\documentclass{article}

% packages
  % basic stuff for rendering math
  \usepackage[letterpaper, top=1in, bottom=1in, left=1in, right=1in]{geometry}
  \usepackage[utf8]{inputenc}
  \usepackage[english]{babel}
  \usepackage{amsmath} 
  \usepackage{amssymb}
  % \usepackage{amsthm}

  % extra math symbols and utilities
  \usepackage{mathtools}        % for extra stuff like \coloneqq
  \usepackage{mathrsfs}         % for extra stuff like \mathsrc{}
  \usepackage{centernot}        % for the centernot arrow 
  \usepackage{bm}               % for better boldsymbol/mathbf 
  \usepackage{enumitem}         % better control over enumerate, itemize
  \usepackage{hyperref}         % for hypertext linking
  \usepackage{fancyvrb}          % for better verbatim environments
  \usepackage{newverbs}         % for texttt{}
  \usepackage{xcolor}           % for colored text 
  \usepackage{listings}         % to include code
  \usepackage{lstautogobble}    % helper package for code
  \usepackage{parcolumns}       % for side by side columns for two column code
  

  % page layout
  \usepackage{fancyhdr}         % for headers and footers 
  \usepackage{lastpage}         % to include last page number in footer 
  \usepackage{parskip}          % for no indentation and space between paragraphs    
  \usepackage[T1]{fontenc}      % to include \textbackslash
  \usepackage{footnote}
  \usepackage{etoolbox}

  % for custom environments
  \usepackage{tcolorbox}        % for better colored boxes in custom environments
  \tcbuselibrary{breakable}     % to allow tcolorboxes to break across pages

  % figures
  \usepackage{pgfplots}
  \pgfplotsset{compat=1.18}
  \usepackage{float}            % for [H] figure placement
  \usepackage{tikz}
  \usepackage{tikz-cd}
  \usepackage{circuitikz}
  \usetikzlibrary{arrows}
  \usetikzlibrary{positioning}
  \usetikzlibrary{calc}
  \usepackage{graphicx}
  \usepackage{caption} 
  \usepackage{subcaption}
  \captionsetup{font=small}

  % for tabular stuff 
  \usepackage{dcolumn}

  \usepackage[nottoc]{tocbibind}
  \pdfsuppresswarningpagegroup=1
  \hfuzz=5.002pt                % ignore overfull hbox badness warnings below this limit

% Custom Environments
  \newtcolorbox[auto counter, number within=section]{example}[1][]
  {
    colframe = orange!25,
    colback  = orange!10,
    coltitle = orange!20!black,  
    breakable, 
    title = \textbf{Question \thetcbcounter ~(#1)}
  }
  \newtcolorbox[auto counter, number within=section]{definition}[1][]
  {
    colframe = yellow!25,
    colback  = yellow!10,
    coltitle = yellow!20!black,  
    breakable, 
    title = \textbf{Definition \thetcbcounter ~(#1)}
  } 
  \newtcolorbox[auto counter, number within=section]{society}[1][]
  {
    colframe = yellow!25,
    colback  = yellow!10,
    coltitle = yellow!20!black,  
    breakable, 
    title = \textbf{Society \thetcbcounter}
  }
  \newtcolorbox[auto counter, number within=section]{politics}[1][]
  {
    colframe = red!25,
    colback  = red!10,
    coltitle = red!20!black,  
    breakable, 
    title = \textbf{Politics \thetcbcounter ~(#1)}
  }
  \newtcolorbox[auto counter, number within=section]{legal}[1][]
  {
    colframe = blue!25,
    colback  = blue!10,
    coltitle = blue!20!black,  
    breakable, 
    title = \textbf{Legal \thetcbcounter ~(#1)}
  } 
  \newtcolorbox[auto counter, number within=section]{finance}[1][]
  {
    colframe = green!25,
    colback  = green!10,
    coltitle = green!20!black,  
    breakable, 
    title = \textbf{Finance \thetcbcounter ~(#1)}
  } 
  \newtcolorbox[auto counter, number within=section]{religion}[1][]
  {
    colframe = violet!25,
    colback  = violet!10,
    coltitle = violet!20!black,  
    breakable, 
    title = \textbf{Religion \thetcbcounter ~(#1)}
  }
  \newtcolorbox[auto counter, number within=section]{technology}[1][]
  {
    colframe = violet!25,
    colback  = violet!10,
    coltitle = violet!20!black,  
    breakable, 
    title = \textbf{Technology \thetcbcounter ~(#1)}
  }

  \BeforeBeginEnvironment{example}{\savenotes}
  \AfterEndEnvironment{example}{\spewnotes}
  \BeforeBeginEnvironment{lemma}{\savenotes}
  \AfterEndEnvironment{lemma}{\spewnotes}
  \BeforeBeginEnvironment{theorem}{\savenotes}
  \AfterEndEnvironment{theorem}{\spewnotes}
  \BeforeBeginEnvironment{corollary}{\savenotes}
  \AfterEndEnvironment{corollary}{\spewnotes}
  \BeforeBeginEnvironment{proposition}{\savenotes}
  \AfterEndEnvironment{proposition}{\spewnotes}
  \BeforeBeginEnvironment{definition}{\savenotes}
  \AfterEndEnvironment{definition}{\spewnotes}
  \BeforeBeginEnvironment{exercise}{\savenotes}
  \AfterEndEnvironment{exercise}{\spewnotes}
  \BeforeBeginEnvironment{proof}{\savenotes}
  \AfterEndEnvironment{proof}{\spewnotes}
  \BeforeBeginEnvironment{solution}{\savenotes}
  \AfterEndEnvironment{solution}{\spewnotes}
  \BeforeBeginEnvironment{question}{\savenotes}
  \AfterEndEnvironment{question}{\spewnotes}
  \BeforeBeginEnvironment{code}{\savenotes}
  \AfterEndEnvironment{code}{\spewnotes}

  \definecolor{dkgreen}{rgb}{0,0.6,0}
  \definecolor{gray}{rgb}{0.5,0.5,0.5}
  \definecolor{mauve}{rgb}{0.58,0,0.82}
  \definecolor{lightgray}{gray}{0.93}

  % default options for listings (for code)
  \lstset{
    autogobble,
    frame=ltbr,
    language=C,                           % the language of the code
    aboveskip=3mm,
    belowskip=3mm,
    showstringspaces=false,
    columns=fullflexible,
    keepspaces=true,
    basicstyle={\small\ttfamily},
    numbers=left,
    firstnumber=1,                        % start line number at 1
    numberstyle=\tiny\color{gray},
    keywordstyle=\color{blue},
    commentstyle=\color{dkgreen},
    stringstyle=\color{mauve},
    backgroundcolor=\color{lightgray}, 
    breaklines=true,                      % break lines
    breakatwhitespace=true,
    tabsize=3, 
    xleftmargin=2em, 
    framexleftmargin=1.5em, 
    stepnumber=1
  }

% Page style
  \pagestyle{fancy}
  \fancyhead[L]{Greeks}
  \fancyhead[C]{Muchang Bahng}
  \fancyhead[R]{Spring 2024} 
  \fancyfoot[C]{\thepage / \pageref{LastPage}}
  \renewcommand{\footrulewidth}{0.4pt}          % the footer line should be 0.4pt wide
  \renewcommand{\thispagestyle}[1]{}  % needed to include headers in title page

\begin{document}

\title{Ancient Greece}
\author{Muchang Bahng}
\date{Spring 2024}

\maketitle
\tableofcontents
\pagebreak

\section{Growth of Athens}

  \subsection{Maritime Trade}

    Furthermore, the city had now outgrown its local agricultural capacity. The Attic plain and the surrounding hills were more suited to cultivation of olive trees and honey rather than wheat and barley. The population was in the hundreds of thousands, and bread was a basic staple. To feed this massive population, they relied on gain imports through overseas trade. Essentially, traders would take boats and have investors fill them up with silver and other goods. They would sail overseas to trade these items for grain. Finally, they would take the grain back, where they can resell it to the population for profit. However, this process was highly regulated with laws.  
    \begin{enumerate}
      \item $2/3$ of all imported grain were required to go to the city of Athens. Shipping grain to any other port by an Athenian citizen was a capital offense. 
      \item Maritime (overseas) lending was restricted \textit{solely} to the overseas grain trade, meaning that Athenian lenders were only allowed to provide loans for the purpose of financing grain imports. This made sure that merchants would prioritize grain over other goods. 
      \item Dealers were limited in how much grain they were allowed to store, and their mark-ups on resale were capped.  
    \end{enumerate}
    All of these regulations were enforced by market overseers, and rewards were offered to \textit{sycophants}, informers who sold out grain dealers who skirted the rules. Despite these regulations, the market price of grain was very volatile, and during expensive times investors and dealers who received the grain could make a solid profit. 

    It seems quite scary that such a developed society is so dependent on maritime trade for survival. However, as we will see, a complex political/legal system with Athenian finance made this stable. 

  \subsection{Legal System}

    In Athens, the capital of Greece, much of the political system relied on the legal system. 

    \begin{legal}[Jurors]
      The Athenian court system resolved disputes between plaintiffs and defendants through trial by jury. Jurors, typically 500 at a time, were chosen randomly for a day, which was the maximum length of a trial. Plaintiffs and defendants represented themselves, although famous orators were sometimes engaged to compose their speeches. Speaking time was regulated by a water-clock. Jurors did not deliberate together; rather they voted, and the ruling was determined by a majority vote.At the end of the day, the matter was settled. The system was widely used for commercial disputes, many involving the Athenian grain trade. Courts specifically for maritime cases were held from September to April, when ships were not at sea and business could be settled in time for the next season. 
    \end{legal}

    This meant that the average citizen had to be quite knowledgeable about finance and politics, and at the same time the orators had to communicate clearly. 

  \subsection{Courts and Conflicts}

    The purpose of the sophisticated legal system was to resolve conflicts. Let's take a look at an instance. 

    \begin{example}[First Antitrust Case]
      In 386 BCE, there had been a price shock of grain. It was most likely due to a naval blockade of one of Athen's key grain suppliers. This led to constrain in supply and therefore a rise in price.\footnote{Basic economic rule. } A group of Athenian dealers colluded to do two things: 
      \begin{enumerate}
        \item By fixing the price at which they were willing to buy to be low, they used this as leverage to buy grain for cheap prices overseas. 
        \item When they came back, they all sold it for high prices to pocket a nice profit rather than passing the savings on the buyers. 
      \end{enumerate}
      Therefore, they colluded against both suppliers and customers, ultimately receiving the death penalty. This may have been the first recorded attempt in an antitrust case. According to the plaintiff, the consequences of this crime is that due to the fixed low prices of grain, overseas merchants would not sell Athenian merchants grain. This would restrict the flow of grain into the city, raising its prices in Athens to unaffordable levels. 
    \end{example}

    The annual grain imports are estimated to be around $26,000$ tons of wheat, enough to feed over $100,000$ people, meaning that hundreds of Greek vessels would set sail every year. 

    Another source of conflict is through loans. Note that for investments in overseas trading voyages, there is always the possibility of a ship sinking, the loss of goods, etc. Therefore, the lenders would take on this risk, which was good since the lenders can diversify their risk amongst several ships. The actual borrowers had all of their eggs in one basket, having all the incentive to buy grain cheaply overseas and sell it high in Athens to repay their loans. 

    \begin{example}
      In 352 BCE, an investor financed a loan for a merchant voyage from Athens to the Black Sea and back again, to use silver to buy wine, and trade it for grain to ship back into Athens. The plaintiffs argued that the traders first bought the wine, but used it as collateral to borrow from a Chian man. They essentially used the same set of wine to bargain twice over. They ended up owing the man the wine as collateral, so by overpromising to the two investors the only option they had was to throw the wine overboard. It turned out that the Chian man had a more senior claim to the wine than the original investor. 
    \end{example}

  \subsection{Banking}

    The Greek term for bank, \textit{trapeza}, refers to a table on which the banker conducted business. 

  \subsection{Investments in Silver Mines}

    Unlike Mesopotamia, where silver was rare, the land around Athens, especially in the southwest in an area called Laurion, was extremely rich in silver. This silver was mined extensively, flowing into private and public coffers, and then as coinage, flowing out of the Athenian port of Piraeus through the Mediterranean, the Black Sea, and beyond. Therefore, Athenians did not have to have commodities to exchange at other ports. Their silver was accepted everywhere. 


    \begin{politics}[Poletai]
      The Athenian government had a council of 10 magistrates called \textbf{poletai}, appointed annually from among Athens' ten tribes. They broadly represented the citizens. 
    \end{politics}

    These Laurion mining operations were not state owned, but privatized like the grain trade. The poletai were in charge of auctioning off these (state-owned) sites, gaining a bit of profit for the state as well. Then, entrepreneurs would use the lease and invest capital to dig for silver ore. If successful, they set up local smelting operations that required even more capital investment. The silver ore contained substantial amounts of lead. Separating the two metals required smashing, roasting, and reheating ore and used a lot of water, which was not necessarily close to the mines. The operations were both a technical and a financial challenge. Investors took great risks in prospecting for ore and then had to raise more capital once it was discovered. 


    These leases were broadly divided into 3 classes: 
    \begin{enumerate}
      \item Unexplored leases 
      \item Developed 
      \item Previously developed but abandoned
    \end{enumerate}

    \begin{example}[Court on Mining Leases]
      A complex lawsuit that displays the financial framework in Athens is one about silver mines. 

      In 346 BCE, the entrepreneur Pantaenetus leased a mine and then borrowed 100 minas (10,000 drachmas) to buy into a partnership that owned the slaves and workshop to run the mining operation. His partnership share of the slaves and workshop served as collateral for his loan. The trouble started when he and his partners sold the slaves and workshop to another consortium with the understanding that Pantaenetus would continue to rent them. Then Pantaenetus failed to meet the rent and to pay the poletai for the lease.

      The consortium seized the assets—but discovered that they were already pledged as collateral for Pantaenetus’s original loan. The collateral was only a fraction of the enterprise, not the whole. The various creditors argued their claims, and the accusations got personal. Pantaenetus blamed everyone—but particularly the creditors, who were evidently professional financiers. 

      This shows that entire companies can change hands in an instant.
    \end{example}

  \subsection{Democracy}

  \subsection{Coinage}

    \begin{finance}[First Coins are Minted]
      Athens first began minting its famous coins around 600 BCE. The coin had Athena's profile on the front and her owl familiar on the back, with the letters $A\Omega E$ (for Athens). Its main function was to serve as a liquid, ready store of value for the Athenian government. 
    \end{finance}

    One of Athens's great military strengths was her treasure of coined silver, which could pay for soldiers and fleets. This treasure was kept in the famous Parthenon temple of Athena, a total of 36 million drachmas. The spread of these coins through trade well beyond Athens had made this coin pretty much universal, up to the point where other civilizations would copy this design. The owl became an international currency for which Athens had a monopoly. Therefore, the world of the eastern Mediterranean and the Near East looked to Athens in part to supply the currency of international trade.

    Clearly, this was not the first instance of money. After all, the Sumerians used silver as a method of exchange.\footnote{Barley, while treated as some money, was perishable and consequently doesn't completely fit the definition of money. } Before, the weight and purity of these precious metals had to be measured, but it isn't as if a grocer can constantly and reliable weigh out shekels of silver while bargaining over barley, cress, and dates. Therefore, by stamping these coins with a label, the need to do this had gone. This also made it more convenient for smaller-scale transactions that were too trivial to be written down as a contract, or for one-time transactions where the probability of seeing the other party again is slim. 

    The need for money had been discussed from the time of the Greeks. Aristotle and other scholars generally agree that long-distance trade necessitated some convenient and valuable form of storage to be used in exchanges all around the world. 
 
\section{Other}

  Greek rule with Alexander the Great in Babylon. Prices of barley through time recorded and how it doubled after his death. 

  Socrates, Plato, Aristotle. 

\end{document}

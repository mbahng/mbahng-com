\documentclass{article}

% packages
  % basic stuff for rendering math
  \usepackage[letterpaper, top=1in, bottom=1in, left=1in, right=1in]{geometry}
  \usepackage[utf8]{inputenc}
  \usepackage[english]{babel}
  \usepackage{amsmath} 
  \usepackage{amssymb}
  % \usepackage{amsthm}

  % extra math symbols and utilities
  \usepackage{mathtools}        % for extra stuff like \coloneqq
  \usepackage{mathrsfs}         % for extra stuff like \mathsrc{}
  \usepackage{centernot}        % for the centernot arrow 
  \usepackage{bm}               % for better boldsymbol/mathbf 
  \usepackage{enumitem}         % better control over enumerate, itemize
  \usepackage{hyperref}         % for hypertext linking
  \usepackage{fancyvrb}          % for better verbatim environments
  \usepackage{newverbs}         % for texttt{}
  \usepackage{xcolor}           % for colored text 
  \usepackage{listings}         % to include code
  \usepackage{lstautogobble}    % helper package for code
  \usepackage{parcolumns}       % for side by side columns for two column code
  

  % page layout
  \usepackage{fancyhdr}         % for headers and footers 
  \usepackage{lastpage}         % to include last page number in footer 
  \usepackage{parskip}          % for no indentation and space between paragraphs    
  \usepackage[T1]{fontenc}      % to include \textbackslash
  \usepackage{footnote}
  \usepackage{etoolbox}

  % for custom environments
  \usepackage{tcolorbox}        % for better colored boxes in custom environments
  \tcbuselibrary{breakable}     % to allow tcolorboxes to break across pages

  % figures
  \usepackage{pgfplots}
  \pgfplotsset{compat=1.18}
  \usepackage{float}            % for [H] figure placement
  \usepackage{tikz}
  \usepackage{tikz-cd}
  \usepackage{circuitikz}
  \usetikzlibrary{arrows}
  \usetikzlibrary{positioning}
  \usetikzlibrary{calc}
  \usepackage{graphicx}
  \usepackage{caption} 
  \usepackage{subcaption}
  \captionsetup{font=small}

  % for tabular stuff 
  \usepackage{dcolumn}

  \usepackage[nottoc]{tocbibind}
  \pdfsuppresswarningpagegroup=1
  \hfuzz=5.002pt                % ignore overfull hbox badness warnings below this limit

% Custom Environments
  \newtcolorbox[auto counter, number within=section]{example}[1][]
  {
    colframe = orange!25,
    colback  = orange!10,
    coltitle = orange!20!black,  
    breakable, 
    title = \textbf{Question \thetcbcounter ~(#1)}
  }
  \newtcolorbox[auto counter, number within=section]{definition}[1][]
  {
    colframe = yellow!25,
    colback  = yellow!10,
    coltitle = yellow!20!black,  
    breakable, 
    title = \textbf{Definition \thetcbcounter ~(#1)}
  } 
  \newtcolorbox[auto counter, number within=section]{society}[1][]
  {
    colframe = yellow!25,
    colback  = yellow!10,
    coltitle = yellow!20!black,  
    breakable, 
    title = \textbf{Society \thetcbcounter}
  }
  \newtcolorbox[auto counter, number within=section]{politics}[1][]
  {
    colframe = red!25,
    colback  = red!10,
    coltitle = red!20!black,  
    breakable, 
    title = \textbf{Politics \thetcbcounter ~(#1)}
  }
  \newtcolorbox[auto counter, number within=section]{legal}[1][]
  {
    colframe = blue!25,
    colback  = blue!10,
    coltitle = blue!20!black,  
    breakable, 
    title = \textbf{Legal \thetcbcounter ~(#1)}
  } 
  \newtcolorbox[auto counter, number within=section]{finance}[1][]
  {
    colframe = green!25,
    colback  = green!10,
    coltitle = green!20!black,  
    breakable, 
    title = \textbf{Finance \thetcbcounter ~(#1)}
  } 
  \newtcolorbox[auto counter, number within=section]{religion}[1][]
  {
    colframe = violet!25,
    colback  = violet!10,
    coltitle = violet!20!black,  
    breakable, 
    title = \textbf{Religion \thetcbcounter ~(#1)}
  }
  \newtcolorbox[auto counter, number within=section]{technology}[1][]
  {
    colframe = violet!25,
    colback  = violet!10,
    coltitle = violet!20!black,  
    breakable, 
    title = \textbf{Technology \thetcbcounter ~(#1)}
  }

  \BeforeBeginEnvironment{example}{\savenotes}
  \AfterEndEnvironment{example}{\spewnotes}
  \BeforeBeginEnvironment{lemma}{\savenotes}
  \AfterEndEnvironment{lemma}{\spewnotes}
  \BeforeBeginEnvironment{theorem}{\savenotes}
  \AfterEndEnvironment{theorem}{\spewnotes}
  \BeforeBeginEnvironment{corollary}{\savenotes}
  \AfterEndEnvironment{corollary}{\spewnotes}
  \BeforeBeginEnvironment{proposition}{\savenotes}
  \AfterEndEnvironment{proposition}{\spewnotes}
  \BeforeBeginEnvironment{definition}{\savenotes}
  \AfterEndEnvironment{definition}{\spewnotes}
  \BeforeBeginEnvironment{exercise}{\savenotes}
  \AfterEndEnvironment{exercise}{\spewnotes}
  \BeforeBeginEnvironment{proof}{\savenotes}
  \AfterEndEnvironment{proof}{\spewnotes}
  \BeforeBeginEnvironment{solution}{\savenotes}
  \AfterEndEnvironment{solution}{\spewnotes}
  \BeforeBeginEnvironment{question}{\savenotes}
  \AfterEndEnvironment{question}{\spewnotes}
  \BeforeBeginEnvironment{code}{\savenotes}
  \AfterEndEnvironment{code}{\spewnotes}

  \definecolor{dkgreen}{rgb}{0,0.6,0}
  \definecolor{gray}{rgb}{0.5,0.5,0.5}
  \definecolor{mauve}{rgb}{0.58,0,0.82}
  \definecolor{lightgray}{gray}{0.93}

  % default options for listings (for code)
  \lstset{
    autogobble,
    frame=ltbr,
    language=C,                           % the language of the code
    aboveskip=3mm,
    belowskip=3mm,
    showstringspaces=false,
    columns=fullflexible,
    keepspaces=true,
    basicstyle={\small\ttfamily},
    numbers=left,
    firstnumber=1,                        % start line number at 1
    numberstyle=\tiny\color{gray},
    keywordstyle=\color{blue},
    commentstyle=\color{dkgreen},
    stringstyle=\color{mauve},
    backgroundcolor=\color{lightgray}, 
    breaklines=true,                      % break lines
    breakatwhitespace=true,
    tabsize=3, 
    xleftmargin=2em, 
    framexleftmargin=1.5em, 
    stepnumber=1
  }

% Page style
  \pagestyle{fancy}
  \fancyhead[L]{Romans}
  \fancyhead[C]{Muchang Bahng}
  \fancyhead[R]{Spring 2024} 
  \fancyfoot[C]{\thepage / \pageref{LastPage}}
  \renewcommand{\footrulewidth}{0.4pt}          % the footer line should be 0.4pt wide
  \renewcommand{\thispagestyle}[1]{}  % needed to include headers in title page

\begin{document}

\title{The Roman Empire}
\author{Muchang Bahng}
\date{Spring 2024}

\maketitle
\tableofcontents
\pagebreak

\section{} 
  
  Again, Rome's financial structure, like Athens, was developed to supply food to an urban metropolis that had grown well beyond local agricultural capacity. However, Rome's trade network encompassed pretty much all of Europe and North Africa, with distant connections to China and India. 

  Rome's conquest, which happened through finance paying soldiers. To effectively tax and administer newly conquered regions, Rome privatized various functions of the state, including tax collection, military supply, and construction.  

  \subsection{Roman Society}

    In Roman society, there was a sharp division into classes and the dependence of political rank on wealth. Money was a necessary, although not ufficient, requirement for ris- ing to a position of power. Over the course of Roman history—from monarchy, to republic, to empire—Rome was ruled by a small, self- perpetuating oligarchy defined by heredity and property. At is greatest extent, a group of roughly 10,000 people ruled an empire of 60 million. 

    \begin{politics}[Senate]
      Rome's ruling body was the \textbf{senate}, a body of 300-500 people. Membership in the senate required a fortune of $250,000$ denarii, election by Senate members, and perhaps approval by the emperor. This was regularly updated with census checks on the citizens. 
    \end{politics}

    \begin{society}[Social Classes]
      There were three social classes:
      \begin{enumerate}
        \item The \textbf{patricians} were the hereditary descendants of Rome's earliest ruling families. 
        \item The \textbf{equestrian class} were Rome's knights. Their elevated rank derived from a traditional role of providing cavalry to Rome's army. Membership required $100,000$ denarii. 
        \item \textbf{Plebeians} and freedmen (former slaves) were in the lower ranks. 
      \end{enumerate}
    \end{society}

    Because of this connection between wealth and rank, financial cooperation, competition, and intrigue represented important dimensions of political strategy. This led to legal constraints on entrepreneurial activity by politicians. For example, the Lex Claudia, a law passed by the Senate in 218 BCE, limited the carrying capacity of merchant ships owned by senators. It was intended to prevent senators from exploiting their political advantage for economic gain. A senator was expected to make his money from land; to own large estates that grew wheat, wine, and olives; and to sell these locally. Without large ships, exporting produce from senatorial estates was effectively controlled.

    Once a knight achieved the rank of senator, he was theoretically barred from direct participation in the vast and profitable trade of the empire, except through indirect investment, such as lending. Senators not only had to be rich but their active capital was also seriously constrained, even though it was the explicit basis for their eligibility. In short, senators had to maintain great fortunes without direct involvement in lucrative enterprise. Thus, the ability to delegate financial operations—to have plausible deniability of involvement in business— and to separate ownership and control were essential. As we shall see, the Roman financial system evolved institutions that allowed senators precisely these opportunities

    The next rank down, Roman knights and their families—unlike Roman senators—could engage in commerce. They conducted major business operations and manned important government posts. The equestrian class ultimately developed a form of financial organization much like a modern corporation. The corporate structure gave the equestrian class the ability to make equity investments, but it also pre- served Rome’s oligopolistic structure—knights who invested in these companies effectively shared the risk and return of enterprise with their co-investors.

  \subsection{Financial Crisis}

    The first financial crisis recorded was of Rome in 33 CE, which happened in the private sector. But to understand this a bit deeper, let's look at a slightly smaller financial crisis in 49 BCE. 

    In 49 BCE, the civil war between Julius Caesar and Pompey for control of Rome led to a dearth of loans and declining property values. This may be due to a few reasons. 
    \begin{enumerate}
      \item Political instability. Lenders don't know what will happen to the city. The possibility of borrowers escaping the city is high, so nobody wanted to lend. Likewise, land may be destroyed or unusable during the invasions, so estate prices had also fell. 

      \item Since property values are declining, borrowers may have more trouble paying off their debts by selling their land, or using land as collateral.  
    \end{enumerate}

    Due to the increased risk loaners put up extremely high interest rates for loans, which repelled borrowers. To reduce this problem, the Senate capped interest rates at $12\%$. However, this still did not solve the credit crisis, so Caesar took additional measures. 
    \begin{enumerate}
      \item He allowed debt repayments in land at pre-crisis values. This would allow borrowers to have a bigger cash cushion as collateral. 
      \item He canceled interest due on mortgages. This would also encourage borrowers since there is no interest anymore. 
      \item He forbade cash holding, which would essentially force lenders to loan out money rather than holding it under a mattress. 
      \item He required lenders to hold a portion of their wealth in real estate, which also provided an influx of money to prop up estate prices.  
    \end{enumerate}

    This helped the problem, except that he was assassinated. Therefore, 80 years later in 31 CE, the current emperor Tiberius after putting down a coup, had used these same strategies to bring back the credit market. 

\section{Others}



\end{document}

\documentclass{article}

% packages
  % basic stuff for rendering math
  \usepackage[letterpaper, top=1in, bottom=1in, left=1in, right=1in]{geometry}
  \usepackage[utf8]{inputenc}
  \usepackage[english]{babel}
  \usepackage{amsmath} 
  \usepackage{amssymb}
  % \usepackage{amsthm}

  % extra math symbols and utilities
  \usepackage{mathtools}        % for extra stuff like \coloneqq
  \usepackage{mathrsfs}         % for extra stuff like \mathsrc{}
  \usepackage{centernot}        % for the centernot arrow 
  \usepackage{bm}               % for better boldsymbol/mathbf 
  \usepackage{enumitem}         % better control over enumerate, itemize
  \usepackage{hyperref}         % for hypertext linking
  \usepackage{fancyvrb}          % for better verbatim environments
  \usepackage{newverbs}         % for texttt{}
  \usepackage{xcolor}           % for colored text 
  \usepackage{listings}         % to include code
  \usepackage{lstautogobble}    % helper package for code
  \usepackage{parcolumns}       % for side by side columns for two column code
  

  % page layout
  \usepackage{fancyhdr}         % for headers and footers 
  \usepackage{lastpage}         % to include last page number in footer 
  \usepackage{parskip}          % for no indentation and space between paragraphs    
  \usepackage[T1]{fontenc}      % to include \textbackslash
  \usepackage{footnote}
  \usepackage{etoolbox}

  % for custom environments
  \usepackage{tcolorbox}        % for better colored boxes in custom environments
  \tcbuselibrary{breakable}     % to allow tcolorboxes to break across pages

  % figures
  \usepackage{pgfplots}
  \pgfplotsset{compat=1.18}
  \usepackage{float}            % for [H] figure placement
  \usepackage{tikz}
  \usepackage{tikz-cd}
  \usepackage{circuitikz}
  \usetikzlibrary{arrows}
  \usetikzlibrary{positioning}
  \usetikzlibrary{calc}
  \usepackage{graphicx}
  \usepackage{caption} 
  \usepackage{subcaption}
  \captionsetup{font=small}

  % for tabular stuff 
  \usepackage{dcolumn}

  \usepackage[nottoc]{tocbibind}
  \pdfsuppresswarningpagegroup=1
  \hfuzz=5.002pt                % ignore overfull hbox badness warnings below this limit

% Custom Environments
  \newtcolorbox[auto counter, number within=section]{example}[1][]
  {
    colframe = orange!25,
    colback  = orange!10,
    coltitle = orange!20!black,  
    breakable, 
    title = \textbf{Question \thetcbcounter ~(#1)}
  }
  \newtcolorbox[auto counter, number within=section]{definition}[1][]
  {
    colframe = yellow!25,
    colback  = yellow!10,
    coltitle = yellow!20!black,  
    breakable, 
    title = \textbf{Definition \thetcbcounter ~(#1)}
  } 
  \newtcolorbox[auto counter, number within=section]{society}[1][]
  {
    colframe = yellow!25,
    colback  = yellow!10,
    coltitle = yellow!20!black,  
    breakable, 
    title = \textbf{Society \thetcbcounter}
  }
  \newtcolorbox[auto counter, number within=section]{politics}[1][]
  {
    colframe = red!25,
    colback  = red!10,
    coltitle = red!20!black,  
    breakable, 
    title = \textbf{Politics \thetcbcounter ~(#1)}
  }
  \newtcolorbox[auto counter, number within=section]{legal}[1][]
  {
    colframe = blue!25,
    colback  = blue!10,
    coltitle = blue!20!black,  
    breakable, 
    title = \textbf{Legal \thetcbcounter ~(#1)}
  } 
  \newtcolorbox[auto counter, number within=section]{finance}[1][]
  {
    colframe = green!25,
    colback  = green!10,
    coltitle = green!20!black,  
    breakable, 
    title = \textbf{Finance \thetcbcounter ~(#1)}
  } 
  \newtcolorbox[auto counter, number within=section]{religion}[1][]
  {
    colframe = violet!25,
    colback  = violet!10,
    coltitle = violet!20!black,  
    breakable, 
    title = \textbf{Religion \thetcbcounter ~(#1)}
  }
  \newtcolorbox[auto counter, number within=section]{technology}[1][]
  {
    colframe = violet!25,
    colback  = violet!10,
    coltitle = violet!20!black,  
    breakable, 
    title = \textbf{Technology \thetcbcounter ~(#1)}
  }

  \BeforeBeginEnvironment{example}{\savenotes}
  \AfterEndEnvironment{example}{\spewnotes}
  \BeforeBeginEnvironment{lemma}{\savenotes}
  \AfterEndEnvironment{lemma}{\spewnotes}
  \BeforeBeginEnvironment{theorem}{\savenotes}
  \AfterEndEnvironment{theorem}{\spewnotes}
  \BeforeBeginEnvironment{corollary}{\savenotes}
  \AfterEndEnvironment{corollary}{\spewnotes}
  \BeforeBeginEnvironment{proposition}{\savenotes}
  \AfterEndEnvironment{proposition}{\spewnotes}
  \BeforeBeginEnvironment{definition}{\savenotes}
  \AfterEndEnvironment{definition}{\spewnotes}
  \BeforeBeginEnvironment{exercise}{\savenotes}
  \AfterEndEnvironment{exercise}{\spewnotes}
  \BeforeBeginEnvironment{proof}{\savenotes}
  \AfterEndEnvironment{proof}{\spewnotes}
  \BeforeBeginEnvironment{solution}{\savenotes}
  \AfterEndEnvironment{solution}{\spewnotes}
  \BeforeBeginEnvironment{question}{\savenotes}
  \AfterEndEnvironment{question}{\spewnotes}
  \BeforeBeginEnvironment{code}{\savenotes}
  \AfterEndEnvironment{code}{\spewnotes}

  \definecolor{dkgreen}{rgb}{0,0.6,0}
  \definecolor{gray}{rgb}{0.5,0.5,0.5}
  \definecolor{mauve}{rgb}{0.58,0,0.82}
  \definecolor{lightgray}{gray}{0.93}

  % default options for listings (for code)
  \lstset{
    autogobble,
    frame=ltbr,
    language=C,                           % the language of the code
    aboveskip=3mm,
    belowskip=3mm,
    showstringspaces=false,
    columns=fullflexible,
    keepspaces=true,
    basicstyle={\small\ttfamily},
    numbers=left,
    firstnumber=1,                        % start line number at 1
    numberstyle=\tiny\color{gray},
    keywordstyle=\color{blue},
    commentstyle=\color{dkgreen},
    stringstyle=\color{mauve},
    backgroundcolor=\color{lightgray}, 
    breaklines=true,                      % break lines
    breakatwhitespace=true,
    tabsize=3, 
    xleftmargin=2em, 
    framexleftmargin=1.5em, 
    stepnumber=1
  }

% Page style
  \pagestyle{fancy}
  \fancyhead[L]{Romans}
  \fancyhead[C]{Muchang Bahng}
  \fancyhead[R]{Spring 2024} 
  \fancyfoot[C]{\thepage / \pageref{LastPage}}
  \renewcommand{\footrulewidth}{0.4pt}          % the footer line should be 0.4pt wide
  \renewcommand{\thispagestyle}[1]{}  % needed to include headers in title page

\begin{document}

\title{The Roman Empire}
\author{Muchang Bahng}
\date{Spring 2024}

\maketitle
\tableofcontents
\pagebreak

\section{} 
  
  Again, Rome's financial structure, like Athens, was developed to supply food to an urban metropolis that had grown well beyond local agricultural capacity. However, Rome's trade network encompassed pretty much all of Europe and North Africa, with distant connections to China and India. 

  Rome's conquest, which happened through finance paying soldiers. To effectively tax and administer newly conquered regions, Rome privatized various functions of the state, including tax collection, military supply, and construction.  

  As we have seen in Athens, there were bankers as well in Rome. The private bankers were called argentarii, and they appear in Roman history as early as the mid-fourth century BCE. The name “argentarii” suggests they had their origin as money changers, as did, perhaps the Athenian bankers. Argentarii supplied a variety of banking services, including taking deposits, transferring by check or account, advancing money to clients, lending to bidders at auction, and money transfer via bills of exchange. They had their own guild and occupied shops along the Via Sacra in the Forum. As in Athens, bankers may also have operated nearer the wharves of the city. 

  \subsection{Roman Society}

    In Roman society, there was a sharp division into classes and the dependence of political rank on wealth. Money was a necessary, although not ufficient, requirement for ris- ing to a position of power. Over the course of Roman history—from monarchy, to republic, to empire—Rome was ruled by a small, self- perpetuating oligarchy defined by heredity and property. At is greatest extent, a group of roughly 10,000 people ruled an empire of 60 million. 

    \begin{politics}[Senate]
      Rome's ruling body was the \textbf{senate}, a body of 300-500 people. Membership in the senate required a fortune of $250,000$ denarii, election by Senate members, and perhaps approval by the emperor. This was regularly updated with census checks on the citizens. 
    \end{politics}

    \begin{society}[Social Classes]
      There were three social classes:
      \begin{enumerate}
        \item The \textbf{patricians} were the hereditary descendants of Rome's earliest ruling families. 
        \item The \textbf{equestrian class} were Rome's knights. Their elevated rank derived from a traditional role of providing cavalry to Rome's army. Membership required $100,000$ denarii. 
        \item \textbf{Plebeians} and freedmen (former slaves) were in the lower ranks. 
      \end{enumerate}
    \end{society}

    Because of this connection between wealth and rank, financial cooperation, competition, and intrigue represented important dimensions of political strategy. This led to legal constraints on entrepreneurial activity by politicians. For example, the Lex Claudia, a law passed by the Senate in 218 BCE, limited the carrying capacity of merchant ships owned by senators. It was intended to prevent senators from exploiting their political advantage for economic gain. A senator was expected to make his money from land; to own large estates that grew wheat, wine, and olives; and to sell these locally. Without large ships, exporting produce from senatorial estates was effectively controlled.

    Once a knight achieved the rank of senator, he was theoretically barred from direct participation in the vast and profitable trade of the empire, except through indirect investment, such as lending. Senators not only had to be rich but their active capital was also seriously constrained, even though it was the explicit basis for their eligibility. In short, senators had to maintain great fortunes without direct involvement in lucrative enterprise. Thus, the ability to delegate financial operations—to have plausible deniability of involvement in business— and to separate ownership and control were essential. As we shall see, the Roman financial system evolved institutions that allowed senators precisely these opportunities

    The next rank down, Roman knights and their families—unlike Roman senators—could engage in commerce. They conducted major business operations and manned important government posts. The equestrian class ultimately developed a form of financial organization much like a modern corporation. The corporate structure gave the equestrian class the ability to make equity investments, but it also pre- served Rome’s oligopolistic structure—knights who invested in these companies effectively shared the risk and return of enterprise with their co-investors.

  \subsection{Financial Crisis}

    The first financial crisis recorded was of Rome in 33 CE, which happened in the private sector. But to understand this a bit deeper, let's look at a slightly smaller financial crisis. 
    
    \subsubsection{Debt Crisis with Iran}

      In 60 BCE, Rome was a superpower. It had a professional army, which was superb, but also a heavy drain upon the treasury. Rome had also allowed two great threats to grow up on her frontiers—one was in central Europe, the tribes of Germans and Gauls, and the other was Iran. In less than a century Iran had grown from a divided and weak country into an economic powerhouse with a superb army, much underrated by the Romans. Iran held a great deal of Roman debt, and Rome had a huge negative trade imbalance with Iran. (Think China.)\footnote{https://ocpathink.org/post/the-debt-crisis-in-ancient-rome-lessons-for-today}

      This lack of foresight by the Romans in foreign policy was matched by their lack of foresight in fiscal matters. It was the belief of the most profound thinkers of antiquity, like Plato and Aristotle, that democracies are inherently fiscally irresponsible. The Athenian statesman Pericles (leader from 461 to 420 B.C.) was said to be the first politician to understand the true mechanism of power in a democracy: to get the people to bribe themselves with their own money—creating massive entitlements. This had also come to be true in Rome—the entitlements of the Roman citizens included everything from free food to free entertainment in the form of gladiatorial games and chariot races. Roman citizens also paid no taxes whatsoever. No politician in Rome dared to either cut these entitlements or impose taxes. All tax money was wrung from the provincials, who were not Roman citizens. This taxation system was corrupt and oppressive, and led to the hatred of Roman power by provincials all the way from Spain to Syria.

      However, even with the taxes from the provincials, the Roman treasury was empty in the year 60 B.C. In fact, it had a huge debt. The public debt was matched only by the massive private debt. Every Roman seemed to think he had an unlimited ability to borrow money. This was also true of the Roman Senate.

      Bill after bill was brought forward to balance the budget and to pay down the debt. Massive infusions of money into the economy had no effect. Every sensible bill was blocked by the bitter partisan politics in the Senate.

    \subsubsection{Debt Crisis of 49 BCE}

      In 49 BCE, the civil war between Julius Caesar and Pompey for control of Rome led to a dearth of loans and declining property values. This may be due to a few reasons. 
      \begin{enumerate}
        \item Political instability. Lenders don't know what will happen to the city. The possibility of borrowers escaping the city is high, so nobody wanted to lend. Likewise, land may be destroyed or unusable during the invasions, so estate prices had also fell. 

        \item Since property values are declining, borrowers may have more trouble paying off their debts by selling their land, or using land as collateral.  
      \end{enumerate}

      Due to the increased risk loaners put up extremely high interest rates for loans, which repelled borrowers. To reduce this problem, the Senate capped interest rates at $12\%$. However, this still did not solve the credit crisis, so Caesar took additional measures. 
      \begin{enumerate}
        \item He allowed debt repayments in land at pre-crisis values. This would allow borrowers to have a bigger cash cushion as collateral. 
        \item He canceled interest due on mortgages. This would also encourage borrowers since there is no interest anymore. 
        \item He forbade cash holding, which would essentially force lenders to loan out money rather than holding it under a mattress. 
        \item He required lenders to hold a portion of their wealth in Italian real estate in order to lend at interest. This also provided an influx of money to prop up estate prices.  
      \end{enumerate}

      This helped the problem, except that he was assassinated. Later, the interest was further pushed down to $5\%$. 

    \subsubsection{Debt Crisis and Government Bailout in 33 CE}

      These laws, while still published, fell into disuse over the following decades, but remained on the books and was revived 80 years later in 33 CE, when the emperor Tiberious, after putting down a coup and in the midst of a political turmoil, was planning to use these same strategies to bring back the credit market that was in distress (for the same reasons). It turns out that this brought a flood of allegations against prominent individuals describing widespread violation of the land-owning requirement. The number of cases quickly overwhelmed the court tasked with these matters, which referred the issue to the senate, and the senate in turn referred the issue to Tiberius. Amazingly—and hyperbolically, in all likelihood—Tacitus tells us that every one of the 600 senators was in personal violation of this law, and they sought Tiberius’s indulgence. He instituted a grace period of eighteen months in which all personal finances were to be brought into accordance with the law.\footnote{https://epicenter.wcfia.harvard.edu/blog/financial-crisis-then-and-now}

      To raise money to buy this land, creditors called in all their loans and demanded that they be paid back in their entirety. Debtors, who had no choice but to pay, tried to sell of their own lands to raise funds, but this influx of property (increase in supply) pushed down prices. This was probably further pushed down with the emperor liquidating the confiscated estates of Sejanus's (the coup leader) supporters. Those who could not cover the costs turned to money-lenders who charged extremely high rates, and a great number of debtors were brought into court and ultimately ejected from their lands along with losing their social rank. Ultimately, this resulted in money being taken out of public circulation back into the imperial treasury. 

      To prop up the prices, the senate then passed a law that required creditors to invest $2/3$ of their capital in Italian land. This would force creditors to buy property and force mortgage credit to landholders, but ironically this made matters worse. The creditors who were required to invest in land held onto the funds from the loans they had managed to call in, figuring that they would allow land prices to continue to fall before they made the purchases that would bring them into conformity with the law. The result was a collapse in land values and a shortage of credit that drove up interest rates. 

      At this point Tiberious stepped in. To bring the credit market back, he distributed 100 million sestertii (4\% of the total treasury) to specially chartered banks in order to make available 3-year, interest-free loans.\footnote{Each loan was secured against land of twice its value.} This allowed debtors to pay back their creditors with these loans, restoring the value of real estate and ultimately restoring credit from private lenders. 

      This account tells us that the interrelationship among the major lenders created systemic risk in Roman finance, such as credit contractions and mortgage defaults. Furthermore, the Roman treasury also functioned similarly as the US Treasury traditionally has done with a financial crisis. It relieved the lack of credit through loans and used intermediary institutions to implement the solution. 
      
      One question remains. How did the banks actually receive the 100 million sestertii? It was no t likely that there were Roman treasurers carrying bags of coins to the financial district to the bankers. Rather, it was through a ledger transfer, i.e. a government commitment or guarantee to provide the capital on which the banks could then write mortgages. A government guarantee allowing a banker to draw on the treasury for cash was as good as hard money. 

      \begin{example}[Trajan Financial Bailout in 101 CE]
        Trajan's tax debt forgiveness. Later set up a vast system of charitable foundations in Italy, invested in mortgages, to support poor children and lend to small farmers. Also canceled the tax arrears of the Roman provinces, endearing himself to the Italian populace. 
      \end{example}

      From these two patterns, we can see that much of Rome's wealth was in the hands of private citizens. Roman conquest made the ruling classes of Rome rich. They had to invest their wealth somewhere, and it certainly trickled down through the credit system. However, credit does not align the interests of the borrower and lender—indeed it can lead to conflicts concerning default. In times of economic downturns, politicians turn to debt forgiveness to maintain stability.

  \subsection{Corporate Shareholder Structure}

    In contrast, a corporate structure puts shareholders on an equal footing. If benefits are paid out on a per-share basis, all investors see their wealth grow proportionately with corporate profits. When these shares are publicly traded, and especially when they can be held anonymously, they become a tool for negotiating among parties that might otherwise be in conflict. In the context of Roman politics, publican company shares offered a means by which power struggles among senators, knights, and emperors could be resolved. It is thus not surprising that, as the emperors consolidated their power, the usefulness of the publican companies declined. It is tempting to draw the conclusion that the corporate form first emerged as a means to negotiate and resolve conflicts between political and economic powers.

\section{Fall of the Western Roman Empire}

  Later, the Byzantine empire arose. 



\end{document}

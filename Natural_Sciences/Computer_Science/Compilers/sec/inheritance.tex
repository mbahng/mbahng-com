\section{Inheritance} 

  But so far, these classes are isolated from one another. 


  This is bad because ideally, we want the classes to actually have relationships with each other 

  \begin{definition}[Polymorphism] 
    % property 
    The ability to substitute objects of matching interface for one another at run-time. 
    The idea of \textit{single interface, multiple implementations}. 
  \end{definition}

  \begin{definition}[Toolkit]
    A \textbf{toolkit} is a set of related and reusable classes designed to provide useful, general-purpose functionality. 
  \end{definition}

  \begin{definition}[Framework]
    A \textbf{framework} is a set of cooperating classes that make up a reusable design for a specific class of software. It provides architectural guidance by partitioning the design into abstract classes and defining their responsibilities and collaborations. A developer customizes the framework to a particular application by subclassing and composing instances of framework classes. 
  \end{definition} 

  There are in general two ways: inheritance and object composition. 

  \begin{definition}[Interface Inheritance] 
    \textbf{Interface inheritance} defines a new interface in terms of one or more existing interfaces. 
  \end{definition}

  \begin{definition}[Implementation Inheritance]
    \textbf{Implementation inheritance} defines a new implementation in terms of one or more existing implementations. 
  \end{definition}  

  Class inheritance combines both interface and implementation inheritance, since the subclass.  

  Inheritance is not polymorphism. In inheritance, you get polymorphism when you cast it back to the base class. 

  \href{https://stackoverflow.com/questions/3392352/python-abcs-registering-vs-subclassing}{subclassing vs inheritance}. Invasive vs non-invasive. 
 
  \href{https://peps.python.org/pep-3119/#abcs-vs-duck-typing}{abcs vs duck typing in python (PEP 3119)}.

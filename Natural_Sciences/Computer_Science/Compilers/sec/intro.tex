
Compiling is the process of converting code that you write (essentially a giant string) into assembly, for a specific ISA. You can use a \textit{cross-compilation toolchain}, where you have a machine for one ISA (e.g. x86) but compile it into ARM with---say the \texttt{gccxarm} package. 

Grammars (regular, context-free). 

\begin{enumerate}
  \item \textit{Lexer}. Convert a sequence of characters into a sequence of tokens. Whitespace and comments are not tokens, since they get dropped out. Need to talk about DFA, NFA, and regular expressions. 
  \item \textit{Parsing}. Building the abstract syntax tree. e.g. LL parsing (what we are doing) vs LR parsing. MLyac. Trees make everything explicit and so is easier to work with. 
  \item \textit{Type Checking}. Just a bit of work. 
  \item \textit{IR}. Top of the mountain. 
  \item \textit{Instruction Selection}. 
  \item \textit{Liveness Analysis}. Data flow analysis, which is at the core of a lot of optimization. 
  \item \textit{Register Allocation}. This gives the MIPS assembly, which is text. 
\end{enumerate}

Expression evaluates to a value, while a statement is ...
SMl-NJ, sort of functional.

In the code below, \texttt{if} is an expression (i.e. evaluates to a value), not a statement (which perform an action). 
\begin{lstlisting}
  fun fact(n) = if (n <= 0) then 1 else n * fact(n-1)
\end{lstlisting}

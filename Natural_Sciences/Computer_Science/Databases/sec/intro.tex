This is a course on database languages (SQL, XML, JSON) and database management systems (Postgres, MongoDB). 

\begin{definition}[Data Model]
  A \textbf{data model} is a notation for describing data or information, consisting of 3 parts. 
  \begin{enumerate}
    \item \textit{Structure of the data}. The physical structure (e.g. arrays are contiguous bytes of memory or hashmaps use hashing). This is higher level than simple data structures. 
    \item \textit{Operations on the data}. Usually anything that can be programmed, such as \textbf{querying} (operations that retrieve information), \textbf{modifying} (changing the database), or \textbf{adding/deleting}. 
    \item \textit{Constraints on the data}. Describing what the limitations on the data can be. 
  \end{enumerate}
\end{definition}

There are two general types: relational databases, which are like tables, and semi-structured data models, which follow more of a tree or graph structure (e.g. JSON, XML). We'll cover in the following order: 
\begin{enumerate}
  \item The theory of relational algebra. 
  \item Practical applications with SQL. 
  \item Theory and practice of XML.  
  \item Theory and practice of JSON. 
\end{enumerate}


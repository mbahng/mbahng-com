\section{Decorators}

  Note that in Python, functions are first-class citizens, which means three things: 
  \begin{enumerate}
    \item They can be treated as objects. 
      \begin{lstlisting}
        def shout(text): 
          return text.upper() 

        print(shout('Hello'))  # HELLO 
        yell = shout 
        print(yell('Hello'))   # HELLO
      \end{lstlisting}
    \item They can be passed into another function as an argument. 
      \begin{lstlisting}
        def shout(text): 
          return text.upper() 

        def whisper(text): 
          return text.lower() 

        def greet(func): 
          greeting = func("Hi, How are You.")
          print (greeting) 

        greet(shout)    # HI, HOW ARE YOU.
        greet(whisper)  # hi, how are you. 
      \end{lstlisting}
    \item They can be returned by another function. 
      \begin{lstlisting}
        def create_adder(x): 
          def adder(y): 
            return x+y 

          return adder 

        add_15 = create_adder(15) 
        print(add_15(10)) # 25 
      \end{lstlisting}
  \end{enumerate}

  Say that you have a function \texttt{f} that does something. I want to modify the behavior so that I do something either before of after \texttt{f} is called automatically, but I don't want to manually add code into the function body. What I can do is simply define another function \texttt{wrapper} and call \texttt{f} inside it. 
  \begin{lstlisting}
    def f(): 
        print("Hello world") 

    def wrapper(): 
        print("started") 
        f()
        print("ended") 

    wrapper() # "started\n Hello world\n ended"
  \end{lstlisting}

  Great, we can do this for one function. But what if there were thousands of functions I want to do this for? Rather than creating a wrapper function for each function, I can make a third function called \texttt{decorator} that takes in the original function \texttt{f} and outputs the \texttt{wrapper} function. 

  \begin{lstlisting}
    def decorator(f): 
      def wrapper(): 
        print("started") 
        f()
        print("ended") 

      return wrapper

    def f(): 
      print("Hello world") 

    wrapper = decorator(f)
    wrapper() # "started\n Hello world\n ended"

    decorator(f) # <function decorator.<locals>.wrapper at 0x100b38e00>
    decorator(f)() # "started\n Hello world\n ended"
  \end{lstlisting}

  This way, I can modify any function I want with this behavior, and is known as \textit{function aliasing}. This is essentially what a decorator is. 

  \begin{definition}[Decorators]
    \textbf{Decorators} are used to modify the behavior of your functions without changing its actual code, used with the \texttt{\@} operator. The two are equivalent. 

    \noindent\begin{minipage}{.5\textwidth}
    \begin{lstlisting}[]{Code}
      def decorator(f): 
        def wrapper(): 
          print("started") 
          f()
          print("ended") 

        return wrapper

      def f(): 
        print("Hello world") 

      f = decorator(f)
      f() # "started\n Hello world\n ended"
    \end{lstlisting}
    \end{minipage}
    \hfill
    \begin{minipage}{.49\textwidth}
    \begin{lstlisting}[]{Output}
      def decorator(f): 
        def wrapper(): 
          print("started") 
          f()
          print("ended") 

        return wrapper

      @decorator
      def f(): 
        print("Hello world") 

      f() # "started\n Hello world\n ended"
    \end{lstlisting}
    \end{minipage}

    This means that every time I call the function \texttt{f}, it really calls the function \texttt{decorator} with \texttt{f} passed into it as an argument. With functions that have arguments, the wrapper function should also have the same arguments. Generically, we can just use the \texttt{\*args} and \texttt{\*\*kwargs} arguments to unpack these variables so that \texttt{wrapper}'s arguments always matches those of \texttt{f}'s arguments, but we can modify these arguments for extra functionality as well. 

    \noindent\begin{minipage}{.5\textwidth}
    \begin{lstlisting}[]{Code}
      # generic args and kwargs
      def decorator(f): 
        def wrapper(*args, **kwargs): 
          print("started") 
          f(*args, **kwargs)
          print("ended") 

        return wrapper

      @decorator
      def f(string): 
        print(string) 

      f("Hello World")
      # started
      # Hello World
      # ended
    \end{lstlisting}
    \end{minipage}
    \hfill
    \begin{minipage}{.49\textwidth}
    \begin{lstlisting}[]{Output}
      # custom arguments 
      def decorator(f): 
        def wrapper(string, start_msg): 
          print(start_msg) 
          f(string)
          print("ended") 

        return wrapper

      @decorator
      def f(string): 
        print(string) 

      f("Hello World", "time to go")
      # time to go
      # Hello World
      # ended
    \end{lstlisting}
    \end{minipage}
    If we want to get the return values of this function, we can store the return value in temporary variable \texttt{tmp}, run whatever code after the function \texttt{f}, and finally return \texttt{tmp} in \texttt{wrapper}. 

    \begin{lstlisting}
      def decorator(f): 
          def wrapper(*args, **kwargs): 
              print("started") 
              tmp = f(*args, **kwargs)
              print("ended") 
              return tmp

          return wrapper

      @decorator
      def f(string): 
          return string + "!"

      print(f("Hello World"))
      # started
      # ended
      # Hello World!
    \end{lstlisting}
  \end{definition}

  \begin{example}[Measuring Total and CPU Runtime]
    If we want to find the runtime of a function, we can do this easily. 

    \begin{lstlisting}
      import time 

      def runtime(f): 
        def wrapper(*args, **kwargs): 
          start = time.time() 
          product = f(*args, **kwargs) 
          end = time.time() 
          print(f"Took {end - start} s") 
          return product
        return wrapper

      @runtime
      def dot(list1, list2): 
        res = 0 
        for x, y in zip(list1, list2): 
          res += x * y 
        return res

      x = [1, 2, 3]
      y = [2, 2, 3]
      result = dot(x, y)  # Took 3.814697265625e-06 s 
      print(result)       # 15 
    \end{lstlisting}

    However, this is not accurate as the OS will switch between different processes. Therefore, the process time is more accurate. 

    \begin{lstlisting}
      import numpy as np
      import time

      def cpu_usage(f):
        def wrapper(*args, **kwargs):
          start_cpu = time.process_time()
          result = f(*args, **kwargs)
          end_cpu = time.process_time()
          print(f"CPU time: {end_cpu - start_cpu:.6f} seconds")
          return result
        return wrapper

      @cpu_usage
      def matrix_mult(a, b): 
        return np.matmul(a, b)

      x = np.random.randn(2000, 2000)

      matrix_mult(x, x) # CPU time: 0.772730 seconds
    \end{lstlisting}
  \end{example}

  \begin{example}[Memory Usage]
    We can measure memory usage with the \texttt{psutil} library. 
    \begin{lstlisting}
      import numpy as np
      import psutil, os 

      def memory_usage(f):
        def wrapper(*args, **kwargs):
          process = psutil.Process(os.getpid())
          mem_before = process.memory_info().rss
          result = f(*args, **kwargs)
          mem_after = process.memory_info().rss
          print(f"Memory usage: {(mem_after - mem_before) / 1024 / 1024:.2f} MB")
          return result
        return wrapper

      @memory_usage 
      def matrix_mult(a, b): 
        return np.matmul(a, b)

      x = np.random.randn(2000, 2000)
      matrix_mult(x, x) # Memory usage: 46.81 MB
    \end{lstlisting}
  \end{example}

  \begin{example}[Measuring Function Call Count]
    To measure how many times a function has been called, we can use the decorator. 
    \begin{lstlisting}
      def call_counter(f):
          def wrapper(*args, **kwargs):
              wrapper.count += 1
              print(f"Function '{f.__name__}' called {wrapper.count} times")
              return f(*args, **kwargs)
          wrapper.count = 0
          return wrapper

      @call_counter
      def factorial(x): 
          if x == 1: 
              return 1 
          return x * factorial(x - 1)

      result = factorial(7)
      # Function 'factorial' called 1 times
      # Function 'factorial' called 2 times
      # Function 'factorial' called 3 times
      # Function 'factorial' called 4 times
      # Function 'factorial' called 5 times
      # Function 'factorial' called 6 times
      # Function 'factorial' called 7 times
      print(result)       
      # 5040
    \end{lstlisting}
  \end{example}

  functools.wraps. 


\section{Intractability}

\subsection{Uncomputable Functions}

  Even though NAND-CIRC programs can compute every finite function $f: \{0,1\}^n \longrightarrow \{0,1\}$, NAND-TM programs can \textit{not} compute every function $F: \{0,1\}^* \longrightarrow \{0,1\}$. That is, there exists such a function that is \textit{uncomputable}! 

  \begin{definition}
  Let $HALT: \{0,1\}^* \longrightarrow \{0,1\}$ be the function such that for every string $M \in \{0,1\}^*$, $HALT(M, x) = 1$ if Turing machine $M$ halts on the input $x$ and $HALT(M, x) = 0$ otherwise. 
  \end{definition}

  \begin{theorem}
  The $HALT$ function is not computable. This leads to many other functions also being uncomputable. 
  \end{theorem}

  It is surprising that such a simple program is actually uncomputable. That is, there is no \textit{general procedure} that would determine for an \textit{arbitrary} program $P$ whether it halts or not. 

\subsection{Impossibility of General Software Verification}

  \begin{definition}
  Let there be a program $P$ that computes a function. A \textbf{semantic property} or \textbf{semantic specification} of a program means properties of the \textit{function} that the program computes, as opposed to the properties that depend on the particular syntax/code used by the program. 
  \end{definition}

  \begin{example}
  A semantic property of a program $P$ is the property that whenever $P$ is given an input string with an even number of $1$'s, it outputs $0$. Another example is the property that $P$ will always halt whenever the input ends with a $1$. 

  In contrast the property that a C program contains a comment before every function declaration is not a semantic property, since it depends on the actual source code as opposed to the input/output relation. 
  \end{example}

  \begin{example}
  Consider the following two C programs: 
  \begin{lstlisting}
  int First(int n) {
      if (n<0) return 0; 
      return 2*n;
  }

  int Second(int n) {
      int i = 0;
      int j = 0
      if (n<0) return 0; 
      while (j<n) {
          i = i + 2;
          j= j + 1; 
      }
      return i; 
  }
  \end{lstlisting}
  $\texttt{First}$ and $\texttt{Second}$ are two distinct C programs, but they compute the same function. Therefore, a \textit{semantic property} would either be true for both programs or false for both, since it depends on the function the programs compute. One example of a semantic property is: \textit{The program $P$ computes a function $f$ mapping integers to integers satisfying that $f(n) \geq n$ for every input $n$.} 

  A property is \textit{not semantic} if it depends on the source code rather than the input/output behavior. An example of this would be: \textit{The program contains the variable $\texttt{k}$} or \textit{the program uses the $\texttt{while}$ operation}. 
  \end{example}

  \begin{definition}[Semantic properties]
  A pair of Turing machines $M$ and $M^\prime$ are \textbf{functionally equivalent} if for every $x \in \{0,1\}^*$, $M(x) = M^\prime (x)$ (including when the function outputs $\perp$). 

  A function $F: \{0,1\}^* \longrightarrow \{0,1\}$ is \textbf{semantic} if for every pair of strings $M, M^\prime$ that represent functionally equivalent Turing machines, $F(M) = F(M^\prime)$. Note that we assume that every string represents \textit{some} Turing machine. 
  \end{definition}

  We now present a theorem concerning the Halting problem (the problem of determining whether a Turing machine will halt or not on any arbitrary input). The Halting problem also turns out to be a linchpin of uncomputability. 

  \begin{theorem}[Rice's Theorem]
  Let $F: \{0,1\}^* \longrightarrow \{0,1\}$. If $F$ is semantic and nontrivial, then it is uncomputable. 
  \end{theorem}

  \begin{corollary}
  The following function is uncomputable: 
  \[COMPUTES-PARITY(P) = \begin{cases}1 & P \text{ computes the parity function} \\
  0 & \text{else}
  \end{cases}\]
  \end{corollary}

  Therefore, we can see that the set $\mathbf{R}$ of computable Boolean functions is a proper subset of the set of all functions mapping $\{0, 1\}^* \longrightarrow \{0, 1\}$. 


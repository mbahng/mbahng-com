\section{Text Editing with Neovim} 

  The first thing you do when coding is typing something, and this requires a text editor. Vim is guaranteed to be on every Linux system, so there is no need to install it. However, you may have to install Neovim (which is just a command away). Vim can be a really big pain in the ass to learn, but I got into it when I was watching some video streams from a senior software engineer at Netflix called The Primeagen. He moved around the code like I've never seen, and I was pretty much at the limit of my typing speed, so I decided to give it a try during the 2023 fall semester. My productivity plummetted during the first 2 days (which was quite scary given that I had homework due), but within a few weeks I was faster than before, so if you have the patience, I would recommend learning it. Here is a summary of reasons why I would recommend learning Vim: 
  \begin{enumerate}
    \item It pushes you to know the ins and outs of your editor. As a mechanic with his tools, a programmer should know exactly how to configure their editor.  
    \item The plugin ecosystem is much more diverse than other editors such as VSCode. You can find plugins/extensions for everything. Here is a summmary of them \href{https://github.com/rockerBOO/awesome-neovim\#neovim-lua-development}{here}. 
    \item You're faster. If you're going to be coding for say the next 10 years, then why not spend a month to master something that will make you faster by 10\%? That way, you'll have coded 1 years worth more with a 1 month commitment. I'd take a free 11 months of coding any day. 
    \item Computing clusters and servers will be much easier to navigate since they all run Linux with Vim. 
    \item Vim is lightweight, and you don't have to open up VSCode every time you want to edit a configuration file.  
  \end{enumerate}

  \begin{example}[Vim vs Neovim]
    Experience wise, Vim and Neovim are very similar, and if you configure things rihght, you may not even be able to tell the difference. But there are 3 differences that I want to mention: 
    \begin{enumerate}
      \item Neovim can be configured in Lua, which is much cleaner than Vimscript. 
      \item Neovim provides mouse control right out of the box, which is convenient for me at times and can be easier to transition into, while Vim does not provide any mouse support. 
      \item There are some plugins that are provided in Neovim that are not in Vim. 
    \end{enumerate}
    Either way, the configuration is essentially the same. At startup, the text editor will parse some predetermined configuration file and load those settings. 
  \end{example}

  It may be the case that a remote server does not have neovim installed, or you may not have the permissions to install it. In this case, you can use \textbf{sshfs}, which is a file system client based on the SSH File Transfer Protocol. It allows you to mount a remote directory over SSH. 

\subsection{Configuration Files}

  In Vim, your configuration files are located in \texttt{~/.vimrc} and plugins are located in \texttt{~/.vim/}. In here, you can put in whatever options, keymaps, and plugins you want. All the configuration is written in VimScript. 

  \begin{lstlisting} 
    # options 
    filetype plugin indent on 
    syntax on 
    set background=dark
    set expandtab ts=2 sw=2 ai
    set nu
    set linebreak 
    set relativenumber        
    
    # keymaps
    inoremap <C-j> <esc>dvbi
    inoremap jk <esc>
    nnoremap <C-h> ge
    nnoremap <C-l> w 
  \end{lstlisting}

  In Neovim, I organize it using Lua. It essentially looks for the \texttt{~/.config/nvim/init.lua} file and loads the options from there. We also have the option to import other Lua modules for better file structure with the \texttt{require} keyword. The tree structure of this configuration file should be the following below. The extra \texttt{user} director layer is necessary for isolating configuration files on multiple user environments.  
  
  The init file is the ``main file'' which is parsed first. I generally don't put any explicit options in this file and reserve it only for require statements. It points to the following (group of) files: 
  \begin{enumerate}
    \item \textbf{options.lua}: This is where I store all my options. 
    \item \textbf{keymaps.lua}: All keymaps. 
    \item \textbf{plugins.lua}: First contains a script to automatically install packer if it is not there, and then contains a list of plugins to download. 
    \item \textbf{Plugin Files}: Individual configuration files for each plugin (e.g. if I install a colorscheme plugin, I should choose which specific colorscheme I want from that plugin). 
    \item \textbf{Filetype Configuration Files}: Options/keymaps/plugins to load for a specific filetype. This helps increase convenience and speed since I won't need plugins like VimTex if I am working in JavaScript. 
  \end{enumerate}

  Once you have your basic options and keymaps done, you'll be spending most of your time experimenting with plugins. It is worth to mention some good ones that I use. 
  \begin{enumerate}
    \item \textbf{Packer} as the essential package manager.  
    \item \textbf{Plenary} 
    \item \textbf{Telescope} for quick search and retrieval of files.  
    \item \textbf{Indent-blankline} for folding. 
    \item \textbf{Neoformat} for automatic indent format. 
    \item \textbf{Autopairs} and \textbf{autotag} to automatically close quotation marks and parantheses. 
    \item \textbf{Undotree} to generate and navigate undo history. 
    \item \textbf{Vimtex} for compilation of LaTeX documents. 
    \item \textbf{Onedark} and \textbf{Oceanic Next} for color schemes. 
    \item \textbf{Vim-Startify} for nice looking neovim startup. 
    \item \textbf{Comment} for commenting visual blocks of code. 
  \end{enumerate}

  It is also worthwhile to see how they are actually loaded in the backend. Each plugin is simply a github repo that has been cloned into \texttt{~/.local/share/nvim/site/pack/packer/}, which contains two directories. The packages in \texttt{start/} are loaded up every time Neovim starts, and those in \texttt{opt/} are packages that are loaded up when a command is called in a certain file (known as lazy loading). Therefore, if you have any problems with Neovim, you should probably look into these folders (and possibly delete them and reinstall them using Packer if needed).

\subsection{Troubleshooting}

  A good test to run is \texttt{:checkhealth}, which checks for any errors or warnings in your Neovim configuration. You should aim to have every (non-optional) warning cleared, which usually involves having to install some package, making it executable and/or adding to \texttt{\$PATH}. 

  If you are getting plugin errors, you can also manually delete the plugin directory in `pack/packer` and run `PackerInstall` to re-pull the repos. This may help. 

\subsection{Language Service Providers} 

  If you were to create a text editor from scratch, you would first want to make a buffer and some external program to analyze this buffer (plus some other text files) concurrently. Things like autocompletion, type checking, and syntax checking may all be taken for granted, but it's not, and these are all provided by the \textbf{language service provider}, also known as \textbf{LSP}. LSPs are specific to each language, such as \texttt{pyright} being the mainstream LSP for Python, and \texttt{ts\_ls} for TypeScript. Some of its services have specific names, and overlap a lot. 
  \begin{enumerate}
    \item \textit{Autocompleting} partially typed words with suggestions based on what you typed so far in the current buffer, or from analyzing existing paths of various directories/files. 
    \item \textit{Linting}, which is a general term for finding issues in your code. 
    \item \textit{Type checking} the correct types of variables to find bugs or edge cases in your code. 
    \item \textit{Symbol searching} variables so that you can jump to where they are declared or defined. 
  \end{enumerate}

  The tricky part about LSPs is that they can get quite heavy in computation. For modern laptops this isn't really a problem. For example, on my Macbook Pro M3 I can have a heavy type checker, full autocompletion of every word,  symbol searching of every variable, and linting across \textit{all} files in my current directory (of up to 50 files), all with no noticeable delay. This was quite nice, until I started working on a remote server offering 4 crappy CPUs to work off of, and this just made coding impossible since all of these processes caused a 1 second delay in my writing. Therefore, depending on where you work, LSPs should be lightweight. The balance between functionality and performance is what I think VSCode does very well compared to Neovim. 

\subsection{Snippets}


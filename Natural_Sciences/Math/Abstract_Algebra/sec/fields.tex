\section{Fields}  

  Our final structure is field, which seems to add only a few more conditions to a ring, but again unlocks more structure. Field theory is usually pretty tame compared to groups and rings. The purpose of this section is to really just introduce the definition of a field, plus the construction of the field of fractions---which will be useful for the construction of $\mathbb{Q}$ and the analysis of UFD polynomial rings---and finally ordered fields, which is nice again in the context of $\mathbb{Q}$ and $\mathbb{R}$. 
  
  \begin{definition}[Field]
    A \textbf{field} $(F, +, \times)$ is a commutative, associative ring where every nonzero element is a unit. 
  \end{definition}

  \begin{lemma}[Properties of Addition]
    The properties of addition hold in a field. 
    \begin{enumerate}
      \item If $x + y = x + z$, then $y = z$. 
      \item If $x + y = x$, then $y = 0$. 
      \item If $x + y = 0$, then $y = -x$. 
      \item $(-(-x)) = x$. 
    \end{enumerate}
  \end{lemma}
  \begin{proof}
    For the first, we have 
    \begin{align}
      x + y = x + z & \implies -x + (x + y) = -x + (x + z) && \tag{addition is a function} \\
                    & \implies (-x + x) + y = (-x + x) + z && \tag{$+$ is associative} \\
                    & \implies 0 + y = 0 + z && \tag{definition of additive inverse} \\
                    & \implies y = z && \tag{definition of identity}
    \end{align} 
    For the second, we can set $z = 0$ and apply the first property. For the third, we have 
    \begin{align}
      x + y = 0 & \implies -x + (x + y) = -x + 0 && \tag{addition is a function} \\
                & \implies (-x + x) + y = -x + 0 && \tag{$+$ is associative} \\
                & \implies 0 + y = -x + 0 && \tag{definition of additive inverse} \\
                & \implies y = -x && \tag{definition of identity}
    \end{align}
    For the fourth, we simply follow that if $y$ is an inverse of $z$, then $z$ is an inverse of $y$. Therefore, $-x$ being an inverse of $x$ implies that $x$ is an inverse of $-x$. $-(-x)$ must also be an inverse of $-x$. Since inverses are unique\footnote{This is proved in algebra.}, $x = -(-x)$. 
  \end{proof}

  \begin{lemma}[Properties of Multiplication]
    The properties of multiplication hold in a field. 
    \begin{enumerate}
      \item If $x \neq 0$ and $xy = xz$, then $y = z$. 
      \item If $x \neq 0$ and $xy = x$, then $y = 1$. 
      \item If $x \neq 0$ and $xy = 1$, then $y = x^{-1}$. 
      \item If $x \neq 0$, then $(x^{-1})^{-1} = x$. 
    \end{enumerate}
  \end{lemma}
  \begin{proof}
    The proof is almost identical to the first. Since $x \neq 0$, we can always assume that $x^{-1}$ exists. For the first, we have
    \begin{align}
      x y = x z & \implies x^{-1} (x y) = x^{-1} (x z) && \tag{multiplication is a function} \\
                & \implies (x^{-1} x) y = (x^{-1} x) z && \tag{$\times$ is associative} \\
                & \implies 1 y = 1 z && \tag{definition of multiplicative inverse} \\  
                & \implies y = z && \tag{definition of identity}
    \end{align}
    For the second, we can set $z = 1$ and apply the first property. For the third, we have 
    \begin{align}
      xy = 1 & \implies x^{-1} (x y) = x^{-1} 1 && \tag{multiplication is a function} \\
             & \implies (x^{-1} x) y = x^{-1} 1 && \tag{$\times$ is associative} \\
             & \implies 1 y = x^{-1} 1 && \tag{definition of multiplicative inverse} \\
             & \implies y = x^{-1} && \tag{definition of identity}
    \end{align}
    For the fourth, we simply see that $x^{-1}$ is a multiplicative inverse of both $x$ and $(x^{-1})^{-1}$ in the group $(\mathbb{F} \setminus \{0\}, \times)$, and since inverses are unique, they must be equal. 
  \end{proof}

  \begin{lemma}[Properties of Distribution]
    For any $x, y, z \in \mathbb{F}$, the field axioms satisfy 
    \begin{enumerate}
      \item $0 \cdot x = 0$.
      \item If $x \neq 0$ and $y \neq 0$, then $x y \neq 0$.
      \item $-1 \cdot x = -x$. 
      \item $(-x) y = - (xy) = x (-y)$. 
      \item $(-x) (-y) = xy$. 
    \end{enumerate}
  \end{lemma} 
  \begin{proof}
    For the first, note that 
    \begin{align}
      0 x & = (0 + 0) \cdot x = 0 x + 0x 
    \end{align}
    and subtracting $0x$ from both sides gives $0 = 0x$. For the second, we can claim that $xy \neq 0$ equivalently claiming that it will have an identity. Since $x, y \neq 0$, their inverses exists, and we claim that $(xy)^{-1} = y^{-1} x^{-1}$ is an inverse. We can see that by associativity, 
    \begin{equation}
      (y^{-1} x^{-1}) (xy) = y^{-1} (x^{-1} x) y = y^{-1} y = 1
    \end{equation} 
    For the third, we see that 
    \begin{equation}
      0 = 0 \cdot x = (1 + (-1)) \cdot x = 1 \cdot x + (-1) \cdot x = x + (-1) \cdot x 
    \end{equation}
    which implies that $-1 \cdot x$ is the additive inverse. The fourth follows immediately from the third by the associative property. For the fifth we can see that 
    \begin{align}
      (-x) (-y) & = (-1) x (-1) y && \tag{property 3} \\
                & = (-1) (-1) x y && \tag{$\times$ is commutative} \\
                & = -1 \cdot (-xy) && \tag{property 3} \\
                & = -(-xy) && \tag{property 3} \\
                & = xy && \tag{addition property 4}
    \end{align}
  \end{proof}

  \begin{theorem}[Fields are Euclidean Domains]
    Every field is a Euclidean domain. 
  \end{theorem}
  \begin{proof}
    Given $x, y \in \mathbb{F}$, assume $x y = 0$ with $x \neq 0$. Since $x$ is invertible,
    \begin{equation}
      0 = x^{-1} 0 = x^{-1} (x y) = y
    \end{equation}
    Now assuming that $y \neq 0$, since $y$ is invertible, 
    \begin{equation}
      0 = 0 y^{-1} = (x y) y^{-1} = x
    \end{equation}
  \end{proof}

  With this theorem, we have established the hierarchy in the beginning of this section. So as soon as we see a field, we can immediately apply everything we know, such as Euclidean division, unique factorization, GCDs, etc. The converse is not generally true except for finite fields. 

  \begin{theorem}[Wedderburn's little theorem]
    Every finite integral domain is a field. 
  \end{theorem} 
  \begin{proof}
    Let $R$ be a finite integral domain and $a \in R$ be nonzero. Since every element is regular, the map $x \mapsto ax$ is an injective function. Since $R$ is finite this map is also surjective. In particular, there is some $b \in R$ s.t. $ab = 1$, i.e. $a$ is a unit. 
  \end{proof}

  \begin{theorem}[Integral Domains are Embedded in Fields]
    An integral domain is a ring that is isomorphic to a subring of a field. 
  \end{theorem}
  \begin{proof}
    TBD
  \end{proof}

  \begin{theorem}[Ideals of Fields]
    The only ideals that exist in a field $\mathbb{F}$ is $\{0\}$ and $\mathbb{F}$ itself. 
  \end{theorem}
  \begin{proof}
    Given a nonzero element $x \in \mathbb{F}$, every element of $\mathbb{F}$ can be expressed in the form of $a x$ or $x a$ for some $a \in \mathbb{F}$. 
  \end{proof}

  The ring $\mathbb{Z}_n$ has all the properties of a field except the property of having inverses for all of its nonzero elements. This leads to the following theorem. 

  \begin{theorem}[Quotient Rings as Fields]
    \label{thm:quotient_ring_fields}
    Let $R$ be a nontrivial commutative ring and $I \subset R$ an ideal. $R/I$ is a field iff $I$ is a maximal ideal. 
  \end{theorem}
  \begin{proof}
    TBD
  \end{proof}

  \begin{corollary}[Integer Quotient Rings as Finite Fields]
    The ring $(\mathbb{Z}_{n}, +, \times)$ is a field if and only if $n$ is a prime number. 
  \end{corollary}
  \begin{proof}
    This proof is a one-liner given the previous theorem, but let's provide an alternative proof. 
    \begin{enumerate}
      \item $(\rightarrow)$ We prove the contrapositive. Assume that $n$ is composite $\implies n = k l$ for $k, n \in \mathbb{N} \implies k, n \neq 0$, but 
      \begin{equation}
        [k]_n [l]_n = [k l]_n = [n]_n = 0
      \end{equation}
      meaning that $\mathbb{Z}_n$ contains $0$ divisors and is not a field. 

      \item $(\leftarrow)$ Given that $n$ is prime, let $[a]_n \neq 0$, i.e. $[a]_n \neq [0]_n, [1]_n$. The set of $n$ elements 
      \begin{equation}
        [0]_n, [a]_n, [2a]_n, ..., [(n-1)a]_n
      \end{equation}
      are all distinct. Indeed, if $[k a]_n = [l a]_n$, then $[(k-l) a]_n = 0 \implies n = (k-l) a \iff n$ is not prime. Since the elements are distinct, exactly one of them must be $[1]_n$, say $[p a]_n \implies$ the inverse $[p]_n$ exists. 
    \end{enumerate}
  \end{proof}

  \begin{corollary}[Invertibility in $\mathbb{Z}_n$]
    For any $n$, $[k]_n$ is invertible in the ring $\mathbb{Z}_n$ if and only if $n$ and $k$ are relatively prime. 
  \end{corollary} 
  \begin{proof}
    TBD
  \end{proof}

  We will talk about finite fields again, which are extremely important in Galois theory and in practical applications in e.g. cryptography. 

\subsection{Field of Fractions and the Rationals}

  Given an integral domain, there is a common way to construct a field from it. We simply just ``add'' all the multiplicative inverses. Doing so with the integers and polynomials creates the field of rational numbers and rational functions. To formalize this a bit more, we claim that any integral domain $R$ is a subring of a larger field $F$. Actually, we can prove an even stronger claim about \textit{commutative rings}, i.e. every commutative ring is a subring of a larger ring $S$ in which every nonzero element of $R$ that is not a zero divisor is a unit in $S$. 

  \begin{definition}[Ring/Field of Fractions]
    Let $R$ be a commutative ring and $D$ be any nonempty subset of $R$\footnote{This is basically a subgroup of units.} that 
    \begin{enumerate}
      \item does not contain $0$, 
      \item does not contain zero divisors, 
      \item is closed under multiplication. 
    \end{enumerate}
    Then, there exists a commutative subring $Q \supset R$, called the \textbf{ring of fractions}, with the properties. 
    \begin{enumerate}
      \item \textit{Fraction}. Every element of $Q$ is of the form $r d^{-1}$, for some $r \in R, d \in D$. If $D = R \setminus \{0\}$, then $Q$ is a field, called the \textbf{field of fractions}. 
      \item \textit{Uniqueness}. $Q$ is the smallest ring containing $R$ in which every elements of $D$ become units in the following sense. Let $S$ be any commutaticve ring with identity and let $\varphi: R \to S$ be any injective ring homomorphism such that $\varphi(d)$ is a unit in $S$ for every $d \in D$. Then there is an injective homomorphism $\phi: Q \to S$ s.t. $\phi |_R = \varphi$.\footnote{In other words, any ring containing an isomorphic copy of $R$ in which all the elements of $D$ become units must also contain an isomorphic copy of $Q$.}
    \end{enumerate}
  \end{definition}
  \begin{proof}
    Let $F = \{(r, d) \in R \times D \mid r \in R, d \in D \}$ and define the relation $\sim$ on $F$ by 
    \begin{equation}
      (r, d) \sim (s, e) \iff re = sd
    \end{equation}
    It is indeed an equivalence relation. 
    \begin{enumerate}
      \item \textit{Reflexive}. $(r, d) \sim (r, d)$ since $rd = rd$. 
      \item \textit{Symmetric}. Let $(r, d) \sim (s, e)$. Then $re = sd \iff sd = re$, and so $(s, e) \sim (r, d)$. 
      \item \textit{Transitive}. Let $(r, d) \sim (s, e)$ and $(s, e) \sim (t, f)$. Then $re - sd = 0$ and $sf - te = 0$. Multiplying the first and second equations by $f$ and $d$ respectively and adding them gives $(rf - td) e = 0$. $e \neq 0$ or $e$ is not a zero divisor, so $rf = td \iff (r, d) \sim (t, f)$. 
    \end{enumerate}
    Let us denote the equivalence class of $(r, d)$ as $\frac{r}{d}$, and let $Q$ be the set of equivalence classes under $\sim$. Note that $\frac{r}{d} = \frac{re}{de}$ for all $e \in D$, since $D$ is closed under multiplication.\footnote{This is why we can't multiply by $\frac{0}{0}$.} We now define addition and multiplication on $Q$ as 
    \begin{equation}
      \frac{a}{b} + \frac{c}{d} \coloneqq \frac{ad + bc}{bd}, \qquad \frac{a}{b} \times \frac{c}{d} = \frac{ac}{bd}
    \end{equation}
    which is again well defined since $D$ is closed under multiplication and do not depend on the choice of representatives from the equivalence class. 
    \begin{enumerate}
      \item Verify the additive identity. 
      \begin{equation}
        (a, b) + (0, c) = (ac + 0b, bc) = (ac, bc) \sim (a, b)
      \end{equation}
      \item Verify the multiplicative identity. 
      \begin{equation}
        (a, b) \times (c, c) = (ac, bc) \sim (a, b)
      \end{equation}
      \item Additive inverse is actually an inverse. 
      \begin{equation}
        (a, b) + (-a, b) = (ab + (-ba), bb) = (0, bb) \sim (0, 1)
      \end{equation}
      \item Multiplicative inverse is actually an inverse. 
      \begin{equation}
        (a, b) \times (b, a) = (ab, ba) = (ab, ab) \sim (1, 1)
      \end{equation}
      \item Addition is commutative. 
      \begin{equation}
        (a, b) + (c, d) = (ad + bc, bd) = (cb + ad, bd) = (c, d) + (a, b)
      \end{equation}
      \item Addition is associative. 
      \begin{align}
        (a, b) + ((c, d) + (e, f)) & = (a, b) + (cf + de, df) \\
                                   & = (adf + bcf + bde, bdf) \\
                                   & = (ad + bc, bd) + (e, f) \\
                                   & = ((a, b) + (c, d)) + (e, f)
      \end{align}
      \item Multiplication is commutative. 
      \begin{equation}
        (a, b) \times (c, d) = (ac, bd) = (ca, db) = (c, d) \times (a, b)
      \end{equation}
      \item Multiplication is associative. 
      \begin{align}
        (a, b) \times ((c, d) \times (e, f)) & = (a, b) \times (ce, df) \\ 
                                             & = (ace, bdf) \\
                                             & = (ac, bd) \times (e, f) \\
                                             & = ((a, b) \times (c, d)) \times (e, f)
      \end{align}
      \item Multiplication distributes over addition. 
        \begin{align}
          (a, b) \times ((c, d) + (e, f)) & = (a, b) \times (c, d) + (a, b) \times (e, f) \\
                                          & = (ac, bd) + (ae, bf) \\
                                          & = (abcf + abde, b^2 df) \\
                                          & = (acf + ade, bdf) \\
                                          & = (a, b) \times (cf + de, df)
        \end{align}
    \end{enumerate}
    Next we embed $R$ into $Q$ by defining 
    \begin{equation}
      \iota: R \to Q, \qquad \iota(r) = \frac{rd}{d} \text{ for any } d \in D
    \end{equation}
    Since $\frac{rd}{d} = \frac{re}{e}$ for all $d, e \in D$, $\iota(r)$ does not depend on the choice of $d \in D$ (for now, we assume that $1 \in D$ and continue the proof with this assumption, though the general case is also pretty simple). Since $D$ is closed under multiplication, one checks directly that $\iota$ is a ring homomorphism. 
    \begin{enumerate}
      \item Preservation of addition. 
        \begin{align}
          \iota(a) +_{Q} \iota(b) & = (a, 1) +_{Q} (b, 1) \\
                                           & = (1a +_{R} 1b, 1^2) \\
                                           & = (a +_{R} b, 1) \\
                                           & = \iota(a +_{R} b) 
        \end{align}
      \item Preservation of multiplication. 
        \begin{align}
          \iota(a) \times_{Q} \iota(b) & = (a, 1) \times_{Q} (b, 1) \\
                                                & = (a \times_{R} b, 1^2) \\
                                                & = (a \times_{R} b, 1) \\
                                                & = \iota(a \times_{R} b, 1)
        \end{align}
      \item Preservation of multiplicative identity. 
        \begin{equation}
          \iota(1_{R}) = (1, 1) = 1_{Q}
        \end{equation}
    \end{enumerate}
    We then prove $\iota$ is injective since 
    \begin{equation}
      \iota(r) = 0 \iff \frac{rd}{d} = \frac{0}{d} \iff rd^2 = 0 \iff r = 0
    \end{equation}
    because $d$---and hence $d^2$--- is neither $0$ nor a zero divisor. We therefore have $\iota(R) \simeq Q$. 

    Next, note that each $d \in D$ has a multiplicative inverse in $Q$. That is, if $d$ is represented by the fraction $\frac{de}{e}$, then its multiplicative inverse is $\frac{e}{de}$. Therefore, every element of $Q$ of the form $r d^{-1}$ for some $r \in R, d \in D$.\footnote{Since $(r, d) = (r, \frac{de}{e}) = (r \cdot \frac{e}{de}, 1) = (r d^{-1}, 1) \in Q$.} It follows immediately that if $D = R \setminus \{0\}$, then every nonzero element of $Q$ is a unit and so $Q$ is a field.  

    For the uniqueness property, assume that $\varphi: R \to S$ is an injective ring homomorphism such that $\varphi(d)$ is a unit in $S$ for all $d \in D$. Extend $\varphi$ to a map $\phi: Q \to S$ by defining 
    \begin{equation}
      \phi(r d^{-1})  = \varphi(r) \varphi(d)^{-1} 
    \end{equation}
    for all $r \in R, d \in D$. This map is well defined, since $r d^{-1} = s e^{-1}$ implies $re = sd \implies \varphi(r) \varphi(e) = \varphi(s) \varphi(d)$, and so 
    \begin{equation}
      \phi(r d^{-1}) = \varphi(r) \varphi(d)^{-1} = \varphi(s) \varphi(e)^{-1} = \phi(s e^{-1})
    \end{equation}
    $\phi$ is indeed a ring homomorphism, and it is injectivce since $rd^{-1} \in \ker{\phi}$ implies $r \in \ker{\phi} \cap R = \ker{\varphi}$. Since $\varphi$ is injective this forces $r$ and hence $rd^{-1}$ to be $0$. 
  \end{proof}

  Therefore, non-zero divisors get upgraded to units, while zero-divisors... well stay the same. The relative properties of these are quite simple, since $R$ being an integral domain implies that it will have a \textit{field} of fractions. We now derive the rational numbers as a field of fractions of the integers, which is straightforward. 

  \begin{definition}[Rational Numbers]
    The field of fractions of $\mathbb{Z}$ is called the \textbf{rational numbers}. More specifically, the \textbf{rational numbers} $(\mathbb{Q}, +_{\mathbb{Q}}, \times_{\mathbb{Q}})$ is the quotient space on $\mathbb{Z} \times \mathbb{Z} \setminus \{0\}$ with the equivalence relation $\sim$ 
    \begin{equation}
      (a, b) \sim (c, d) \iff a \times_{\mathbb{Z}} d = b \times_{\mathbb{Z}} c
    \end{equation} 
    and the operation defined 
    \begin{enumerate}
      \item The additive and multiplicative identities are 
      \begin{equation}
        0_{\mathbb{Q}} \coloneqq (0_{\mathbb{Z}}, a), \;\;\; 1_{\mathbb{Q}} \coloneqq (a, a)
      \end{equation}

      \item Addition on $\mathbb{Q}$ is defined 
      \begin{equation}
        (a, b) +_{\mathbb{Q}} (c, d) \coloneqq \big( (a \times_{\mathbb{Z}} d) +_{\mathbb{Z}} (b \times_{\mathbb{Z}} c), b \times_{\mathbb{Z}} d \big) 
      \end{equation}

      \item The additive inverse is defined 
      \begin{equation}
        -(a, b) \coloneqq (-a, b)
      \end{equation}

      \item Multiplication on $\mathbb{Q}$ is defined 
      \begin{equation}
        (a, b) \times_{\mathbb{Q}} (c, d) \coloneqq \big( a \times_{\mathbb{Z}} c, b \times_{\mathbb{Z}} d \big)
      \end{equation} 

      \item The multiplicative inverse is defined 
      \begin{equation}
        (a, b)^{-1} \coloneqq (b, a)
      \end{equation}
    \end{enumerate}
  \end{definition}

  So for every commutative ring there is an associated ring of fractions. A natural question to ask is whether this is unique. Apparently, it is not. 

  \begin{corollary}
    Let $R$ be an integral domain and let $Q$ be the field of fractiohns of $R$. If a field $F$ contains a subring $R^\prime$ isomorphic to $R$, then the subfield of $F$ generated by $R^\prime$ is isomorphic to $Q$. 
  \end{corollary}
  \begin{proof}
    
  \end{proof}
  
  \begin{lemma}[Rationals are a Minimal Field]
    Every subfield of $\mathbb{C}$ contains $\mathbb{Q}$. 
  \end{lemma}
  \begin{proof}
    Must contain $0$ and $1$. Keep adding $1$ and inverting it to get $\mathbb{Z}$. Now $\mathbb{Z}$ must contain units so $1/n$ also contained. Then multiply the elements to get $\mathbb{Q}$. 
  \end{proof} 

  Once we define polynomials in the next section, we will construct the field of rational functions in the same manner. 

\subsection{Ordered Fields}

  Great, so we have established that $\mathbb{Q}$ is a field. The next property we want to formalize is order. There are countless ways to do it, but I just take the difference and claim that it is greater than $0$. Note that given a set, we can really put whatever order we want on it. However, consider the field with the following order. 
  \begin{equation}
    \mathbb{F} = \{0, 1\}, \; 0 < 1
  \end{equation} 
  This does not behave well with respect to its operations because for example if we have $0 < 1$, then adding the same element to both sides should preserve the ordering. But this is not the case since $0 + 1 = 1 > 1 + 1 = 0$. While it may be easy to define an order, we would like it to be an ordered field. 

  \begin{definition}[Ordered Field]
    An \textbf{ordered field} is a field that has an order satisfying 
    \begin{enumerate}
      \item $y < z \implies x + y < x + z$ for all $x \in \mathbb{F}$. 
      \item $x > 0, y > 0 \implies xy > 0$. 
    \end{enumerate}
  \end{definition}

  \begin{theorem}[Properties of an Ordered Field]
    In an totally ordered field, 
    \begin{enumerate}
      \item $x > 0 \implies -x < 0$. 
      \item $x \neq 0 \implies x^2 > 0$. 
      \item If $x > 0$, then $y < z \implies xy < xz$. 
    \end{enumerate}
  \end{theorem} 
  \begin{proof}
    The first property is a single-liner 
    \begin{equation}
      0 < x \implies 0 + -x < x + -x \implies -x < 0 
    \end{equation}
    For the second property, it must be the case that $x > 0$ or $x < 0$. If $x > 0$, then by definition $x^2 > 0$. If $x < 0$, then 
    \begin{equation}
      x^2 = 1 \cdot x^2 = (-1)^2 \cdot x^2 = (-1 \cdot x)^2 = (-x)^2
    \end{equation}
    and since $-x > 0$ from the first property, we have $x^2 = (-x)^2 > 0$. For the third, we use the distributive property. 
    \begin{align}
      y < z & \implies 0 < z - y \\ 
            & \implies 0 = x 0 < x(z - y) = xz - xy \\
            & \implies xy < xz
    \end{align}
  \end{proof}

  \begin{theorem}[Ordered Field Structure]
    Second, $\mathbb{Q}$ is an ordered field. The order $\leq_{\mathbb{Q}}$ defined on the rationals as 
    \begin{equation}
      (a, b) \leq_{\mathbb{Q}} (c, d) \iff ad \leq_{\mathbb{Z}} bc
    \end{equation}
    is a total order. Remember that WLOG we can assume $b, d > 0$.  
  \end{theorem}
  \begin{proof}
    For the order property, we have 
    \begin{enumerate}
      \item Reflexive. 
      \begin{equation}
        (a, b) \leq_{\mathbb{Q}} (a, b) \iff ab \leq_{\mathbb{Z}} ab
      \end{equation} 

      \item Antisymmetric. 
      \begin{align}
        (a, b) \leq_{\mathbb{Q}} (c, d) & \implies ad \leq_{\mathbb{Z}} bc
        (c, d) \leq_{\mathbb{Q}} (a, b) & \implies bc \leq_{\mathbb{Z}} ad
      \end{align} 
      This implies that both $ad = bc$, which by definition means that they are in the same equivalence class. 

      \item Transitivity. Assume that $(a, b) \leq (c, d)$ and $(c, d) \leq (e, f)$. Then, we notice that $b, d, f > 0$ and therefore by the ordered ring property\footnote{If $a \leq b$ and $0 \leq c$, then $ac \leq bc$.} of $\mathbb{Z}$, we have 
      \begin{align}
        (a, b) \leq_{\mathbb{Q}} (c, d) & \implies ad \leq_{\mathbb{Z}} bc \implies adf \leq_{\mathbb{Z}} bcf \\ 
        (c, d) \leq_{\mathbb{Q}} (e, f) & \implies cf \leq_{\mathbb{Z}} de \implies bcf \leq_{\mathbb{Z}} bde
      \end{align}
      Therefore from transitivity of the ordering on $\mathbb{Z}$ we have $adf \leq bde$. By the ordered ring property\footnote{If $a \leq b$, then $a + c \leq b + c$.}  we have $0 \leq bde - adf = d(be - af)$. But notice that $d > 0$ from our definition of rationals, and therefore it must be the case that $0 \leq be - af \implies af \leq_{\mathbb{Z}} be$, which by definition means $(a, b) \leq_{\mathbb{Q}} (e, f)$. 
    \end{enumerate}
    For the ordered field property, we have 
    \begin{enumerate}
      \item Assume that $y = (a, b) \leq (c, d) = z$. Let $x = (e, f)$. Then $x + y = (af + be, bf)$, $x + z = (cf + de, df)$. Therefore 
      \begin{align}
        (af + be) df & = adf^2 + bedf \\ 
                     & \leq bcf^2 + bedf \\
                     & = (cf + de) bf
      \end{align} 
      But $(af + be) df = (cf + de) bf$ is equivalent to saying $(af + be, bf) \leq_{\mathbb{Q}} (cf + de, df)$, i.e. $x + y \leq x + z$!  

      \item Let $x = (a, b), y = (c, d)$. Since $0 < x, 0 < y$, by construction this means that $0 < a, 0 < c$ (since $b, d > 0$ in the canonical rational form). By the ordered ring property of the integers, $0 < ac$. So 
      \begin{equation}
        0 < ac \iff 0 \cdot bd < ac \cdot 1 \iff (0, 1) < (ac, bd)  \iff 0_{\mathbb{Q}} < (a, c) \times_{\mathbb{Q}} (b, d) = x y
      \end{equation}
    \end{enumerate}
  \end{proof} 

  Great, so we have shown that the rationals have an ordered field structure and that there is a canonical ring embedding from $\mathbb{Z}$ to $\mathbb{Q}$. It remains to show that this ring homomorphism is an \textit{ordered} ring homomorphism. 

  \begin{theorem}[Canonical Injection of $\mathbb{Z}$ to $\mathbb{Q}$ is an Ordered Ring Homomorphism]
    The canonical injection $\iota: \mathbb{Z} \rightarrow \mathbb{Q}$ defined $\iota(a) = (a, 1)$\footnote{which is well defined since we can arbitrarily choose the denominator.} is an ordered ring homomorphism. That is, for $a, b \in \mathbb{Z}$, 
    \begin{equation}
      a \leq_{\mathbb{Z}} b \iff \iota(a) \leq_{\mathbb{Q}} \iota(b)
    \end{equation}
  \end{theorem}
  \begin{proof} 
    We have already proved that this is a ring homomorphism. To show that it preserves the order, we have 
    \begin{align}
      a \leq_{\mathbb{Z}} b & \iff a \cdot 1 \leq_{\mathbb{Z}} b \cdot 1 \\
                            & \iff (a, 1) \leq_{\mathbb{Q}} (b, 1) \\
                            & \iff \iota(a) \leq_{\mathbb{Q}} \iota(b)
    \end{align}
  \end{proof} 

  \begin{theorem}[Finite Fields]
    There are no finite ordered fields. 
  \end{theorem} 
  \begin{proof}
    Assume $\mathbb{F}$ is such an ordered field. It must be the case that $0, 1 \in \mathbb{F}$, with $0 < 1$. Therefore, we also have $0 + 1 < 1 + 1 \implies 1 < 1 + 1$. Repeating this we get 
    \begin{equation}
      0 < 1 < 1 + 1 < 1 + 1 + 1 < \ldots
    \end{equation}
    where these elements must be distinct (since only one of $>, <, =$ must be true for a totally ordered set). Since this can be done for a countably infinite number of times, $\mathbb{F}$ cannot be finite. 
  \end{proof}

  Great, so we have pretty much constructed the rational numbers, with the exception that we still need the topology/metric/norm on these numbers, but this won't be too relevant for now. 

\subsection{The Real Numbers}

  The next step is to formally construct the real numbers. There are generally two ways of doing this: with Dedekind cuts, with Cauchy sequences, or with compact nested intervals. Using Cauchy sequences or compactness requires us to introduce the metric and the topology, while Dedekind cuts is purely based on the order which we have established. To make the construction as minimal as possible, I will use the Dedekind cuts method, and in topology/analysis, we can compare all three of these methods (and determine their equivalence!). 

  \begin{definition}[Dedekind Cut] 
    A \textbf{Dedekind cut} is a partition of the rationals $\mathbb{Q} = A \sqcup A^\prime$ satisfying the three properties.\footnote{This can really be defined for any totally ordered set. } 
    \begin{enumerate}
      \item $A \neq \emptyset$ and $A \neq \mathbb{Q}$.\footnote{By relaxing this property, we can actually complete $\mathbb{Q}$ to the extended real number line. }
      \item $x < y$ for all $x \in A, y \in A^\prime$. 
      \item The maximum element of $A$ does not exist in $\mathbb{Q}$. 
    \end{enumerate}
    The minimum of $A^\prime$ may exist in $\mathbb{Q}$, and if it does, the cut is said to be \textbf{generated} by $\min A^\prime$. 
  \end{definition}

  Note that in $\mathbb{Q}$, there will be two types of cuts: 
  \begin{enumerate}
    \item ones that are generated by rational numbers, such as 
    \begin{equation}
      A = \{x \in \mathbb{Q} \mid x < 2/3 \}, A^\prime = \{ x \in \mathbb{Q} \mid x \geq 2/3 \} 
    \end{equation}
    \item and the ones that are not 
    \begin{equation}
      A = \{x \in \mathbb{Q} \mid x^2 < 2 \}, A^\prime = \{x \in \mathbb{Q} \mid x^2 \geq 2 \}
    \end{equation}
  \end{enumerate}
  We can intuitively see that the set of all Dedekind cuts $(A, A^\prime)$ will ``extend'' the rationals into a bigger set. We can then define some operations and an order to construct this into an ordered field, and finally it will have the property that we call ``completeness.''

  \begin{definition}[Dedekind Completeness]
    A totally ordered algebraic field $\mathbb{F}$ is \textbf{complete} if every Dedekind cut of $\mathbb{F}$ is generated by an element of $\mathbb{F}$. 
  \end{definition} 

  \begin{theorem}[Rationals are Not Dedekind-Complete]
    $\mathbb{Q}$ is not Dedekind-complete. 
  \end{theorem}
  \begin{proof}
    The counter-example is given above for the cut 
    \begin{equation}
      A = \{x \in \mathbb{Q} \mid x^2 < 2 \}, A^\prime = \{x \in \mathbb{Q} \mid x^2 \geq 2 \}
    \end{equation}
  \end{proof} 

  Now we have the tools to define the reals, giving us the beefy theorem. 

  \begin{theorem}[Reals as the Dedekind-Completion of Rationals]
    Let $\mathbb{R}$ be the set of all Dedekind cuts $(A, A^\prime)$ of $\mathbb{Q}$ of $\mathbb{Q}$. For convenience we can uniquely represent $(A, A^\prime)$ with just $A$ since $A^\prime = \mathbb{Q} \setminus A$. By doing this we can intuitively think of a real number as being represented by the set of all smaller rational numbers. Let $A, B$ be two Dedekind cuts. Then, we define the following order and operations. 
    \begin{enumerate}
      \item \textit{Order}. $A \leq_{\mathbb{R}} B \iff A \subset B$. 
      \item \textit{Addition}. $A +_{\mathbb{R}} B \coloneqq \{ a +_{\mathbb{Q}} b \mid a \in A, b \in B \}$. 
      \item \textit{Additive Identity}. $0_{\mathbb{R}} \coloneqq \{x \in \mathbb{Q} \mid x < 0 \}$. 
      \item \textit{Additive Inverse}. $-B \coloneqq \{ a - b \mid a < 0 , b \in (\mathbb{Q} \setminus B) \}$.
      \item \textit{Multiplication}. If $A, B \geq 0$, then we define $A \times_{\mathbb{R}} B \coloneqq \{ a \times_{\mathbb{Q}} b \mid a \in A, b \in B, a, b \geq 0\} \cup 0_{\mathbb{R}}$. If $A$ or $B$ is negative, then we use the identity $A \times B = -(A \times_{\mathbb{R}} -B) = -(-A \times_{\mathbb{R}} B) = (-A \times_{\mathbb{R}} -B)$ to convert $A, B$ to both positives and apply the previous definition. 
      \item \textit{Multiplicative Identity}. $1_{\mathbb{R}} = \{x \in \mathbb{Q} \mid x < 1 \}$. 
      \item \textit{Multiplicative Inverse}. If $B > 0$, $B^{-1} \coloneqq \{ a \times_{\mathbb{Q}} b^{-1} \mid a \in 1_{\mathbb{R}}, b \in (\mathbb{Q} \setminus B) \}$. If $B$ is negative, then we compute $B^{-1} = -((-B)^{-1})$ by first converting to a positive number and then applying the definition above. 
    \end{enumerate}
    We claim that $(\mathbb{R}, +_{\mathbb{R}}, \times_{\mathbb{R}}, \leq_{\mathbb{R}})$ is a totally ordered field, and the canonical injection $\iota: \mathbb{Q} \rightarrow \mathbb{R}$ defined 
    \begin{equation}
      \iota(q) = \{x \in \mathbb{Q} \mid x < q \}
    \end{equation}
    is an ordered field isomorphism. Finally, by construction $\mathbb{R}$ is Dedekind-complete. 
  \end{theorem} 

  \begin{definition}[Least Upper Bound Property]
    A totally ordered set $\mathbb{F}$ has the \textbf{least upper bound} property if every nonempty set of $F$ having an upper bound must have a least upper bound (supremum) in $F$. 
  \end{definition} 

  \begin{theorem}[Dedekind Completeness Equals Least-Upper-Bound Property]
    Dedekind completeness is equivalent to the least upper bound property. 
  \end{theorem}
  \begin{proof}
    
  \end{proof}

  It is also important to note that $\mathbb{R}$ satisfies the Archimidean principle, which is fundamental in analysis, and that Cauchy/nested interval completeness does \textit{not} imply Archimidean, while Dedekind-completeness does. However, this again is not very relevant in a purely algeraic sense. 

  \begin{definition}[Archimidean Principle]
    An ordered ring $(X, +, \cdot, \leq)$ that embeds the naturals $\mathbb{N}$\footnote{as in, there exists an ordered ring homomorphism $\iota: \mathbb{N} \rightarrow X$} is said to obey the \textbf{Archimedean principle} if given any $x, y \in X$ s.t. $x, y > 0$, there exists an $n \in \mathbb{N}$ s.t. $\iota(n) \cdot x > y$. Usually, we don't care about the canonical injection and write $nx > y$. 
  \end{definition}

  By the canonical injections $\mathbb{N} \rightarrow \mathbb{Z} \rightarrow \mathbb{Q} \rightarrow \mathbb{R}$, we can talk about whether this set has the Archimedean property. In fact Dedekind completeness does imply it. 

  \begin{theorem}[Reals are Archimidean]
    $\mathbb{R}$ satisfies the Archimedean principle. 
  \end{theorem}
  \begin{proof}
    Assume that this property doesn't hold. Then for any fixed $x$, $nx < y$ for all $n \in \mathbb{N}$. Consider the set 
    \begin{equation}
      A = \bigcup_{n \in \mathbb{N}} (-\infty, nx), \;\;\; B = \mathbb{R} \setminus A
    \end{equation}
    $A$ by definition is nonempty, and $B$ is nonempty since it contains $y$. Then, we can show that $a \in A, b \in B \implies a < b$ using proof by contradiction. Assume that there exists $a^\prime \in A, b^\prime \in B$ s.t. $a^\prime > b^\prime$. Since $a^\prime \in A$, there exists a $n^\prime \in \mathbb{N}$ s.t. $a^\prime \in (-\infty, n^\prime x) \iff a^\prime < n^\prime x$. But by transitivity of order, this means $b^\prime < n^\prime x \iff b^\prime \in (-\infty, n^\prime x) \implies b^\prime \in A$. 

    Going back to the main proof, we see that $A$ is upper bounded by $y$, and so by the least upper bound property it has a supremum $z = \sup{A}$. 
    \begin{enumerate}
      \item If $z \in A$, then by the induction principle\footnote{Note that $\mathbb{N}$ is defined recursively as $1 \in \mathbb{N}$ and if $n \in \mathbb{N}$, then $n+1 \in \mathbb{N}$. } $z + x \in A$, contradicting that $z$ is an upper bound. 
      \item If $z \not\in A$, then by the induction principle\footnote{The contrapositive of the recursive definition of $\mathbb{N}$ is: if $n \not\in \mathbb{N}$, then $n-1 \not\in \mathbb{N}$.} $z-x \not\in A \implies z-x \in B$. Since every element of $B$ upper bounds $A$ and since $x > 0$, this means that $z-x < z$ is a smaller upper bound of $A$, contradicting that $z$ is a least upper bound. 
    \end{enumerate}
    Therefore, it must be the case that $nx > y$ for some $n \in \mathbb{N}$. 
  \end{proof}

\subsection{Exercises} 

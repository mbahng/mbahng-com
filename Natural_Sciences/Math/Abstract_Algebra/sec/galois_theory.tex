\section{Galois Theory}

  In general, given a polynomial $f(x)$, we would like to find effective ways to factor it or find its roots. This depends on which polynomial ring $R[x]$ that we view $f(x)$ as an element of, and finding such a unique factorization requires us to work in UFDs at least. But solving even one of problems is quite hard with ring theory alone, and we need to combine our ideas of polynomial rings, groups, and vector spaces to develop effective means of solving polynomials. This is known as \textit{Galois theory}. 

  One nice property where finding the roots and the factorization coincide is when $f(x)$ splits in $R[x]$, and we usually construct these rings that allow $f(x)$ to split by \textit{extending} existing rings $R$ into a bigger ring $S$. 

  \begin{definition}[Ring Extension]
    A pair of rings $R, S$ where $R$ is a subring of $S$ is called a \textbf{ring extension}, denoted $R \hookrightarrow S$, $S/R$ or $R \subset S$. If we have multiple extensions $R \subset S \subset T$, this is called a \textbf{tower}. 
  \end{definition}

  We have already come across several examples of extensions. 

  \begin{example}[Ring Extensions]
    Listed.
    \begin{enumerate}
      \item $\mathbb{Z} \subset \mathbb{Q}$ is a ring extension.
      \item $\mathbb{Z} \subset \mathbb{Z}[i]$ is a ring extension. 
      \item $\mathbb{Q} \subset \mathbb{R} \subset \mathbb{C}$ is a ring extension tower. 
    \end{enumerate}
  \end{example}

  Now we present the motivation for working with extensions of fields that rings in general. Recall that a field is trivially a vector space, followed by the more general result. 

  \begin{theorem}[Fields are a Vector Space over Subfields]
    \label{thm:fields_vector_space}
    Let $F$ be a subfield of $K$. Then $K$ is a $F$-vector space. 
  \end{theorem}
  \begin{proof}
    A $F$-vector space has $0$, addition, and multiplication by $F$. $K$ indeed has $0$, addition, and we can multiply any element of $K$ by an element of $F$. The extra axioms follow but are too verbose to write a full proof. 
  \end{proof}

  \begin{corollary} 
    $\mathbb{R}$ is an infinite-dimensional vector space over $\mathbb{Q}$. 
  \end{corollary}
  \begin{proof}
    The fact that it is a vector space immediately follows from $\mathbb{Q} \hookrightarrow \mathbb{R}$. For dimensionality, the outline is to show that $\{\sqrt{p} \mid p \text{ prime }\}$ are linearly independent. This takes work to prove and won't do it. 
  \end{proof}

  Therefore, by constructing a subfield $F$ of a field $K$, we can model $K$ as a $F$-vector space (though it may be finite or infinite-dimensional). This additional structure warrants a name. 

  \begin{definition}[Field Extension]
    If $F \subset K$ are fields, then this is called a \textbf{field extension}. Its \textbf{degree} is the $F$-dimension of $K$, denoted
    \begin{equation}
      [K:F] \coloneqq \dim_F (K) 
    \end{equation}
  \end{definition} 

  As we concatenate field extensions, sometimes called a \textit{tower}, the dimensions behave nicely as well. 

  \begin{theorem}[Tower Rule]
    If $E \hookrightarrow F \hookrightarrow K$ are two field extensions, then $E \hookrightarrow K$ is a field extension with degree 
    \begin{equation}
      [K:E] = [K:E] [E:F]
    \end{equation}
  \end{theorem}
  \begin{proof}
    Let $\alpha_1, \ldots \alpha_m$ be a basis for $E$ over $F$ and $\beta_1, \ldots, \beta_n$ be a basis for $K$ over $E$. We claim that $\{\alpha_i \beta_j\}$ is a basis for $K$ over $F$, with multiplication done in the field $K$. We check linear indepdence. Let $\beta \in K$ be arbitrary. Then by the $E$-basis, we have 
    \begin{equation}
      \beta = \sum_{j=1}^n x_j \beta_j
    \end{equation} 
    But since $x_j \in E$, there are elements 
    \begin{equation}
      x_j = \sum_{i=1}^m y_{ij} \alpha_i
    \end{equation}
    and so combining we get 
    \begin{equation}
      \beta = \sum_{j=1}^n \sum_{i=1}^m y_{ij} (\alpha_i \beta_j) 
    \end{equation}
    To prove linear independence, suppose $\beta = 0$. Then we have 
    \begin{equation}
      0 = \sum_{j=1}^n \sum_{i=1}^m y_{ij} (\alpha_i \beta_j) = \sum_{j=1}^n \bigg( \sum_{i=1}^m y_{ij} \alpha_i \bigg) \beta_j 
    \end{equation}
    Since $\beta_1, \ldots, \beta_n$ are linearly indpendent, we must have $\sum_{i=1}^m y_{ij} \alpha_i = 0$ for all $j$. But since $\alpha_i$'s are linear independent, this means $y_{ij} = 0$ for all $i, j$. 
  \end{proof} 

  In fact, it is a natural question to find the ``minimal field'' $F$ such that $f(x)$ splits in $F[x]$, which is called the \textit{splitting field} of $f(x)$. To do this, we must start with $F$ and extend it to $K$ such that $K$ is a field. This is done through \textit{adjoining} an element $\alpha$ to $F$, getting a larger set $F[\alpha]$. This guarantees both a ring and vector space structure but requires another condition (that $\alpha$ be a root of an irreducible polynomial in $F[x]$) to be a field. Once this is done, we have successfully created a field extension, and finding the splitting field is not too hard from here. 

  What's next? Now that we can construct the splitting field $K/F$, we want to try and find some \textit{symmetries} between its roots. That is, if $\alpha, \beta$ are the roots of some $f(x) \in F[x]$, can we find a transformation group $G$ such that knowing one root allows us to uncover the other roots? There is in fact such a structure out there, known as the \textit{Galois group}---an automorphism group on the finite set of roots. This will be our roadmap in this section. 

\subsection{Ring Extensions Through Adjoining} 

  We deal with the first problem of constructing ring extensions. We have already seen the idea of taking ring $R$ and adjoining an element $x$ to it to get a minimal ring containing both $R$ (as a subring) and $x$. This was the polynomial ring. Now we can replace the indeterminate $x$ with an element $\alpha$ from a larger ring $S/R$ and do the exact same thing to get a ring extension of $R$. 

  \begin{definition}[Adjoining Ring] 
    \label{def:adjoining_ring}
    Given ring extension $R \hookrightarrow S$ with finite $\alpha = \{\alpha_1, \ldots, \alpha_n\} \subset S$, the \textbf{ring $R$ adjoined by $\alpha$} is defined in the two equivalent ways. 
    \begin{enumerate}
      \item $R[\alpha]$ is the minimal ring containing $R$ and $\alpha$. 

      \item We take the $\alpha_i$'s and map it through all polynomials in $R[x_1, \ldots, x_n]$. 
      \begin{equation}
        R[\alpha] = R[\alpha_1, \ldots, \alpha_n] \coloneqq \{ f(\alpha_1, \ldots, \alpha_n) \in S \mid f \in R[x_1, \ldots, x_n]\} \subset S
      \end{equation} 

      \item We can iteratively construct $R[\alpha]$.\footnote{Note that if $R \hookrightarrow S$ is a ring extension, then $R[\alpha]$ is a ring, and so it makes sense to write $R[\alpha][\beta] \subset S$. It had better be the case that this ring is consistent with equivalent constructions.} 
      \begin{equation}
        R \subset R[\alpha_1] \subset R[\alpha_1, \alpha_2] \subset \ldots \subset R[\alpha_1, \ldots, \alpha_n] 
      \end{equation}
    \end{enumerate}
  \end{definition} 
  \begin{proof}
    We show a few things. 
    \begin{enumerate}
      \item \textit{$R[\alpha]$ in (2) is a ring}. Given two elements $\phi, \gamma \in F[\alpha]$, there exists polynomials $f, g \in F[x]$ s.t. $\phi = f(\alpha), \gamma = g(\alpha)$. Since $F[x]$ is a ring, we see that 
      \begin{align}
        \phi + \gamma & = f(\alpha) + g(\alpha) = (f + g)(\alpha) \\
        \phi \cdot \gamma & = f(\alpha) \cdot g(\alpha) = (fg)(\alpha)
      \end{align} 
      Furthermore, it is easy to check that $0$ and $1$ are the images of $\alpha$ through the $0$ and $1$ polynomials. 
      \item \textit{(1) $\implies$ (2)}. This is pretty obvious since $R$ is a subring of (2) and $\alpha$ is just the image of $\alpha$ under $f(x) = x$. 
      \item \textit{(2) $\iff$ (3)}. By induction, it suffices to show that $R[\alpha, \beta] = R[\alpha][\beta] = R[\beta][\alpha]$. 
      \item \textit{(3) $\implies$ (1)}. By induction, it suffices to show that $R[\alpha_1]$ is the minimal field a
    \end{enumerate}
  \end{proof} 

  What allows us to make this inclusion proper is that the $\alpha \in K$, which does not necessarily have to be in $F$, \textit{extends} this field a bit further, but since we can only map the one element $\alpha$, it may not cover all of $K$. Most of the times, we will work with adjoining rings of univariate polynomial elements. 
  \begin{equation}
    F[\alpha] \coloneqq \{ f(\alpha) \in F \mid f \in F[x]\} \subset K
  \end{equation} 
  Let's go through some examples.

  \begin{example}[Radical Extensions of $\sqrt{2}$]
    Let $F = \mathbb{Q}$ and $K = \mathbb{C}$. We claim $\mathbb{Q}[\sqrt{2}] = \{a + b \sqrt{2} \mid a, b \in \mathbb{Q} \}$.
    \begin{enumerate}
      \item $\mathbb{Q}[\sqrt{2}] \subset \{a + b \sqrt{2} \mid a, b \in \mathbb{Q} \}$. $\mathbb{Q}[\sqrt{2}]$ are elements of the form
      \begin{equation}
        f(\sqrt{2}) = a_n (\sqrt{2})^n + a_{n-1} (\sqrt{2})^{n-1} + \ldots + a_2 (\sqrt{2})^2 + a_1 \sqrt{2} + a_0
      \end{equation} 
      This can be written by collecting terms, of the form $a + b \sqrt{2}$. 

      \item $\mathbb{Q}[\sqrt{2}] \supset \{a + b \sqrt{2} \mid a, b \in \mathbb{Q} \}$. Given an element $a + b \sqrt{2}$, this is clearly in $\mathbb{Q}[\sqrt{2}]$ since it is the image of $\sqrt{2}$ under the polynomial $f(x) = a + bx$. 
    \end{enumerate}
  \end{example} 

  Given this, we may extrapolate this pattern and claim that $\mathbb{Q}[\sqrt{2} + \sqrt{3}]$ consists of all numbers of form $a + (\sqrt{2} + \sqrt{3}) b$. However, this is \textit{not} the case. 

  \begin{example}[Radical Extensions of $\sqrt{2} + \sqrt{3}$]
    Given any element $\beta \in \mathbb{Q}[\sqrt{2} + \sqrt{3}]$, it is by definition of the form 
    \begin{equation}
      \beta = \sum_{k=0}^n a_k (\sqrt{2} + \sqrt{3})^k 
    \end{equation} 
    Clearly $1, \sqrt{2} + \sqrt{3} \in \mathbb{Q}[\sqrt{2} + \sqrt{3}]$ by mapping $\sqrt{2} + \sqrt{3}$ through the polynomials $f(x) = 1$ and $f(x) = $. However, we can see that $(\sqrt{2} + \sqrt{3})^2 = 5 + \sqrt{6}$,\footnote{where we use $\sqrt{6}$ as notation for $\sqrt{2} \cdot \sqrt{3}$} and so $\sqrt{6} \in \mathbb{Q}[\sqrt{2} + \sqrt{3}]$. Furthermore, we have $(\sqrt{2} + \sqrt{3})^3 = 11 \sqrt{2} + 9 \sqrt{3}$, and so with the ring properties we can conclude that 
    \begin{align}
      \frac{1}{2} \big[ (11 \sqrt{2} + 9 \sqrt{3}) - 9 (\sqrt{2} + \sqrt{3})\big] = \sqrt{2} & \in \mathbb{Q}[\sqrt{2} + \sqrt{3}] \\
      -\frac{1}{2} \big[ (11 \sqrt{2} + 9 \sqrt{3}) - 11 (\sqrt{2} + \sqrt{3})\big] = \sqrt{3} & \in \mathbb{Q}[\sqrt{2} + \sqrt{3}] \\
    \end{align} 
    If we go a bit further, we can show that 
    \begin{equation}
      \mathbb{Q}[\sqrt{2} + \sqrt{3}] = \{a + b \sqrt{2} + c \sqrt{3} + d\sqrt{6} \mid a, b, c, d \in \mathbb{Q} \}
    \end{equation}
  \end{example}

  \begin{example}[Showing Two Extensions are Equal]
    We claim that $\mathbb{Q}[\sqrt{6}, \sqrt{5}] = \mathbb{Q}[\sqrt{6} + \sqrt{5}]$. 
    \begin{enumerate}
      \item $\mathbb{Q}[\sqrt{6}, \sqrt{5}] \subset \mathbb{Q}[\sqrt{6} + \sqrt{5}]$ because the sum of $\sqrt{6}$ and $\sqrt{5}$ lies in $\mathbb{Q}[\sqrt{6}, \sqrt{5}]$. 
      \item $\mathbb{Q}[\sqrt{6} + \sqrt{5}] \subset \mathbb{Q}[\sqrt{6}, \sqrt{5}]$. We have 
        \begin{equation}
          \frac{1}{\sqrt{5} + \sqrt{6}} = \frac{1}{\sqrt{5} + \sqrt{6}} \frac{\sqrt{5} - \sqrt{6}}{\sqrt{5} - \sqrt{6}} = \sqrt{6} - \sqrt{5} \in \mathbb{Q}[\sqrt{5} + \sqrt{6}] 
        \end{equation}
        which implies that 
        \begin{equation}
          \sqrt{6} = \frac{1}{2} ((\sqrt{6} + \sqrt{5}) + (\sqrt{6} - \sqrt{5})) \in \mathbb{Q}[\sqrt{5} + \sqrt{6}] 
        \end{equation}
        and so $\sqrt{5} = (\sqrt{6} + \sqrt{5}) - \sqrt{6} \in \mathbb{Q}[\sqrt{5} + \sqrt{6}]$. 
    \end{enumerate}
  \end{example}

  So far, we've gotten used to modeling fields as vector spaces. In general, an adjoining ring $R[\alpha]$ is another ring containing $R$ as a subring. Now if $F$ is a field, then $F \subset F[\alpha]$ is a ring extension, and since we are working with the base field $F$, we can model $F[\alpha]$ as a vector space. 

  \begin{theorem}[Adjoining Field is Finite-Dimensional Vector Space]
    If $F \subset K$ is a field and $\alpha \in K$, then 
    \begin{enumerate}
      \item $F[\alpha]$ is a \textit{finite-dimensional} vector space over $F$. 
      \item If $f(x) = a_n x^n + \ldots + a_0$ is any polynomial with root $\alpha$, then $\{1, \alpha, \ldots, \alpha^{n-1}\}$ spans $F[\alpha]$.\footnote{Note that this does not mean that it is a basis.} 
    \end{enumerate}
  \end{theorem}
  \begin{proof}
    An element of $F[\alpha]$ is of the form 
    \begin{equation}
      f(\alpha) = \sum_{k=0}^n a_k \alpha^k
    \end{equation} 
    for some $f \in F[x]$, and so it is immediate that $\{\alpha^k\}_{k \in \mathbb{N}_0}$ spans $F[\alpha]$. We claim that $\alpha^{n-1+i}$ is in $S$ for all $i > 0$. By induction, if $i = 1$, then 
    \begin{equation}
      \alpha^n = -\frac{1}{a_n} \big( a_{n-1} \alpha^{n-1} + \ldots + a_0 \big)
    \end{equation}
    which proves the claim. Now assume that $\alpha^n, \alpha^{n+1}, \ldots, \alpha^{n-1+i} \in \Span\{1, \ldots, \alpha^{n-1}\}$. Then 
    \begin{equation}
      \alpha^i f(\alpha) = 0 \implies a_n \alpha^{n+i} + \alpha_{n-1} \alpha^{n+i-1} + \ldots + a_0 \alpha^i = 0 
    \end{equation}
    and so 
    \begin{equation}
      \alpha^{n+i} = -\frac{1}{a_n} \big(a_{n-1} \alpha^{n+i-1} + \ldots + a_0 \alpha^i)
    \end{equation}
    which means that $\alpha^{n+i} \in \Span\{1, \ldots, \alpha^{n-1}\}$, completing the proof. 
  \end{proof} 

\subsection{Field Extensions}

  This is nice since we have a vector space structure on $F[\alpha]$ unlike just a ring structure on $R[\alpha]$. What we should think is that $F[\alpha]$ ends up becoming both a ring and a vector space, but not yet a field. We would like to find conditions in which $F[\alpha]$ indeed does become a field, which at this point it is commonly denoted $F(\alpha)$ (rather than square brakcets to emphasize it is a field). It turns out that it will happen if any only if $\alpha$ is \textit{algebraic}.  

  \begin{definition}[Algebraic Numbers]
    Given a field extension $F \subset K$, an element $\alpha \in K$ is \textbf{algebraic over $F$} if $\alpha$ is some root of $f(x) \in F[x]$.\footnote{By default we have $F = \mathbb{Q}$.}
  \end{definition}

  \begin{example}[Algebraic Numbers]
    We list a few examples. 
    \begin{enumerate}
      \item $\sqrt{2} \in \mathbb{R}$ is algebraic over $\mathbb{Q}$ since it is a root of $x^2 - 2 \in \mathbb{Q}[x]$. 
      \item $i \in \mathbb{C}$ is algebraic over $\mathbb{R}$ since it is a root of $x^2 + 1 \in \mathbb{R}[x]$. It is also a root of $x^4 + 2 x^2 + 1 \in \mathbb{R}[x]$. 
    \end{enumerate}
  \end{example}

  Note that given an element $\alpha \in K$ that is algebraic over $F$, there may be multiple polynomials $f(x) \in F[x]$ that has $\alpha$ as a root. In fact, this set forms an ideal. 

  \begin{lemma}[Polynomials Vanishing at $\alpha$ Forms an Ideal]
    Let $K/F$ be a field extension and fix $\alpha \in K$ that is algebraic over $F$. Then the set of polynomials $f(x) \in F[x]$ with root $\alpha$ forms an ideal over $F[x]$. 
  \end{lemma}
  \begin{proof}
    Let us denote the set as $I$. We prove the two properties of ideals.  
    \begin{enumerate}
      \item Consider $f(x), g(x) \in I$. Then $(f + g)(\alpha) = f(\alpha) + g(\alpha) = 0 + 0 = 0 \implies (f + g)(x) \in I$. 
      \item Consider $h(x) \in F[x], f(x) \in I$. Then $(hf)(\alpha) = h(\alpha) f(\alpha) = h(\alpha) 0 = 0 \implies (hf)(x) \in I$. 
    \end{enumerate}
  \end{proof}

  Since $F$ is a field, $F[x]$ is a PID, and so it must be generated by some element. The scaling of the coefficients doesn't really matter (since we are working in a field so we can always divide the leading coefficient), so we can assume that it is monic. This is called the \textit{minimal polynomial}. 

  \begin{definition}[Minimal Polynomial]
    \label{def:min_poly}
    Let $F \subset K$ be a field extension and $\alpha \in K$. The \textbf{minimal polynomial} of $\alpha$ is defined in the equivalent ways. 
    \begin{enumerate}
      \item It is the monic polynomial of least degree among all polynomials in $F[x]$ having $\alpha$ as a root. 
      \item It is the generator of the ideal of all polynomials in $F[x]$ with root $\alpha$. 
    \end{enumerate}
    We claim that it always exists and is unique. 
  \end{definition} 
  \begin{proof}
    Take the evaluation homomorphism $\ev_\alpha: F[x] \to K$ and look at the ideal $\ker{\ev_\alpha}$. Since $F[x]$ is a PID, call the generator $f(x)$ and normalize the leading coefficient to $1$. We claim that this is the minimal polynomial. This proves existence. 

    TBD. 
  \end{proof}

  \begin{example}[Minimal Polynomials]
    Let $\mathbb{Q} \subset \mathbb{R}$ be a field extension. 
    \begin{enumerate}
      \item The minimal polynomial for $\alpha = \sqrt{2} \in \mathbb{R}$ in $\mathbb{Q}[x]$ is 
      \begin{equation}
        x^2 - 2 = (x + \sqrt{2}) (x - \sqrt{2})
      \end{equation}

      \item The minimal polynomial for $\alpha = \sqrt{2} \in \mathbb{R}$ in $\mathbb{R}[x]$ is 
      \begin{equation}
        x - \sqrt{2}
      \end{equation}

      \item The minimal polynomial for $\alpha = \sqrt{2} + \sqrt{3}$ in $\mathbb{Q}[x]$ is 
      \begin{equation}
        x^4 - 10x^2 + 1 = (x - (\sqrt{2} + \sqrt{3}))(x - (\sqrt{2} - \sqrt{3}))(x - (-\sqrt{2} + \sqrt{3}))(x - (-\sqrt{2} - \sqrt{3}))
      \end{equation}
    \end{enumerate}
  \end{example}

  Note that the minimal polynomial is also irreducible. Let's prove this. 

  \begin{lemma}[Minimal Polynomial is Irreducible]
    Let $K/F$ be a field extension. The minimal polynomial $f(x) \in F[x]$ of $\alpha \in K$ is irreducible in $F[x]$. 
  \end{lemma}
  \begin{proof}
    Assume that $f(x)$ is reducible. Then we factor it into $f(x) = g(x) h(x)$, which implies that 
    \begin{equation}
      0 = f(\alpha) = g(\alpha) h(\alpha) \implies g(\alpha) = 0 \text{ or } h(\alpha) = 0
    \end{equation}
    since $F$ is a field and therefore does not contain zero divisors. Choosing both $g(x), h(x)$ to be of degree strictly lower than $f(x)$ would contradict the minimality requirement of $f(x)$, so $f(x)$ must be irreducible. 
  \end{proof}

  Since the minimal polynomial is irreducible, we can see that from Bezout's identity that the minimal polynomial generates a maximal ideal, and so the quotient ring is actually a field. 

  \begin{theorem}[Quotient Ring of Minimal Polynomial is a Field]
    Let $F$ be a field and $f(x) \in F[x]$, then 
    \begin{equation}
      f(x) \text{ is irreducible in } F[x] \iff \frac{F[x]}{\langle f(x) \rangle} \text{ is a field}
    \end{equation} 
  \end{theorem}
  \begin{proof}
    We first claim that $f(x)$ is irreducible if and only if $f(x)$ is a maximal ideal.  
    \begin{enumerate}
      \item If $f(x)$ is irreducible, then consider the ideal $I = \langle f(x) \rangle$ and add some $g(x) \in F[x] \setminus I$. Since $F[x]$ is a PID, $f(x)$ is prime, and so $\gcd(f, g) = 1$. From Bezout's identity, there must exist $a(x), b(x) \in F[x]$ s.t. $a(x) f(x) + b(x) g(x) = 1$, which implies that any larger ideal than $\langle f(x) \rangle$ must be the entire $F[x]$. 

      \item If $f(x)$ is reducible, then let $f(x) = g(x) h(x)$. Then $\langle f(x) \rangle \subsetneq \langle g(x) \rangle \subsetneq F[x]$, and so $\langle f(x) \rangle$ is not maximal. 
    \end{enumerate}
    Then with \ref{thm:quotient_ring_fields}, we know that $I$ is a maximal ideal if and only if $F[x]/I$ is a field, and the proof is complete. 
  \end{proof}

  Now we went through all this trouble of determining sufficient conditions for quotient rings to be fields, when our original goal was on adjoining rings. It seemed like a long detour, but this finally pays off, as now we have both a necessary and sufficient condition to reach our original goal. 

  \begin{corollary}[Conditions For Adjoining Ring to be a Field]
    Let $F \hookrightarrow K$ be a field extension and $\alpha \in K$. 
    \begin{enumerate}
      \item If $\alpha$ is algebraic over $F$---i.e. it is the root of some $g(x) \in F[x]$---then $F(\alpha) \subset K$ is a field. 
      \item The $F$-dimension of $K$ $[F(\alpha):F]$ is the degree of the minimal polynomial of $\alpha$.\footnote{It is clear that if there exists \textit{some} $f(x) \in F[x]$ that has root $\alpha$, then it may not be the \textit{unique} one. The dimension resides specifically in the unique minimal polynomial. } 
    \end{enumerate}
  \end{corollary}
  \begin{proof} 
    Let $f(x)$ be the minimal polynomial of $\alpha$ of degree $n$, and we know that $g(x) \in \langle f(x) \rangle \implies g(x) = h(x) f(x)$ for some $h(x) \in F[x]$. Take the surjective ring homomorphism $\ev_\alpha: F[x] \to F[\alpha]$. We know from \ref{def:min_poly} that $\ker(\ev_\alpha) = \langle f(x) \rangle$, and by the first isomorphism theorem of rings, we have 
    \begin{equation}
      \frac{F[x]}{\langle f(x) \rangle} = \frac{F[x]}{\ker(\ev_\alpha)} \simeq F[\alpha]
    \end{equation} 
    which proves the first claim. For dimension, we know that $\{1, \ldots, x^{n-1}\}$ is a basis. 
  \end{proof}
  \begin{proof}
    This alternative proof is constructive in that it actually shows how to compute multiplicative inverses. 

    It is clear that $F[\alpha]$ is a commutative ring since $F$ is a field. So it remains to show that every nonzero element of $\beta \in F[\alpha]$ is a unit. By definition $\beta = p(\alpha)$ for some polynomial $p \in F[x]$. Factor $f \in F[x]$ as the product of irreducible polynomials. Then $\alpha$ must be a root of one of those irreducible factors, say $g(x)$. Note that $g(x) \nmid p(x)$ since $p(\alpha) \neq 0$. Since $g$ is irreducible, we know that $\gcd(g, p) = 1$ and so $\exists s, t \in F[x]$ s.t. 
    \begin{equation}
      1 = s p + t g \implies 1 = s(\alpha) p(\alpha) + t(\alpha) g(\alpha) = s(\alpha) p(\alpha)
    \end{equation}  
    Therefore we have found a multiplicative inverse $s = p^{-1} \in F[\alpha]$. 
  \end{proof} 
  \begin{proof}
    We can also prove it using the vector space structure, though this is for finite-dimensional vector spaces. Treating $F[\alpha]$as a finite-dimensional vector space over $F$, let us define the $F$-linear function\footnote{linearity is easy to check}
    \begin{equation}
      m_b: F[\alpha] \rightarrow F[\alpha], \qquad m_b (\beta) = b\beta
    \end{equation} 
    Since $F[\alpha] \subset K$, $F[\alpha]$ is an integral domain. Thus $\not\exists \beta \in F[\alpha] \setminus \{0\}$ s.t. $b \beta = 0$. This means that the kernel of $m_b$ is $0$, and so $m_b$ is injective. By the rank-nullity theorem, it is bijective, and so there exists a $\beta \in F[\alpha]$ s.t. $b \beta = 1 \implies b$ is a unit. 
  \end{proof} 

  Since $F[\alpha]$ is the smallest ring containing both $F$ and $\alpha$, it immediately follows that it is the smallest \textit{field} containing $F$ and $\alpha$. Therefore, we have unified the two constructions of adjoining fields and quotient rings. 

  \begin{example}[Some Adjoining Fields]
    Here are some examples. 
    \begin{enumerate}
      \item Given any algebraic number $\alpha \in \mathbb{C}$, by definition there exists a $f(x) \in \mathbb{Q}[x]$ with roots $\alpha$, and so $\mathbb{Q}(\alpha) \subset \mathbb{C}$.\footnote{In fact it was historically so common to work solely in subfields of $\mathbb{C}$ that an \textit{algebraic number} meant by default algebraic with respec to the field extension $\mathbb{C}/F$.}
      \item $\mathbb{Q}(\sqrt{3} i)$ is a field of dimension 2 since $\sqrt{3}i$ is a root of the polynomial $f(x) = x^2 + 3$. 
      \item However, $\mathbb{Q}[\pi]$ is not a field since $\pi$ is a \textit{transcendental number}. However we will not prove it now. 
    \end{enumerate}
  \end{example}

  Note that we now have the field extension tower $F \subset F[\alpha] \subset K$. By the tower property, $[K:F] = [F:F(\alpha)] [F(\alpha):F]$, and so $\deg{f(x)} = [F:F(\alpha)]$ must divide $[K:F]$. 

  Now recall quotient rings, which do not necessarily preserve the properties of the original ring. That is, if $F$ is a field, then $F/I$ may not be a field. Using the fundamental ring homomorphism theorem, we can precisely correlate certain quotient maps with adjoining fields. Recall that given a field extension $F \subset K$, the evaluation function $\ev_\alpha: F[x] \rightarrow K$ defined $f(x) \mapsto f(\alpha)$ is a homomorphism. 

  \begin{example}[Simple Quotient Rings as Field Adjoined with 1 Variable]
    Consider the following. 
    \begin{enumerate}
      \item Since $x^2 + 1 \in \mathbb{Z}_7 [x]$ is irreducible, $\mathbb{Z}_7 [x] / \langle x^2 + 1 \rangle$ is a field. 

      \item Since $x^2 + 1 \in \mathbb{R}[x]$ is irreducible of degree 2, so the quotient ring is a field. Furthermore, $i \in \mathbb{C}$ is a root of the degree 2 minimal polynomial, so we have the isomorphism 
      \begin{equation}
        \frac{\mathbb{R}[x]}{\ker(\ev_i)} = \frac{\mathbb{R}[x]}{\langle x^2 + 1 \rangle} \simeq \mathbb{R}[i] = \mathbb{C}
      \end{equation}
      induced by the evaluation map
      \begin{equation}
        \phi: \frac{\mathbb{R}[x]}{\langle x^2 + 1 \rangle} \rightarrow \mathbb{C}, \qquad \phi\big( f(x) \pmod{\langle x^2 + 1 \rangle} \big) = f(i)
      \end{equation}
      Therefore $\mathbb{R}(i)$ has $\mathbb{R}$-dimension 2, which makes sense since we view $\mathbb{C}$ as being isormophic to $\mathbb{R}^2$. 

      \item The evaluation map 
      \begin{equation}
        \ev_{\sqrt{2}}: \mathbb{Q}[x] \mapsto \mathbb{Q}[\sqrt{2}], \qquad \ev_{\sqrt{2}} (f) = f(\sqrt{2}) 
      \end{equation}
      is a homomorphism. Furthermore, it has a kernel $\langle x^2 - 2 \rangle$ since $(x^2 - 2)$ is an irreducible polynomial in $\mathbb{Q}[x]$ containing the root $\sqrt{2}$. Therefore by the fundamental ring homomorphism theorem we have 
      \begin{equation}
        \frac{\mathbb{Q}[x]}{\langle x^2 - 2 \rangle} \simeq \mathbb{Q}[\sqrt{2}]
      \end{equation}
    \end{enumerate}
  \end{example}

  \begin{example}[Extensions of $\sqrt{2}$ and $i$]
    We claim that 
    \begin{equation}
      \mathbb{Q}[\sqrt{2}, i] = \{ a + b \sqrt{2} + ci + d(\sqrt{2} i) \mid a, b, c, d \in \mathbb{Q}\}
    \end{equation}
    From the previous example, we know that $\mathbb{Q}[\sqrt{2}]$ are all numbers of the form $a + b\sqrt{2}$. Now we take $i \in \mathbb{C}$ and map it through all polynomials with coefficients in $\mathbb{Z}[\sqrt{2}]$, which will be of form 
    \begin{equation}
      f(i) = (a_n + b_n \sqrt{2}) i^n + (a_{n-1} + b_{n-1}\sqrt{2}) i^{n-1} + \ldots + (a_2 + b_2 \sqrt{2}) i^2 + (a_1 + b_1 \sqrt{2}) i + (a_0 + b_0 \sqrt{2})
    \end{equation} 
    However, we can see that since $i^2 = -1$, we only need to consider up to degree 1 polynomials of form 
    \begin{equation}
      (a + b \sqrt{2}) + (c + d \sqrt{2}) i 
    \end{equation}
    which is clearly of the desired form. For the other way around, this is trivial since we can construct a linear polynomial as before. 
  \end{example} 

  \begin{example}
    We claim $\mathbb{Q}[\sqrt{3} + i] = \mathbb{Q}[\sqrt{3}, i]$. 
    \begin{enumerate}
      \item $\mathbb{Q}[\sqrt{3} + i] \subset \mathbb{Q}[\sqrt{3}, i]$
      \item $\mathbb{Q}[\sqrt{3} + i] \supset \mathbb{Q}[\sqrt{3}, i]$. Note that 
        \begin{align}
          (\sqrt{3} + i)^3 = 8i & \implies i \in \mathbb{Q}[\sqrt{3} + i] \\
                                & \implies (\sqrt{3} + i) - i = \sqrt{3} \in \mathbb{Q}[\sqrt{3} + i] 
        \end{align}
        Therefore, $\mathbb{Q}[\sqrt{3} + i]$ contains the elements $1, \sqrt{3}, i$, which form the basis of $\mathbb{Q}[\sqrt{3}, i]$. 
    \end{enumerate}
  \end{example}

  \begin{example}[Extensions of $\sqrt{3}i$ and $\sqrt{3}, i$]
    We claim that $\mathbb{Q}[\sqrt{3} i] \subsetneq \mathbb{Q}[\sqrt{3}, i]$. 
    \begin{enumerate}
      \item We can see that $\{1, \sqrt{3}i \}$ span $\mathbb{Q}[\sqrt{3}i ]$ as a $\mathbb{Q}$-vector space. Therefore, 
      \begin{equation}
        \sqrt{3}, i \in \mathbb{Q}[\sqrt{3}, i] \implies \sqrt{3} i \in \mathbb{Q}[\sqrt{3}, i]
      \end{equation} 
      implies that $\mathbb{Q}[\sqrt{3} i] \subset \mathbb{Q}[\sqrt{3}, i]$. 

      \item To prove proper inclusion, we claim that $i \not\in \mathbb{Q}[\sqrt{3}i]$. Assuming that it can, we represent it in the basis $i = b_0 + b_1 \sqrt{3} i$, and so
      \begin{equation}
        -1 = (b_0 + b_1 \sqrt{3} i)^2 = (b_0^2 - 3b_1^2) + 2b_0 b_1 \sqrt{3} i
      \end{equation}
      Therefore we must have $2b_0 b_1 \sqrt{3} = 0 \implies b_0$ or $b_1$ should be $0$. If $b_0 = 0$, then $b_0^2 - 3b_1^2 = -3 b_1^2 \implies b_1^2 = 1/3$, which is not possible since $b_1^2 \in \mathbb{Q}$. If $b_1 = 0$, then $b_0 - 3 b_1^2 = b_0^2 > 0$, and so it cannot be $-1$. 
    \end{enumerate}
  \end{example}

  The most significant property is that a field contains inverses. So how do we compute such an inverse? 

  \begin{example}[Computing Inverses in Field Extensions]
    We can do two problems. 
    \begin{enumerate}
      \item Given $\beta = p(\alpha) = \alpha^2 + \alpha - 1 \in \mathbb{Q}[\alpha]$, where $\alpha$ is a root of $f(\alpha) = \alpha^3 + \alpha + 1$, we first know that $\beta$ must have a multiplicative inverse since $\mathbb{Q}[\alpha]$ is a field. Applying the Euclidean algorithm, we have 
      \begin{equation}
        1 = \frac{1}{3} \big\{ (x+1) f(x) - (x^2 + 2) p(x)\big\} = -\frac{1}{3} (\alpha^2 + 2) p(\alpha)
      \end{equation}
      and so $\beta^{-1} = (\alpha^2 + \alpha - 1)^{-1} = -\frac{1}{3} (\alpha^2 + 2)$. We can check that 
      \begin{align}
        -\frac{1}{3} (\alpha^2 + 2) (\alpha^2 + \alpha - 1) & = -\frac{1}{3} (\alpha^4 + \alpha^3 + \alpha^2 + 2 \alpha - 2) \\
                                                            & = -\frac{1}{3} (\alpha^3 + \alpha - 2) \\
                                                            & = -\frac{1}{3} (-3) = 1
      \end{align}
      \item Given $f(x) = x^3 + 3 \in \mathbb{Q}[x]$, we know (from Eisenstein) that it is irreducible. Therefore we know that $\mathbb{Q}[x]/\langle f(x) \rangle$ is a field. Furthermore, it contains a root $\alpha$, and so 
      \begin{equation}
        \frac{F[x]}{\langle x^3 + x \rangle} \simeq F[\alpha]
      \end{equation}
      Now take an arbitrary element of the field, say $f(\alpha) = \alpha^2 - 1$. Then it must have an inverse call it $g(\alpha)$ that satisfies 
      \begin{equation}
        1 = f(\alpha) g(\alpha) \text{ in } \frac{\mathbb{Q}[x]}{\langle x^3 + 3 \rangle}
      \end{equation}
      But by working in the original ring $\mathbb{Q}[x]$ and not the quotient, this would look something like. 
      \begin{equation}
        1 = f(x) g(x) + k(x) (x^3 + 1) \text{ in } \mathbb{Q}[x]
      \end{equation} 
      for some $k(\alpha)$. Note that we are guaranteed to get this form since $f(x)$ is irreducible and so $\gcd(f, g) = 1$. So we use Euclidean division. 
      \begin{align}
        x^3 + 3 & = (x) (x^2 - 1) + (x + 3) \label{hi}\\
        x^2 - 1 & = (x - 3) (x + 3) + 8
      \end{align}
      Therefore by substituting we have 
      \begin{equation}
        8 = (x - 3) (x^3 + 3) - (x^2 - 2x - 1) (x^2 - 1) 
      \end{equation} 
      and so by taking $\pmod{x^3 + 3}$, we have 
      \begin{equation}
        1 = -\frac{1}{8} (x^2 - 3x - 1) (x^2 - 1) \implies \frac{1}{\alpha^2 - 1} = -\frac{1}{8}(\alpha^2 - 3 \alpha - 1)
      \end{equation}
    \end{enumerate}
  \end{example}

  \begin{example}[Irreducible Quotient Rings]
    On the other hand, if we consider the ring $\mathbb{Z}_5 [x] / \langle x^2 - 2 \rangle$, then $x^2 - 2$ is irreducible in $\mathbb{Z}_5 [x]$ (just plug in $0, 1, 2, 3, 4$), and so since $\mathbb{Z}_5$ is a field, the quotient ring is a field. Then we have 
    \begin{equation}
      \frac{\mathbb{Z}_5 [x]}{\langle x^2 - 2 \rangle} \simeq \mathbb{Z}_5 [\sqrt{2}] = \{a + b \sqrt{2} \mid a, b \in \mathbb{Z}_5 \}
    \end{equation}
  \end{example} 

  Remember that we should not be careless and just mindlessly assume that every quotient ring is a field. This is only when the ideal is generated by a minimal polynomial. Here is an example where the ideal is reducible, but fortunately using the chinese remainder theorem, we can first reduce it as a direct product of quotient rings, and then apply our theorem to represent them as a field structure. 

  \begin{example}[Quotient Ring as Product Rings with Chinese Remainder Theorem]
    Consider the following ring $\mathbb{Z}_5 [x] / \langle x^2 + 1 \rangle$. Then $x^2 + 1$ is reducible since $x^2 + 1 = (x + 2) (x + 3)$, which implies $\langle x^2 + 1 \rangle = \langle x + 2 \rangle \cap \langle x + 3 \rangle$, and so by the Chinese remainder theorem, we have 
    \begin{equation}
      \frac{\mathbb{Z}_5 [x]}{\langle x^2 + 1 \rangle} = \frac{\mathbb{Z}_5 [x]}{\langle x + 2 \rangle \cap \langle x + 3 \rangle} = \frac{\mathbb{Z}_5 [x]}{\langle x + 2 \rangle} \times \frac{\mathbb{Z}_5 [x]}{\langle x + 3 \rangle} \simeq \mathbb{Z}_5 \times \mathbb{Z}_5
    \end{equation}
  \end{example} 

\subsection{Splitting Fields} 

  Great, so we have established conditions for which a field adjoined with an element becomes another field. All that is left to do is to find a splitting field.

  Remember that by previously establishing that $\mathbb{C}$ is algebraically closed, this gives us a ``safe space'' to work in, in the sense that if we take any subfield $F \subset \mathbb{C}$ and find a polynomial $f(x) \in F[x]$, we are \textit{guaranteed} to find a linear factorization of $f$ in $\mathbb{C}[x]$. Therefore, if $K$ is algebraically closed and $F \subset K$ is a field extension, $f(x) \in F[x]$ is guaranteed to \textit{split} completely into linear factors. This is true for \textit{all} $f(x) \in F[x]$, but now if we \textit{fix} $f(x) \in F[x]$, perhaps we don't need the entire field $K$ to split $f(x)$. Maybe we can work in a slightly larger field $E$---such that $F \subset E \subset K$---where $f(x)$ splits in $E$. This process of finding such a minimal field is important to understand the behavior of roots of such polynomials. 

  \begin{definition}[Splitting Field]
    Given a field extension $F \subset K$ and a polynomial $f \in F[x]$, 
    \begin{enumerate}
      \item $f$ \textbf{splits} in $K$ if $f$ can be written as the product of linear polynomials in $K[x]$. 
      \item If $f$ splits in $K$ and there exists no field $E$ s.t. $F \subsetneq E \subsetneq K$, then $K$ is called a \textbf{splitting field} of $f$.\footnote{i.e. the splitting field is the ``smallest'' field that splits $f$.} 
    \end{enumerate}
    We claim that a splitting field always exists for any $f(x) \in F[x]$.
  \end{definition}
  \begin{proof}
    Let us decompose $f(x)$ into irreducible factors over $F$: $f(x) = f_1(x)f_2(x)\cdots f_k(x)$. If all of them are linear, then $F$ is the splitting field for $f(x)$. Otherwise, we can assume that $f_1(x)$ is not linear. Then we can consider the field $K_1 = F[x]/\langle f_1(x) \rangle$, where $f_1(x)$ has a root $\alpha$. Then $f_1(x) = (x - \alpha)g_1(x)$, and $f(x)$ has at least one (but maybe more) linear factor over $K_1$. If all irreducible factors over $K_1$ are linear, stop, otherwise there is an irreducible factor of degree at least 2, and we can repeat the procedure and add its root. Since a polynomial of degree $n$ has at most $n$ roots, the process will eventually stop and all factors will be linear in some extension of $F$.
  \end{proof} 

  It also turns out to be unique but we will prove this at the end of the section. 

  \begin{example}[Don't Need Necessarily Complex Numbers to Split]
    Consider the following subfields of $\mathbb{C}$ and observe that they are enough to split a given polynomial. 
    \begin{enumerate}
      \item Let $f(x) = x^2 - 1$. If $f(x) \in \mathbb{R}[x]$, it does split in $\mathbb{R}$. In fact, even if we consider it as an element of $\mathbb{Z}_2 [x]$, it still splits into $(x + 1)(x - 1)$. 
      \item Let $f(x) = x^2 - 2$. If $f(x) \in \mathbb{Q}[x]$, it doesn't split in $\mathbb{Q}$ since the roots $\pm \sqrt{2} \not\in \mathbb{Q}$, but $\pm \sqrt{2}$ are real numbers, so $f(x)$ does in fact split in $\mathbb{R}$ since it splits into $(x + \sqrt{2}) (x - \sqrt{2})$. However, maybe it is not the (smallest) splitting field. 
      \item Let $f(x) = x^2 + 1$. We can see that if we consider it as an element of $\mathbb{Q}[x]$ or $\mathbb{R}[x]$, neither fields split $f(x)$ since $\pm i$ are its roots and therefore are contained in the coefficients of its linear factors. We know that it definitely splits in $\mathbb{C}$, but can we find a smaller field that splits $f(x)$? Perhaps.  
    \end{enumerate}
  \end{example}

  So how does one find a splitting field? Note that in the example above, we have found that there were some roots $\alpha$ of certain polynomials $f(x) \in F[x]$ are not contained in $F$. Therefore, what we want to do is find the smallest field $F$ containing both $F$ and $\alpha$ (plus any other $\alpha$'s). This is precisely the adjoining field $F[\alpha]$, which guarantees unique factorization since $F[\alpha]$ is a Euclidean domain. 

  \begin{lemma}[Square-Free Extensions as Splitting Field]
    If $a$ is not a perfect square in $F$ then $F(\sqrt{a})$ is the splitting field of $f(x) = x^2 - a$. 
  \end{lemma}
  \begin{proof}
    TBD
  \end{proof}

  We first provide some straightforward examples of computing splitting fields. 

  \begin{example}[Straightforward Computation of Splitting Fields]
    For some polynomials, finding their roots is trivial. 
    \begin{enumerate}
      \item Let $f(x) = (x^2 - 2) \in \mathbb{Q}[x]$. Then the splitting field is $\mathbb{Q}(\sqrt{2})$. 

      \item Let $f(x) = (x^2 - 2)(x^2 - 3) \in \mathbb{Q}[x]$. Then the splitting field is $\mathbb{Q}(\sqrt{2}, \sqrt{3})$. 
        
      \begin{figure}[H]
        \centering 
        \begin{tikzcd}
          & \mathbb{Q}(\sqrt{2}, \sqrt{3}) \arrow[ld, "2"'] \arrow[d, "2"] \arrow[rd, "2"] & \\
          \mathbb{Q}(\sqrt{2}) \arrow[rd, "2"'] & \mathbb{Q}(\sqrt{6}) \arrow[d, "2"] & \mathbb{Q}(\sqrt{3}) \arrow[ld, "2"] \\
          & \mathbb{Q} &
        \end{tikzcd}
        \caption{Diagram of known subfields of $\mathbb{Q}(\sqrt{2}, \sqrt{3})$.} 
      \end{figure}
    \end{enumerate}
    Note that 
  \end{example}

  \begin{example}[Slightly Harder Computation of Splitting Fields]
    Note that often, the splitting field is ``smaller'' that one might suspect. 
    \begin{enumerate}
      \item Let $f(x) = x^2 + 2x + 2 \in \mathbb{Q}[x]$. Then the roots of $f(x)$ are $-1 \pm i$, so 
      \begin{equation}
        f(x) = (x - (-1 + i)) (x - (-1 - i)) 
      \end{equation}
      and we can show that $\mathbb{Q}[-1 - i, -1+i] = \mathbb{Q}[i]$ is the splitting field of $f$. It has dimension $2$ since $f(x)$ is a 2nd degree polynomial that is minimal. 

      \item Let $f(x) = x^2 - 2x - 1 \in \mathbb{Q}[x]$. The roots are $1 \pm \sqrt{2}$, and so 
      \begin{equation}
        f(x) = (x - (1 + \sqrt{2})) (x - (1 - \sqrt{2}))
      \end{equation}
      and so $\mathbb{Q}[\sqrt{2}]$ is the splitting field of $f$. Since $f(x)$ is a 2nd degree polynomial that is minimal. 

      \item Let $f(x) = x^6 - 1 \in \mathbb{Q}[x]$. We can factor 
      \begin{equation}
        f(x) = (x-1) (x + 1) (x^2 + x + 1) (x^2 - x + 1)
      \end{equation} 
      and the non-rational roots are $\frac{\pm 1 \pm \sqrt{3} i}{2}$. Thus the splitting field of $f$ is $\mathbb{Q}[\sqrt{3} i]$. The dimension is not 6 because $f(x)$ is reducible over $\mathbb{Q}$. It is $2$. It is easier to look at this not as a quotient ring, but use the first theorem to find a minimal polynomial with root $\sqrt{3} i$. Indeed, such a polynomial is $x^2 + 3$, which has degree $2$. 

      \item Let $f(x) = x^4 - 2 \in \mathbb{Q}[x]$, which has roots $\{ \sqrt[4]{2}, \sqrt[4]{2} e^{\frac{2\pi i}{4}}, \sqrt[4]{2} e^{\frac{4\pi i}{4}}, \sqrt[4]{2} e^{\frac{6\pi i}{4}} \}$ and thus the splitting field is 
      \begin{equation}
        \mathbb{Q} \big( \sqrt[4]{2}, \sqrt[4]{2} e^{\frac{2\pi i}{4}}, \sqrt[4]{2} e^{\frac{4\pi i}{4}}, \sqrt[4]{2} e^{\frac{6\pi i}{4}} \big) = \mathbb{Q}(\sqrt[4]{2}, e^{\frac{2\pi i}{4}})
      \end{equation}
      The inclusion $\subset$ is easy to prove, and to prove $\supset$, we can see that since we are working in a field, 
      \begin{equation}
        e^{2 \pi i / 4} = \frac{\sqrt[4]{2} e^{2\pi i/4}}{\sqrt[4]{2}} \in \mathbb{Q} \big( \sqrt[4]{2}, \sqrt[4]{2} e^{\frac{2\pi i}{4}}, \sqrt[4]{2} e^{\frac{4\pi i}{4}}, \sqrt[4]{2} e^{\frac{6\pi i}{4}} \big) 
      \end{equation}
      which implies that $\sqrt[4]{2} \in \mathbb{Q} \big( \sqrt[4]{2}, \sqrt[4]{2} e^{\frac{2\pi i}{4}}, \sqrt[4]{2} e^{\frac{4\pi i}{4}}, \sqrt[4]{2} e^{\frac{6\pi i}{4}} \big)$. $f(x)$ is certainly minimal in $\mathbb{Q}[x]$ containing all roots, so the dimension of the field is $4$. It is spanned by the powers of $\sqrt[4]{2}$. 
    \end{enumerate}
  \end{example}

  \begin{theorem}[Bounds on Degree of Splitting Field]
    A splitting field of a polynomial $f(x) \in F[x]$ over $F$ is of degree at most $n!$ over $F$. 
  \end{theorem}
  \begin{proof}
    The polynomial can have at most $n$ roots, call them $\alpha_1, \ldots, \alpha_n$. 
    \begin{enumerate}
      \item Then, assume that $F$ is irreducible and so $[F[\alpha_1]: F] = n$. 
      \item Now factor $f(x) = (x - \alpha_1) g(x) \in F[\alpha_1][x]$. If $g(x)$---which has degree $n-1$---is irreducible, $[F[\alpha_1, \alpha_2]:F[\alpha_1]] = n-1$. 
    \end{enumerate}
    We will get the maximal degree at each step if the factored polynomial is irreducible. Then using the tower property, we have 
    \begin{equation}
      [F[\alpha_1, \ldots, \alpha_n]:F] = \prod_{i=0}^n [F[\alpha_1, \ldots, \alpha_{i+1}]: F[\alpha_1, \ldots, \alpha_i]] = n!
    \end{equation}
  \end{proof}

  \begin{example}[Splitting Field of $x^p - 1$]
    Let $p$ be prime, consider the polynomial $f(x) = x^p - 1$ over $\mathbb{Q}$. Let $\omega = e^{2\pi i/p}$, then $\omega$ is a root of $f(x)$ and we have
    \begin{equation}
      x^p - 1 = (x - 1)(x - \omega)(x - \omega^2)\cdots(x - \omega^{p-1}),
    \end{equation}
    so all roots of $f(x)$ belong to the extension $\mathbb{Q}(\omega)$. Since the minimal polynomial for $\omega$ equals $x^{p-1} + \ldots + 1$, we have $[\mathbb{Q}(\omega) : \mathbb{Q}] = p - 1$.
  \end{example}

  \begin{example}[Splitting Field of $x^p - 2$, $p$ Prime]
    Consider $f(x) = x^p - 2 \in \mathbb{Q}[x]$. If $\alpha$ is a root, i.e. $\alpha^p = 2$, then $(\zeta \alpha)^p = 2$, where $\zeta$ is any $p$th root of unity. Hence the $p$ complex roots of $f(x)$ are $\zeta \sqrt[p]{2}$ for the $p$ roots of unity $\zeta$. It then follows that the splitting field is $\mathbb{Q}(\sqrt[p]{2}, \zeta)$. Now to compute the degree of this field extension, note that
    \begin{equation}
      [\mathbb{Q}(\sqrt[p]{2}, \zeta): \mathbb{Q}] = [\mathbb{Q}(\sqrt[p]{2}, \zeta): \mathbb{Q}(\sqrt[p]{2})] [\mathbb{Q}(\sqrt[p]{2}): \mathbb{Q}] = (p - 1) p
    \end{equation}
    which is true since it is easy to show that $[\mathbb{Q}[\sqrt[p]{2}]: \mathbb{Q}] =  p$, and $x^2 - 2 = (x - \sqrt[p]{2}) (x^{p-1} + \ldots + 1) \in \mathbb{Q}[\sqrt[p]{2}]$, where $x^{p-1} + \ldots + 1$ turns out to be irreducible---and hence the degree is $n-1$. We could have done this the other way by decomposing 
    \begin{equation}
      [\mathbb{Q}(\sqrt[p]{2}, \zeta): \mathbb{Q}] = [\mathbb{Q}(\sqrt[p]{2}, \zeta): \mathbb{Q}(\zeta)] [\mathbb{Q}(\zeta): \mathbb{Q}] = (p - 1) p
    \end{equation}
    which would yield the same result as proven in \ref{def:adjoining_ring}.

    \begin{figure}[H]
      \centering 
      \begin{tikzcd}[row sep=small, column sep=small]
        & \mathbb{Q}(\sqrt[p]{2}, \zeta) \arrow[ddl, "p"'] \arrow[dr, "p-1"] \arrow[drrr, "p-1"] & & & \\  
        & & \mathbb{Q}(\zeta^{0} \sqrt[p]{2}) \arrow[ddl, "p"] & \ldots & \mathbb{Q}(\zeta^{p-1} \sqrt[p]{2}) \arrow[ddlll, "p"] \\ 
        \mathbb{Q}(\zeta) \arrow[dr, "p-1"'] & & & & \\
        & \mathbb{Q} & & & 
      \end{tikzcd} 
      \caption{Diagram of known subfields of $\mathbb{Q}(\sqrt[p]{2}, \zeta)$. } 
      \label{fig:splittingfieldx_2}
    \end{figure}
  \end{example}

  Note that the order in which we added the adjoined the elements to the base field did not matter in the example above. We now formalize this. 

  \begin{theorem}[Extending Isomorphisms of Fields]
    \label{thm:extending_field_iso}
    Let $\phi: F \to F^\prime$ be an isomorphism of fields.
    \begin{enumerate}
      \item Let $f(x) \in F[x]$ be an irreducible polynomial with root $\alpha$ in some field extension $K \supset F$.
      \item Let $f^\prime(x) = \phi(f(x)) \in F^\prime[x]$ and let $\beta$ be a root of $f^\prime(x)$ in some field extension $K^\prime \supset F^\prime$. 
    \end{enumerate}
    Then there is a unique isomorphism $\bar{\phi} : F[\alpha] \to F^\prime [\alpha^\prime]$ that is an extension of $\phi$ (i.e. behaves the same under $F$) and carries $\alpha$ to $\alpha$. 
    \begin{figure}[H]
      \centering 
      \begin{tikzcd}
        K \arrow[r, "\bar{\phi}"] \arrow[d] & K^\prime \arrow[d] \\
        F \arrow[r, "\phi"] & F^\prime
      \end{tikzcd}
      \caption{Commutative diagram.} 
      \label{fig:field_isomorphism_extension}
    \end{figure}
  \end{theorem}
  \begin{proof}
    We use induction on the degree $n$ of $f(x)$. Recall that a field (i.e. a ring) isomorphism $\phi: F \to F^\prime$ induces a ring isomorphism $\Tilde{\phi}: F[x] \to F^\prime [x]$. jSo, if $f(x)$ and $f^\prime (x)$ correspond to one another under this isomorphism, then the irreducible factors of $f(x) \in F[x]$ correspond to the irreducible factors of $f^\prime (x) \in F^\prime [x]$. 

    If $f(x)$ has all its roots in $F$ then $f(x)$ splits completely in $F[x]$ and $f^\prime (x)$ splits completely in $F^\prime [x]$---with its linear factors being the images of the linear factors for $f(x)$. Hence $F = K$ and $F^\prime = K^\prime$, and in this case we may take $\bar{\phi} = \phi$. This shows the result is true for $n = 1$ and in the case where all the irreducible factors of $f(x)$ have degree $1$. 

    Assume now by induction that the theorem holds for any field field $F$, isomorphism $\phi$, and polynomial $f(x) \in F[x]$ of degree $< n$. Let $p(x)$ be an irreducible factor of $f(x) \in F[x]$ of degree at least $2$ and let $p^\prime (x)$ be the corresponding irreducible factor of $f^\prime (x) \in F^\prime [x]$. If $\alpha \in K$ is a root of $p(x)$ and $\beta \in K^\prime$ is a root of $p^\prime (x)$, then we can extend $\phi$ to an isomorphism $\phi^\prime: F(\alpha) \to F^\prime (\beta)$. 

    \begin{figure}[H]
      \centering 
      \begin{tikzcd}
        F(\alpha) \arrow[r, "\phi^\prime"] \arrow[d] & F^\prime (\beta) \arrow[d] \\
        F \arrow[r, "\phi"] & F^\prime
      \end{tikzcd}
    \end{figure}

    Let $F_1 = F(\alpha), F_1^\prime = F^\prime (\beta)$, so that we have the isomorphism $\phi^\prime :F_1 \to F_1^\prime$. We have $f(x) = (x - \alpha) f_1 (x)$ over $F_1$ with $\deg{f(x)} = n-1$ and $f^\prime (x) = (x - \beta) f_1^\prime(x)$. $K$ is a splitting field for $f_1 (x)$ over $F_1$ since: all the roots of $f_1 (x)$ are in $K$ and if they were contained in any smaller extension $L$ containing $F_1$, then, since $F_1$ contains $\alpha$, $L$ would also contain all the roots of $f(x)$, which would contradict the minimality of $K$ as the splitting field of $f(x)$ over $F$. Similarly, $K^\prime$ is a splitting field for $f_1^\prime (x)$ over $F_1^\prime$. Since the degrees of $f_1 (x)$ and $f_1^\prime (x)$ are less than $n$, by induction there exists a map $\phi: K \to K^\prime$ extending the isomorphism $\sigma^\prime: F_1 \to F_1^\prime$. This gives the extended diagram. 

    \begin{figure}[H]
      \centering 
      \begin{tikzcd}
        K \arrow[r, "\phi^\prime"] \arrow[d] & K^\prime \arrow[d] \\
        F_1 \arrow[r, "\phi^\prime"] \arrow[d] & F^\prime_1 \arrow[d] \\
        F \arrow[r, "\phi"] & F^\prime
      \end{tikzcd}
    \end{figure}
  \end{proof}

  \begin{corollary}[Splitting Field is Unique]
    The splitting field of $f(x) \in F[x]$ is unique up to isomorphism. 
  \end{corollary}
  \begin{proof}
    Take $\phi$ to be the identity mapping from $F$ to itself, and $K, K^\prime$ to be two splitting fields for $f(x) = f^\prime (x)$. 
  \end{proof}

  So far, we have fixed one polynomial and studied the unique splitting field of $f(x)$. Now what if we unfix $f(x)$? 

  \begin{definition}[Algebraic Closure]
    The field $\overline{F}$ is called an \textbf{algebraic closure} of $F$ if $\overline{F}$ is algebraic over $F$ and if every polynomial $f(x) \in F[x]$ splits completely over $\overline{F}$. That is, $\overline{F}$ contains all the elements algebraic over $F$. 
  \end{definition}

  \begin{lemma}[Algebraic Closures are Algebraically Closed]
    Let $\overline{F}$ be an algeraic closure of $F$, then $\overline{F}$ is algebraically closed. 
  \end{lemma}
  \begin{proof}
    
  \end{proof}

  \begin{theorem}
    For any field $F$ there exists an algebraically closed field $K$ containing $F$.
  \end{theorem}
  \begin{proof}
    
  \end{proof}

  In integral domains, by taking the field of fractions we can get a nice set of formulas often introduced in high-school math competitions. 

  \begin{theorem}[Viete's Formulas]
    Let $R[x]$ be an integral domain $f(x) \in R[x]$, and $F$ be the field of fractions of $R$. If $f(x)$ splits in an algebraically closed field extension $K$\footnote{Normally we take $R = \mathbb{Z}$, its field of fractions to be $\mathbb{Q}$, and its algebraically closed extension to be $K$.} then
    \begin{equation}
      f(x) = a_0 \prod_{i = 1}^{n} (x - \alpha_i)
    \end{equation}
    for some $\alpha_1, \ldots, \alpha_n \in K$, then the coefficients of $f$ can be presented with the formulas
    \begin{align}
      \sum_{i=1}^n \alpha_i & = - \frac{a_1}{a_0} \\
      \sum_{i_1 < i_2} \alpha_{i_1} \alpha_{i_2} & = \frac{a_2}{a_0} \\
      \sum_{i_1< ...< i_k} \prod_{j = 1}^{k} \alpha_{i_j} & = (-1)^k \frac{a_k}{a_0} \\
      \alpha_1 \alpha_2 \alpha_3 ... \alpha_n & = (-1)^n \frac{a_n}{a_0}
    \end{align}
  \end{theorem}
  \begin{proof}
    
  \end{proof}

\subsection{Finite Fields and Separability of Extensions} 

  One property of polynomials that we have defined---yet have not studied much---was the multiplicity of its roots. Let's just introduce a quick definition and then provide a nice theorem to check muliplicity of roots.  

  \begin{definition}[Separability of Polynoials]
    A polynomial $f(x) \in F$ is \textbf{separable} if it has no repeated roots, i.e. no root of multiplicity greater than $1$. A polynomial which is not separable is called inseparable. 
  \end{definition}

  Now we introduce the criterion to check separability. We introduce the derivative, which does coincide with the definition seen in analysis, but note that this is purely algebraic and should not be seen to have any connection with derivatives in analysis.  

  \begin{definition}[Derivative]
    The \textbf{derivative} of a polynomial $f(x) = a_n x^n + a_{n-1} x^{n-1} + \ldots + a_1 x + a_0 \in F[x]$ is defined as an operator $D_x : F[x] \to F[x]$ 
    \begin{equation}
      D_x f(x) = f^\prime (x) \coloneqq n a_n x^{n-1} + (n-1) a_{n-1} x^{n-2} + \ldots + 2 a_2 x + a_1 \in F[x] 
    \end{equation}
  \end{definition}

  \begin{lemma}[Properties of the Polynomial Derivative]
    $D_x: F[x] \to F[x]$ satisfies the properties. 
    \begin{enumerate}
      \item \textit{Linearity}. 
        \begin{align}
          D_x (cf)(x) & = c D_x f(x) \\
          D_x (f + g)(x) & = D_x f(x) + D_x g(x) 
        \end{align}

      \item \textit{Product Rule}. 
        \begin{equation}
          D_x (fg)(x) = f(x) \big( D_x g(x) \big) + \big( D_x f(x) \big) g(x)
        \end{equation}
    \end{enumerate}
  \end{lemma}
  \begin{proof}
    Trivial through straightforward computation. 
  \end{proof}

  \begin{theorem}[Conditions for Separability]
    Let $f(x) \in F[x]$. 
    \begin{enumerate}
      \item $f(x)$ has multiple roots $\alpha$ if and only if $\alpha$ is also a root of $D_x f(x)$, i.e. $f(x)$ and $D_x f(x)$ are both divisible by the minimal polynomial for $\alpha$. 
      \item $f(x)$ is separable if and only if it is relatively prime to $D_x f(x)$, i.e. $\gcd\big( f(x), D_x f(x) \big) = 1$. 
    \end{enumerate}
  \end{theorem}
  \begin{proof}
    We prove bidirectionally. 
    \begin{enumerate}
      \item $(\rightarrow)$. Suppose that $\alpha$ is a multiple root of $f(x)$. Then over a splitting field, 
      \begin{equation}
        f(x) = (x - \alpha)^n g(x)
      \end{equation}
      for some $n \geq 2$ and some polynomial $g(x)$. Taking derivatives we get 
      \begin{equation}
        D_x f(x) = n (x - \alpha)^{n-1} g(x) + (x - \alpha)^n D_x g(x) 
      \end{equation}
      which implies that $\alpha$ is a root of $(D_x f)(x)$. 

      \item $(\leftarrow)$. Conversely, suppose that $\alpha$ is a root of both $f(x)$ and $D_x f(x)$. Then we write $f(x) = (x - \alpha) h(x)$ for some polynomial $h(x)$ and taking the derivative, we get 
        \begin{equation}
          D_x f(x) = h(x) + (x - \alpha) D_x h(x) \implies 0 = D_x f(\alpha) = h(\alpha)
        \end{equation} 
        and so $\alpha$ is a root of $h$. Hence $h(x) = (x - \alpha) h_1 (x)$ for some polynomial $h_1 (x)$, and so $f(x) = (x - a)^2 h_1 (x)$. 
    \end{enumerate}
  \end{proof}

  \begin{corollary}[Separability over Field of Characteristic $0$]
    \label{thm:sep_char}
    Every irreducible polynomial over a field of characteristic $0$ is separable. A polynomial over such a field is separable if and only if it is the product of distinct irreducible polynomials. 
  \end{corollary}
  \begin{proof}
    Since every field of characteristic $0$ contains $\mathbb{Q}$ as a subfield, it suffices to focus on $\mathbb{Q}$. Let $f(x) \in \mathbb{Q}[x]$ be irreducible and $\alpha \in \mathbb{C}$ be its root with multiplicity $k > 1$. Then 
    \begin{equation}
      f(x) = (x - \alpha)^k g(x)
    \end{equation}
    for some $g(x)$. Taking the derivative we have 
    \begin{equation}
      f^\prime(x) = k (x - \alpha)^{k-1} g(x) + (x - \alpha)^k g^\prime (x) = (x - \alpha)^{k-1} \big( k g(x) + (x - \alpha) g^\prime(x) \big)
    \end{equation}
    which also has $\alpha$ as a root. So since $(x - \alpha)$ divides both $f$ and $f^\prime$, it divides $\gcd(f, f^\prime)$. 
  \end{proof}

  By estabalishing a bit of theory of separability, can now detour here to find a classification of \textit{all} finite fields, which is quite a powerful result. We have all the tools needed for this. We know that a field---as an integral domain---has characteristic $0$ or prime $p$. We also know that a field is a vector space, at least over itself. But now that we have shown that a field can be modeled as a vector space, we can apply this to finite fields. 

  \begin{theorem}[Characteristic Determines Base Field of Vector Space]
    \label{thm:char_field}
    Given a field $F$, 
    \begin{enumerate}
      \item If $\Char(F) = p$, then $F$ is a vector space over $\mathbb{Z}_p$. 
      \item If $\Char(F) = 0$, then $F$ is a vector space over $\mathbb{Q}$. 
    \end{enumerate}
  \end{theorem}
  \begin{proof}
    
  \end{proof} 

  \begin{definition}[Frobenius Endomorphism]
    Let $F_p$ be a field of characteristic $p$. Then the map 
    \begin{equation}
      \phi: F_p \to F_p, \qquad \phi(a) = a^p
    \end{equation}
    is an injective field endormophism, called the \textbf{Frobenius endomorphism}. That is, it satisfies the following. 
    \begin{equation}
      (a + b)^p = a^p + b^p, \qquad (ab)^p = a^p b^p
    \end{equation}
    which is often called the \textit{Freshman's dream}. 
  \end{definition}
  \begin{proof}
    We prove the following properties. 
    \begin{enumerate}
      \item \textit{Addition}. We have 
      \begin{equation}
        (a + b)^p = \sum_{k = 0}^p \binom{p}{k} a^{p-k} b^{k}
      \end{equation}
      It is clear that 
      \begin{equation}
        \binom{p}{k} = \frac{p (p-1) ... (p - k+1)}{k!}
      \end{equation}
      is divisible by $p$ for all $k \neq 0, p$, so all the middle terms must cancel out to $0$. 

      \item \textit{Multiplication}. 

      \item \textit{Injectivity}. 
    \end{enumerate}
  \end{proof}

  Therefore, just from the characteristic we can classify all fields as vector spaces over either $\mathbb{Q}$ or $\mathbb{Z}_p$. Now if we focus on finite fields, we can do a reverse classification. 

  \begin{theorem}[Finite Fields Have Cardinality $p^d$]
    Let $F$ be a finite field. Then $|F| = p^n$ for some $n \in \mathbb{N}$. 
  \end{theorem}
  \begin{proof}
    $F$ is a vector space over $\mathbb{Z}_p$ from \ref{thm:char_field}. Since $F$ has finitely many elements, $F$ has a finite spanning set, which implies $\dim_{\mathbb{Z}_p} F \leq + \infty$. Let $d$ be the dimension and $\{b_1, \ldots, b_d\}$ be the basis. The elements of $F$ are 
    \begin{equation}
      a_1 b_1 + \ldots + a_d b_d
    \end{equation}
    with $a_1, \ldots a_d \in \mathbb{Z}_p$. Thus there are $p^d$ elements of $F$, so $F \simeq \mathbb{Z}_p^d$. 
  \end{proof}

  In fact, for \textit{every} prime power there exists a unique field. Therefore we can create a bijection by proving the converse. 

  \begin{theorem}[Field for Every $p^d$]
    For every prime $p$ and $n \in \mathbb{N}$, there exists a field with $q = p^d$ elements, unique up to isomorphism.  
  \end{theorem} 
  \begin{proof} 
    Let $f(x) = x^q - x \in \mathbb{Z}_p [x]$. Then this polynomial has a splitting field $K \supset \mathbb{Z}$. Now we claim the roots of $f(x)$ in $K$ are distinct and form a subfield $F_q \subset K$. This will complete the proof since $F_q \subset K$ and $K \subset F_q \implies K = F_q$. Assume $\alpha, \beta \in K$ are roots of $f(x)$, and so $\alpha^p = \alpha$ and $\beta^p = \beta$
    \begin{enumerate}
      \item $\alpha + \beta \in K$ since by a modification of Freshman's dream, $(\alpha + \beta)^p = \alpha^p + \beta^p = \alpha + \beta$.\footnote{We induct on $n$ for $q = p^n$. For $n=1$, this is trivial by Freshmans dream. Now assume it holds for some $n \in \mathbb{N}$. Then $(x + y)^{p^{n+1}} = ( (x + y)^{p^n} )^p = (x^{p^n} + y^{p^n})^p = (x^{p^n})^p + (y^{p^n})^p = x^{p^{n+1}} + y^{p^{n+1}}$. }
      \item $(-\alpha)^q = (-1)^q \alpha^q = (-1)^q \alpha = -\alpha$ since $-1 = 1$ or $q$ is odd. 
      \item $\alpha \beta \in K$ since $\mathbb{Z}_p$ is a field and so $(\alpha \beta)^p = \alpha^p \beta^p = \alpha \beta$. 
      \item For multiplicative inverses, let $\alpha \neq 0$. Then $(\alpha^{-1})^p = (\alpha^{p})^{-1} = \alpha^{-1}$. 
      \item For all $p$, $0$ and $1$ are roots so $0, 1 \in K$. 
    \end{enumerate}
    Now we show that $K$ consists of distinct roots. Certainly $0 \in K$ with multiplicity $1$ since $f(x) = x (x^{q-1} - x)$. Now suppose nonzero $r \in K$ is a root with multiplicity $m$. The multiplicity of $r$ is the multiplicity of $0$ of 
    \begin{equation}
      f(x + r) = (x + r)^q - (x + r) = x^q + r^q - x - r = x^q - x
    \end{equation}
    where the final step follows from $0 = r^q - r$ since $r \in K$. Therefore $r$ has multiplicity $1$. Since $K[x]$ has unique factorization property, it follows that $m=1$ and every $r$ is a simple root. 

    To show that every field with $p^n$ elements is unique, let $F$ be such a field. We claim that $\Char(F) = p \implies \mathbb{Z}_p \subset F$. We claim that every element of $F$ is a root of $f(x) = x^q - x \in \mathbb{Z}_p [x]$, where $F$ is the splitting field. Let $G = F^\ast$ be the multiplicative group of units. Since $F$ is a field, then $|F^\ast| = |F| - 1 = p^d - 1$, and by constructing the cyclic group $\langle g \rangle \subset G$ for any $g \in G$, we know by Lagrange's theorem that $g^{|G|} = 1_G$, which implies that for all $x \in F$, 
    \begin{enumerate}
      \item If $x \neq 0$ then $x^{p^d - 1} = x \implies x^{p^d} = x$ and so $x \in K$. 
      \item If $x = 0$ then $x^{p^d} - x = 0$ and so $x \in K$. 
    \end{enumerate}
    Therefore $F \subset K$ with $|F| = |K|$ both finite, and so $F = K$. 
  \end{proof} 

  From this, we can write for every prime $p$ and natural $n$ the finite field of order $p^n$ as $\mathbb{F}_{p^n}$. It is clear that if $n = 1$ then $\mathbb{F}_p \simeq \mathbb{Z}_p$. The final result we will show is a hierarchy of subfields. 

  \begin{theorem}[Hierarchy of Fields]
    For a given prime $p$, if $p^m < p^n$, then 
    \begin{equation}
      F_{q^m} \subset F_{q^n} \iff m \mid n
    \end{equation}
  \end{theorem}

\subsection{Automorphism Groups} 

  Now we delve into the heart of Galois theory, which considers the relation of the group of permutations of the roots of $f(x)$ to the algebraic structure of its splitting field. The connection is given by the fundamental theorem of Galois theory. 

  As we stated in the beginning of this section, we want to look at symmetries of the roots of a polynomial. More concretely, given a polynomial $f(x) \in F[x]$, we can construct its splitting field $K$. TBD

  More concretely, rather than directly computing all roots $\alpha_1, \ldots, \alpha_n$ of a polynomial $f(x) \in F[x]$---which may be extremely hard---we can try to look at a certain transformation group of its permutations---that is, it's symmetries. But it's not just simply 

  Once we know that field extensions are vector spaces, what constitutes linear maps? A first guess would be a ring homomorphism, but this may not be true. Therefore, we need some additional constraint. 

  \begin{definition}[Field Automorphisms]
    Let $K/F$ be a field extension. 
    \begin{enumerate}
      \item An \textbf{$F$-automorphism of $K$} is a ring homomorphism $\sigma: K \to K$ such that $\sigma(a) = a$ for all $a \in F$.\footnote{Note that we denote it $\sigma$ since it is analogous to a permutation---as we will see soon.}
      \item The set of all $F$-automorphisms of $K$ under composition is a subgroup of the autormorphism group of $K$, called the \textbf{$F$-automorphism group of $K$} and denoted $\Aut(K/F)$.\footnote{Sometimes---in bad taste---this is introduced as the Galois group, but technically we need some extra conditions. To minimize confusion, I will refer to this as the automorphism group.}
    \end{enumerate}
  \end{definition}
  \begin{proof}
    This is indeed a group under composition. The identity map $\iota \in \Aut(K/F)$. The composition is clearly closed. Now given that $\sigma \in \Aut(K/F)$, $\phi^{-1}$ is also an automorphism that is constant on $F$, so it is also in $\Aut(K/F)$. 
  \end{proof}

  \begin{lemma}[Linear Map]
    An $F$-automorphism of $K$ $\sigma: K \to K$ is a linear map of $F$-vector spaces. 
  \end{lemma}
  \begin{proof}
    
  \end{proof}

  Essentially, the $F$-automorphism group is a transformation subgroup of the ring automorphism group of $K$ that doesn't vary $F \subset K$. Let's provide a few examples to derive some of the automorphism groups. 

  \begin{example}[Computing $\Aut(\mathbb{Q}(\sqrt{2})/\mathbb{Q})$]
    Given the field extension $\mathbb{Q} \subset \mathbb{Q}(\sqrt{2})$, we have for any $a, b \in \mathbb{Q}$ and $\phi \in \Aut(\mathbb{Q}(\sqrt{2})/\mathbb{Q})$, we have 
    \begin{equation}
      \phi(a + b \sqrt{2}) = \phi(a) + \phi(b \sqrt{2}) = a + b \phi(\sqrt{2}) 
    \end{equation}
    So $\phi$ is completely determined by the value of $\phi(\sqrt{2})$. Now let $\phi(\sqrt{2}) = \alpha + \beta \sqrt{2}$ for some $\alpha, \beta \in \mathbb{Q}$. We have 
    \begin{align}
      2 = \phi(2) = \phi(\sqrt{2} \sqrt{2}) = \phi(\sqrt{2})^2 = (\alpha + \beta\sqrt{2})^2 = (\alpha^2 + 2 \beta^2) + 2 \alpha \beta \sqrt{2} 
    \end{align}
    Therefore $\alpha \beta = 0$ and $\alpha^2 + 2 \beta^2 = 2$. With further casework, we must have $\alpha = 0, \beta = \pm 1$. In conclusion, there are exactly two $\mathbb{Q}$-automorphisms of $\mathbb{Q}(\sqrt{2})$. 
    \begin{enumerate}
      \item The identity map $\iota(a + b \sqrt{2}) = a + b \sqrt{2}$, and 
      \item The conjugation map $\phi(a + b \sqrt{2}) = a - b \sqrt{2}$. 
    \end{enumerate}
  \end{example}

  \begin{example}[Computing $\Aut(\mathbb{Q}(2^{1/3})/\mathbb{Q})$]
    Given the field extension $\mathbb{Q} \subset \mathbb{Q}(\sqrt[3]{2})$, let us write $\xi = \sqrt[3]{2}$. Then since $\mathbb{Q}(\xi)$ is a vector space with basis $1, \xi, \xi^2$, we can write any element as $a + b \xi + c \xi^2$, and so by definition an element $\phi \in \Aut(\mathbb{Q}(\xi)/\mathbb{Q})$ must satisfy 
    \begin{equation}
      \phi(a + b \xi + c \xi^2) = a + b \phi(\xi) + c \phi(\xi)^2
    \end{equation}
    So the action is completely determined by the value of $\phi(\xi)$. Suppose $\phi(\xi) = \alpha + \beta \xi + \gamma \xi^2$ for some $\alpha, \beta, \gamma \in \mathbb{Q}$. Through some derivation we have 
    \begin{align}
      2 & = \phi(2) = \phi(\xi^3) = (\phi(\xi))^3 = (\alpha + \beta \xi + \gamma \xi^2)^3 \\ 
        & = (\alpha^3 + 2\beta^3 + 4\gamma^3) + 3(\alpha^2\beta + \beta^2\gamma + \gamma^2\alpha)\xi + 3(\alpha\beta^2 + \beta\gamma^2 + \gamma\alpha^2)\xi^2
    \end{align} 
    From the linear independence of $1, \xi, \xi^2$ we can see that 
    \begin{align}
      \alpha^3 + 2\beta^3 + 4\gamma^3 &= 2 \\
      \alpha^2\beta + \beta^2\gamma + \gamma^2\alpha &= 0 \\
      \alpha\beta^2 + \beta\gamma^2 + \gamma\alpha^2 &= 0
    \end{align}
    which turns out to have the only solution $\alpha = \gamma = 0, \beta = 1$. Therefore, the only $\mathbb{Q}$-automorphism of $\mathbb{Q}(\sqrt[3]{2})$ is the idnetity map. 
  \end{example} 

  Now let's look at how an $F$-automorphism acts on the roots of a polynomial. Let $f(x) = x^n + a_{n-1} x^{n-1} + \ldots + a_1 x + a_0 \in F[x]$, $F \subset K$ a field extension, and suppose $f(x)$ has roots $\alpha_1, \ldots, \alpha_m$ lying in $K$ (and perhaps other roots lying in a further extension). It turns out that any $F$-automorphism of $K$ must permute $\alpha_1, \ldots, \alpha_m$, so the roots stay ``within'' the polynomial. 

  \begin{lemma}[$F$-Automorphims Permute Roots]
    \label{thm:f_auto_permute}
    Let $K/F$ be a field extension and let $\sigma \in \Aut(K/F)$. 
    \begin{enumerate}
      \item If $\alpha \in K$ is algebraic over $F$---i.e. there exists $f(x) \in F[x]$ s.t. $f(\alpha) = 0$---then $\sigma (\alpha) \in K$ is also a root of $f(x)$.\footnote{However it may not be a permutation! The next point describes this a bit more precisely.}

      \item Let $f(x) \in F[x]$ have roots $\alpha_1, \ldots, \alpha_m \in K$ (with possibly other roots outside of $K$). Then there exists a well-defined group homomorphism 
      \begin{equation}
        \phi: \Aut(K/F) \to S_m, \qquad \phi(\sigma)(i) \coloneqq j \text{ s.t. } \sigma(\alpha_j) = \alpha_i
      \end{equation}

      \item If $K$ is the splitting field of $f(x)$, then $\phi$ is injective and so $\im(\phi)$ is a subgroup of $S_m$. 
    \end{enumerate}
  \end{lemma}
  \begin{proof}
    Listed. 
    \begin{enumerate}
      \item We know that $f(\alpha) = 0$. Now we can use the ring homomorphism properties to find
      \begin{align}
        \sigma\big( f(\alpha) \big) & = \sigma \big( a_n \alpha^n + a_{n-1} \alpha^{n-1} + \ldots + a_1 \alpha + a_0 \big) \\  
                                    & = \sigma(a_n) \sigma(\alpha)^n + \sigma(a_{n-1}) \sigma(\alpha)^{n-1} + \ldots + \sigma(a_1) \sigma(\alpha) + \sigma(a_0) \\ 
                                    & = a_n \sigma(\alpha)^n + a_{n-1} \sigma(\alpha)^{n-1} + \ldots + a_1 \sigma(\alpha) + a_0
      \end{align}
      which implies that $f(\sigma(\alpha)) = 0$. 

      \item Now assume that $S = \{\alpha_1, \ldots, \alpha_n\}$ is contained in $K$. We know that $\phi(\alpha_j) \in S$. The map $\phi \mapsto \phi(\alpha_j)$ is indeed a group homomorphism $\Aut(K/F) \to \mathrm{Perm}(S)$. 

      \item If $K$ is the splitting field, then $K = F[S]$, and so if $\phi \in \Aut(K/F)$ fixes all the $\alpha_j$'s, then it must fix all of $K$. 
    \end{enumerate}
  \end{proof} 

  \begin{example}[Recomputing $\Aut(\mathbb{Q}(2^{1/3})/\mathbb{Q})$]
    Therefore, we can get a much simpler solution of the Automorphism group of $\mathbb{Q}(\sqrt[3]{2})$ over $\mathbb{Q}$. Since $\xi = \sqrt[3]{2}$ is a root of the irreducible polynomial $x^3 - 2$, any $\phi \in \Aut(\mathbb{Q}[\xi]/\mathbb{Q})$ must carry $\xi$ to some other root of $x^3 - 2$. The other roots are in the complex plane and not in $\mathbb{Q}(\sqrt[3]{2})$, and so $\phi$ must carry $\xi \mapsto \xi$. Thus, $\phi$ must be the identity. 
  \end{example}

  \begin{example}[Computing $\Aut(\mathbb{Q}(1 + \sqrt{2}) / \mathbb{Q})$]
    Let $f(x) = x^2 - 2x - 1 \in \mathbb{Q}[x]$, which is the minimal polynomial of $1 + \sqrt{2} \in \mathbb{R}$. One root of $f(x)$ is $1 + \sqrt{2} \in \mathbb{Q}(\sqrt{2})$. But we know that there exists a $\phi \in \Aut(\mathbb{Q}(\sqrt{2})/\mathbb{Q})$ that conjugates, and so $1 - \sqrt{2}$ must also be a root. 
  \end{example} 

  \begin{example}[Computing $\Aut(\mathbb{C}/\mathbb{R})$]
    It follows that $\Aut(\mathbb{C}/\mathbb{R})$ is a group of order $2$ generated by complex conjugation. 
  \end{example} 

  So given any field extension $K/F$, we can associate it with a group $\Aut(K/F)$. One can also reverse this process and associate to each group of automorphisms a field extension. In fact, the subgroup property is not ecessary, and we can just talk about subsets. 

  \begin{theorem}[]
    Let $K$ be a field, and let $H$ be a subset of the group of automorphisms $\Aut(K)$. Then the collection $F$ of element of $K$ fixed by all the elements of $H$ is a subfield of $K$. 
  \end{theorem}
  \begin{proof}
    Let $h \in H$ and let $a, b \in F$. Then by definition $h(a) = a, h(b) = b$, and so $h(a + b) = h(a) + h(b)$, $h(ab) = h(a) h(b)$, and $h(a)^{-1} = a^{-1}$. So $F$ is closed, hence a subfield of $K$. 
  \end{proof}

  \begin{theorem}[Inclusion Reversing Association between Groups and Fields]
    Therefore, the association of groups to fields and fields to group are inclusion reversing. 
    \begin{enumerate}
      \item If $F_1 \subset F_2 \subset K$ are two subfields of $K$, then $\Aut(K/F_2) \subset \Aut(K/F_1)$. 
      \item If $H_1 \subset H_2 \subset \Aut(K)$ are to subgroups of automorphisms of $K$ with fixed fields $F_1$ and $F_2$, respectively, then $F_2 \subset F_1$. 
    \end{enumerate}
  \end{theorem}
  \begin{proof}
    
  \end{proof}

  Given a subfield $F$ of $K$, the associated group is the collection of $F$-automorphisms of $K$. Given a group of $F$-automorphisms of $K$, the associated extension is defined by taqking $F$ to be the fixed field of the automorphisms. 
  \begin{enumerate}
    \item Given the subfield $\mathbb{Q} \subset \mathbb{Q}(\sqrt{2})$, the automorphism group is $\{1, \sigma\}$, and given the group, the fixed field is $\mathbb{Q}$. 
      \begin{equation}
        \mathbb{Q}(\sqrt{2}) \supset \mathbb{Q} \longrightarrow \{1, \sigma\} \longrightarrow \mathbb{Q}
      \end{equation}
      Therefore, there is a duality between the subfield $\mathbb{Q}$ and the group $\{1, \sigma\}$. 

    \item Given the subfield $\mathbb{Q} \subset \mathbb{Q}(\sqrt[3]{2})$, we obtain the trivial group $\{e\}$, which induces the fixed field $\mathbb{Q}(\sqrt[3]{2})$. 
      \begin{equation}
        \mathbb{Q}(\sqrt[3]{2}) \supset \mathbb{Q} \longrightarrow \{e\} \longrightarrow \mathbb{Q}(\sqrt[3]{2})
      \end{equation}
      In here, we lose the duality. 
  \end{enumerate}

  Let's investigate why we lose the duality for the second example. The trivial group $\{e\}$ does not have enough automorphisms to force the fixed field to be $\mathbb{Q}$ rather than the full $\mathbb{Q}(\sqrt[3]{2})$. This is because the other roots of the minimal polynomial $x^3 - 2 \in \mathbb{Q}[x]$---which can be the images of $\sqrt[3]{2}$ under an automorphism---lie outside of $\mathbb{Q}(\sqrt[3]{2})$. We now make precise the notion of fields with ``enough'' automorphisms. Essentially we want to count the number of elements in the automorphism group. How do we do this? Well we take a field extension $K/F$, take the identity automorphism of $F$, and see how many ways we can extend it into an automorphism of $K$. 

  Recall \ref{thm:extending_field_iso}, which states that any isomorphism $\phi: F \to F^\prime$ can be extended to an isomorphism $\bar{\phi}: K \to K^\prime$ for $f^\prime (x) = \phi(f(x)) \in F^\prime[x]$. We build on this theorem. 

  \begin{theorem}[Bound on Number of Field Isomorphism Extensions]
    Let $F \subset K, F^\prime \subset K^\prime$ be field extensions with an isomorphism $\phi:F \to F^\prime$. Then, the numbers of extensions $\bar{\phi}: K \to K^\prime$ is bounded by $[K:F]$. 
  \end{theorem}
  \begin{proof}
    We now show by induction on $[K : F]$ that the number of such extensions is at most $[K : F]$, with equality if $f(x)$ is separable over $F$. If $[K : F] = 1$ then $K = F$, $K' = F'$, $\sigma = \varphi$ and the number of extensions is 1. If $[K : F] > 1$ then $f(x)$ has at least one irreducible factor $p(x)$ of degree $> 1$ with corresponding irreducible factor $p'(x)$ of $f'(x)$. Let $\alpha$ be a fixed root of $p(x)$. If $\sigma$ is any extension of $\varphi$ to $K$, then $\sigma$ restricted to the subfield $F(\alpha)$ of $K$ is an isomorphism $\tau$ of $F(\alpha)$ with some subfield of $K'$. The isomorphism $\tau$ is completely determined by its action on $\alpha$, i.e., by $\tau\alpha$, since $\alpha$ generates $F(\alpha)$ over $F$. Just as in Proposition 2, we see that $\tau\alpha$ must be some root $\beta$ of $p'(x)$. Then we have a diagram

    \begin{figure}[H]
      \centering 
      \begin{tikzcd}
        K \arrow[r, "\sigma"] \arrow[d] & K' \arrow[d] \\
        F(\alpha) \arrow[r, "\tau"] \arrow[d] & F'(\beta) \arrow[d] \\
        F \arrow[r, "\varphi"] & F'
      \end{tikzcd}
    \end{figure}

    Conversely, for any $\beta$ a root of $p^\prime(x)$ there are extensions $\tau$ and $\sigma$ giving such a diagram (this is Theorem 13.8 and Theorem 13.27). Hence to count the number of extensions $\sigma$ we need only count the possible number of these diagrams.

    The number of extensions of $\varphi$ to an isomorphism $\tau$ is equal to the number of distinct roots $\beta$ of $p'(x)$. Since the degree of $p(x)$ and $p'(x)$ are both equal to $[F(\alpha) : F]$, we see that the number of extensions of $\varphi$ to a $\tau$ is at most $[F(\alpha) : F]$, with equality if the roots of $p(x)$ are distinct.

    Since $K$ is also the splitting field of $f(x)$ over $F(\alpha)$, $K'$ is the splitting field of $f'(x)$ over $F'(\beta)$, and $[K : F(\alpha)] < [K : F]$, we may apply our induction hypothesis to these field extensions. By induction, the number of extensions of $\tau$ to $\sigma$ is $\leq [K : F(\alpha)]$, with equality if $f(x)$ has distinct roots.

    From $[K : F] = [K : F(\alpha)][F(\alpha) : F]$ it follows that the number of extensions of $\varphi$ to $\sigma$ is $\leq [K : F]$. We have equality if $p(x)$ and $f(x)$ have distinct roots, which is equivalent to $f(x)$ having distinct roots since $p(x)$ is a factor of $f(x)$, completing the proof by induction. 
  \end{proof}

  \begin{corollary}[Bound on Order of Automorphism Group]
    Let $K$ be the splitting field of a polynomial $f(x) \in F[x]$. Then, 
    \begin{equation}
      | \Aut(K/F) | \leq [E:F]
    \end{equation}
    with equality if $f(x)$ is separable over $F$. 
  \end{corollary}

\subsection{Galois Extensions and Galois Groups}

  If we reach this bound, then this is equivalent to saying that the automorphism group $H$ has ``enough'' elements to recover $F$ (given field extension $K/F$) as the fixed field of $H$. 

  \begin{definition}[Galois Field Extension]
    A field extension $F \subset K$ is \textbf{Galois} if $|\Aut(K/F)| = [K:F]$---or equivalently, the degree of the minimal polynomial $f(x) \in F[x]$ that splits in $K$. 
  \end{definition}

  \begin{definition}[Galois Group of Galois Field Extension]
    Given a Galois field extension $K/F$, the automorphism group $\Aut(K/F)$ is called the \textbf{Galois Group}, denoted $\mathcal{G}(K/F)$. 
  \end{definition}

  Therefore, we can think of an automorphism group as Galois if we have reached the maximal number of automorphisms. 

  \begin{example}[Simple Examples]
    We review the derived Galois groups above and see if the field extensions are Galois. 
    \begin{enumerate}
      \item $\mathbb{Q}(\sqrt{2})$ is a Galois extension of $\mathbb{Q}$. Since we know that $G = \mathcal{G}(\mathbb{Q}(\sqrt{2})/\mathbb{Q})$ has order $2$, which is the same as the dimension of $\mathbb{Q}(\sqrt{2})$, i.e. the degree of the minimal polynomial $x^2 - 2 \in \mathbb{Q}[x]$. 

      \item $\mathbb{Q}(\sqrt[3]{2})$ is not a Galois extension of $\mathbb{Q}$ since $|\mathcal{G}(\mathbb{Q}[\sqrt[3]{2}]/\mathbb{Q})| = 1$ but $[\mathbb{Q}[\sqrt[3]{2}] : \mathbb{Q}] = 3$. 

      \item Let $K = \mathbb{Q}[\sqrt[3]{2}, i \sqrt{3}]$ is the splitting field of $f(x) = x^3 - 2$. Then $[K:\mathbb{Q}] = 6$. $\mathcal{G}(K/\mathbb{Q}) \simeq S_3$ has order $6$ since we showed 6 $\mathbb{Q}$-automorphisms of $K$, but now we know that there can be no more. 

      \item Let $\alpha = \sqrt[7]{2}$. $\mathbb{Q} \subset \mathbb{Q}[\alpha]$ is not a Galois extension. $x^7 - 2$ is a polynomial with root $\alpha$, and by Eisenstein $x^7 - 2$ is irreducible. So $f(x) = x^7 - 2$ is the minimal polynomial of $\alpha$. This means that the number of $\mathbb{Q}$-embeddings $\mathbb{Q}[\alpha] \to \mathbb{Q}[\alpha]$ is in bijection with the number of roots of $f(x)$ in $\mathbb{Q}[\alpha]$. All $7$ roots of $f(x)$ are $\sqrt[7]{2} e^{2\pi i j/7}$ for $j = 0, \ldots, 6$, which has one real root. So there is $1$ $\mathbb{Q}$-embedding $\mathbb{Q}[\alpha] \to \mathbb{Q}[\alpha]$. But $[\mathbb{Q}[\sqrt[7]{2}] : \mathbb{Q}] = \deg{f(x)} = 7$. Since $\mathbb{Q}[\alpha] \simeq \mathbb{Q}[x]/{\langle x^7 - 2 \rangle}$, which are polynomials of degree at most $6$. 
    \end{enumerate}
  \end{example}

  These examples suggest that $K$ will be a Galois extension of $F$ whenever $K$ is a splitting field of some polynomial $f(x) \in F[x]$. To establish this, we must produce $[K:F]$ $F$-automorphisms of $K$ under these circumstances. 

  \begin{theorem}[Splitting Fields of Separable Polynomials are Galois]
    \label{thm:splitting_separable}
    If $K$ is the splitting field over $F$ of a separable polynomial $f(x)$ then $K/F$ is a Galois extension. 
  \end{theorem}
  \begin{proof}
    Since $K$ is the splitting field of $f(x)$ over $F$, we have $F[r_1, \ldots, r_m]$ where the $r_i$ are the distinct roots of $f(x)$. We show by induction that there are $[F[r_1, \ldots, r_j]:F]$ $F$-embeddings of $F[r_1, \ldots, r_j] \to K$. For $j = 1$, $r_1$ is a root of some irreducible factor $f_1 (x)$ of $f(x)$. 
    \begin{equation}
      F[r_1] \simeq \frac{F[x]}{\langle f_1 (x) \rangle}
    \end{equation} 
    and the set of $F$-embeddings $F[r_1] \to K$ is in bijection with the set of roots $\alpha \in K$ of $f_1 (x)$. By hypothesis , $f(x)$ has no repeated roots, which implies that the number of $F$-embeddings $F[r_1] \to K$ is $\deg{f_1 (x)} = [F[r_1]: F]$ which gives the base case. For the inductive step, we know there are 
    \begin{equation}
      [F[r_1, \ldots, r_{j-1}]: F] 
    \end{equation}
    $F$-embeddings $F[r_1, \ldots, r_{j-1}] \xrightarrow{\phi} K$. For each, we will show that there are exactly $[F[r_1, \ldots, r_j]: F[r_1, \ldots, r_{j-1}]]$ extensions of $\phi$ which completes the proof. Because 
    \begin{align}
      F[r_1, \ldots, r_j]: F] & = F[r_1, \ldots, r_j]: F[r_1, \ldots, r_{j-1}]] F[r_1, \ldots, r_{j-1}]: F]
    \end{align} 
    Let $g(x)$ be the minimal polynomial of $r_j$ over $F[r_1, \ldots, r_{j-1}]$. Since $g(r_j) = 0$, $g$ divides one of the irreducible factors of $f(x)$ in $F[r_1, \ldots, r_{j-1}] [x]$ which implies it has no repeated roots. Then 
    \begin{equation}
      F[r_1, \ldots, r_j] = \frac{E[x]}{\langle g(x)\rangle} 
    \end{equation}
    The number of $E$-embeddings is equal to the number of roots in in $g$ which is $\deg{g} = F[r_1, \ldots, r_j: F]$.
  \end{proof} 

  It turns out that the converse of the theorem is also true, as we will see later. This gives us a complete characterization of Galois extensions. The following is immediate. 

  \begin{corollary}[Splitting Fields of $\mathbb{Q}$ are Automatically Galois]
    If $K$ is a splitting field of $f(x) \in \mathbb{Q}[x]$, then $K/\mathbb{Q}$ is Galois. 
  \end{corollary}
  \begin{proof}
    All irreducible factors of $f(x) \in \mathbb{Q}[x]$ has no repeated roots and hence is separable from \ref{thm:sep_char}. Therefore, from \ref{thm:splitting_separable}, since $K$ is the splitting field, $K/\mathbb{Q}$ is Galois. 
  \end{proof}

  What about the converse? There are two steps to proving that for any Galois field extension $F \subset K$, there exists a polynomial $f(x) \in F[x]$ that splits in $K$. 

  \begin{lemma}[Irreducible Polynomial with Root in Galois Extension Splits] 
    % Shifrin 6.9
    \label{thm:6.9}
    Let $F \subset K$ be a Galois extension of fields. Let $f(x) \in F[x]$ be an irreducible polynomial with a root $\alpha \in K$. Then $f(x)$ splits in $K$, i.e. all other roots must be in $K$.  
  \end{lemma} 
  \begin{proof}
    Let $\sigma_j$ for $j = 1, \ldots, n$ be the elements of $\mathcal{G}(K/F)$, and set $\alpha_j = \sigma_j (\alpha)$. Then define 
    \begin{equation}
      h(x) = \prod_{i=1}^n (x - \alpha_i)  \in K[x]
    \end{equation}
    We claim that the coefficients of $h(x)$ are fixed by any element of $\mathcal{G}(K/F)$. This is because by \ref{thm:f_auto_permute}, the coefficients obtained by taking the binomial expansion stay invariant. Therefore, all coefficients of $h(x)$ are in $F$, and so $h(x) \in F[x]$. Since $g(x)$ and $h(x)$ have a common root $\alpha \in K$, and since $g(x)$ is irreducible in $F[x]$, we have $g(x) \mid h(x)$. Therefore $g(x)$ as a factor of $h(x)$ which splits in $K$, also splits in $K[x]$. 
  \end{proof}

  \begin{theorem}[Every Galois Extension has a Splitting Field]
    Let $F \subset K$ be a Galois extension. Then $K$ is the splitting field of a polynomial $f(x) \in F[x]$. 
  \end{theorem}
  \begin{proof}
    Let $\alpha_1, \ldots, \alpha_k$ be a basis for $K$ over $F$, and for each $j = 1, \ldots, k$, let $g_j (x) \in F[x]$ be an irreducible polynomial with root $\alpha_j$. Then by \ref{thm:6.9}, $K$ is the splitting field $g_j(x)$ splits in $K$, which implies that 
    \begin{equation}
      f(x) = g_1 (x) \ldots g_k(x) \in F[x]
    \end{equation}
    also splits in $K$, and it does in no smaller fields since any splitting field must contain all the $\alpha_j$'s. 
  \end{proof}

  When computing Galois groups, it only makes sense to talk about it with respec to Galois field extensions. Therefore, if we are given a field extension, we must prove that it is a Galois extension, then find the order of its Galois group, then compute the elements using the theorems we have above. But since we have established equivalent conditions for a Galois group to exist through polynomials, we can talk about a Galois group w.r.t. a separable polynomial. 

  \begin{definition}[Galois Group of Separable Polynomial]
    Given a separable polynomial $f(x) \in F[x]$, its \textbf{Galois group} is defined to be the Galois group of the splitting field of $f(x)$ over $F$. 
  \end{definition}

  If we are given a polynomial $f(x) \in F[x]$, when we must find that no irreducible factor of it has repeated roots, and then we find the splitting field $F/K$ to compute the Galois group. Let's do some practice. 

  \begin{example}[Computing Galois Groups]
    Now let's compute the Galois group of $\mathbb{Q}(\sqrt{2}, \sqrt{3})$ over $\mathbb{Q}$. 
    \begin{enumerate}
      \item $\mathbb{Q}(\sqrt{2}, \sqrt{3})$ is the splitting field of $f(x) = (x^2 - 2)(x^2 - 3)$ over $\mathbb{Q}$. Then the field extension $\mathbb{Q} \subset \mathbb{Q}(\sqrt{2}, \sqrt{3})$ is Galois. 

      \item Therefore the Galois group of this extension exists. Since $\sqrt{2}, \sqrt{3}$ are not linearly dependent, we have $[\mathbb{Q}(\sqrt{2}, \sqrt{3}):\mathbb{Q}] = 4$. Thus the Galois group has order $4$. Now it remains to compute the 4 elements. 

      \item Since $\mathbb{Q}(\sqrt{2}, \sqrt{3})$ is a $\mathbb{Q}$-vector space of dimension 4, we know that every element is of the form $a + b \sqrt{2} + c \sqrt{3} + d \sqrt{6}$. Since 
      \begin{equation}
        \phi(a + b \sqrt{2} + c \sqrt{3} + d \sqrt{6}) = a + b \phi(\sqrt{2}) + c \phi(\sqrt{3}) + d \phi(\sqrt{2}) \phi(\sqrt{3}), 
      \end{equation}
      every $\mathbb{Q}$-automorphism is determined by the values $\phi(2), \phi(3)$. 

    \item From \ref{thm:f_auto_permute}, we know that elements $\phi \in \mathcal{G}(\mathbb{Q}(\sqrt{2}, \sqrt{3}) / \mathbb{Q})$ permute the roots of $f(x)$. We immediately see that if $\phi(\sqrt{2}) = \sqrt{3}$
      \begin{equation}
        2 = \phi(2) = \phi(\sqrt{2})^2 = \sqrt{3}^2 = 3
      \end{equation}
      which is a contradiction. The same logic follows for $\sqrt{2} \mapsto \pm \sqrt{3}, \sqrt{3} \mapsto \pm \sqrt{2}$. So it must be the case that $\phi(\sqrt{2}) = \pm \sqrt{2}$ and $\phi(\sqrt{3}) = \pm \sqrt{3}$. 
    \end{enumerate}
    Therefore, we are able to deduce that the 4 automorphisms are. 
    \begin{align}
      \phi_1 (a + b \sqrt{2} + c \sqrt{3} + d \sqrt{6}) & = a + b \sqrt{2} + c \sqrt{3} + d \sqrt{6} \\
      \phi_1 (a + b \sqrt{2} + c \sqrt{3} + d \sqrt{6}) & = a - b \sqrt{2} + c \sqrt{3} - d \sqrt{6} \\
      \phi_1 (a + b \sqrt{2} + c \sqrt{3} + d \sqrt{6}) & = a + b \sqrt{2} - c \sqrt{3} - d \sqrt{6} \\
      \phi_1 (a + b \sqrt{2} + c \sqrt{3} + d \sqrt{6}) & = a - b \sqrt{2} - c \sqrt{3} + d \sqrt{6}
    \end{align}
    which is isormophic to the Klein 4-group. 
  \end{example}

  \begin{example}[Computing Galois Groups of 3 Adjoining Elements]
    For $\mathbb{Q}(\sqrt{2}, \sqrt{3}, \sqrt{5})$, follow the same method, as it's the splitting field of $(x^2 - 2)(x^2 - 3)(x^2 - 5)$. Consider the automorphisms:
    \begin{align}
      \sqrt{2} &\mapsto -\sqrt{2} \\
      \sqrt{3} &\mapsto \sqrt{3} \\
      \sqrt{5} &\mapsto \sqrt{5}
    \end{align}
    and
    \begin{align}
      \sqrt{2} &\mapsto \sqrt{2} \\
      \sqrt{3} &\mapsto -\sqrt{3} \\
      \sqrt{5} &\mapsto \sqrt{5}
    \end{align}
    and
    \begin{align}
      \sqrt{2} &\mapsto \sqrt{2} \\
      \sqrt{3} &\mapsto \sqrt{3} \\
      \sqrt{5} &\mapsto -\sqrt{5}
    \end{align}
    which generates the group. We can see that this is $\mathbb{Z}_2^3$. 
  \end{example}

\subsection{Fundamental Theorem of Galois Theory}

  Notice from our previous example that there seems to be a similarity between the known subgroups of the Klein-4 groups and the known subfields of $\mathbb{Q}(\sqrt{2}, \sqrt{3})$. In fact this bijection is not a coincidence, and this is precisely the statement of the fundamental theorem of Galois theory. 

  \begin{figure}[H]
    \centering
    \begin{subfigure}[b]{0.48\textwidth}
      \centering
      \begin{tikzcd}[row sep=large, column sep=large]
        & \{1\} \arrow[dl, "2"'] \arrow[d, "2"] \arrow[dr, "2"] & \\
        \{1, \tau\} \arrow[dr, "2"'] & \{1, \sigma\tau\} \arrow[d, "2"] & \{1, \sigma\} \arrow[dl, "2"] \\
        & \{1, \sigma, \tau, \sigma\tau\} &
      \end{tikzcd}
      \caption{Group structure diagram}
      \label{fig:group_structure}
    \end{subfigure}
    \hfill 
    \begin{subfigure}[b]{0.48\textwidth}
      \centering
      \begin{tikzcd}[row sep=large, column sep=large]
        & \mathbb{Q}(\sqrt{2}, \sqrt{3}) \arrow[dl, "2"'] \arrow[d, "2"] \arrow[dr, "2"] & \\
        \mathbb{Q}(\sqrt{2}) \arrow[dr, "2"'] & \mathbb{Q}(\sqrt{6}) \arrow[d, "2"] & \mathbb{Q}(\sqrt{3}) \arrow[dl, "2"] \\
        & \mathbb{Q} &
      \end{tikzcd}
      \caption{Field extension diagram}
      \label{fig:field_extension}
    \end{subfigure}
    \caption{Lattice diagrams showing group structure and corresponding field extensions}
    \label{fig:lattice_diagrams}
  \end{figure}

  Also take a look at this for the polynomial $x^3 - 2 \in \mathbb{Q}[x]$. 

  \begin{figure}[H]
    \centering
    \begin{subfigure}[b]{0.4\textwidth}
      \centering
      \begin{tikzcd}[row sep=small, column sep=small]
        & \{e\} \arrow[ddl, "3"'] \arrow[drr, ""] \arrow[dr, ""] \arrow[drrr, "2"] & & & \\ 
        & & \langle f \rangle \arrow[ddl, ""] & \langle rf \rangle \arrow[ddll, ""] & \langle r^2 f \rangle \arrow[ddlll, "3"] \\ 
        \langle r \rangle \arrow[dr, "2"'] & & & & \\
        & \mathbb{Q} & & & 
      \end{tikzcd} 
      \caption{}
    \end{subfigure}
    \hfill 
    \begin{subfigure}[b]{0.58\textwidth}
      \centering
      \begin{tikzcd}[row sep=small, column sep=small]
        & \mathbb{Q}(\sqrt[3]{2}, \zeta) \arrow[ddl, "3"'] \arrow[drr, ""] \arrow[dr, ""] \arrow[drrr, "2"] & & & \\ 
        & & \mathbb{Q}(\sqrt[3]{2}) \arrow[ddl, ""] & \mathbb{Q}(\zeta \sqrt[3]{2}) \arrow[ddll, ""] & \mathbb{Q}(\zeta^2 \sqrt[3]{2}) \arrow[ddlll, "3"] \\ 
        \mathbb{Q}(\zeta) \arrow[dr, "2"'] & & & & \\
        & \mathbb{Q} & & & 
      \end{tikzcd} 
      \caption{}
    \end{subfigure}
    \caption{}
  \end{figure}

  \begin{theorem}[Fundamental Theorem of Galois Theory]
    Let $K/F$ be a Galois extension and let $G = \mathcal{G}(K/F)$. Then there is a bijection 
    \begin{equation}
      \{\text{Fields } E \mid F \subset E \subset K \} \simeq \{\text{Subgroups } H \mid \{e\} \subset H \subset G \}
    \end{equation}
    where given automorphism group $H$, $E$ is the fixed field of all elements in $H$; and given field $E$, $H$ is the corresponding group formed by the elements of $G$ fixing $E$. 
    Under this correspondence, we have 
    \begin{enumerate}
      \item If $E_1 \sim H_1, E_2 \sim H_2$, then $E_1 \subset E_2$ iff $H_2 \subset H_1$. As a consequence $F \simeq G$ and $K \simeq \{e\}$. 

      \item $[K:E] = |H|$ and $[E:F] = |G:H|$.\footnote{Note the left is the degree of $E$ over $\mathbb{Q}$ while the right is the number of cosets of $H$ in $G$!} 
      \item $K/E$ is always Galois, with Galois group 
        \begin{equation}
          \mathcal{G}(K/E) = H
        \end{equation}

      \item $E$ is Galois over $F$ iff $H$ is a normal subgroup of $G$. If this is the case, then the Galois group is isomorphic to the quotient group 
        \begin{equation}
          \mathcal{G}(E/F) \cong G/H
        \end{equation}

      \item If $E_1, E_2$ correspond to $H_1, H_2$ respectively, then $E_1 \cap E_2 \sim \langle H_1, H_2 \rangle$ and the composite field $E_1 E_2 \sim H_1 \cap H_2$. Hence the lattice of subfields of $K$ containing $F$ and the lattice of subgroups are ``dual.'' 
    \end{enumerate}
  \end{theorem}
  \begin{proof}
    
  \end{proof}

  \begin{example}[]
    Since all the subgroups of an abelian group are normal, all the subfields of $\mathbb{Q}(\sqrt{2}, \sqrt{3})$ are Galois extensions of $\mathbb{Q}$. 
  \end{example}

\subsection{Galois Groups of Polynomials} 

  \begin{definition}[Elementary Symmetric Functions] 
    Let $x_1, x_2, \ldots, x_n$ be indeterminates. The \textbf{elementary symmetric functions} $s_1, s_2, \ldots, s_n$ are defined as 
    \begin{align}
      s_1 & = x_1 + x_2 + \ldots + x_n \\ 
      s_2 & = x_1 x_2 + x_1 x_3 + \ldots + x_i x_j + x_{n-1} x_n \\ 
      \vdots & = \vdots \\
      s_n & = x_1 x_2 \ldots x_n
    \end{align}
  \end{definition}

  \begin{definition}[General Polynomial of Degree $n$]
    The \textbf{general polynomial of degree $n$} is the polynomial 
    \begin{equation}
      f(x) = \prod_{i=1}^n (x - x_i)
    \end{equation}
    whose roots are the indeterminates $x_1, \ldots, x_n$.  
  \end{definition}

  \begin{theorem}
    The fixed field of the symmetric group $S_n$ acting on the field of rational functions in $n$ variables $F(x_1, \ldots, x_n)$ is the field of rational functions in the eleemntary symmetric functions $F(s_1, \ldots, s_n)$. 
  \end{theorem}

  \begin{definition}[Symmetric]
    A rational function $f(x_1, \ldots, x_n)$ is \textbf{symmetric} if it is not changed by any permutation of the variables $x_1, \ldots x_n$. 
  \end{definition}

  \begin{theorem}[Fundamental Theorem on Symmetric Functions]
    Any symmetric functions in the variables $x_1, x_2, \ldots, x_n$ is a rational function in the elementary symmetric functions $s_1, s_2, \ldots, s_n$. 
  \end{theorem}

  \begin{theorem}
    The general polynomial 
    \begin{equation}
      x^n - s_1 x^{n-1} + s_2 x^ {n-2} + \ldots + (-1)^n s_n 
    \end{equation}
    over the field $F(s_1, \ldots, s_n)$ is separable with Galois group $S_n$. 
  \end{theorem}

  The well known discriminant of a quadratic equation 
  \begin{equation}
    f(x) = ax^2 + bx + c
  \end{equation}
  is known in the form $\nabla = b^2 - 4ac$. However, we will present it in a slightly different manner. 

  \begin{definition}
    The \textbf{discriminant} $D(\varphi)$ of a quadratic polynomial
    \begin{equation}
      \varphi = a_0 x^2 + a_1 x + a_2 \in \mathbb{C}[x]
    \end{equation}
    with $c_1, c_2 \in \mathbb{C}$ as its roots is defined
    \begin{equation}
      D(\varphi) = a_1^2 - 4 a_0 a_2 = a_0^2 \bigg( \Big(\frac{a_1}{a_0} \Big)^2 - \frac{4 a_2}{a_0} \bigg) = a_0^2 \big( (c_1 + c_2)^2 - 4 c_1 c_2 \big) = a_0^2 (c_1 - c_2)^2
    \end{equation}
    Clearly, the value of $D(\varphi)$ can tell us three things
    \begin{enumerate}
      \item $c_1, c_2 \in \mathbb{R}, c_1 \neq c_2$. Then $c_1 - c_2$ is a nonzero real number and $D(\varphi) > 0$. 
      \item $c_1 = c_2 \in \mathbb{R}$. Then $c_1 - c_2 = 0$ and $D(\varphi) = 0$. 
      \item $c_1, c_2 \in \mathbb{C}, c_1 = \bar{c}_2$. Then, $c_1 - c_2$ is a nonzero strictly imaginary number and $D(\varphi) < 0$. 
    \end{enumerate}
  \end{definition}

  \begin{definition}
    We can generalize this notion of the discriminant to arbitrary polynomials
    \begin{equation}
      \varphi = a_0 x^n + a_1 x^{n-1} + ... + a_{n-1} x + a_n \in \mathbb{F}[x], \; a_0 \neq 0
    \end{equation}
    The discriminant $D(\varphi)$ of the polynomial above is defined
    \begin{equation}
      D(\varphi) \equiv a_0^{2n-2} \prod_{i>j} (c_i - c_j)^2
    \end{equation}
    The $a_0$ term isn't very important in this formula, since it does not affect whether $D(\varphi)$ is positive, negative, or zero. 
  \end{definition}

  \begin{definition}
    A polynomial 
    \begin{equation}
      \varphi = a_0 x^n + a_1 x^{n-1} + ... + a_{n-1} x + a_n \in \mathbb{F}[x], \; a_0 \neq 0
    \end{equation}
    where $a_1 = 0$ is called \textbf{depressed}. A depressed cubic polynomial is of form
    \begin{equation}
      \varphi = x^3 + p x + q
    \end{equation}
  \end{definition}

  \begin{theorem}
    Every monic (leading coefficeint $=1$) polynomial (and non-monic ones) 
    \begin{equation}
      \varphi = x^n + a_1 x^{n-1} + ... + a_{n-1} x + a_n \in \mathbb{F}[x], \; a_0 \neq 0
    \end{equation}
    can be turned into a depressed polynomial with the change of variable
    \begin{equation}
      x = y - \frac{a_1}{n}
    \end{equation}
    to get the polynomial 
    \begin{equation}
      \psi = y^n + b_2 y^{n-2} + ... + b_{n-1} y + b_n
    \end{equation}
  \end{theorem}

  \begin{lemma}
    A cubic polynomial 
    \begin{equation}
      \varphi = a_0 x^3 + a_1 x^2 + a_2 x + a_3 \in \mathbb{R}[x]
    \end{equation}
    with roots $c_1, c_2, c_3 \in \mathbb{C}$ has discriminant
    \begin{equation}
      D(\varphi) \equiv a_0^4 (c_1 - c_2)^2 (c_1 - c_3)^2 (c_2 - c_3)^2
    \end{equation}
    With a bit of evaluation, it can also be expressed in terms of its coefficients as
    \begin{equation}
      D(\varphi) = a_1^2 a_2^2 - 4a_1^3 a_3 - 4a_0 a_2^3 + 18 a_0 a_1 a_2 a_3 - 27 a_0^2 a_3^2
    \end{equation}
    Again, three possibilities can occur (up to reordering of its roots). 
    \begin{enumerate}
        \item $c_1, c_2, c_3$ are distinct real numbers. Then $D(\varphi) > 0$. 
        \item $c_1, c_2, c_3 \in \mathbb{R}, c_1 = c_2$. Then $D(\varphi) = 0$. 
        \item $c_1 \in \mathbb{R}, c_2 = \bar{c}_3 \not\in \mathbb{R}$. Then $D(\varphi) < 0$. 
    \end{enumerate}
    Furthermore, the cubic formula used to find the roots of the polynomial is 
    \begin{equation}
      c_{1, 2, 3} = \sqrt[3]{-\frac{q}{2} + \sqrt{\frac{p^3}{27} + \frac{q^2}{4}}} + \sqrt[3]{-\frac{q}{2} - \sqrt{\frac{p^3}{27} + \frac{q^2}{4}}}
    \end{equation}
    known as \textbf{Cardano's formula}, after the mathematician Gerolamo Cardano. 
  \end{lemma}

\subsection{Solvable and Radical Extensions}


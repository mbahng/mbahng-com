\section{Subgroups} 

  We have seen a few examples of subgroups, but we will heavily elaborate on here. We know that given a set, we can define an equivalence relation on it to get a quotient set. Now if we have a group, defining any such equivalence relation may not be compatible with the group structure. Therefore, it would be nice to have some principles in which we can construct such compatible equivalence classes, i.e. through a \textbf{congruence relation} that preserves the operations. 

  We introduce some standard notation. 

  \begin{definition}[Subgroup of Integer Multiples]
    The set $k \mathbb{Z}$ is the set of all integer multiples of $k$. This is a group under addition. 
  \end{definition}

\subsection{Cosets}

  Fortunately, we can do such a thing by taking a subgroup $H \subset G$ and ``shifting'' it to form the cosets of $G$, which are the equivalence classes. 
  
  \begin{definition}[Coset]
    Given a group $G$, $a \in G$, and subgroup $H$, 
    \begin{enumerate}
      \item A \textbf{left coset} is $a H \coloneqq \{a h \mid h \in H \}$. 
      \item A \textbf{right coset} is $H a \coloneqq \{h a \mid h \in H \}$. 
      \item When $G$ is abelian, the \textbf{coset} is denoted $a + H$. 
    \end{enumerate}
    With this, we can take arbitrary elements $a, b \in G$ and determine if they are in the same coset as such. Since $a \in aH$, $b \in aH$ iff $b = ah$ for some $h \in H$. Therefore, we have the equivalence relation. 
    \begin{equation}
      a \equiv b \pmod{H} \iff a = b h \text{ for some } h \in H
    \end{equation}
  \end{definition}
  \begin{proof}
    We show that this indeed forms an equivalence class. 
    \begin{enumerate}
      \item \textit{Reflexive}. $a \equiv a \pmod{H}$ since $e \in H \implies a = a e$. 
      \item \textit{Symmetric}. Let $a \equiv b \pmod{H}$. Then $a = bh$ for some $h \in H$, but since $H$ is a group, $h^{-1} \in H \implies a h^{-1} = b \implies b \equiv a \pmod{H}$. 
      \item \textit{Transitive}. Let $a \equiv b \pmod{H}$ and $b \equiv c \pmod{H}$. Then $a = bh$ and $b = ch^\prime$ for some $h, h^\prime \in H$. But then 
      \begin{equation}
        a = bh = (ch^\prime) h = c(h^\prime h)
      \end{equation}
      where $h^\prime h \in H$ due to closure. 
    \end{enumerate}
  \end{proof} 

  Note that a coset is \textit{not} a subgroup. It is only the case that $eH = H$ is a subgroup, but for $a \neq e$, $aH$ does not even contain the identity. We should think of a coset as a \textit{translation} of the subgroup $H$. 

  \begin{example}[Familiar Cosets]
    Here are some examples. Note that all it takes is to find \textit{some} subgroup, and the cosets will naturally pop up. 
    \begin{enumerate}
      \item Let $H = 2 \mathbb{Z} \subset (\mathbb{Z}, +)$ be the even integers. Then $0 + H$ and $1 + H$ are the even and odd integers, respectively. 
      \item Let $H = \{e, f\} \subset \Dih(3)$. Then 
      \begin{equation}
        H = \{e, f\}, rH = \{r, rf\}, r^2 H = \{r^2, r^2 f\} 
      \end{equation}
      are the cosets. 
    \end{enumerate}
  \end{example}

  With this partitioning scheme in mind, the following theorem on the order of such groups becomes very intuitive, and has a lot of consequences. 

  \begin{theorem}[Lagrange's Theorem]
    Let $G$ be a finite group and $H$ its subgroup. Then 
    \begin{equation}
      |G| = [G:H] |H|
    \end{equation}
    where $[G:H]$, called the \textbf{index of $H$}, is the number of cosets in $G$. Therefore, the order of a subgroup of a finite group divides the order of the group. 
  \end{theorem}
  \begin{proof}
    The union of the $[G:H]$ disjoint cosets is all of $G$. On the other hand, every $H$ is in one-to-one correspondence with each coset $aH$, so every coset has $|H|$ elements. Therefore, there are $[G:H] |H|$ elements altogether. 
  \end{proof}

  Therefore, Lagrange's theorem says that \textit{given} that you find a subgroup, the order of the subgroup must divide the order of $G$. However, that doesn't mean that such a subgroup may even exist. For example, there is a group of order 12 having no subgroup of order 6. 

  \begin{corollary}
    The order of any element of a finite group divides the order of the group. 
  \end{corollary}
  \begin{proof}
    Take any $a \in G$ and construct the cyclic subgroup $\langle a \rangle \subset G$. Then by Lagrange's theorem, $|a| = |\langle a \rangle|$ divides $|G|$. 
  \end{proof}

  \begin{corollary}
    Every finite group of a prime order is cyclic. 
  \end{corollary}
  \begin{proof}
    Let $a \in G$ be any element other than the identity $e$, and consider $\langle a \rangle \subset G$. The order must divide $|G|$ which is prime, so $|a| = 1$ or $|G|$. But $|a| \neq 1$ since we did not choose the identity, so $|a| = |G| \implies \langle a \rangle = G$. 
  \end{proof}

  \begin{corollary}
    If $|G| = n$, then for every $a \in G$ $a^n = e$. 
  \end{corollary}
  \begin{proof}
    Let $|a| = k$. Then $k \mid n$, and so $a^n = a^{kl} = (a^k)^l = e^l = e$. 
  \end{proof}

  \begin{corollary}[Fermant's Little Theorem]
    Let $p$ be a prime number. The multiplicative group $\mathbb{Z}_{p} \setminus \{0\}$ of the field $\mathbb{Z}_{p}$ is an abelian group of order $p-1 \implies g^{p-1} = 1$ for all $g \in \mathbb{Z}_{p} \setminus \{0\}$. So,
    \begin{equation}
      a^{p-1} \equiv 1 \iff a^{p} \equiv a \pmod{p}
    \end{equation}
  \end{corollary} 

  We can generalize this. 

  \begin{definition}[Euler's Totient Function]
    \textbf{Euler's Totient Function}, denoted $\varphi(n)$, consists of all the numbers less than or equal to $n$ that are coprime to $n$. 
  \end{definition}

  \begin{theorem}[Euler's Theorem]
    For any $n$, the order of the group $\mathbb{Z}_{n} \setminus \{0\}$ of invertible elements of the ring $\mathbb{Z}_{n}$ equals $\varphi(n)$, where $\varphi$ is Euler's totient function. In other words with $G = \mathbb{Z}_{n} \setminus \{0\}$, 
    \begin{equation}
      a^{\varphi(n)} \equiv 1 \pmod{n}, \; \text{ where $a$ is coprime to $n$}
    \end{equation}
  \end{theorem}

  \begin{example}
    In $\mathbb{Z}_{125} \setminus \{0\}$, $\varphi(125) = 125 - 25 = 100 \implies 2^{100} \equiv 1 \pmod{125}$
  \end{example}

\subsection{Normal Subgroups}

  By introducing cosets, we have successfully constructed an equivalence relation on $G$. This set of cosets is indeed a partition of $G$, but we would like to endow it with a group structure that respects that of $G$. That is, let $a, b \in G$ and its corresponding cosets be $aH, bH$. Then, we would like to define an operation $\cdot$ on the cosets such that 
  \begin{equation}
    (aH) \cdot (bH) \coloneqq (ab)H
  \end{equation} 
  That is, we would like to upgrade the equivalence relation to a \textit{congruence relation}. If we try to show that this is indeed a well-defined operation, we run into some trouble. Suppose $aH = a^\prime H$ and $bH = b^\prime H$. Then with our definition, we should be able to derive that $(aH)(bH) = (a^\prime H) (b^\prime H)$ through the equation 
  \begin{equation}
     (aH) (bH) = (ab)H = (a^\prime b^\prime) H = (a^\prime H) (b^\prime H) 
  \end{equation}
  We have $a^\prime = a h_1$, $b^\prime = b h_2$, and $a^\prime b^\prime = ab h$. Then, 
  \begin{align}
    (ab) H = (a^\prime b^\prime) H & \implies a^\prime b^\prime = abh \text{ for some } h \in H \\
                                   & \implies a h_1 b h_2 = abh \text{ for some } h_1, h_2, h \in H
  \end{align}
  But the final statement is not true in general. In an abelian group, we could just swap $h_1$ and $b$ to derive it completely, but perhaps there is a weaker condition on just the subgroup $H$ that allows us to ``swap'' the two. 

  \begin{definition}[Normal Subgroups]
    A subgroup $N \subset G$ is a \textbf{normal subgroup} iff the left cosets equal the right cosets. That is, $\forall g \in G, h \in H$. 
    \begin{equation}
      g^{-1} h g \in H
    \end{equation}
    We call $g^{-1} h g$ the \textbf{conjugate} of $h$ by $g$. 
  \end{definition} 

  \begin{example}[Normal Subgroups]
    For intuition, we provide some examples of normal subgroups. 
    \begin{enumerate}
      \item If $G$ is abelian, every subgroup is normal. So $(2\mathbb{Z}, +)$ is normal, and $(\mathbb{Q}, \times) \subset (\mathbb{R}, \times)$ is also normal. 
      \item Given $G = (\mathbb{R} \setminus \{0\}, \times)$, let $H = (\mathbb{R}^+, \times) \subset G$ be a subgroup. Then $H$ is normal since for any $g \in \mathbb{R}$, $g, g^{-1}$ are either both positive or both negative, and so $g h g^{-1} > 0 \implies g h g^{-1} \in H$. $H$ and $(-1)H$ are two cosets of $\mathbb{R}$. 
      \item $\SL_n (\mathbb{F}) \subset \GL_n (\mathbb{F})$ is a normal subgroup since the determinant of the inverse is ihe inverse of the determinant, and so for any $g \in \GL_n (\mathbb{F})$, 
      \begin{equation}
        \det(g h g^{-1}) = \det(g) \det(h) = \det(g^{-1}) = \det(g) \cdot 1 \cdot \frac{1}{\det(g)} = 1 \implies g h g^{-1} \in \SL_n (\mathbb{F}) 
      \end{equation}

      \item The subgroup $H = \{e, r^2\} \subset \Dih(4)$ is a normal subgroup. It is clearly a subgroup isomorphic to $Z_2$, and to see normality, note that $r^2$ commutes with any $g = r^n \in \Dih(4)$. If $g$ contains a flip, then we can just check the 4 cases knowing that $f r = r^3 f$. 
      \begin{align}
        f r^2 f^{-1} & = f r^2 f = (f r)(r f) = r^3 f r f = r^3 r^3 f^2 = r^2 \\ 
        (rf) r^3 (rf)^{-1} & = \ldots = r^2
      \end{align}
      Therefore $\Dih(4)/H$ has order 4, which means it must be isomorphic to either the cyclic group or the Klein 4 group. It turns out it's the Klein 4 group. 

      \item The subgroup $H = \{e, r, r^2, r^3 \} \subset \Dih(4)$ is a normal subgroup because 
      \begin{align}
        \underbrace{(f^j r^i)}_{g} \underbrace{(r^l)}_{h} 
        \underbrace{(r^{-i} f^{-j})}_{g^{-1}} & = f^j r{i + l - i} f^{-j} \\  
                                              & = f^j r^l f^{-j} \\
                                              & = f^j r^l f_j \\
                                              & = r^{l + 3j}
      \end{align}
      where we used the fact that $frf = r^3 = r^{-1}$ in the penultimate step. So $|\Dih(4)/H| = 2 \implies \Dih(4)/H \simeq Z_2$ with generator $fh$. 
    \end{enumerate}
  \end{example}

  \begin{example}[Subgroups that are Not Normal]
    Here are some subgroups that are not normal. 
    \begin{enumerate}
      \item Given $G = \Dih(3)$, $H = \{e, f\}$ is not normal since $rf r^{-1} = r f r^2 = r^2 f \not\in H$. 
      \item The subgroup 
      \begin{equation}
        H = \bigg\{ \begin{pmatrix} a & b \\ 0 & c \end{pmatrix} \; \bigg| \; ac \neq 0 \bigg\} \subset \GL_2 (\mathbb{R})
      \end{equation} 
      is not normal since 
      \begin{equation}
        h = \begin{pmatrix} 1 & 1 \\ 0 & 1 \end{pmatrix} \in H , a = \begin{pmatrix} 1 & 0  \\ 1 & 1 \end{pmatrix} \in \GL_2 (\mathbb{R}) \implies a h a^{-1} = \begin{pmatrix} 0 & 1 \\ -1 & 2 \end{pmatrix} \not\in H
      \end{equation}
    \end{enumerate}
  \end{example}

  Finally, we present some relevent results of alternating subgroups. 

  \begin{theorem}[Alternating Group is Normal in Symmetric]
    $A_n$ is a normal subgroup of $S_n$, of index\footnote{i.e. the number of cosets} $2$. 
  \end{theorem}
  \begin{proof}
    
  \end{proof} 

  \begin{lemma}[Cycles in Alternating Group]
    \label{thm:cycles_alt}
    We have the following. 
    \begin{enumerate}
      \item Every element of $A_n$ can be written as the product of 3-cycles. 
      \item If $n \geq 4$, $H$ is a normal subgroup of $A_n$, and $H$ contains one 3-cycle, then $H = A_n$. 
    \end{enumerate}
  \end{lemma}
  \begin{proof}
    Since we've proved that every permutation is the product of transpositions, it suffices to prove that the product of two transpositions can be written as the product of 3-cycles. We check this case by case, where distinct symbols represent distinct values. 
    \begin{enumerate}
      \item $(\alpha \; \beta) ( \gamma \; \delta) = (\alpha \; \beta \; \gamma) (\beta \; \gamma \; \delta)$ 
      \item $(\alpha \; \beta)(\alpha \; \gamma) = (\alpha \; \gamma \; \beta)$ 
      \item $(\alpha \; \beta) (\alpha \; \beta) = e$
    \end{enumerate}
    Therefore every even permutation is the product of 3-cycles. 
  \end{proof}

  \begin{definition}[Simple Group]
    A \textbf{simple group} is a group with no proper normal subgroup. That is, the only normal subgroups are the trivial group and itself. 
  \end{definition}

  \begin{theorem}[Alternating Groups are Simple]
    For $n \geq 5$, $A_n$ is a simple group. 
  \end{theorem}
  \begin{proof}
    Let $H \subset A_n$ be a normal subgroup containing more than the identity. If we can find a single 3-cycle in $H$, then it follows from \ref{thm:cycles_alt} that $H = A_n$. Let $\gamma \in H$, $\gamma \neq e$, and write $\gamma = \gamma_1 \ldots \gamma_m$ as a product of disjoint cycles. We have 4 cases. 
    \begin{enumerate}
      \item Let $k \geq 4$ and suppose that some factor, say $\gamma_1$ is a $k$-cycle. WLOG let us assume that $\gamma_1 = (1 \ldots k)$. Since $H$ is normal, $(1, 2, 3) \gamma (1, 2, 3)^{-1} \in H$ and $(1, 2, 3)$ commutes with all the factors of $\gamma$ except $\gamma_1$ (since the cycles are disjoint and so $\gamma_i$ for $i \neq 1$ does not contain $1, 2, 3$). Thus letting 
      \begin{equation}
        \sigma = (1, 2, 3) \gamma (1, 2, 3)^{-1} = (2, 3, 1, 4, \ldots, k) \gamma_2 \ldots \gamma_m \in H
      \end{equation} 
      since $H$ is a group we have 
      \begin{align}
        \sigma \gamma^{-1} & = \begin{pmatrix} 2 & 3 & 1 & 4 \ldots & k \end{pmatrix} \begin{pmatrix} 1 & 2 & 3 & 4 & \ldots & k \end{pmatrix}^{-1} \\
                           & = \begin{pmatrix} 2 & 3 & 1 & 4 & \ldots & k \end{pmatrix} \begin{pmatrix} k & \ldots & 4 & 3 & 2 & 1 \end{pmatrix} = \begin{pmatrix} 1 & 2 & 4 \end{pmatrix}
      \end{align}

      \item Suppose $\gamma$ has at least two 3-cycles as factors, say $\gamma_1 = (1, 2, 3), \gamma_2 = (4, 5, 6)$. Then 
      \begin{equation}
        \sigma = (3, 4, 5) \gamma (3, 4, 5)^{-1} = (1, 2, 4) (3, 6, 5) \gamma_3 \ldots \gamma_m \in H
      \end{equation}
      and again we have 
      \begin{align}
        \sigma \gamma ^{-1} & = 
        \begin{pmatrix} 1 & 2 & 4 \end{pmatrix}
        \begin{pmatrix} 3 & 6 & 5 \end{pmatrix}
        \begin{pmatrix} 4 & 5 & 6 \end{pmatrix}^{-1}
        \begin{pmatrix} 1 & 2 & 3 \end{pmatrix}^{-1} \\ 
                          & = \begin{pmatrix} 1 & 6 & 3 & 4 & 5 \end{pmatrix}
      \end{align}
      which is a 5-cycle, and we are done by case 1. 

      \item Suppose $\gamma$ has precisely one 3-cycle factor and all others are transposisions. If the 3-cycle is $\gamma_1 = (1, 2, 3)$, then $\gamma^2 = (1, 2, 3)^2 = (1, 3, 2)$ is a $3$-cycle. 

      \item Suppose $\gamma$ is the product of disjoint transpositions. Say $\gamma_1 = (1, 2), \gamma_2 = (3, 4)$. Then as before 
      \begin{equation}
        \sigma = \begin{pmatrix} 1 & 2 & 4 \end{pmatrix} \gamma \begin{pmatrix} 1 & 2 & 4 \end{pmatrix}^{-1}  \implies \sigma \gamma^{-1} = \begin{pmatrix} 1 & 4 \end{pmatrix} \begin{pmatrix} 2 & 3 \end{pmatrix} \in H
      \end{equation}
      Since $n \geq 5$ by our theorem hypothesis, the permutation $\tau = (2, 3, 5) \in A_n$, and so 
      \begin{align}
        \tau \begin{pmatrix} 1 & 4 \end{pmatrix} \begin{pmatrix} 2 & 3 \end{pmatrix} \tau^{-1} & = \begin{pmatrix} 1 & 4 \end{pmatrix} \begin{pmatrix} 3 & 5 \end{pmatrix} \in H \\
            \implies & \begin{pmatrix} 1 & 4 \end{pmatrix} \begin{pmatrix} 2 & 3 \end{pmatrix} \begin{pmatrix} 1 & 4 \end{pmatrix} \begin{pmatrix} 3 & 5 \end{pmatrix} = \begin{pmatrix} 2 & 5 & 3 \end{pmatrix} \in H
      \end{align}
    \end{enumerate}
  \end{proof}

\subsection{Quotient Groups}

  Now that we know about normal subgroups, this allows us to endow on the quotient set a group structure. 

  \begin{definition}[Quotient Group]
    Given a group $G$ and a normal subgroup $H$, the \textbf{quotient group} $G/H$ is the group of left cosets $aH$ with 
    \begin{enumerate}
      \item the operation $(aH) \cdot (bH) \coloneqq (ab)H$ 
      \item the identity element $eH$. 
      \item inverses $(aH)^{-1}) = (a^{-1})H$. 
    \end{enumerate}
    and order $|G/H| = |G| / |H|$.  
  \end{definition}
  \begin{proof}
    We verify the properties of a group. 
    \begin{enumerate}
      \item Suppose as above that $aH = a^\prime H$ and $bH = b^\prime H$. Then $a^\prime = ah$ and $b^\prime = bk$ for some $h, k \in H$. Since $H$ is normal, $b^{-1} h b = h^\prime$ for some $h^\prime \in H$. Therefore, 
      \begin{equation}
        a^\prime b^\prime = (ah) (bk) = a(hb) k = (ab h^\prime) k = (ab)(h^\prime k) \in (ab) H
      \end{equation}
      and so $(ab)H = (a^\prime b^\prime)H$. 

      \item $eH$ is indeed the identity since $(aH)(eH) = (ae)H = aH$ and $(eH)(aH) = (ea)H = aH$. 
      \item Inverses are the same logic. 
      \item Associativity follows from associativity in $G$. 
    \end{enumerate}
    Finally, by Lagrange's theorem, the order is as stated. 
  \end{proof}

  Since the quotient defines a \textit{congruence} class, this makes it a group homomorphism. 

  \begin{theorem}[Quotient Maps are Homomorphisms]
    The map $p: G \rightarrow G/H$ is a group homomorphism. 
  \end{theorem}
  \begin{proof}
    Follows immediately from the definition. 
  \end{proof} 

  It's a bit hard thinking of an intuitive picture of a normal subgroup. Unless you sit down and try to prove that a subgroup is normal, it's difficult to tell right away. The following lemma characterizes normal subgroups in a different manner. 

  \begin{lemma}[Normal Subgroup as Kernel]
    \label{thm:normal_kernel}
    A subgroup $H \subset G$ is normal if and only if there exists a group homomorphism $\phi: G \rightarrow G^\prime$ with $\ker{\phi} = H$. 
  \end{lemma}
  \begin{proof}
    We prove bidirectionally. 
    \begin{enumerate}
      \item $(\rightarrow)$. Since $H$ is normal, we can form the quotient group $G/H$. Let $\phi: G \rightarrow G/H$ be defined $\phi(a) = aH$. Then, 
      \begin{align}
        \ker{\phi} = \phi^{-1}(eH) & = \{a \in G \mid aH = eH = H \} \\
                                   & = \{a \in G \mid a \in H \}
      \end{align}
      Therefore, $\phi$ is a homomorphism because $\phi(ab) = abH = (aH)(bH)$.   

      \item $(\leftarrow)$ Assume there is a group homomorphism $\phi$. Then, $\ker{\phi} \subset G$ is a subgroup proven in \ref{thm:kernels_subgroup}. Now consider any $g \in G$. Then 
      \begin{equation}
        \phi(g h g^{-1}) = \phi(g) \phi(h) \phi(g^{-1}) = \phi(g) \cdot e \cdot \phi(g)^{-1} = e \implies g h g^{-1} \in \ker{\phi}
      \end{equation}
    \end{enumerate}
  \end{proof}

  Now that we can construct quotient groups, we would like to see if they are isomorphic to any current groups that we know. More specifically, if we have a normal subgroup $H \subset G$, we can cleverly think of some other group $G^\prime$ and construct a group homomorphism $f: G \to G^\prime$ such that $H = \ker{f}$. If we can do this, then we can construct a nice isomrophism from $G/H$ to $G^\prime$. Recall a similar theorem in point set topology: given a topological space $(X, \mathscr{T})$ and its quotient space, if we can construct a map from $X$ to a cleverly chosen space $Z$ that agrees with the quotient, then this induces a homeomorphism $X \cong Z$. 

  \begin{theorem}[Fundamental Group Homomorphism Theorem]
    Let $f: G \to G^\prime$ be a surjective homomorphism.\footnote{Sometimes called an \textit{epimorphism}.} Then $G/{\ker{f}} \simeq G^\prime$.\footnote{Note that if $f$ is not surjective, we can just have it be surjective by restricting $G^\prime$ to be the image of $f$. }

    \begin{figure}[H]
      \centering 
      \begin{tikzcd}
        G \arrow[r, "f"] \arrow[d, "p"] & G' \\
        G/\ker{f} \arrow[ur, "\bar{f}"'] &
      \end{tikzcd}
      \caption{Given $f$ and the projection map $p: G \to G/{\ker{f}}$, this induces an isomorphism $\bar{f}$ such that $f = \bar{f} \circ p$.} 
      \label{fig:group_fund_homo_theorem}
    \end{figure}
  \end{theorem}
  \begin{proof}
    Let $H = \ker{f}$, which is then a normal subgroup from \ref{thm:normal_kernel}. Now we define a homomorphism 
    \begin{equation}
      \bar{f}: G/H \to G^\prime, \qquad \bar{f}(aH) = f(a)
    \end{equation}
    We check the following. 
    \begin{enumerate}
      \item $\bar{f}$ is well defined. If we have $a, a^\prime \in G$ with $aH = a^\prime H$, then $a^\prime = a h$ for some $h \in H = \ker{f}$. So $f(a^\prime) = f(ah) = f(a) f(h) = f(a)$. 

      \item $\bar{f}$ is a homomorphism. We see that 
      \begin{align}
        \bar{f}((aH)(bH)) & = \bar{f}((ab)H) \\ 
                          & = f(ab) \\
                          & = f(a) f(b) \\  
                          & = \bar{f}(aH) \bar{f}(bH) 
      \end{align}

      \item $\bar{f}$ is surjective. This is trivially true since if not, then $f = \bar{f} \circ p$ cannot be surjective. 

      \item $\bar{f}$ is injective. By \ref{thm:kernels_subgroup}, it suffices to show that $\ker{\bar{f}}$ is trivial. Suppose $aH \in \ker{\bar{f}}$. Then $\bar{f}(aH) = f(a) = e_{G^\prime} \implies a \in H \implies aH = eH$. 
    \end{enumerate}
  \end{proof}

  \begin{example}[Cyclic Groups]
    $(k \mathbb{Z}, +) \subset (\mathbb{Z}, +)$ is a normal subgroup. Our intuition might tell us that the cosets of the form $k \mathbb{Z}, 1 + k \mathbb{Z}, \ldots, (k-1) + k\mathbb{Z}$ behave like integers modulo $k$, i.e. a cyclic group. Therefore, we can construct the map 
    \begin{equation}
      f: \mathbb{Z} \to Z_k, \qquad f(x) = x \pmod{k}
    \end{equation} 
    This is a homomorphism and also $\ker{f} = k \mathbb{Z}$, and so by the fundamental homomorphism theorem 
    \begin{equation}
      \frac{\mathbb{Z}}{k \mathbb{Z}} \simeq Z_k 
    \end{equation}
    By establishing the connection between the integers and cyclic groups, we establish the notation $Z_k = \mathbb{Z}_k$. 
  \end{example}

  \begin{example}[Quotient of Reals over Integers]
    We can see that $(\mathbb{Z}, +) \subset (\mathbb{R}, +)$ is a normal subgroup. Our intuition might tell us that the cosets (which are disconnected sets consisting of isolated points $\{\ldots, x - 1, x, x + 1, \ldots\}$) behave sort of like the rotations on a circle $S^1$. Therefore, let us construct a map 
    \begin{equation}
      f: \mathbb{R} \to S^1, \qquad f(x) = \cos{2\pi x} + i \sin{2\pi x} \subset \mathbb{C}
    \end{equation}
    Since $f(x + y) = f(x) f(y)$, it follows that $f$ is a homomorphism. On the other hand, $\ker{f} = \{ x \in \mathbb{R} \mid \cos{2 \pi x} = 1, \sin{2 \pi x} = 0 \} = \mathbb{Z}$. Therefore by the fundamental homomorphism theorem, we have
     
    \begin{equation}
      \mathbb{R}/\mathbb{Z} \simeq S^1
    \end{equation}
  \end{example}

  \begin{example}[Determinant]
    The determinant $\det: \GL_n (\mathbb{F}) \to \mathbb{F}^\ast = \mathbb{F} \setminus \{0\}$ is a surjective group homomorphism (under multiplication on $\mathbb{F}$). Therefore, 
    \begin{equation}
      \frac{\GL_n (\mathbb{F})}{\SL_n (\mathbb{F})} \simeq \mathbb{F}^\ast
    \end{equation}
  \end{example}

\subsection{Orbits and Stabilizers}

  \begin{definition}[Orbits]
    Let $G$ be a transformation group on set $X$. Points $x, y \in X$ are equivalent with respect to $G$ if there exists an element $g \in G$ such that $y = g x$. This has already been defined through the equivalence of figures before. This relation splits $X$ into equivalence classes, called \textbf{orbits}. Note that cosets are the equivalence classes of the transformation group $G$; oribits are those of $X$. We denote it as
    \begin{equation}
      Gx \equiv \{ g x \;|\;g \in G \}
    \end{equation}
  \end{definition}

  By definition, transitive transformation groups have only one orbit.

  \begin{definition}
    The subgroup $G_{x} \subset G$, where $G_{x} \equiv \{ g \in G | g x = x\}$ is called the \textbf{stabilizer} of $x$.
  \end{definition}

  \begin{example}
    The orbits of $O(2)$ are concentric circles around the origin, as well as the origin itself. The stabilizer of $0$ is the entire $O(2)$.
  \end{example}

  \begin{example}
    The group $S_n$ is transitive on the set $\{1, 2, ..., n\}$. The stabilizer of $k, (1 \leq k \leq n)$ is the subgroup $H_{k} \simeq S_{n-1}$, where $H_k$ is the permutation group that does not move $k$ at all. 
  \end{example}

  \begin{theorem}
    There exists a 1-to-1 injective correspondence between an orbit $G_x$ and the set $G / G_{x}$ of cosets, which maps a point $y = g x \in G x $ to the coset $g G_x$. 
  \end{theorem}

  \begin{corollary}
    If $G$ is a finite group, then 
    \begin{equation}
      |G| = |G_x| |G x|
    \end{equation}
    In fact, there exists a precise relation between the stabilizers of points of the same orbit, regardless of $G$ being finite or infinite: 
    \begin{equation}
      G_{g x} = g G_{x} g^{-1}
    \end{equation}
  \end{corollary}

\subsection{Centralizers and Normalizers} 

\subsection{Lattice of Subgroups} 

\subsection{Exercises}

  \begin{exercise}[Shifrin 6.2.2]
    Prove that $\mathbb{Z}_7^{\times} \cong \mathbb{Z}_6$. (It is crucial to remember that we multiply in $\mathbb{Z}_7^{\times}$ and add in $\mathbb{Z}_6$.)
  \end{exercise}
  \begin{solution}
    Both groups are of order 6, and so $\mathbb{Z}_7^\times$---which is indeed a group (since it is the group of units of the ring $(\mathbb{Z}_7, +, \times)$)---must be isomorphic to either $\mathbb{Z}_6$ or $S_3$. However, $S_3$ is not abelian, while $\mathbb{Z}^\times_7$ is, so it must be the case that it is isomorphic to $\mathbb{Z}_6$. 
  \end{solution}

  \begin{exercise}[Shifrin 6.2.15.a/b]
    The \textbf{dihedral group} of order $2n$, denoted $\mathcal{D}_n$, is given by $\{\rho^i\psi^j : 0 \leq i < n, 0 \leq j \leq 1\}$ subject to the rules $\rho^n = e$, $\psi^2 = e$, and $\psi\rho\psi^{-1} = \rho^{-1}$.
    \begin{enumerate}
      \item Check this is really a group. That is, what is $(\rho^i\psi^j)^{-1}$, and what is the product $(\rho^i\psi^j)(\rho^k\psi^\ell)$?
      \item Check that $\mathcal{T} \cong \mathcal{D}_3$ and $S_q \cong \mathcal{D}_4$.
    \end{enumerate}
  \end{exercise}
  \begin{solution}
    We check the properties of a group. The following identity is useful: 
    \begin{equation}
      (\psi \rho \psi^{-1})^{n-i} = (\rho^{-1})^{n-i} \implies \psi \rho^{n-i} \psi^{-1} = \rho^i \implies \psi \rho^{n-i} = \rho^i \psi
    \end{equation}
    \begin{enumerate}
      \item \textit{Closure}. From simplifying according to the first two rules, we will automatically adjust the exponents to be $i, k < n$ (by subtracting out multiples of $n$) and $j \in \{0, 1\}$ (by subtracting out multiples of $2$). Going case by case, 
      \begin{enumerate}
        \item $j = 0, l = 0$. $\rho^i \rho^k = \rho^{i+k}$. 
        \item $j = 0, l = 1$. $\rho^i \rho^k \psi = \rho^{i+k} \psi$. 
        \item $j = 1, l = 0$. $\rho^i \psi \rho^k = \rho^i \rho^{n-k} \psi = \rho^{n-k+i} \psi$. 
        \item $j = 1, l = 1$. $\rho^i \psi \rho^k \psi = \rho^i \psi \psi \rho^{n-k} = \rho^i \rho^{n-k} = \rho^{n-k+i}$. 
      \end{enumerate}
      \item \textit{Identity}. The identity is $e = \rho^0 \psi^0$. We can see that $e \rho^i \psi^j = \rho^i \psi^j e = \rho^{i+0} \psi^j$. 
      \item \textit{Inverse}. We have $\psi \rho \psi^{-1} = \psi \rho \psi = \rho^{-1} \implies \psi \rho = \rho^{-1} \psi^{-1} = (\psi \rho)^{-1}$. Therefore, 
      \begin{equation}
        (\rho^i \psi^j)^{-1} = \begin{cases} 
          \rho^{n - i} & \text{ if } j = 0  \\
          \rho^{i} \psi & \text{ if } j = 1
        \end{cases}
      \end{equation}
      which are both of the correct form and therefore in $\mathcal{D}_n$. To verify, we see that $\rho^i \rho^{n-i} = \rho^n = e$, and $(\rho^i \psi) (\rho^i \psi) = \rho^i \psi \psi \rho^{n-i} = \rho^i \rho^{n-i} = e$.  
      \item \textit{Associativity}. Can also be proven tediously but problem only asked to state the product and inverse.  
    \end{enumerate} 

    For (b) for $\mathcal{T}$, we can explicitly look at the multiplication tables and see that they are isomorphic. We denote $r_1, r_2$ as the 120 and 240 degree rotations, and $f_1, f_2, f_3$ as the flips across each axis. 

    \begin{figure}[H]
      \centering
      \begin{subfigure}[b]{0.48\textwidth}
        \centering
        \begin{tabular}{c|cccccc}
          & $e$ & $\rho$ & $\rho^2$ & $\psi$ & $\rho\psi$ & $\rho^2\psi$ \\
          \hline
          $e$ & $e$ & $\rho$ & $\rho^2$ & $\psi$ & $\rho\psi$ & $\rho^2\psi$ \\
          $\rho$ & $\rho$ & $\rho^2$ & $e$ & $\rho^2\psi$ & $\psi$ & $\rho\psi$ \\
          $\rho^2$ & $\rho^2$ & $e$ & $\rho$ & $\rho\psi$ & $\rho^2\psi$ & $\psi$ \\
          $\psi$ & $\psi$ & $\rho^2\psi$ & $\rho\psi$ & $e$ & $\rho^2$ & $\rho$ \\
          $\rho\psi$ & $\rho\psi$ & $\psi$ & $\rho^2\psi$ & $\rho$ & $e$ & $\rho^2$ \\
          $\rho^2\psi$ & $\rho^2\psi$ & $\rho\psi$ & $\psi$ & $\rho^2$ & $\rho$ & $e$
        \end{tabular}
        \caption{$\mathcal{D}_3$}
      \end{subfigure}
      \hfill 
      \begin{subfigure}[b]{0.48\textwidth}
        \centering
        \begin{tabular}{c|cccccc}
          & $e$ & $r_1$ & $r_2$ & $f_1$ & $f_2$ & $f_3$ \\
          \hline
          $e$ & $e$ & $r_1$ & $r_2$ & $f_1$ & $f_2$ & $f_3$ \\
          $r_1$ & $r_1$ & $r_2$ & $e$ & $f_3$ & $f_1$ & $f_2$ \\
          $r_2$ & $r_2$ & $e$ & $r_1$ & $f_2$ & $f_3$ & $f_1$ \\
          $f_1$ & $f_1$ & $f_2$ & $f_3$ & $e$ & $r_2$ & $r_1$ \\
          $f_2$ & $f_2$ & $f_3$ & $f_1$ & $r_1$ & $e$ & $r_2$ \\
          $f_3$ & $f_3$ & $f_1$ & $f_2$ & $r_2$ & $r_1$ & $e$
        \end{tabular}
        \caption{$\mathcal{T}$}
      \end{subfigure}
    \end{figure}

    For $S_q$, it is tedious to write the full table, so we construct the isormorphisms using the generators. For $S_q$, the symmetry group of the square consists of 8 elements: the 4 rotations $r_1, r_2, r_3, r_4$ (of 90, 180, 270, and 360=0 degrees), and the flips $f_1, f_2, f_3, f_4$ (across each axis). Now we construct the function $g: \mathcal{D}_3 \rightarrow \mathcal{T}$ such that $f(\rho) = r_1$ and $f(\psi) = f_1$. Then we can see that 
    \begin{equation}
      g(\rho^4) = g(e) = e = r_1^4 = g(\rho^4), \qquad g(\psi^2) = g(e) = e = f_1^2 = g(\psi)^2
    \end{equation}
    since 90 degrees rotated 4 times is $0$ degrees, the identity, and two flips across the same axis is also the identity. Finally, we have 
    \begin{equation}
      g(\psi \rho \psi) = g(\rho^{-1}) = r_1^{-1} = r_3 = f_1 r_1 f_1 = g(\psi) g(\rho) g(\psi)
    \end{equation}
    Where $r_1^{-1} = r_3$ since a rotation of 270 after a 90 is the same as rotation by 360=0, and $r_3 = f_1 r_1 f_1$ is the change of basis symmetry observed in Shifrin Example 6.1.5. Therefore the rules match, making it a homomorphism, and since the order is the same ($\mathcal{D}_3$ has $4 \times 2 = 8$ elements from looking at the indices), this is an isomorphism. 
  \end{solution}

  \begin{exercise}[Shifrin 6.3.8]
    Let $H \subset G$ be a subgroup, and let $a \in G$ be given. Prove that $aHa^{-1} \subset G$ is a subgroup (called a \textbf{conjugate subgroup} of $H$). Prove, moreover, that it is isomorphic to $H$ (cf. Exercise 6.2.12).
  \end{exercise}
  \begin{solution}
    Let $x, y \in aHa^{-1}$. Then $x = a h_x a^{-1}, y = a h_y a^{-1}$ for some $h_x, h_y \in H$. Therefore, 
    \begin{enumerate}
      \item It is closed. $xy = (a h_x a^{-1}) (a h_y a^{-1}) = a h_x (a^{-1} a) h_y a^{-1} = a h_x h_y a^{-1} \in aHa^{-1}$ since $h_x h_y \in H$ by closure. 
      \item It has an identity since $e \in H \implies a e a^{-1} = a a^{-1} = e \in aHa^{-1}$. 
      \item It has inverses since given $x \in H$ as above with inverses $x^{-1}$, we see that $(a x a^{-1})^{-1} = (a^{-1})^{-1} x^{-1} a^{-1} = a x^{-1} a^{-1} \in a H a^{-1}$ since $x^{-1} \in H$ by $H$ being a group. 
      \item Associativity is inherited from $G$. 
    \end{enumerate} 
    It suffices to show that this is injective, since the map $\iota : H \rightarrow a H a^{-1}$ is surjective by definition. Given $x, y \in a H a^{-1}$ with $x = y$, we have $a h_x a^{-1} = a h_y a^{-1}$, and multiplying by $a$ on the right and then $a^{-1}$ on the left, we get $h_x = h_y$.
  \end{solution}

  \begin{exercise}[Shifrin 6.3.11]
    Prove that a group of order $n$ has a proper subgroup if and only if $n$ is composite.
  \end{exercise}
  \begin{solution}
    We prove bidirectionally. Call the group $G$ and subgroup $H$. 
    \begin{enumerate}
      \item $(\rightarrow)$. Assume $n$ is prime. Then by Lagrange's theorem $|H|$ must divide $n$, and so $|H| = 1$ or $n$, neither of which results in a proper subgroup. 
      \item $(\leftarrow)$. Assume $G$ has a proper subgroup $H$. Since it is proper, $|H| \neq 1, n$. Then by Lagrange's theorem, $|H|$ divides $n$, which implies that $n$ is composite. 
    \end{enumerate}
  \end{solution}

  \begin{exercise}[Shifrin 6.3.13]
    Suppose $H, K \subset G$ are subgroups of orders $5$ and $8$, respectively. Prove that $H \cap K = \{e\}$.
  \end{exercise}
  \begin{solution}
    Let us take an arbitrary element in $x \in H \cap K$ and consider the cyclic group $\langle x \rangle$. By Lagrange's Theorem, the order $|x|$ in $H$ must be either $1$ or $5$, while the order in $K$ must be $1, 2, 4, 8$. Therefore, $|x| = 1$ and so $x = e$. 
  \end{solution}

  \begin{exercise}[Shifrin 6.3.17]
    \begin{enumerate}
      \item Prove that a group $G$ of even order has an element of order $2$. (Hint: If $a \neq e$, $a$ has order $2$ if and only if $a = a^{-1}$.)
      \item Suppose $m$ is odd, $|G| = 2m$, and $G$ is abelian. Prove $G$ has precisely one element of order $2$. (Hint: If there were two, they would provide a Klein four-group.)
      \item Prove that if $G$ has exactly one element of order $2$, then it must be in the center of $G$.
    \end{enumerate}
  \end{exercise}
  \begin{solution}
    Listed. 
    \begin{enumerate}
      \item Assume the contrary and take $H = G \setminus \{e\}$. Then $|H|$ is odd, and since no element has order $2$, every element must be associated with a unique inverse $a, a^{-1}$. But this cannot happen since $|H|$ is odd. Therefore there must be at least one element of order $2$. 

      \item It has at least 1 element of order 2 from (1). Now assume that there are two, call them $a, b$. Then $ab \neq a, b$ and $ab$ also has order $2$ since $(ab)(ab) = abba = aa = e$. Therefore, calling $c = ab$, we have $ac = ca = aab = b$ and $bc = cb = abb = a$. This fully defines the multiplication table for the Klein 4 group $K$ of order $4$. Therefore, by Lagrange's theorem, we have found a subgroup $K$ and so $|K|$ must divide $G$. However, this would mean that $m$ must be even, a contradiction. Therefore there is only one such unique $a$. 

      \item Given $a \in G$ with $|a| = 2$, we wish to show that it is an element of $Z = \{ b \in G \mid bx = xb \forall x \in G\}$.\footnote{I am using the definition of center defined in Shifrin 6.3.7.} Consider $z = x^{-1} a x$. We have 
      \begin{equation}
        z^2 = (x^{-1} a x)^2 = x^{-1} a x x^{-1} a x = x^{-1} a^2 x = x^{-1} x = e
      \end{equation}
      which means that $z$ also has order $2$. But since this is unique, it must be that $z = a$. Therefore, by multiplying $x$ on the left, we get 
      \begin{equation}
        x^{-1} a x = a \implies ax = xa
      \end{equation}
    \end{enumerate}
  \end{solution}
  
  \begin{exercise}[Assigned]
    Find all group homomorphisms $\mathbb{Z}_n \to \mathbb{Z}_m$. (Your answer will depend on $n$ and $m$.) 
  \end{exercise}
  \begin{solution}
    Given a homomorphism, $f$, we must have $f(0) = 0$. Let $f(1) = k$. Note that the value of $f(1) = k$ completely determines the homomorphism since the image of every other $l \in \mathbb{Z}_n$ is defined by 
    \begin{equation}
      f(l) = f(\underbrace{1 + \ldots + 1}_{l \text{ times}}) = \underbrace{k + \ldots + k}_{l \text{ times}}
    \end{equation}
    Since the image of $f$ must be a cyclic subgroup of $\mathbb{Z}_m$, we must satisfy 
    \begin{align}
      0 = f(0) & = f(\underbrace{1 + \ldots + 1}_{n \text{ times}}) \\
               & = \underbrace{k + \ldots + k}_{n \text{ times}} 
    \end{align}
    and so $m \mid nk$. Therefore, $k$ must be a multiple of $m/\gcd(n, m)$. So all homomorphisms are determined by the set 
    \begin{equation}
      \bigg\{ k = \frac{a m}{\gcd(n, m)} \; \bigg| \; a \in \mathbb{N}, 0 \leq k \leq m-1 \bigg\}
    \end{equation}
    which we can see ranges from $0 \leq a < \gcd(n, m)$, and so the total number of homomorphisms is $\gcd(n, m)$. Note that there is always the trivial homomorphism when $a = 0$, i.e. everything maps to $0$. For example, if we have $f: \mathbb{Z}_{14} \to \mathbb{Z}_{21}$, we have $k = 0, 3, 6, 9, 12, 15, 18$. 
  \end{solution}


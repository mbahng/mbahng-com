\section{Jan 8} 

  We start off with some useful geometry, which doesn't seem relevant now but will be helpful later on. 

  \begin{definition}[Complex Projective Line]
    Define equivalence relation on $\mathbb{C}^2 \setminus \{ 0\}$ by specifiying $\zeta \sim \xi$ iff $\exists 0 = c \in \mathbb{C}$ s.t. $\xi = c \zeta$. Here, $0 \neq \zeta, \xi \in \mathbb{C}^2$. 

    That is, $\xi \sim \zeta$ means that they span the same 1-dimensional linear subspace. Then, we define the \textbf{complex projective line} as 
    \begin{equation}
      \mathbb{CP}^1 = (\mathbb{C}^2 \setminus \{0\}) / \sim
    \end{equation} 
    The $\mathbb{C}$ stands for complex, but we will omit it since this is about complex analysis. The set of all 1-dimensional linear subspaces is called the \textbf{Grassmanian}, denoted
    \begin{equation}
      \mathrm{Gr}(1, \mathbb{C}^2) 
    \end{equation}
    These spaces are all the same. 
  \end{definition}

  For notation, given $\zeta = (\zeta_0, \zeta_1) \neq 0 \in \mathbb{C}^2$ , we denote $[\zeta] = [\zeta_0 : \zeta_1] \in \mathbb{P}^1$ be the line spanned by $\zeta$. 

  A philosophy of complex geometry is that to study a space, you should study the functions on that space. So we will study \textit{holomorphic functions} on the space. So what are the functions are $\mathbb{P}^1$? We care about holomorphic functions, and the only holomorphic functions on $\mathbb{P}^1$ tend to be complex. 

  Let's let $\mathbb{C}[\zeta_0, \zeta_1]$ the ring of complex polynomials in two variables, and let $\mathbb{C}[\zeta_0, \zeta_1]_d \subset \mathbb{C}[\zeta_0, \zeta_1]$ be the vector space of homogeneous polynomials of degree $d$. 

  Suppose $P \in \mathbb{C}[\zeta_0, \zeta_1]$. Then $P$ is homogeneous of degree $d$ iff $P(\lambda \zeta_0, \lambda \zeta_1) = \lambda^d P (\zeta_0, \zeta_1)$ for all $\zeta \in (\zeta_0, \zeta_1) \in \mathbb{C}^2$ and for all $\lambda \in \mathbb{C}$. 

  Fix a line $[\zeta] \in \mathbb{P}^1$ in $\mathbb{C}^2$. This line is paramterized by $\lambda \in \mathbb{C} \to \lambda \zeta \in \mathbb{C}^2$. If we restrict $P$ to this line, it takes the values $P(\lambda \zeta) = \lambda^d P (\zeta)$. Therefore, $P$ does \textit{not} define a function on $\mathbb{P}^1$. Intuitively, we can think of this as ``if it vanishes at 1 (nonzero) point on the line, it vanishes at all points on the line.'' But $\{P = 0\} \subset \mathbb{P}^1$ is well-defined, because $P(\zeta) = 0 \implies P(c \zeta) = 0$ for all $c \in \mathbb{C} \implies P(\xi) = 0$ for all $\xi \sim \zeta$. 

  \begin{theorem}
    Suppose $P, Q \in \mathbb{C}[\zeta_0, \zeta_1]_d$. We claim the map $F: \mathbb{P}^1 \to \mathbb{P}^1$ defined by 
    \begin{equation}
      [\zeta] \mapsto [ P(\zeta), Q(\zeta)] 
    \end{equation}
    is well-defined\footnote{Maps of this form are called \textit{rational functions}, which has some connection to what you already might think of as rational functions.} as long as $\{P = 0 \} \cap \{Q = 0\} = \{0\} \subset \mathbb{C}^2$. 
  \end{theorem}
  \begin{proof}
    Suppose $\zeta \neq 0 \in \mathbb{C}^2$, but $P(\zeta), Q(\zeta) = 0$, i.e. is in the $0$-locus of both polynomials. Then $[P(\zeta) : Q(\zeta)] = [0 : 0]$ is not defined. So $\{P = 0\} \cap \{Q = 0\} = \{0\}$ is necessary. 

    To see that $F$ is well-defined, we must show that $F$ takes the same value on $[\zeta]$ and $[c \zeta]$ for all $0 \neq c \in \mathbb{C}$. 
    \begin{align}
      F([c \zeta]) & = [P(c \zeta) : Q(c \zeta)] \\ 
                   & = [ c^d P(\zeta) : c^d Q(\zeta)] \\ 
                   & = [P(\zeta) : Q(\zeta)] = F([\zeta])
    \end{align}
    So the definition of $F$ is independent of the generator of the line, so this is nice. 
  \end{proof}

  \begin{example}
    If $A \in \mathrm{GL}_2 \mathbb{C}$, then 
    \begin{equation}
      [\zeta] \mapsto [A \zeta] 
    \end{equation}
    is a well-defined map $[A] : \mathbb{P}^1 \to \mathbb{P}^1$. Maps of this form are called Mobius transformations.\footnote{very important since they are very flexible and can be used to normalize problems to make solutions much easier.} Just do the same thing. 
  \end{example}

  We know that $\mathbb{P}^1$ is the 1-point compactification of $\mathbb{C}$. We can write $\mathbb{P}^1 = U_0 \cup U_1$, where 
  \begin{align}
    U_0 & = \{[\zeta_0 : \zeta_1] \in \mathbb{P}^1 \colon \zeta_0 \neq 0 \} \\
    U_1 & = \{[\zeta_0 : \zeta_1] \in \mathbb{P}^1 \colon \zeta_1 \neq 0 \}
  \end{align}
  This is handy, since it gives me a sneaky way to put the complex plane into the projective line. Define map $\mathbb{C} \to U_1$, 
  \begin{equation}
    z \mapsto [z : 1] \in U_1 \subset \mathbb{P}^1
  \end{equation}
  This maps is bijective, with inverse 
  \begin{equation}
    [\zeta_0 : \zeta_1] = \big[ \frac{\zeta_0}{\zeta_1} : 1 \big] \mapsto \frac{\zeta_0}{\zeta_1} = z
  \end{equation}

  What we missed is 
  \begin{align}
    \mathbb{P}^1 \setminus U_1 & = \{ [\zeta_0: 0] \colon \zeta_0 \neq 0\} \\ 
                               & = \{ [1 : 0] \}
  \end{align}
  which is just one point. We call $[1 : 0] = \infty$, i.e. the point at infinity. So we have 
  \begin{equation}
    \mathbb{P}^1 = U_1 \sqcup \{[1 : 0]\}  = \mathbb{C} \cup \{\infty\}
  \end{equation}
  Conway calls this space $\mathbb{C}_{\infty}$, but we also call is $\mathbb{R} \cup \{\infty\} = \mathbb{RP}^1$. But to talk about a compactification, we must extend the topology of $\mathbb{C}$ to $\mathbb{P}^1$ by declaring 
  \begin{equation}
    B(\infty, \frac{1}{r}) = \{ z \in \mathbb{C} \mid |z| > r \} \cup \{\infty\}
  \end{equation}
  to be in the basis of $+\infty = [1 : 0] \in \mathbb{P}^1$. 

  \begin{exercise}
    Show that rational functions $F = [P : Q] : \mathbb{P}^1 \to \mathbb{P}^1$ are continuous. 
  \end{exercise}

  Finally, $\mathbb{P}^1$ is also known as the Riemann sphere because $\mathbb{P}^1$ is homeomorphic to $S^2$. We can identify $\mathbb{R}^3$ with $\mathbb{C} \times \mathbb{R}$, embed $S^2$ in $\mathbb{C} \times \mathbb{R}$ s.t. 
  \begin{equation}
    S^2 = \{ (z, t) \in \mathbb{C} \times \mathbb{R} \colon |z|^2 + |t|^2 = 1 \}
  \end{equation}
  and the map is 
  \begin{equation}
    \phi(z) = \bigg( \frac{2z}{|z|^2 + 1}, \frac{|z|^2 - 1}{|z|^2 + 1} \bigg) \in S^2 \subset \mathbb{C} \times \mathbb{R}
  \end{equation}

\subsection{Rational Functions} 

  Suppose $P \in \mathbb{C}[\zeta_0 : \zeta_1]_d$. Then $p(z) = P(z, 1) \in \mathbb{C}[z]$ is a polynomial of degree $\leq d$. Now conversely, suppose that $p(z) \in \mathbb{C}[z]$ is a polynomial of degree $d$. We get a homogeneous polynomial of degree $d$ in $\mathbb{C}[\zeta_0, \zeta_1]$ by a process called ``homoegenization.'' 
  \begin{equation}
    P(\zeta) = P(\zeta_0, \zeta_1) = \zeta_1^d p (\zeta_0 / \zeta_1)
  \end{equation} 
  If $p(z) = \sum_{j=0}^d a_j z^j$. Then
  \begin{equation}
    P(\zeta) = \zeta_1^d \sum_{j=0}^d a_j \Big( \frac{\zeta_0}{\zeta_1} \Big)^j = \sum_{j=0}^d a_j \zeta_0^j \zeta_1^{d-j}
  \end{equation}
  So we have a way of toggling back and forth between two variable homogenous polynomials and one variable nonhomogeneous polynomials. 

  Given $P, Q \in \mathbb{C}[\zeta_0, \zeta_1]_d$, we have the map $F = [P : Q] : \mathbb{P}^1 \to \mathbb{P}^1$. Then, we have 

  \begin{figure}[H]
    \centering 
    \begin{tikzcd}
      \mathbb{P}^1 \arrow[r, "F = {[P : Q]}"] & \mathbb{P}^1 \\
      \mathbb{C} \arrow[u, "i"] & 
    \end{tikzcd}
    \caption{} 
  \end{figure}

  Notice that $F \circ i : z \in \mathbb{C} \mapsto [z : 1] \mapsto [p(z) : q(z)] = \big[ \frac{p(z)}{q(z)} : 1 \big]$. When convenient, we will conflate $F$ with $\frac{p(z)}{q(z)}$. 


\subsection{Mobius Transformations} 

  Let $A \in \mathrm{GL}_2 \mathbb{C}$ and let $[A] : \mathbb{P}^1 \to \mathbb{P}^1$ be the Mobius transformation $[\zeta] \to [A \zeta]$. It is not hard to see that  
  \begin{align}
    [A] [B] & = [AB] \\ 
    [A]^{-1} & = [A^{-1}]
  \end{align}
  Therefore, the set of all mobius transformations, 
  \begin{equation}
    \mathrm{PGL}_2 \mathbb{C} = \{ [A] : \mathbb{P}^1 \to \mathbb{P}^1 \text{ s.t. } A \in \mathrm{GL}_2 \mathbb{C} \}
  \end{equation}
  is a group. And the map $\mathrm{GL}_2 \mathbb{C} \to \mathrm{PGL}_2 \mathbb{C}$ is a group homomorphism. The kernel is the scalar multiple of identity maps. Suppose $A \in \mathrm{ker}$, so $[A] : \mathbb{P}^1 \to \mathbb{P}^1$ is the identity, so this means that it preserves every 1-dimensional subspaces of $\mathbb{C}^2$. Therefore, the whole space must be an eigenspace. Therefore, 
  \begin{equation}
    \mathrm{PGL}_2 \mathbb{C} = \mathrm{GL}_2 \mathbb{C}  (C \times I) = \mathrm{SL}_2 \mathbb{C} / \{\pm I\} 
  \end{equation}
  Let's write 
  \begin{equation}
    A = \begin{pmatrix} a & b \\ c & d \end{pmatrix} \in \mathrm{GL}_2 \mathbb{C}
  \end{equation} 
  Then, the composition $[A] \circ \iota : \mathbb{C} \to \mathbb{P}^1$ sends 
  \begin{equation}
    z \xrightarrow{\iota} [z : 1] \xrightarrow{[A]} [az + b : cz + d] = \bigg[ \frac{az + b}{cz + d} : 1 \bigg]
  \end{equation}
  When convenient, we will conflate $[A]$ with $S_A (z) = \frac{az + b}{cz + d}$. 

  \begin{exercise}
    Suppose $\xi_0, \xi_1, \xi_2$ are pairwise disjoint points in $\mathbb{P}^1 = \mathbb{C}_\infty$. Then, the Mobius transformation 
    \begin{equation}
      S(z) = \frac{z - \xi_0}{z - \xi_\infty} \cdot \frac{\xi_1 - \xi_\infty}{\xi_1 - \xi_0}
    \end{equation}
    maps $\xi_0 \to 0$, $\xi_1 \to 1$, $\xi_\infty \to \infty$. Therefore, given that I choose three such numbers and you choose three such numbers, there is a Mobius transformation that can send my three numbers to your three numbers simply by mapping it through $0, 1, \infty$. Show that 
    \begin{enumerate}
      \item This Mobius transformation satisfying $\xi_0 \to 0$, $\xi_1 \to 1$, $\xi_\infty \to \infty$ is unique. There are many ways to approach this, with the alegebraic argument a bit tedious, but the geometric is another way. 
      \item Suppose a Mobius transformation $T$ fixes 3 distinct points in $\mathbb{C}_\infty$. Show that $T$ is the identity. This is a straightforward consequence of (a). 
    \end{enumerate}
  \end{exercise}

\section{Jan 13} 

  Recall that 
  \begin{equation}
    A = \begin{bmatrix} a & b \\ c & d \end{bmatrix}, [A]: \mathbb{P}^1 \to \mathbb{P}^1 
  \end{equation}
  defines a Mobius transformation by $[\xi] \mapsto [A \xi]$. Also, given $\iota: \mathbb{C} \to \mathbb{P}^1$ defined $z \mapsto [z : 1]$ given by $[A] \circ \iota (z) = [az + b : cz + d] = \big[ \frac{az + b}{cz + d} : 1 \big]$. 

  We will study actions of Mobius transformations on lines and circles. We will declare lines to be circles of infinite radius. There is a unique circle passing through 3 distinct points $\xi_0, \xi_1, \xi_2 \in \mathbb{P}^1 = \mathbb{C}_{\infty}$. If one of the points is $\infty$ or all three points are colinear, then we have a line. 

  \begin{exercise}
    Every Mobius transformation maps circles to circles. This is pretty surprising statement, and is surprisingly easy to prove. First show that every Mobius transformation $S(z) = \frac{az + b}{cz + d}$ can be written as a composition of 
    \begin{enumerate}
      \item homotheties: $z \mapsto rz$ for $0 < r \in \mathbb{R}$. 
      \item rotations: $z \mapsto e^{i \theta z}$ for $\theta \in \mathbb{R}$. 
      \item translations: $z \mapsto z + p$ for $p \in \mathbb{C}$. 
      \item inversion: $z \mapsto \frac{1}{z}$. 
    \end{enumerate}
    The first three clearly sends circles to circles, but for inversions, you can show it with algebraic calculations. 
  \end{exercise}

  \begin{corollary}
    The Mobius transformation 
    \begin{equation}
      S(z) = \frac{z - \xi_0}{z - \xi_\infty} \frac{\xi_1 - \xi_\infty}{\xi_1 - \xi_0} 
    \end{equation}
    maps the circle through $\xi_0, \xi_1, \xi_\infty \in \mathbb{C}_\infty$ to the circle $\mathbb{R}_\infty \subset \mathbb{C}_\infty$ (real line). 
  \end{corollary}

  \begin{corollary}
    The group $\mathrm{PGL}_2 (\mathbb{C})$ of Mobius transformations acts transitively on the set of circles. 
  \end{corollary}
  \begin{proof}

  \end{proof}

  \begin{corollary}
    A point $z$ lies on the circle through $\xi_0, \xi_1, \xi_2 \in \mathbb{C}_\infty$ iff 
    \begin{equation}
      \underbrace{(z, \xi_1, \xi_0, \xi_{\infty})}_{\text{cross ratio}} = \frac{z - \xi_0}{z - \xi_\infty} \frac{\xi_1 - \xi_\infty}{\xi_1 - \xi_0} \in \mathbb{R}_\infty
    \end{equation}
  \end{corollary}

  We will use Mobius transformations to analyze functions. 

  \begin{theorem}
    The subgroup of Mobius transformations preserving the disc $B = \{z \in \mathbb{C} \colon \|z\| = 1\}$ acts transitively on $B$, i.e. given $w, w^\prime \in B$, there exists a Mobius transformation $S(z)$ s.t. $S(B) = B$, $S(w) = w^\prime$. 
  \end{theorem}
  \begin{proof}
    Very messy to do directly, but becomes easy when we think of it as linear maps. Define a Hermitian form $\phi$ in $\mathbb{C}^2$ by 
    \begin{equation}
      \phi(\zeta, \xi) = - \zeta_0 \bar{\xi}_0 + \zeta_1 \bar{\xi}_1, \quad \zeta = \begin{bmatrix} \zeta_0 \\ \zeta_1 \end{bmatrix}, \xi = \begin{bmatrix} \xi_0 \\ \xi_1 \end{bmatrix} \in \mathbb{C}^2
    \end{equation}
    We can set 
    \begin{equation}
      H = \begin{bmatrix} -1 & 0 \\ 0 & 1 \end{bmatrix} 
    \end{equation}
    Then, $h(\zeta, \xi) = \zeta^T H \bar{\xi}$. The automorphism group of $\phi$ is 
    \begin{align}
      \mathrm{GL}_2 (\mathbb{C}, \phi) & = \{ A \in \mathrm{GL}_2 (\mathbb{C}) \colon \phi(\zeta, \xi) = \phi(A \zeta, A \xi) \forall \zeta, \xi \in \mathbb{C}^2 \} \\ 
                                       & = \{A \in \mathrm{GL}_2 (\mathbb{C}) \colon A^T H \bar{A} = H \} \\ 
                                       & \simeq U(1, 1)
    \end{align}
    To write it out, 
    \begin{equation}
      A = \begin{bmatrix} a & b \\ c & d \end{bmatrix} = \begin{bmatrix} \zeta & \xi \end{bmatrix}
    \end{equation}
    Then we have the characterization
    \begin{equation}
      A \in \mathrm{GL}_2 (\mathbb{C}, \phi) \iff \begin{cases} 
        \phi(\zeta, \zeta) & = -1 \\ 
        \phi(\xi, \xi) & = 1 \\ 
        \phi(\zeta, \xi) & = 0
      \end{cases}
    \end{equation}
    For sake of argument, let's write $\zeta = (z, 1)^T$. Then $\phi(\zeta, \zeta) = - \|z\|^2 + 1$. Then we can identify $\mathbb{P}^1 = B \sqcup S^\prime \sqcup B^\ast$, where 
    \begin{align}
      B & = \{ [\zeta] \in \mathbb{P}^1 \coloneq \phi(\zeta, \zeta) > 0 \} = \{ [z : 1] \in \mathbb{P}^1 \colon \|z\| < 1 \} \\ 
      S^1 = \partial B & = \{ [\zeta] \in \mathbb{P}^2 \colon \phi(\zeta, \zeta) = 0 \} = \{[z : 1] \in \mathbb{P}^1 \colon \|z\| = \} \\ 
      B^\ast & = \{[\zeta] \in \mathbb{P}^1 \colon \phi(\zeta, \zeta) < 0 \} = \{[z : 1] \in \mathbb{P}^1 \colon \|z\| \geq 1 \} \\ 
             & = \{[1 : z] \in \mathbb{P}^1 \colon \|z\| \leq 1 \} 
    \end{align}
    So every matrix gives me a Mobius transformation. By construction, if the transformation preserves $\phi$, it preserves each of these sets. That is, if $A \in \mathrm{GL}_2 (\mathbb{C}, \phi)$, then the Mobius transformation $[A] : \mathbb{P}^2 \to \mathbb{P}^2$ preserves $B, S^1, B^\ast$. 

    Fix $w, w^\prime \in B$. It remains to find $A \in \mathrm{GL}_2 (\mathbb{C}, \phi)$  s.t. $S_A (w) = w^\prime$. Fix $\xi = [w : 1] \in B$ s.t. $\phi(\xi , \xi) = 1$. $\zeta \in [1 : w] \in B^\ast$ s.t. $\phi(\zeta, \zeta) = -1$. Observe that $\phi(\zeta, \xi) = 0 \implies A = [\zeta, \xi] \in \mathrm{GL}_2 (A, \phi)$. Similarly, fix $\xi^\prime \in [w^\prime : 1] \in B, \zeta^\prime \in [1 : \bar{w}^\prime ] \in B^\ast$. Then 
    \begin{equation}
      \begin{cases} 
        \phi(\xi^\prime, \xi^\prime) & = 1 \\ 
        \phi(\zeta^\prime, \zeta^\prime) & = -1 \\ 
        \phi(\zeta, \xi) & = 0 
      \end{cases} \implies A^\prime = [\zeta^\prime, \xi^\prime] \in \mathrm{GL}_2 (\mathbb{C}, \phi)
    \end{equation}
    Therefore, $\theta = [0 : 1] \in B$. 
    \begin{equation}
      A \begin{bmatrix} 0 \\ 1 \end{bmatrix} = \xi, A^\prime \begin{bmatrix} 0 \\ 1 \end{bmatrix} = \xi^\prime \implies B = A^{\prime} \circ A^{-1} \text{ maps } \xi \mapsto \xi^\prime
    \end{equation}
    which implies that $S_B (w) = w^\prime$. 
  \end{proof}


  \begin{definition}[Differentiable]
    Let $U \subset \mathbb{C}$ be open and $p \in U$. Then $f: U \subset \mathbb{C} \to \mathbb{C}$ is \textbf{(complex) differentiable at $p$} if the limit 
    \begin{equation}
      f^\prime (p) \coloneqq \lim_{z \to p} \frac{f(z) - f(p)}{z - p} 
    \end{equation}
    exists. 
  \end{definition}

  \begin{example}
    \begin{enumerate}
      \item \textit{Constant}. If $f(z) = c$ is constant, then $f(z) - f(p) = 0 \implies f^\prime (p) = 0$ for all $p \in \mathbb{C}$. 
      \item \textit{Identity}. If $f(z) = z$, then 
        \begin{equation}
          \frac{f(z) - f(p)}{z - p} = \frac{z - p}{z - p} = 1
        \end{equation}
        so $f^\prime (p) = 1$ for all $p \in \mathbb{C}$. 
    \end{enumerate}
  \end{example}

  \begin{theorem}
    Suppose $f, g$ are $\mathbb{C}$-differentiable at $p \in \mathbb{C}$. Then 
    \begin{enumerate}
      \item $f + g$ is differentiable, with 
        \begin{equation}
          (f + g)^\prime (p) = f^\prime (p) + g^\prime (p)
        \end{equation}

      \item $fg$ is differentiable, with 
        \begin{equation}
          (fg)^\prime (p) = f^\prime (p) g(p) + f(p) g^\prime (p)
        \end{equation} 

      \item If $g(p) \neq 0$, then $f/g$ is differentiable at $p$ with 
        \begin{equation}
          (f/g)^\prime (p) = \frac{f^\prime (p) g (p) - f(p) g^\prime (p)}{g(p)^2}
        \end{equation}
    \end{enumerate}
  \end{theorem}
  \begin{proof}
    Same as in real analysis. 
  \end{proof}

  \begin{corollary}
    Polynomials $f(z) \in \mathbb{C}[z]$ are $\mathbb{C}$-differentiable at every point $p \in \mathbb{C}$. Likewise, so are rational functions. 
  \end{corollary}

  \begin{example}
    $f(z) = \|z\|^2 = z \bar{z}$ is complex differentiable only at $p = 0 \in \mathbb{C}$. 
  \end{example}

  Note that if $f$ is $\mathbb{C}$ differentiable at $p$, then there is a ``Taylor formula'' saying that 
  \begin{equation}
    f(p + z) = f(p) + f^\prime (p) z + \epsilon (z)
  \end{equation}
  where $\frac{\epsilon(z)}{z} \to 0$ as $z \to 0$. The Lagrange's form may not be as well-defined nor as useful, but will have to look into this. 

  Let's compare complex differentiability to real differentiability. Write 
  \begin{equation}
    f(x, y) = u(x, y) + i v(x, y), \quad z = x + i y \in \mathbb{C}
  \end{equation} 
  Think of $f$ as a map $f: U \subset \mathbb{R}^2 \to \mathbb{R}^2$. Then $f$ is $\mathbb{R}$-differentiable at $p$ if the $\mathbb{R}$ derivative at $p$ exists. If it exists the $\mathbb{R}$-derivative is the unique linear map $Df(p): \mathbb{R}^2 \to \mathbb{R}^2$ s.t. 
  \begin{equation}
    \lim_{(x, y) \to (0, 0)} \frac{\| f(p + (x, y))}{} ..
  \end{equation} 
  Then as a consequence of the definition, we have Taylor's formula, 
  \begin{equation}
    f(p + (x, y)) = f(p) + Df(p)(x, y) + \epsilon (x, y) \text{ with } \frac{\|\epsilon(x, y)\|}{\|(x, y)\|} \to 0 \text{ as } (x, y) \to 0
  \end{equation}
  So $Df(p)$ has matrix representation (w.r.t. the standard basis). 
  \begin{equation}
    \begin{bmatrix} u_x (p) & v_x (p) \\ u_y (p) & v_y (p) \end{bmatrix}
  \end{equation}
  where the subscripts represent partial derivatives. Notice that $x = \frac{1}{2} (z + \bar{z}), y = \frac{i}{2} (\bar{z} - z)$. This motivates 
  \begin{align}
    \frac{\partial}{\partial z} & = \frac{\partial x}{\partial z} \frac{\partial}{\partial x} + \frac{\partial y}{\partial z} \frac{\partial}{\partial y} = \frac{1}{2} \bigg( \frac{\partial}{\partial x} - i \frac{\partial}{\partial y} \bigg) \\ 
    \frac{\partial}{\partial \bar{z}} & = \frac{\partial x}{\partial \bar{z}} \frac{\partial}{\partial x} + \frac{\partial y}{\partial \bar{z}} \frac{\partial}{\partial y} = \frac{1}{2} \bigg( \frac{\partial}{\partial x} + i \frac{\partial}{\partial y} \bigg)
  \end{align}
  This is motivates and not implies since this isn't a true (strict) change of coordinates. 

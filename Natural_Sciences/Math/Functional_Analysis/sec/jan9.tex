\section{Jan 9} 

  Assume knowledge of 403 and 531 basically. Dunford \& Schwartz. Recall the definition of a \textit{normed vector space}. It induces a metric and hence a topology. A metric space is a Banach space if it is complete. It's often that if you have infinite-dim vector spaces, you have a lot of topologies to choose from, each of which have different roles. We will find out how to utilize each of them. Some examples are $\mathbb{R}^n$ with the $p$-norm, $\ell^p$ such that $\big( \sum_{j=1}^\infty |x_j|^p \big) < \infty$, or $L^p (\mathbb{R}^n)$. We will begin with the simplest topology. 

  \begin{definition}[Bounded Linear Operators]
    Let $(X, \|\cdot\|_X)$ and $(Y, \|\cdot\|_Y)$ be 2 normed spaces. Then, a linear map $A: X \to Y$ is called \textbf{bounded} if there exists a $c \geq 0$, s.t. $\forall v \in X$, 
    \begin{equation}
      \|Av \|_Y \leq c \|v\|_X
    \end{equation}
    The smallest such $c$ is denoted $\|A\| = \|A\|_{\mathrm{sup}} = \|A\|_\infty$, i.e. the spectral norm. i.e. 
    \begin{equation}
      \|A\| \coloneqq \sup_{w \in X \setminus \{0\}} \frac{\|Aw\|_Y}{\|w\|_X}
    \end{equation}
    Note that due to linearity, we can take the sup over the unit sphere as well. 
  \end{definition}

  Set 
  \begin{equation}
    L(X, Y) \coloneqq \{A : X \to Y \coloneq A \text{ linear, bounded}\}
  \end{equation}
  which is a normed vector space. The topology is called the \textit{uniform operator topology}. 

  \begin{theorem}[1.2.2.]
    Let $X, Y$ be two normed vector spaces. Let $A: X \to Y$ be linear. TFAE. 
    \begin{enumerate}
      \item $A$ is bounded. 
      \item $A$ is continuous. 
      \item $A$ is continuous at $0$. 
    \end{enumerate}
  \end{theorem}
  \begin{proof}
    Listed. 
    \begin{enumerate}
      \item $(1 \to 2)$. The one-liner is that $A$ is Lipschitz, which is continuous, since 
        \begin{equation}
          \| Au - Av \|_Y = \| A(u - v) \|_Y \leq \|A\| \|u - v\|
        \end{equation}
      \item $(2 \to 3)$. Clear. 
      \item $(3 \to 1)$. For every $\epsilon > 0$, $\exists \delta_\epsilon > 0$ s.t. 
        \begin{equation}
          A^{-1} \big( B_Y (0, \epsilon) \big) \supset B_X (0, \delta_\epsilon)
        \end{equation}
        where $A^{-1}$ indicates the preimage. In other words, $|u|_X < \delta_\epsilon \implies |Au|_Y < \epsilon$. Let $w \in X \setminus \{0\}$. Then, 
        \begin{equation}
          w = |w| \cdot \frac{2}{\delta_\epsilon} \frac{w}{|w|} \frac{\delta_\epsilon}{2}
        \end{equation}
        So by linearity, we have 
        \begin{equation}
          Aw = \frac{|w| 2}{\delta_\epsilon} A \bigg( \frac{w}{|w|} \frac{\delta_\epsilon}{2} \bigg) \implies |A w| \leq \frac{2 |w|}{\delta_\epsilon} \epsilon
        \end{equation}
        So 
        \begin{equation}
          \|A\| \leq \frac{2 \epsilon}{\delta_\epsilon}
        \end{equation}
    \end{enumerate}
  \end{proof}

  So basically, boundedness is equivalent to continuity, which is why most analysts look at bounded operators. Let's review some notation from linear algebra, given $A \in L(X, Y)$. Then, 
  \begin{enumerate}
    \item $\mathrm{ker}(A) \subset X$ is a closed subspace. This was obvious in finite dimensions, but in infinite dimensions, you need the boundedness. 
    \item $\mathrm{Im}(A) \subset Y$ is a subspace, but it is not closed, which is not as good. 
    \item $\mathrm{koker}(A) \coloneqq Y / \mathrm{Im}(A)$ is also a vector space, but it can be nasty, since if $\mathrm{Im}(A)$ is not closed, then the quotient space is not Hausdorff. 
  \end{enumerate} 

  \begin{definition}
    2 norms $|\cdot|_1, |\cdot|_2$ on $X$ are equivalent if $\exists c_1 \leq c_2$ s.t. $\forall v \in X$, 
    \begin{equation}
      c_1 |v_1| \leq |v|_2 \leq c_2 |v_1|
    \end{equation}
  \end{definition}

  It is easy to see that equivalent norms induce the same topology, and equivalence of norms is an equivalence relation. 

  \begin{theorem}
    Let $X$ be any finite-dimensional vector space. Then, any two norms on $X$ are equivalent. 
  \end{theorem}
  \begin{proof}
    Choose a basis $\{e_1, \ldots, e_n\}$ for $X$. Let's define the Euclidean norm $\| \sum_{j=1}^n x_j e_j\|_2 \coloneqq \sqrt{\sum_j x_j^2}$. Let $|\cdot|$ be any other norm on $X$. Then, by the triangle inequality, followed by Cauchy-Schwartz, 
    \begin{equation}
      \bigg| \sum_{j=1}^n x_j e_j \bigg| \sum_j |x_j| |e_j|  \leq \sqrt{\sum_j x_j^2} \underbrace{\sqrt{\sum_j |e_j|^2}}_{= c_1} \implies |v| \leq c_1 \|v\|_2
    \end{equation}

    Now let $S^{n-1} = \|\cdot\|^{-1} (1)$, i.e. the preimage of the Euclidean norm of $1$. So $| \cdot |$ is continuous w.r.t. the $\|\cdot\|_2$-topology,\footnote{it is trivially continuous w.r.t. its own topology that it induces.} Since it is well known that $S^{n-1}$ is compact, $|\cdot|$ attains its minimum on $S^{n-1}$, call it $c_2$. For $v \neq 0$, 
    \begin{equation}
      |v| = \|v\|_2 \bigg| \frac{v}{\|v\|_2} \bigg| \geq c_2 \|v_2\|
    \end{equation}

    We mixed up the $c_1, c_2$ labels, but the idea is the same. 
    \begin{equation}
      c_2 \|v\|+2 \leq |v| \leq c_1 \|v\|_2 
    \end{equation}
  \end{proof}

  This establishes that we know the metric structure of $\mathbb{R}^n$ very well, and so $\mathbb{R}^n$ is pretty boring. Unless you're in stats and you study very high-dimensional analysis. 

  \begin{corollary}[1.2.6]
    Every finite dimensional normed vector space is Banach. 
  \end{corollary}
  \begin{proof}
    Every finite-dimensional vector space is isomorphic to $\mathbb{R}^n$, which is complete under all norms. 
  \end{proof}

  \begin{corollary}[1.2.7]
    Let $X$ be a normed vector space. Then every finite-dimensional subspace is closed. 
  \end{corollary}
  \begin{proof}
    Let $Y \subset X$ be finite dimensional. Then $Y$ is complete, hence is closed. 
  \end{proof}

  \begin{corollary}
    Let $X$ be a finite-dimensional normed vector space. Then $K \subset X$ is compact iff it is closed and bounded. 
  \end{corollary}

  \begin{corollary}[1.2.9]
    Let $X, Y$ be two normed vector spaces, with $\dim(X) < +\infty$. Then every linear function $f: X \to Y$ is bounded. 
  \end{corollary}
  \begin{proof}
    Let $A: X \to Y$ be linear. Define the graph norm on $X$ to be 
    \begin{equation}
      |w|_A \coloneqq |w|_X + |Aw|_Y
    \end{equation}
    But this is equivalent to Euclidean norm since all norms on finite-dimensional vector spaces are equivalent. So $\exists c_1$ s.t. 
    \begin{equation}
      |w|_X + |Aw|_Y \leq c_1 |w|_X \implies |Aw|_Y \leq c_1 |w|_X
    \end{equation}
  \end{proof}

  So it tells us that any linear map from a finite-dimensional vector space to any vector space is continuous. 

  \begin{theorem}[1.2.11]
    Let $X$ be a normed vector space, $B \subset X$ denote the unit ball, $S \subset X$ denote the unit sphere, then TFAE. 
    \begin{enumerate}
      \item $\dim(X) < +\infty$. 
      \item $B$ is compact. 
      \item $S$ is compact. 
    \end{enumerate}
  \end{theorem}
    

  If this is Hilbert space, then the proof is trivial. But this is hard in Banach spaces. It's analogous to doing linear vs nonlinear PDEs. 

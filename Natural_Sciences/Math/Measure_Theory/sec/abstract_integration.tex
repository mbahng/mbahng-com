\section{Integration over General Measure Spaces}

\subsection{Radon-Nikodym Theorem} 

  Let $(X, \mathcal{M})$ be a measure space. For measure $\mu$ on $(X, \mathcal{M})$ and $f$ a nonnegative function on $X$ that is measurable w.r.t. $\mathcal{M}$, define the set function $\nu$ on $\mathcal{M}$ by 
  \begin{equation}
    \nu(E) \coloneqq \int_E f \, d\mu
  \end{equation}
  Then, by additivity of integration and by MCT, this is indeed a measure. 

  \begin{definition}[Absolutely Continuous Measures]
    On a given measurable space $(X, \mathcal{M})$, a measure $\nu$ is said to be \textbf{absolutely continuous} w.r.t. the measure $\mu$ if for all $E \in \mathcal{M}$, 
    \begin{equation}
      \mu(E) = 0 \implies \nu(E) = 0
    \end{equation}
    So therefore, $\mu$ acts as an upper bound of $\nu$ over $0$ measure sets. 
  \end{definition}

  \begin{theorem}
    Let $(X, \mathcal{M}, \mu)$ be a measure space and $\nu$ a finite easure on the measurable space $(X, \mathcal{M})$. Then $\nu$ is absolutely continuous w.r.t. $\mu$ iff for each $\epsilon > 0$, $\exists \delta > 0$ s.t. for all $E \in \mathcal{M}$, 
    \begin{equation}
      \mu(E) < \delta \implies \nu(E) < \epsilon 
    \end{equation}
  \end{theorem}

  \begin{theorem}[Radon-Nikodym Theorem]
    Let $(X, \mathcal{M}, \mu)$ be a $\sigma$-finite measure space and $\nu$ a $\sigma$-finite measure defined on the measurable space $(X, \mathcal{M})$ that is absolutely continuous w.r.t. $\mu$. Then there is a nonnegative function $f$ on $X$ that is measurable w.r.t. $\mathcal{M}$ for which 
    \begin{equation}
      \nu(E) = \int_E f \,d \mu
    \end{equation}
    for all $E \in \mathcal{M}$. Furthermore, $f$ is unique in the sense that if $g$ is any nonnegative measurable function on $X$ that also has this property, then $g = f$ $\mu$-a.e. 
  \end{theorem}
  \begin{proof}
    
  \end{proof}

  \begin{corollary}
    Let $(X, \mathcal{M}, \mu)$ be a $\sigma$-finite measure space and $\nu$ a finite signed measure on the measurable space $(X, \mathcal{M})$ that is AC w.r.t. $\mu$, then there is a function $f$ that is integrable over $X$ w.r.t. $\mu$, and 
    \begin{equation}
      \nu (E) = \int_E f \, d\mu 
    \end{equation}
    for all $E \in \mathcal{M}$. 
  \end{corollary}

  \begin{theorem}[Lebesgue Decomposition Theorem]
    Let $(X, \mathcal{M}, \mu)$ be a $\sigma$-finite measure space and $\nu$ a $\sigma$-finite measure on the measurable space $(X, \mathcal{M})$. Then there is a measure $\nu_0$ on $\mathcal{M}$, singular w.r.t. $\mu$, and a measure $\nu_1$ on $\mathcal{M}$---AC w.r.t. $\mu$---for which $\nu = \nu_0 + \nu_1$. The measures $\nu_0, \nu_1$ are unique. 
  \end{theorem}
  \begin{proof}
    
  \end{proof}

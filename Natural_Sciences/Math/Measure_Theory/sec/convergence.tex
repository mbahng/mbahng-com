\section{Convergence} 

  \begin{definition}[Convergence in Measure]
    \textbf{$f_n \to f$ in measure} if for every $\eta > 0$, 
    \begin{equation}
      \lim_{n \to \infty} m \big( \{x \mid |f_n (x) - f(x)|  > \eta \}\big) = 0
    \end{equation}
  \end{definition}

  So we have 3 types of convergence: uniform convergence, a.e. convergence, and now convergence in measure. Now we want to relate this convergence to the ones we already have. 

  \begin{theorem}
    Suppose $E$ is measurable, $m(E) < +\infty$, and $f_n \to f$ a.e. in $E$ (assume $f_n$ all measurable). Then , $f_n \to f$ in measure. 
  \end{theorem}
  \begin{proof}
    Observe that if $f_n \to f$ uniformly, then it converges in measure, because given some $\eta > 0$, $\exists N$ s.t. 
    \begin{equation}
      \{ x \mid |f_n (x) - f(x) | > \eta \} = \emptyset
    \end{equation}
    by definition. It doesn't go to $0$; it is $0$. You can guess why we started with this, because now we can directly use Egorov's theorem. Fix any $\epsilon > 0$. Find $E_0 \subset E$ s.t. $m (E \setminus E_0) < \epsilon$, and $f_n \to f$ uniformly on $E_0$. It follows that for all $\eta > 0$, 
    \begin{equation}
      m(\{x \mid |f_n (x) - f(x)| > \eta\}) \leq \epsilon 
    \end{equation}
    for all $n \geq N(\eta)$. Since this is true for every $\epsilon > 0$, so this implies
    \begin{equation}
      \lim_{n \to \infty} m(\{ x \mid |f_n (x) - f(x)| > \eta\}) = 0
    \end{equation} 
    for every $\eta > 0$. 
  \end{proof}

  A few remarks. First, if the measure of $E$ is infinite, this need not be true. Consider $f_n (x) = \chi_{[n, n+1]} (x)$. Then, this converges to $0$ pointwise, but it does not converge to $0$ in measure. There is always a measure $1$ set where $f$ is $1$. Where the proof breaks down is in Egorov's theorem, since it does not work when $m(E) = +\infty$. 

  The second remark is that the converse is not true. Consider $[0, 1]$ and the sequence of functions 
  \begin{equation}
    \chi_{[0, 1/2]}, \chi_{[1/2, 1]}, \chi_{[0, 1/4]}, \chi_{[1/4, 1/2]}, \chi_{[1/2, 3/4]}, \ldots 
  \end{equation}
  Then $f_n \to 0$ in measure since the size shrinks at the rate of $2^{-n}$. However, it doesn't converge a.e. since for any point $x \in [0, 1]$, the function will be $1$ eventually as we hit the subinterval containing $x$, like ``waves.'' So $f_n(x)$ diverges for all $x \in [0, 1]$. So indeed, convergence in measure is the weakest type of convergence. 

  Here is a sort-of converse. 

  \begin{theorem}[Riesz]
    Suppose $f_n \to f$ in measure. Then, there exists a subsequence $f_{n_k} \to f$ a.e. 
  \end{theorem}
  \begin{proof}
    For every $k$, find $n_k$ s.t. for all $n \geq n_k$,
    \begin{equation}
      m(\underbrace{\{x \mid |f_n(x) - f(x)| > 1/k\}}_{E_k}) < 2^{-k}
    \end{equation}
    Then, 
    \begin{equation}
      \sum_{k=1}^\infty m(E_k) < +\infty
    \end{equation}
    By Borel-Cantelli, the set of all $x$'s that are in infinitely many $E_k$ have measure $0$. So, almost everywhere, $x$ is only in a finite number of $E_k$. So for a.e., $x$, there exists $N(x)$ s.t. $x \not\in E_k$ for all $k \geq N(x)$. This means 
    \begin{equation}
      | f_{n_k} (x) - f(x)| < 1/k 
    \end{equation}
    for all $k \geq N(x)$. Therefore, $f_{n_k} (x) \to f(x)$ for a.e. $x$. 
  \end{proof}

  In the example above, we can just skip the functions that evaluate $x$ to $1$.  

  Practically, proving convergence in measure is still pretty good since we can pass in a subsequence that converges a.e. Here is a corollary. 

  \begin{corollary}
    Let $f_n \geq 0$, integrable on $E$. Then, 
    \begin{equation}
      \lim_{n \to +\infty} \int_E f_n \,dx = 0  \iff f_n \to 0 \text{ in measure}
    \end{equation}
    $f_n$ are tight and uniformly integrable. 
  \end{corollary}
  \begin{proof}
    We prove bidirectionally. 
    \begin{enumerate}
      \item $(\rightarrow)$. Tight, uniformly integrable is true be definition. Also, $f_n \to 0$ in measure by Chebyshev. 
        \begin{equation}
          m(\{x \mid f_n (x) > \eta\}) \leq \frac{1}{\eta} \int_E f_n \,dx 
        \end{equation}
      \item $(\leftarrow)$ For the opposite, we use the previous theorem. Find $f_{n_k}$ s.t. that it converges to $0$ a.e., and then use Vitali's convergence theorem. 
    \end{enumerate}
  \end{proof}

  In general, if $f_n \to 0$ in measure, it doesn't mean that the integral will go to $0$ since you can take larger and larger bumps. So we need extra assumptions. 

  \begin{lemma} 
    Suppose $f$ is bounded, and there exists measurable sequences of functions $\phi_n, \psi_n$ s.t. 
    \begin{equation}
      \psi_n (x) \leq f(x) \leq \psi_n(x) \quad \forall x \in E
    \end{equation}
    and 
    \begin{equation}
      \lim_{n \to +\infty} \int_E (\psi_n - \phi_n) = 0
    \end{equation}
    Then, there exists $\Tilde{phi}_n \to f$ and $\Tilde{psi}_n \to f$ a.e. 
  \end{lemma}
  \begin{proof}
    Define 
    \begin{equation}
      \Tilde{\phi}_n (x) = \max\{\phi_1 (x) , \ldots, \phi_n (x) \}, \quad \Tilde{\psi}_n (x) = \min\{\psi_1 (x) , \ldots, \psi_n (x) \}
    \end{equation}
    We still have $\Tilde{\phi}_n (x) \leq f(x) \leq \Tilde{\phi}_n (x)$ for all $n$ and for all $x$. Also, $\Tilde{\phi}_n (x)$ is increasing, $\Tilde{\psi}_n (x)$ is decreasing. Now define 
    \begin{equation}
      \phi^\ast (x) \coloneqq \lim_{n \to \infty} \Tilde{\phi}_n (x), \qquad \psi^\ast (x) \coloneqq \lim_{n \to \infty} \Tilde{\psi}_n (x)
    \end{equation}
    Observe that 
    \begin{equation}
      \int (\Tilde{\psi}_n - \Tilde{\phi}_n) \leq \int (\psi_n - \phi_n) \implies \int (\Tilde{\psi}_n - \Tilde{\phi}_n) \to 0 \text{ as } n \to \infty 
    \end{equation}
    Also, 
    \begin{equation}
      \int (\underbrace{\psi^\ast (x) - \phi^\ast(x)}_{\geq 0}) \,dx \leq \int (\Tilde{\psi}^\ast - \Tilde{\phi}^\ast) 
    \end{equation}
    for all $n$. Therefore, 
    \begin{equation}
      \int (\psi^\ast (x) - \phi^\ast (x)) = 0 \implies \psi^\ast (x) = \phi^\ast (x) \text{ a.e.}
    \end{equation}
    And so $f(x)$, which is sandwiched between $\psi^\ast$ and $\phi^\ast$, must be equal a.e. We didn't assume that $f$ was measurable, but these $\psi_n, \phi_n$ is measurable by assumption. 
  \end{proof}

  Now, we can prove this master theorem. 

  \begin{theorem}[Characterization of Lebesgue Integrability]
    Let $f$ be bounded on measurable set $E$ of finite measure. Then $f$ is Lebesgue integrable iff $f$ is measurable. 
  \end{theorem}
  \begin{proof}
    The backward implication is true in general. We want to show that $f$ is measurable. Recall that for bounded functions, we defined Lebesgue integrals with $\underline{L}f$ and $\overline{L}f$. Therefore, we can find simple $\phi_n, \psi_n$ s.t. $\phi_n \leq f \leq \psi_n$, and $\int \psi_n - \int \phi_n \leq 1/n$. Now we are exactly in the setting of the lemma, and so by the lemma, we can find measurable $\Tilde{\psi}_n (x) \to f$ a.e. (in fact, $\Tilde{\psi}$ will be simple). Since the limit of measurable functions is measurable, $f$ is measurable.
  \end{proof}

  This is a very reasonable criterion, and you can't really hope for more then Lebesgue measurability. This following theorem on Riemmann integrability is much more restrictive, while for above, measurable functions can be very wild. 

  \begin{theorem}[Characterization of Riemann Integrability]
    $f$ is Riemann integrable on $[a, b]$ if the set of its discontinuities has measure $0$. 
  \end{theorem}
  \begin{proof}
    Not stated. In book. 
  \end{proof}




\section{Convergence} 

  \begin{definition}[Convergence in Measure]
    Let $(f_n)$ be a sequence of measurable and finite\footnote{We add finite condition since we avoid dealing with $+\infty - \infty$.} a.e. \textbf{$f_n \to f$ in measure} if for every $\eta > 0$, 
    \begin{equation}
      \lim_{n \to \infty} m \big( \{x \mid |f_n (x) - f(x)|  > \eta \}\big) = 0
    \end{equation}
    Colloquially, the set over which $f_n$ and $f$ differ too much is small. 
  \end{definition}

  So we have 3 types of convergence: uniform convergence, a.e. convergence, and now convergence in measure. Now we want to relate this convergence to the ones we already have. 

  \begin{theorem}
    Suppose $E$ is measurable, $m(E) < +\infty$, and $f_n \to f$ a.e. in $E$ (assume $f_n$ all measurable). Then , $f_n \to f$ in measure. 
  \end{theorem}
  \begin{proof}
    Observe that if $f_n \to f$ uniformly, then it converges in measure, because given some $\eta > 0$, $\exists N$ s.t. 
    \begin{equation}
      \{ x \mid |f_n (x) - f(x) | > \eta \} = \emptyset
    \end{equation}
    by definition. It doesn't go to $0$; it is $0$. You can guess why we started with this, because now we can directly use Egorov's theorem. Fix any $\epsilon > 0$. Find $E_0 \subset E$ s.t. $m (E \setminus E_0) < \epsilon$, and $f_n \to f$ uniformly on $E_0$. It follows that for all $\eta > 0$, 
    \begin{equation}
      m(\{x \mid |f_n (x) - f(x)| > \eta\}) \leq \epsilon 
    \end{equation}
    for all $n \geq N(\eta)$. Since this is true for every $\epsilon > 0$, so this implies
    \begin{equation}
      \lim_{n \to \infty} m(\{ x \mid |f_n (x) - f(x)| > \eta\}) = 0
    \end{equation} 
    for every $\eta > 0$. 
  \end{proof}

  A few remarks. First, if the measure of $E$ is infinite, this need not be true. Consider $f_n (x) = \chi_{[n, n+1]} (x)$. Then, this converges to $0$ pointwise, but it does not converge to $0$ in measure. There is always a measure $1$ set where $f$ is $1$. Where the proof breaks down is in Egorov's theorem, since it does not work when $m(E) = +\infty$. 

  The second remark is that the converse is not true. Consider $[0, 1]$ and the sequence of functions 
  \begin{equation}
    \chi_{[0, 1/2]}, \chi_{[1/2, 1]}, \chi_{[0, 1/4]}, \chi_{[1/4, 1/2]}, \chi_{[1/2, 3/4]}, \ldots 
  \end{equation}
  Then $f_n \to 0$ in measure since the size shrinks at the rate of $2^{-n}$. However, it doesn't converge a.e. since for any point $x \in [0, 1]$, the function will be $1$ eventually as we hit the subinterval containing $x$, like ``waves.'' So $f_n(x)$ diverges for all $x \in [0, 1]$. So indeed, convergence in measure is the weakest type of convergence. 

  Here is a sort-of converse. 

  \begin{theorem}[Riesz]
    Suppose $f_n \to f$ in measure. Then, there exists a subsequence $f_{n_k} \to f$ a.e. 
  \end{theorem}
  \begin{proof}
    For every $k$, find $n_k$ s.t. for all $n \geq n_k$,
    \begin{equation}
      m(\underbrace{\{x \mid |f_n(x) - f(x)| > 1/k\}}_{E_k}) < 2^{-k}
    \end{equation}
    Then, 
    \begin{equation}
      \sum_{k=1}^\infty m(E_k) < +\infty
    \end{equation}
    By Borel-Cantelli, the set of all $x$'s that are in infinitely many $E_k$ have measure $0$. So, almost everywhere, $x$ is only in a finite number of $E_k$. So for a.e., $x$, there exists $N(x)$ s.t. $x \not\in E_k$ for all $k \geq N(x)$. This means 
    \begin{equation}
      | f_{n_k} (x) - f(x)| < 1/k 
    \end{equation}
    for all $k \geq N(x)$. Therefore, $f_{n_k} (x) \to f(x)$ for a.e. $x$. 
  \end{proof}

  In the example above, we can just skip the functions that evaluate $x$ to $1$. 

  Practically, proving convergence in measure is still pretty good since we can pass in a subsequence that converges a.e. Here is a corollary. 

  \begin{corollary}
    Let $f_n \geq 0$, integrable on $E$. Then, 
    \begin{equation}
      \lim_{n \to +\infty} \int_E f_n \,dx = 0  \iff f_n \to 0 \text{ in measure}
    \end{equation}
    $f_n$ are tight and uniformly integrable. 
  \end{corollary}
  \begin{proof}
    We prove bidirectionally. 
    \begin{enumerate}
      \item $(\rightarrow)$. Tight, uniformly integrable is true be definition. Also, $f_n \to 0$ in measure by Chebyshev. 
        \begin{equation}
          m(\{x \mid f_n (x) > \eta\}) \leq \frac{1}{\eta} \int_E f_n \,dx 
        \end{equation}
      \item $(\leftarrow)$ For the opposite, we use the previous theorem. Find $f_{n_k}$ s.t. that it converges to $0$ a.e., and then use Vitali's convergence theorem. 
    \end{enumerate}
  \end{proof}

  In general, if $f_n \to 0$ in measure, it doesn't mean that the integral will go to $0$ since you can take larger and larger bumps. So we need extra assumptions. 

\subsection{Exercises}

  \begin{exercise}[Math 631 Fall 2025, Final Exam Exercise 1]
    Let $f_{n}(x)=\frac{nx}{1+n^{2}x^{4}}$ defined on $E=(0,1)$. Does $f_{n}$ converge in measure? a.e.? uniformly? Does there exist integrable $g(x)$ such that $|f_{n}(x)|\le g(x)$ for all $n$?
  \end{exercise}
  \begin{solution}
    For each $x$, $f_{n}(x)\rightarrow0$ as $n\rightarrow\infty$, so $f_{n}(x)$ converges a.e. and in measure since a.e. convergence implies convergence in measure. Observe that
    \begin{equation}
      f_{n}^{\prime}(x)=\frac{n(1+n^{2}x^{4})-nx4n^{2}x^{3}}{(1+n^{2}x^{4})^{2}}.
    \end{equation}
    The numerator is equal to zero if $n-3n^{3}x^{4}=0$; the only root on $(0,1)$ is $x=3^{-1/4}n^{-1/2}$. This suggests that we should look at $f_{n}(x)$ for $x\sim n^{-1/2}$ to get an idea of its maximal value. Indeed,
    \begin{equation}
      f_{n}(n^{-1/2})=\frac{1}{2}n^{1/2},
    \end{equation}
    and so $f_{n}$ does not converge to 0 uniformly (and can't converge to anything else due to pointwise convergence to zero). Also, given $x$ close to zero, set $n=\lfloor\frac{1}{x^{2}}\rfloor$. Then
    \begin{equation}
      f_{n}(x)\ge\frac{1}{2}x\lfloor\frac{1}{x^{2}}\rfloor\ge\frac{1}{2x}+\frac{x}{2}(\lfloor\frac{1}{x^{2}}\rfloor-\frac{1}{x^{2}})\ge\frac{1}{2x}-\frac{x}{2}.
    \end{equation}
    So if $x$ is sufficiently small, then $f_{n}(x)\ge\frac{1}{4x}$. This estimate implies that $g(x)\ge\frac{1}{4x}$ for all sufficiently small $x$, and so cannot be integrable.
  \end{solution}

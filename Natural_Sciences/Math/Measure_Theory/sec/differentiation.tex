\section{Differentiation}

  Now, we will establish differentiation and culminate in the fundamental theorem of calculus. Monotone functions are a nice class of functions to study for differentiation and for constructing more general measures. 

  \begin{theorem}
    Suppose $f$ is monotone, increasing on $[a, b]$. Then, the set of discontinuities of $f$ at most countable. 
  \end{theorem}
  \begin{proof}
    Let $x_k$ be any point of discontinuity. Note that 
    \begin{equation}
      \lim_{x \to x_k^-} f(x), \qquad \lim_{x \to x_k^+} f(x)
    \end{equation}
    both exist by monotonicity, but since there is a discontinuity, we have 
    \begin{equation}
      L_k^- = \lim_{x \to x_k^-} f(x) < \lim_{x \to x_k^+} f(x) = L_k^+
    \end{equation}
    Then, $L_k^+ - L_k^-$ is a jump of $f$ at $x_k$. These intervals $[L_k^-, L_k^+]$ are disjoint due to monotonicity, and each interval contains a rational number. So there can only be at most countable intervals. 
  \end{proof}

  One piece of info from this trick. 

  \begin{definition}
    A point $x$ is a discontinuity of the first kind of $f(x)$ if both one-sided limits exist. 
  \end{definition}

  Now here's a generalization for not necessarily monotone functions. 

  \begin{theorem}[Detour]
    The set of discontinuities of the first kind is countable. 
  \end{theorem}
  \begin{proof}
    Idea of the proof. Look at some jump discontinuity and record the jump $\eta > 0$. Then, find $\delta > 0$ s.t. if $0 < y - x < \delta$, then 
    \begin{equation}
      \big| f(x)  - \lim_{y \to x^+} f(y) \big|  < \frac{\eta}{10}
    \end{equation}
    Then look at the rectangle on the graph associated with each jump. Because the limits exist, you can pick the rectangles so small that they are completely disjoint. Look at picture. 
  \end{proof}

  Now back to monotone functions. 

  \begin{theorem}
    For any countable set $C \subset (a, b)$ (where the interval doesn't need to be bounded), there exists monotonically increasing $f$ with a jump at each $x \in C$ and continuous at every $x \not\in C$. 
  \end{theorem}
  \begin{proof}
    Let $x_1, x_2, \ldots$ be $C$, and define 
    \begin{equation}
      f(x) = \sum_{x_k \leq x, x \in C} 2^{-k}
    \end{equation}
    The sum is increasing and convergent (since it's dominated by geometric series). $f$ also has a jump of $2^{-k}$ at every $x_k$. 

    Now we prove continuity. Suppose $x \not\in C$. Take $N \in \mathbb{N}$. Find $\delta_N > 0$ s.t. 
    \begin{equation}
      x_1, x_2, \ldots, x_N \not\in (x - \delta_N, x + \delta_N)
    \end{equation}
    which is possible since this is a finite set. The remaining sum can only add up to $2^{-N}$, and so $f(x + \delta_N) - f(x - \delta_N) \leq 2^{-N}$. 
  \end{proof}




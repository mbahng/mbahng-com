\section{Integration}

\subsection{Radon-Nikodym Theorem} 

  Let $(X, \mathcal{M})$ be a measure space. For measure $\mu$ on $(X, \mathcal{M})$ and $f$ a nonnegative function on $X$ that is measurable w.r.t. $\mathcal{M}$, define the set function $\nu$ on $\mathcal{M}$ by 
  \begin{equation}
    \nu(E) \coloneqq \int_E f \, d\mu
  \end{equation}
  Then, by additivity of integration and by MCT, this is indeed a measure. 

  \begin{definition}[Absolutely Continuous Measures]
    On a given measurable space $(X, \mathcal{M})$, a measure $\nu$ is said to be \textbf{absolutely continuous} w.r.t. the measure $\mu$ if for all $E \in \mathcal{M}$, 
    \begin{equation}
      \mu(E) = 0 \implies \nu(E) = 0
    \end{equation}
    So therefore, $\mu$ acts as an upper bound of $\nu$ over $0$ measure sets. 
  \end{definition}

  \begin{theorem}
    Let $(X, \mathcal{M}, \mu)$ be a measure space and $\nu$ a finite easure on the measurable space $(X, \mathcal{M})$. Then $\nu$ is absolutely continuous w.r.t. $\mu$ iff for each $\epsilon > 0$, $\exists \delta > 0$ s.t. for all $E \in \mathcal{M}$, 
    \begin{equation}
      \mu(E) < \delta \implies \nu(E) < \epsilon 
    \end{equation}
  \end{theorem}

  \begin{theorem}[Radon-Nikodym Theorem][def:rn-der]
    Let $(X, \mathcal{M}, \mu)$ be a $\sigma$-finite measure space and $\nu$ a $\sigma$-finite measure defined on the measurable space $(X, \mathcal{M})$ that is absolutely continuous w.r.t. $\mu$. Then there is a nonnegative function $f$ on $X$---called the \textbf{Radon-Nikodym derivative}---that is measurable w.r.t. $\mathcal{M}$ for which 
    \begin{equation}
      \nu(E) = \int_E f \,d \mu
    \end{equation}
    for all $E \in \mathcal{M}$. Furthermore, $f$ is unique in the sense that if $g$ is any nonnegative measurable function on $X$ that also has this property, then $g = f$ $\mu$-a.e. 
  \end{theorem}
  \begin{proof}
    
  \end{proof}

  \begin{corollary}
    Let $(X, \mathcal{M}, \mu)$ be a $\sigma$-finite measure space and $\nu$ a finite signed measure on the measurable space $(X, \mathcal{M})$ that is AC w.r.t. $\mu$, then there is a function $f$ that is integrable over $X$ w.r.t. $\mu$, and 
    \begin{equation}
      \nu (E) = \int_E f \, d\mu 
    \end{equation}
    for all $E \in \mathcal{M}$. 
  \end{corollary}

  \begin{theorem}[Lebesgue Decomposition Theorem]
    Let $(X, \mathcal{M}, \mu)$ be a $\sigma$-finite measure space and $\nu$ a $\sigma$-finite measure on the measurable space $(X, \mathcal{M})$. Then there is a measure $\nu_0$ on $\mathcal{M}$, singular w.r.t. $\mu$, and a measure $\nu_1$ on $\mathcal{M}$---AC w.r.t. $\mu$---for which $\nu = \nu_0 + \nu_1$. The measures $\nu_0, \nu_1$ are unique. 
  \end{theorem}
  \begin{proof}
    
  \end{proof}

\subsection{Fubini's and Tonelli's Theorem}

  \begin{theorem}[Fubini's Theorem]
    Let $(X, \mathcal{A}, \mu)$, $(Y, \mathcal{B}, \nu)$ be two measure spaces and $\nu$ be complete. Let $f$ be integrable over $X \times Y$ w.r.t. to the product measure $\mu \times \nu$. Then for almost all $x \in X$, the $x$-section of $f$, $f(x, \cdot)$, is integrable over $Y$ with respect to $\nu$, and 
    \begin{equation}
      \int_{X \times Y} f \, d(\mu \times \nu) = \int_X \bigg[ \int_Y f(x, y) \,d\nu(y) \bigg] \, d\mu(x)
    \end{equation}
  \end{theorem}

  \begin{theorem}[Tonelli's Theorem]
    Let $(X, \mathcal{A}, \mu)$, $(Y, \mathcal{B}, \nu)$ be two measure spaces and $\nu$ be complete.Let $f$ be a nonnegative $(\mu \times \nu)$-measurable function on $X \times Y$. Then, 
    \begin{enumerate}
      \item For almost all $x \in X$, the $x$-section of $f$, $f(x, \cdot)$, is $\nu$-measurable. 
      \item The function defined almost everywhere on $x$ by 
        \begin{equation}
          x \mapsto \int_Y f(x, y) \,d\nu(y)
        \end{equation}
        is $\mu$-measurable. 
      \item Finally, 
        \begin{equation}
          \int_{X \times Y} f \,d(\mu \times \nu) = \int_X \bigg[ \int_Y f(x, y) \,d\nu(y) \bigg] \,d \mu(x)
        \end{equation}
    \end{enumerate}
  \end{theorem}

\subsection{Exercises} 

  \begin{exercise}[Math 631 Fall 2025, Final Exam Exercise 4]
    Let $A=[-1,1]\times[-1,1]$. Let $f(x,y)=\frac{xy}{(x^{2}+y^{2})^{2}}$ (when $x=0$ or $y=0$ we set $f(x,y)=0$). Prove that the iterated integrals exist and are equal:
    \begin{equation}
      \int_{-1}^{1}(\int_{-1}^{1}f(x,y)dx)dy=\int_{-1}^{1}(\int_{-1}^{1}f(x,y)dy)dx,
    \end{equation}
    but the double integral $\int_{A}f(x,y) dxdy$ does not exist.
  \end{exercise}
  \begin{solution}
    Note that $f$ is odd with respect to both $x$ and $y$. Therefore
    \begin{equation}
      \int_{-1}^{1}f(x,y)dx = \int_{-1}^{1}f(x,y)dy=0
    \end{equation}
    for all $y$ and $x$ respectively. Hence both iterated integrals are zero. Now passing to polar coordinates, we have that $|f(r,\theta)|=\frac{|\sin 2\theta|}{2r^{2}}$. Therefore, for every $\epsilon>0$,
    \begin{equation}
      \int_{A}|f(x,y)|dxdy\ge\int_{\epsilon}^{1}\int_{0}^{\pi/2}\frac{\sin 2\theta}{r^{2}}rd\theta dr\ge c\int_{\epsilon}^{1}r^{-1}dr.
    \end{equation}
    Since $\int_{\epsilon}^{1}r^{-1}dr$ diverges as $\epsilon\rightarrow0$, $f$ is not integrable over $A$.
  \end{solution}


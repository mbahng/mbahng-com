\section{Jordan Measure}

  We would like to generalize the concepts of size, which are specified as length/area/volume depending on the dimension of the space we live in. The most intuitive notion of size are those of segments, rectangles, and boxes, and these are the simplest forms of sets that we will work with. 

  Such that for ``simple'' sets $A$ where we know what the area is, the outer measure of $A$ should coincide with the area of $A$. Let's first start by defining what a ``simple'' set is. 

  \begin{definition}[Box] 
    An \textbf{box} $E \subset \mathbb{R}^n$ is defined recursively as follows. 
    \begin{enumerate}
      \item An \textbf{interval} $I \subset \mathbb{R}$ is one of the sets $(a, b), [a, b), (a, b], [a, b]$ for $a, b \in \mathbb{R}$. 
      \item For $n > 1$, an box $E \subset \mathbb{R}^n$ is $E = I_1 \times \ldots \times I_n$ for intervals $I_1, \ldots, I_k$. 
    \end{enumerate}
  \end{definition} 

  \begin{definition}[Size of a Box]
    The \textbf{size} of a box $E = I_1 \times \ldots \times I_n \subset \mathbb{R}^n$ is defined as follows. 
    \begin{enumerate}
      \item The \textbf{length} of an interval $I$ is $\ell(I) \coloneqq b - a$. 
      \item The \textbf{size} of $E$ is $|E| \coloneqq \prod_{i=1}^n (b_i - a_i)$. 
    \end{enumerate}
  \end{definition} 

  Now we can combine these to get an elementary set. 

  \begin{definition}[Elementary Set]
    An \textbf{elementary set} is a set $E \subset \mathbb{R}^n$ that is a finite union of boxes. 
  \end{definition}

  We would like to have some nice properties of these elementary sets. 

  \begin{lemma}[Boolean Closure of Elementary Sets]
    Given two elementary sets $E, F \subset \mathbb{R}^n$, 
    \begin{enumerate}
      \item $E \cup F$ is elementary. 
      \item $E \cap F$ is elementary. 
      \item $E \setminus F$ is elementary. 
    \end{enumerate}
  \end{lemma}
  \begin{proof}
    
  \end{proof}

  \begin{lemma}[Disjoint Finite Union of Elementary Sets]
    Let $E \subset \mathbb{R}^n$ be an elementary set. Then, $E$ can be expressed as a finite union of \textit{disjoint} boxes. 
  \end{lemma}
  \begin{proof}
    
  \end{proof}

  \begin{definition}[Elementary Measure]
    The \textbf{elementary measure} of an elementary set $E \subset \mathbb{R}^n$ is defined as the sum of the sizes of each box in a partition: 
    \begin{equation}
      m(E) \coloneqq \sum_{i=1}^k |B_k|
    \end{equation}
    We claim that this sum is invariant depending on the partition, and hence, well defined. 
  \end{definition}
  \begin{proof}
    
  \end{proof}

  This elementary measure clearly extends the notion of size, since 
  \begin{equation}
    m(B) = s(B)
  \end{equation}
  whenever $B$ is elementary. Furthermore, we can deduce finite additivity and nonnegativity. These are really trivial but we state them as theorems to establish a pattern. 

  \begin{lemma}[Fundamental Properties of Elementary Measure]
    The elementary measure satisfies the following. 
    \begin{enumerate}
      \item \textit{Nonnegativity}. For any elementary set $E$, $m(E) \geq 0$. 
      \item \textit{Finite Additivity} Given $E_1, \ldots, E_n$ are disjoint elementary sets, 
      \begin{equation}
        m(E_1 \cup \ldots \cup E_k) = m(E_1) + \ldots + m(E_k)
      \end{equation}

      \item \textit{Monotonicity}. Given elementary sets $E \subset F$, we have 
      \begin{equation}
        m(E) \leq m(F)
      \end{equation}

      \item \textit{Finite Subadditivity}. Let $E_1, \ldots, E_n$ be any elementary sets (not necessarily disjoint). Then 
      \begin{equation}
        m(E_1 \cup \ldots \cup E_n) \leq m(E_1) + \ldots + m(E_n)
      \end{equation}

      \item \textit{Translation Invariance}. For $x \in \mathbb{R}^n$ and elementary set $E \subset \mathbb{R}^n$, 
      \begin{equation}
        m(E) = m(x + E) 
      \end{equation}
    \end{enumerate}
  \end{lemma}

  It turns out that these properties uniquely determine an elementary measure. 

  \begin{theorem}[Uniqueness of Elementary Measure]
    
  \end{theorem}

\subsection{Jordan Measure}

  Now, we define the outer and inner measure, which are defined for \textit{all} subsets of $\mathbb{R}^n$. 

  \begin{definition}[Jordan Outer, Inner Measure]
    Let $E \subset \mathbb{R}^n$. 
    \begin{enumerate}
      \item The \textbf{Jordan inner measure} is defined 
      \begin{equation}
        m_\ast (E) \coloneqq \sup_{A \subset E, A \text{ elementary}} m(A)
      \end{equation}

      \item The \textbf{Jordan outer measure} is defined 
      \begin{equation}
        m_\ast (E) \coloneqq \inf_{B \supset E, B \text{ elementary}} m(B)
      \end{equation}
    \end{enumerate}
    Note that if $E$ is unbounded, then there exists no elementary set that is a superset of $E$, and so the infimum of such a set is $+\infty$ conventionally. 
  \end{definition}

  This is where our first big leap in construction comes in. Before, we have defined elementary boxes, which are pretty much guaranteed to have a well-defined elementary measure. Here, we \textit{begin} with a function on the power set of $\mathbb{R}^n$, and then we will filter the power set to those subsets that behave nicely. 

  \begin{definition}[Jordan Measurable Set, Jordan Measure]
    Let $E \subset \mathbb{R}^n$ be bounded.\footnote{Note that by convention, we don't consider unbounded sets to be Jordan measurable.} If $m_\ast (E) = m^\ast (E)$, then $E$ is said to be \textbf{Jordan-measurable}, and we define 
    \begin{equation}
      m(E) \coloneqq m_\ast (E) = m^\ast (E)
    \end{equation}
    as the \textbf{Jordan measure} of $E$. 
  \end{definition}

  Note first of all that Jordan measure is a generalization of elementary measure, since if $E$ is elementary, then we can set $A = E = B$ to achieve these bounds. Furthermore, by monotonicity, we can never get past them, and will always have 
  \begin{equation}
    m(A) \leq m(E) \leq m(B)
  \end{equation}
  where $m$ is the elementary measure. So, we can overload the notation and just write $m$ to denote elementary and Jordan measure. Second, note that the Jordan measure shares the same properties. 

  \begin{lemma}[Boolean Closure of Jordan Measurable Sets]
    Given two Jordan-measurable sets $E, F \subset \mathbb{R}^n$, 
    \begin{enumerate}
      \item $E \cup F$ is elementary. 
      \item $E \cap F$ is elementary. 
      \item $E \setminus F$ is elementary. 
    \end{enumerate}
  \end{lemma}
  \begin{proof}
    
  \end{proof}

  The properties of the Jordan measure parallel those of elementary measure. 

  \begin{theorem}[Fundamental Properties of Jordan Measure]
    The elementary measure satisfies the following. 
    \begin{enumerate}
      \item \textit{Nonnegativity}. For any elementary set $E$, $m(E) \geq 0$. 
      \item \textit{Finite Additivity} Given $E_1, \ldots, E_n$ are disjoint elementary sets, 
      \begin{equation}
        m(E_1 \cup \ldots \cup E_k) = m(E_1) + \ldots + m(E_k)
      \end{equation}

      \item \textit{Monotonicity}. Given elementary sets $E \subset F$, we have 
      \begin{equation}
        m(E) \leq m(F)
      \end{equation}

      \item \textit{Finite Subadditivity}. Let $E_1, \ldots, E_n$ be any elementary sets (not necessarily disjoint). Then 
      \begin{equation}
        m(E_1 \cup \ldots \cup E_n) \leq m(E_1) + \ldots + m(E_n)
      \end{equation}

      \item \textit{Translation Invariance}. For $x \in \mathbb{R}^n$ and elementary set $E \subset \mathbb{R}^n$, 
      \begin{equation}
        m(E) = m(x + E) 
      \end{equation}
    \end{enumerate}
  \end{theorem}
  \begin{proof}
    
  \end{proof}

  Jordan measurable sets are sets that are ``almost'' elementary, but a few sets already come to mind that are not Jordan measurable. 

  \begin{example}[Rationals in Unit Interval]
    $\mathbb{Q} \cap [0, 1]$ is not Jordan measurable. 
  \end{example} 

  It may be hard to tell directly whether something is Jordan measurable. This is where the ``Cauchy criterion'' of Jordan measurable sets comes in. 

  \begin{theorem}[Equivalent Notions]
    $E$ is Jordan measurable iff $\forall \epsilon > 0$, $\exists$ elementary sets $A \subset E \subset B$ s.t. $m(B \setminus A) < \epsilon$. 
  \end{theorem}
  \begin{proof}
    
  \end{proof}

  Note how the previous lemma is very similar to \hyperref[real-thm:cauchy-riemann-integrability]{this theorem on Riemann integrability}. 

  \begin{example}[Regions Under Graphs are Jordan Measurable]
    
  \end{example}

  \begin{example}[Triangle is Jordan Measurable]
    
  \end{example}

  \begin{example}[Compact Convex Polytopes are Jordan Measurable]
    
  \end{example}

  \begin{example}[Open and Closed Balls in Euclidean Space are Jordan Measurable]
    
  \end{example}

  \begin{example}[Subsets of Jordan Null Sets have 0 Jordan Measure]
    
  \end{example}

  \begin{theorem}[Uniqueness of Jordan Measure]
    
  \end{theorem}

  \begin{theorem}[Topological Approximations of Jordan Measurable Sets]
    Let $E \subset \mathbb{R}^n$ be a bounded set. Then, 
    \begin{enumerate}
      \item $E$ and its closure $\overline{E}$ have the same Jordan outer measure. 
      \item $E$ and its interior $E^\circ$ have the same Jordan outer measure. 
      \item $E$ is Jordan measurable iff the topological boundary $\partial E$ has Jordan outer measure $0$. 
    \end{enumerate}
  \end{theorem}
  \begin{proof}
    
  \end{proof}

  \begin{example}[Bullet Riddled Square]
    Show that both sets have a Jordan inner measure $0$ and Jordan outer measure $1$. 
    \begin{enumerate}
      \item $[0, 1]^2 \setminus \mathbb{Q}^2$. 
      \item $[0, 1]^2 \cap \mathbb{Q}^2$. 
    \end{enumerate}
  \end{example}

  Finally, a little teaser theorem. 

  \begin{theorem}[Caratheodory Property]
    Let $E \subset \mathbb{R}^n$ be a bounded set, and $F \subset \mathbb{R}^n$ be an elementary set. Show that 
    \begin{equation}
      m^\ast (E) = m^\ast (E \cap F) + m^\ast(E \setminus F)
    \end{equation}
    where $m^\ast$ is the Jordan outer measure. 
  \end{theorem}
  \begin{proof}
    
  \end{proof}

\subsection{Riemann Integration} 

  Now, we connect the Riemann integral to the Jordan measure. 

  \begin{theorem}[Jordan Measure with Riemann Integral]
    If $E \subset [a, b]$ is Jordan measurable, then the indicator function $\mathbbm{1}_E$ is Riemann integrable, and 
    \begin{equation}
      \int_a^b \mathbbm{1}_{E} \,dx = m(E)
    \end{equation}
  \end{theorem}



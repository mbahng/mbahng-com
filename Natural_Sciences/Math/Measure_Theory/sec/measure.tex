\section{Measure}

  Now we generalize to abstract measure spaces. Recall how we constructed the Lebesgue measure: (1) We took a set function $\ell$ that assigns lengths to all intervals in $\mathbb{R}$. (2) We have used this length to define the Lebesgue outer measure as an outer approximation of these intervals. (3) We then use the outer measure to define measurable sets with Caratheodory's criterion, which states that measurable sets should split any set nicely into measurable sets. (4) We verify that the collection of all measurable sets is a $\sigma$-algebra. (5) We define the Lebesgue measure as the restriction of the Lebesgue outer measure to the collection of mesaurable sets. This is called the \textit{Caratheodory} construction of Lebesgue measure, and we can generalize this to an abstract space $X$ under certain conditions.  

  First, we want all measurable sets to be a $\sigma$-algebra, which defines all the well-behaved sets. 

  \begin{definition}[Measurable Space] 
    A \textbf{measurable space} is a tuple $(X, \mathcal{M})$ consisting of a set $X$ with a $\sigma$-algebra of subsets of $X$. Elements of $\mathcal{M}$ are called \textbf{measurable sets}. 
  \end{definition}

  \begin{definition}[Measure, Measure Space]
    A \textbf{measure} $\mu$ on a measurable space $(X, \mathcal{M})$ is a set function $\mu: \mathcal{M} \to [0, +\infty]$ satisfying the following. 
    \begin{enumerate}
      \item \textit{Null empty set}. $\mu(\emptyset) = 0$. 
      \item \textit{Countable Additivity}. For all countable collections $\{A_k\}_{k=1}^\infty$ of pairwise disjoint\footnote{Disjointness is clearly important since if it wasn't, then $\mu(A) = \mu(A \cup A) = 2 \mu(A)$, which is absurd. } subsets $A_k \subset 2^{X}$,
      \begin{equation}
        \mu \bigg( \bigsqcup_{k=1}^\infty A_k \bigg) = \sum_{k=1}^\infty \mu(A_k)
      \end{equation}
    \end{enumerate} 
  \end{definition}

  \begin{example}[Measure Spaces]
    The following are all valid measure spaces. 
    \begin{enumerate}
      \item $(\mathbb{R}, \mathcal{L}, m)$, where $\mathcal{L}$ is the set of all Lebesgue-measurable sets. 
      \item $(\mathbb{R}, \mathcal{B}, m)$, where $\mathcal{B}$ is the set of all Borel-measurable sets. 
      \item $(X, 2^X, \eta)$, where $\eta(E)$ is the cardinality of the set. This is called the \textbf{counting measure}. 
      \item $(X, 2^X, \delta_{x_0})$ where $x_0 \in X$ and $\delta_{x_0}(E) = 1$ if $x_0 \in E$ and $0$ if else. This is called the \textbf{Dirac measure}. 
    \end{enumerate}
  \end{example}

  From just this definition, we can restore all of our familiar properties. 

  \begin{theorem}[Axiomatic Properties of Measure]
    Let $(X, \mathcal{M}, \mu)$ be a measure space. 
    \begin{enumerate}
      \item \textit{Finite Additivity}. For any finite disjoint collection $\{E_k\}_{k=1}^n$ of measurable sets, 
        \begin{equation}
          \mu \bigg( \bigcup_{k=1}^n E_k \bigg) = \sum_{k=1}^n \mu(E_k)
        \end{equation}
      \item \textit{Monotonicity}. If $A \subset B$ are both measurable sets, then 
        \begin{equation}
          \mu(A) \leq \mu(B)
        \end{equation}
      \item \textit{Excision}. If $A \subset B$ are both measurable sets and $\mu(A) < +\infty$, then 
        \begin{equation}
          \mu(B \setminus A) = \mu(B) - \mu(A)
        \end{equation}
      \item \textit{Countable Monotonicity}. For any countable collection $\{E_k\}_{k=1}^\infty$ of measurable sets that covers a measurable set $E$, 
        \begin{equation}
          \mu(E) \leq \sum_{k=1}^\infty \mu(E_k)
        \end{equation}
    \end{enumerate}
  \end{theorem}
  \begin{proof}
    Listed. 
    \begin{enumerate}
      \item \textit{Finite Additivity}. 
      \item \textit{Monotonicity}. Let $B \setminus A \coloneqq B \cap A^c$. Then, since $A$ and $B \setminus A$ are disjoint, we have 
        \begin{equation}
          \mu(B) = \mu\big( A \cup (B \setminus A) \big) = \mu(A) + \mu(B \setminus A) \geq \mu(A)
        \end{equation}

      \item \textit{Excision}.
      \item \textit{Countable Monotonicity}. We again try to divide this union into disjoint sets. Let $A_i^\prime = A \cap A_i$, and let $B_1 = A_1^\prime$ with 
        \begin{equation}
          B_i = A_i \setminus \bigcup_{j=1}^{i-1} A^\prime_j
        \end{equation}
        Since $B_i$'s are disjoint with $B_i \subset A_i$, we can use the first property to get 
        \begin{equation}
          \mu(A) = \sum_{i=1}^\infty \mu(B_i) \leq \sum_{i=1}^\infty \mu(A_i)
        \end{equation}

    \end{enumerate}
  \end{proof}

  \begin{theorem}[Continuity of Measure]
    Let $(X, \mathcal{M}, \mu)$ be a measure space. 
    \begin{enumerate}
      \item \textit{Continuity from Below}. If $\{A_k\}_{k=1}^\infty$ is an ascending sequence of measurable sets, then 
        \begin{equation}
          \mu \bigg( \bigcup_{k=1}^\infty A_k \bigg) = \lim_{k \to \infty} A_k
        \end{equation}
      \item \textit{Continuity from Above}. If $\{B_k\}_{k=1}^\infty$ is a descending sequence of measurable sets and $\mu(B_1) < +\infty$, then 
        \begin{equation}
          \mu \bigg( \bigcap_{k=1}^\infty B_k \bigg) = \lim_{k \to \infty} \mu(B_k)
        \end{equation}
    \end{enumerate}
  \end{theorem}
  \begin{proof}
    Listeed. 
    \begin{enumerate}
      \item \textit{Continuity from Below}. With the fact that $\mu(A_k)$ must be nondecreasing, we can use real analysis and see that it is bounded by $\infty$, meaning that it must have a limit. But why does this limit equal to the left hand side? We can see that 
        \begin{align}
          \mu\bigg( \bigcup_{k=1}^\infty A_k \bigg) & = \mu(A_1) + \sum_{k=2}^\infty \mu(B_k) \\
          & = \mu(A_1) + \lim_{k \rightarrow \infty} \sum_{k=2}^\infty \mu(B_k) \\
          & = \lim_{k \rightarrow \infty} \mu(A_1 \cup B_2 \cup \ldots B_k)  = \lim_{k \rightarrow \infty} \mu(A_k) 
        \end{align}
        where $B_k = A_k \setminus A_{k-1}$. 

      \item \textit{Continuity from Above}. The $\mu(A_1) < \infty$ is a necessary condition, since if we take $A_k = [k, \infty)$ on the real number line, then we have $\cap_{k=1}^\infty A_k = \emptyset$, but the limit of the measure is $\infty$. Well we can define $B_k = A_k \setminus A_{k+1}$ and write $\cap_{k=1}^\infty A_k = A_1 \setminus \cup_{k=1}^\infty B_k$, which means that 
        \begin{align}
          \mu\bigg( \bigcap_{k=1}^\infty A_k \bigg) & = \mu\bigg( A_1 \setminus \bigcup_{k=1}^\infty B_k \bigg) \\
          & = \mu(A_1) - \mu\bigg( \bigcup_{k=1}^\infty B_k\bigg) \\
          & = \mu(A_1) - \sum_{k=1}^\infty \mu(B_k) \\
          & = \mu(A_1) - \lim_{K \rightarrow \infty} \sum_{k=1}^K \mu(B_k) \\
          & = \lim_{K \rightarrow \infty} \bigg( \mu(A_1) - \sum_{k=1}^K \mu(B_k) \bigg) \\
          & = \lim_{K \rightarrow \infty} \mu \bigg( A_1 \setminus \bigcup_{k=1}^K B_k \bigg) = \lim_{K \rightarrow \infty} \mu(A_K)
        \end{align}
        Now the first line uses the fact that if $A \subset B$, then $\mu(B \setminus A) + \mu(A) = \mu(B)$, and with the further assumption that $\mu(A) < \infty$, we can subtract on both sides like we do with regular arithmetic. 
    \end{enumerate}
  \end{proof}

  \begin{definition}[Almost Everywhere]
    For a measure space $(X, \mathcal{M}, \mu)$ and a measurable subset $E$ of $X$, we say that a property $P$ holds \textbf{almost everywhere} on $E$ if it holds for all $E \setminus E_0$ for some measurable subset $E_0$ where $\mu(E_0) = 0$.
  \end{definition}

  \begin{lemma}[Borel-Cantelli Lemma]
    Let $(X, \mathcal{M}, \mu)$ be a measure space and $\{E_k\}_{k=1}^\infty$ be a countable collection of measurable sets for which $\sum_{k=1}^\infty \mu(E_k) < +\infty$. Then, 
    \begin{equation}
      \mu \big( \limsup_k E_k \big) \coloneqq \mu \bigg( \bigcap_{n = 1}^\infty \bigcup_{k \geq n} E_k \bigg) = 0
    \end{equation}
    That is, almost all $x \in X$ belong to at most a finite number of the $E_k$'s. 
  \end{lemma}
  \begin{proof}
    By continuity of $\mu$ from above and countable monotonicity of $\mu$, 
    \begin{equation}
      \mu \bigg( \bigcup_{n=1}^\infty \bigg[ \bigcap_{k \geq n} E_k \bigg] \bigg) = \lim_{n \to \infty} \mu \bigg( \bigcup_{k \geq n} E_k \bigg) \leq \lim_{n \to\infty} \sum_{k = n}^\infty \mu(E_k) = 0
    \end{equation}
    since the series converges. 
  \end{proof}

  Now here comes the new part. 

  \begin{definition}[Finite, $\sigma$-Finite Measures and Measurable Sets]
    Let $(X, \mathcal{M}, \mu)$ be a measure space. 
    \begin{enumerate}
      \item The measure $\mu$ is \textbf{finite} if $\mu(X) < +\infty$. 
      \item The measure $\mu$ is \textbf{$\sigma$-finite} if $X$ is the union of a countable collection of measurable sets, each of which has finite measure. 
    \end{enumerate}
    Let $E$ be a measurable set. 
    \begin{enumerate}
      \item $E$ is of \textbf{finite measure} if $\mu(E) < +\infty$. 
      \item $E$ is \textbf{$\sigma$-finite} if $E$ is the union of a countable collection of measurable sets, each of which has finite measure. 
    \end{enumerate}
    Finiteness implies $\sigma$-finiteness. 
  \end{definition}

  \begin{example}
    Listed. 
    \begin{enumerate}
      \item The Lebesgue measure on $[0, 1]$ is a finite measure. 
      \item The Lebesgue measure on $\mathbb{R}$ is a $\sigma$-finite measure. 
      \item The counting measure on an uncountable set is not $\sigma$-finite.  
    \end{enumerate}
  \end{example}

  \begin{definition}[Complete Metric Spaces]
    A measure space $(X, \mathcal{M}, \mu)$ is \textbf{complete} if $\mathcal{M}$ contains all subsets of sets of measure $0$. 
  \end{definition}

  \begin{example}
    Listed. 
    \begin{enumerate}
      \item $(\mathbb{R}, \mathcal{L}, m)$ is complete. 
      \item $(\mathbb{R}, \mathcal{B}, m)$ is complete since we shows that the Cantor set (a Borel set of Lebesgue measure $0$), contains a subset that is not Borel. 
    \end{enumerate}
  \end{example}

  \begin{theorem}[Every Measure Space can be Completed]
    Let $(X, \mathcal{M}, \mu)$ be a measure space. Define $\mathcal{M}_0$ to be the collection of subsets $E$ of $X$ of the form $E = A \cup B$ where 
    \begin{enumerate}
      \item $B \in \mathcal{M}$, 
      \item $A \subset C$ for some $C \in \mathcal{M}$ for which $\mu(C) = 0$.\footnote{So we are basically splitting $E$ into a measure $0$ part $C$ and everything else $B$.}
    \end{enumerate}
    For such a set $E$, define $\mu_0 (E) = \mu(B)$. Then, $\mathcal{M}_0$ is a $\sigma$-algebra that contains $\mathcal{M}$, $\mu_0$ is a measure that extends $\mu$, and $(X, \mathcal{M}_0, \mu_0)$ is a complete measure space. 
  \end{theorem}
  \begin{proof}
    
  \end{proof}

\subsection{Signed Measures}

  \begin{lemma}[Sums and Positive Multiples of Measures are Measures]
    If $\mu_1$ and $\mu_2$ are two measures defined on the same measurable space $(X, \mathcal{M})$, then for $\alpha, \beta > 0$, the following is a measure as well. 
    \begin{equation}
      \mu_3 (E) = \alpha \cdot \mu_1 (E) + \beta \cdot \mu_2 (E)
    \end{equation}
  \end{lemma}
  \begin{proof}
    
  \end{proof}

  Note that we can't always define differences of such measures 
  \begin{equation}
    \nu(E) = \mu_1 (E) - \mu_2 (E)
  \end{equation}
  since $\nu$ may not always be nonnegative. Furthermore, it may not even be defined if $\mu_1 (E) - \mu_2 (E) = \infty - \infty$. 

  \begin{definition}[Signed Measure]
    A \textbf{signed measure} $\nu$ on the measurable space $(X, \mathcal{M})$ is a function $\nu: \mathcal{M} \to [-\infty, +\infty]$ satisfying 
    \begin{enumerate}
      \item \textit{Well-Defined}. $\nu$ assumes at most one of the values $+\infty, -\infty$. 
      \item \textit{Null Empty Set}. $\nu(\emptyset) = 0$ 
      \item \textit{Countable Additivity}. For any countable collection $\{E_k\}_{k=1}^\infty$ of disjoint measurable sets, 
        \begin{equation}
          \nu \bigg( \bigcup_{k=1}^\infty E_k \bigg) = \sum_{k=1}^\infty \nu(E_k)
        \end{equation}
        where the series $\sum_{k} \nu(E_k)$ converges absolutely if $\nu \big( \cup_{k} E_k \big)$ is finite. 
    \end{enumerate}
  \end{definition} 

  \begin{definition}[Positive, Negative, Null Sets]
    Let $\nu$ be a signed measure on $(X, \mathcal{M})$ and $A \in \mathcal{M}$. 
    \begin{enumerate}
      \item $A$ is \textbf{positive} w.r.t. $\nu$ if for all measurable $E \subset A$, $\nu(E) \geq 0$. 
      \item $A$ is \textbf{negative} w.r.t. $\nu$ if for all measurable $E \subset B$, $\nu(E) \leq 0$. 
      \item $A$ is \textbf{null} w.r.t. $\nu$ if for all measurable $E \subset B$, $\nu(E) = 0$.\footnote{By monotoncity of measure, a set is null if and only if it has measure $0$.}
    \end{enumerate}
  \end{definition}

  \begin{lemma}[Hahn's Lemma]
    Let $\nu$ be a signed measure on $(X, \mathcal{M})$ and $E$ a measurable set for which $0 < \nu(E) < +\infty$. Then, there is a measurable subset $A \subset E$ that is positive and of positive measure. 
  \end{lemma}

  \begin{theorem}[Hahn Decomposition Theorem]
    Let $\nu$ be a signed measure on the measurable space $(X, \mathcal{M})$. Then, there is a positive set $A$ and a negative set $B$, both with respect to $\nu$, for which 
    \begin{equation}
      X = A \sqcup B
    \end{equation}
    That is, we can always decompose $X$ into a positive and negative measure parts, and this is called the \textbf{Hahn decomposition} of $X$ w.r.t. $\nu$.\footnote{Note that this may not be unique, since if $A \cup B$ is a Hahn decomposition, then by excising a null set $E$ from $A$ and adding to $B$, $(A \setminus E) \cup (B \cup E)$ is also a Hahn decomposition.} 
  \end{theorem}

  Therefore, the measure of $\nu^+ (E) = \nu (E \cap A)$ and $\nu^- (E) = -\nu(E \cap B)$. This decomposition is nice since we know that the positive parts and the negative parts are ``nicely separated.'' Let's formalize this notion. 

  \begin{definition}[Mutually Singular Measures]
    Two measures $\nu_1, \nu_2$ are said to be \textbf{mutually singular}, denoted $\nu_1 \perp \nu_2$, if $X = A \cup B$ with $\nu_1 (A) = \nu_2 (B) = 0$. 
  \end{definition}

  \begin{theorem}[Jordan Decomposition Theorem]
    Let $\nu$ be a signed measure on the measurable space $(X, \mathcal{M})$. Then, there is a unique pair of mutually singular measures $\nu^+, \nu^-$ on $(X, \mathcal{M})$ for which 
    \begin{equation}
      \nu = \nu^+ - \nu^- 
    \end{equation}
    called the \textbf{Jordan decomposition}, with $\nu^+, \nu^-$ called the positive and negative parts of $\nu$. Since $\nu$ assumes at most one of the values $\pm \infty$, either $\nu^+, \nu^-$ must be finite. 
  \end{theorem}
  \begin{proof}
    We have already proven the first part as 
    \begin{equation}
      \nu^+ (E) = \nu (E \cap A), \quad \nu^- (E) = -\nu(E \cap B)
    \end{equation}
  \end{proof}

  \begin{example}[]
    Let $f: \mathbb{R} \to \mathbb{R}$ be a function that is Lebesgue integrable over $\mathbb{R}$, and define 
    \begin{equation}
      \nu(E) \coloneqq \int_E f \,dm
    \end{equation}
    Then, from the countable additivity of integration, $\nu$ is a signed measure on the measurable space $(\mathbb{R}, \mathcal{L})$. Define 
    \begin{equation}
      A = \{x \in \mathbb{R} \mid f(x) \geq 0 \}, \quad B = \{x \in \mathbb{R} \mid f(x) < 0 \}
    \end{equation}
    and 
    \begin{equation}
      \nu^+ (E) = \int_{A \cap E} f \,dm, \quad \nu^- (E) = - \int_{B \cap E} f \,dm
    \end{equation}
    Then, $A, B$ is a Hahn decomposition of $\mathbb{R}$ w.r.t. the signed measure $\nu$. Moreover, $v = v^+ = v^-$ is a Jordan decomposition of $\nu$. 
  \end{example}

\subsection{Carathéodory Construction of Measurable Sets} 

  \begin{definition}[Outer Measure]
    Given a space $X$, an \textbf{outer measure} is a function $\mu^\ast : 2^X \to [0, +\infty]$ satisfying either the two properties. 
    \begin{enumerate}
      \item \textit{Null Empty Set}. $\mu^\ast(\emptyset) = 0$. 
      \item \textit{Countable Monotonicity}. For arbitrary subset $A, B_1, B_2, \ldots$, 
      \begin{equation}
        A \subset \bigcup_{k=1}^\infty B_k \implies \mu(A) \leq \sum_{k=1}^\infty \mu(B_k)
      \end{equation} 
    \end{enumerate}
  \end{definition}

  \begin{theorem}[Construction of Outer Measure]
    Let $\mathcal{S}$ be a collection of subsets in $X$ and $\mu: \mathcal{S} \to [0, +\infty]$ be a set functions. Define $\mu^\ast(\emptyset) = 0$ and 
    \begin{equation}
      \mu^\ast (E) \coloneqq \inf \bigg\{ \sum_{k=1}^\infty \mu(E_k) \colon E \subset \bigcup_k E_k \bigg\}
    \end{equation}
    where the infimum of an empty set is $+\infty$. Then, the set function $\mu^\ast : 2^X \to [0, +\infty]$ is an outer measure called the \textbf{outer measure induced by $\mu$}. 
  \end{theorem}
  \begin{proof}
    
  \end{proof}

  \begin{definition}[Carathéodory's criterion]
    Given outer measure $\mu^\ast$ on $X$, a set $E \subset X$  is called \textbf{$\mu^\ast$-measurable} if for every set $A \subset X$, 
    \begin{equation}
      \mu^\ast (A \cap E) + \mu^\ast (A \cap E^c) = m^\ast (A) 
    \end{equation}
  \end{definition} 

  \begin{theorem}[Measurable Sets is a $\sigma$-Algebra]
    Let $\mu^\ast$ be an outer measure on $2^X$. Then, the collection $\mathcal{M}$ of sets that are measurable w.r.t. $\mu^\ast$, also called $\mu^\ast$-measurable, is a $\sigma$-algebra. If $\overline{\mu}$ is the restriction of $\mu^\ast$ to $\mathcal{M}$, then $(X, \mathcal{M}, \overline{\mu})$ is a complete measure space. 
  \end{theorem}
  \begin{proof}
    It is clear that complements are measurable by symmetricity of Carathéodory's criterion. The processes of proving this is identical to that of Lebesgue measure: first prove that finite union of measurable sets is measurable. Then show that for any $A \subset X$ and a finite disjoint collection $\{E_k\}_{k=1}^n$, we have 
    \begin{equation}
      \mu^\ast \bigg( A \cap \bigg[ \bigcup_{k=1}^n E_k \bigg] \bigg) = \sum_{k=1}^n \mu^\ast (A \cap E_k) 
    \end{equation}
    Finally, we prove that union of countable collection of measurable sets is measurable, where one direction is easy and the other is done by using the previous proposition and taking $n \to \infty$. 
  \end{proof}

  \begin{definition}[Carathéodory Measure]
    Let $\mathcal{S}$ be a collection of subsets of $X$, $\mu: \mathcal{S} \to [0, +\infty]$ a set function, and $\mu^\ast$ the outer measure induced by $\mu$. The measure $\overline{\mu}$ that is the restriction of $\mu^\ast$ to the $\sigma$-algebra $\mathcal{M}$ of $\mu^\ast$-measurable sets is called the \textbf{Carathéodory measure induced by $\mu$}. 
  \end{definition}

  Recall the regularity properties of Lebesgue measurable sets. That is, $E \subset \mathbb{R}$ is measurable iff there exists a $G_\delta$-set $G$ s.t. $E \subset G$ and $m^\ast (G \setminus E) = 0$. The following is a generalization of this. 

  \begin{theorem}[Regularity of $\mu^\ast$-measurable Sets]
    Let $\mu: \mathcal{S} \to [0, +\infty]$ be a set function defined on a collection $\mathcal{S}$ of subsets of a set $X$ and $\overline{\mu}: \mathcal{M} \to [0, +\infty]$, the Carathéodory measure induced by $\mu$. Let $E$ be a subset of $X$ for which $\mu^\ast (E) < +\infty$. Then, there is a subset $A \subset X$ for which 
    \begin{equation}
      A \in S_{\sigma \delta}, \quad E \subset A, \quad \mu^\ast (E) = \mu^\ast (A)
    \end{equation}
    Furthermore, if $E$ and seach set in $\mathcal{S}$ is $\mu^\ast$-measurable, then so is $A$, and 
    \begin{equation}
      \overline{\mu}(A \setminus E) = 0
    \end{equation}
  \end{theorem}
  \begin{proof}
  \end{proof}

  We can also generalize this further by introducing a increasing, continuous function $F: \mathbb{R} \rightarrow \mathbb{R}$ and defining the outer measure to be 
  \begin{equation}
   \lambda^\ast (A) = \inf_{C_A} \sum_{j=1}^\infty \big( F(b_j) - F(a_j) \big) 
  \end{equation}

\subsection{Premeasures} 

  Note that given a set function $\mu$ over $\mathcal{S}$, its Carathéodory extension $\overline{\mu}$ need not agree with $\mu$ for sets in $\mathcal{S}$. We would like to find adequate assumptions such that $\overline{\mu}$ is an extension of $\mu$. 

  \begin{definition}[Premeasure]
    Let $\mathcal{S}$ be a collection of subsets of $X$ and $\mu: \mathcal{S} \to [0, +\infty]$ a set function. Then, $\mu$ is called a \textbf{premeasure} if  
    \begin{enumerate}
      \item $\mu$ is finitely additive, 
      \item $\mu$ is countably monotone, 
      \item if $\emptyset \in \mathcal{S}$, then $\mu(\emptyset) = 0$. 
    \end{enumerate}
  \end{definition}

  \begin{theorem}[Premeasure Condition for Carathéodory Measure]
    Let $\mathcal{S} \subset 2^X$ and $\mu: \mathcal{S} \to [0, +\infty]$ a set function. In order for the Carathéodory measure induced by $\mu$ be an extension of $\mu$, it is necessary that $\mu$ is a premeasure.  
  \end{theorem}
  \begin{proof}
    
  \end{proof}

  So being a premeasure is a necessary but not sufficient condition, but if we impose on $\mathcal{S}$ a finer set-theoretic structure, this necessary condition is also sufficient. 

  \begin{definition}[Closure Under Relative Complements]
    A collection $\mathcal{S} \subset 2^X$ is said to be closed w.r.t. the formation of relative complements provided 
    \begin{equation}
      A, B \in \mathcal{S} \implies A \setminus B \in \mathcal{S}
    \end{equation}
  \end{definition}

  \begin{theorem}[Premeasure over Set Closured Under Relative Complements Induces Carathéodory Extension]
    Let $\mu: \mathcal{S} \to [0, +\infty]$ be a premeasure on $\mathcal{S}$ that is closed w.r.t. the formation of relative complements. Then, the Carathéodory measure $\overline{\mu}: \mathcal{M} \to [0, +\infty]$ induced by $\mu$ is an extension of $\mu$, called the \textbf{Carathéodory extension} of $\mu$. 
  \end{theorem}
  \begin{proof}
    
  \end{proof}

  However, a number of natural premeasures such as the premeasure length defined on the collection of bounded intervals of reals numbers, are defined on collections of sets that are not closed w.r.t. relative complements. So we consider alternate conditions for extending measures. 

  \begin{definition}[Semiring]
    A nonempty collection $\mathcal{S}$ of subsets of a set $X$ is a \textbf{semiring} if 
    \begin{enumerate}
      \item \textit{Closure under finite intersections}. 
        \begin{equation}
          A, B \in \mathcal{S} \implies A \cap B \in \mathcal{S}
        \end{equation}
      \item \textit{Disjoint decomposition of relative complements}. 
        \begin{equation}
          A, B \in \mathcal{S} \implies A \setminus B = \bigsqcup_{k=1}^n C_k
        \end{equation}
        for some collection $C_k \in \mathcal{S}$. 
    \end{enumerate}
  \end{definition}

  \begin{theorem}[Carathéodory-Hahn Theorem]
    Let $\mu: \mathcal{S} \to [0, +\infty]$ be a premeasures on a semiring $\mathcal{S}$ of subsets of $X$. 
    \begin{enumerate}
      \item Then, the Carathéodory measure $\overline{\mu}$ induced by $\mu$ is an extension of $\mu$. 
      \item Furthermore, if $\mu$ is $\sigma$-finite, then so is $\overline{\mu}$, and $\overline{\mu}$ is the unique measure on the $\sigma$-algebra of $\mu^\ast$-measurable sets that extends $\mu$. 
    \end{enumerate}
  \end{theorem}

\subsection{Product Measure} 

  Before, we saw how we can construct measures using Carathéodory construction. We will consider how to create product measures, which will be an extension of a set functions using the Carathéodory-Hahn theorem. 

  \begin{definition}[Measurable Rectangle]
    Let $(X, \mathcal{A}, \mu)$, $(Y, \mathcal{B}, \nu)$ be two measure spaces. Consider the product space $X \times Y$. If $A \in \mathcal{A}, B \in \mathcal{B}$, then the set $A \times B$ is called a \textbf{measurable rectangle}. 
  \end{definition}

  \begin{lemma} 
    Let $\{A_k \times B_k\}_{k=1}^\infty$ be a countable disjoint collection of measurable rectangles whose union is also a measurable rectangle $A \times B$. Then, 
    \begin{equation}
      \mu(A) \times \nu(B) = \sum_{k=1}^\infty \mu(A_k) \times \nu(B_k)
    \end{equation}
  \end{lemma}
  \begin{proof}
    
  \end{proof}

  We want to set up the conditions to invoke the Carathéodory-Hahn theorem. This naturally leads to the following. 

  \begin{theorem}
    Let $\mathcal{R}$ be the collection of measurable rectangles in $X \times Y$ and for a measurable rectangle $A \times B$, define 
    \begin{equation}
      \lambda(A \times B) = \mu(A) \cdot \nu(B)
    \end{equation}
    Then, $\mathcal{R}$ is a semiring and $\lambda: \mathcal{R} \to [0, +\infty]$ is a premeasure. 
  \end{theorem}
  \begin{proof}
    
  \end{proof}

  \begin{definition}[Product Measure]
    Let $(X, \mathcal{A}, \mu)$, $(Y, \mathcal{B}, \nu)$ be two measure spaces, $\mathcal{R}$ the collection of measurable rectangles contained in $X \times Y$, and $\lambda$ the premeasure defined on $\mathcal{R}$ by 
    \begin{equation}
      \lambda(A \times B) = \mu(A) \cdot \nu(B)
    \end{equation}
    for all $A \times B \in \mathcal{R}$. Then, the \textbf{product measure} $\lambda = \mu \times \nu$ is the Carathéodory extension of $\lambda: \mathcal{R} \to [0, +\infty]$ defined on the $\sigma$-algebra of $(\mu \times \nu)^\ast$-measurable subsets of $X \times Y$. 
  \end{definition}

\subsection{Stieltjes Construction of Measure}

  Let $\mathbb{R}^n$ be the continuum and $\mathcal{R}^n$ be the \textbf{Borel $\boldsymbol{\sigma}$-algebra}, defined as the $\sigma$-algebra generated by the open sets of $\mathbb{R}^n$. 

  \begin{example}[Stieltjes Measure Function]
    Measures on $(\mathbb{R}, \mathcal{R})$ are defined by giving a \textbf{Stieltjes measure function} with the following properties: 
    \begin{enumerate}
      \item $F$ is nondecreasing 
      \item $F$ is right continuous: 
      \begin{equation}
        \lim_{y \downarrow x} F(y) = F(x)
      \end{equation}
    \end{enumerate}
  \end{example}

  \begin{theorem}
    Associated with each Stieltjes measure function $F$ there is a unique measure $\mu$ on $(\mathbb{R}, \mathcal{R})$ with 
    \begin{equation}
      \mu((a, b]) = F(b) - F(a)
    \end{equation}
    When $F(x) = x$, then the resulting measure is called the \textbf{Lebesgue measure}. 
  \end{theorem}

  This is quite a hard proof, but we outline the construction of this measure on $\mathbb{R}$. First, we would like to define a "nice" set of half-open half-closed intervals, which we show is a semialgebra $\mathcal{S}$. We can easily define a measure $\mu$ on this semialgebra. We can extend this semialgebra to an algebra $\overline{\mathcal{S}}$, along with a proper extension $\overline{\mu}$ that is a unique measure on $\overline{\mathcal{S}}$. 

  \begin{definition}[Semialgebra, Algebra]
    A collection $\mathcal{S}$ of sets is said to be a \textbf{semialgebra} if 
    \begin{enumerate}
      \item it is closed under intersection 
      \item If $S \in \mathcal{S}$, then $S^c$ is a finite disjoint union of sets in $\mathcal{S}$
    \end{enumerate}
    A collection $\mathcal{A}$ of subsets is said to be an \textbf{algebra} if 
    \begin{enumerate}
      \item it is closed under union 
      \item it is closed under complementation
      \item the first two imply that it is closed under intersection
    \end{enumerate}
    We can see that a set that is a $\sigma$-algebra $\implies$ it is an algebra. 
  \end{definition}

  Here is an example of a semialgebra, which we will utilize in building a measure on $\mathbb{R}^n$.  

  \begin{example}
    Let $\mathcal{S}_d$ be the empty set plus all sets of the form 
    \begin{equation}
      (a_1, b_1] \times \ldots \times (a_d, b_d] \subset \mathbb{R}^d
    \end{equation}
    where $-\infty \leq a_i < b_i \leq +\infty$. $\mathcal{S}_d$ is a semialgebra since 
    \begin{equation}
      \bigg( \prod_i (a_i^1 , b_i^1] \bigg) \cap \bigg( \prod_i (a_i^2, b_i^2] \bigg) = \prod_i (\max\{a_i^1, a_i^2\}, \min\{b_i^1, b_i^2\}]
    \end{equation}
    and ...
  \end{example}

  Now, we show that we can extend this semialgebra to an algebra. 

  \begin{lemma}
    If $\mathcal{S}$ is a semialgebra, then $\overline{\mathcal{S}} = \{\text{finite disjoint unions of sets in } \mathcal{S}\}$ is an algebra, called the algebra generated by $\mathcal{S}$. 
  \end{lemma}
  \begin{proof}

  \end{proof}

  \begin{example}
    Given $\mathbb{R}$ and its semialgebra $\mathcal{S}_1$, then $\overline{\mathcal{S}}_1$ consists of the empty set and all sets of the form 
    \begin{equation}
      \bigcup_{i=1}^n (a_i, b_i] \text{ where } -\infty \leq a_i < b_i \leq +\infty
    \end{equation}
  \end{example}

  Now as for extending our measure function to $\overline{\mathcal{S}}$, we can simply use the properties. Note that since since an algebra is constructed from finite disjoint unions of a semialgebra, given that the finite collection $\{A_i\}_{i=1}^n$ all reside in $\mathcal{S}$ and are disjoint, then their disjoint union must be in $\overline{\mathcal{S}}$ and must be measurable, defined as 
  \begin{equation}
    \overline{\mu} \bigg( \bigsqcup_{i=1}^n A_i \bigg) = \sum_{i=1}^n \mu(A_i)
  \end{equation}

  \begin{definition}[$\sigma$-finite measure]
    Given a measure on an algebra $\mathcal{A}$, $\mu$ is said to be \textbf{$\boldsymbol{\sigma}$-finite} if there is a sequence of sets $A_1, A_2, \ldots \in \mathcal{A}$ s.t. $\mu(A_i) < \infty$ and $\cup_i A_i = \Omega$ . 
  \end{definition}

  \begin{theorem}
    Let $\mathcal{S}$ be a semialgebra and let $\mu$ defined on $\mathcal{S}$ have $\mu(\emptyset) = 0$. Suppose 
    \begin{enumerate}
      \item if $S \in \mathcal{S}$ is a finite disjoint union of sets $\{S_i\}_{i=1}^n$, then 
      \begin{equation}
        \mu(S) = \sum_{i=1}^n \mu(S_i)
      \end{equation}
      \item f $S$ is a countably infinite disjoint union of sets $\{S_j\}_{j=1}^\infty$, then 
      \begin{equation}
        \mu(S) \leq \sum_{j=1}^\infty \mu(S_j)
      \end{equation}
    \end{enumerate}
    Then, $\mu$ has a unique extension $\bar{\mu}$ that is a measure on $\overline{\mathcal{S}}$, the algebra generated by $\mathcal{S}$. If $\bar{\mu}$ is $\sigma$-finite, then there is a unique extension $\nu$ that is a measure on $\sigma(\mathcal{S})$ (the smallest $\sigma$-algebra containing $\mathcal{S}$). 
  \end{theorem}


\section{Measure}

  Now we generalize to abstract measure spaces. Recall how we constructed the Lebesgue measure: (1) We took a set function $\ell$ that assigns lengths to all intervals in $\mathbb{R}$. (2) We have used this length to define the Lebesgue outer measure as an outer approximation of these intervals. (3) We then use the outer measure to define measurable sets with Caratheodory's criterion, which states that measurable sets should split any set nicely into measurable sets. (4) We verify that the collection of all measurable sets is a $\sigma$-algebra. (5) We define the Lebesgue measure as the restriction of the Lebesgue outer measure to the collection of mesaurable sets. This is called the \textit{Caratheodory} construction of Lebesgue measure, and we can generalize this to an abstract space $X$ under certain conditions.  

  First, we want all measurable sets to be a $\sigma$-algebra, which defines all the well-behaved sets. 

  \begin{definition}[Measurable Space] 
    A \textbf{measurable space} is a tuple $(X, \mathcal{M})$ consisting of a set $X$ with a $\sigma$-algebra of subsets of $X$. Elements of $\mathcal{M}$ are called \textbf{measurable sets}. 
  \end{definition}

  \begin{definition}[Measure, Measure Space]
    A \textbf{measure} $\mu$ on a measurable space $(X, \mathcal{M})$ is a set function $\mu: \mathcal{M} \to [0, +\infty]$ satisfying the following. 
    \begin{enumerate}
      \item \textit{Null empty set}. $\mu(\emptyset) = 0$. 
      \item \textit{Countable Additivity}. For all countable collections $\{A_k\}_{k=1}^\infty$ of pairwise disjoint\footnote{Disjointness is clearly important since if it wasn't, then $\mu(A) = \mu(A \cup A) = 2 \mu(A)$, which is absurd. } subsets $A_k \subset 2^{X}$,
      \begin{equation}
        \mu \bigg( \bigsqcup_{k=1}^\infty A_k \bigg) = \sum_{k=1}^\infty \mu(A_k)
      \end{equation}
    \end{enumerate} 
  \end{definition}

  \begin{example}[Measure Spaces]
    The following are all valid measure spaces. 
    \begin{enumerate}
      \item $(\mathbb{R}, \mathcal{L}, m)$, where $\mathcal{L}$ is the set of all Lebesgue-measurable sets. 
      \item $(\mathbb{R}, \mathcal{B}, m)$, where $\mathcal{B}$ is the set of all Borel-measurable sets. 
      \item $(X, 2^X, \eta)$, where $\eta(E)$ is the cardinality of the set. This is called the \textbf{counting measure}. 
      \item $(X, 2^X, \delta_{x_0})$ where $x_0 \in X$ and $\delta_{x_0}(E) = 1$ if $x_0 \in E$ and $0$ if else. This is called the \textbf{Dirac measure}. 
    \end{enumerate}
  \end{example}

  From just this definition, we can restore all of our familiar properties. 

  \begin{theorem}[Axiomatic Properties of Measure]
    Let $(X, \mathcal{M}, \mu)$ be a measure space. 
    \begin{enumerate}
      \item \textit{Finite Additivity}. For any finite disjoint collection $\{E_k\}_{k=1}^n$ of measurable sets, 
        \begin{equation}
          \mu \bigg( \bigcup_{k=1}^n E_k \bigg) = \sum_{k=1}^n \mu(E_k)
        \end{equation}
      \item \textit{Monotonicity}. If $A \subset B$ are both measurable sets, then 
        \begin{equation}
          \mu(A) \leq \mu(B)
        \end{equation}
      \item \textit{Excision}. If $A \subset B$ are both measurable sets and $\mu(A) < +\infty$, then 
        \begin{equation}
          \mu(B \setminus A) = \mu(B) - \mu(A)
        \end{equation}
      \item \textit{Countable Monotonicity}. For any countable collection $\{E_k\}_{k=1}^\infty$ of measurable sets that covers a measurable set $E$, 
        \begin{equation}
          \mu(E) \leq \sum_{k=1}^\infty \mu(E_k)
        \end{equation}
    \end{enumerate}
  \end{theorem}
  \begin{proof}
    Listed. 
    \begin{enumerate}
      \item \textit{Finite Additivity}. 
      \item \textit{Monotonicity}. 
      \item \textit{Excision}.
      \item \textit{Countable Monotonicity}. 
    \end{enumerate}
  \end{proof}

  \begin{theorem}[Continuity of Measure]
    Let $(X, \mathcal{M}, \mu)$ be a measure space. 
    \begin{enumerate}
      \item \textit{Continuity from Below}. If $\{A_k\}_{k=1}^\infty$ is an ascending sequence of measurable sets, then 
        \begin{equation}
          \mu \bigg( \bigcup_{k=1}^\infty A_k \bigg) = \lim_{k \to \infty} A_k
        \end{equation}
      \item \textit{Continuity from Above}. If $\{B_k\}_{k=1}^\infty$ is a descending sequence of measurable sets and $\mu(B_1) < +\infty$, then 
        \begin{equation}
          \mu \bigg( \bigcap_{k=1}^\infty B_k \bigg) = \lim_{k \to \infty} \mu(B_k)
        \end{equation}
    \end{enumerate}
  \end{theorem}
  \begin{proof}
    
  \end{proof}

  \begin{definition}[Almost Everywhere]
    For a measure space $(X, \mathcal{M}, \mu)$ and a measurable subset $E$ of $X$, we say that a property $P$ holds \textbf{almost everywhere} on $E$ if it holds for all $E \setminus E_0$ for some measurable subset $E_0$ where $\mu(E_0) = 0$.
  \end{definition}

  \begin{lemma}[Borel-Cantelli Lemma]
    Let $(X, \mathcal{M}, \mu)$ be a measure space and $\{E_k\}_{k=1}^\infty$ be a countable collection of measurable sets for which $\sum_{k=1}^\infty \mu(E_k) < +\infty$. Then, 
    \begin{equation}
      \mu \big( \limsup_k E_k \big) \coloneqq \mu \bigg( \bigcap_{n = 1}^\infty \bigcup_{k \geq n} E_k \bigg) = 0
    \end{equation}
    That is, almost all $x \in X$ belong to at most a finite number of the $E_k$'s. 
  \end{lemma}
  \begin{proof}
    By continuity of $\mu$ from above and countable monotonicity of $\mu$, 
    \begin{equation}
      \mu \bigg( \bigcup_{n=1}^\infty \bigg[ \bigcap_{k \geq n} E_k \bigg] \bigg) = \lim_{n \to \infty} \mu \bigg( \bigcup_{k \geq n} E_k \bigg) \leq \lim_{n \to\infty} \sum_{k = n}^\infty \mu(E_k) = 0
    \end{equation}
    since the series converges. 
  \end{proof}

  Now here comes the new part. 

  \begin{definition}[Finite, $\sigma$-Finite Measures and Measurable Sets]
    Let $(X, \mathcal{M}, \mu)$ be a measure space. 
    \begin{enumerate}
      \item The measure $\mu$ is \textbf{finite} if $\mu(X) < +\infty$. 
      \item The measure $\mu$ is \textbf{$\sigma$-finite} if $X$ is the union of a countable collection of measurable sets, each of which has finite measure. 
    \end{enumerate}
    Let $E$ be a measurable set. 
    \begin{enumerate}
      \item $E$ is of \textbf{finite measure} if $\mu(E) < +\infty$. 
      \item $E$ is \textbf{$\sigma$-finite} if $E$ is the union of a countable collection of measurable sets, each of which has finite measure. 
    \end{enumerate}
    Finiteness implies $\sigma$-finiteness. 
  \end{definition}

  \begin{example}
    Listed. 
    \begin{enumerate}
      \item The Lebesgue measure on $[0, 1]$ is a finite measure. 
      \item The Lebesgue measure on $\mathbb{R}$ is a $\sigma$-finite measure. 
      \item The counting measure on an uncountable set is not $\sigma$-finite.  
    \end{enumerate}
  \end{example}

  \begin{definition}[Complete Metric Spaces]
    A measure space $(X, \mathcal{M}, \mu)$ is \textbf{complete} if $\mathcal{M}$ contains all subsets of sets of measure $0$. 
  \end{definition}

  \begin{example}
    Listed. 
    \begin{enumerate}
      \item $(\mathbb{R}, \mathcal{L}, m)$ is complete. 
      \item $(\mathbb{R}, \mathcal{B}, m)$ is complete since we shows that the Cantor set (a Borel set of Lebesgue measure $0$), contains a subset that is not Borel. 
    \end{enumerate}
  \end{example}

  \begin{theorem}[Every Measure Space can be Completed]
    Let $(X, \mathcal{M}, \mu)$ be a measure space. Define $\mathcal{M}_0$ to be the collection of subsets $E$ of $X$ of the form $E = A \cup B$ where 
    \begin{enumerate}
      \item $B \in \mathcal{M}$, 
      \item $A \subset C$ for some $C \in \mathcal{M}$ for which $\mu(C) = 0$.\footnote{So we are basically splitting $E$ into a measure $0$ part $C$ and everything else $B$.}
    \end{enumerate}
    For such a set $E$, define $\mu_0 (E) = \mu(B)$. Then, $\mathcal{M}_0$ is a $\sigma$-algebra that contains $\mathcal{M}$, $\mu_0$ is a measure that extends $\mu$, and $(X, \mathcal{M}_0, \mu_0)$ is a complete measure space. 
  \end{theorem}
  \begin{proof}
    
  \end{proof}

\subsection{Signed Measures}

  \begin{lemma}[Sums and Positive Multiples of Measures are Measures]
    If $\mu_1$ and $\mu_2$ are two measures defined on the same measurable space $(X, \mathcal{M})$, then for $\alpha, \beta > 0$, the following is a measure as well. 
    \begin{equation}
      \mu_3 (E) = \alpha \cdot \mu_1 (E) + \beta \cdot \mu_2 (E)
    \end{equation}
  \end{lemma}
  \begin{proof}
    
  \end{proof}

  Note that we can't always define differences of such measures 
  \begin{equation}
    \nu(E) = \mu_1 (E) - \mu_2 (E)
  \end{equation}
  since $\nu$ may not always be nonnegative. Furthermore, it may not even be defined if $\mu_1 (E) - \mu_2 (E) = \infty - \infty$. 

  \begin{definition}[Signed Measure]
    A \textbf{signed measure} $\nu$ on the measurable space $(X, \mathcal{M})$ is a function $\nu: \mathcal{M} \to [-\infty, +\infty]$ satisfying 
    \begin{enumerate}
      \item \textit{Well-Defined}. $\nu$ assumes at most one of the values $+\infty, -\infty$. 
      \item \textit{Null Empty Set}. $\nu(\emptyset) = 0$ 
      \item \textit{Countable Additivity}. For any countable collection $\{E_k\}_{k=1}^\infty$ of disjoint measurable sets, 
        \begin{equation}
          \nu \bigg( \bigcup_{k=1}^\infty E_k \bigg) = \sum_{k=1}^\infty \nu(E_k)
        \end{equation}
        where the series $\sum_{k} \nu(E_k)$ converges absolutely if $\nu \big( \cup_{k} E_k \big)$ is finite. 
    \end{enumerate}
  \end{definition} 

  \begin{definition}[Positive, Negative, Null Sets]
    Let $\nu$ be a signed measure on $(X, \mathcal{M})$ and $A \in \mathcal{M}$. 
    \begin{enumerate}
      \item $A$ is \textbf{positive} w.r.t. $\nu$ if for all measurable $E \subset A$, $\nu(E) \geq 0$. 
      \item $A$ is \textbf{negative} w.r.t. $\nu$ if for all measurable $E \subset B$, $\nu(E) \leq 0$. 
      \item $A$ is \textbf{null} w.r.t. $\nu$ if for all measurable $E \subset B$, $\nu(E) = 0$.\footnote{By monotoncity of measure, a set is null if and only if it has measure $0$.}
    \end{enumerate}
  \end{definition}

  \begin{lemma}[Hahn's Lemma]
    Let $\nu$ be a signed measure on $(X, \mathcal{M})$ and $E$ a measurable set for which $0 < \nu(E) < +\infty$. Then, there is a measurable subset $A \subset E$ that is positive and of positive measure. 
  \end{lemma}

  \begin{theorem}[Hahn Decomposition Theorem]
    Let $\nu$ be a signed measure on the measurable space $(X, \mathcal{M})$. Then, there is a positive set $A$ and a negative set $B$, both with respect to $\nu$, for which 
    \begin{equation}
      X = A \sqcup B
    \end{equation}
    That is, we can always decompose $X$ into a positive and negative measure parts, and this is called the \textbf{Hahn decomposition} of $X$ w.r.t. $\nu$.\footnote{Note that this may not be unique, since if $A \cup B$ is a Hahn decomposition, then by excising a null set $E$ from $A$ and adding to $B$, $(A \setminus E) \cup (B \cup E)$ is also a Hahn decomposition.} 
  \end{theorem}

  Therefore, the measure of $\nu^+ (E) = \nu (E \cap A)$ and $\nu^- (E) = -\nu(E \cap B)$. This decomposition is nice since we know that the positive parts and the negative parts are ``nicely separated.'' Let's formalize this notion. 

  \begin{definition}[Mutually Singular Measures]
    Two measures $\nu_1, \nu_2$ are said to be \textbf{mutually singular}, denoted $\nu_1 \perp \nu_2$, if $X = A \cup B$ with $\nu_1 (A) = \nu_2 (B) = 0$. 
  \end{definition}

  \begin{theorem}[Jordan Decomposition Theorem]
    Let $\nu$ be a signed measure on the measurable space $(X, \mathcal{M})$. Then, there is a unique pair of mutually singular measures $\nu^+, \nu^-$ on $(X, \mathcal{M})$ for which 
    \begin{equation}
      \nu = \nu^+ - \nu^- 
    \end{equation}
    called the \textbf{Jordan decomposition}, with $\nu^+, \nu^-$ called the positive and negative parts of $\nu$. Since $\nu$ assumes at most one of the values $\pm \infty$, either $\nu^+, \nu^-$ must be finite. 
  \end{theorem}
  \begin{proof}
    We have already proven the first part as 
    \begin{equation}
      \nu^+ (E) = \nu (E \cap A), \quad \nu^- (E) = -\nu(E \cap B)
    \end{equation}
  \end{proof}

  \begin{example}[]
    Let $f: \mathbb{R} \to \mathbb{R}$ be a function that is Lebesgue integrable over $\mathbb{R}$, and define 
    \begin{equation}
      \nu(E) \coloneqq \int_E f \,dm
    \end{equation}
    Then, from the countable additivity of integration, $\nu$ is a signed measure on the measurable space $(\mathbb{R}, \mathcal{L})$. Define 
    \begin{equation}
      A = \{x \in \mathbb{R} \mid f(x) \geq 0 \}, \quad B = \{x \in \mathbb{R} \mid f(x) < 0 \}
    \end{equation}
    and 
    \begin{equation}
      \nu^+ (E) = \int_{A \cap E} f \,dm, \quad \nu^- (E) = - \int_{B \cap E} f \,dm
    \end{equation}
    Then, $A, B$ is a Hahn decomposition of $\mathbb{R}$ w.r.t. the signed measure $\nu$. Moreover, $v = v^+ = v^-$ is a Jordan decomposition of $\nu$. 
  \end{example}

\subsection{Construction of Outer Measures and Measurable Sets} 

  \begin{definition}[Outer Measure]
    Given a space $X$, an \textbf{outer measure} is a function $\mu^\ast : 2^X \to [0, +\infty]$ satisfying either the two properties. 
    \begin{enumerate}
      \item \textit{Null Empty Set}. $\mu^\ast(\emptyset) = 0$. 
      \item \textit{Countable Monotonicity}. For arbitrary subset $A, B_1, B_2, \ldots$, 
      \begin{equation}
        A \subset \bigcup_{k=1}^\infty B_k \implies \mu(A) \leq \sum_{k=1}^\infty \mu(B_k)
      \end{equation} 
    \end{enumerate}
  \end{definition}

  \begin{theorem}[Construction of Outer Measure]
    Let $\mathcal{S}$ be a collection of subsets in $X$ and $\mu: \mathcal{S} \to [0, +\infty]$ be a set functions. Define $\mu^\ast(\emptyset) = 0$ and 
    \begin{equation}
      \mu^\ast (E) \coloneqq \inf \bigg\{ \sum_{k=1}^\infty \mu(E_k) \colon E \subset \bigcup_k E_k \bigg\}
    \end{equation}
    where the infimum of an empty set is $+\infty$. Then, the set function $\mu^\ast : 2^X \to [0, +\infty]$ is an outer measure called the \textbf{outer measure induced by $\mu$}. 
  \end{theorem}
  \begin{proof}
    
  \end{proof}

  \begin{definition}[Carathéodory's criterion]
    Given outer measure $\mu^\ast$ on $X$, a set $E \subset X$  is called \textbf{$\mu^\ast$-measurable} if for every set $A \subset X$, 
    \begin{equation}
      \mu^\ast (A \cap E) + \mu^\ast (A \cap E^c) = m^\ast (A) 
    \end{equation}
  \end{definition} 

  \begin{theorem}[Measurable Sets is a $\sigma$-Algebra]
    Let $\mu^\ast$ be an outer measure on $2^X$. Then, the collection $\mathcal{M}$ of sets that are measurable w.r.t. $\mu^\ast$, also called $\mu^\ast$-measurable, is a $\sigma$-algebra. If $\overline{\mu}$ is the restriction of $\mu^\ast$ to $\mathcal{M}$, then $(X, \mathcal{M}, \overline{\mu})$ is a complete measure space. 
  \end{theorem}
  \begin{proof}
    It is clear that complements are measurable by symmetricity of Carathéodory's criterion. The processes of proving this is identical to that of Lebesgue measure: first prove that finite union of measurable sets is measurable. Then show that for any $A \subset X$ and a finite disjoint collection $\{E_k\}_{k=1}^n$, we have 
    \begin{equation}
      \mu^\ast \bigg( A \cap \bigg[ \bigcup_{k=1}^n E_k \bigg] \bigg) = \sum_{k=1}^n \mu^\ast (A \cap E_k) 
    \end{equation}
    Finally, we prove that union of countable collection of measurable sets is measurable, where one direction is easy and the other is done by using the previous proposition and taking $n \to \infty$. 
  \end{proof}

  \begin{definition}[Carathéodory Measure]
    Let $\mathcal{S}$ be a collection of subsets of $X$, $\mu: \mathcal{S} \to [0, +\infty]$ a set function, and $\mu^\ast$ the outer measure induced by $\mu$. The measure $\overline{\mu}$ that is the restriction of $\mu^\ast$ to the $\sigma$-algebra $\mathcal{M}$ of $\mu^\ast$-measurable sets is called the \textbf{Carathéodory measure induced by $\mu$}. 
  \end{definition}

  Recall the regularity properties of Lebesgue measurable sets. That is, $E \subset \mathbb{R}$ is measurable iff there exists a $G_\delta$-set $G$ s.t. $E \subset G$ and $m^\ast (G \setminus E) = 0$. The following is a generalization of this. 

  \begin{theorem}[Regularity of $\mu^\ast$-measurable Sets]
    Let $\mu: \mathcal{S} \to [0, +\infty]$ be a set function defined on a collection $\mathcal{S}$ of subsets of a set $X$ and $\overline{\mu}: \mathcal{M} \to [0, +\infty]$, the Carathéodory measure induced by $\mu$. Let $E$ be a subset of $X$ for which $\mu^\ast (E) < +\infty$. Then, there is a subset $A \subset X$ for which 
    \begin{equation}
      A \in S_{\sigma \delta}, \quad E \subset A, \quad \mu^\ast (E) = \mu^\ast (A)
    \end{equation}
    Furthermore, if $E$ and seach set in $\mathcal{S}$ is $\mu^\ast$-measurable, then so is $A$, and 
    \begin{equation}
      \overline{\mu}(A \setminus E) = 0
    \end{equation}
  \end{theorem}
  \begin{proof}
  \end{proof}

  We can also generalize this further by introducing a increasing, continuous function $F: \mathbb{R} \rightarrow \mathbb{R}$ and defining the outer measure to be 
  \begin{equation}
   \lambda^\ast (A) = \inf_{C_A} \sum_{j=1}^\infty \big( F(b_j) - F(a_j) \big) 
  \end{equation}

\subsection{Premeasures} 

  Note that given a set function $\mu$ over $\mathcal{S}$, its Carathéodory extension $\overline{\mu}$ need not agree with $\mu$ for sets in $\mathcal{S}$. We would like to find adequate assumptions such that $\overline{\mu}$ is an extension of $\mu$. 

  \begin{definition}[Premeasure]
    Let $\mathcal{S}$ be a collection of subsets of $X$ and $\mu: \mathcal{S} \to [0, +\infty]$ a set function. Then, $\mu$ is called a \textbf{premeasure} if  
    \begin{enumerate}
      \item $\mu$ is finitely additive, 
      \item $\mu$ is countably monotone, 
      \item if $\emptyset \in \mathcal{S}$, then $\mu(\emptyset) = 0$. 
    \end{enumerate}
  \end{definition}

  \begin{theorem}[Premeasure Condition for Carathéodory Measure]
    Let $\mathcal{S} \subset 2^X$ and $\mu: \mathcal{S} \to [0, +\infty]$ a set function. In order for the Carathéodory measure induced by $\mu$ be an extension of $\mu$, it is necessary that $\mu$ is a premeasure.  
  \end{theorem}
  \begin{proof}
    
  \end{proof}

  So being a premeasure is a necessary but not sufficient condition, but if we impose on $\mathcal{S}$ a finer set-theoretic structure, this necessary condition is also sufficient. 

  \begin{definition}[Closure Under Relative Complements]
    A collection $\mathcal{S} \subset 2^X$ is said to be closed w.r.t. the formation of relative complements provided 
    \begin{equation}
      A, B \in \mathcal{S} \implies A \setminus B \in \mathcal{S}
    \end{equation}
  \end{definition}

  \begin{theorem}[Premeasure over Set Closured Under Relative Complements Induces Carathéodory Extension]
    Let $\mu: \mathcal{S} \to [0, +\infty]$ be a premeasure on $\mathcal{S}$ that is closed w.r.t. the formation of relative complements. Then, the Carathéodory measure $\overline{\mu}: \mathcal{M} \to [0, +\infty]$ induced by $\mu$ is an extension of $\mu$, called the \textbf{Carathéodory extension} of $\mu$. 
  \end{theorem}
  \begin{proof}
    
  \end{proof}

  However, a number of natural premeasures such as the premeasure length defined on the collection of bounded intervals of reals numbers, are defined on collections of sets that are not closed w.r.t. relative complements. So we consider alternate conditions for extending measures. 

  \begin{definition}[Semiring]
    A nonempty collection $\mathcal{S}$ of subsets of a set $X$ is a \textbf{semiring} if 
    \begin{enumerate}
      \item \textit{Closure under finite intersections}. 
        \begin{equation}
          A, B \in \mathcal{S} \implies A \cap B \in \mathcal{S}
        \end{equation}
      \item \textit{Disjoint decomposition of relative complements}. 
        \begin{equation}
          A, B \in \mathcal{S} \implies A \setminus B = \bigsqcup_{k=1}^n C_k
        \end{equation}
        for some collection $C_k \in \mathcal{S}$. 
    \end{enumerate}
  \end{definition}

  \begin{theorem}[Carathéodory-Hahn Theorem]
    Let $\mu: \mathcal{S} \to [0, +\infty]$ be a premeasures on a semiring $\mathcal{S}$ of subsets of $X$. 
    \begin{enumerate}
      \item Then, the Carathéodory measure $\overline{\mu}$ induced by $\mu$ is an extension of $\mu$. 
      \item Furthermore, if $\mu$ is $\sigma$-finite, then so is $\overline{\mu}$, and $\overline{\mu}$ is the unique measure on the $\sigma$-algebra of $\mu^\ast$-measurable sets that extends $\mu$. 
    \end{enumerate}
  \end{theorem}

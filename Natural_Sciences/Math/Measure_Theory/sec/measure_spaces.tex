\section{Measure Spaces}

\subsection{Sigma Algebra}

  \begin{definition}[$F_\sigma$ Sets]
    A \textbf{$F_\sigma$-set} is a subset of a topological space that is a countable union of closed sets. 
  \end{definition}

  \begin{definition}[$G_\delta$ Sets]
    A \textbf{$G_\delta$-set} is a subset of a topological space  that is a countable intersection of open sets. 
  \end{definition}

  \begin{lemma}
    The complement of a $F_\sigma$ set is a $G_\delta$ set. 
  \end{lemma} 
  
  Now, given any set $X$, we can construct its power set $2^X$. But we can't naively just give a measure to every $A \in 2^X$, since for certain spaces, this causes nasty contradictions shown through the Banach-Tarski Paradox.\footnote{Given any two bounded subsets $A$ and $B$ of $\mathbb{R}^n$ where $n \geq 3$, both of which have a nonempty interior, there are partitions of $A$ and $B$ into a finite number of disjoint subsets, $A = A_1 \cup \ldots \cup A_k$, $B = B_1 \cup \ldots \cup B_k$, such that $A_i$ and $B_i$ are congruent for every $i \in [k]$.} A nice set of subsets of $X$ to work with is the $\sigma$-algebra of $X$. 

  A $\sigma$-algebra is similar to the topology $\tau$ of topological space. Both $\mathcal{A}$ and $\tau$ require $\emptyset$ and $X$ to be in it. The three differences are that (i) $\tau$ does not allow compelmentation, (ii) $\tau$ allows any (even uncountable) union of sets (condition is strengthened), and (iii) $\tau$ allows only finite intersection of sets (condition is weakened). Now in order to construct $\sigma$-algebras, the following theorems are useful since they allow us to construct $\sigma$-algebras from other $\sigma$-algebras. It turns out that the intersection of $\sigma$-algebras is a $\sigma$-algebra, but not for unions. 

  \begin{theorem}[Intersection of Sigma Algebras is a Sigma Algebra]
    Let $\{\mathcal{A}_k\}$ be a family of $\sigma$-algebras of $X$. Then, $\cap \mathcal{A}_k$ is also a $\sigma$-algebra of $X$. 
  \end{theorem}
  \begin{proof}
    Clearly, $\emptyset, X$ is in $\cap \mathcal{A}_k$. To prove complementation, 
    \begin{equation}
      A \in \bigcap \mathcal{A}_k \implies A \in \mathcal{A}_k \; \forall k \implies A^c \in A_k \; \forall k \implies A^c \in \bigcap \mathcal{A}_k
    \end{equation}
    To prove countable union, let $\{A_j\}_{j \in J}$ be some countable family of subsets in $\cap \mathcal{A}_k$. Then, 
    \begin{equation}
      A_j \in \bigcap \mathcal{A}_k \; \forall j \in J \implies A_j \in \mathcal{A}_k \; \forall k \forall j \implies \bigcup A_j \in \mathcal{A}_k \; \forall k \implies \bigcup A_j \in \bigcap \mathcal{A}_k
    \end{equation}
  \end{proof}

  This allows us to easily prove the following theorem, which just establishes the existence of $\sigma$-algebras. 

  \begin{theorem}[Unique Smallest Sigma Algebra]
    Let $F \subset 2^X$. Then there exists a unique smallest $\sigma$-algebra $\sigma(F)$ containing $F$, called the $\sigma$-algebra \textbf{generated} by $F$. 
  \end{theorem}
  \begin{proof}
    Let us denote $\mathcal{M}$ as the set of all possible $\sigma$-algebras $\mathcal{B}$ of $X$. $\mathcal{M}$ is nonempty since it contains $2^X$. Then, the intersection 
    \begin{equation}
      \bigcap_{\mathcal{B} \in \mathcal{M}} \mathcal{B}
    \end{equation}
    is the unique smallest $\sigma$-algebra. 
  \end{proof} 

  With this guarantee, we can now define what it means for a set of subsets to \textit{generate} a $\sigma$-algebra. 

  \begin{definition}[$\sigma$-Algebra Generated by a Set]
    Given a collection of sets $\mathscr{C}$, the $\sigma$-algebra \textbf{generated} by $\mathscr{C}$ is the unique smallest $\sigma$-algebra containing $\mathscr{C}$, denoted $\sigma(\mathscr{C})$. 
  \end{definition} 

  This gives us a convenient way to construct $\sigma$-algebras. The general method is to identify a collection of ``important'' subsets that we would like to be included in the $\sigma$-algebra, and then just generate it.   

  \begin{definition}[Borel $\sigma$-algebra]
    The \textbf{Borel $\boldsymbol{\sigma}$-algebra} of a topological space $(X, \mathscr{T})$ is the $\sigma$-algebra generated by the topology $\mathscr{T}$, denoted $\mathcal{B}(X) \coloneqq \sigma(\mathscr{T})$. An element of the Borel algebra is called a \textbf{Borel set}. 
  \end{definition}

  Note that the Borel algebra contains: 
  \begin{enumerate}
    \item all open sets, 
    \item all closed sets due to closure under complements, 
    \item all $G_\delta$ sets due to closure under countable unions, 
    \item all $F_\sigma$ sets due to closure under countable intersection. 
  \end{enumerate}

  \begin{definition}[Measure Space]
    A \textbf{measure set} is a tuple $(X, \mathcal{A})$, where $X$ is an arbitrary space and $\mathcal{A}$ a $\sigma$-algebra. 
  \end{definition}


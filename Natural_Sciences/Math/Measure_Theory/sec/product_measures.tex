\section{Product Measures} 

  Before, we saw how we can construct measures using Carathéodory construction. We will consider how to create product measures, which will be an extension of a set functions using the Carathéodory-Hahn theorem. 

  \begin{definition}[Measurable Rectangle]
    Let $(X, \mathcal{A}, \mu)$, $(Y, \mathcal{B}, \nu)$ be two measure spaces. Consider the product space $X \times Y$. If $A \in \mathcal{A}, B \in \mathcal{B}$, then the set $A \times B$ is called a \textbf{measurable rectangle}. 
  \end{definition}

  \begin{lemma} 
    Let $\{A_k \times B_k\}_{k=1}^\infty$ be a countable disjoint collection of measurable rectangles whose union is also a measurable rectangle $A \times B$. Then, 
    \begin{equation}
      \mu(A) \times \nu(B) = \sum_{k=1}^\infty \mu(A_k) \times \nu(B_k)
    \end{equation}
  \end{lemma}
  \begin{proof}
    
  \end{proof}

  We want to set up the conditions to invoke the Carathéodory-Hahn theorem. This naturally leads to the following. 

  \begin{theorem}
    Let $\mathcal{R}$ be the collection of measurable rectangles in $X \times Y$ and for a measurable rectangle $A \times B$, define 
    \begin{equation}
      \lambda(A \times B) = \mu(A) \cdot \nu(B)
    \end{equation}
    Then, $\mathcal{R}$ is a semiring and $\lambda: \mathcal{R} \to [0, +\infty]$ is a premeasure. 
  \end{theorem}
  \begin{proof}
    
  \end{proof}

  \begin{definition}[Product Measure]
    Let $(X, \mathcal{A}, \mu)$, $(Y, \mathcal{B}, \nu)$ be two measure spaces, $\mathcal{R}$ the collection of measurable rectangles contained in $X \times Y$, and $\lambda$ the premeasure defined on $\mathcal{R}$ by 
    \begin{equation}
      \lambda(A \times B) = \mu(A) \cdot \nu(B)
    \end{equation}
    for all $A \times B \in \mathcal{R}$. Then, the \textbf{product measure} $\lambda = \mu \times \nu$ is the Carathéodory extension of $\lambda: \mathcal{R} \to [0, +\infty]$ defined on the $\sigma$-algebra of $(\mu \times \nu)^\ast$-measurable subsets of $X \times Y$. 
  \end{definition}

  \begin{theorem}[Fubini's Theorem]
    Let $(X, \mathcal{A}, \mu)$, $(Y, \mathcal{B}, \nu)$ be two measure spaces and $\nu$ be complete. Let $f$ be integrable over $X \times Y$ w.r.t. to the product measure $\mu \times \nu$. Then for almost all $x \in X$, the $x$-section of $f$, $f(x, \cdot)$, is integrable over $Y$ with respect to $\nu$, and 
    \begin{equation}
      \int_{X \times Y} f \, d(\mu \times \nu) = \int_X \bigg[ \int_Y f(x, y) \,d\nu(y) \bigg] \, d\mu(x)
    \end{equation}
  \end{theorem}

  \begin{theorem}[Tonelli's Theorem]
    Let $(X, \mathcal{A}, \mu)$, $(Y, \mathcal{B}, \nu)$ be two measure spaces and $\nu$ be complete.Let $f$ be a nonnegative $(\mu \times \nu)$-measurable function on $X \times Y$. Then, 
    \begin{enumerate}
      \item For almost all $x \in X$, the $x$-section of $f$, $f(x, \cdot)$, is $\nu$-measurable. 
      \item The function defined almost everywhere on $x$ by 
        \begin{equation}
          x \mapsto \int_Y f(x, y) \,d\nu(y)
        \end{equation}
        is $\mu$-measurable. 
      \item Finally, 
        \begin{equation}
          \int_{X \times Y} f \,d(\mu \times \nu) = \int_X \bigg[ \int_Y f(x, y) \,d\nu(y) \bigg] \,d \mu(x)
        \end{equation}
    \end{enumerate}
  \end{theorem}

\subsection{Exercises} 

  \begin{exercise}[Math 631 Fall 2025, Final Exam Exercise 4]
    Let $A=[-1,1]\times[-1,1]$. Let $f(x,y)=\frac{xy}{(x^{2}+y^{2})^{2}}$ (when $x=0$ or $y=0$ we set $f(x,y)=0$). Prove that the iterated integrals exist and are equal:
    \begin{equation}
      \int_{-1}^{1}(\int_{-1}^{1}f(x,y)dx)dy=\int_{-1}^{1}(\int_{-1}^{1}f(x,y)dy)dx,
    \end{equation}
    but the double integral $\int_{A}f(x,y) dxdy$ does not exist.
  \end{exercise}
  \begin{solution}
    Note that $f$ is odd with respect to both $x$ and $y$. Therefore
    \begin{equation}
      \int_{-1}^{1}f(x,y)dx = \int_{-1}^{1}f(x,y)dy=0
    \end{equation}
    for all $y$ and $x$ respectively. Hence both iterated integrals are zero. Now passing to polar coordinates, we have that $|f(r,\theta)|=\frac{|\sin 2\theta|}{2r^{2}}$. Therefore, for every $\epsilon>0$,
    \begin{equation}
      \int_{A}|f(x,y)|dxdy\ge\int_{\epsilon}^{1}\int_{0}^{\pi/2}\frac{\sin 2\theta}{r^{2}}rd\theta dr\ge c\int_{\epsilon}^{1}r^{-1}dr.
    \end{equation}
    Since $\int_{\epsilon}^{1}r^{-1}dr$ diverges as $\epsilon\rightarrow0$, $f$ is not integrable over $A$.
  \end{solution}

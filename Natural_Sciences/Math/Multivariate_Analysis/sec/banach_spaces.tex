\section{Banach Spaces}  

  Note that both the domain and codomain of a function $f: \mathbb{R}^n \to \mathbb{R}^m$ are vector spaces. However, Euclidean spaces have a lot of structure on them, and it is nice to identify the essential properties we need from these sets. It is essential that we work in vector spaces because when we define a derivative, we usually see some form that looks like 
  \begin{equation}
    \frac{f(x + h) - f(x)}{h}
  \end{equation} 
  Note the operations used here. First, we want a notion of addition in the domain ($x + h$) and the codomain $f(x + h) - f(x)$, along with some scalar multiplication when we multiply by $1/h$. A vector space precisely supports these operations and therefore is a natural choice. It is immediate that to define convergence, we definitely need a topology. We will see later that we want to define multivariate derivatives by adding a norm to this term, requiring the use of a normed vector space. Completion is clearly essential as we have seen in single-variable analysis. 

  \begin{definition}[Banach Space]
    A \textbf{Banach space} is a normed completed vector space. 
  \end{definition}

  Note that by extending the dimension, we have essentially lost the ordering $\leq$ on these spaces, along with the field properties. Therefore, we will need to adapt our definitions accordingly. 

\subsection{Coordinate Systems}  

  Frenet frame? 

In here, we extend the results of univariate real analysis to multivariate and/or vector-valued functions. In practice, multivariate calculus is used, and there are many new results that arise in the multivariate case. The case for continuity and convergence of multivariate functions is very straightforward, since these are topological properties. However, the definition of the derivative and the integral will need to be generalized. 


In continuity, to prove that a limit is something, we just use $\epsilon$-$\delta$. However, to actually \textit{compute} what the limit is, we have multiple ways to do this in practice. 
\begin{enumerate}
  \item Just compute the function assuming it is continuous. 
  \item Take some sort of path $p$ and take the univariate limit. 
\end{enumerate}

We can also show that the multivariate limit doesn't exist by taking two sequences or two path functions where the limits do not equal each other. 



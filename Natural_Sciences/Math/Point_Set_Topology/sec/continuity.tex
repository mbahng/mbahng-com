\section{Limits and Continuity} 

\subsection{Limits of Sequences} 

  Recall what a \hyperref[st-def:sequence]{sequence} is. 

  \begin{definition}[Convergence of Sequence in Topological Space][def:sequence]
    A sequence $(x_n)$ of points in topological space $(X, \T)$ is said to \textbf{converge} to the point $x \in X$ if $\forall$ open neighborhoods $U_x \ni x$, $\exists N \in \mathbb{N}$ such that
    \begin{equation}
      n \geq N \implies x_n \in U 
    \end{equation}
    If there exists no limit, then $(x_n)$ is said to \textbf{diverge}. 
  \end{definition}

  Note that $x$ being the limit of a sequence $(x_i)$ is stronger than the claim that $x$ is a limit point of $\{x_i\}$. If we consider the sequence $0, 1, 0, 1, \ldots$, we can see that both $0$ and $1$ are limit points, but the limit does not exist. We would like to define some notion of limit points in the language of sequences. We can precisely do this by treating a sequence as a set and talking about subsequential limits. 

  \begin{definition}[Subsequences]
    A \textbf{subsequence} of $(x_n)_n$ is a sequence $(x_{n_k})_k$, where $(n_k)_k$ is a strictly increasing infinite subsequence of $1, 2, 3, \ldots$. 
  \end{definition}

  \begin{definition}[Partial Limits]
    The \textbf{partial limit} of a sequence $(x_n)$ is the limit of any of its subsequence.  

    \begin{figure}[H]
      \centering 
      \begin{tikzpicture}[scale=0.8]
        % Define the axis
        \draw[->] (0,0) -- (8,0) node[right] {$n$};
        \draw[->] (0,0) -- (0,4.5) node[above] {$a_n$};
        
        % Draw grid lines
        \foreach \x in {1,2,3,4,5,6,7}
            \draw[dotted] (\x,0) -- (\x,4);
        \foreach \y in {1,2,3,4}
            \draw[dotted] (0,\y) -- (7,\y);
            
        % Draw tick marks on x-axis
        \foreach \x in {1,2,3,4,5,6,7}
            \draw (\x,0.1) -- (\x,-0.1);
            
        % Draw tick marks on y-axis
        \foreach \y in {1,2,3,4}
            \draw (0.1,\y) -- (-0.1,\y);
        
        % Plot the sequence values
        \filldraw[black] (1,0.5) circle (3pt);
        \filldraw[black] (2,1) circle (3pt);
        \filldraw[black] (3,0.165) circle (3pt);
        \filldraw[black] (4,2) circle (3pt);
        \filldraw[black] (5,0.1) circle (3pt);
        \filldraw[black] (6,3) circle (3pt);
        \filldraw[black] (7,0.07) circle (3pt);
      \end{tikzpicture}
      \caption{Two partial limits of the sequence $a_n = 1/n$ for $n$ odd and $n/2$ for $n$ even, is $+\infty$ and $0$. } 
      \label{fig:partial_limit}
    \end{figure}
  \end{definition}

  \begin{lemma}[Partial Limits Equivalent to Limit Point]
    Given a sequence $(x_n)$, $x$ is a limit point of $(x_n)$ iff there exists a subsequence of $(x_n)$ that converges to $x$. 
  \end{lemma}

  \begin{theorem}[Convergence of Sequence in Metric Space]
    Let $(X, d)$ be a metric space. $x$ is the limit of $(x_n)$ if it satisfies one of the two equivalent conditions: 
    \begin{enumerate}
      \item if $x_n \to x$ under the metric topology generated by $d$
      \item $\forall \epsilon > 0$, $\exists N \in \mathbb{N}$ such that
      \begin{equation}
        n \geq N \implies d(x_n, x) < \epsilon 
      \end{equation}
    \end{enumerate}
  \end{theorem}

\subsection{Limits of Functions}

  Now we talk about limits of functions. We will talk about a variable $x$ approaching a particular value $a \in X$, denoted $x \to a$. But this isn't clear. When we talk about the concept of something approaching another thing, we have established two definitions. 
  \begin{enumerate}
    \item A \textit{sequence} can approach to its limit, which is a \textit{point}. 
    \item A \textit{point} can be a limit point of a \textit{set}. 
  \end{enumerate} 
  When we write $x \to p$, we are talking about some indeterminate variable $x$ and a point $p$, it isn't immediately clear what this means. As we will soon define, this will refer to a neighborhood of $p$ or equivalently to \textit{all} sequences converging to $p$. So informally, we can think of $x \to p$ as notation for all sequences $(x_n) \to p$.   

  \begin{definition}[Limit of Function between Topological Spaces][def:limfunc-top]
    Let $f: E \subset X \to Y$ be a map between topological spaces and $p \in X$ be a limit point of $E$. The limit of $f$ at $x$ is any\footnote{Note that this limit $q$ may not be unique unless $Y$ is Hausdorff.} point $q \in Y$ satisfying the following: For all open neighborhoods $V_q \subset Y$ of $q$, there exists a punctured open neighborhood $\mathring{U}_p \subset X$ of $x$ s.t. 
    \begin{equation}
      f\big( \mathring{U}_p \cap E \big) \subset V_q
    \end{equation}
  \end{definition}

  Note that while the definition may look technically complicated, it makes sense. First, we want $p$ to be a limit point since a function can ``tend toward'' some boundary. We also want to take the punctured open neighborhood to ensure that $x \neq p$, since functions can jump at $p$. Finally, we want to map $\mathring{U}_p \cap E$ since that is what $f$ is defined over. 

  \begin{theorem}[Limit of a Function between Metric Spaces][def:limfunc-met]
    Let $f: E \subset X \to Y$ be a map between metric spaces and $p \in X$ be a limit point of $E$. We say $f(x) \to q$ as $x \to p$, i.e. 
    \begin{equation}
      \lim_{x \to p} f(x) = q
    \end{equation} 
    if it meets the following equivalent conditions. 
    \begin{enumerate}
      \item \textit{$\epsilon$-$\delta$ Definition}. If $\forall \epsilon > 0$, $\exists \delta > 0$ s.t. $0 < d_X (x, p) < \delta \implies d_Y (f(x), q)) < \epsilon$.

      \begin{figure}[H]
        \centering 
        \begin{tikzpicture}
          \node at (0, 3) {$X$};
          \node at (6.5, 3) {$Y$};
          
          % Function arrow
          \draw[->, thick, bend left=20] (2.5,1.5) to node[above] {$f$} (6.5,1.5);
          
          % Blue blob 1 - made bigger but still in top left
          \draw[blue, thick, dashed] plot [smooth cycle, tension=0.8] coordinates {(0.2,1.7) (0.6,1.2) (1.0,0.9) (1.5,1.0) (1.8,1.4) (1.6,1.9) (1.3,2.3) (0.9,2.5) (0.5,2.4) (0.2,2.2)};
          \fill[blue, pattern=north west lines, pattern color=blue, opacity=0.2] plot [smooth cycle, tension=0.8] coordinates {(0.2,1.7) (0.6,1.2) (1.0,0.9) (1.5,1.0) (1.8,1.4) (1.6,1.9) (1.3,2.3) (0.9,2.5) (0.5,2.4) (0.2,2.2)};
          \node at (1.0,0.4) [blue] {$f^{-1}(B_\epsilon(q))$};
          
          % Blue blob 2 - in bottom right
          \draw[blue, thick, dashed] plot [smooth cycle, tension=0.7] coordinates {(1.8,0.8) (2.2,0.6) (2.6,0.7) (2.8,1.0) (2.7,1.3) (2.3,1.2) (2.0,1.1)};
          \fill[blue, pattern=north west lines, pattern color=blue, opacity=0.2] plot [smooth cycle, tension=0.7] coordinates {(1.8,0.8) (2.2,0.6) (2.6,0.7) (2.8,1.0) (2.7,1.3) (2.3,1.2) (2.0,1.1)};
          
          % Left neighborhood (red, hatched circle) - moved inside V1
          \draw[red, thick, dashed] (1.2,1.6) circle (0.4);
          \fill[red, pattern=north east lines, pattern color=red, opacity=0.3] (1.2,1.6) circle (0.4);
          \node at (2.2,2) [red] {$\mathring{B}_\delta(p)$};
          
          % Point p - hollow and red - also moved
          \draw (1.2,1.6) circle (0.05);
          \node at (1.2,1.6) [below right] {$p$};
          
          % Right neighborhood (blue circle)
          \draw[blue, thick, dashed] (8,1.5) circle (1);
          \fill[blue, pattern=north west lines, pattern color=blue, opacity=0.2] (8,1.5) circle (1);
          \node at (8.5,2.6) [blue] {$B_\epsilon(q)$};
          
          % Point q - where f(p) would be, now just a point reference
          \draw (8,1.5) circle (0.05);
          \node at (8,1.5) [below right] {$q$};
          
          % Epsilon visualization - dotted line segment not touching the blue boundary
          \draw[blue, thick, dotted] (8,1.5) -- (9,1.5) node[midway, above] {$\varepsilon$};
        \end{tikzpicture}
        \caption{Said in one line, the preimage of any open ball around $y = f(x)$ must contain some open deleted open ball around $x$.} 
        \label{fig:limit_function}
      \end{figure}

      \item \textit{Sequential Definition}. If for all sequences $(x_n) \to p$, $f(x_n) \to q$.

      \begin{figure}[H]
        \centering 
        \begin{tikzpicture}[scale=1]
          % Space labels without axes
          \node at (1, 3) {$X$};
          \node at (7.5, 3) {$Y$};
          
          % Function arrow
          \draw[->, thick, bend left=20] (3.5,1.5) to node[above] {$f$} (7.5,1.5);
          
          % Point a in domain - shifted by 1
          \fill (2.5,1.5) circle (0.07);
          \node at (2.5,1.5) [above right] {$p$};
          
          % Blue sequence (x_n) in domain - curved path - shifted by 1
          \filldraw[blue] (1.3,0.6) circle (0.03);
          \filldraw[blue] (1.7,0.8) circle (0.03);
          \filldraw[blue] (1.9,1.0) circle (0.03);
          \filldraw[blue] (2.1,1.3) circle (0.03);
          \filldraw[blue] (2.3,1.4) circle (0.03);
          \node at (1.6,0.4) [blue] {$(x_n)$};
          
          % Red sequence (y_n) in domain - curved path - shifted by 1
          \filldraw[red] (2.7,1.6) circle (0.03);
          \filldraw[red] (2.9,1.7) circle (0.03);
          \filldraw[red] (3.0,1.8) circle (0.03);
          \filldraw[red] (3.2,1.85) circle (0.03);
          \filldraw[red] (3.5,1.9) circle (0.03);
          \node at (3.4,2.0) [red] {$(y_n)$};
          
          % Point A in codomain
          \fill (8,1.5) circle (0.07);
          \node at (8,1.5) [above right] {$q$};
          
          % Blue sequence (f(x_n)) in codomain - curved path
          \filldraw[blue] (7.1,0.7) circle (0.03);
          \filldraw[blue] (7.3,0.9) circle (0.03);
          \filldraw[blue] (7.5,1.1) circle (0.03);
          \filldraw[blue] (7.7,1.2) circle (0.03);
          \filldraw[blue] (7.8,1.4) circle (0.03);
          \node at (7.2,0.4) [blue] {$(f(x_n))$};
          
          % Red sequence (f(y_n)) in codomain - curved path
          \filldraw[red] (8.2,1.7) circle (0.03);
          \filldraw[red] (8.5,1.8) circle (0.03);
          \filldraw[red] (8.7,2.0) circle (0.03);
          \filldraw[red] (8.9,2.3) circle (0.03);
          \filldraw[red] (9.1,2.5) circle (0.03);
          \node at (9.2,1.9) [red] {$(f(y_n))$};
        \end{tikzpicture}
        \caption{For every sequence that converges to the left, the new sequence mapped through $f$ converges to $q$. Note that we choose the points $x_n$ to be in the "deleted" neighborhood $E\setminus a$ (neighborhood $E$ with point $a$ removed) to force us to choose a sequence that is not $a, a, \ldots$. That is, it forces us to choose different points for the sequence. } 
        \label{fig:sequential_limit_def}
      \end{figure}
    \end{enumerate}
  \end{theorem}
  \begin{proof}
    We prove equivalence. 
    \begin{enumerate}
      \item $(\to)$. Assume $\lim_{x \to p} f(x) = q$. Let $(x_n) \in E$ s.t. $x_n \to p$ with $x_n \neq p$. We wish to show that $f(x_n) \to q$. Let $\epsilon > 0$. Then $\exists \delta > 0$ s.t. $0 < d_X (x, p) < \delta \implies d_Y (f(x), q) < \epsilon$. Since $\delta > 0$, by definition $\exists N \in \mathbb{N}$ s.t. if $n \geq N$, $d_X (x_n , p) < \delta \implies d_Y (f(x_n), q)$. 
    \end{enumerate}
  \end{proof}

  Sometimes, the $\epsilon$-$\delta$ definition is good, but a lot of the times the sequential definition is good enough and more insightful. Note also that the topological definition of a limit does not include the sequential definition because it is not true.\footnote{\href{https://math.stackexchange.com/a/3151525/616717}{https://math.stackexchange.com/a/3151525/616717}}

  \begin{example}[Counterexample]
    
  \end{example}

\subsection{Continuous Functions}

  Note that from set theory, we can construct functions as a subset of Cartesian product of two spaces $X, Y$. There is nothing new here. 

  \begin{definition}[Continuous Function]
    A function $f$ between 2 topological spaces $(X, \T_{X})$ and $(Y, \T_{Y})$ is \textbf{continuous at $x \in X$} if the preimage of every open neighborhood of $f(x) \in Y$ is an open neighborhood of $x \in X$.
    \begin{equation}
      U_{f(x)} \in \T_{Y} \implies x \in f^{-1}(U_{f(x)}) \in \T_{X}
    \end{equation} 
    $f$ is said to be \textbf{continuous} (at all points) if the preimage of every open set in $Y$ is an open set in $X$.\footnote{Note that continuity of a function $f$ is not only determined by the function itself, but also by the topologies of $X$ and $Y$.}
  \end{definition}

  Note that it is easier for $f$ to be continuous when the $\T_X$ is finer (since there are more open sets in $X$ for the preimage of $V \subset Y$ to map to) or $\T_Y$ is coarser (since there are fewer open sets that we have to check to map to open sets of $X$). 

  \begin{theorem}[Sufficient Properties for Continuity]
    Let $X, Y$, be topological spaces and let $f: X \longrightarrow Y$. Then, the following are equivalent to $f$ being continuous. 
    \begin{enumerate}
      \item The preimage of every basis element $B \in \T_Y$ is open in $X$. 
      \item For every closed set $B$ in $Y$, the set $f^{-1} (B)$ is closed in $X$. 
      \item For every subset $A$ of $X$, $f(\bar{A}) \subset \bar{f(A)}$. 
    \end{enumerate}
  \end{theorem}  
  \begin{proof}
    Listed. 
    \begin{enumerate}
      \item An arbitrary open set $V$ of $Y$ can be written as $V = \cup_{\alpha \in J} b_\alpha$. Then, 
      \begin{equation}
        f^{-1} (V) = f^{-1} \Big( \bigcup_{\alpha \in J} b_\alpha \Big) = \bigcup_{\alpha \in J} f^{-1} (b_\alpha)
      \end{equation}
    \end{enumerate}
  \end{proof}

  Great, so we have a few ways in which we can check continuity of a function. There are a few special cases. 

  \begin{lemma}[Trivially Continuous Functions]
    We have the following for general topological spaces. 
    \begin{enumerate}
      \item The identity function $\id: (X, \T_1) \rightarrow (X, \T_2)$ is continuous if $\T_1 \supset \T_2$. 
      \item A constant function $f: (X, \T) \rightarrow (Y, \T_2)$ is always continuous, regardless of the topologies. 
    \end{enumerate}
  \end{lemma}
  \begin{proof}
    If we take an open set $U \in \T_2$, its preimage is the same set $U$, which is guaranteed to be in $\T_1$ since $\T_1$ is finer than $\T_2$. 
  \end{proof}

\subsection{Construction of Continuous Functions} 

  \begin{theorem}[Arithmetic on Real Continuous Functions]
    If $X$ is a topological space, and if $f, g: X \longrightarrow \mathbb{R}$ are continuous, then $f + g$, $f-g$, and $f \cdot g$ are also continuous. $f / g$ is continuous if $g(x) \neq 0$ for all $x \in X$. 
  \end{theorem}

  \begin{theorem}[Analytic Continuity = Topological Continuity] 
    Given metric spaces with their induced metric topologies $(X, \T_X, d_X)$ and $(Y, \T_Y, d_Y)$. The following are equivalent. 
    \begin{enumerate}
      \item $f: X \rightarrow Y$ is continuous at $x$. 
      \item For every $\delta > 0$, there exists an $\epsilon = \epsilon(\delta) > 0$ such that for all $z \in X$, $d_X (x, z) < \epsilon \implies d_Y (f(x), f(x)) < \delta$.\footnote{This is the definition of continuity at a point in analysis.} 
    \end{enumerate}
  \end{theorem}
  \begin{proof}
    ($\rightarrow$) Assume $f$ is continuous according to the $\epsilon - \delta$ definition. Let $U$ be any open set in $Y$ containing the point $y$, and let $x$ be an element in $f^{-1}(U)$ such that $y = f(x)$. We must prove that $f^{-1}(U)$ is also open. Since open sets contain neighborhoods (e.g. open balls) of all of its points, we can claim that, since $U$ is open by assumption, there exists an open ball $B_y$ around $y$ with radius $\epsilon > 0$. This guarantees the existence of a point $z \in U$ such that $\rho(y, z) < \epsilon$ for any $\epsilon > 0$ that we choose. Since $f$ is continuous, for every $\epsilon >0$ that we chose previously, there exists a $\delta >0$ such that $d(x, w) \implies \rho(f(x), f(w)) < \epsilon$. Since $\rho(f(x), f(w)) < \epsilon$, we can conclude that $f(w) \in B_y \subset U$ when $d(x, w) < \delta$. Therefore, $d(x, w) < \delta \implies w \in f^{-1}(U)$. But this is equivalent to saying that if $w \in B_(x, \delta)$, then $w \in f^{-1}(U)$, which means that every single point $x \in f^{-1}(U)$ contains an open ball neighborhood fully contained in $f^{-1}(U)$. So, by definition, $f^{-1}(U)$ is open. 


    ($\leftarrow$) Assume $f^{-1}(U)$ is open when $U$ is an open set in $Y$, i.e. $f$ is continuous under the topological definition. Let us define the open ball 
    \begin{equation}
      B(f(x), \epsilon) \coloneqq \{ y \in Y \mid \rho(f(x), y) < \epsilon\} \in \mathscr{T}_Y
    \end{equation}
    By our assumption, $f^{-1} \big( B(f(x), \epsilon) \big)$ is an open set in $\mathscr{T}_X$, and clearly, $x \in f^{-1} \big( B(f(x), \epsilon) \big)$ since $f^{-1}$ maps the point $f(x) \in B(f(x), \epsilon)$ to $x \in f^{-1} \big( B(f(x), \epsilon) \big)$. But since $f^{-1} \big( B(f(x), \epsilon) \big)$ is open, we can construct an open ball around $x$ with radius $\delta$ fully contained within the open set. Moreover, by selecting a point $p \in B(f(x), \delta) \subset f^{-1}\big( B(f(x), \epsilon) \big)$, we can guarantee that $f(p) \in B(f(x), \epsilon)$. This is precisely the $\epsilon - \delta$ definition of continuity. That is, given an $\epsilon > 0$ to be the radius of an open ball $B(f(x), \epsilon)$ in $Y$, we can always choose a $\delta > 0$ to be the radius of the open ball $B(x, \delta)$ in $X$ that is fully contained within the preimage of $B(f(x), \epsilon)$. In mathematical notation, 
    \begin{equation}
      p \in B(x, \delta) \subset f^{-1} \big( B(f(x), \epsilon) \big) \implies f(p) \in f\big( B(x, \delta) \big) \subset B(f(x), \epsilon)
    \end{equation}
    or equivalently in terms of metrics,
    \begin{equation}
      d(x, p) < \delta \implies \rho (f(x), f(p)) < \epsilon
    \end{equation}
  \end{proof} 

  \begin{lemma}[Composition of Continuous Functions]
    If $f: X \rightarrow Y$ and $g: Y \rightarrow Z$ is continuous, then $g \circ f :X \rightarrow Z$ is continuous. 
  \end{lemma}

\subsection{Open and Closed Maps}

  Open and closed functions map open/closed sets to open/closed sets, unlike continuous functions which take the preimage. However, they do are not natural and most maps are not open nor closed, so this is a pretty special condition. 

  \begin{definition}[Open, Closed Maps]
    A map $f: X \rightarrow Y$ is said to be 
    \begin{enumerate}
      \item \textbf{open} if it maps open sets of $X$ to open sets of $Y$. 
      \item \textbf{closed} if it maps open sets of $X$ to closed sets of $Y$. 
    \end{enumerate}
    Note that open and closed maps are completely independent. A map may be open, closed, neither, or both. 
  \end{definition}

  \begin{example}[Open but Not Closed]
    The projection $\pi_1: X \times Y \rightarrow X$ is an open map but but closed. Consider $\pi_1: \mathbb{R}^2 \rightarrow \mathbb{R}$ with $S = \{(x, y) \in \mathbb{R}^2 \mid xy = 1 \}$. Then $\pi_1 (S) = \mathbb{R} \setminus \{0\}$, which is not closed.\footnote{In open maps, the typical behavior is that points are ``copied,'' i.e. for projections, the preimage of $\pi_1^{-1} (x) = x \times Y$, where all $y \in Y$ are copied.}
  \end{example}

  \begin{example}[Closed but Not Open]
    $f: \mathbb{R} \rightarrow \mathbb{R}$ with $f(x) = x^2$ is closed but not open since $f(\mathbb{R}) = [0, +\infty)$ which is not open. 
  \end{example}

\subsection{Homeomorphisms}

  \begin{definition}[Homeomorphism]
    A bijective, bicontinuous function $f: X \longrightarrow Y$ between two topological spaces is called a \textbf{homeomorphism} between $X$ and $Y$. If there exists at least one homeomorphism between $X$ and $Y$, then $X$ is said to be \textbf{homeomorphic} to $Y$, denoted $X \cong Y$. 

    \begin{figure}[H]
      \centering 
      \includegraphics[scale=0.4]{img/Homeomorphism_of_Plane.png}
      \caption{The visual below shows a homeomorphism between the plane $X$ and the surface $Y$.}
      \label{fig:homeomorphism_plane}
    \end{figure}
  \end{definition}

  \begin{theorem}[Sufficient Properties of Homeomorphism]
    Suppose $f: X \rightarrow Y$ is a bijection. TFAE. 
    \begin{enumerate}
      \item $U \subset Y$ is open iff $f^{-1} (U)$ is open. 
      \item $U \subset X$ is open iff $f(U)$ is open. 
      \item $f$ is a homeomorphism. 
    \end{enumerate}
  \end{theorem} 

  Note that we may have functions that are bijective and continuous, but not bicontinuous. In order to construct such examples one of the easiest things we can do is just endow the codomain with the discrete topology, which guarantees continuity. 

  \begin{example}[Bijective and Continuous but not Homeomorphism]
    $\mathbb{Z}$ and $\mathbb{Q}$ are countable sets, so there is a bijection between them. If we give each of them the metric topology, $\mathbb{Z}$ ends up having the discrete topology (take the $0.5$-ball around each integer), whereas for $\mathbb{Q}$, we will see later that by the density of the rationals there are an infinite number of rationals in $(q - r, q + r)$ for $q \in \mathbb{Q}$. Note that this bijection $f: \mathbb{Z} \rightarrow \mathbb{Q}$ is continuous (since $\mathbb{Z}$ has discrete topology) but not bicontinuous. 
  \end{example}


  \begin{example}[Comparability and Homeomorphic Spaces]
    Consider the set $X = \{a, b\}$ with the two topologies $\T_3 = \{\emptyset, \{a\}, X\}$ and $\T_4 = \{\emptyset, \{b\}, X\}$. They are not comparable but they seem ``similar'' in a way in that if we swap all the $a$'s and $b$'s in $\T_3$, then we get $\T_4$. We can make this rigorous by defining $f: (X, \T_3) \rightarrow (X, \T_4)$ with $f(a) = b, f(b) = a$, and showing that it is a homeomorphism. 
  \end{example}

  In fact, a homeomorphism $f$ is an equivalence relation between two topological spaces. This partitions the set of all topological spaces into \textbf{homeomorphism classes}. Analogous to how isomorphisms preserve algebraic structures, homeomorphisms preserve topological structure between topological spaces. 

  \begin{example}[Homeomorphism Classes of 2D Manifolds]
    There is an infinite family of 2-dimensional manifolds, call them $M$ and $N$, and each set in each family is not homeomorphic to another.  
    \begin{enumerate}
      \item $M_0 = S^2$ (sphere). $M_1 = T^2$ (torus). $M_2$ is a donut with two holes. $M_3$ has three holes, and so on. 
      \item $N_1$ is the Mobius strip. $N_2$ is the Klein bottle. 
    \end{enumerate}
  \end{example}

  Additionally, not only does a homeomorphism give a bijective correspondence between points in $X$ and $Y$, but it also determines a bijection between \textbf{the set of all open sets in $X$ and $Y$} (that is, a bijection between their topologies)! This bijection then allows two spaces that are homeormophic to have the same topological properties. 

  \begin{theorem}[Preservation of Topological Properties]
    A homeomorphism $f$ between two topological spaces $(X, \mathscr{T}_{x})$ and $(Y, \mathscr{T}_{Y})$ preserves all topological properties (e.g. separability, countability, compactness, (path) connectedness) of $X$ onto $Y$ and $Y$ onto $X$. 
  \end{theorem}

  \begin{definition}[Embedding]
    Suppose that $f: X \longrightarrow Y$ an injective continuous map with $X, Y$ topological spaces. Let $Z \coloneqq \im{f}$. Then, the function
    \begin{equation}
      f^\prime: X \longrightarrow Z \subset Y
    \end{equation}
    obtained by restricting the codomain of $f$ is bijective. If $f^\prime$ happens to be a homeomorphism of $X$ with $Z$, then we say that the map
    \begin{equation}
      f: X \longrightarrow Y
    \end{equation}
    is a \textbf{topological embedding}, or more simply an \textbf{embedding}, of $X$ in $Y$. 
  \end{definition} 

  A homeomorphism can be useful, but we can work a lot more flexibly with it by knowing that the restriction of a homeomorphism is a homeomorphism. 

  \begin{theorem}[Restriction of Homeomorphism is Homeomorphism]
    If $f: X \rightarrow Y$ is a homeomorphism, then for any $x \in X$, the restriction 
    \begin{equation}
      f|_{X \setminus \{x\}} : X \setminus \{x\} \rightarrow Y \setminus \{f(x)\}
    \end{equation}
    is also a homeomorphism. 
  \end{theorem}

\subsection{Local Homeomorphisms} 




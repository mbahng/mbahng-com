\section{Countability}

  \begin{definition}[1st-Countability]
    A space $X$ is said to have a countable basis at $x$ if there exists a sequence $N_1, N_2, ...$ of open neighborhoods of $x$ such that for any neighborhood $N$ of $x$, there exists an integer $i$ such that $N_i \in N$. That is, the countable basis of neighborhoods get arbitrarily small around $x$. A space $X$ satisfying this axiom at every point $x \in X$ is said to be a \textbf{first-countable space}. 
  \end{definition}

  In particular, every metric space is first-countable, since we can construct the sequence of open balls $B (x, \frac{1}{n})$ for each $n \in \mathbb{N}$ which forms a countable basis at $x$. We now generalize some previous statements about metric spaces to statements about first-countable spaces. 

  \begin{theorem}
    Let $X$ be a space satisfying the first countability axiom, and let $A \subset X$. 
    \begin{enumerate}
      \item $x \in \bar{A}$ if and only if there exists a sequence of points in $A$ converging to $x$. 
      \item The function $f: X \longrightarrow Y$ is continuous if and only if for every convergent sequence $(x_n) \rightarrow x$ in $X$, the sequence $\big( f(x_n)\big) \rightarrow f(x)$ in $Y$. 
    \end{enumerate}
  \end{theorem}

  \begin{definition}[2nd-Countability]
    A topological space $X$ is said to satisfy the \textbf{second countability axiom} if $X$ has a countable basis for its topology.
  \end{definition}

  \begin{proposition}
    Second countability implies first countability. 
  \end{proposition}
  \begin{proof}
    If $\B$ is a countable basis for the topology of $X$, then the subset of $\B$ consisting of elements containing the point $x$ is a countable basis at $x$. 
  \end{proof}

  \begin{example}
    The real line $\mathbb{R}$ is second countable. We can contrust a countable basis as the set of all open intervals $(a, b)$ with rational end points. Likewise, $\mathbb{R}^n$ has a countable basis, which is the collection of all products of intervals having rational end points. Additionally, $\mathbb{R}^\omega$ has a countable basis. It is the collection of all products
    \begin{equation}
      \prod_{n \in \mathbb{N}} U_n
    \end{equation}
    where $U_n$ is an open interval with rational endpoints for finitely many values of $n$ and $U_n = \mathbb{R}$ for all other values of $n$. 
  \end{example}

  \begin{example}
    In the uniform topology, $\mathbb{R}^\omega$ satisfies the first countability axiom (since it is metrizable). 
  \end{example}

  \begin{theorem}
    A subspace of a first and second countable space is first and second countable, respectively. A countable product of first and second countable space is first and second countable, respectively. 
  \end{theorem}

  \begin{theorem}
    A subset $A$ of space $X$ is said to be \textbf{dense} in $X$ if $\bar{A} = X$. 
  \end{theorem}

  \begin{theorem}
    Suppose that $X$ has a countable basis. Then, 
    \begin{enumerate}
      \item Every open cover of $X$ has a countable subcover. 
      \item There exists a countable subset of $X$ which is dense in $X$. 
    \end{enumerate}
  \end{theorem}
  \begin{proof}
    Listed. 
    \begin{enumerate}
      \item Let $\B = \{B_n\}_{n \in \mathbb{N}}$ be a countable basis for $X$, and let $\mathscr{A}$ be an open covering of $X$. For each integer $n \in \mathbb{N}$, chose an element $A_n \in \mathscr{A}$ containing the basis element $B_n$. The newly formed collection $\mathscr{A}^\prime$ of all the $A_n$'s is countable since it is indexed according to a subset of $\mathbb{N}$. Furthermore, since $B_n \subset A_n$ for every $B_n$ in the basis, the $A_n$ clearly covers $X$. 

      \item From each nonempty basis element $B_n$, we choose a point $x_n$. The set 
      \begin{equation}
        D \equiv \{x_n \; | \; n \in \mathbb{N}\}
      \end{equation}
      is dense in $X$, since given any $x \in X$, every open basis element $B_x$ about $x$ intersects $D$. That is, 
      \begin{equation}
        B_x \cap D \neq \emptyset
      \end{equation}
      meaning that the set of points $x_n$ get arbitrarily close to $x$. 
    \end{enumerate}
  \end{proof}

  \begin{definition}[Lindelof Space]
    A space for which every open covering contains a countable subcovering is called a \textbf{Lindelof space}. 
  \end{definition}


Let's first talk about why we need analysis in general in the first place. Algebra allows us to define certain algebraic structures, which are essentially sets with operations. These operations are defined to have a finite number of arguments. For example, let's take a look at the negation $x \mapsto -x$ and the addition $x, y \mapsto x + y$ operations in a group $G$. We can compose these operations up to any finite length $n$, removing the parentheses due to associativity, but note that the ``sum'' below is not a single operation. It is a composition of $n-1$ operations. 
\begin{align}
  & x_1 + x_2 + \ldots + x_n \in G  \\
  & -(-(\ldots(-x))) \in G
\end{align} 
This is still well defined due to closure, but what if we wanted to do this an infinite number of times? 
\begin{align}
  x_1 + x_2 + \ldots & = ? \\
  \ldots(-(-x)) & = ? 
\end{align} 
For someone who has learned about sequences and series in high school, this may not be a big jump in logic, but it is. The objects above are not even well-defined and trying to define them with algebraic tools is equivalent to the famous Zeno's paradox. So we simply need to add more tools in order to define these new mathematical objects, which we call \textit{series}. To define series, we need to first define sequences. Can we do this with algebra? Yes, since we can simply model it as a function. 

\begin{definition}[Sequence]
  A sequence is a function $f: \mathbb{N} \rightarrow X$. We usually denote a sequence by writing out the first few terms of the sequence, followed by an ellipsis. 
  \begin{equation}
    a_1 = f(1), a_2 = f(2), \ldots
  \end{equation}
  or as an indexed set over the naturals $\{a_i\}_{i \in \mathbb{N}}$. 
\end{definition}

Therefore, we can consider series as a sequence of finite sums, each element which is well-defined. 
\begin{equation}
  x_1, x_1 + x_2, x_1 + x_2 + x_3, \ldots
\end{equation} 
For any $n \in \mathbb{N}$, we can get the value of $a_n = \sum_{i=1}^n a_i$, but can we say something about the limiting behavior of $a_n$? That is, maybe we can just slap a value $x$ onto this series such that it doesn't ``break'' any of the rules we have in the finite sense. Unfortunately, it is not possible to define such values for all series, but it is possible for some of them, which we call \textit{convergent series}. To rigorously determine which ones are convergent and which ones are not, we need the tools of topology and analysis. Defining the concept of sequences that model infinitely composed operations is what allows us to define differentiation and integration. 

Great, we've motivated the need for analysis, but before jumping straight into real analysis, let's talk about what analysis in general works with. It studies functions of the form $f: X \rightarrow Y$, and minimally both $X, Y$ must be \textit{Banach spaces}, i.e. complete normed vector spaces over some field $\mathbb{F}$. Almost all flavors of analysis, including real ($\mathbb{R}$), complex ($\mathbb{C}$), multivariate ($\mathbb{R}^n$), p-adic, and functional (infinite-dimensional Banach spaces) analysis require \textit{at least} a Banach space structure. Why are Banach spaces so great? Well if we were to define convergence in $X$ or $Y$, then it only makes sense to talk about convergence with respect to a topology. So $X, Y$ must at least be topological spaces. It would also be bad if we were to take a sequence in $X$ and find out that it converges to some element outside of $X$. Therefore, we want a notion of \textit{completeness} in the sense that all sequences that ``get closer,'' i.e. Cauchy sequences, actually converge in $X$. Unfortunately, while convergence of sequences is preserved under homeomorphisms (and is thus a topological property), convergence of Cauchy sequences is not.\footnote{Consider the sequence $a_n = 1/(n+1)$ in $(0, 1)$ and the map $f(x) = 1/x$ to the set $(1, +\infty)$. $a_n$ is Cauchy but $f(a_n)$ is not.} Furthermore, the notion of uniform convergence is a metric space property, not a topological one. Therefore, the concept of distances is crucial to the construction of analysis. As for the norm, I'm still not sure why we need this.\footnote{Aspinwall and Ng told me this, but I'm not sure why. The Frechet derivative seems like it can be purely defined with a metric. } 

But in college courses such as real and complex analysis, why do we say we work over the \textit{fields} $\mathbb{R}$ and $\mathbb{C}$ rather than the Banach spaces $\mathbb{R}$ and $\mathbb{C}$? This is because of the following theorem. 

\begin{theorem}
  Every field $\mathbb{F}$ is a $1$-dimensional vector space over itself. 
\end{theorem}

Therefore, when we talk about the \textit{field} $\mathbb{R}$, we are really treating it as a vector space $\mathbb{R}$ over the field $\mathbb{R}$.\footnote{Thanks to Prof. Lenny Ng for clarifying this.} Every other structure beyond this is a ``bonus'' property that gives us extra tools to prove stronger properties. The most notable is the total ordering on $\mathbb{R}$, which allows us to define upper/lower bounds and other real-analysis specific theorems like the intermediate value theorem or the mean value theorem. Other structures include the inner product or the measure. 

Now that we've taken in the big picture, for each type of analysis, we should construct the underlying relevant Banach space. At the very least, we can with the tools of set theory and algebra define the rationals $\mathbb{Q}$ as an ordered field over the quotient space $\mathbb{Z} \times \mathbb{Z} / \sim$. Furthermore, $\mathbb{Q}$ itself is a normed vector space (over $\mathbb{Q}$)\footnote{Note that while we define the norm and metric to usually map to $\mathbb{R}^+$, $\mathbb{R}$ isn't even defined yet and so to avoid circular definitions, we define the norm on the rationals to have codomain $\mathbb{Q}$. } and the only thing we need now is completeness. 
\begin{enumerate}
  \item If the norm on $\mathbb{Q}$ is defined as the normal absolute value (Euclidean norm), completing it gives $\mathbb{R}$ as an ordered field which also has a compatible order as that of $\mathbb{Q}$. We study functions mapping to and from $\mathbb{R}$ with \textit{single-variable real analysis}. 
  \item If we take the \textit{p-adic} norm, then completing it with respect to this gives the \textit{p-adic numbers}, which also forms a field but loses the ordering. We deal with functions over the p-adics with \textit{p-adic analysis}. 
  \item We can construct $\mathbb{C}$ by taking $\mathbb{R}^2$ and endowing it with a bit more structure. We get \textit{complex analysis}. 
  \item We can construct $\mathbb{R}^n$ and $\mathbb{C}^n$ by easily defining its vector space structure and then endowing it with a norm, and showing that it is complete with respect to the norm-induced metric. This is known as \textit{multivariate analysis}. 
  \item With all these defined, we can define Banach function spaces like $L^p$ and perform analysis on operators $f: L^p \rightarrow L^q$. This is \textit{functional analysis}. 
\end{enumerate}
What we have talked about so far was Cauchy completeness, but there is a different type of completeness called \textit{Dedekind completeness}, also equivalently known as the \textit{least-upper-bound (LUB) property}, defined only on ordered sets (with no other structure). It turns out that in an ordered field, the two forms of completeness are equivalent.\footnote{Actually, this is not true. Dedekind completeness is equivalent to Cauchy completeness plus the Archimedean property. An example of a Cauchy-complete non Archimidean field is the field $F$ of rational functions over $\mathbb{R}$, with positive cone consisting of those functions $f/g$ such that the leading coefficients of $f, g$ have the same algebraic sign. The Cauchy completion of this into the equivalence classes of Cauchy sequences in $F$ results in a non-Archimedean field. } Therefore, many real analysis textbooks tend to use Dedekind completeness when constructing the reals, but Cauchy completeness is in a sense more ``fundamental.'' We will go through both independent constructions of $\mathbb{R}$ involving both types of completeness since both are used in future theorems. 

\begin{enumerate}
  \item \textit{Construction from Cauchy Sequences}. We verify that $\mathbb{Q}$ is a field and endow it with the standard Euclidean metric $d(x, y) = |x - y|$. We can then construct a new quotient space $S$ of Cauchy sequences in $\mathbb{Q}$, define all the ordered field operations/relations, and finally show that $S$ satisfies Cauchy completeness. Most would end here and claim that this is $\mathbb{R}$, but we must also prove the Archimedean property with this order. Once done, now we can truly claim $S = \mathbb{R}$. 

  \item \textit{Construction from Dedekind Cuts}. We verify that $\mathbb{Q}$ is a field, put an order on it, and verify that it is an ordered field. We then construct a new set $D$ of \textit{Dedekind cuts} from $\mathbb{Q}$, define the compatible ordered field operations/relations, and show that this new set $D$ satisfies the least-upper bound property. We claim that $D = \mathbb{R}$. 
\end{enumerate} 


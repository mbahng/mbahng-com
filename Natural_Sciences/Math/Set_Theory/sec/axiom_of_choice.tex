\section{Axiom of Choice}

  The axioms up to this point are pretty much undisputed and completes ZF set theory. Now that we've defined a function, let's quickly extend \hyperref[def:cart-prod]{our previous definition of a Cartesian product} into an arbitrary union of sets. 

  \begin{definition}[Choice Function]
    Given a set $X$ of sets, a \textbf{choice function} of $X$ is a function $f: X \to \cup X$, that assigns each $S \in X$ to one of its elements $f(S) \in S$. 
  \end{definition}

  \begin{example}
    Given $X = \{\{1, 4, 7\}, \{9\}, \{2, 7\}\}$, one possible choice function is 
    \begin{align}
      f(\{1, 4, 7\}) & = 4 \\ 
      f(\{9\}) & = 9 \\ 
      f(\{2, 7\}) & = 2
    \end{align}
  \end{example}

 
  \begin{definition}[Cartesian Product][def:cart-prod-2]
    If $\{X_\alpha\}_{\alpha \in A}$ is an indexed family of sets, then their \textbf{Cartesian product} is defined as the set of choice functions of $\{X_\alpha\}_{\alpha \in A}$. 
    \begin{equation}
      \prod_{\alpha \in A} X_\alpha \coloneqq \bigg\{ f: A \rightarrow \bigcup_{\alpha \in A} X_\alpha \;\Big|\; \forall \alpha \in A, f(\alpha) \in X_\alpha \bigg\} 
    \end{equation}
  \end{definition} 

  Therefore, we have used the power set axiom to define a finite Cartesian product, to then define a function, to then define a general Cartesian product. But this detail is irrelevant later on. Note also that this definition of Cartesian product is not the same as that of the previous definition. The binary Cartesian product is defined as $(a, b) = \{\{a\}, \{a, b\}\}$ while this defines as a function $f: \{1, 2\} \rightarrow A, B$. But once we have overwritten the old definition (which is still necessary!) we can just forget about it and use this new definition of Cartesian product since there is a canonical bijection between them. It is a lot less annoying to think of ordered tuples as just tuples rather than as sets of sets. 

  However, in our definition, we just call this a ``set of functions'' and have never proved that it actually contains anything. But we can see obviously that if this Cartesian product is nonempty then there exists a choice function, and if there exists a choice function then the Cartesian product is nonempty. It would be ideal if we can prove one of the two conditions, but it turns out we can't, and therefore we introduce the final axiom, called the \textit{axiom of choice}. Though controversial, it is required in the proofs of some notable theorems. If we include this axiom of choice, then we have ZFC set theory. The axiom of choice has many equivalent definitions. 

  Colloquially, the axiom of choice says that a Cartesian product of a collection\footnote{Note that this does not have to be finite} of non-empty sets is non-empty. That is, it is possible to construct a new set by choosing one element from each set, even if the collection is infinite. 

  \begin{axiom}[Axiom of Choice] 
    Let $X$ be any set of nonempty sets. Then, AC states the equivalent things: 
    \begin{enumerate}
      \item There exists a choice function on $X$. 
      \item The cartesian product $\prod X$ is nonempty. 
      \item There exists a set containing exactly one element from each set in the collection. 
    \end{enumerate}
  \end{axiom} 

  The controversy around AC is that is is nonconstructive by nature, and one cannot write down a specific formula or rule to define such a choice function. This is in contrast to---say, the axiom of infinity, since you can construct such a set inductively. For example, take a look at a choice function for the set of all subsets of the reals, which is considered as an ``unruly'' set. 

  \begin{example}[Choice Function on Power Set of Reals]
    Let $I$ be the set of all nonempty subsets of $\mathbb{R}$, and $X_i = i \in I$. Then an element $f$ in $\prod_{i \in I} X_i$ is a function which picks an element $f(T) \in T$ for every nonempty $T \subset \mathbb{R}$. How do you \textit{define} such an $f$? 
    \begin{enumerate}
      \item We might say, \textit{pick the minimum element}, but subsets like $(0, 1)$ does not have a minimum. 
      \item We could write a rule that says \textit{pick 1 if it is in the set, otherwise 2, otherwise 3, and so on}, but the choice function would not be defined for subsets that don't contain any natural number, such as $\{\pi, 2 \pi, e\}$. 
    \end{enumerate}
  \end{example}

  Therefore, there is no canonical choice of an element in a nonempty set of real numbers. But AC tells us that we don't have to worry about this. It gives us such a function, even if we cannot ``write it down'' (which means, construct it from the other ZF axioms).  However, there are still sets which this is possible. 

  \begin{example}[Choice Function on Power Set of Naturals]
    Let $I$ be the set of all nonempty subsets of $\mathbb{N}$. Then, we can define a choice function $f(T) = \min(T)$, which is always defined due to the well-ordering principle. 
  \end{example}

  \begin{example}[Choice Function on Open Sets of Reals]
    If we let $I$ be the set of all nonempty \textit{open} subsets of $\mathbb{R}$, then there is a choice function. Choose any bijection $\tau: \mathbb{N} \rightarrow \mathbb{Q}$, and then assign to each nonempty open subset $U \subset \mathbb{R}$ the element $\tau (\min\{n \in \mathbb{N} \mid \tau(n) \in U\})$. This works since $U \cap \mathbb{Q} \neq \emptyset$, and by the well ordering principle, we are guaranteed a minimum element. 
  \end{example} 

\subsection{Well-Ordering and Zorn's Lemma}

  The examples indicate that we must try to find some representative element of every subset of a set. This motivates the definition, followed by two additional axioms that turn out to be equivalent to AC. 

  \begin{definition}[Well-Ordered]
    A set $X$ is \textbf{well-ordered} by a strict total order $\leq$ if every nonempty subset of $X$ has a least element under $\leq$. 
  \end{definition}

  Therefore, if the well-ordering theorem holds, then we can see that every subset of the reals has such a least element, and therefore we can construct a choice function, which supports AC. It turns out that the converse is true as well. If AC is true, we can see generally that we would like to use a choice function to select a representative element of each set in $X$. Then we can use these to construct an order. 

  \begin{axiom}[Well-Ordering Theorem]
    Every set can be well-ordered. 
  \end{axiom}

  Despite the seeming equivalence between AC and the well-ordering theorem, the this result seems to be the most counterintuitive, since it claims the existence of such a total order on $\mathbb{R}$ such that \textit{every} nonempty subset of $\mathbb{R}$ has a minimum! Nobody has been able to explicitly construct such an ordering for the reals, and at first glance, perhaps one may try to \textit{prove} that such a well-ordering cannot exist. Let's move onto the second axiom. 

  \begin{axiom}[Zorn's Lemma]
    Let $X$ be a partially ordered set that satisfies the two properties. 
    \begin{enumerate}
      \item $P$ is nonempty. 
      \item Every \textbf{chain} (a subset $A \subset P$ where $A$ is totally ordered) has an upper bound in $P$. 
    \end{enumerate}
    Then $P$ has at least one maximal element. 
  \end{axiom} 

  The validity of Zorn's lemma is a bit ambiguous, which motivates the following quote from Jerry Bona: \textit{The axiom of choice is obviously true, the well-ordering principle obviously false, and who can tell about Zorn's lemma?} Ironically, all three results turn out to be equivalent. 

  \begin{theorem}[Equivalence]
    The following are equivalent. 
    \begin{enumerate}
      \item Axiom of Choice. 
      \item Well-Ordering Theorem. 
      \item Zorn's Lemma. 
    \end{enumerate}
  \end{theorem}
  \begin{proof}
    
  \end{proof}
 
\subsection{Banach-Tarski Paradox}

  Let $G_3$ be the group of all 3-dimensional rigid transformations $x \mapsto Ax + b$, where $x \in \mathbb{R}^3, A \in \mathrm{SO}(3), b \in \mathbb{R}^3$. 

  \begin{definition}[Equidecomposability]
    Let $G$ be a group acting on set $X$. We say that $A, B \subset X$ are \textbf{$G$-equidecomposable}, written $A \sim_G B$, if both sets have a decomposition 
    \begin{equation}
      A = A_1 \cup \ldots \cup A_n, \quad B = B_1 \cup \ldots \cup B_n
    \end{equation} 
    and $A_i = g_i B_i$ for some $g_i \in G$. We claim that $\sim_G$ is an equivalence relation on $2^X$. 
  \end{definition}
  \begin{proof}
    Listed. 
    \begin{enumerate}
      \item \textit{Reflexive}. Clearly, $A = e A$ where $e \in G$ is the identity transformation. 
      \item \textit{Symmetric}. If $A \sim_G B$, then we see that $A_i = g_i B_i$, but this means that $B_i = g_i^{-1} A_i$ for $g_i^{-1} \in G$, so $B \sim_G A$. 
      \item \textit{Transitive}. If $A \sim B$, $B \sim C$, then from $A \sim B$, we have 
        \begin{equation}
          A = A_1 \cup \ldots \cup A_n, \quad B = B_1 \cup \ldots \cup B_n, \qquad A_i = f_i B_i \text{ for } f_i \in G
        \end{equation}
        From $B \sim C$, we have 
        \begin{equation}
          B = B_1^\prime \cup \ldots \cup B_m^\prime, \quad C = C_1 \cup \ldots \cup C_m, \qquad B_j^\prime = g_j C_j \text{ for } g_j \in G
        \end{equation}

        Now we can take the common partition, which can have at most $n \cdot m$ partitions. 
        \begin{equation}
          A = \bigcup_{i, j} \big( A_i \cap f_i B_j^\prime \big), \quad C = \bigcup_{i, j} \big( g_j^{-1} B_i \cap C_j \big) 
        \end{equation}
        and see that 
        \begin{align}
          C & \mapsto \bigcup_j g_j \bigg( \bigcup_i \big( g_j^{-1} B_i \cap C_j \big) \bigg) = \bigcup_j \bigcup_i \big( B_i \cap g_j C_j \big) = \bigcup_j \bigcup_i \big( B_i \cap B^\prime_j \big) \\ 
            & \mapsto \bigcup_i f_i \bigg( \bigcup_j \big( B_i \cap B^\prime_j \big) \bigg) = \bigcup_i \bigcup_j \big( f_i B_i \cap f_i B_j^\prime \big) = \bigcup_{ij} \big( A_i \cap f_i B^\prime_j \big) = A
        \end{align}
    \end{enumerate}
  \end{proof}

  \begin{definition}[Paradoxical Sets]
    Let $G$ be a group acting on set $X$, and let $E \subset X$ be nonempty. Then, $E$ is \textbf{$G$-paradoxical} if 
    \begin{equation}
      E = A \sqcup B, \qquad E \sim_G A, \quad E \sim_G B
    \end{equation}
  \end{definition}

  Note that this essentially means that $E$ can be duplicated since 
  \begin{equation}
    E \sim A \cup B \sim g_1 A \cup g_2 B \sim g_1 E \cup g_2 E
  \end{equation}

  \begin{theorem}[The Banach-Schröder-Bernstein Theorem]
    Let $G$ be a group acting on set $X$ and $A, B \subset X$. If $A$ is $G$-equidecomposable with a subset of $B$ and $B$ is $G$-equidecomposable with a subset of $A$, then $A \sim_G B$. 
  \end{theorem}

  \begin{corollary}[Conditions for Paradoxical][thm:disjoint-subset-paradoxical]
    Let $G$ be a group acting on set $X$. Then, $A \subset X$ is $G$-paradoxical if it contains disjoint subsets $A_1, A_2 \subset A$ both equidecomposable with $A$. 
  \end{corollary}

  \begin{example}[Vitali Paradox]
    Let $\mathrm{SO}(2)$ be the group of rotations in $\mathbb{R}^2$ and $S^1$ be the unit circle. For $p_1, p_2 \in S^1$, let $p_1 \sim p_2$ if the angle of rotation between them is a rational multiple of $2\pi$, which is an equivalence relation. Let us invoke the axiom of choice to define the choice set $C$ where each element contains a representative element of each equivalence class. Then, each point in $S^1$ can be expressed as an element of $C$, rotated by some rational $q \in [0, 1)$. By enumerating the rationals in the unit interval $(q_n)$, we get 
    \begin{equation}
      S^1 = q_1 C \sqcup q_2 C \sqcup q_3 C \sqcup \ldots = C_1 \sqcup C_2 \sqcup C_3 \sqcup \ldots
    \end{equation} 
    We can end up recreating $S^1$ be using only the sets of even or odd indices by applying a suitable rotation to them. 
    \begin{align}
      S^1 & = C_1 \sqcup \underbrace{C_3 + (q_2 - q_3)}_{C_2} \sqcup \underbrace{C_5 + (q_3 - q_5)}_{C_3} \sqcup \ldots \\ 
      S^1 & = \underbrace{C_2 + (q_1 - q_2)}_{C_1} \sqcup \underbrace{C_3 + (q_2 - q_3)}_{C_2} \sqcup \underbrace{C_5 + (q_3 - q_5)}_{C_3} \sqcup \ldots
    \end{align}
    Therefore, $S^1$ has a decomposition into two subsets such that each of them is ``\textit{countably} $\mathrm{SO}(2)$-equidecomposable'' with $S^1$, indicating that $S^1$ is ``countably $\mathrm{SO}(2)$-paradoxical.''
  \end{example}

  We say that a set $S$ can generate a group $G$. 

  \begin{definition}[Free Group]
    Let $G$ be the group generated by $S$. Then, $G$ is \textbf{free} if it satisfies the following equivalent definitions. 
    \begin{enumerate}
      \item No nonempty reduced word in $S$ represents the identity element in $G$. 
      \item Every element in $G$ can be represented by exactly one reduced word of $S$. 
    \end{enumerate}
    The number of elements of $S$---called the generators---is the \textbf{rank} of the free group. 
  \end{definition}

  \begin{example}[Free Group of Rank 1]
    Let $S = \{1\}$. Then, it generates the group $(\mathbb{Z}, +)$. Similarly, we can think of an irrational rotation in $S^1$, which will also give us a free generator. 
  \end{example}

  Let's extend this by one more dimension. 

  \begin{lemma}
    If a free group $G$ is of rank $2$, then it is $G$-paradoxical, where we view $G$ as acting on itself. 
  \end{lemma}
  \begin{proof}
    Let $G$ be freely generated by $S = \{\rho, \tau\}$, and for each $g \in \{\rho, \tau, \rho^{-1}, \tau^{-1}\}$, define $G_g$ as the set of all elements from $G$ represented by reduced words in $S$ having the leftmost letter as $g$. Since $G$ is a free group, we can partition $G$ as 
    \begin{equation}
      G = \{e\} \sqcup G_\rho \sqcup G_\tau \sqcup G_{\rho^{-1}} \sqcup G_{\tau^{-1}}
    \end{equation}
    Note that by separating out the first letter, we can decompose $G_\rho = \{\rho\} \sqcup \rho G_{\rho} \sqcup \rho G_\tau \sqcup \rho G_{\tau^{-1}}$. By transforming all elements by $\rho^{-1}$, we can ``remove'' the element as $\rho^{-1} G_{\rho} = \{e\} \sqcup G_\rho \sqcup G_\tau \sqcup G_{\tau^{-1}}$. Therefore, $G = \rho^{-1} G_\rho \sqcup G_{\rho^{-1}}$, and so 
    \begin{equation}
      G \sim_G G_\rho \sqcup G_{\rho^{-1}} 
    \end{equation}
    Similarly, we have $G \sim_G G_\tau \sqcup G_{\tau^{-1}}$. Since $G$ is $G$-equidecomposable with its two disjoint subsets $G_{\rho} \sqcup G_{\rho^{-1}}$ and $G_{\tau} \sqcup G_{\tau^{-1}}$, \hyperref[thm:disjoint-subset-paradoxical]{it follows that} $G$ is $G$-paradoxical. 
  \end{proof}

  \begin{theorem}
    $\mathrm{SO}(3)$ has a subgroup that is free on two generators. 
  \end{theorem}
  \begin{proof}
    The general idea is to take motivation from the irrational angles as free generators. 
  \end{proof}

  \begin{theorem}
    Let $G$ be a $G$-paradoxical group acting on set $X$. If only the identity in $G$ has any fixed points in $X$, then $X$ is also $G$-paradoxical. 
  \end{theorem}

  \begin{theorem}[Simplified Hausdorff Paradox]
    Let $S^2$ be the unit sphere in $\mathbb{R}^3$. There is a countable subset $D \subset S^2$  such that $S^2 \setminus D$ is $\mathrm{SO}(3)$-paradoxical.
  \end{theorem}
  \begin{proof}
    
  \end{proof}

  \begin{theorem}
    $S^2$ is $SO(3)$-paradoxical. 
  \end{theorem} 

  \begin{theorem}[Banach-Tarski Paradox]
    Let $B^3$ be the closed unit ball in $\mathbb{R}^3$. Then, $B^3 \setminus \{0\}$ is $\mathrm{SO}(3)$-paradoxical. 
  \end{theorem}

\subsection{Nonmeasurable Sets}

\subsection{Countable Choice} 

\subsection{Dependent Choice}

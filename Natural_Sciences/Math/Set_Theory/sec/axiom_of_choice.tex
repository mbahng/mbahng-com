\section{Axiom of Choice}

  The axioms up to this point are pretty much undisputed and completes ZF set theory. Now that we've defined a function, let's quickly extend the definition of a Cartesian product into an arbitrary union of sets. 
  
  \begin{definition}[Cartesian Product]
    If $\{X_\alpha\}_{\alpha \in A}$ is an indexed family of sets, then their \textbf{Cartesian product} is defined as a set of functions. That is, 
    \begin{equation}
      \prod_{\alpha \in A} X_\alpha \coloneqq \bigg\{ f: A \rightarrow \bigcup_{\alpha \in A} X_\alpha \;\Big|\; \forall \alpha \in A, f(\alpha) \in X_\alpha \bigg\} 
    \end{equation}
    Each function $f$ is called a \textbf{choice function}, which assigns to each $X_\alpha$ some element $f(\alpha) \in X_\alpha$. 
  \end{definition} 

  Therefore, we have used the power set axiom to define a finite Cartesian product, to then define a function, to then define a general Cartesian product. But this detail is irrelevant later on. Note also that this definition of Cartesian product is not the same as that of the previous definition. The binary Cartesian product is defined as $(a, b) = \{\{a\}, \{a, b\}\}$ while this defines as a function $f: \{1, 2\} \rightarrow A, B$. But once we have overwritten the old definition (which is still necessary!) we can just forget about it and use this new definition of Cartesian product since there is a canonical bijection between them. It is a lot less annoying to think of ordered tuples as just tuples rather than as sets of sets. 

  However, in our definition, we just call this a ``set of functions'' and have never proved that it actually contains anything. But we can see obviously that if this Cartesian product is nonempty then there exists a choice function, and if there exists a choice function then the Cartesian product is nonempty. It would be ideal if we can prove one of the two conditions, but it turns out we can't, and therefore we introduce the final axiom, called the \textit{axiom of choice}. Though controversial, it is required in the proofs of some notable theorems. If we include this axiom of choice, then we have ZFC set theory. The axiom of choice has many equivalent definitions. 

  Colloquially, the axiom of choice says that a Cartesian product of a collection\footnote{Note that this does not have to be finite} of non-empty sets is non-empty. That is, it is possible to construct a new set by choosing one element from each set, even if the collection is infinite. 

  \begin{axiom}[Axiom of Choice]
    Let us have an indexed family $X = \{S_i\}_{i \in I}$ of nonempty sets. Then the axiom states the following, which are all equivalent. 
    \begin{enumerate}
      \item There exists an indexed set $\{x_i\}_{i \in I}$ such that $x_i \in S_i$ for every $i \in I$. 
      \item $\prod_{i \in I} S_i$ is nonempty. 
      \item There exists a choice function $f: I \rightarrow \cup_{i \in I} S_i$. 
    \end{enumerate}
  \end{axiom}

  The existence of a choice function when $X$ is finite is easily proved from the ZF axioms, and AC only matters for certain infinite sets. One may argue that if each $S_i$ is nonempty, then choose $s_i \in S_i$ and you're done! While this is an intuition for why the axiom of choice may be true, we can't make the \textit{choice} of all the infinitely many $s_i$ in any ``canonical'' fashion. That is,k while this works for any single $i$ at a time, this doesn't define a function $i \mapsto s_i$. Note that for any sets where you \textit{can} make this choice (e.g. there is a total ordering on $X$, so choose the minimum element), AC holds as a theorem and not as an axiom. 

  \begin{example}
    Let $I$ be the set of all nonempty subsets of $\mathbb{R}$, and $X_i = i \in I$. Then an element $f$ in $\prod_{i \in I} X_i$ is a function which picks an element $f(T) \in T$ for every nonempty $T \subset \mathbb{R}$. How do you \textit{define} such an $f$? If we have $\mathbb{N}$ instead of $\mathbb{R}$, we could take $f(T) = \min(T)$, but this doesn't work for $\mathbb{R}$. Therefore, there is no canonical choice of an element in a nonempty set of real numbers. But AC tells us that we don't have to worry about this. It gives us such a function, even if we cannot ``write it down'' (which means, construct it from the other ZF axioms).  

    If we let $I$ be the set of all nonempty \textit{open} subsets of $\mathbb{R}$, then there is a choice function. Choose any bijection $\tau: \mathbb{N} \rightarrow \mathbb{Q}$, and then assign to each nonempty open subset $U \subset \mathbb{R}$ the element $\tau (\min\{n \in \mathbb{N} \mid \tau(n) \in U\})$. This works since $U \cap \mathbb{Q} \neq \emptyset$. 
  \end{example}

  It is characterized as nonconstructive because it asserts the existence of a choice function but says nothing about how to construct one, unlike the axiom of infinity. This choice function was used in the proof of the following, which turns out to be equivalent.  

  \begin{axiom}[Axiom of Well-Ordering]
    For any set $X$, there exists a binary relation $R$ which \textit{well-orders} $X$, i.e. is a total order and has the property that every nonempty subset of $X$ has a least element under the order $R$. 
    \begin{equation}
      \forall X \exists R (R \text{ well-orders } X)
    \end{equation}
  \end{axiom}

  We can see generally that we would like to use a choice function to select a representative element of each set in $X$. Then we can use these to construct an order. Finally, we state the last form of the axiom of choice. 

  \begin{axiom}[Zorn's Lemma]
    Let $X$ be a partially ordered set that satisfies the two properties. 
    \begin{enumerate}
      \item $P$ is nonempty. 
      \item Every \textit{chain} (a subset $A \subset P$ where $A$ is totally ordered) has an upper bound in $P$. 
    \end{enumerate}
    Then $P$ has at least one maximal element. 
  \end{axiom}

  Zorn's lemma is required to show that every vector space has a basis. 


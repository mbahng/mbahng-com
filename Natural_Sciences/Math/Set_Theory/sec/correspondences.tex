\section{Correspondences} 

\subsection{Axiom of Power Set}

  Now that we have constructed the Von Neumann ordinals, we are allowed to do \textit{indexing}. 

  \begin{axiom}[Axiom of Power Set]
    \label{power-set-axiom}
    The axiom of power set states that for any set $A$, there is a set $B$ that contains every subset of $A$. 
    \begin{equation}
      \forall A \exists B \forall S (S \subset A \implies S \in B)
    \end{equation}
    The axiom of schema of restricted comprehension is then used to define the power set as the unique subset of such $B$ containing the subset of $A$ exactly. 
    \begin{equation}
      \mathcal{P}(A) = 2^A = \{X \in B \mid X \subset A \}
    \end{equation}
  \end{axiom} 

  The existence of the power set allows us to define subfamilies of subsets of a given set, namely things such as the topology or $\sigma$-algebra. But a perhaps more important consequence is the ability to construct Cartesian products of sets. So far, a set of say $\{a, b\}$ is an \textit{unordered} pair since by the axiom of extensionality, $\{a, b\} = \{b, a\}$. We would like to consider some way to order the elements into $(a, b)$, and this ordered pair $(a, b)$ must be a set. There are many ways to do this, but the most established is to take $(a, b)$ as an element of the power set of the power set of the union of sets. 

  \begin{definition}[Cartesian Product]
    The power set axiom allows for the definition of the \textbf{Cartesian product} of two sets $A$ and $B$. Note that if $a \in A, b \in B$, then by the axiom of union $a, b \in A \cup B$ and by the axiom of power set $\{a\}, \{a, b\} \in \mathcal{P}(A \cup B)$. Therefore, using the axiom of power set again we can define
    \begin{equation}
      (a, b) \coloneqq \{\{a\}, \{a, b\}\} \in \mathcal{P}(\mathcal{P}(A \cup B))
    \end{equation} 
    and the Cartesian product is defined 
    \begin{equation}
      A \times B \coloneqq \{ (a, b) \in \mathcal{P}(\mathcal{P}(A \cup B)) \mid a \in A \land b \in B \}
    \end{equation}
    which is a valid set by the axiom schema of specification. 
  \end{definition} 

  \begin{theorem}
    $(a, b) = (a^\prime, b^\prime)$ if and only if $a = a^\prime$ and $b = b^\prime$. 
  \end{theorem}
  \begin{proof}
    The backwards implication is trivial. For the forward, let us assume that $\{\{a\}, \{a, b\}\} = \{\{a^\prime\}, \{a^\prime, b^\prime\}\}$. If $a \neq b$, then $\{a\} = \{a^\prime\}$ and $\{a, b\} = \{a^\prime, b^\prime\}$. So first $a = a^\prime$ and then $\{a, b\} = \{a, b^\prime\}$ implies $b = b^\prime$. If $a = b$, then $\{\{a\}, \{a, a\}\} = \{\{a\}\}$. So $\{a\} = \{a^\prime\}$ and $\{a\} = \{a^\prime, b^\prime\}$, and we get $a = a^\prime = b^\prime$, and so $a = a^\prime$ and $b = b^\prime$. 
  \end{proof}

  \begin{theorem}
    We have 
    \begin{equation}
      A \times (B \cup C) = (A \times B) \cup (A \times C)
    \end{equation}
  \end{theorem}

  From this we can define the Cartesian product of any finite collection of sets recursively. It is indeed the case that $(X \times Y) \times Z$ is a different set from $X \times (Y \times Z)$, but as we will see in later functions, we can define a canonical bijection between them, treating them as equivalent. Furthermore, notice that we have not defined the Cartesian product of infinite sets yet. We can define them using functions actually. 

  The definition of Cartesian products allows us to formally define \textbf{correspondences}. The most notable correspondences are \textit{functions}, \textit{order relations}, and \textit{equivalence relations}. 

  \begin{definition}[Correspondence, Relation]
    A \textbf{correspondence}, or a \textbf{binary relation}, $R$ on a set $A$ is a subset of $A \times A$. We write $aRb$ if and only if $(a,b) \in R$.\footnote{It is a way of describing precisely which two elements are related to one another.} But not all relations may be meaningful or interesting. Therefore we usually like to have certain properties on these relations, including but not limited to 
    \begin{enumerate}
      \item \textit{Reflexive}. For all $a \in A$, $aRa$
      \item \textit{Symmetric}. For all $a,b \in A$, if $aRb$ then $bRa$
      \item \textit{Antisymmetric}. For all $a,b \in A$, if $aRb$ and $bRa$ then $a=b$
      \item \textit{Transitive}. For all $a,b,c \in A$, if $aRb$ and $bRc$ then $aRc$
      \item \textit{Total}. For all $a,b \in A$, either $aRb$ or $bRa$
    \end{enumerate}
  \end{definition} 

\subsection{Functions} 

  We explore our first---and most universally used---relation. 

  \begin{definition}[Function]
    Given two sets $X, Y$, a function $f: X \rightarrow Y$ is a subset $f \subset X \times Y$ satisfying the following
    \begin{enumerate}
      \item For all $x \in X$, there exists $y \in Y$ s.t. $(x, y) \in f$.\footnote{This says that $f$ must be defined for all inputs in $X$.}
      \item For all $x \in X$ and $y, y^\prime \in Y$, if $(x, y) \in f$ and $(x, y^\prime) \in f$, then $y = y^\prime$.\footnote{In other words, $f$ must map one element to exactly one element.} 
    \end{enumerate}
    The set $X$ is said to be the \textbf{domain} and $Y$ the \textbf{codomain}. 

    \begin{figure}[H]
      \centering 
      \begin{tikzcd}
        X \arrow[r, "f"'] & Y 
      \end{tikzcd}
      \caption{A diagram representing the function $f: X \rightarrow Y$.} 
      \label{fig:function at}
    \end{figure}
  \end{definition} 

  \begin{lemma} 
    Given functions $f, g$, $f = g$ if and only if the domains of $f$ and $g$ are equal and $f(x) = g(x)$ for all $x \in \mathrm{domain}(f)$. 
  \end{lemma} 

\subsubsection{Sets Mapped Through Functions}

  \begin{definition}[Image, Preimage]
    Given some $f: X \rightarrow Y$ and $A \subset X$, the \textbf{image} of $A$ under $f$ is defined 
    \begin{equation}
      f(A) \coloneqq \{y \in Y \mid \exists x \in X (f(x) = y)\}
    \end{equation}
    Given $B \subset Y$, the \textbf{preimage} of $B$ under $f$ is defined 
    \begin{equation}
      f^{-1} (B) \coloneqq \{ x \in X \mid f(x) \in B \}
    \end{equation}
  \end{definition} 

  Now let's see how these operations behavior under functions. 

  \begin{theorem}[Preservation Under Mapping Back and Forth]
    \label{preserve_back_forth}
    Given $f: A \rightarrow B$, with $A_0, A_1 \subset A$ and $B_0, B_1 \subset B$, the following hold 
    \begin{enumerate}
      \item $A_0 \subset f^{-1} (f(A_0))$, with equality holding if $f$ is injective. 
      \item $f(f^{-1}(B_0)) \subset B_0$, with equality holding if $f$ is surjective. 
    \end{enumerate}
  \end{theorem} 
  \begin{proof} 
    Listed. 
    \begin{enumerate}
      \item Assume that $x \in A_0$. Then $f(x) \in f(A_0)$. The preimage is 
      \begin{equation}
        f^{-1} (f(A_0)) \coloneqq \{ y \in A \mid f(y) \in f(A_0) \}
      \end{equation}
      and $x$ certainly satisfies the condition that $f(x) \in f(A_0)$. Therefore $x \in f^{-1} (f(A_0))$ and so $A_0 \subset f^{-1} (f(A_0))$. 

      Now assume that $f$ is injective. It suffices to prove that $f^{-1} (f(A_0)) \subset A_0$ since the other direction is proven for all functions. We prove this by proving the contrapositive, i.e. $x \not\in A_0 \implies x \not\in f^{-1} (f(A_0))$. Suppose $x \not\in A_0 \implies f(x) \not\in f(A_0) \implies f^{-1} (f(x)) \not\subset f^{-1} (f(A_0))$ by definition of the image and preimage. But note that since $f$ is injective, $f^{-1} (f(x)) = x$.\footnote{More specifically, if we treat $x$ as the singleton set, $f(x)$ is also a singleton set by definition of a function. Since $f$ is injective, the preimage of a singleton set must be a singleton set. If it were not, then there exists $x, y$ with $x \neq y$ that maps to the same $z$, which contradicts the definition of injectivity.} and thus $x \not\in f^{-1} (f(A_0))$. 

      \item We prove this using the contrapositive. Assume that $x \not\in B_0$. Then, with abuse of notation, we have by definition of the preimage and the image $f^{-1} (x) \not\subset f^{-1} (B_0) \implies f(f^{-1} (x)) \not\subset f(f^{-1}(B_0))$. But $f (f^{-1} (x)) = \{x\}$, since we are just mapping the preimage of $x$ back across to $f$. Therefore, $x \notin f( f^{-1} (B_0))$. 

      Now assume that $f$ is surjective. It suffices to prove that $B_0 \subset f (f^{-1}(B_0))$. Assume $y \in B_0$. Since $f$ is surjective, we know that $f^{-1} (y)$ is nonempty in $A$. We can state $f^{-1}(y) \subset f^{-1} (B_0)$\footnote{The formal proof of this is given in Munkres 1.2.2.a.} which then implies $f(f^{-1} (y)) \subset f (f^{-1} (B_0))$.\footnote{Again formal proof of this given in Munkres 1.2.2.e.} But $f (f^{-1} (y)) = y$ as mentioned previously, and so $y \in f(f^{-1} (B_0))$. 
    \end{enumerate}
  \end{proof}

  \begin{example}[Counterexamples]
    To see why equality does not hold in general for the two cases, look at the counterexamples below. 
    \begin{enumerate}
      \item $A_0 \not\supset f^{-1} (f(A_0))$. 
      \item $f(f^{-1}(B_0)) \not\supset B_0$. Consider $X = Y = \{0, 1\}$ and $f: X \rightarrow Y$ defined $f(0) = f(1) = 0$. Consider $C = Y$. We have $f^{-1} (C) = f^{-1} (0) \cup f^{-1} (1) = X \cup \emptyset = X$. Then $f(f^{-1} (C)) = f(X) = \{0\} \neq C$. 
    \end{enumerate}
  \end{example}

  \begin{theorem}[Preservation Under Preimages]
    \label{preserve_preimage}
    Given $f: A \rightarrow B$, with $A_\alpha \subset A$ and $B_\alpha \subset B$, $f$ preserves the inclusion, union, intersection, and set difference under the preimage. 
    \begin{enumerate}
      \item \textit{Inclusion}. $B_0 \subset B_1 \implies f^{-1} (B_0) \subset f^{-1} (B_1)$. 
      \item \textit{Union}. $f^{-1} (\cup B_\alpha) = \cup_\alpha f^{-1} (B_\alpha)$. 
      \item \textit{Intersection}. $f^{-1} (\cap B_\alpha) = \cap_\alpha f^{-1} (B_\alpha)$.
      \item \textit{Set Difference}. $f^{-1}(B_0 \setminus B_1) = f^{-1} (B_0) \setminus f^{-1} (B_1)$. 
    \end{enumerate}
  \end{theorem} 
  \begin{proof}
    Listed. 
    \begin{enumerate}
      \item \textit{Inclusion}. If $x \in B_0$, then $f^{-1} (x) \subset A$ maps to $x$ by definition. But since $x \in B_0$, $f^{-1} (x)$ maps to a point in $B_0$, and so $f^{-1} (x) \subset f^{-1} (B_0)$. Since $B_0 \subset B_1$ by assumption, $x \in B_1$, and by the previous logic but with $B_0$ replaced by $B_1$ we have $f^{-1}(x) \subset f^{-1} (B_1)$. We have just proved that $f^{-1} (x) \in f^{-1} (B_0)  \implies f^{-1} (x) \in f^{-1} (B_1)$, and so $f^{-1} (B_0) \subset f^{-1} (B_1)$. 

      \item \textit{Union}. We prove bidirectionally. 
      \begin{enumerate}
        \item $f^{-1} (B_0 \cup B_1) \subset f^{-1} (B_0) \cup f^{-1} (B_1)$. Let $x \in f^{-1} (B_0 \cup B_1)$ which by definition of the preimage means $f(x) \in B_0 \cup B_1$. Therefore $f(x) \in B_0$ or $B_1$. Without loss of generality, let $f(x) \in B_0$. Then we have 
          \begin{equation}
            x \in f^{-1} (f(x)) \subset f^{-1} (B_0)
          \end{equation} 
          where the first inclusion comes from [Munkres 1.2.1.a] when treating $A_0 = \{x\}$, and the second subset comes from [Munkres 1.2.2.a] when treating $B_0 = \{f(x)\}, B_1 = B_1$. Therefore $x \in f^{-1} (B_0) \subset f^{-1} (B_0) \cup f^{-1} (B_1)$. 
        \item $f^{-1} (B_0) \cup f^{-1} (B_1) \subset f^{-1} (B_0 \cup B_1)$. Let $x \in f^{-1}(B_0) \cup f^{-1} (B_1)$. Without loss of generality, let $x \in f^{-1}(B_0)$ which by definition of the preimage implies $f(x) \in B_0 \subset (B_0 \cup B_1) \implies f(x) \in (B_0 \cup B_1)$. Therefore, we have 
          \begin{equation}
            x \in f^{-1} (f(x)) \subset f^{-1} (B_0 \cup B_1)
          \end{equation} 
          where the inclusion claim comes from [Munkres 1.2.1.a] when treating $A_0 = \{x\}$, and the subset claim comes from [Munkres 1.2.2.a] when treating $B_0 = \{f(x)\}, B_1 = B_0 \cup B_1$. Therefore $x \in f^{-1} (B_0 \cup B_1)$. 
      \end{enumerate}
      Therefore, $f^{-1} (B_0) \cup f^{-1} (B_1) = f^{-1} (B_0 \cup B_1)$. 

      \item \textit{Intersection}. We prove bidirectionally. 
      \begin{enumerate}
        \item $f^{-1} (B_0 \cap B_1) \subset f^{-1} (B_0) \cap f^{-1} (B_1)$. Assume $x \in f^{-1} (B_0 \cap B_1)$, which by definition of the preimage means $f(x) \in B_0 \cap B_1$. So 
          \begin{align}
            f(x) \in B_0 & \implies x \in f^{-1} (f(x)) \subset f^{-1} (B_0) \\
            f(x) \in B_1 & \implies x \in f^{-1} (f(x)) \subset f^{-1} (B_1)
          \end{align}
          where the inclusion claim comes from [Munkres 1.2.1.a] when treating $A_0 = \{x\}$, and the subset claim comes from [Munkres 1.2.2.a] when treating $f(x)$ as a singleton set. Therefore $x$ is in both of the preimages and so $x \in f^{-1} (B_0) \cap f^{-1} (B_1)$. 
        \item $f^{-1} (B_0) \cap f^{-1} (B_1) \subset f^{-1} (B_0 \cap B_1)$. Let $x \in f^{-1} (B_0) \cap f^{-1} (B_1)$. Then by definition of intersection and preimage, 
          \begin{align}
            x \in f^{-1} (B_0) & \implies f(x) \in B_0 \\
            x \in f^{-1} (B_1) & \implies f(x) \in B_1 
          \end{align} 
          and so $f(x) \in B_0 \cap B_1$ by definition of intersection. This means by definition of the preimage that $x \in f^{-1}(B_0 \cap B_1)$. 
      \end{enumerate}

      \item \textit{Set Difference}. We prove bidirectionally. 
      \begin{enumerate}
        \item $f^{-1} (B_0 \setminus B_1) \subset f^{-1} (B_0) \setminus f^{-1} (B_1)$. Let $x \in f^{-1}(B_0 \setminus B_1)$ which by definition of preimage means $f(x) \in B|0 \setminus B_1$. This implies two things. First, 
          \begin{equation}
            f(x) \in B_0 \implies x \in f^{-1} (f(x)) \subset f^{-1} (B_0)
          \end{equation}
          where the inclusion comes from [Munkres 1.2.1.a] when treating $A_0 = \{x\}$ as the single set, and the subset claim comes from [Munkres 1.2.2.a] stating that inclusions are preserved under the preimage operator. Secondly, we claim that  
          \begin{align}
            f(x) \not\in B_1 & \implies x \not\in f^{-1} (B_1)
          \end{align}
          since if $x \in f^{-1} (B_1)$, then $f(x) \in B_1$ by definition of the preimage. 

        \item $f^{-1} (B_0) \setminus f^{-1} (B_1) \subset f^{-1} (B_0 \setminus B_1)$. Let $x \in f^{-1} (B_0) \setminus f^{-1} (B_1)$. Then the following holds 
          \begin{align}
            x \in f^{-1} (B_0) & \implies f(x) \in B_0 \\
            x \not\in f^{-1} (B_1) & \implies f(x) \not\in B_1
          \end{align} 
          from the definition of the preimage and the contrapositive of its implication. Therefore $f(x) \in B_0 \setminus B_1$ which by definition of the preimage $x \in f^{-1} (B_0 \setminus B_1)$. 
      \end{enumerate}
    \end{enumerate}
  \end{proof}

  \begin{theorem}[Preservation Under Images]
    \label{preserve_image}
    Given $f: A \rightarrow B$, with $A_\alpha \subset A$ and $B_\alpha \subset B$, $f$ preserves the inclusion and union under the image, but inclusion properties for the intersection and set difference hold. 
    \begin{enumerate}
      \item \textit{Inclusion}. $A_0 \subset A_1 \implies f(A_0) \subset f(A_1)$. 
      \item \textit{Union}. $f(\cup_\alpha A_\alpha) = \cup_\alpha f(A_\alpha)$. 
      \item \textit{Intersection}. $f(\cap_\alpha A_\alpha) \subset \cap_\alpha f(A_\alpha)$, and equality holds if $f$ is injective. 
      \item \textit{Set Difference}. $f(A_0 \setminus A_1) \supset f(A_0) \setminus f(A_1)$, and equality holds if $f$ is injective. 
    \end{enumerate}
  \end{theorem} 
  \begin{proof}
    Listed. 
    \begin{enumerate}
      \item \textit{Inclusion}. Let $x \in A_0$. Then by definition of the image $f(x) \in f(A_0)$. Since $A_0 \subset A_1$, then $x \in A_1$ and it immediately follows that $f(x) \in f(A_1)$. Therefore $f(A_0) \subset f(A_1)$. 

      \item \textit{Union}. We prove bidirectionally. 
      \begin{enumerate}
        \item $f(A_0 \cup A_1) \subset f(A_0) \cup f(A_1)$. Let $y \in f(A_0 \cup A_1)$. Then by definition there exists some $x \in A_0 \cup A_1$ s.t. $f(x) = y$. WLOG let $x \in A_0$. Then by definition $y = f(x) \in f(A_0) \subset f(A_0) \cup f(A_1)$. 

        \item $f(A_0) \cup f(A_1) \subset f(A_0 \cup A_1)$. Let $y \in f(A_0) \cup f(A_1)$. WLOG $y \in f(A_0)$, and there exists some $x \in A_0$ s.t. $f(x) = y$. Since $x \in A_0$, $x \in A_0 \cup A_1$, and by definition $y = f(x) = f(A_0) \cup f(A_1)$. 
      \end{enumerate}

      \item \textit{Intersection}. Assume that $y \in f(A_0 \cap A_1)$. Then by definition there exists some $x \in A_0 \cap A_1$ s.t. $f(x) = y$. So we have 
      \begin{align}
        x \in A_0 & \implies f(x) \in f(A_0) \\
        x \in A_1 & \implies f(x) \in f(A_1)
      \end{align} 
      and therefore $y = f(x) \in f(A_0) \cap f(A_1)$. 

      To prove equality, it suffices to show that $f(A_0) \cap f(A_1) \subset f(A_0 \cap A_1)$ if $f$ is injective. Assume that $y \in f(A_0) \cap f(A_1)$. Then $y \in f(A_0)$, and so there exists an $x \in A_0$ s.t. $y = f(x) \in f(A_0)$. By the same logic there exists an $x^\prime \in A_1$ s.t. $y = f(x^\prime) \in f(A_1)$. But since $f$ is injective, this implies that $x = x^\prime$. So $x \in A_0 \cap A_1$, and so $y = f(x) \in f(A_0 \cap A_1)$. 

      \item \textit{Set Difference}. Assume that $y \in f(A_0) \setminus f(A_1)$. Since $y \in f(A_0)$, there exists some $x \in A_0$ s.t. $y = f(x)$. Since $y \not\in f(A_1)$, there exists no $x^\prime \in A_1$ s.t. $y = f(x^\prime)$.  Therefore, $x \in A_0 \setminus A_1 \implies y = f(x) \in f(A_0 \setminus A_1)$. 

      To prove equality, it suffices to show that $f(A_0 \setminus A_1) \subset f(A_0) \setminus f(A_1)$ if $f$ is injective. Assume that $y \in f(A_0 \setminus A_1)$. Then there exists some $x \in A_0 \setminus A_1$ s.t. $f(x) = y$. We claim that $x$ is unique since if there were two $x, x^\prime$, then $f(x) = f(x^\prime)$ with $x \neq x^\prime$, which means $f$ is not injective. We see that $x \in A_0 \implies y = f(x) \in f(A_0)$, and $x \not\in A_1 \implies y = f(x) \not\in f(A_1)$. Therefore, $x \in f(A_0) \setminus f(A_1)$. 
    \end{enumerate}
  \end{proof} 

  \begin{example}[Intersection Not Necessarily Preserved]
    Note that intersection is not necessarily preserved. To see why, look at the counterexample. Consider $A = \{0, 1\}, B = \{1, 2\}$, and define 
    \begin{equation}
      f(0) = f(2) = 0, f(1) = 1
    \end{equation} 
    Then $f(A) = f(B) = \{0, 1\} \implies f(A) \cap f(B) = \{0, 1\}$. On the other hand, we have $A \cap B = \{1\} \implies f(A \cap B) = \{1\}$. 
  \end{example}

\subsubsection{Composite Functions}

  \begin{definition}[Composition]
    Given functions $f: X \rightarrow Y$, $g: Y \rightarrow Z$, we define the \textbf{composite}, denoted $g \circ f$ or $g(f(\cdot))$, of $f$ and $g$ as the subset
    \begin{equation}
      g \circ f \coloneqq \{ (x, z) \in X \times Z \mid \exists y \in Y (f(x) = y \land f(y) = z) \}
    \end{equation} 
  \end{definition}

  \begin{theorem}[Compositions]
    A composite is a function. 
    \begin{figure}[H]
      \centering 
      \begin{tikzcd}
        X \arrow[r, "f"] \arrow[rd, "g \circ f"'] & Y \arrow[d, "g"] \\
        & Z
      \end{tikzcd}
      \caption{Commutative diagram representing a composition of functions.} 
      \label{fig:composition}
    \end{figure}
  \end{theorem}
  \begin{proof}
    Using the definition above, we prove the two properties. 
    \begin{enumerate}
      \item For all $x \in X$, there exists $y \in Y$ s.t. $(x, y) \in f$. Similarly, for all $y \in Y$, there exists $z \in Z$ s.t. $(y, z) \in g$. Therefore, for all $x \in X$, there exists a $y \in Y$, which follows that there exists also a $z \in Z$. Therefore $g \circ f$ is defined for all inputs in $X$. 
      \item For all $x \in X$ and $z, z^\prime \in Z$, say that $(x, z), (x, z^\prime) \in g \circ f$. Then by definition of composition there exists a $y, y^\prime \in Y$ s.t. $f(x) = y, f(y) = z$ and $f(x) = y^\prime, f(y^\prime) = z^\prime$. Since $f$ is a function, $y = y^\prime$. Since $g$ is a function, $y = y^\prime \implies z = z^\prime$. Therefore $g \circ f$ is a function. 
    \end{enumerate}
  \end{proof}

  For the computer science students, note that a function behaves precisely like functional dependencies in a relational database. A composition represents a natural join. 

  \begin{theorem}[Associativity]
    Composition is associative. That is, consider $f: Y \rightarrow Z, g: X \rightarrow Y, h: W \rightarrow X$ functions. Then 
    \begin{equation}
      (f \circ g) \circ h = f \circ (g \circ h)
    \end{equation} 
    Therefore, we write this as 
    \begin{equation}
      f \circ g \circ h
    \end{equation} 
    \begin{figure}[H]
      \centering 
      \begin{tikzcd}
        & X \arrow[r, "g"] \arrow[rrd, "g \circ h"'] & Y \arrow[rd, "h"] & \\ 
        W \arrow[ru, "f"] \arrow[rru, "g \circ f"'] & & & Z 
      \end{tikzcd}
      \caption{} 
      \label{fig:associative}
    \end{figure}
  \end{theorem}
  \begin{proof}
    Consider any $w \in W$, and let us label $x = h(w)$, $y = g(x)$, $z = f(y)$, where $x, y, z$ must be uniquely determined by $w$ since it is a function. Then, 
    \begin{align}
      ((f \circ g) \circ h) (w) & = (f \circ g) (h(w)) = (f \circ g) (x) = z  \\
      (f \circ (g \circ h)) (w) & = f ((g \circ h)(w)) = f (y) = z
    \end{align}
    and they coincide for all $w \in W$. 
  \end{proof} 

  If we are familiar with algebra, this gives the set of functions $\{f: X \rightarrow X\}$ the structure of a \textit{monoid} under composition. We can also talk about commutativity. 

  \begin{definition}[Commutativity]
    Two functions $f, g: X \rightarrow X$ are said to be \textbf{commute} if 
    \begin{equation}
      f \circ g = g \circ f
    \end{equation} 
    \begin{figure}[H]
      \centering 
      \begin{tikzcd}
        X \arrow[r, "f"] \arrow[d, "g"] & X \arrow[d, "g"] \\
        X \arrow[r, "f"] & X 
      \end{tikzcd}
      \caption{Commutative diagram representing commuting functions $f, g$. } 
      \label{fig:commutative}
    \end{figure}
  \end{definition}

  \begin{theorem}[Composition]
    Let $f: X \rightarrow Y$ and $g: Y \rightarrow Z$. 
    \begin{enumerate}
      \item $f$ injective and $g$ injective $\implies$ $g \circ f$ injective. 
      \item $f$ surjective and $g$ surjective $\implies$ $g \circ f$ surjective. 
      \item $f$ bijective and $g$ bijective $\implies$ $g \circ f$ bijective. 
    \end{enumerate}
  \end{theorem}

\subsubsection{Injective, Surjective, Bijective Functions}

  \begin{definition}[Injectivity, Surjectivity, Bijectivity]
    A function $f: X \rightarrow Y$ is said to be 
    \begin{enumerate}
      \item \textbf{injective} if $\forall x \in X, \forall x^\prime \in X \big( f(x) = f(x^\prime) \implies x = x^\prime \big)$. 
      \item \textbf{surjective} if $\forall y \in Y \exists x \in X (y = f(x))$. 
      \item \textbf{bijective} if it is injective and surjective. 
    \end{enumerate}
  \end{definition}

  \begin{definition}[Inverse Function]
    If a function $f: X \rightarrow Y$ is bijective, then there exists an \textbf{inverse function} $f^{-1}: Y \rightarrow X$ satisfying 
    \begin{equation}
      \forall x \in X \big[ f(f^{-1}(x)) = f^{-1} (f(x)) = x \big]
    \end{equation}
  \end{definition} 

  \begin{definition}[Restriction, Extension]
    If $f: X \rightarrow Y$ and $X_0 \subset X$, we define the \textbf{restriction} of $f$ to $X_0$ to be the function mapping to $Y$ whose rule is 
    \begin{equation}
      f|_{X_0} \coloneqq \{ (x, f(x)) \in f x \in X_0 \}
    \end{equation} 
    Letting $g: X_0 \rightarrow Y$, any function $f: X \supset X_0 \rightarrow Y$ satisfying $f(x) = g(x)$ for all $x \in X_0$ is said to be an \textbf{extension} of $g$ to $X$. 
  \end{definition} 

  \begin{theorem}[Injectivity/Surjectivity]
    Let $f: X \rightarrow Y$, $g: Y \rightarrow Z$, and $h = g \circ f$. The following hold: 
    \begin{enumerate}
      \item $h$ injective $\implies$ $f$ injective. 
      \item $h$ surjective $\implies$ $g$ surjective. 
      \item $h$ bijective $\implies$ $f$ injective and $g$ bijective. 
    \end{enumerate}
  \end{theorem} 

  \begin{corollary}[Bijection Equals Existence of Inverse]
    $f: X \rightarrow Y$ has a inverse function $f^{-1}: B \rightarrow A$ iff it is bijective. 
  \end{corollary}

  \begin{corollary}[Decomposition]
    Any function $h: X \rightarrow Y$ can be decomposed to the form $h = g \circ f$, where $f$ is injective and $g$ is surjective. 
  \end{corollary}
  \begin{proof}
    Given $X$, let us define an equivalence class where for any $x, y \in X$, $x \sim y$ iff $f(x) = f(y)$. Call this quotient space $X / \sim$. Then we can define the mappings. 
    \begin{enumerate}
      \item $\iota: X \rightarrow X / \sim$ which maps each element to its equivalence class. $\iota(x) = [x]$
      \item $f^\prime: X / \sim \rightarrow Y$ which maps each class to the element of $Y$ that it maps to. $f^\prime ([x]) = f(x)$. 
    \end{enumerate}
    $\iota$ is surjective since for every $[x] \in X / \sim$, there exists at least one element $x \in X$ that maps to it. $f^\prime$ is injective since have squished all the points $x$ that map to the same $y$ into a single class $[x]$. 

    \begin{figure}[H]
      \centering 
      \begin{tikzcd}
        X \arrow[r, "f"] \arrow[d, "\iota"] & Y \\
        X / \sim \arrow[ru, "f^\prime"'] & 
      \end{tikzcd}
      \caption{Decomposition of $f$ into surjective $\iota$ and injective $f^\prime$. } 
      \label{fig:decomposition}
    \end{figure}
  \end{proof}

  \begin{theorem}[Inverse of Inverses]
    If $f$ is bijective, then $f = (f^{-1})^{-1}$. 
  \end{theorem}

  \begin{theorem}[Inverse of Compositions]
    If $f, g$ are both bijective, then 
    \begin{equation}
      (f \circ g)^{-1} = g^{-1} \circ f^{-1}
    \end{equation}
  \end{theorem}

  \begin{theorem}[Finite Set Mappings]
    Suppose $X$ and $Y$ are finite sets, each with $n$ elements, and $f: X \rightarrow Y$. If $f$ is injective or bijective, then $f$ is bijective. 
  \end{theorem} 

\subsubsection{Cartesian Products as Functions} 

  Remember that we have used the power set to construct the binary Cartesian product of sets. Now we extend this to arbitrary Cartesian products using functions. 

  \begin{lemma} 
    Let $A$ and $B$ be sets. Then the set of all functions $f: A \rightarrow B$ is denoted $B^A$. We claim that this set exists. 
  \end{lemma}
  \begin{proof}
    
  \end{proof}

  \begin{definition}[Cartesian Product]
    Let $I$ be some set (used for indexing) and $S = \{S_i\}_{i \in I}$ be an indexed set of sets. Then we define the \textbf{Cartesian product} of $S_i$'s as 
    \begin{equation}
      \prod S = \prod_{i \in I} S_i \coloneqq \{f \in (\cup S)^I \mid f(i) \in S_i \; \forall i \in I \}
    \end{equation}
    Note that we need the property since we want $f(i)$ to map specifically into $S_i$. 
  \end{definition}

  With this more general construction, we can pretty much forget about the previous construction of binary Cartesian products. 

\subsection{Order Relations} 
  
  \begin{definition}[Partial Order]
    An \textbf{order} is a relation $R$, usually denoted $\leq$, that satisfies the following properties. 
    \begin{enumerate}
      \item \textit{Reflexive}. For all $a \in A$, $a \leq a$
      \item \textit{Antisymmetric}. For all $a,b \in A$, if $a \leq b$ and $b \leq a$ then $a=b$
      \item \textit{Transitive}. For all $a,b,c \in A$, if $a \leq b$ and $b \leq c$ then $a \leq c$
    \end{enumerate} 
    Note that when we say $x \leq y$, this means "$x$ is related to $y$" (but does not necessarily mean that $y$ is related to $x$), or "$x$ is less than or equal to $y$." A set $X$ with a partial order is called a partially ordered set. 
  \end{definition} 

  \begin{example}[Partially Ordered Sets]
    We list some examples of partially ordered sets. 
    \begin{enumerate}
      \item The real numbers ordered by the standard "less-than-or-equal" relation $\leq$ (totally ordered set as well). 
      \item The set of subsets of a given set $X$ ordered by inclusion. That is, the power set $2^X$ with the partial order $\subseteq$ is partially ordered. 
      \item The set of natural numbers equipped with the relation of divisibility. 
      \item The set of subspaces of a vector space ordered by inclusion. 
      \item For a partially ordered set $P$, the sequence space containing all sequences of elements from $P$, where sequence $a$ precedes sequence $b$ if every item in $a$ precedes the corresponding item in $b$. 
    \end{enumerate}
  \end{example} 

  \begin{definition}[Comparable Elements]
    Given elements $a, b$ of partially ordered set $A$, if either $a \leq b$ or $b \leq a$, then $a$ and $b$ are \textbf{comparable}. Otherwise, they are \textbf{incomparable}. 
  \end{definition}

  While partial ordering is nice, we would often want a stricter structure so that the order ``encompasses'' the whole set, i.e. every element is comparable. This property is sometimes known as \textit{totality}. 

  \begin{definition}[Total Order]
    A partial order in which every pair of elements is comparable is called a \textbf{total order}, or \textbf{linear order}. Note that from this $\leq$ relation, we can similarly define 
    \begin{enumerate}
      \item $a < b \iff (a \leq b) \land (a \neq b)$. 
      \item $a \geq b \iff \neg(a < b)$. 
      \item $a > b \iff \neg(a \leq b)$. 
    \end{enumerate} 
  \end{definition}

  Almost always when we talk about ordered sets, we mean totally ordered sets. So we will work with them by default and define following terms according to totally ordered sets. For convenience of notation, we also write $a < x < b \iff (a < x) \land (x < b)$. 

  \begin{definition}[Interval]
    Given a totally ordered set $X$, we denote the \textbf{intervals} as 
    \begin{enumerate}
      \item $(a, b) \coloneqq \{x \in X \mid a < x < b \}$
      \item $[a, b) \coloneqq \{x \in X \mid a \leq x < b \}$
      \item $(a, b] \coloneqq \{x \in X \mid a < x \leq b \}$
      \item $[a, b] \coloneqq \{x \in X \mid a \leq x \leq b \}$
    \end{enumerate}
  \end{definition} 

  \begin{definition}[Extrema]
    Given a totally ordered set $X$, 
    \begin{enumerate}
      \item $x \in X$ is a \textbf{maximum} $X$ if $y \leq x$ for all $y \in X$. 
      \item $x \in X$ is a \textbf{minimum} $X$ if $x \leq y$ for all $y \in X$. 
    \end{enumerate}
  \end{definition}

  \begin{definition}[Bounds]
    Given a totally ordered set $X$ and some subset $S \subset X$. 
    \begin{enumerate}
      \item $x \in X$ is an \textbf{upper bound} of $S$ if $x \geq y$ for all $y \in S$. 
      \item $x \in X$ is a \textbf{lower bound} of $S$ if $x \leq y$ for all $y \in S$. 
      \item $x \in X$ is a \textbf{supremum}, or \textbf{least upper bound}, of $S$ if $x$ is the minimum of the set of all upper bounds of $S$. 
      \item $x \in X$ is a \textbf{infimum}, or \textbf{greatest lower bound}, of $S$ if $x$ is the maximum of the set of all lower bounds of $S$. 
    \end{enumerate}
  \end{definition}

  Note that we have defined max/min separately from the concept of bounds. You can define the maximum of a set with just knowing the set, but the bounds require \textit{both} some subset $S$ with respect to an enclosing set $X$.\footnote{For example, it makes sense to define the maximum of a set $S = [0, 1]$ by itself, but not an upper bound for it. If $X = \mathbb{Q}$, then the supremum is $1$, but if $X$ was the set of all irrationals, then this has no supremum.} Intuitively, the main difference between the supremum/infimum and maximum/minimum is that the supremum/infimum accounts for limit points of the subset $S$. 

  \begin{definition}[Convexity on Ordered Sets]
    Given a totally ordered set $(X, \leq)$, a subset $S$ is said to be \textbf{convex} if for all $a, b \in S$, 
    \begin{equation}
      a \leq c \leq b \implies c \in S
    \end{equation}
  \end{definition}

\subsection{Equivalence Relations}

  \begin{definition}[Equivalence Relation]
    An \textbf{equivalence relation} on a set $A$ is a relation, denoted $\sim$ satisfying 
    \begin{enumerate}
      \item \textit{Reflexive}. For all $a \in A$, $a \sim a$
      \item \textit{Symmetric}. For all $a,b \in A$, if $a \sim b$ then $b \sim a$
      \item \textit{Transitive}. For all $a,b,c \in A$, if $a \sim b$ and $b \sim c$ then $a \sim c$
    \end{enumerate}
    Given an equivalence relation, we can define an \textbf{equivalence class} as 
    \begin{equation}
      [y] \coloneqq \{ x \in A \mid x \sim y \}
    \end{equation}
  \end{definition}

  \begin{definition}[Partition]
    A \textbf{partition} of a set $X$ is a collection of disjoint nonempty subset of $X$ whose union is all of $A$. 
  \end{definition} 

  \begin{theorem}[Quotient Space, Map]
    \label{partition}
    The set of equivalence classes of a set $X$ with an equivalence relation $\sim$ is a partition of $X$, denoted as the \textbf{quotient set} $X/\sim$. Therefore, the map $\iota: X \rightarrow X/\sim$ is well-defined and is called a \textbf{quotient map}. 
  \end{theorem} 
  \begin{proof}
    Assume the contrary. If $X$ has one element, then its equivalence class is $[x]$ and this is trivially proven. If $X$ has at least 2 elements, let us call them $x, y \in X$ with $x \neq y$. $[x], [y]$ are their equivalence classes. Clearly due to reflexivity, $x \in [x]$ and $y \in [y]$ and so they are nonempty. Since we assumed that this is not a partition, there exists some $z \in X$ in both $[x], [y]$. But $z \in [x] \implies z \sim x$ and $z \in [y] \implies y \sim z$. So by transitivity, $x \sim z$, meaning that $[x] = [y]$. Therefore they must be the same element of a partition. 
  \end{proof}

  \begin{example}[Circles]
    $M$ is the set of circles in $\mathbb{R}^{2}$. Given $a, b \in M$, $a \sim b$ iff the radii are equal in length. We can denote each equivalence class by $\{ r \}$, where $r$ is the length of the radius. We can define addition as 
    \begin{equation}
      \{ a \} + \{ b \} \equiv \{ a + b\}
    \end{equation}
  \end{example}


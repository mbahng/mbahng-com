\section{Analytic Functions} 

\subsection{Exponential and Logarithmic Functions} 

\subsection{Trigonometric Functions} 

\subsection{Fourier Series}

\subsection{Gamma Function}

\subsection{Exercises} 

  \begin{exercise}[Rudin 8.1]
    Define
    \begin{equation}
      f(x) = \begin{cases} e^{-1/x^2} & (x \neq 0), \\ 0 & (x = 0). \end{cases}
    \end{equation}
    Prove that $f$ has derivatives of all orders at $x = 0$, and that $f^{(n)}(0) = 0$ for $n = 1, 2, 3, \dots$.
  \end{exercise}
  \begin{solution}

  \end{solution}

  \begin{exercise}[Rudin 8.2]
    Let $a_{ij}$ be the number in the $i$th row and $j$th column of the array
    \begin{align*}
      -1 & \quad 0 & \quad 0 & \quad 0 & \dots \\
      \frac{1}{2} & \quad -1 & \quad 0 & \quad 0 & \dots \\
      \frac{1}{4} & \quad \frac{1}{2} & \quad -1 & \quad 0 & \dots \\
      \frac{1}{8} & \quad \frac{1}{4} & \quad \frac{1}{2} & \quad -1 & \dots \\
      \dots & \dots & \dots & \dots & \dots
    \end{align*}
    so that
    \begin{equation}
      a_{ij} = \begin{cases} 0 & (i < j), \\ -1 & (i = j), \\ 2^{j-i} & (i > j). \end{cases}
    \end{equation}
    Prove that
    \begin{equation}
      \sum_i \sum_j a_{ij} = -2, \quad \sum_j \sum_i a_{ij} = 0.
    \end{equation}
  \end{exercise}
  \begin{solution}

  \end{solution}

  \begin{exercise}[Rudin 8.3]
    Prove that
    \begin{equation}
      \sum_i \sum_j a_{ij} = \sum_j \sum_i a_{ij}
    \end{equation}
    if $a_{ij} \geq 0$ for all $i$ and $j$ (the case $+\infty = +\infty$ may occur).
  \end{exercise}
  \begin{solution}

  \end{solution}

  \begin{exercise}[Rudin 8.4]
    Prove the following limit relations:
    \begin{enumerate}
      \item $\lim_{x \to 0} \frac{b^x - 1}{x} = \log b \quad (b > 0)$.
      \item $\lim_{x \to 0} \frac{\log (1+x)}{x} = 1$.
      \item $\lim_{x \to 0} (1+x)^{1/x} = e$.
      \item $\lim_{n \to \infty} (1 + \frac{x}{n})^n = e^x$.
    \end{enumerate}
  \end{exercise}
  \begin{solution}

  \end{solution}

  \begin{exercise}[Rudin 8.5]
    Find the following limits:
    \begin{enumerate}
      \item $\lim_{x \to 0} \frac{e - (1+x)^{1/x}}{x}$.
      \item $\lim_{n \to \infty} \frac{n}{\log n} [n^{1/n} - 1]$.
      \item $\lim_{x \to 0} \frac{\tan x - x}{x(1 - \cos x)}$.
      \item $\lim_{x \to 0} \frac{x - \sin x}{\tan x - x}$.
    \end{enumerate}
  \end{exercise}
  \begin{solution}

  \end{solution}

  \begin{exercise}[Rudin 8.6]
    Suppose $f(x)f(y) = f(x+y)$ for all real $x$ and $y$.
    \begin{enumerate}
      \item Assuming that $f$ is differentiable and not zero, prove that
        \begin{equation}
          f(x) = e^{cx}
        \end{equation}
        where $c$ is a constant.
      \item Prove the same thing, assuming only that $f$ is continuous.
    \end{enumerate}
  \end{exercise}
  \begin{solution}

  \end{solution}

  \begin{exercise}[Rudin 8.7]
    If $0 < x < \frac{\pi}{2}$, prove that
    \begin{equation}
      \frac{2}{\pi} < \frac{\sin x}{x} < 1.
    \end{equation}
  \end{exercise}
  \begin{solution}

  \end{solution}

  \begin{exercise}[Rudin 8.8]
    For $n = 0, 1, 2, \dots$, and $x$ real, prove that
    \begin{equation}
      |\sin nx| \leq n |\sin x|.
    \end{equation}
    Note that this inequality may be false for other values of $n$. For instance, $|\sin \frac{1}{2}\pi| > \frac{1}{2} |\sin \pi|$.
  \end{exercise}
  \begin{solution}

  \end{solution}

  \begin{exercise}[Rudin 8.9]
    \begin{enumerate}
      \item Put $s_N = 1 + (\frac{1}{2}) + \dots + (1/N)$. Prove that
        \begin{equation}
          \lim_{N \to \infty} (s_N - \log N)
        \end{equation}
        exists. (The limit, often denoted by $\gamma$, is called Euler's constant.)
      \item Roughly how large must $m$ be so that $N = 10^m$ satisfies $s_N > 100$?
    \end{enumerate}
  \end{exercise}
  \begin{solution}

  \end{solution}

  \begin{exercise}[Rudin 8.10]
    Prove that $\sum 1/p$ diverges; the sum extends over all primes.
  \end{exercise}
  \begin{solution}

  \end{solution}

  \begin{exercise}[Rudin 8.11]
    Suppose $f \in \mathscr{R}$ on $[0, A]$ for all $A < \infty$, and $f(x) \to 1$ as $x \to +\infty$. Prove that
    \begin{equation}
      \lim_{t \to 0} t \int_0^\infty e^{-tx} f(x) \, dx = 1 \quad (t > 0).
    \end{equation}
  \end{exercise}
  \begin{solution}

  \end{solution}

  \begin{exercise}[Rudin 8.12]
    Suppose $0 < \delta < \pi, f(x) = 1$ if $|x| \leq \delta, f(x) = 0$ if $\delta < |x| \leq \pi$, and $f(x + 2\pi) = f(x)$ for all $x$.
    \begin{enumerate}
      \item Compute the Fourier coefficients of $f$.
      \item Conclude that
        \begin{equation}
          \sum_{n=1}^\infty \frac{\sin (n\delta)}{n} = \frac{\pi - \delta}{2} \quad (0 < \delta < \pi).
        \end{equation}
      \item Deduce from Parseval's theorem that
        \begin{equation}
          \sum_{n=1}^\infty \frac{\sin^2 (n\delta)}{n^2 \delta} = \frac{\pi - \delta}{2}.
        \end{equation}
      \item Let $\delta \to 0$ and prove that
        \begin{equation}
          \int_0^\infty \left( \frac{\sin x}{x} \right)^2 \, dx = \frac{\pi}{2}.
        \end{equation}
      \item Put $\delta = \pi/2$ in (c). What do you get?
    \end{enumerate}
  \end{exercise}
  \begin{solution}

  \end{solution}

  \begin{exercise}[Rudin 8.13]
    Put $f(x) = x$ if $0 \leq x < 2\pi$, and apply Parseval's theorem to conclude that
    \begin{equation}
      \sum_{n=1}^\infty \frac{1}{n^2} = \frac{\pi^2}{6}.
    \end{equation}
  \end{exercise}
  \begin{solution}

  \end{solution}

  \begin{exercise}[Rudin 8.14]
    If $f(x) = (\pi - |x|)^2$ on $[-\pi, \pi]$, prove that
    \begin{equation}
      f(x) = \frac{\pi^2}{3} + \sum_{n=1}^\infty \frac{4}{n^2} \cos nx
    \end{equation}
    and deduce that
    \begin{equation}
      \sum_{n=1}^\infty \frac{1}{n^2} = \frac{\pi^2}{6}, \quad \sum_{n=1}^\infty \frac{1}{n^4} = \frac{\pi^4}{90}.
    \end{equation}
  \end{exercise}
  \begin{solution}

  \end{solution}

  \begin{exercise}[Rudin 8.15]
    With $D_n$ as defined in (77), put
    \begin{equation}
      K_N(x) = \frac{1}{N+1} \sum_{n=0}^N D_n(x).
    \end{equation}
    Prove that
    \begin{equation}
      K_N(x) = \frac{1}{N+1} \cdot \frac{1 - \cos (N+1)x}{1 - \cos x}
    \end{equation}
    and that
    \begin{enumerate}
      \item $K_N \geq 0$,
      \item $\frac{1}{2\pi} \int_{-\pi}^\pi K_N(x) \, dx = 1$,
      \item $K_N(x) \leq \frac{1}{N+1} \cdot \frac{2}{1 - \cos \delta}$ if $0 < \delta \leq |x| \leq \pi$.
    \end{enumerate}
    If $s_N = s_N(f; x)$ is the $N$th partial sum of the Fourier series of $f$, consider the arithmetic means
    \begin{equation}
      \sigma_N = \frac{s_0 + s_1 + \cdots + s_N}{N+1}.
    \end{equation}
    Prove that
    \begin{equation}
      \sigma_N(f; x) = \frac{1}{2\pi} \int_{-\pi}^\pi f(x-t) K_N(t) \, dt,
    \end{equation}
    and hence prove Fejér's theorem.
  \end{exercise}
  \begin{solution}

  \end{solution}

  \begin{exercise}[Rudin 8.16]
    Prove a pointwise version of Fejér's theorem: If $f \in \mathscr{R}$ and $f(x+), f(x-)$ exist for some $x$, then
    \begin{equation}
      \lim_{N \to \infty} \sigma_N(f; x) = \frac{1}{2} [f(x+) + f(x-)].
    \end{equation}
  \end{exercise}
  \begin{solution}

  \end{solution}

  \begin{exercise}[Rudin 8.17]
    Assume $f$ is bounded and monotonic on $[-\pi, \pi]$, with Fourier coefficients $c_n$.
    \begin{enumerate}
      \item Use Exercise 17 of Chap. 6 to prove that $\{nc_n\}$ is a bounded sequence.
      \item Combine (a) with Exercise 16 and with Exercise 14(e) of Chap. 3, to conclude that
        \begin{equation}
          \lim_{N \to \infty} s_N(f; x) = \frac{1}{2} [f(x+) + f(x-)]
        \end{equation}
        for every $x$.
      \item Assume only that $f \in \mathscr{R}$ on $[-\pi, \pi]$ and that $f$ is monotonic in some segment $(\alpha, \beta) \subset [-\pi, \pi]$. Prove that the conclusion of (b) holds for every $x \in (\alpha, \beta)$.
    \end{enumerate}
  \end{exercise}
  \begin{solution}

  \end{solution}

  \begin{exercise}[Rudin 8.18]
    Define
    \begin{align}
      f(x) &= x^3 - \sin^2 x \tan x \\
      g(x) &= 2x^2 - \sin^2 x - x \tan x.
    \end{align}
    Find out, for each of these two functions, whether it is positive or negative for all $x \in (0, \pi/2)$, or whether it changes sign. Prove your answer.
  \end{exercise}
  \begin{solution}

  \end{solution}

  \begin{exercise}[Rudin 8.19]
    Suppose $f$ is a continuous function on $R^1, f(x+2\pi) = f(x)$, and $\alpha/\pi$ is irrational. Prove that
    \begin{equation}
      \lim_{N \to \infty} \frac{1}{N} \sum_{n=1}^N f(x + n\alpha) = \frac{1}{2\pi} \int_{-\pi}^\pi f(t) \, dt
    \end{equation}
    for every $x$. \textit{Hint: Do it first for $f(x) = e^{ikx}$.}
  \end{exercise}
  \begin{solution}

  \end{solution}

  \begin{exercise}[Rudin 8.20]
    The following simple computation yields a good approximation to Stirling's formula. For $m = 1, 2, 3, \dots$, define
    \begin{equation}
      f(x) = (m+1-x) \log m + (x-m) \log (m+1)
    \end{equation}
    if $m \leq x \leq m+1$, and define
    \begin{equation}
      g(x) = \frac{x}{m} - 1 + \log m
    \end{equation}
    if $m - \frac{1}{2} \leq x < m + \frac{1}{2}$. Conclude that
    \begin{equation}
      \frac{7}{8} < \log(n!) - (n + \frac{1}{2}) \log n + n < 1
    \end{equation}
    for $n = 2, 3, 4, \dots$. Thus
    \begin{equation}
      e^{7/8} < \frac{n!}{(n/e)^n \sqrt{n}} < e.
    \end{equation}
  \end{exercise}
  \begin{solution}

  \end{solution}

  \begin{exercise}[Rudin 8.21]
    Let $L_n = \frac{1}{2\pi} \int_{-\pi}^\pi |D_n(t)| \, dt \quad (n = 1, 2, 3, \dots)$. Prove that there exists a constant $C > 0$ such that $L_n > C \log n$, or, more precisely, that the sequence
    \begin{equation}
      \left\{ L_n - \frac{4}{\pi^2} \log n \right\}
    \end{equation}
    is bounded.
  \end{exercise}
  \begin{solution}

  \end{solution}

  \begin{exercise}[Rudin 8.22]
    If $\alpha$ is real and $-1 < x < 1$, prove Newton's binomial theorem
    \begin{equation}
      (1+x)^\alpha = 1 + \sum_{n=1}^\infty \frac{\alpha(\alpha-1) \cdots (\alpha-n+1)}{n!} x^n.
    \end{equation}
    Show also that
    \begin{equation}
      (1-x)^{-\alpha} = \sum_{n=0}^\infty \frac{\Gamma (n+\alpha)}{n! \Gamma(\alpha)} x^n
    \end{equation}
    if $-1 < x < 1$ and $\alpha > 0$.
  \end{exercise}
  \begin{solution}

  \end{solution}

  \begin{exercise}[Rudin 8.23]
    Let $\gamma$ be a continuously differentiable closed curve in the complex plane, with parameter interval $[a, b]$, and assume that $\gamma(t) \neq 0$ for every $t \in [a, b]$. Define the index of $\gamma$ to be
    \begin{equation}
      \text{Ind} (\gamma) = \frac{1}{2\pi i} \int_a^b \frac{\gamma'(t)}{\gamma(t)} \, dt.
    \end{equation}
    Prove that $\text{Ind} (\gamma)$ is always an integer.
  \end{exercise}
  \begin{solution}

  \end{solution}

  \begin{exercise}[Rudin 8.24]
    Let $\gamma$ be as in Exercise 23, and assume in addition that the range of $\gamma$ does not intersect the negative real axis. Prove that $\text{Ind} (\gamma) = 0$.
  \end{exercise}
  \begin{solution}

  \end{solution}

  \begin{exercise}[Rudin 8.25]
    Suppose $\gamma_1$ and $\gamma_2$ are curves as in Exercise 23, and $|\gamma_1(t) - \gamma_2(t)| < |\gamma_1(t)|$ for $a \leq t \leq b$. Prove that $\text{Ind} (\gamma_1) = \text{Ind} (\gamma_2)$.
  \end{exercise}
  \begin{solution}

  \end{solution}

  \begin{exercise}[Rudin 8.26]
    Let $\gamma$ be a closed curve in the complex plane (not necessarily differentiable) with parameter interval $[0, 2\pi]$, such that $\gamma(t) \neq 0$ for every $t \in [0, 2\pi]$. Prove that there exists a common value $\text{Ind} (\gamma)$ and that the statements of Exercises 24 and 25 hold without any differentiability assumption.
  \end{exercise}
  \begin{solution}

  \end{solution}

  \begin{exercise}[Rudin 8.27]
    Let $f$ be a continuous complex function defined in the complex plane. Suppose there is a positive integer $n$ and a complex number $c \neq 0$ such that
    \begin{equation}
      \lim_{|z| \to \infty} z^{-n} f(z) = c.
    \end{equation}
    Prove that $f(z) = 0$ for at least one complex number $z$.
  \end{exercise}
  \begin{solution}

  \end{solution}

  \begin{exercise}[Rudin 8.28]
    Let $\bar{D}$ be the closed unit disc in the complex plane. Let $g$ be a continuous mapping of $\bar{D}$ into the unit circle $T$. Prove that $g(z) = -z$ for at least one $z \in T$.
  \end{exercise}
  \begin{solution}

  \end{solution}

  \begin{exercise}[Rudin 8.29]
    Prove that every continuous mapping $f$ of $\bar{D}$ into $\bar{D}$ has a fixed point in $\bar{D}$. (This is the 2-dimensional case of Brouwer's fixed-point theorem.)
  \end{exercise}
  \begin{solution}

  \end{solution}

  \begin{exercise}[Rudin 8.30]
    Use Stirling's formula to prove that
    \begin{equation}
      \lim_{x \to \infty} \frac{\Gamma (x+c)}{x^c \Gamma (x)} = 1
    \end{equation}
    for every real constant $c$.
  \end{exercise}
  \begin{solution}

  \end{solution}

  \begin{exercise}[Rudin 8.31]
    In the proof of Theorem 7.26 it was shown that
    \begin{equation}
      \int_{-1}^1 (1-x^2)^n \, dx \geq \frac{4}{3\sqrt{n}}
    \end{equation}
    for $n = 1, 2, 3, \dots$. Use Theorem 8.20 and Exercise 30 to show the more precise result
    \begin{equation}
      \lim_{n \to \infty} \sqrt{n} \int_{-1}^1 (1-x^2)^n \, dx = \sqrt{\pi}.
    \end{equation}
  \end{exercise}
  \begin{solution}

  \end{solution}

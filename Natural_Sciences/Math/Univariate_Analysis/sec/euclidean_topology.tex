\section{Euclidean Topology} 

  With the construction of the real line and the real space, the extra properties of completeness, norm, and order (for the real line) allows us to restate these topological properties in terms of these ``higher-order'' properties. It also proves much more results than for general topological spaces. Therefore, the next few sections will focus on reiterating the topological properties of $\mathbb{R}$ and $\mathbb{R}^n$ (this can be done slightly more generally for metric spaces, but we talk about this in point-set topology). 

  \begin{theorem}[Euclidean Topology]
    The \textbf{Euclidean topology} $\mathscr{T}$ on $\mathbb{R}^n$ is defined in the equivalent ways: 
    \begin{enumerate}
      \item If $n = 1$, it is the order topology generated by the basis of open intervals $(a, b)$ for $a < b$. 
      \item It is the metric topology generated by the basis of open balls under the $L^p$ metric for $p \geq 1$. 
      \item It is the $n$-fold product topology of the Euclidean topology on $\mathbb{R}$. 
    \end{enumerate}
  \end{theorem}
  \begin{proof} 
    Let's call these sets $\mathscr{T}_1, \mathscr{T}_2, \mathscr{T}_3$. We first show that these are indeed topologies. 
    \begin{enumerate}
      \item This is proven in the \hyperref[pst-def:order-topology]{definition of the order topology}. 
      \item This is proven for the $L^2$ metric in the \hyperref[pst-def:metric-topology]{definition of the metric topology}, and then we prove \hyperref[pst-thm:lp-norms-euclidean-topology]{equivalence to $L^p$ metrics}. 
      \item This is proven in the \hyperref[pst-def:product-topology]{definition of the product topology}. 
    \end{enumerate}

    Now it remains to show that these topologies are all the same. For $n=1$, it suffices to show $\mathscr{T}_1 = \mathscr{T}_2$. We do this by comparing their bases; let $\mathscr{B}_k$ be the basis of $\mathscr{T}_k$. 
    \begin{enumerate}
      \item $\mathscr{T}_1 \subset \mathscr{T}_2$. Given any open interval $(a, b) \in \mathscr{B}_1$, $(a, b)$ is the open ball with center $\frac{a + b}{2}$ and radius $\frac{b - a}{2}$ and so it is in $\mathscr{B}_2$. 
      \item $\mathscr{T}_2 \subset \mathscr{T}_1$. Given an open interval with center $x$ and radius $r$, the corresponding interval $(x - r, x + r)$ is in $\mathscr{B}_1$. 
    \end{enumerate}
    Now we show that $\mathscr{T}_2 = \mathscr{T}_3$ for $\mathbb{R}^n$. 
    \begin{enumerate}
      \item We show that the product topology is the same as the metric topology under the $L^\infty$ norm. 
      \item The equivalence of all $L^p$ norms is already proven \hyperref[pst-thm:lp-norms-euclidean-topology]{here}. 
    \end{enumerate}
  \end{proof}

  \begin{example}[Examples of Open and Not Open Sets]
    Here are some examples of sets which are open and not open. 
    \begin{enumerate}
      \item $U=\{(x,y)\in \mathbb{R}^2 : x^2+y^2 \neq 1\}$ is open since for every point $x \in U$, we just need to find a radius $\epsilon > 0$ that is smaller than its distance to the unit circle. 
      \item $(a, b) \times (c, d) \subset \mathbb{R}^2$ is open since given a point $x$, we can take the minimum of its distance between the two sides of the rectangle and construct an open ball. 
      \item $S=\{(x,y)\in \mathbb{R}^2:xy\neq 0\}$ is open since given a point $x \in S$, we can take the minimum of the distance between it and the $x$ and $y$ axes. 
      \item The set of all complex $z$ such that $|z| \leq 1$ is not open since we cannot construct open balls at the boundary points that are fully contained in the set. 
      \item The set $S = \{1/n\}_{n \in \mathbb{N}}$ is not open since given any point $x = 1/n$, we can construct an open ball with radius $\epsilon < 1/(n+1)$, which contains irrationals that are not in $S$. 
    \end{enumerate}
  \end{example}

  \begin{corollary}
    Every closed bounded interval in $\mathbb{R}$ is compact. 
  \end{corollary}
  \begin{proof}
    Let $I = [a, b]$. Then if $x, y \in I$, $|x - y| \leq b - a = \delta$. Now by contradiction, suppose that there exists an open cover $\{U_\alpha\}$ of $I$ which contains no finite subcover of $I$. Then letting $c = (a + b)/2$, at least one of the two intervals $[a, c], [c, b]$ cannot have a finite subcovering (otherwise their finite union can be covered). WLOG let it be $[a, c]$. We keep subdividing and get the sequence of nested intervals. 
    \begin{equation}
      I \supset I_1 \supset I_2 \supset \ldots 
    \end{equation}
    We know that $I_n$ is not covered by any finite subcollection of $\{U_\alpha\}$ and if $x, y \in I_n$, then $|x - y| < 2^{-n} \delta$. From the nested intervals theorem, there exists a point $z$ lying in every $I_n$. There must then be an open neighborhood $U_z$ in the open cover, and by definition of openness there exists a $\epsilon > 0$ s.t. $z \in B_\epsilon (z) \subset U_z$. By the Archimidean property, we can set $n$ so large that $2^{-n} \delta < \epsilon$ and this means that $B_\epsilon (z) \supset I_n$, which contradicts the fact that $I_n$ is not covered by a finite subcollection. Therefore $I$ is compact. 
  \end{proof}


  Now here are some useful properties. 

  \begin{theorem}[Countability]
    $\mathbb{R}^n$ is 2nd countable. 
  \end{theorem}

  \begin{theorem}[Separability]
    $\mathbb{R}^n$ is Hausdorff. 
  \end{theorem}
  \begin{proof}
    It is a metric topology. 
  \end{proof}

  \begin{theorem}[Connectedness]
    $\mathbb{R}^n$ is 
    \begin{enumerate}
      \item connected, 
      \item path connected, 
      \item locally connected, 
      \item locally path connected
    \end{enumerate}
  \end{theorem}

  \begin{theorem}[Compactness]
    $\mathbb{R}^n$ is 
    \begin{enumerate}
      \item not compact, 
      \item locally compact. 
    \end{enumerate}
  \end{theorem}

  \begin{theorem}[Convexity]
    An open ball is convex in a normed vector space. What happens if we weaken it to a metric? 
  \end{theorem}
  \begin{proof}
    The normed part is important here, as the properties of the metric is not sufficient. Given $B_r (p)$, $x, y \in B_r (p)$ implies that $||x - p|| < r$ and $||y - p ||<r$. Therefore, 
    \begin{align}
      ||t x + (1 - t)y - p|| & = ||t x - tp + (1 - t) y - (1 - t) p|| \\
      & \leq t ||x - p|| + (1 - t) ||y - p|| \\
      & = t r + (1 - t) r = r 
    \end{align}
  \end{proof}

  \begin{theorem}
    Let $E$ be a nonempty set of real numbers which is bounded above. Let $y = \sup{E}$. Then $y \in \overline{E}$. Hence $y \in E$ if $E$ is closed. 
  \end{theorem}
  \begin{proof}
    Assume that $y$ is not a limit point of $E$. Then, there exists some $\epsilon > 0$ s.t. $(y - \epsilon, y + \epsilon)$ does not intersect with $E$. This means that $y - \epsilon$ is an upper bound of $E$, and so $y$ is not the supremum. 
  \end{proof}

  The general notion of compactness for topological spaces is not needed for analysis. Rather, we make use of the following theorem which allows us to focus on the compactness of subsets in Euclidean spaces $\mathbb{R}^n$. 

  \begin{theorem}[Heine-Borel]
    Let $E \subset \mathbb{R}^n$. The following are equivalent. 
    \begin{enumerate}
      \item $E$ is closed and bounded. 
      \item $E$ is compact. 
      \item $E$ is sequentially compact. 
      \item $E$ is limit point compact. 
    \end{enumerate}
  \end{theorem}
  \begin{proof}
    To prove the equivalence of the last three, note that $X$ is a metric space, so it is Hausdorff. Therefore, all three forms of compactness is the same. 
  \end{proof}

  \begin{example}[Open Sets in Real Plane are Not Compact]
    An open set in $\mathbb{R}^2$ is not compact. Take the open rectangle $ R = (0,1)^2 \subset \mathbb{R}^2$. There exists an infinite cover of $R$
    \begin{equation}
      R = \bigcup_{n=0}^\infty \big(0,1\big) \times \bigg( 0, \frac{ 2^{n+1} - 1}{2^{n+1}} \bigg)
    \end{equation}
    that does not have a finite subcover. 
  \end{example}

  \begin{corollary}[Bolzano-Weierstrass Theorem]
    Every bounded sequence in $\mathbb{R}^n$ has a limit point, i.e. contains a convergent subsequence. 
  \end{corollary}
  \begin{proof}
    The fact that the infinite sequence is bounded means that there exists some closed subset $I \in \mathbb{R}^n$ that contains all point of the sequence. By definition $I$ is compact, so by the Heine-Borel theorem, every cover of $I$ has a finite subcover. 

    Now, assume that there exists an infinite sequence in $I$ that is not convergent, i.e. has no limit point. Then, each point $x_i \in I$ would have a neighborhood $U(x_i)$ containing at most a finite number of points in the sequence. We can define $I$ such that the union of the neighborhoods is a cover of $I$. That is, 
    \[I \subset \bigcup_{i=1}^\infty U(x_i)\]
    However, since every $U(x_i)$ contains at most a finite number of points, we must have an infinite open neighborhoods to cover $I \implies$ we cannot have a finite subcover. This contradicts the fact that $I$ is compact. 
  \end{proof}
  \begin{proof}
    It suffices to prove that there exists a monotonic sequence within a bounded sequence $(x_n)$. 
  \end{proof}

  \begin{theorem}[Connectedness]
    A subset $E$ of the real line $\mathbb{R}$ is connected if and only if it has the following property: if $x \in E, y \in E$ and $x < z < y$, then $z \in E$. 
  \end{theorem}
  \begin{proof}

  \end{proof}

  \begin{definition}[Perfect Sets]
    A set $P$ is perfect if it is closed and all of its points are limit points of $P$. In other words, the limit points of $P$ and $P$ itself coincide. 
    \begin{equation}
      P^\prime = P
    \end{equation}
  \end{definition}

  \begin{theorem}
    Let $P$ be a nonempty perfect set in $\mathbb{R}^k$. Then $P$ is uncountable. 
  \end{theorem}


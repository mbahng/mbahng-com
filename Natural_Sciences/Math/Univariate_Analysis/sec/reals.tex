\section{The Real Numbers}

  By constructing $\mathbb{Q}$ and its topology in my algebra and topology notes, we can talk about convergence. The first question to ask (if you were the first person inventing the reals) is ``how do I know that there exists some other numbers at all?'' The first clue is trying to find the side length of a square with area $2$. As we see, this number is not rational. 

  \begin{theorem}[$\sqrt{2}$ is Not Rational]
    \label{thm:sqrt2-irrational}
    There exists no $x \in \mathbb{Q}$ s.t. $x^2 = 2$. 
  \end{theorem}
  \begin{proof}
    Assume such a number $x = p/q$ exists, where $\mathrm{gcd}(p, q) = 1$. Then, by the field axioms of $\mathbb{Q}$, we can deduce that 
    \begin{equation}
      2 = \frac{p^2}{q^2} \implies 2 q^2 = p^2
    \end{equation}
    This implies that $2 \mid p$, so we have $p = 2p_0$, and we can write $2 q^2 = 4 p_0^2$. Dividing both sides by $2$, we get $q^2 = 2p_0^2$, which implies that $2 \mid q$. This contradicts the fact that $p$ and $q$ are coprime. 
  \end{proof} 

  We can ``imagine'' that a square with area $2$ certainly exists, but the fact that its side length is undefined is certainly unsettling. I don't know about you, but I would try to ``invent'' $\sqrt{2}$. We can do this in 4 distinct ways, though some may be more similar than others. 

  \begin{enumerate}
    \item \textit{Dedekind Completeness}. I would define the set of all rationals such that $x^2 < 2$, and try to define $\sqrt{2}$ as the max or supremum of this set. We will quickly find that neither the max nor the supremum exists in $\mathbb{Q}$, and this motivates the definition for \textit{Dedekind completeness}. By construction, we can also see that every nonempty subset of this set must have a least upper bound---that is, the specific Dedekind cut.  

    \item \textit{Cauchy Completeness}. I write out the decimal expansion one by one, which gives our first exposure to sequences. 
    \begin{equation}
      1, 1.4, 1.41, 1.414, \ldots
    \end{equation} 
    It is clear that on $\mathbb{Q}$, this sequence does not converge. Our intuition tells that that if the terms get closer and closer to each other, they must be getting closer and closer to \textit{something}, though that something is not in $\mathbb{Q}$. This motivates the definition for \textit{Cauchy completeness}. 

    \item \textit{Nested Interval Completeness}. I would write out maybe some nested intervals so that $\sqrt{2}$ \textit{must} lie within each interval. 
    \begin{equation}
      [1, 2] \supset [1.4, 1.5] \supset [1.41, 1.42] \supset \ldots 
    \end{equation}
    This motivates the definition of \textit{nested-interval completeness}. 

    \item \textit{Least Upper Bound Completeness}. I would like to add more elements such that every subset of $\mathbb{Q}$ that is bounded from above has a least upper bound. Therefore, we can just define  
    \begin{equation}
      \sqrt{2} \coloneqq \sup\{x \in \mathbb{Q} \mid x^2 < 2\}
    \end{equation}
    which is guaranteed to exist by construction. 
  \end{enumerate}

  All four of these methods points at the same intuition that there should not be any ``gaps'' or ``missing points'' in the set that we will construct to be $\mathbb{R}$, which is the general notion of \textit{completeness}. This contrasts with the rational numbers, whose corresponding number line has a "gap" at each irrational value. Even though constructing the reals with one method is sufficient, knowing the different flavors in which completeness manifests is very useful. It allows us to view properties of $\mathbb{R}$ through different lens. 

  The main division between these four properties is that the first two are methods to directly \textit{construct} the reals from $\mathbb{Q}$, while the latter two are more \textit{axiomatic properties} that we use to verify completeness for a given set. In the next two sections, we will take the rationals and add on extra elements using Dedekind cuts and Cauchy sequences. However, it isn't as conventional (though possible) to construct them as the single point contained in a sequence of nested intervals nor as a supremum of all upper-bounded sets. In fact, an alternative way to construct the reals is to define it axiomatically---as a totally ordered field with either the LUB property or the nested interval property.\footnote{In fact, you also need the Archimidean principle, but we'll talk about this later.} 

  Therefore, in the following sections, we will 
  \begin{enumerate}
    \item first define the relevant notion of completeness, 
    \item show that the rationals are not complete 
    \item and then construct the completed version of the rationals as a version of the reals. 
  \end{enumerate}
  Once we have done this for all three versions, we will unify them by proving they are all equivalent. 

  There is a second essential property of the reals that is not talked about as often is the Archimidean principle. 

  \begin{definition}[Archimidean Principle]
    An ordered ring $(X, +, \cdot, \leq)$ that embeds the naturals $\mathbb{N}$\footnote{as in, there exists an ordered ring homomorphism $\iota: \mathbb{N} \rightarrow X$} is said to obey the \textbf{Archimedean principle} if given any $x, y \in X$ s.t. $x, y > 0$, there exists an $n \in \mathbb{N}$ s.t. $\iota(n) \cdot x > y$. Usually, we don't care about the canonical injection and write $nx > y$. 
  \end{definition} 
  
  \begin{lemma}[Rationals are Archimidean]
    $\mathbb{Q}$ satisfies the Archimidean principle. 
  \end{lemma}
  \begin{proof}
    Take any $x = p_1/q_1, y = p_2 / q_2 \in \mathbb{Q}$. Then, take $n = q_1 p_2$, and we get 
    \begin{equation}
      n x = q_1 p_2 \frac{p_1}{q_1} = p_1 p_2 > \frac{p_2}{q_2} = y \iff p_1 > \frac{1}{q_2}
    \end{equation}
    which must be true since $p_1 \geq 1 \geq \frac{1}{q_2}$. 
  \end{proof}

  Usually, we just construct $\mathbb{R}$ right out of $\mathbb{Q}$, and it turns out that the Archimidean principle just trivially follows. However, if we construct $\mathbb{R}$ axiomatically (without the rationals), it needs to be stated. In this axiomatic formulation, we will find that certain types of completeness---like Dedekind completeness---actually \textit{implies} the Archimidean principle, while others---like Cauchy and nested-intervals completeness---does not. Therefore, in a sense, Dedekind completeness is the ``strongest'' form of completeness. 

\subsection{Dedekind Completeness}  

  This is an explicit construction from the rationals. 

  \begin{definition}[Dedekind Cut] 
    A \textbf{Dedekind cut} is a partition of the rationals $\mathbb{Q} = A \sqcup A^\prime$ satisfying the three properties.\footnote{This can really be defined for any totally ordered set. } 
    \begin{enumerate}
      \item $A \neq \emptyset$ and $A \neq \mathbb{Q}$.\footnote{By relaxing this property, we can actually complete $\mathbb{Q}$ to the extended real number line. }
      \item $x < y$ for all $x \in A, y \in A^\prime$. 
      \item The maximum element of $A$ does not exist in $\mathbb{Q}$. 
    \end{enumerate}
    The minimum of $A^\prime$ may exist in $\mathbb{Q}$, and if it does, the cut is said to be \textbf{generated} by $\min A^\prime$. 
  \end{definition}

  Note that in $\mathbb{Q}$, there will be two types of cuts: 
  \begin{enumerate}
    \item ones that are generated by rational numbers, such as 
    \begin{equation}
      A = \{x \in \mathbb{Q} \mid x < 2/3 \}, A^\prime = \{ x \in \mathbb{Q} \mid x \geq 2/3 \} 
    \end{equation}
    \item and the ones that are not 
    \begin{equation}
      A = \{x \in \mathbb{Q} \mid x^2 < 2 \}, A^\prime = \{x \in \mathbb{Q} \mid x^2 \geq 2 \}
    \end{equation}
  \end{enumerate}
  We can intuitively see that the set of all Dedekind cuts $(A, A^\prime)$ will ``extend'' the rationals into a bigger set. We can then define some operations and an order to construct this into an ordered field, and finally it will have the property that we call ``completeness.''

  \begin{definition}[Dedekind Completeness]
    A totally ordered algebraic field $\mathbb{F}$ is \textbf{Dedekind-complete} if every Dedekind cut of $\mathbb{F}$ is generated by an element of $\mathbb{F}$. 
  \end{definition}

  From above, $\mathbb{Q}$ is not Dedekind-complete since the counter-example is given above for the cut 
  \begin{equation}
    A = \{x \in \mathbb{Q} \mid x^2 < 2 \}, A^\prime = \{x \in \mathbb{Q} \mid x^2 \geq 2 \}
  \end{equation}

  These Dedekind cuts are simply subsets of the power set of $\mathbb{Q}$, which always exists due to the \hyperref[st-power-set-axiom]{power set axiom}. Therefore, we can simply use the axiom of restricted comprehension (?) to create a well-defined set of Dedekind cuts.    

  \begin{definition}[Reals as the Dedekind-Completion of Rationals]
    Let $\mathbb{R}_D$ be the set of all Dedekind cuts $A$\footnote{For convenience we can uniquely represent $(A, A^\prime)$ with just $A$ since $A^\prime = \mathbb{Q} \setminus A$. }  of $\mathbb{Q}$. 

    \begin{equation}
      \mathbb{R}_D \coloneqq \{ A \in 2^{\mathbb{Q}} \mid (A, A^c) \text{ is a Dedekind cut}\}
    \end{equation}
    By doing this we can intuitively think of a real number as being represented by the set of all smaller rational numbers. Let $A, B \in \mathbb{R}_D$ two Dedekind cuts. Then, we define the following order and operations. 
    \begin{enumerate}
      \item \textit{Order}. $A \leq_{\mathbb{R}} B \iff A \subset B$. 
      \item \textit{Addition}. $A +_{\mathbb{R}} B \coloneqq \{ a +_{\mathbb{Q}} b \mid a \in A, b \in B \}$. 
      \item \textit{Additive Identity}. $0_{\mathbb{R}} \coloneqq \{x \in \mathbb{Q} \mid x < 0 \}$. 
      \item \textit{Additive Inverse}. $-B \coloneqq \{ a - b \mid a < 0 , b \in (\mathbb{Q} \setminus B) \}$.
      \item \textit{Multiplication}. If $A, B \geq 0$, then we define $A \times_{\mathbb{R}} B \coloneqq \{ a \times_{\mathbb{Q}} b \mid a \in A, b \in B, a, b \geq 0\} \cup 0_{\mathbb{R}}$. If $A$ or $B$ is negative, then we use the identity $A \times B = -(A \times_{\mathbb{R}} -B) = -(-A \times_{\mathbb{R}} B) = (-A \times_{\mathbb{R}} -B)$ to convert $A, B$ to both positives and apply the previous definition. 
      \item \textit{Multiplicative Identity}. $1_{\mathbb{R}} = \{x \in \mathbb{Q} \mid x < 1 \}$. 
      \item \textit{Multiplicative Inverse}. If $B > 0$, $B^{-1} \coloneqq \{ a \times_{\mathbb{Q}} b^{-1} \mid a \in 1_{\mathbb{R}}, b \in (\mathbb{Q} \setminus B) \}$. If $B$ is negative, then we compute $B^{-1} = -((-B)^{-1})$ by first converting to a positive number and then applying the definition above. 
    \end{enumerate}

    We claim that $(\mathbb{R}, +_{\mathbb{R}}, \times_{\mathbb{R}}, \leq_{\mathbb{R}})$ is a Dedekind-complete totally ordered field, and the canonical injection $\iota: \mathbb{Q} \rightarrow \mathbb{R}$ defined 
    \begin{equation}
      \iota(q) = \{x \in \mathbb{Q} \mid x < q \}
    \end{equation}
    is an ordered field isomorphism. 
  \end{definition} 
  \begin{proof}
    
  \end{proof}

  By the canonical injections $\mathbb{N} \rightarrow \mathbb{Z} \rightarrow \mathbb{Q} \rightarrow \mathbb{R}$, we can talk about whether this set has the Archimedean property. By construction, Archimidean is trivial since $\mathbb{R}_D$ contains $\mathbb{Q}$ which is Archimidean. 

  \begin{theorem}[Dedekind Reals is Archimedean]
    $\mathbb{R}_D$ satisfies the Archimedean principle. 
  \end{theorem}
  \begin{proof}
    
  \end{proof}

  \begin{definition}[Axiomatic Construction of Dedekind-Reals]
    $\mathbb{R}_D^\prime$ is a totally ordered field that is Dedekind complete.
  \end{definition}

  \begin{theorem}[Axiomatic Dedekind Reals is Archimedean]
    $\mathbb{R}_D^\prime$ satisfies the Archimedean principle. 
  \end{theorem}
  \begin{proof}
    Assume that this property doesn't hold. Then for any fixed $x$, $nx < y$ for all $n \in \mathbb{N}$. Consider the set 
    \begin{equation}
      A = \bigcup_{n \in \mathbb{N}} (-\infty, nx), \qquad B = \mathbb{R} \setminus A
    \end{equation}
    $A$ by definition is nonempty, and $B$ is nonempty since it contains $y$. Then, we can show that $a \in A, b \in B \implies a < b$ using proof by contradiction. Assume that there exists $a^\prime \in A, b^\prime \in B$ s.t. $a^\prime > b^\prime$. Since $a^\prime \in A$, there exists a $n^\prime \in \mathbb{N}$ s.t. $a^\prime \in (-\infty, n^\prime x) \iff a^\prime < n^\prime x$. But by transitivity of order, this means $b^\prime < n^\prime x \iff b^\prime \in (-\infty, n^\prime x) \implies b^\prime \in A$. 

    Going back to the main proof, we see that $A$ is upper bounded by $y$, and so by the least upper bound property it has a supremum $z = \sup{A}$. 
    \begin{enumerate}
      \item If $z \in A$, then by the induction principle\footnote{Note that $\mathbb{N}$ is defined recursively as $1 \in \mathbb{N}$ and if $n \in \mathbb{N}$, then $n+1 \in \mathbb{N}$. } $z + x \in A$, contradicting that $z$ is an upper bound. 
      \item If $z \not\in A$, then by the induction principle\footnote{The contrapositive of the recursive definition of $\mathbb{N}$ is: if $n \not\in \mathbb{N}$, then $n-1 \not\in \mathbb{N}$.} $z-x \not\in A \implies z-x \in B$. Since every element of $B$ upper bounds $A$ and since $x > 0$, this means that $z-x < z$ is a smaller upper bound of $A$, contradicting that $z$ is a least upper bound. 
    \end{enumerate}
    Therefore, it must be the case that $nx > y$ for some $n \in \mathbb{N}$. 
  \end{proof}

\subsection{Cauchy Completeness} 

  In many cases we are not working with ordered fields, and so different types of completeness may be more useful. In actual practice, you tend to use Cauchy completeness which only assumes a metric structure. 

  \begin{definition}[Cauchy Sequence]
    A sequence $(x_n)_n$ in a metric space $(X, d)$ is a \textbf{Cauchy sequence} if $\forall \epsilon > 0$, $\exists N \in \mathbb{N}$ s.t.  
    \begin{equation}
      n, m \geq N \implies d(x_n, x_m) < \epsilon
    \end{equation}
  \end{definition}

  To motivate this definition, note that in a general topological space $X$, we can define convergence of a sequence $x_n \to x$ perfectly fine. However, take some subset $U \subset X$, and let $x$ be a limit point of $U$. In this case, $x_n$ does not converge in $U$, but it does converge to something outside of $U$---namely, $x \in X$. This is similar to $\mathbb{Q} \subset \mathbb{R}$, where $x$ is an irrational point. However, we are trying to \textit{construct} $\mathbb{R}$, so Cauchy convergence allows us to speak of convergence without actually referring to \textit{what} a sequence is converging to. 

  Note that it is not sufficient to say that a sequence is Cauchy by claiming that each term becomes arbitrarily close to the preceding term. That is, 
  \begin{equation}
    \lim_{n \rightarrow \infty} d(x_{n+1}, x_{n}) = 0
  \end{equation}
  It is often more convenient to think of the limit of the \textit{diameter} of rest of the sequence. That is, a sequence is Cauchy if 
  \begin{equation}
    \lim_{n \rightarrow \infty} \mathrm{diam}\{x_{m}\}_{m \geq n} = 0
  \end{equation}

  \begin{example}[Adjacent Terms Converging Doesn't Imply Sequence is Cauchy]
    For example, look at the sequence 
    \begin{equation}
      a_n = \sqrt{n} \implies a_{n+1} - a_{n} = \frac{1}{\sqrt{n+1} + \sqrt{n}} < \frac{1}{2\sqrt{n}}
    \end{equation}
    However, it is clear that $a_n$ gets arbitrarily large, meaning that a finite interval can contain at most a finite number of terms in $\{a_n\}$. 
  \end{example}

  It is trivial that convergence implies Cauchy convergence, but the other direction is not true. Therefore, we would like to work in a space where these two are equivalent, and this is called completeness. Therefore, we can construct the reals as equivalence classes over Cauchy sequences. Rather than using the order, we take advantage of the metric. 

  \begin{definition}[Cauchy Completeness]
    A metric space $(X, d)$ is \textbf{Cauchy complete} if every Cauchy sequence in that space converges to an element in $X$. 
  \end{definition} 
  
  $\mathbb{Q}$ is not Cauchy-complete. Let $a_n$ be the largest number $x$ up to the $n$th decimal expansion such that $x^2$ does not exceed $2$. The first few terms are 
  \begin{equation}
    1.4, 1.41, 1.414, \ldots
  \end{equation}
  In this case, we can see that this is Cauchy since at the $n$th element and on, the first $n$ decimal places are kept fixed and so the most that the rest of the sequence can deviate by is $10^{-n}$. 

  \begin{definition}[Reals as the Cauchy-Completion of the Rationals]
    Let $\mathbb{R}_C$ be the quotient space of all Cauchy (under the Euclidean metric) sequences $(x_n)$ of rational numbers with the equivalence relation $(x_n) = (y_n)$ iff their difference tends to $0$.\footnote{This equivalence class reflects that the same real number can be approximated in many different sequences. In fact, this shows \textit{by definition} that $1, 1, \ldots$ and $0.9, 0.99, 0.999, \ldots$ are the same number!} That is, for every rational $\epsilon > 0$, there exists an integer $N$ s.t. for all naturals $n > N$, $|x_n - y_n| < \epsilon$. 
    \begin{enumerate}
      \item \textit{Order}. $(x_n) \leq_{\mathbb{R}} (y_n)$ iff $x = y$ or there exists $N \in \mathbb{N}$ s.t. $x_n \leq_{\mathbb{Q}} y_n$ for all $n > N$. 
      \item \textit{Addition}. $(x_n) + (y_n) \coloneqq (x_n + y_n)$. 
      \item \textit{Additive Identity}. $0_{\mathbb{R}} \coloneqq (0_{\mathbb{Q}})$. 
      \item \textit{Additive Inverse}. $-(x_n) \coloneqq (-x_n)$. 
      \item \textit{Multiplication}. $(x_n) \times_{\mathbb{R}} (y_n) = (x_n \times_{\mathbb{Q}} y_n)$. 
      \item \textit{Multiplicative Identity}. $1_{\mathbb{R}} \coloneqq (1)$. 
      \item \textit{Multiplicative Inverse}. $(x_n)^{-1} \coloneqq (x_n^{-1})$. 
    \end{enumerate}
    We claim that $(\mathbb{R}, +_{\mathbb{R}}, \times_{\mathbb{R}}, \leq_{\mathbb{R}})$ is a totally ordered field, and the canonical injection $\iota: \mathbb{Q} \rightarrow \mathbb{R}$ defined 
    \begin{equation}
      \iota(q) = (q)
    \end{equation} 
    is an ordered field isomorphism. Finally, by construction $\mathbb{R}$ is Cauchy-complete. 
  \end{definition}
  \begin{proof}
    
  \end{proof}

  \begin{theorem}[Cauchy Reals is Archimidean]
    $\mathbb{R}_C$ satisfies the Archimedean principle. 
  \end{theorem}
  \begin{proof}
    
  \end{proof}

  The best thing about Cauchy completeness is that we can just take $\mathbb{Q}^n$ to create $\mathbb{R}^n$. It becomes quite general. However, note that first, Cauchy completion depends on \textit{which} metric you use to complete it. 

  \begin{example}[P-adic Numbers]
    Let $p$ be a prime number. For a non-zero rational $x = p^k \cdot \frac{a}{b}$ where $p \nmid a, b$, define the \textbf{p-adic norm}
    \begin{equation}
      |x|_p = p^{-k}, \qquad |0|_p = 0
    \end{equation}
    This measures divisibility by $p$: the more $p$ divides $x$, the smaller $|x|_p$. For example, $|8|_2 = 2^{-3} = 1/8$ and $|3|_2 = 1$.

    The \textbf{p-adic numbers} $\mathbb{Q}_p$ are the Cauchy completion of $\mathbb{Q}$ with respect to the p-adic metric $d_p(x,y) = |x - y|_p$. This set not only does not satisfy the Archimidean principle; it doesn't even have a natural ordering! 
  \end{example} 

  \begin{definition}[Axiomatic Construction of Cauchy-Reals]
    $\mathbb{R}_D^\prime$ is a totally ordered field that is Cauchy complete and that satisfies the Archimidean principle.
  \end{definition}

  Note that we require the extra Archimidean assumption in the axiomatic construction. 

  \begin{example}[Ordered Cauchy-Complete Fields that are Not Archimidean]
    Provide examples of ordered, Cauchy-complete fields that are not Archimedean. 
  \end{example}

\subsection{Least Upper Bound Completeness}

  \begin{definition}[Least Upper Bound Property]
    A totally ordered algebraic field $\mathbb{F}$ (must it be a field?) is \textbf{least-upper-bound complete}, or is said to satisfy the \textbf{least upper bound (LUB) property}, if every nonempty set of $\mathbb{F}$ having an upper bound must have a least upper bound (supremum) in $\mathbb{F}$. 
  \end{definition} 

  \begin{theorem}[LUB is Equivalent to GLB]
    A set $(\mathbb{F}, \leq)$ has the least upper bound property iff it has the \textit{greatest lower bound property}, which states that every set bounded below has a greatest lower bound. 
  \end{theorem}
  \begin{proof}
    We will prove one direction since the other is the same logic. Let $S \subset X$ be a nonempty set that is bounded below by some $l \in X$. Let $L \subset X$ be the set of all lower bounds of $S$. Since $l$ exists, it is nonempty. Furthermore, $L$ is bounded above by any element of $S$. Due to LUB property $L$ has a least upper bound, call it $z = \sup{L}$. We claim that $z = \inf{S}$. 
    \begin{enumerate}
      \item $z$ is a lower bound of $S$. Assume that it is not. Then there exists $s \in S$ s.t. $s < z$. But by construction $s$ is an upper bound for $L$ and so $z$ s not the \textit{least} upper bound, a contradiction. 
      \item $z$ is a \textit{greatest} lower bound. Assume that $z$ is not. Then there exists a $z^\prime \in X$ s.t. $z < z^\prime \leq s$ for all $s \in S$. But since $z^\prime, z$ are lower bounds, this means $z, z^\prime \in L$ by definition and $z < z^\prime$ contradicts the fact that $z$ is an upper bound of $L$. 
    \end{enumerate}
    We are done. 
  \end{proof}

  $\mathbb{Q}$ does not satisfy the least upper bound property, but proving this can be tricky for the first time. We state this as a lemma. 

  \begin{theorem}[Rationals Doesn't Satisfy LUB Property]
    $\mathbb{Q}$ does not satisfy the LUB property. 
  \end{theorem}
  \begin{proof}
    Assume it does, and let us denote 
    \begin{equation}
      p \coloneqq \sup \{x \mid \mathbb{Q} \mid x^2 < 2\} \in \mathbb{Q}
    \end{equation}
    The key here is to find another rational that we can always ``squeeze'' in between $p$ and $2$. This can be done with the Archimidean principle, which is already satsified in $\mathbb{Q}$. Since we have \href{thm:sqrt2-irrational}{proved that there exists no rational that squares to $2$}, we only need to consider the two cases. 
    \begin{enumerate}
      \item $p^2 < 2$. Take $\epsilon \in \mathbb{Q}$ so small that 
      \begin{equation}
        p^2 + 2 p \epsilon + \epsilon^2 = (p + \epsilon)^2 < 2
      \end{equation}
      To show complete steps, we can see that by density of reals, there exists some rational $r$ s.t. $0 < r < 2 - p^2$. Therefore, we can invoke Archimidean principle to find a $n \in \mathbb{N}$ s.t. $2 p/n < r$. Therefore, $p$ is not an upper bound, so this cannot be true. 

      \item $p^2 > 2$. We can again take an $\epsilon \in \mathbb{Q}$ so small that 
      \begin{equation}
        p^2 - 2 p \epsilon + \epsilon^2 = (p - \epsilon)^2 > 2
      \end{equation}
      Therefore, $p$ is not least, so this cannot be true. 
    \end{enumerate}
  \end{proof}

  \begin{definition}[Axiomatic Construction of Reals with LUB Property]
    $\mathbb{R}_I^\prime$ is a totally ordered field that satisfies the least upper bound property.
  \end{definition}

  Note that we don't need to explicitly assume Archimidean principle here. The LUB property is strong enough that it implies Archimidean!

  \begin{theorem}[LUB Property Implies Archimidean]
    $\mathbb{R}_I^\prime$ is Archimidean. 
  \end{theorem}

  Now let's see how our previous constructions of the reals relate to the LUB property. 

  \begin{theorem}[Dedekind Completed Reals Satisfies LUB Property]
    $\mathbb{R}_D$ satisfies the least upper bound property. 
  \end{theorem}
  \begin{proof}
    Let $\mathcal{A}$ be a nonempty subset of $\mathbb{R}_D$ bounded from above by $T$. Then, $\forall A \in \mathcal{A}$, $A \coloneqq (A, A^c)$ is a Dedekind cut, and we can define
    \begin{equation}
      (B, B^\prime) \coloneqq \bigg( \bigcup_{A \in \mathcal{A}} A, \bigcap_{A \in \mathcal{A}} A^c \bigg)
    \end{equation} 
    We claim that this is a Dedekind cut. 
    \begin{enumerate}
      \item First, it is nonempty set $\mathcal{A} \neq \emptyset$ and so for each $A \in \mathcal{A}$, $A \subset \mathbb{Q}$ is nonempty. It is also not all of $\mathbb{Q}$ since $T$ is an upper bound of $\mathcal{A}$, and so $T \geq a \; \forall a \in A \; \forall A \in \mathcal{A}$, which implies that 
      \begin{equation}
        T \not\in \bigcup_{A \in \mathcal{A}} A
      \end{equation}

      \item Now let $x \in B, y \in B^\prime$. Then, $x \in A_0$ for some $A_0 \in \mathcal{A}$, and since $y$ is in the intersection of all the corresponding $A^c$, it must be in the corresponding $A_0^c$. Therefore we invoke the Dedekind cut property of $(A_0, A_0^c)$ and see $x < y$. 

      \item Finally, for the sake of contradiction, let $m \in \mathbb{Q}$ be the maximum of $B$. Then, $m \in A^\ast$ for some $A^\ast \in \mathcal{A}$. But $m$ is an upper bound for the whole $B$, so this means that $m = \max\{A^\ast\}$, which contradicts the fact that lower cut cannot have a rational maximum. 
    \end{enumerate}
    Now we claim that $B$ is the supremum. It is an upper bound since 
    \begin{equation}
      A \subset B = \bigcup_{A \in \mathcal{A}} A \quad \forall A \in \mathcal{A}
    \end{equation}
    To prove least, we should see that if $(S, S^c)$ is another upper bound of $\mathcal{A}$, then $A \subset S$ for all $A \in \mathcal{A} \implies B = \bigcup_{A \in \mathcal{A}} A \subset S$, which establishes that $B$ is least. 
  \end{proof}

\subsection{Nested Intervals Completeness}

  The next flavor we present is nested-intervals completeness.  This is the least popular way to construct the reals, and it is used more as a post-hoc tool to analyze the reals after you construct it using either of the two previous methods. 

  \begin{definition}[Nested Interval Completeness]
    Let $\mathbb{F}$ be a totally ordered algebraic field. Let $I_n= [a_n, b_n]$ ($a_n < b_n$) be a sequence of decreasing nested intervals that are 
    \begin{enumerate}
      \item closed, 
      \item bounded, 
      \item nested, $I_1 \supset I_2 \supset I_3 \supset \ldots$ 
      \item and decreasing to $0$ in the sense that $\lim_{n \to \infty} b_n - a_n = 0$. 
    \end{enumerate}
    $\mathbb{F}$ is \textbf{nested-interval complete} if the intersection of all of these intervals $I_n$ contains exactly one point. 
    \begin{equation}
      \bigcup_{n=1}^\infty I_n \in \mathbb{F}
    \end{equation}
  \end{definition}

  Note that defining nested intervals requires only an ordered field. One may look at this and try to ask if this is a specific instance of the following conjecture: The intersection of a nested sequence of nonempty closed sets in a topological space has exactly 1 point. This claim may not even make sense, actually. If we define nested in terms of proper subsets, then for a finite topological space a sequence cannot exist since we will run out of open sets and so this claim is vacuously true and false. If we allow $S_n = S_{n+1}$ then we can just select $X \supset X \supset \ldots$, which is obviously not true. However, a slightly weaker claim is that every proper nested non-empty closed sets has a non-empty intersection is a consequence of compactness. 

  \begin{theorem}[Rationals are Not Nested-Interval Complete]
    $\mathbb{Q}$ is not nested-interval complete. 
  \end{theorem}
  \begin{proof}
    Consider the intervals $[a_i, b_i]$ where $a_i$ is the largest number $x$ up to the $n$th decimal expansion such that $x^2$ does not exceed $2$, and $b_i$ is the smallest number $x$ up to the $n$th decimal expansion such that $x^2$ is not smaller than $2$. The first few terms are 
    \begin{equation}
      [1.4, 1.5], [1.41, 1.42], [1.414, 1.415], \ldots
    \end{equation}
  \end{proof}

  \begin{theorem}[Cantor's Intersection Theorem]
    $\mathbb{R}$ is nested-interval complete. 
  \end{theorem}
  \begin{proof}
    We prove this by first proving the claim that given nested, closed, and bounded sets $I_n$ (not even necessarily intervals), then 
    \begin{equation}
      \bigcap_{n=1}^\infty I_n \neq \emptyset
    \end{equation}
    Suppose this is not true. For every $x \in \mathbb{R}$, there exists a $n_x$ s.t. $x \not\in I_n$ (and all later $I_m$ for $m > n$). Let $O_n = I_n^c$ open sets. Then, $\mathbb{R} \subset \cup_n O_n$ In particular, $I_1 \subset \cup_n O_n$. But $I_1$ is closed and bounded. So we can extract a finite subcover. $O_{n_1}, O_{n_2}, \ldots, O_{n_m}$ (ordered $n_1 < n_2 < \ldots< n_m$). Then since $O_n$ are increasing, $I_1 \subset O_{n_m} = I_{n_m}^c$. But $I_{n_m} \subset I_1$, a contradiction. 

    Now with this, we know that because the limits of the endpoints of the intervals go to $0$, then there cannot be more than 2 points in the intersection. Thus there must be 1 unique point. 
  \end{proof} 

  \begin{definition}[Axiomatic Construction of Reals with Nested Intervals]
    $\mathbb{R}_I$ is a totally ordered field that 
    \begin{enumerate}
      \item satisfies the nested intervals completeness, and 
      \item satisfies the Archimidean principle.
    \end{enumerate}
  \end{definition}

\subsection{Properties of the Real Line}  

  Perfect, now all that remains is to unite the two constructions of the reals. 

  \begin{theorem}[Dedekind and Cauchy Complete Reals are Isomorphic]
    Given that $\mathbb{R}_D$ is the Dedekind-completed version of the rationals and $\mathbb{R}_C$ is the Cauchy-completed version of the rationals, we claim that the two are isomorphic as ordered fields. 
    \begin{equation}
      \mathbb{R}_D \simeq \mathbb{R}_C
    \end{equation}
  \end{theorem}
  \begin{proof}
    
  \end{proof}

  Therefore, it doesn't really matter which one we talk about, and we can refer to \textit{the} real numbers as a single set. Great! Now we can finally feel satisfied about defining metrics, norms, and inner products as mappings to the codomain $\mathbb{R}$. 

  \begin{definition}[Reals (as Construction from Rationals)]
    The \textbf{reals} $\mathbb{R}$ is the totally ordered complete Archimedean field constructed as the completion\footnote{either one} of $\mathbb{Q}$. 
  \end{definition}

  So far, we have taken the completion of the rationals as our main mode of construction. However, we can take an axiomatic approach, and it turns out that there is only one set, up to isomorphism, that satisfies all these properties. 

  \begin{theorem}[Axiomatic Definition of Reals]
    The \textbf{real numbers}, denoted $\mathbb{R}$, is any totally ordered complete Archimedean field. $\mathbb{R}$ is unique up to field isomorphism. That is, if two individuals construct two ordered complete Archimedean fields $\mathbb{R}_A$ and $\mathbb{R}_B$, then 
    \begin{equation}
      \mathbb{R}_A \simeq \mathbb{R}_B
    \end{equation}
  \end{theorem} 
  \begin{proof}
    The proof is actually much longer than I expected, so I draw a general outline.\footnote{Followed from \href{https://math.ucr.edu/~res/math205A/uniqreals.pdf}{here}.} We want to show how to construct an isomorphism $f: \mathbb{R}_A \rightarrow \mathbb{R}_B$. 
    \begin{enumerate}
      \item Realize that there are unique embeddings of $\mathbb{N}$ in $\mathbb{R}_A$ and $\mathbb{R}_B$ that preserve the inductive principle, the order, closure of addition, and closure of multiplication, the additive identity, and the multiplicative identity. Call these ordered doubly-monoid (since it's a monoid w.r.t. $+$ and $\times$) homomorphisms $\iota_A, \iota_B$. 
      \item Construct an isomorphism $f_1: \iota_A(\mathbb{N}) \rightarrow \iota_B(\mathbb{N})$ that preserves the inductive principle, order, addition, and multiplication. This is easy to do by just constructing $f_1 = \iota_B \circ \iota_A^{-1}$. 
      \item Extend $f_1$ to the ordered ring isomorphism $f_2$ by explicitly defining what it means to map additive inverses, i.e. negative numbers. 
      \item Extend $f_2$ to the ordered field isomorphism $f_3$ by explicitly defining what it means to map multiplicative inverses, i.e. reciprocals. 
      \item Extend $f_3$ to the ordered field isomorphism on the entire domain $\mathbb{R}_A$ and codomain $\mathbb{R}_B$. There is no additional operations that we need to support, but we should explicitly show that this is both injective and surjective, which completes our proof. 
    \end{enumerate}
  \end{proof}

  It seems that the real numbers is \textit{any} set that satisfies the definition above. Therefore, a line $\mathbb{L}$ with $+$ associated with the translation of $\mathbb{L}$ along itself and $\cdot$ associated with the "stretching/compressing" of the line around the additive origin $0$ is a valid representation of the reals. $\mathbb{R}$ can also be represented as an uncountable list of numbers with possibly infinite decimal points, known as the decimal number system. 
  \begin{equation}
    \ldots, -2.583\ldots, \ldots , 0, \ldots, 1.2343\ldots, \ldots, \sqrt{2}, \ldots
  \end{equation}

  The first property we should know is that the reals are uncountable. 

  \begin{theorem}[Cantor's Diagonalization]
    $\mathbb{R}$ is uncountable.
  \end{theorem}
  \begin{proof}
    We proceed by contradiction. Suppose the real numbers are countable. Then there exists a bijection $f: \mathbb{N} \to \mathbb{R}$. This means we can list all real numbers in $[0,1]$ as an infinite sequence.\footnote{This must be explicitly proven, but we can take the set of all Cauchy sequences of rationals in their decimal expansion and construct the reals this way.}
    
    \begin{align}
      f(1) &= 0.a_{11}a_{12}a_{13}\dots \\
      f(2) &= 0.a_{21}a_{22}a_{23}\dots \\
      f(3) &= 0.a_{31}a_{32}a_{33}\dots \\
      &\vdots
    \end{align}
    
    where each $a_{ij}$ is a digit between 0 and 9.
    
    Now construct a new real number $r = 0.r_1r_2r_3\dots$ where:
    \begin{equation}
      r_n = \begin{cases}
        1 & \text{if } a_{nn} \neq 1 \\
        2 & \text{if } a_{nn} = 1
      \end{cases}
    \end{equation}
    This number $r$ is different from $f(n)$ for every $n \in \mathbb{N}$, since $r$ differs from $f(n)$ in the $n$th decimal place. Therefore $r \in [0,1]$ but $r \notin \text{range}(f)$, contradicting that $f$ is surjective. Thus our assumption that the real numbers are countable must be false.
  \end{proof}

  With this, we can add the inner product, metric, and topology. 

\subsection{Exponentials, Roots, and Logarithms} 

  Now we will focus on some other operations that become well-defined in the reals. We know that $x^{n}$ for $n \in \mathbb{N}$ denotes repeated multiplication and $x^{-1}$ denotes the multiplicative inverse. We need to build up on this notation. As a general outline, we will show that $x^{-n}$ is well defined, then $x^q, q \in \mathbb{Q}$ is well-defined, and finally $x^r, r \in \mathbb{R}$ is well-defined. For the naturals, we have defined $x^n$ as the repeated multiplication of $n$. It is trivial that the canonical injection $\iota_0: \mathbb{N} \rightarrow \mathbb{R}$ commutes with the exponential map of naturals. We prove that $\iota_1: \mathbb{Z} \rightarrow \mathbb{R}$ also commutes. 

  \begin{lemma}[Integer Exponents]
    We have 
    \begin{enumerate}
      \item For $x_1, \ldots, x_n \in \mathbb{R}$, $(x_1 \ldots x_n)^{-1} = x_n^{-1} \ldots x_1^{-1}$. 
      \item For $x \in \mathbb{R}$, $x > 0$, $(x^n)^{-1} = (x^{-1})^n$. This value is denoted $x^{-n}$. 
      \item For $x \in \mathbb{R}$ and $w, z \in \mathbb{Z}$, $x^{w + z} = x^w x^z$. 
      \item For $w, z \in \mathbb{Z}$, $x^{wz} = (x^z)^w = (x^w)^z$. 
    \end{enumerate}
  \end{lemma}
  \begin{proof}
    Listed. 
    \begin{enumerate}
      \item The proof is trivial, but for $n = 2$ and $x_1 = x, x_2 = y$, we see that by associativity, $(x^{-1} x^{-1}) (x y) = y^{-1} (x^{-1} x) y = y^{-1} y = 1$ and we know inverses are unique. 
      \item Set $x_i = x$ using (1). 
      \item If $w, z > 0$ this is trivial by the associative property. If either or both are negative, say $w < 0 < z$, then we set $w^\prime = -w > 0$ and using (2) we know that 
      \begin{equation}
        x^{w} x^{z} = (x^{-1})^{w^\prime} x^z = x^{-w^\prime + z} = x^{w + z}
      \end{equation}
      by associativity in the second last equality. 
    \end{enumerate}
  \end{proof}

  Therefore, we have successfully defined $x^z$ for all $z \in \mathbb{Z}$, and if $z$ is negative, we're allowed to ``swap'' the $-1$ and $|z|$ in the exponents. Now we want to extend this into rational exponents, first by proving the existence and uniqueness of $n$th roots for any real. The proof is a little involved, but the general idea is that we want to use the LUB property to define the $n$th root as the supremum of a set.  

  \begin{theorem}[Existence of Nth Roots]
    For any real $x > 0$ and every $n \in \mathbb{N}$ there is one and only one positive real $y \in \mathbb{R}$ s.t. $y^n = x$. This is denoted $x^{1/n}$. 
  \end{theorem}
  \begin{proof}
    Let $E$ be the set consisting of all reals $t \in \mathbb{R}$ s.t. $t^n < x$. We show that 
    \begin{enumerate}
      \item it is nonempty. Consider $t = x/(1+x)$. Then $0 \leq t < 1 \implies t^n \leq t < x$. Thus $t \in E$ and $E$ is nonempty. 
      \item it is bounded. Consider any number $s = 1 + x$. Then $s^n \geq s > x$, so $s \not\in E$, and $s = 1 + x$ is an upper bound of $E$. 
    \end{enumerate}
    Therefore, $E$ is a nonempty set that is upper bounded, so it has a least upper bound, called $y = \sup{E}$. We claim that $y^n = x$, proving by contradiction. For both cases, we use the fact that the identity $b^n - a^n = (b - a) (b^{n-1} + b^{n-2} a + \ldots + a^{n-1})$ gives the inequality 
    \begin{equation}
      b^n - a^n < (b-a) n b^{n-1} \text{ for } 0 < a < b
    \end{equation}
    \begin{enumerate}
      \item Assume $y^n < x$. Then we choose a fixed $0 < h < 1$ s.t. 
      \begin{equation}
        h < \frac{x - y^n}{n(y + 1)^{n-1}}
      \end{equation}
      Then by putting $a = y, b = y + h$, we have 
      \begin{equation}
        (y + h)^n - y^n < hn (y + h)^{n-1} < hn (y + 1)^{n-1} < x - y^n 
      \end{equation}
      and thus $y^n < (y + h)^n < x$. This means that $y + h \in E$, and so $y$ is not an upper bound. 

      \item Assume $y^n > x$. Then we set a fixed number 
      \begin{equation}
        k = \frac{y^n - x}{n y^{n-1}} 
      \end{equation}
      Then $0 < k < y$. If we take any $t \in \mathbb{R}$ s.t. $t \geq y - k$, this implies that $t^n \geq (y -k)^n \implies -t^n \geq -(y - k)^n$, and so 
      \begin{equation}
        y^n - t^n \leq y^n - (y - k)^n < k ny^{n-1} = y^n - x
      \end{equation}
      Thus $t^n > x$ and $t \not\in E$. So it must be the case that $t < y - k$, and so $y - k$ is an upper bound of $E$, contradicting that $y$ is least. 
    \end{enumerate}
  \end{proof}

  At this point, rooting has been introduced as sort of an independent map from exponentiation. We show that they have the nice property of commuting. 

  \begin{lemma}[Rooting and Exponentiation Commute]
    For $p \in \mathbb{Z}, q \in \mathbb{N}$ and $x \in \mathbb{R}$ with $x > 0$, we have 
    \begin{equation}
      (x^{p})^{1/q} = (x^{1/q})^p
    \end{equation}
  \end{lemma}
  \begin{proof}
    If $p > 0$, then let $r = (x^p)^{1/q}$. By definition $r^q = x^p$. Let $s = x^{1/q}$ By definition $s^q = x$. Therefore $r^q = (s^q)^p = s^{qp}$ from the lemma on integer exponents. But since roots are well-defined and unique 
    \begin{equation}
      r = (r^q)^{1/q} = (s^{qp})^{1/q} = s^p \implies (x^p)^{1/q} = (x^{1/q})^p
    \end{equation}
    If $p = 0$, this is trivially $0$, and if $p < 0$ the by the same logic with $p = -p^\prime$ for $p^\prime > 0$ and $y = x^{-1} > 0$. we know 
    \begin{align}
      (x^p)^{1/q} = \big( (y^{-1})^{-p^\prime} \big)^{1/q} = (y^{-(-p^\prime)})^{1/q} & = (y^{p^\prime})^{1/q} \\ 
                         & = (y^{1/q})^{p^\prime} = ((x^{-1})^{1/q})^{p^\prime} = (x^{1/q})^{-p^\prime} = (x^{1/q})^p
    \end{align}
  \end{proof}

  \begin{theorem}[Rational Exponential Function]
    Given $m, p \in \mathbb{Z}$ and $n, q \in \mathbb{N}$, prove that 
    \begin{equation}
      (b^m)^{1/n} = (b^p){1/q}
    \end{equation}
    Hence it makes sens to define $b^r = (b^m)^{1/n}$, since every element of the equivalence class $r$ of each rational number maps to the same value. 
  \end{theorem} 
  \begin{proof}
    Since $m/n = p/q \implies mq = np$, 
    \begin{align}
      b^{mq} = b^{np} & \implies (b^m)^q = (b^p)^n \\
                      & \implies b^m = ((b^m)^q)^{1/q} = ((b^p)^n)^{1/q} \\
                      & \implies b^m = ((b^p)^{1/q})^n \\
                      & \implies (b^m)^{1/n} = (b^p)^{1/q}
    \end{align}
    Therefore we can define for any $r \in \mathbb{Q}$ 
    \begin{equation}
      x^r = x^{m/n} = (x^{m})^{1/n} = (x^{1/n})^m
    \end{equation}
    where the final equality holds from the commutativity of rooting and exponentiation. 
  \end{proof}

  It turns out that this is a homomorphism. 

  \begin{corollary}[Rational Exponential Function is a Homomorphism]
    The rational exponential function is a homomorphism. That is, given $r, s \in \mathbb{Q}$ and $x \in \mathbb{R}$, 
    \begin{equation}
      x^{r + s} = x^r \cdot x^s
    \end{equation}
  \end{corollary}
  \begin{proof}
    Let $r = m/n, s = p/q$. Then 
    \begin{align}
      x^{r+s} = x^{m/n + p/q} & = x^{\frac{mq + np}{nq}} && \tag{addition on $\mathbb{Q}$}\\
                              & = (x^{mq + np})^{1/nq} && \tag{exp and roots commute}\\
                              & = (x^{mq} + x^{np})^{1/nq} && \tag{int exp lemma}\\
                              & = (x^{mq})^{1/nq} (x^{np})^{1/nq} && \tag{int exp lemma}\\
                              & = x^{mq/nq} x^{np/nq} && \tag{exp and roots commute} \\
                              & = x^{m/n} x^{p/q} && \tag{relation from $\mathbb{Q}$}
    \end{align}
  \end{proof}

  With rational exponents defined, we can use the least upper bound property to define a consistent extension of a real exponent. 
  
  \begin{lemma} 
    If $r \in \mathbb{Q}$ with $r \geq 0$, then for $x \in \mathbb{R}$, $x > 1$, $1 \leq b^r$. 
  \end{lemma}
  \begin{proof}
    Let $r = m/n$. Then $x^r = x^{m/n} = (x^m)^{1/n}$. Since $1 < x$, and $m \geq 0$, we have 
    \begin{equation}
      1 \leq x \leq x^2 \leq \ldots \leq x^m \implies 1 \leq b^m
    \end{equation}
    Now set $y = x^{m/n}$ and assume that $y < 1$. Then 
    \begin{equation}
      x^m = y^n < y^{n-1} < \ldots < y < 1
    \end{equation}
    and so $x^m < 1$, which is a contradiction. So it must be the case that $y > 1$. 
  \end{proof}

  \begin{lemma}[Monotonicity of Rational Exponents]
    If $x, y \in \mathbb{R}$, then for any rational $r \in \mathbb{Q}$ with $r < x + y$, there exists a $p, q \in \mathbb{Q}$ s.t. $p < x, q < y$ and $p + q = r$. The converse is true as well. 
  \end{lemma}
  \begin{proof}
    $r < x + y \implies r - y < x$. By density of $\mathbb{Q}$ in $\mathbb{R}$, we can choose $r - y < p < x$. Then $-r + y > -p > x \implies r - r + y > r - p > r - x \implies y > r - p > r - x$, and we set $q = r - p$. We are done. The converse is trivial since given $p, q \in \mathbb{Q}$ with $p < x, q < y$, then by the ordered field properties $p + q < x + y$. 
  \end{proof}

  \begin{corollary}[Real Exponential Function]
    Given $x\in \mathbb{R}$, we define 
    \begin{equation}
      B(x) \coloneqq \{ x^q \in \mathbb{R} \mid q \in \mathbb{Q}, \; q \leq x \}
    \end{equation}
    We claim that given $r \in \mathbb{R}$, 
    \begin{equation}
      x^r \coloneqq \sup B(r)
    \end{equation}
    is well-defined and is a homomorphism extension of the rational exponential function. That is, 
    \begin{equation}
      \sup{B(x + y)} = \sup{B(x)} \cdot \sup{B(y)}
    \end{equation}
  \end{corollary}
  \begin{proof}
    To show that $x^r \coloneqq \sup B(r)$ where $B(r) = \{x^t \in \mathbb{R} \mid t \in \mathbb{Q}, t \leq r \}$, 
    \begin{enumerate}
      \item We show it's an upper bound. Assume it wasn't. Then $x^r < x^t$ for some $t \in \mathbb{Q}$ satisfying $t \leq r$. But $t \leq r \implies 0 \leq r - t$, and by the previous lemma, $1 \leq x^{r - t}$. So $1 \leq x^{r-t} = x^{r} x^{-t} = x^r (x^t)^{-1} \implies x^t \leq x^r$, which is a contradiction. 
      \item We show that it is least. Assume that it is not. Then $\exists r^\prime \in \mathbb{Q}$ s.t. $x^t \leq x^{r^\prime}$ and $r^\prime < r$. Now let $s \in \mathbb{Q}$ be an element between $r^\prime$ and $r$, which is guaranteed to exist due to density of rationals in reals. But $s < r$, so by definition $x^s \in B(r)$, but 
      \begin{align}
        0 < s - r^\prime & \implies 1 < b^{s - r^\prime} \\
                         & \implies b^{r^\prime} (b^{r^\prime})^{-1} < b^s (b^{-r^\prime}) \\
                         & \implies 1 < b^s (b^{r^\prime})^{-1} \\
                         & \implies b^{r^\prime} < b^s
      \end{align}
      and so $b^{r^\prime}$ is not an upper bound for $B(r)$. By contradiction, $b^r$ is least. 
    \end{enumerate}
    Since this is defined, the analogous definition for real numbers is consistent with that of hte rationals, and it is upper bounded by the Archimedean principle, so such a supremum must exist. Note that $t$ is rational. For the second part, from the previous lemma and the homomorphism properties of the rational exponent, 
    \begin{align}
      B(x + y) = B^\prime (x + y) & \coloneqq \{b^{p+q} \in \mathbb{R} \mid p, q \in \mathbb{Q}, p \leq x, q \leq y\} \\
                                  & = \{b^p b^q \in \mathbb{R} \mid p, q \in \mathbb{Q}, p \leq x, q \leq y\} \\
    \end{align}
    Therefore we can treat $B$ and $B^\prime$ as the same set. 
    \begin{enumerate}
      \item Prove upper bound $\sup{B(x + y)} \leq \sup{B(x)} \sup{B(y)}$. Given $\alpha \in B^\prime (x + y)$, there exists $p_\alpha, q_{\alpha} \in \mathbb{Q}$ (with $p_\alpha < x$, $q_\alpha < y$) s.t. $b^{p_{\alpha}} b^{q_{\alpha}} = \alpha$. But 
      \begin{equation}
        b^{p_{\alpha}} b^{q_{\alpha}} \leq \sup_{p_{\alpha}} \{ b^{p_{\alpha}}\} \cdot \sup_{q_{\alpha}} \{b^{q_{\alpha}}\} = \sup{B(x)} \sup{B(y)}
      \end{equation}

    \item To prove least, assume there exists $K \in \mathbb{R}$ s.t. $\sup{B^\prime(x + y)} \leq K < \sup{B(x)} \sup{B(y)}$. Then, since the image of $b^x$ is always positive, we assume $0 < K$. We bound its factors as so: $K < \sup{B(x)} \sup{B(y)} \implies K/\sup{B(x)} < \sup{B(y)}$. By density of the rationals, there exists a $\beta \in \mathbb{Q}$, s.t. 
    \begin{equation}
      \frac{K}{\sup{B(x)}} < \beta < \sup{B(y)}
    \end{equation}
    This means $K/\beta < \sup{B(x)}$ and $\beta < \sup{B(y)}$. But this means that there exists $\phi, \gamma \in B(x), B(y)$ s.t. $K/\beta < \phi, \beta < \gamma \implies K = (K/\beta) \cdot \beta < \phi \gamma \implies \phi \gamma \in B^\prime(x + y)$ by definition. So $K$ is not an upper bound. 
    \end{enumerate}
  \end{proof}

  Furthermore, this is an isomorphism, and the inverse is defined. Let's define this analytically. 

  \begin{theorem}[Logarithm]
    For $b > 1$ and $y > 0$, there is a unique real number $x$ s.t. $b^x = y$. We claim 
    \begin{equation}
      x = \sup\{ w \in \mathbb{R} b^w < y \}
    \end{equation}
    $x$ is called the \textbf{logarithm of $y$ to the base $b$}. 
  \end{theorem}
  \begin{proof}
    We use the inequality $b^n - 1 \leq n (b-1)$ for all $n \in \mathbb{N}$.\footnote{We prove by induction. For $n=1$ $b^1 - 1 \leq 1 (b-1)$. Assume that this holds for some $n$. Then $b^{n+1} - 1 = b^{n+1} - b + b - 1 = b (b^n - 1) + (b-1) \geq bn (b-1) + (b-1) = (bn + 1) (b-1) \geq (n+1) (b-1)$, where the last step follows since $b \geq 1 \implies bn \geq n \implies bn + 1 \geq n + 1$. } By substituting $b = b^{1/n}$ (valid since $b > 1 \iff b^{1/n} > 1$) so $b-1 \geq n(b^{1/n} - 1)$. Now set some $t > 1$, and by Archimidean principle, we can choose some $n \in \mathbb{N}$ s.t. $n > \frac{b-1}{t-1}$. Then $n (t-1) > b-1$, and with the inequality derived we get 
    \begin{equation}
      n (t - 1) > b - 1 \geq n (b^{1/n} - 1) \implies t > b^{1/n}
    \end{equation} 
    This allows us to prove 2 things. 
    \begin{enumerate}
      \item If $w$ satisfies $b^w < y$, then $b^{w + (1/n)} < y$ for sufficiently large $n$. Setting $t = y b^{-w}$ (which is greater than $1$ since $b^w < y$) gives $y \cdot b^{-w} > b^{1/n} \implies b^w b^{1/n} < y \implies b^{w + (1/n)} < y$. 
      \item If $w$ satisfies $b^w > y$, then $b^{w - (1/n)} > y$ for sufficiently large $n$. Setting $t = b^w y^{-1}$ (which is greater than $1$ since $b^w > y$) gives $b^w y^{-1} > b^{1/n} \implies b^{w - (1/n)} > y$. 
    \end{enumerate}
    Now we can prove existence. Let $A$ the set of all $w$ s.t. $b^w < y$. We claim that $x = \sup{A}$. 
    \begin{enumerate}
      \item Assume that $b^x < y$. We know that there exists $n \in \mathbb{N}$ s.t. $b^{x + (1/n)} < y \implies x + (1/n) \in A$, contradicting that $x$ is an upper bound. 
      \item Assume that $b^x > $. We know that there exists $n \in \mathbb{N}$ s.t. $b^{x - (1/n)} > y \implies x - (1/n)$ is also an upper bound for $A$, contradicting that $x$ is least. Therefore $b^x = y$. 
    \end{enumerate}
    We now prove uniqueness. Assume that there are two such $x$'s , call them $x, x^\prime$. By total ordering and $x \neq x^\prime$, WLOG let $x > x^\prime \implies x - x^\prime > 0 \implies b^{x - x^\prime} > 1$. By density of rationals, since we can choose $r \in \mathbb{R}$ s.t. $0 < r < x - x^\prime$, we have $1 < b^r < b^{x - x^\prime}$ and so $B(r) \subset B(x - x^\prime)$. Since $1 < b^{x - x^\prime} \implies 1 \cdot b^{x^\prime} < b^{x - x^\prime} \cdot b^{x^\prime} = b^x$, we have $b^{x^\prime} < b^x$ and they cannot both by $y$. So $x = x^\prime$. 
  \end{proof}

\subsection{Extended Reals} 

  Often, we deal with numbers that are not finite, and we would like to have a system to incorporate $\pm \infty$ into the real line. Most first courses glaze over this, but it's important to see the construction as well. The problem is that we can't really add in these numbers without breaking a lot of the algebraic properties, but let's see for ourselves. It should be pretty obvious that we want (note the strict inequalities)
  \begin{equation}
    -\infty < x < +\infty \quad \forall x \in \mathbb{R}
  \end{equation} 
  To define addition, we can't make $x + \infty$ a finite number since then 
  \begin{equation}
    \infty \leq x + \infty = y 
  \end{equation}
  which is a contradiction. So $x + \infty = +\infty$. We keep doing this but the main problem comes in with trying to define $\infty - \infty$ or $\infty/\infty$, which are known as \textbf{indeterminate forms}. These are particularly bad since we cannot deduce $x = y$ from $x + \infty = y + \infty$ or from $x \cdot \infty = y \cdot \infty$. The solution to this is to \textit{simply avoid them}\footnote{as far as I know} by making these indeterminate terms undefined. 

  \begin{definition}[Extended Real Number Line]
    The \textbf{extended real number line} is the set $\mathbb{R} \cup \{\pm \infty\}$ with the following operations. 
    \begin{enumerate}
      \item \textit{Order}. $-\infty \leq x$ and $x \leq +\infty$ for all $x \in \overline{\mathbb{R}}$. 
      \item \textit{Addition}. 
        \begin{align}
          \forall x \in \mathbb{R}, & x + \infty = \infty + x = +\infty \\
          \forall x \in \mathbb{R}, & x - \infty = \infty - x = -\infty \\ 
          & + \infty + \infty = +\infty \\
          & - \infty - \infty = -\infty \\ 
          & +\infty - \infty, -\infty + \infty \text{ are undefined}
        \end{align}
      \item \textit{Multiplication}.\footnote{The fact that $0 \cdot \infty = 0$ might sound odd. Look at the extension into hyperreals later.} 
      \begin{align} 
        \forall x \in \mathbb{R} \setminus \{0\}, & x \cdot +\infty = +\infty \cdot x = \begin{cases} +\infty \text{ if } x > 0 \\ -\infty \text{ if } x < 0 \end{cases} \\
        \forall x \in \mathbb{R} \setminus \{0\}, & x \cdot -\infty = -\infty \cdot x = \begin{cases} -\infty \text{ if } x < 0 \\ -\infty \text{ if } x > 0 \end{cases} \\
        & 0 \cdot +\infty = +\infty \cdot 0 = 0 \\
        & 0 \cdot -\infty = -\infty \cdot 0 = 0 \\
        & +\infty \cdot +\infty = -\infty \cdot -\infty = +\infty \\ 
        & +\infty \cdot -\infty = -\infty \cdot +\infty = +\infty \\ 
      \end{align}
    \end{enumerate}
  \end{definition} 

  It turns out that this is still Dedekind-complete, which is nice. Unfortunately this is not even a field since the multiplicative inverse of $\pm \infty$ is not defined. Furthermore, we lose the Archimedean property. 

  The general rule of thumb is that if one wishes to use cancellation, this is only safe if one can guarantee that the numbers we work with are all finite. If we must work with infinity, another way is to work with the nonnegative reals. 

  \begin{definition}[Extended Real Number Line]
    The \textbf{extended nonnegative reals} is the set $\mathbb{R}_{\geq 0} \cup \{+\infty\}$ with the following operations. 
    \begin{enumerate}
      \item \textit{Order}. $x \leq +\infty$ for all $x \in \overline{\mathbb{R}}$. 
      \item \textit{Addition}. 
      \begin{align}
        \forall x \in [0, +\infty], +\infty + x = x + \infty = +\infty 
      \end{align}
    \item \textit{Multiplication}.
      \begin{align}
        \forall x \in (0, +\infty], & +\infty \cdot x = x \cdot +\infty = +\infty \\ 
                                    & 0 \cdot +\infty = +\infty \cdot 0 = 0 
      \end{align}
    \end{enumerate}
  \end{definition} 

  There is a tradeoff here: we can work with infinity, or negative numbers, but not both. Also, note that if we define $\infty \cdot 0 = 0$, the multiplication becomes \textit{upward continuous}, but not \textit{downwards continuous}. This leads to an asymmetry when defining integrals, but in univariate analysis we will only work with bounded functions, and this will not hinder us until measure theory. 

\subsection{Hyperreals}

  The loss of the field property of the extended reals is quite bad, and we might want to recover this. Therefore, we can add more elements that serve to be the multiplicative inverse of infinity. We call these inverses \textit{infinitesimals} and the new set the \textit{hyperreal numbers}. 

  \begin{theorem}[Hyperreals]
    The \textbf{hyperreals} 
  \end{theorem}

  In fact, when Newton first invented calculus, the hyperreals were what he worked with, and you can surprisingly build a good chunk of calculus with this. Even though this is a dead topic at this point, a lot of modern notation is based off of this number system, so it's good to see how it works. For example, when we write the integral 
  \begin{equation}
    \int f(x) \,dx
  \end{equation} 
  we are saying that we are taking the uncountable sum of the terms $f(x) \,dx$, the multiplication of the real number $f(x)$ and the infinitesimal number $dx$ living in the hyperreals. Unfortunately, we cannot fully construct a rigorous theory of calculus with only infinitesimals. However, in practice (especially physics) people tend to manipulate and do algebra with infinitesimals, so having a good foundation on what you can and cannot do with them is practical. While the focus won't be on \textit{smooth infinitesimal analysis (SIA)}, I will include some alternate constructions later purely with infinitesimals. 

\subsection{Some Algebraic Inequalities}

  We also introduce various inequalities that may be useful for producing future results. The following lemmas can be proved with elementary algebra on the field of reals. 

  \begin{lemma}[Bernoulli's Inequality]
    For any $x \in \mathbb{R}$ and $n \in \mathbb{N}$, we have 
    \begin{equation}
      (1 + x)^n \geq 1 + nx
    \end{equation}
  \end{lemma}

  \begin{lemma}[Young's Inequalities]
    If $a>0$ and $b>0$, and the numbers $p$ and $p$ are such that $p \neq 0, 1$ and $q \neq 0, 1$, and $\frac{1}{p} + \frac{1}{q} = 1$, then 
    \begin{align*}
        a^{\frac{1}{p}} b^{\frac{1}{q}} \leq \frac{1}{p} a + \frac{1}{q} b \text{  if } p > 1 \\
        a^{\frac{1}{p}} b^{\frac{1}{q}} \geq \frac{1}{p} a + \frac{1}{q} b \text{  if } p < 1
    \end{align*}
    and equality holds in both cases if and only if $a = b$. 
  \end{lemma}

  \begin{lemma}[Holder's Inequalities]
    Let $x_i \geq 0, y_i \geq 0$ for $i = 1, 2, ..., n$, and let $\frac{1}{p} + \frac{1}{q} = 1$. Then, 
    \begin{align*}
        &\sum_{i=1}^n x_i y_i \leq \bigg( \sum_{i=1} x_i^p \bigg)^{\frac{1}{p}} \, \bigg( \sum_{i=1} y_i^q \bigg)^{\frac{1}{q}} \text{  for } p > 1 \\
        &\sum_{i=1}^n x_i y_i \geq \bigg( \sum_{i=1} x_i^p \bigg)^{\frac{1}{p}} \, \bigg( \sum_{i=1} y_i^q \bigg)^{\frac{1}{q}} \text{  for } p < 1, p \neq 0
    \end{align*}
  \end{lemma}

  \begin{lemma}[Minkowski's Inequalities]
    Let $x_i \geq 0, y_i \geq 0$ for $i = 1, 2, ... ,n$. Then, 
    \begin{align*}
        \bigg( \sum_{i=1}^n (x_i + y_i)^p \bigg)^{\frac{1}{p}} & \leq \bigg( \sum_{i=1}^n x_i^p \bigg)^\frac{1}{p} + \bigg( \sum_{i=1}^n y_i^p \bigg)^{\frac{1}{p}} \text{  when } p > 1 \\
        \bigg( \sum_{i=1}^n (x_i + y_i)^p \bigg)^{\frac{1}{p}} & \geq \bigg( \sum_{i=1}^n x_i^p \bigg)^\frac{1}{p} + \bigg( \sum_{i=1}^n y_i^p \bigg)^{\frac{1}{p}} \text{  when } p < 1, p \neq 0
    \end{align*}
  \end{lemma}


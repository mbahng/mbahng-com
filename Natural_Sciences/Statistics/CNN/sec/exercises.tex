\section{Exercises}

  \begin{example}[Tarokh, Duke ECE685]
    Consider an RGB image $X = [X_0, X_1, X_2]$ with three channels, and given as follows 

    \begin{equation}
      X_0 = \begin{bmatrix} 2 & 1 & 0 & 0 \\ 0 & 0 & 2 & 1 \\ 0 & 2 & 0 & 1 \\ 2 & 1 & 0 & 1 \end{bmatrix}, \;\; 
      X_1 = \begin{bmatrix} 2 & 2 & 0 & 0 \\ 0 & 0 & 2 & 1 \\ 0 & 0 & 2 & 0 \\ 0 & 1 & 0 & 1 \end{bmatrix}, \;\; 
      X_2 = \begin{bmatrix} 2 & 1 & 0 & 0 \\ 0 & 0 & 2 & 1 \\ 2 & 0 & 0 & 0 \\ 0 & 1 & 0 & 1 \end{bmatrix}
      \label{eq:tarokh_conv_exercise}
    \end{equation}
    
    The image is passed through the convolutional filter with the weights $W = [W_0, W_1, W_2] \in \mathbb{R}^{3 \times 3 \times 3}$ and step size $1$, and given as follows 

    \begin{equation}
      W_0 = \begin{bmatrix} 1 & 0 & 0 \\ 0 & -2 & 0 \\ 0 & 0 & -1 \end{bmatrix}, \;\; 
      W_1 = \begin{bmatrix} 1 & 2 & 0 \\ 2 & 0 & - 1 \\ 0 & -1 & 1 \end{bmatrix}, \;\; 
      W_2 = \begin{bmatrix} 0 & 0 & -2 \\ 0 & 1 & 2 \\ -2 & 2 & 0 \end{bmatrix} 
      \label{eq:tarokh_conv_exercise2}
    \end{equation}   

    The output of the convolutional filter is given as 
    \begin{equation}
      Y = \mathrm{ReLU} \bigg( \sum_{i=0}^2 (X_i^\prime \ast W_i) + 2 \cdot 1_{4 \times 4}\bigg)
    \end{equation}
    where $Y$ is the output image, $X^\prime$ is the input image after applying $0$ padding around the edges, and $\ast$ is the discrete convolution operator. Compute the output $Y$, and then apply max pooling on nonoverlapping $2 \times 2$ submatrices, and then apply average pooling on non-overlapping $2 \times 2$ submatrices. 
  \end{example}
  \begin{solution}
    We can compute 
    \begin{align*} 
      X_0 \ast  W_0 & = \begin{bmatrix} -4 & -4 & -1 & 0 \\ -2 & 2 & -4 & -2 \\ -1 & -4 & -1 & 0 \\ -4 & -2 & 2 & -2 \end{bmatrix} \\
      X_1 \ast W_1 & = \begin{bmatrix} -2 & 6 & 3 & -1 \\ 4 & 6 & -1 & 4 \\ 1 & -3 & 5 & 7 \\ -1 & 0 & 5 & 2 \end{bmatrix} \\
      X_2 \ast W_2 & = \begin{bmatrix} 4 & 1 & 4 & -2 \\ 2 & 0 & 4 & 11 \\ 2 & -2 & -4 & 2 \\ 2 & 1 & 2 & 1 \end{bmatrix}
    \end{align*}
    and so we get 
    \begin{equation}
      Y = \begin{bmatrix} 0 & 5 & 8 & 0 \\ 6 & 10 & 1 & 5 \\ 4 &  0 & 2 & 11 \\ 0 & 1 & 11 & 3 \end{bmatrix}
    \end{equation}
    Maxpooling and average pooling gives us 
    \begin{equation}
      \mathrm{max}(Y) = \begin{bmatrix} 10 & 8 \\ 4 & 11 \end{bmatrix} \text{ and } \mathrm{avg}(Y) = \begin{bmatrix} 21/4 & 7/2 \\ 5/4 & 27/4 \end{bmatrix}
    \end{equation}
  \end{solution}


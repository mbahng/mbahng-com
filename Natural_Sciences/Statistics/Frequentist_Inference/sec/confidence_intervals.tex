\section{Confidence Intervals}

  Recall that the central limit theorem says that given a sequence of iid random variables $x_1, \ldots, x_n$ coming from a random variable with true mean $\mu$ and variance $\sigma^2$, the sample mean is similar to a $\mathcal{N}(\mu, \sigma^2 / n)$ random variable. That is, the sample mean converges in distribution 
  \begin{equation}
    \overline{X}_n \xrightarrow{dist} \mathcal{N} \Big( \mu, \frac{\sigma^2}{n} \Big)
  \end{equation}
  as $n \rightarrow \infty$. Another way to state it is that the normalized sample mean is similar to a standard Gaussian. 
  \begin{equation}
    \frac{\overline{x}_n - \mu}{\sigma_{\overline{x}_n}} = \frac{\overline{x}_n - \mu}{\sigma / \sqrt{n}} \xrightarrow{dist} \mathcal{N}(0, 1)
  \end{equation}
  So, given that we have enough samples, I will perfectly understand its fluctuations. Now let's introduce some definitions that will allow us to unify some ideas into simpler notation: the realized value $x$, the number of standard deviations it is away from the mean, and the probability that it takes that value (or more extreme). 

  \begin{definition}[z-score]
    Given a $\mathcal{N}(\mu, \sigma^2)$ distribution, the \textbf{z-score} of a number $x \in \mathbb{R}$ is defined to be the number of standard deviations away from the mean. 
    \begin{equation}
      z = \frac{x - \mu}{\sigma}
    \end{equation}
  \end{definition}

  \begin{definition}[Percentile]
    Given $X \sim \mathcal{N}(0, 1)$ and significance level $\alpha \in [0, 1]$, let us define $q_{\alpha} \in \mathbb{R}$ as the point where 
    \begin{equation}
      \mathbb{P}(X \geq q_{\alpha}) = \alpha
    \end{equation}
    i.e. the $100\alpha$th percentile of the standard normal. Note that given $X \sim \mathcal{N}(0, 1)$, we have 
    \begin{equation}
      \mathbb{P} (|X| > q_{\alpha/2}) = \alpha
    \end{equation}
  \end{definition}

  Now given $x_1, \ldots, x_n$ from a population $X$ with mean $\mu$ and standard deviation $\sigma$, let $\overline{x}_n$ be the sampling distribution of the mean. By virtue of the central limit theorem, we can write
  \begin{equation}
    \mathbb{P} \bigg( \bigg| \frac{\overline{X}_n - \mu}{\sigma / \sqrt{n}} \bigg| \geq q_{\alpha/2} \bigg) \approx \alpha \iff \mathbb{P} \bigg( \bigg| \frac{\overline{X}_n - \mu}{\sigma \sqrt{n}} \bigg| \leq q_{\alpha/2} \bigg) \approx 1 - \alpha
  \end{equation}
  which implies that with probability $1 - \alpha$, we have 
  \begin{equation}
    \overline{X}_n \in \bigg[ \mu - q_{\alpha/2} \frac{\sigma}{\sqrt{n}}, \mu + q_{\alpha/2} 
    \frac{\sigma}{\sqrt{n}} \bigg] \iff \mu \in \bigg[ \overline{X}_n - q_{\alpha/2} \frac{\sigma}{\sqrt{n}}, \overline{X}_n + q_{\alpha/2} \frac{\sigma}{\sqrt{n}} \bigg]
  \end{equation}
  This is how we construct a confidence interval. In other words, as $n$ becomes large (ideally at least $30$), the probability that an interval around our sample mean contains the actual mean $\mu$ can be approximated by a Gaussian. But note that CI requires to know the actual standard deviation $\sigma$. There are three ways to deal with this: 
  \begin{enumerate}
    \item This may actually be known from the start, especially if we are working with calibrated devices with standard devices that have been experimentally verified.

    \item We can simply bound $\sigma$, depending on what kind of random variable we are working with. For example, given $X \sim \mathrm{Bernoulli}(p)$, its standard deviation is bounded by $\sigma = \sqrt{p (1 - p)} \leq \frac{1}{2}$, so we can create a confidence interval that is larger than any other confidence interval we can make if we had known the true $\sigma$. 
    \begin{equation}
      p \in \bigg[ \overline{X}_n - q_{\alpha/2} \, \frac{1}{2 \sqrt{n}}, \overline{X}_n + q_{\alpha/2} \,\frac{1}{2 \sqrt{n}} \bigg]
    \end{equation}

    \item We can approximate $\sigma$ with the sample standard deviation $S$, which turns out to be an unbiased estimator. 
  \end{enumerate}

  \begin{example}[Proportion of Right-Side Kissers]
    We have observed $80$ out of $124$ right-side kisses, resulting in a sample estimate of $\widehat{p} = 0.645$. Given that we want a confidence interval of $95\%$, we want an $\alpha = 0.05$, implying a the value $q_{\alpha/2} = q_{0.025} = 1.96$. So, with probability $0.95$, we have 
    \begin{equation}
      p \in \bigg[ 0.645 - \frac{1.96}{2 \sqrt{124}}, 0.645 + \frac{1.96}{2 \sqrt{124}} \bigg] = [ 0.56, 0.73 ]
    \end{equation}
    If we had, say $3$ observations, rather than $124$, we would have a $95\%$ confidence interval of $p \in [0.10, 1.23]$, which is terrible, but in this case even CLT is not valid. 
  \end{example}

  \begin{example}[Proportion of Voters]
    Given that we sample $n = 100$ people from a city's population to ask whether they support candidate A or B, we have $54$ people who support candidate $A$, so $\widehat{p} = 0.54$. Say that we want a 95\% confidence interval, which leads to $q_{\alpha /2} = q_{0.025} = 1.96$. So, with probability $0.95$, we have 
    \begin{equation}
      p \in \bigg[ 0.54 - 1.96\,\frac{\sigma}{\sqrt{100}}, 0.54 + 1.96\,\frac{\sigma}{\sqrt{100}} \bigg]
    \end{equation}
    and by substituting $\sigma$ for $S = \sqrt{0.54 (1 - 0.54)} \approx 0.5$, we get 
    \begin{equation}
      p \in \bigg[ 0.54 - 1.96\,\frac{0.284}{\sqrt{100}}, 0.54 + 1.96\,\frac{0.284}{\sqrt{100}} \bigg] = [0.44, 0.64]
    \end{equation}
  \end{example}

  An interpretation of confidence intervals is that if you keep on sampling $\overline{x}$ or $\widehat{p}$ and construct 95\% CIs, then 95\% of the time these intervals will contain the true mean $\mu$ or proportion $p$ (or more if we had bounded the CI with a bigger interval). 

  \begin{example}
    We survey 6250 teachers to ask whether they think computers are essential for teaching. 250 were randomly selected and 142 felt that they were essential. Let's construct a 99\% confidence interval for the proportion of teachers who felt that computers were essential. We would like to construct a CI for the true $\mu = p$, and we have $\overline{x} = 142/250 = 0.568$. 
    \begin{enumerate}
      \item 99\% confidence corresponds to $\alpha = 0.01$, which corresponds to a z-score of $q_{\alpha/2} = 2.576$. 
      \item The parent distribution is $\mathrm{Bernoulli}(p)$, with $\mu = p$ and $\sigma = \sqrt{p (1 - p)}$. The sampling distribution of $\overline{x}$ has $\mu_{\overline{x}} = p$ also and $\sigma_{\overline{x}} = \sigma / \sqrt{n}$. 
      \item We need to know the details of the sampling distribution, but we don't know $\sigma$, which is needed to calculate $\sigma_{\overline{x}}$. However, we can estimate it using the sample standard deviation $S = \sqrt{0.568 (1 - 0.568)} = 0.5$. 
      \item Our sampling distribution has standard deviation $\sigma_{\overline{x}} \approx S / \sqrt{n} = 0.5 / \sqrt{250} = 0.031$, and our z-score was $2.576$, so our 99\% confidence interval is $2.576$ standard deviations from our mean. That is, with probability $0.99$, 
      \begin{equation}
        p \in \big[ 0.568 - 2.576 \cdot 0.031, 0.568 + 2.576 \cdot 0.031 \big] = \big[ 0.488144, 0.647856 \big]
      \end{equation}
    \end{enumerate}
  \end{example}

\subsection{CIs for means, proportions, and variances}

\subsection{Bootstrap confidence intervals}


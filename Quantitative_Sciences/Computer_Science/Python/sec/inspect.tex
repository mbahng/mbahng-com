\section{Inspect} 

  \texttt{inspect} is a module that allows you to get live information about live objects such as modules, classes, and functions. 

  
  \begin{definition}[\texttt{getsource}]
    The \texttt{getsource} method allows you to see the text of live objects. 

    \begin{figure}[H]
      \centering 
      \begin{lstlisting}
        >>> import inspect 
        >>> backbone_module = construct_backbone('resnet50[pretraining=inaturalist]')
        >>> model = backbone_module.embedded_model 
        >>> print(inspect.getsource(model.forward))
            def forward(self, x):
                x = self.conv1(x)
                x = self.bn1(x)
                x = self.relu(x)
                x = self.maxpool(x)

                x = self.layer1(x)
                x = self.layer2(x)
                x = self.layer3(x)
                x = self.layer4(x)

                return x

        >>> print(inspect.getsource(model.__class__))
        class ResNet_features(nn.Module):
            """
            the convolutional layers of ResNet
            the average pooling and final fully convolutional layer is removed
            """

            def __init__(self, block, layers, num_classes=1000, zero_init_residual=False):
                super(ResNet_features, self).__init__()
                ...
            ...
      \end{lstlisting}
      \caption{Say that you have some torch model that is either inaccessible or is hidden away through so many imports that you have a hard time accessing it. Rather than going through several files and having to parse which methods are relevant, is overwritten, or called, you can just inspect the methods and classes directly.}
    \end{figure}
  \end{definition}

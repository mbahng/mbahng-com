After coding in Python for about 4 to 5 years, I realized that my coding practices have not changed, and I should try to grow on them. These notes have four purposes. Learn some intermediate Python through different syntax, methods, and classes. Learn how Python and its data structures are implemented, specifically CPython (C notes are previously done). Establish best practices by going through different case studies of codebase design. Learn the APIs of some broad Python packages, mostly in the standard library that are pretty up in the dependency tree. 

All of these can be found in either: 
\begin{enumerate}
  \item The \href{https://docs.python.org/3/reference}{official Python language reference}, which describes the exact syntax and semantics of the Python language. 
  \item The \href{https://docs.python.org/3/library/index.html}{official Python standard library}, which describes the standard library (the built-in modules) that is distributed with Python. 
  \item The \href{https://peps.python.org/}{Index of Python enhacement proposals (PEP)}, which is a series of design documents providing information to the Python community. It is used to describe new features of Python and its processes of development. 
\end{enumerate}

\begin{definition}[Object]
  Every object has an \textbf{identity}, a \textbf{type}, and a \textbf{value}. 
\end{definition}

\begin{theorem}[]
  In CPython, \texttt{id(x)} is the memory address where \texttt{x} is stored. 
\end{theorem}

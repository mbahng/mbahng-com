\section{Context Free Languages}

\subsection{Context Free Grammars}

  When a person designs a programming language, they need to determine its \textit{syntax}. That is, the designer decides which strings correspond to valid programs, and which ones do not (i.e. which strings contain a syntax error). To ensure that a compiler or interpreter always halts when checking for syntax errors, language designers typically \textit{do not} use a general Turing-complete mechanism to express their syntax. Rather, they use a \textit{restricted} computational model, most often being \textit{context free grammars}. 

  Consider the function $ARITH: \Sigma^* \longrightarrow \{0,1\}$ that takes as input a string $x$ over alphabet 
  \[\Sigma = \{(, ), +, -, \times, \div, 0, 1, 2, 3, 4, 5, 6, 7, 8, 9\}\]
  and returns $1$ if and only if the string $x$ represents a valid arithmetic expression. Intuitively, we build expressions by applying an operation such as $+, -, \times, \div$ to smaller expressions or enclosing them in parentheses. More precisely, we can make the following definitions: 
  \begin{enumerate}
      \item A \textit{digit} is one of the symbols $0, 1, 2, 3, 4, 5, 6, 7, 8, 9$. 
      \item A \textit{number} is a sequence of digits (we will drop the condition that the sequence does not have a leading zero)
      \item An \textit{operation} is one of $+, -, \times, \div$. 
      \item An \textit{expression} has either the form 
      \begin{enumerate}
          \item \textit{"number"}
          \item \textit{"sub-expression1 operation sub-expression2}
          \item \textit{"(sub-expression1)"}
      \end{enumerate}
      where "sub-expression1" and "sub-expression2" are themselves expressions. Note that this is a recursive function. 
  \end{enumerate}
  A context free grammar (CFG) is a formal way of specifying such conditions, consisting of a set of ruels that tell us how to generate strings from smaller components. 

  \begin{definition}[Context Free Grammar]
  Let $\Sigma$ be some finite set. A \textbf{context free grammar (CFG) over $\Sigma$} is a triple $(V, R, s)$ such that: 
  \begin{enumerate}
      \item $V$, known as the \textit{variables}, is a set disjoint from $\Sigma$
      \item $s \in V$ is known as the \textit{initial variable}
      \item $R$ is a set of \textit{rules}. Each rule is a pair $(v, z)$ with $v \in V$ and $z \in (\Sigma \cup V)^*$. We often write the rule $(v, z)$ as 
      \[v \implies z\]
      and say that the string $z$ \textit{can be derived} from the variable $v$. 
  \end{enumerate}
  \end{definition}

  \begin{example}
  The example of well-formed arithmetic expressions can be captured formally by the following context free grammar. 
  \begin{enumerate}
      \item The alphabet $\Sigma$ is $\{(, ), +, -, \times, \div, 0, 1, 2, 3, 4, 5, 6, 7, 8, 9\}$. 
      \item The variables are $V = \{expression, number, digit, operation\}$
      \item The rules are the set $R$ containing the following 19 rules: 
      \begin{enumerate}
          \item 4 Rules: $operation \implies +$, $operation \implies -$, $operation \implies \times$, $operation \implies \div$
          \item 10 Rules: $digit \implies 0$, $digit \implies 1$, ..., $digit \implies 9$
          \item Rule: $number \implies digit$
          \item Rule: $number \implies digit number$
          \item Rule: $expression \implies number$
          \item Rule: $expression \implies expression\; operation\; expression$
          \item Rule: $expression \implies (expression)$
      \end{enumerate}
      \item The starting variable is $expression$.
  \end{enumerate}
  \end{example}


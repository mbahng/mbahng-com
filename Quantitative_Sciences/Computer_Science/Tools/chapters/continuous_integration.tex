\section{Continuous Integration (CI) with Git Actions and Docker} 

  \textbf{Continuous integration (CI)}, or \textbf{continuous development (CD)}, refers to any automated process that runs whenever you perform some action on a repository. These can include: 
  \begin{enumerate}
    \item Compiling your package upon pushing to a git branch. This saves the time of manually compiling it yourself. 
    \item Compiling and/or running unit tests on your package, over possibly different compiler/interpreter versions on different operating systems and different architectures, whenever someone opens a pull request. This is usually done by automatically creating docker images and running a script that sets up the environment for your system. 
    \item Automatically publishing a new package version to PyPI upon a push to the master branch of a repository. 
  \end{enumerate} 

  \textbf{Github actions} provide \textbf{workflow scripts} that you can include in your repository's \texttt{github/workflows/} directory that automates this. They are essentially yaml files that activate upon some command, whether that'd be a push to a branch, a pull request, or even the completion/failure of another workflow. This gives great convenience in deploying code. 


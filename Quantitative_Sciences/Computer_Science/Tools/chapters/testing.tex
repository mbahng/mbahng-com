\section{Unit and Integration Tests} 

  In general, if you can reduce the number of lines of code to accomplish a task, it is seen as a good thing. You are (most of the time) simplifying the logic and therefore reducing the probability of there being bugs. This leads to our first rule. 

  \begin{theorem}
    Code is not an asset. It's a liability. 
  \end{theorem}

  However, codebases must grow, and therefore they must be maintained. As a student, I never worked with unit tests since I never developed something large or complex enough to warrant them. This is the case for most college students, and unless you start working---either during an internship or full-time---you may not even know unit testing even exists. When I wrote my first unit tests during my internship in junior year, the existing testing suite was massive and a lot to take in. I was supposed to write separate tests for the new features I was working on, but I felt like I was winging it since I never properly learned how to write tests. I've read up on a few books and used them to write this guide. 
  \begin{enumerate}
    \item Korikov's \textit{Unit Testing: Principles, Practices, Patterns}
  \end{enumerate} 

  There are two general schools of unit testing (classical/Detroit and London), and I present the classical school's definition below. 

  \begin{definition}[Unit Test]
    In classical unit testing, a \textbf{unit test} is a test that satisfies the 3 properties. 
    \begin{enumerate}
      \item \textit{Atomicity}. It verifies a single unit of behavior.\footnote{This may or may not encompass units of \textit{code}, which are generally classes in OOP. }
      \item \textit{Efficiency}. It does it quickly, and 
      \item \textit{Isolation}. It does it in isolation from other tests.\footnote{TBD}
    \end{enumerate}
  \end{definition}

  The London school still asserts efficiency, but the atomicity and isolation requirements are stricter. The basic idea is that say you have two classes $A, B$ with $B$ a child of $A$. You want to test the behavior of each class and therefore have test suites $T_A, T_B$ for each class. Now say that there is a bug in $T_A$. Then $T_A$ and $T_B$ will fail, even though the singular problem was in $A$. This doesn't provide the isolation that we need, since $T_C$ for any class $C$ should fail if and only if there is a bug in $C$. This is why in the London school, you make \textit{double/mock classes}, which are minimal copies of the class which are created for testing only. 

  \begin{definition}[Integration Test]
    An \textbf{integration test} is a test doesn't satisfy precisely one of the three properties. 
    \begin{enumerate}
      \item \textit{Atomicity}. You may need to see how two units of code act together. 
      \item \textit{Efficiency}. This is subjective depending on your time constraints. 
      \item \textit{Isolation}. Multiple tests may use a shared dependency. 
    \end{enumerate}
    An \textbf{end-to-end test} is usually defined as not satisfying both atomicity and isolation, which often means that it doesn't run quickly either.  
  \end{definition} 

  What should unit tests achieve? 
  \begin{enumerate}
    \item \textit{Protection against regressions}.\footnote{A regression is a software bug.} When you add a new feature that introduces a new bug, the tests should find that bug and report it before the features is pushed into production. Having good protection reduces the probability of tests passing that should be actually failing, i.e. false negatives. 
    \item \textit{Resistance to Refactoring}. A test should still be runnable after refactoring your code, i.e. should not be about implementation details. This is a binary attribute: a test either has resistance or it does not. If you don't do this, then you will generate a lot of tests that should pass but fail due to refactoring, i.e. false positives. This brittleness dilutes your ability and willingness to react to problems in the code. 
    \item \textit{Fast Feedback}. Should be fast. 
    \item \textit{Maintainability}. Should be maintainable. 
  \end{enumerate}

  With this in mind, we will talk about the ways in which we can write tests. 

\subsection{Structure of Unit Tests}

  To write a proper unit test, use the following, also called the Given-When-Then pattern. 
  \begin{enumerate}
    \item \textit{Arrange}. Bring the system under test (SUT) and dependencies to a desired state. This is usually the largest of the three, but if it's significantly larger than the act and assert sections combined, then it's a good idea to extract the arrangements either into private methods or a separate function. 
    \item \textit{Act}. Call methods on SUT, pass the prepared dependencies, and capture the output (if any). 
    \item \textit{Assert}. Verify the outcome. This should be a single line of a method call, and be careful if it is not because then this indicates that an atomic behavior is really 2 methods, and there is an incorrect design choice somewhere. Then it can be followed by one or more assertions. 
    \item \textit{Teardown}. Depending on the language, this may be necessary if there is no automatic garbage disposal. 
  \end{enumerate}

  If you have an arange, act, assert, act, assert, etc., then this is a sign that it's an integration test. Alternatively, you should refactor it so that it is a sequence of unit tests. 

  \begin{theorem}[Avoid If Statements in Tests]
    An if statement is a conditional, and this is branching behavior that we don't want in a linear sequence of code in a unit test. 
  \end{theorem}  

  Assert statements should be outputted by verbose messages. 

\subsubsection{Output Based Testing}

  \begin{definition}[Output-Based (Functional) Testing]
    
  \end{definition}

\subsubsection{State Based Testing}

  \begin{definition}[State-Based Testing]
    
  \end{definition}

\subsubsection{Communication Based Testing}

  \begin{definition}[Communication-Based Testing]
    
  \end{definition}

\subsection{Continuous Integration (CI) with Git Actions and Docker} 

  \textbf{Continuous integration (CI)}, or \textbf{continuous development (CD)}, refers to any automated process that runs whenever you perform some action on a repository. These can include: 
  \begin{enumerate}
    \item Compiling your package upon pushing to a git branch. This saves the time of manually compiling it yourself. 
    \item Compiling and/or running unit tests on your package, over possibly different compiler/interpreter versions on different operating systems and different architectures, whenever someone opens a pull request. This is usually done by automatically creating docker images and running a script that sets up the environment for your system. 
    \item Automatically publishing a new package version to PyPI upon a push to the master branch of a repository. 
  \end{enumerate} 

  \textbf{Github actions} provide \textbf{workflow scripts} that you can include in your repository's \texttt{github/workflows/} directory that automates this. They are essentially yaml files that activate upon some command, whether that'd be a push to a branch, a pull request, or even the completion/failure of another workflow. This gives great convenience in deploying code. 



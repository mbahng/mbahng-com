\section{Classification of Groups} 

\subsection{Direct Products}

  \begin{definition}[Direct Product]
    The \textbf{direct product} of two groups $(G, \cdot)$ and $(H, \ast)$ is the set 
    \begin{equation}
      G \times H \equiv \{ (g, h) \mid g \in G, h \in H \}
    \end{equation} 
    equipped with the operation 
    \begin{equation}
      (g_1, h_1) \cdot (g_2, h_2) \coloneqq (g_1 \cdot g_2, h_1 \ast h_2)
    \end{equation}
  \end{definition}
  \begin{proof}
    It is pretty trivial to see that this is a group. 
  \end{proof}

  \begin{example}[General Affine Group]
    The \textbf{general affine group} is defined 
    \begin{equation}
      \GA(V) \equiv \Tran(V) \times \GL(V)
    \end{equation}
  \end{example}

  \begin{example}[Galileo Group]
    The \textbf{Galileo Group} is the transformation group of spacetime symmetries that are used to transform between two reference frames which differ only by constant relative motion within the constructs of Newtonian physics. It is denoted 
    \begin{equation}
      \Tran \mathbb{R}^{4} \times H \times \O(3)
    \end{equation}
    where $H$ is the group of transformations of the form 
    \begin{equation}
      (x, y, z, t) \mapsto (x+at, y+bt, z+ct, t)
    \end{equation}
  \end{example}

  \begin{example}[Poincaré Group]
    The \textbf{Poincaré Group} is the symmetry group of spacetime within the principles of relativistic mechanics, denoted
    \begin{equation}
      G = \Tran \mathbb{R}^{4} \times \O_{3,1}
    \end{equation}
    where $\O_{3,1}$ is the group of linear transformations preserving the polynomial 
    \begin{equation}
      x^{2} + y^{2} + z^{2} - t^{2}
    \end{equation}
  \end{example} 

\subsection{Semidirect Products} 

\subsection{Classification of Finite Abelian Groups} 

  \begin{theorem}[Groups of Order 1, 2, 3]
    We have the following. 
    \begin{enumerate}
      \item There is only one group of order 1. 
        \begin{equation}
          Z_1 \simeq S_1 \simeq A_2
        \end{equation}

      \item There is only one group of order 2. 
        \begin{equation}
          Z_2 \simeq S_2 \simeq D_2
        \end{equation}

      \item There is only one group of order 3. 
        \begin{equation}
          Z_3 = A_3
        \end{equation}
    \end{enumerate}
  \end{theorem}

  \begin{theorem}[Groups of Order 4]
    There are two groups of order 4. 
    \begin{equation}
      Z_4, \qquad Z_2^2 \simeq D_4
    \end{equation}
  \end{theorem}

\subsection{Group Extensions}

\subsection{Classification of Simple Groups of Small Order}

  \begin{theorem}[Classification of Simple Groups of Small Order]
    The following are the only groups of order $n$. You can notice that it is dominated by direct products of cyclic groups, since they exist for every order, while the other types increase in order very fast.   

    \begin{figure}[H]
      \centering
      \begin{tabular}{|c|l|l|}
      \hline
      $n$ & \textbf{Abelian Groups} & \textbf{Non-Abelian Groups} \\
      \hline
      1 & $\{e\}$ (trivial group) & None \\
      \hline
      2 & $\mathbb{Z}_2 = S_2 = \text{Dih}(1)$ & None \\
      \hline
      3 & $\mathbb{Z}_3 = A_3$ & None \\
      \hline
      4 & $\mathbb{Z}_4$, $\mathbb{Z}_2 \times \mathbb{Z}_2 = \text{Dih}(2)$ & None \\
      \hline
      5 & $\mathbb{Z}_5$ & None \\
      \hline
      6 & $\mathbb{Z}_6 = \mathbb{Z}_3 \times \mathbb{Z}_2$ & $S_3 = \text{Dih}(3)$ \\
      \hline
      7 & $\mathbb{Z}_7$ & None \\
      \hline
      8 & $\mathbb{Z}_8$, $\mathbb{Z}_4 \times \mathbb{Z}_2$, $\mathbb{Z}_2 \times \mathbb{Z}_2 \times \mathbb{Z}_2$ & $D_4 = \text{Dih}(4)$, $Q_8$ (quaternion) \\
      \hline
      9 & $\mathbb{Z}_9$, $\mathbb{Z}_3 \times \mathbb{Z}_3$ & None \\
      \hline
      10 & $\mathbb{Z}_{10} = \mathbb{Z}_5 \times \mathbb{Z}_2$ & $D_5 = \text{Dih}(5)$ \\
      \hline
      11 & $\mathbb{Z}_{11}$ & None \\
      \hline
      12 & $\mathbb{Z}_{12} = \mathbb{Z}_4 \times \mathbb{Z}_3$, $\mathbb{Z}_6 \times \mathbb{Z}_2$, $\mathbb{Z}_2 \times \mathbb{Z}_2 \times \mathbb{Z}_3$ & $A_4$, $D_6 = \text{Dih}(6)$, $\mathbb{Z}_3 \rtimes \mathbb{Z}_4$ (dicyclic) \\
      \hline
      \end{tabular}
      \caption{Classification of groups up to order 12.}
      \label{tab:groups_up_to_order_10}
    \end{figure}
  \end{theorem}



\section{Lipschitz Concentration and Transportation Inequalities}

\subsection{Concentration in Metric Spaces}

  Recall what a Lipschitz function is. 

  \begin{definition}[Lipschitz Function]
  Let $(X, d)$ be a matrix space. A function $f: X \rightarrow \mathbb{R}$ is called $L$-\textbf{Lipschitz} if $|f(x) - f(y)| \leq L \, d(x, y)$ for all $x, y \in X$. The family of all $1$-Lipschitz functions is denoted $\Lip(X)$. 
  \end{definition}

  Remember that given iid $X_1, \ldots, X_n \sim N(0, 1)$, Gaussian concentration states that the random variable is $|| ||\nabla f||^2 ||_\infty$-subgaussian. But we can write it in an equivalent way in terms of a Lipschitz property. 

  \begin{lemma}
  Let $f: \mathbb{R}^n \rightarrow \mathbb{R}$ be a $C^1$ function. Then, $|| ||\nabla f||^2 ||_\infty \leq L^2$ if and only if $f$ is $L$-lipschitz. 
  \end{lemma}

  Therefore, if given random vector $X \sim N(0, I)$, then $f(X)$ is $1$-subgaussian for every $f \in \Lip(\mathbb{R}^n, ||\cdot||)$. 


\section{Exercises} 

\subsection{Vector Spaces and Dual Spaces} 

  \begin{exercise}[Lax 1.1] 
    Show that the zero of vector addition is unique.
  \end{exercise}

  \begin{exercise}[Lax 1.2] 
    Show that the vector with all components zero serves as the zero element of classical vector addition.
  \end{exercise}

  \begin{exercise}[Lax 1.3] 
    Show that (i) and (iv) are isomorphic.
  \end{exercise}

  \begin{exercise}[Lax 1.4] 
    Show that if $S$ has $n$ elements, (i) and (iii) are isomorphic.
  \end{exercise}

  \begin{exercise}[Lax 1.5] 
    Show that when $K = \mathbb{R}$, (iv) is isomorphic with (iii) when $S$ consists of $n$ distinct points of $\mathbb{R}$.
  \end{exercise}

  \begin{exercise}[Lax 1.6] 
    Denote by $X$ the linear space of all polynomials $p(t)$ of degree $< n$, and denote by $Y$ the set of polynomials that are zero at $t_1,\ldots,t_j$, $j < n$.
    (i) Show that $Y$ is a subspace of $X$.
    (ii) Determine dim $Y$.
    (iii) Determine dim $X/Y$.
  \end{exercise}

  \begin{exercise}[Lax 1.7] 
    Prove (i)-(iii) above. Show furthermore that if $x_1 \equiv x_2$, then $kx_1 \equiv kx_2$ for every scalar $k$.
  \end{exercise}

  \begin{exercise}[Lax 1.9] 
    Show that the set of all linear combinations of $x_1,\ldots,x_j$ is a subspace of $X$, and that it is the smallest subspace of $X$ containing $x_1,\ldots,x_j$. This is called the subspace spanned by $x_1,\ldots,x_j$.
  \end{exercise}

  \begin{exercise}[Lax 1.10] 
    Show that if the vectors $x_1,\ldots,x_j$ are linearly independent, then none of the $x_i$ is the zero vector.
  \end{exercise}

  \begin{exercise}[Lax 1.15] 
    Show that the above definition of addition and multiplication by scalars is independent of the choice of representatives in the congruence class.
  \end{exercise}

  \begin{exercise}[Lax 1.11] 
    Prove that if $X$ is finite dimensional and the direct sum of $Y_1,\ldots,Y_m$, then
    \[ \dim X = \sum \dim Y_j. \]
  \end{exercise}

  \begin{exercise}[Lax 1.12] 
    Show that every finite-dimensional space $X$ over $K$ is isomorphic to $K^n$, $n = \dim X$. Show that this isomorphism is not unique when $n$ is $>1$.
  \end{exercise}

  \begin{exercise}[Lax 1.14] 
    Show that two congruence classes are either identical or disjoint.
  \end{exercise}

  \begin{exercise}[Lax 1.18] 
    Show that
    \[ \dim X_1 \oplus X_2 = \dim X_1 + \dim X_2. \]
  \end{exercise}

  \begin{exercise}[Lax 1.19] 
    $X$ a linear space, $Y$ a subspace. Show that $Y \oplus X/Y$ is isomorphic to $X$.
  \end{exercise}

  \begin{exercise}[Lax 1.17] 
    Prove Corollary 6': A subspace $Y$ of a finite-dimensional linear space $X$ whose dimension is the same as the dimension of $X$ is all of $X$.
  \end{exercise}

  \begin{exercise}[Lax 2.1] 
    Given a nonzero vector $x_1$ in $X$, show that there is a linear function $l$ such that...
  \end{exercise}

  \begin{exercise}[Lax 2.2] 
    Verify that $Y^\perp$ is a subspace of $X'$.
  \end{exercise}

  \begin{exercise}[Lax 2.3] 
    Prove Theorem 6: Denote by $Y$ the smallest subspace containing $S$:
    \[S^\perp = Y^\perp.\]
  \end{exercise}

  \begin{exercise}[Lax 2.4] 
    In Theorem 6 take the interval $I$ to be $[-1, 1]$, and take $n$ to be 3. Choose the three points to be $t_1 = -a$, $t_2 = 0$, and $t_3 = a$.
    
    \begin{enumerate}
      \item[(i)] Determine the weights $m_1, m_2, m_3$ so that (9) holds for all polynomials of degree $<3$.
      \item[(ii)] Show that for $a > \sqrt{1/3}$, all three weights are positive.
      \item[(iii)] Show that for $a = \sqrt{3/5}$, (9) holds for all polynomials of degree $<6$.
    \end{enumerate}
  \end{exercise}

  \begin{exercise}[Lax 2.5] 
    In Theorem 6 take the interval $I$ to be $[-1, 1]$, and take $n$ to be 4. Choose the four points to be $-a, -b, b, a$.
    
    \begin{enumerate}
      \item[(i)] Determine the weights $m_1, m_2, m_3$, and $m_4$ so that (9) holds for all polynomials of degree $<4$.
      \item[(ii)] For what values of $a$ and $b$ are the weights positive?
    \end{enumerate}
  \end{exercise}

  \begin{exercise}[Lax 2.6] 
    Let $\mathcal{P}_2$ be the linear space of all polynomials
    \[p(x) = a_0 + a_1x + a_2x^2\]
    
    with real coefficients and degree $\leq 2$. Let $\xi_1, \xi_2, \xi_3$ be three distinct real numbers, and then define
    \[\ell_j = p(\xi_j) \quad \text{for} \quad j = 1,2,3.\]
    
    \begin{enumerate}
      \item[(a)] Show that $\ell_1, \ell_2, \ell_3$ are linearly independent linear functions on $\mathcal{P}_2$.
      \item[(b)] Show that $\ell_1, \ell_2, \ell_3$ is a basis for the dual space $\mathcal{P}_2'$.
      \item[(c)] 
        \begin{enumerate}
          \item[(1)] Suppose $\{e_1, \ldots, e_n\}$ is a basis for the vector space $V$. Show there exist linear functions $\{\ell_1, \ldots \ell_n\}$ in the dual space $V'$ defined by
          \[\ell_i(e_j) = \begin{cases} 1 & \text{if } i = j, \\ 0 & \text{if } i \neq j. \end{cases}\]
        
          Show that $\{\ell_1, \ldots, \ell_n\}$ is a basis of $V'$, called the \textit{dual basis}.
          \item[(2)] Find the polynomials $p_1(x), p_2(x), p_3(x)$ in $\mathcal{P}_2$ for which $\ell_1, \ell_2, \ell_3$ is the dual basis in $\mathcal{P}_2'$.
        \end{enumerate}
    \end{enumerate}
  \end{exercise}

  \begin{exercise}[Lax 2.7] 
    Let W be the subspace of $\mathbb{R}^4$ spanned by $(1, 0, -1, 2)$ and $(2, 3, 1, 1)$.
    Which linear functions $\ell(x) = c_1x_1 + c_2x_2 + c_3x_3 + c_4x_4$ are in the annihilator of W?
  \end{exercise}

\subsection{Class}

  \begin{exercise}[Math 403 Spring 2020, Week 1.2] 
    If $e_1, e_2, \ldots, e_n$ is a basis for $V$ and we define linear maps $e_i^\prime$ to $K$ by
    \[e_i^\prime(e_j) = \delta_{ij},\]
    
    then prove $e_1^\prime, e_2^\prime, \ldots, e_n^\prime$ form a basis for $V^\prime$.
  \end{exercise}

  \begin{exercise}[Math 403 Spring 2020, Week 4.2] 
    Find the Jordan Normal Form, $J$, of
    \[A = \begin{pmatrix} 
      4 & -2 & -6 \\
      -1 & 2 & 2 \\
      1 & -1 & -1
    \end{pmatrix}\]
    
    and a matrix $S$ such that $J = S^{-1}AS$.
  \end{exercise}

  \begin{exercise}[Math 403 Spring 2020, Week 4.3] 
    Let $d^n(\alpha)$ denote the dimension of the nullspace of $(\alpha I - A)^n$ for a matrix $A$ and a scalar $\alpha$. In the following examples we list all the possible nonzero values of $d^n(\alpha)$. In each case find the Jordan Normal Form of $A$ or prove $A$ cannot exist.
    \begin{enumerate}
      \item $d^1(1) = 2$, and $d^n(1) = 3$ if $n \geq 2$.
      \item $d^n(2) = 1$ if $n \geq 1$; $d^1(1) = 2$, $d^n(1) = 4$ if $n \geq 2$.
      \item $d^1(1) = 1$, $d^n(1) = 3$ if $n \geq 2$.
    \end{enumerate}
  \end{exercise}

  \begin{exercise}[Math 403 Spring 2020, Week 7.1] 
    Let
    \[\sigma_x = \begin{pmatrix} 0 & 1 \\ 1 & 0 \end{pmatrix}, \quad
    \sigma_y = \begin{pmatrix} 0 & i \\ -i & 0 \end{pmatrix}, \quad
    \sigma_z = \begin{pmatrix} 1 & 0 \\ 0 & -1 \end{pmatrix}.\]
    
    Compute $\Delta(\sigma_x)\Delta(\sigma_y)$ if the state vector is $\begin{pmatrix} 1 \\ 0 \end{pmatrix}$. Is this consistent with the Heisenberg Uncertainty Principle?
  \end{exercise}

  \begin{exercise}[Math 403 Spring 2020, Week 8.1] 
    Compute the singular value decompositions of the following matrices
    \begin{enumerate}
      \item $(1 \quad 1)$
      
      \item $\begin{pmatrix} 1 & 1 \\ 0 & 1 \end{pmatrix}$
      
      \item $\begin{pmatrix} 0 & 1 \\ -1 & 0 \end{pmatrix}$
    \end{enumerate}
  \end{exercise}

  \begin{exercise}[Math 403 Spring 2020, Week 8.2] 
    Let $M$ be any matrix over $\mathbb{R}$ and let $M^{(k)}$ be the rank $k$ approximation of $M$ we did in class. That is $M^{(k)}$ is obtained from an SVD of $M$ where we set all but the $k$ largest singular values to 0.
    \begin{enumerate}
      \item Show that $B = M^{(k)}$ is not the \textit{unique} matrix that minimizes the norm
      \[||M-B||_2\]
      if $\sigma_k = \sigma_{k+1}$ where $\sigma_1 \geq \sigma_2 \geq \cdots$ are the singular values of $M$.
      
      \item Discuss if $B = M^{(k)}$ is the unique matrix that minimizes this norm if $\sigma_k > \sigma_{k+1}$.
    \end{enumerate}
  \end{exercise}

  \begin{exercise}[Math 403 Spring 2020, Week 9.1] 
    Let
    \[A = \begin{pmatrix} 
      0 & 0 & 0 & 1/3 \\
      1/2 & 0 & 1 & 1/3 \\
      0 & 1 & 0 & 1/3 \\
      1/2 & 0 & 0 & 0
    \end{pmatrix}.\]
    
    Compute the asymptotic behaviour of $A^N v$ where $v$ has coordinates $(1/4, 1/4, 1/4, 1/4)$ and $N$ is large.
  \end{exercise}

  \begin{exercise}[Math 403 Spring 2020, Week 10.1] 
    Prove $v \otimes 0 = 0$ in $V \otimes V$ for any vector $v \in V$.
  \end{exercise}

  \begin{exercise}[Math 403 Spring 2020, Week 10.2] 
    Prove there is a natural isomorphism between the vector spaces $\text{Hom}(V,V)$ and $V \otimes V^*$.
  \end{exercise}

  \begin{exercise}[Math 403 Spring 2020, Week 10.3] 
    If $\{e1,e2,...\}$ is a basis for $V$, which vector in $V \otimes V^*$ is identified with the identity in $\text{Hom}(V,V)$ under this isomorphism.
  \end{exercise}

  \begin{exercise}[Math 403 Spring 2020, Week 10.4] 
    Given a map $f : Y \otimes X \to Z$ what is the naturally associated map $Y \to X^* \otimes Z$? Give your answer in terms of components assuming bases are given for all the spaces involved.
  \end{exercise}

  \begin{exercise}[Math 403 Spring 2020, Week 10.5] 
    Let $V = \mathbb{R}^3$ with a basis $\{e_1, e_2, e_3\}$ and with the standard inner product. Consider the vector $v \in V \otimes V$ which if we write in terms of components
    \[v = \sum_{ij} m_{ij}e_i \otimes e_j,\]
    
    then the matrix $M$ with entries $m_{ij}$ is given by
    \[M = \begin{pmatrix} 1 & 2 & -1 \\ 1 & -2 & -1 \\ -1 & 0 & 1 \end{pmatrix}.\]
    
    Find vectors $x_1, x_2, y_1, y_2 \in V$ such that
    \[v = x_1 \otimes y_1 + x_2 \otimes y_2,\]
    
    with $x_1$ perpendicular to $x_2$, and $y_1$ perpendicular to $y_2$.
  \end{exercise}

  \begin{exercise}[Math 403 Spring 2020, Week 10.6] 
    What does the exterior algebra of $\mathbb{R}^3$ look like? What is the wedge product of two vectors?
  \end{exercise}

  \begin{exercise}[Math 403 Spring 2020, Week 11.1] 
    The Baker-Campbell-Hausdorff formula says $e^A e^B = e^{A*B}$, where
    \[A * B = \sum_{k=1}^{\infty} F_k\]
    and $F_k$ is of total order $k$ in $A$ and $B$. We have
    \[F_1 = A+B\]
    \[F_2 = \frac{1}{2}[A,B].\]
    Compute $F_3$ expressing it in terms of commutators.
  \end{exercise}

  \begin{exercise}[Math 403 Spring 2020, Week 11.2] 
    Let $H = \begin{pmatrix} 1 & 0 \\ 0 & -1 \end{pmatrix}$ and $S = \exp(i\theta H) \in \text{SU}(2)$. Let $M$ be a traceless self-adjoint $2 \times 2$ matrix. Then let $M$ correspond to a three dimensional vector with components $(x, y, z)$ by decomposing using the Pauli matrices as
    \[M = x\sigma_x + y\sigma_y + z\sigma_z.\]
    
    Suppose we define
    \[M' = S^{-1}MS.\]
    
    Show $M'$ is also traceless and self-adjoint. If $M'$ is similarly associated to a three dimensional vector, show $S$ thereby induces a linear transformation on $\mathbb{R}^3$. What exactly is this transformation?
  \end{exercise}

  \begin{exercise}[Math 403 Spring 2020, Week 12.1] 
    Recall that the Baker-Campbell-Hausdorff formula says the group law is given by $e^A e^B = e^{A*B}$, where
    \[A*B = \sum_{k=1}^{\infty} F_k\]
    and $F_k$ is of total order $k$ in $A$ and $B$. We have
    \[F_1 = A+B\]
    \[F_2 = \frac{1}{2}[A,B].\]
    You computed $F_3$ last week and expressed it purely in terms of the bracket. Let $A$ and $B$ be elements of the vector space $V$ and let the bracket be a bilinear form $V \otimes V \to V$. Now think of this as an arbitrary map not necessarily given by commutators.
    \begin{enumerate}
      \item Prove $-(-B)*(-A) = A*B$ and thus the bracket obeys $[A,B] = -[B,A]$.
      \item Using your knowledge of $F_3$, prove that associativity of the group law implies the Jacobi identity
      \[[A,[B,C]]+[B,[C,A]]+[C,[A,B]]=0.\]
    \end{enumerate}
  \end{exercise}

  \begin{exercise}[Math 403 Spring 2020, Week 12.2] 
    Let $n$ denote the irreducible representation of $\mathfrak{sl}_2\mathbb{C}$ of dimension $n$. In class we showed $2 \otimes 2 = 3 \oplus 1$. Perform similar decompositions for:
    \begin{enumerate}
      \item $3 \otimes 2$ 
      \item $\Sym^2 3$ 
      \item $\Lambda^3 4$. 
    \end{enumerate}
  \end{exercise}

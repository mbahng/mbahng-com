\documentclass{article}

  % packages
    % basic stuff for rendering math
    \usepackage[letterpaper, top=1in, bottom=1in, left=1in, right=1in]{geometry}
    \usepackage[utf8]{inputenc}
    \usepackage[english]{babel}
    \usepackage{amsmath} 
    \usepackage{amssymb}
    % \usepackage{amsthm}

    % extra math symbols and utilities
    \usepackage{mathtools}        % for extra stuff like \coloneqq
    \usepackage{mathrsfs}         % for extra stuff like \mathsrc{}
    \usepackage{centernot}        % for the centernot arrow 
    \usepackage{bm}               % for better boldsymbol/mathbf 
    \usepackage{enumitem}         % better control over enumerate, itemize
    \usepackage{hyperref}         % for hypertext linking
    \usepackage{fancyvrb}          % for better verbatim environments
    \usepackage{newverbs}         % for texttt{}
    \usepackage{xcolor}           % for colored text 
    \usepackage{listings}         % to include code
    \usepackage{lstautogobble}    % helper package for code
    \usepackage{parcolumns}       % for side by side columns for two column code
    

    % page layout
    \usepackage{fancyhdr}         % for headers and footers 
    \usepackage{lastpage}         % to include last page number in footer 
    \usepackage{parskip}          % for no indentation and space between paragraphs    
    \usepackage[T1]{fontenc}      % to include \textbackslash
    \usepackage{footnote}
    \usepackage{etoolbox}

    % for custom environments
    \usepackage{tcolorbox}        % for better colored boxes in custom environments
    \tcbuselibrary{breakable}     % to allow tcolorboxes to break across pages

    % figures
    \usepackage{pgfplots}
    \pgfplotsset{compat=1.18}
    \usepackage{float}            % for [H] figure placement
    \usepackage{tikz}
    \usepackage{tikz-cd}
    \usepackage{circuitikz}
    \usetikzlibrary{arrows}
    \usetikzlibrary{positioning}
    \usetikzlibrary{calc}
    \usepackage{graphicx}
    \usepackage{algorithmic}
    \usepackage{caption} 
    \usepackage{subcaption}
    \captionsetup{font=small}

    % for tabular stuff 
    \usepackage{dcolumn}

    \usepackage[nottoc]{tocbibind}
    \pdfsuppresswarningpagegroup=1
    \hfuzz=5.002pt                % ignore overfull hbox badness warnings below this limit

  % New and replaced operators
    \DeclareMathOperator*{\card}{card}
    \DeclareMathOperator*{\argmin}{\arg\!\min}
    \DeclareMathOperator*{\argmax}{\arg\!\max}
    \newcommand{\qed}{\hfill$\blacksquare$}     % I like QED squares to be black

  % Custom Environments
    \newtcolorbox[auto counter, number within=section]{question}[1][]
    {
      colframe = orange!25,
      colback  = orange!10,
      coltitle = orange!20!black,  
      breakable, 
      title = \textbf{Question \thetcbcounter ~(#1)}
    }

    \newtcolorbox[auto counter, number within=section]{exercise}[1][]
    {
      colframe = teal!25,
      colback  = teal!10,
      coltitle = teal!20!black,  
      breakable, 
      title = \textbf{Exercise \thetcbcounter ~(#1)}
    }
    \newtcolorbox[auto counter, number within=section]{solution}[1][]
    {
      colframe = violet!25,
      colback  = violet!10,
      coltitle = violet!20!black,  
      breakable, 
      title = \textbf{Solution \thetcbcounter}
    }
    \newtcolorbox[auto counter, number within=section]{lemma}[1][]
    {
      colframe = red!25,
      colback  = red!10,
      coltitle = red!20!black,  
      breakable, 
      title = \textbf{Lemma \thetcbcounter ~(#1)}
    }
    \newtcolorbox[auto counter, number within=section]{theorem}[1][]
    {
      colframe = red!25,
      colback  = red!10,
      coltitle = red!20!black,  
      breakable, 
      title = \textbf{Theorem \thetcbcounter ~(#1)}
    } 
    \newtcolorbox[auto counter, number within=section]{proposition}[1][]
    {
      colframe = red!25,
      colback  = red!10,
      coltitle = red!20!black,  
      breakable, 
      title = \textbf{Proposition \thetcbcounter ~(#1)}
    } 
    \newtcolorbox[auto counter, number within=section]{corollary}[1][]
    {
      colframe = red!25,
      colback  = red!10,
      coltitle = red!20!black,  
      breakable, 
      title = \textbf{Corollary \thetcbcounter ~(#1)}
    } 
    \newtcolorbox[auto counter, number within=section]{proof}[1][]
    {
      colframe = orange!25,
      colback  = orange!10,
      coltitle = orange!20!black,  
      breakable, 
      title = \textbf{Proof. }
    } 
    \newtcolorbox[auto counter, number within=section]{definition}[1][]
    {
      colframe = yellow!25,
      colback  = yellow!10,
      coltitle = yellow!20!black,  
      breakable, 
      title = \textbf{Definition \thetcbcounter ~(#1)}
    } 
    \newtcolorbox[auto counter, number within=section]{example}[1][]
    {
      colframe = blue!25,
      colback  = blue!10,
      coltitle = blue!20!black,  
      breakable, 
      title = \textbf{Example \thetcbcounter ~(#1)}
    } 
    \newtcolorbox[auto counter, number within=section]{axiom}[1][]
    {
      colframe = green!25,
      colback  = green!10,
      coltitle = green!20!black,  
      breakable, 
      title = \textbf{Axiom \thetcbcounter ~(#1)}
    } 
    \newtcolorbox[auto counter, number within=section]{algo}[1][]
    {
      colframe = green!25,
      colback  = green!10,
      coltitle = green!20!black,  
      breakable, 
      title = \textbf{Algorithm \thetcbcounter ~(#1)}
    } 

    \BeforeBeginEnvironment{example}{\savenotes}
    \AfterEndEnvironment{example}{\spewnotes}
    \BeforeBeginEnvironment{lemma}{\savenotes}
    \AfterEndEnvironment{lemma}{\spewnotes}
    \BeforeBeginEnvironment{theorem}{\savenotes}
    \AfterEndEnvironment{theorem}{\spewnotes}
    \BeforeBeginEnvironment{corollary}{\savenotes}
    \AfterEndEnvironment{corollary}{\spewnotes}
    \BeforeBeginEnvironment{proposition}{\savenotes}
    \AfterEndEnvironment{proposition}{\spewnotes}
    \BeforeBeginEnvironment{definition}{\savenotes}
    \AfterEndEnvironment{definition}{\spewnotes}
    \BeforeBeginEnvironment{exercise}{\savenotes}
    \AfterEndEnvironment{exercise}{\spewnotes}
    \BeforeBeginEnvironment{proof}{\savenotes}
    \AfterEndEnvironment{proof}{\spewnotes}
    \BeforeBeginEnvironment{solution}{\savenotes}
    \AfterEndEnvironment{solution}{\spewnotes}
    \BeforeBeginEnvironment{question}{\savenotes}
    \AfterEndEnvironment{question}{\spewnotes}
    \BeforeBeginEnvironment{axiom}{\savenotes}
    \AfterEndEnvironment{axiom}{\spewnotes}
    \BeforeBeginEnvironment{algo}{\savenotes}
    \AfterEndEnvironment{algo}{\spewnotes}

    \definecolor{dkgreen}{rgb}{0,0.6,0}
    \definecolor{gray}{rgb}{0.5,0.5,0.5}
    \definecolor{mauve}{rgb}{0.58,0,0.82}
    \definecolor{darkblue}{rgb}{0,0,139}
    \definecolor{lightgray}{gray}{0.93}
    \renewcommand{\algorithmiccomment}[1]{\hfill$\triangleright$\textcolor{blue}{#1}}

    % default options for listings (for code)
    \lstset{
      autogobble,
      frame=ltbr,
      language=Python,
      aboveskip=3mm,
      belowskip=3mm,
      showstringspaces=false,
      columns=fullflexible,
      keepspaces=true,
      basicstyle={\small\ttfamily},
      numbers=left,
      firstnumber=1,                        % start line number at 1
      numberstyle=\tiny\color{gray},
      keywordstyle=\color{blue},
      commentstyle=\color{dkgreen},
      stringstyle=\color{mauve},
      backgroundcolor=\color{lightgray}, 
      breaklines=true,                      % break lines
      breakatwhitespace=true,
      tabsize=3, 
      xleftmargin=2em, 
      framexleftmargin=1.5em, 
      stepnumber=1
    }

  % Page style
    \pagestyle{fancy}
    \fancyhead[L]{Logic}
    \fancyhead[C]{Muchang Bahng}
    \fancyhead[R]{Spring 2025} 
    \fancyfoot[C]{\thepage / \pageref{LastPage}}
    \renewcommand{\footrulewidth}{0.4pt}          % the footer line should be 0.4pt wide
    \renewcommand{\thispagestyle}[1]{}  % needed to include headers in title page

\begin{document}

\title{Logic}
\author{Muchang Bahng}
\date{Spring 2025}

\maketitle
\tableofcontents
\pagebreak

\section{Propositional Logic} 

  Philosophers still debate about what a proposition really means. As a complete beginner, I mention some interpretations of it, but I by no means claim that this is the definitive definition. 

  \begin{definition}[Possible World]
    A \textbf{possible world} is a complete and consistent way the world is or could have been. 
  \end{definition} 

  The \textbf{language} of propositional logic consists of just two things: propositions and connectives. 

  \begin{definition}[Proposition]
    A \textbf{proposition} does not have a formal definition, but we can describe it in the following ways.  
    \begin{enumerate}
      \item They can be understood as an indicator function $f: W \rightarrow \{T, F\}$\footnote{$T, F$ stands for True, False.} that takes in a possible world and returns a truth value. We can also model it with the preimage of $f$ under $T$, i.e. the characteristic set of $f$. 
      \item They deal with \textbf{statements}, which are defined as declarative sentences having a truth value. 
    \end{enumerate} 
    Propositions are either true or false. 
  \end{definition}

  \begin{example}
    The proposition that \textit{the sky is blue} is represented as the function that returns $T$ for every possible world where the sky is blue. 
  \end{example}

  These declarative sentences are contrasted with questions, such as \textit{how are you doing?} and imperative statements such as \textit{please run my models}. Such non-declarative sentences have no truth value. 

  A statement can contain one or more other statements as parts. For example, compound sentences form simpler sentences. 

  \begin{definition}[Connectives]
    Statements are combined with \textbf{logical connectives}. 
    \begin{table}[H]
      \centering
      \begin{tabular}{|l|l|}
      \hline
      \textbf{Connective} & \textbf{Symbols} \\
      \hline
      AND & $A \wedge B$, $A \cdot B$, $AB$, $A \& B$, $A \&\& B$ \\
      \hline
      OR & $A \vee B$, $A + B$, $A \mid B$, $A \parallel B$ \\
      \hline
      NOT & $\neg A$, $-A$, $\overline{A}$, $\sim A$ \\
      \hline
      NAND & $\overline{A \wedge B}$, $A \mid B$, $\overline{A \cdot B}$ \\
      \hline
      NOR & $\overline{A \vee B}$, $A \downarrow B$, $\overline{A + B}$ \\
      \hline
      XOR & $A \veebar B$, $A \oplus B$ \\
      \hline
      XNOR & $A \odot B$ \\
      \hline
      IMPLIES & $A \Rightarrow B$, $A \supset B$, $A \rightarrow B$ \\
      \hline
      EQUIVALENT & $A \equiv B$, $A \Leftrightarrow B$, $A \leftrightarrow B$ \\
      \hline
      NONEQUIVALENT & $A \not\equiv B$, $A \not\Leftrightarrow B$, $A \not\leftrightarrow B$ \\
      \hline
      \end{tabular}
      \caption{Logical Connectives and Their Symbols}
      \label{tab:logical-connectives}
    \end{table} 
  \end{definition} 

  \begin{definition}[Propositional Formula]
    Propositions, represented by letters and denoted \textbf{propositional variables}, along with these symbols for connectives, combine to make a \textbf{propositional formula}. 
  \end{definition}

  Propositional logic is not concerned with the structures of propositions beyond the point where they cannot be decomposed any more by logical connectives. 

\subsection{Arguments}

  At this point we may look at a set of propositions $P_1, \ldots, P_n$ and try come to a logical conclusion $Q$. This is called an argument. 

  \begin{definition}[Argument]
    Let $P$ be a set of propositions, called the \textbf{premises}. Let $Q$ be a proposition, called the \textbf{conclusion}. Then an \textbf{argument} is an attempt to deduce $Q$ from $P$. It is written in the forms 
    \begin{enumerate}
      \item If $P$, then $Q$.  
      \item $P \implies Q$
    \end{enumerate}
    An argument is \textbf{valid} if and only if  
    \begin{enumerate}
      \item It is necessary that if $P$ is true, $Q$ is true. 
      \item It is impossible for $P$ to be true, while $Q$ is false. 
    \end{enumerate}
  \end{definition}

  \begin{example}
    The following is an argument. 
    \begin{enumerate}
      \item If \textit{it is raining}, then \textit{it is cloudy}. 
      \item \textit{It is raining}. 
      \item \textit{Therefore it is cloudy}.
    \end{enumerate}
  \end{example} 

  Logic in general aims to specify valid arguments. This is done by defining a valid argument as one in which its conclusion is a logical consequence of its premises. Determining whether a proposition is a a logical consequence of another proposition is the process of \textbf{deductive argument}, which has rules. These rules, called \textbf{rules of inference}, determines the ``legal moves'' from one or more premises to the conclusion. We give 2 familiar ones. 

  \begin{definition}[Modus Ponens]
    \textbf{Modus ponens} is a deductive argument form and rule of inference.\footnote{In some literature it is treated as an axiom, though most people think of it as a rule.} The argument states that given the premises
    \begin{enumerate}
      \item $P \implies Q$ 
      \item $P$
    \end{enumerate}
    Then our conclusion is $Q$. 
  \end{definition} 

  The next one is the familiar statement that a statement is equivalent to its contrapositive. 

  \begin{definition}[Modus Tollens]
    \textbf{Modus tollens} is a deductive argument form and a rule of inference. The argument states that given the premises 
    \begin{enumerate}
      \item $P \implies Q$ 
      \item $\neg Q$ 
    \end{enumerate}
    Then our conclusion is not $P$. 
  \end{definition}


\section{First-Order Logic} 

  In propositional logic, we deal with simple declarative propositions. \textbf{First-order logic} extends this by covering predicates and quantification. Let's motivate them. 

  We can think of predicates as properties. If we say \textit{Socrates is a philosopher} and \textit{Plato is a philosopher}, in propositional logic both these statements, represented as $P$ and $Q$, as utterances that are either true or false, and they are completely independent from one another. However, we may want to view them as an application of a predicate \textit{ $\ast$ is a philosopher} on the entities \textit{Socrates} and \textit{Plato}. This motives the formalism of the domain of discourse and the predicate. 

  \begin{definition}[Domain of Discourse]
    Given an individual $x$, its \textbf{domain of discourse} is the set over which certain variables of interest in some formal treatment may range. 
  \end{definition}

  \begin{definition}[Predicate]
    A \textbf{predicate} $P$ is a symbol that represents a property or a relation of a certain individual $x$ in a domain of discourse. Using predicates, $P(x)$ can be viewed as a proposition about the individual $x$. 
  \end{definition} 

  Note that a predicate itself is not a proposition, since saying \textit{$\ast$ is a philosopher} doesn't have any truth or false meaning to it, akin to a sentence fragment. But it is a placeholder $P(\cdot)$ upon which if an individual $x$ is put in, it makes sense to ask whether $P(x)$ is true. 

  \begin{definition}[Formula]
    A \textbf{formula} is a string of propositions, connectives, predicates, and variables $\phi$ that turns into a proposition once all free variables have been instantiated. 
  \end{definition}

  With predicates alone, all we have really done is add notational convenience. However, if we want to state a proposition not just about $x$, but its domain of discourse, then we can use quantifiers. 

  \begin{definition}[Quantifier]
    A \textbf{quantifier} is an operator that specifies how many individuals in the domain of source satisfy a proposition. The two most used quantifiers are 
    \begin{enumerate}
      \item \textit{Universal Quantification}. $\forall$, which means \textit{for every}. 
      \item \textit{Existential Quantification}. $\exists$, which means \textit{there exists}. 
    \end{enumerate}
  \end{definition} 

  These quantifiers are additional symbols in our language $\mathcal{L}$. If we add the equality symbol, we get first-order logic with equality. 

  \begin{axiom}[Equality]
    \textbf{Equality} is a primitive logical symbol which is always interpreted as the real equality relation between members of the domain of discourse. These equality axioms are: 
    \begin{enumerate}
      \item \textit{Reflexivity}. For each variable $x$, $x = x$. 
      \item \textit{Substitution for Functions}. For all variables $x$ and $y$, and any function symbol $f$, 
        \begin{equation}
          x = y \implies f(x) = f(y)
        \end{equation}
      \item \textit{Substitution for Formulas}. For any variables $x$ and $y$, and any formula $\phi(z)$ with free variable $z$, then 
        \begin{equation}
          x = y \implies (\phi(x) \implies \phi(y))
        \end{equation}
    \end{enumerate}
    Symmetry and transitivity follow from the axioms above. 
  \end{axiom} 

  Ordinary first-order interpretations have a single domain of discourse over which all quantifiers range. \textbf{Many-sorted first-order logic}, or \textbf{typed first-order logic} allows variables to have different \textbf{sorts} or \textbf{types}, i.e. coming from different domains.  

\subsection{Exercises}

  \begin{exercise}[Shifrin Abstract Algebra Appendix 1.1] 
    Negate the following sentences; in each case, indicate whether the original sentence or its negation is a true statement. Be sure to move the ``not" through all the quantifiers.
    \begin{enumerate}
      \item For every integer $n \geq 2$, the number $2^n - 1$ is prime.
      \item There exists a real number $M$ so that for all real numbers $t$, $|\sin t| \leq M$.
      \item For every real number $x > 0$, there exists a real number $y > 0$ so that $xy > 1$.
    \end{enumerate}
  \end{exercise}
  \begin{solution}
    Listed. 
    \begin{enumerate}
      \item \textit{Negation}. For at least one $n \geq 2$, the number $2^n - 1$ is composite (not prime). The negation is true. Consider $n = 4 \implies 2^4 - 1 = 15 = 3 \cdot 5$. 

      \item \textit{Negation}. There exists no real number $M$ such that for all real numbers $t$, $|\sin{t}| \leq M$. The original is true. Pick $M=1$, and by definition $|\sin{t}| \leq 1$. 

      \item \textit{Negation}. For at least one real number $x > 0$, there exists no real number $y > 0$ so that $xy > 1$. The original is true. Given a real number $x > 0$, choose $y = \frac{1}{x} + 1$. Then, 
      \begin{equation}
        xy = x \bigg( \frac{1}{x} + 1 \bigg) = 1 + x > 1
      \end{equation} 
      where the steps follow from the ordered field properties of $\mathbb{R}$. 
    \end{enumerate}
  \end{solution} 

  \begin{exercise}[Shifrin Abstract Algebra Appendix 1.4]
    Suppose $n$ is an odd integer. Prove:
    \begin{enumerate}
      \item The equation $x^2 + x - n = 0$ has no solution $x \in \mathbb{Z}$.
      \item Prove that for any $m \in \mathbb{Z}$, the equation $x^2 + 2mx + 2n = 0$ has no solution $x \in \mathbb{Z}$.
    \end{enumerate}
  \end{exercise}
  \begin{solution}
    We prove by contradiction. Assume such a solution $x$ exists for odd $n$. We consider the two cases where $x$ 
    \begin{enumerate}
      \item is even. 
      \begin{align}
        x \text{ is even} & \implies x \equiv 0 \; (\mathrm{mod } 2) \\
                          & \implies x^2 + x \equiv 0 \; (\mathrm{mod } 2) \\
                          & \implies x^2 + x - n \equiv 1 \; (\mathrm{mod } 2)
      \end{align} 

      \item is odd. 
      \begin{align}
        x \text{ is odd} & \implies x \equiv 1 \; (\mathrm{mod } 2) \\
                         & \implies x^2 + x \equiv 1 + 1 \equiv 0 (\mathrm{mod } 2) \\
                         & \implies x^2 + x - n \equiv 1 (\mathrm{mod } 2) 
      \end{align}
    \end{enumerate}
    Both cases result in the quadratic expression lying in the equivalence class $[1]$ and thus cannot be $0$. This contradicts our assumption that it is a solution. 
    We prove by contradiction. Assume a solution $x$ exists for odd $n$. Note that since $x^2 + 2mx + 2n \equiv x^2 \equiv 0 \; (\mathrm{mod } 2)$, this implies that $x \equiv 0 \; (\mathrm{mod } 2)$.\footnote{This is true if we look at the contrapositive: $x \equiv 1 \implies x^2 \equiv 1$. } Therefore, we can write $x = 2x^\prime$ for some $x^\prime \in \mathbb{Z}$, our assumption is equivalent to the existence of $x^\prime$. Substituting this gives 
    \begin{equation}
      4 x^{\prime 2} + 4 m x^\prime + 2n = 0 \iff 2x^{\prime 2} + 2m x^\prime + n = 0
    \end{equation} 
    Since $2x^{\prime 2} + 2 m x^\prime$ is even, $n$ must be even as well, which contradicts our assumption that $n$ is odd. 
  \end{solution}


\section{Second-Order Logic} 

  First order logic can quantify over individuals, but not over properties. That is, while we can state something like 

  \begin{center}
    \textit{There exists x such that x is a cube.}
  \end{center} 

  we cannot quantify over a predicate. That is, the statement 
  
  \begin{center}
    \textit{There exists a property $P$ such that a cube satisfies $P$.}
  \end{center}

  This statement does not make sense in first-order logic, but makes sense in second-order logic. 



\end{document}

\section{Propositional Logic} 

  Philosophers still debate about what a proposition really means. As a complete beginner, I mention some interpretations of it, but I by no means claim that this is the definitive definition. 

  \begin{definition}[Possible World]
    A \textbf{possible world} is a complete and consistent way the world is or could have been. 
  \end{definition} 

  The \textbf{language} of propositional logic consists of just two things: propositions and connectives. 

  \begin{definition}[Proposition]
    A \textbf{proposition} does not have a formal definition, but we can describe it in the following ways.  
    \begin{enumerate}
      \item They can be understood as an indicator function $f: W \rightarrow \{T, F\}$\footnote{$T, F$ stands for True, False.} that takes in a possible world and returns a truth value. We can also model it with the preimage of $f$ under $T$, i.e. the characteristic set of $f$. 
      \item They deal with \textbf{statements}, which are defined as declarative sentences having a truth value. 
    \end{enumerate} 
    Propositions are either true or false. 
  \end{definition}

  \begin{example}
    The proposition that \textit{the sky is blue} is represented as the function that returns $T$ for every possible world where the sky is blue. 
  \end{example}

  These declarative sentences are contrasted with questions, such as \textit{how are you doing?} and imperative statements such as \textit{please run my models}. Such non-declarative sentences have no truth value. 

  A statement can contain one or more other statements as parts. For example, compound sentences form simpler sentences. 

  \begin{definition}[Connectives]
    Statements are combined with \textbf{logical connectives}. 
    \begin{table}[H]
      \centering
      \begin{tabular}{|l|l|}
      \hline
      \textbf{Connective} & \textbf{Symbols} \\
      \hline
      AND & $A \wedge B$, $A \cdot B$, $AB$, $A \& B$, $A \&\& B$ \\
      \hline
      OR & $A \vee B$, $A + B$, $A \mid B$, $A \parallel B$ \\
      \hline
      NOT & $\neg A$, $-A$, $\overline{A}$, $\sim A$ \\
      \hline
      NAND & $\overline{A \wedge B}$, $A \mid B$, $\overline{A \cdot B}$ \\
      \hline
      NOR & $\overline{A \vee B}$, $A \downarrow B$, $\overline{A + B}$ \\
      \hline
      XOR & $A \veebar B$, $A \oplus B$ \\
      \hline
      XNOR & $A \odot B$ \\
      \hline
      IMPLIES & $A \Rightarrow B$, $A \supset B$, $A \rightarrow B$ \\
      \hline
      EQUIVALENT & $A \equiv B$, $A \Leftrightarrow B$, $A \leftrightarrow B$ \\
      \hline
      NONEQUIVALENT & $A \not\equiv B$, $A \not\Leftrightarrow B$, $A \not\leftrightarrow B$ \\
      \hline
      \end{tabular}
      \caption{Logical Connectives and Their Symbols}
      \label{tab:logical-connectives}
    \end{table} 
  \end{definition} 

  \begin{definition}[Propositional Formula]
    Propositions, represented by letters and denoted \textbf{propositional variables}, along with these symbols for connectives, combine to make a \textbf{propositional formula}. 
  \end{definition}

  Propositional logic is not concerned with the structures of propositions beyond the point where they cannot be decomposed any more by logical connectives. 

\subsection{Arguments}

  At this point we may look at a set of propositions $P_1, \ldots, P_n$ and try come to a logical conclusion $Q$. This is called an argument. 

  \begin{definition}[Argument]
    Let $P$ be a set of propositions, called the \textbf{premises}. Let $Q$ be a proposition, called the \textbf{conclusion}. Then an \textbf{argument} is an attempt to deduce $Q$ from $P$. It is written in the forms 
    \begin{enumerate}
      \item If $P$, then $Q$.  
      \item $P \implies Q$
    \end{enumerate}
    An argument is \textbf{valid} if and only if  
    \begin{enumerate}
      \item It is necessary that if $P$ is true, $Q$ is true. 
      \item It is impossible for $P$ to be true, while $Q$ is false. 
    \end{enumerate}
  \end{definition}

  \begin{example}
    The following is an argument. 
    \begin{enumerate}
      \item If \textit{it is raining}, then \textit{it is cloudy}. 
      \item \textit{It is raining}. 
      \item \textit{Therefore it is cloudy}.
    \end{enumerate}
  \end{example} 

  Logic in general aims to specify valid arguments. This is done by defining a valid argument as one in which its conclusion is a logical consequence of its premises. Determining whether a proposition is a a logical consequence of another proposition is the process of \textbf{deductive argument}, which has rules. These rules, called \textbf{rules of inference}, determines the ``legal moves'' from one or more premises to the conclusion. We give 2 familiar ones. 

  \begin{definition}[Modus Ponens]
    \textbf{Modus ponens} is a deductive argument form and rule of inference.\footnote{In some literature it is treated as an axiom, though most people think of it as a rule.} The argument states that given the premises
    \begin{enumerate}
      \item $P \implies Q$ 
      \item $P$
    \end{enumerate}
    Then our conclusion is $Q$. 
  \end{definition} 

  The next one is the familiar statement that a statement is equivalent to its contrapositive. 

  \begin{definition}[Modus Tollens]
    \textbf{Modus tollens} is a deductive argument form and a rule of inference. The argument states that given the premises 
    \begin{enumerate}
      \item $P \implies Q$ 
      \item $\neg Q$ 
    \end{enumerate}
    Then our conclusion is not $P$. 
  \end{definition}


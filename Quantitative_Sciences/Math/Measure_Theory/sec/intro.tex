In math, we are first taught to solve simple equations like $x^2 - 2x + 4 = 0$ for a certain \textit{number} $x$, but in real world applications, we must now solve for some \textit{function} $f$ satisfying an equation 
\begin{equation}
  \mathcal{L}(f) = 0
\end{equation}
where $\mathcal{L}$ is some operator on functions. This is usually difficult, and many times a solution does not exist. However, we can find approximate solutions, say 
\begin{align*}
  \mathcal{L}(f) & = 1/2 \\
  \mathcal{L}(f) & = 1/4 \\ 
  \mathcal{L}(f) & = 1/8 \\
  \ldots & = \ldots 
\end{align*}
and approximate the solution as 
\begin{equation}
  f = \lim_{n \rightarrow \infty} f_n 
\end{equation}
Given that this limit exists, we can usually define $f$ pointwise using a point-wise limit 
\begin{equation}
  f(x) = \lim_{n \rightarrow \infty} f_n (x) \text{ for all } x
\end{equation}
but the function in total is very ugly and not Riemann integrable. The classic non-Riemann integrable function is the 
\begin{equation}
  f(x) = \chi_{\mathbb{R} \setminus \mathbb{Q}} (x) \coloneqq \begin{cases} 
  1 & \text{ if } x \in \mathbb{R} \setminus \mathbb{Q} \\
  0 & \text{ if} x \in \mathbb{Q} 
  \end{cases}
\end{equation}
Since $\mathbb{Q}$ is countable, we can enumerate $\mathbb{Q} = \{q_n\}_{n=1}^\infty$ and define the sequence of functions 
\begin{equation}
  f_n = 1 - \chi_{\{q_j\}_{j=1}^n}(x)
\end{equation}
that start off with the constant function $1$ and then "removes" points in $\mathbb{Q}$, setting their image to $0$. It is clear that since we are removing points, every function in the sequence has an integral (from $0$ to $1$) of $1$, and therefore the integral of $f$ should also be $1$. 
\begin{equation}
  \int_0^1 f_n \, dx = 1 \implies \int_0^1 f \,dx = \int_0^1 \lim_{n \rightarrow \infty} f_n \,dx = \lim_{n \rightarrow \infty} \int_0^1 f_n \,dx
\end{equation}
What is crucial for mathematicians to work with is the capability to take the limit from inside the integral to outside the integral. The problem is that $f$ is not a Riemann integral function. 

\begin{definition}[Riemann Integrable Function]
  Given a function $f: [0, 1] \longrightarrow \mathbb{R}$, let us consider some partition of $[0, 1]$ into intervals $P = \{I_0, I_1, \ldots, I_N\}$, then, for each $I \in P$, we can take the supremum $M_I = \sup_{x \in I} f(x)$ and infimum $m_I = \inf_{x \in I} f(x)$ and bound $f$ by the upper and lower Riemann sums. 
  \begin{equation}
    \sum_{I \in P} m_I |I| \leq \int_0^1 f \,dx \leq \sum_{I \in P} M_I |I| 
  \end{equation}
  where $|I|$ is the length of interval $I$. If we take \textit{all} possible partitions, the bound should still hold. 
  \begin{equation}
    m = \sup_P \Big\{ \sum_{I \in P} m_I |I| \Big\} \leq \int_0^1 f \,dx \leq \inf_P \Big\{ \sum_{I \in P} M_I |I| \Big\} = M
  \end{equation}
  and if the lower bound is equal to the upper bound $m = M$, then the integral is this number and $f$ is considered Riemann integrable. 
\end{definition}

Now since $\mathbb{Q}$ is dense in $\mathbb{R}$, for every interval $I$ in every partition $P$ will have $m_I = 0$ and $M_I = 1$ for the Riemann function, meaning that the lower bound will always be $0$ and the upper bound will always be $1$. So, $\int_0^1 \chi_{\mathbb{R} \setminus \mathbb{Q}} (x)$ can take on any value in $[0, 1]$, which isn't helpful. The fact that we can't integrate this really simple function is a problem. For nice functions, we can partition it so that the base of each Riemann rectangle is a nice interval, while the base of the Riemann function is an "interval with holes." The problem really boils down to measuring what the "length" of this set is. So the problem with the Riemann integral isn't the integral itself, but the fact that we can't give a meaningful size to the set $\mathbb{R} \setminus \mathbb{Q}$. Now mathematicians in the 19th century thought that as long as we solve this problem, we should be good to go, but Banach and Tarski proved that there exists sets that cannot be measured with their famous paradox, which says that you can take any set $P$, disassemble it into a finite set of pieces, and rearrange it (under isometry and translations) so that it has a different size than the original $P$. So, we have to exclude some sets that are not measurable. The collection of sets that we \textit{can} measure is called the $\sigma$-algebra. 


\section{Euclidean Topology} 

  With the construction of the real line and the real space, the extra properties of completeness, norm, and order (for the real line) allows us to restate these topological properties in terms of these ``higher-order'' properties. It also proves much more results than for general topological spaces. Therefore, the next few sections will focus on reiterating the topological properties of $\mathbb{R}$ and $\mathbb{R}^n$ (this can be done slightly more generally for metric spaces, but we talk about this in point-set topology). In this section, we will restate the notion of open sets, limit points, compactness, connectedness, and separability. Then we can continue in the next section sequences and their limits, and after that we describe continuity. Once this is done, we can focus constructing the derivative and integral, which are unique to Banach spaces.  

\subsection{Open Sets} 

  It is well-known that the set of open-balls of a metric space $(X, d)$ is indeed a topology, which we prove in point-set topology. Once we prove this, we have access to a whole suite of theorems on topological spaces that we can just apply to $\mathbb{R}^n$. We will restate many of these topological theorems for completeness but will not prove them. However, if any of these theorems use any other structure, such as order/metrics/norms/completeness, we will have to prove them. 

  \begin{definition}[Topology]
    Let $X$ be a set and $\mathscr{T}$ be a family of subsets of $X$. Then $\mathscr{T}$ is a \textbf{topology} on $X$\footnote{I will use script letters to denote topologies and capital letters to denote sets.} if it satisfies the following properties. 
    \begin{enumerate}
      \item \textit{Contains Empty and Whole Set}: 
      \begin{equation}
        \emptyset, X \in \mathscr{T}
      \end{equation}

      \item \textit{Closure Under Union}. If $\{U_\alpha\}_{\alpha \in A}$ is a class of sets in $\mathscr{T}$, then 
      \begin{equation}
        \bigcup_{\alpha \in A} U_\alpha \in \mathscr{T}
      \end{equation}

      \item \textit{Closure Under Finite Intersection}: If $U_1, \ldots, U_n$ is a finite class of sets in $\mathscr{T}$, then 
      \begin{equation}
       \bigcap_{i=1}^{n} U_i \in \mathscr{T}
      \end{equation}
    \end{enumerate}
    A \textbf{topological space} is denoted $(X, \mathscr{T})$. 
  \end{definition}

  \begin{theorem}[Euclidean Topology]
    Let $\tau_{\mathbb{R}}$ (which we denote as $\mathscr{T}$) be the set of subsets $S$ of $(\mathbb{R}^n, || \cdot ||)$ satisfying the property that if $x \in S$, then there exists an open $\epsilon$-ball $B(x, \epsilon)$ s.t. $B \subset S$. $\mathscr{T}$ is a topology of $\mathbb{R}^n$. 
  \end{theorem} 
  \begin{proof} 
    We prove the following three properties. 
    \begin{enumerate}
      \item $\emptyset, \mathbb{R}^n$ are open. 
      \item For any collection $\{G_\alpha\}_\alpha$ of open sets, $\cup_\alpha G_\alpha$ is open.  
      \item For any finite collection $G_1, \ldots, G_n$ of open sets, $\cap_{i=1}^n G_i$ is open. 
    \end{enumerate}
    Listed. 
    \begin{enumerate}
      \item Let $x \in \cup_\alpha G_\alpha$. Then, $x \in G_k$ for some $k$ and since $G_k$ is open, there exists a $B_\epsilon (x) \subset G_k \subset \cup_{\alpha} G_\alpha$, proving that $\cup_\alpha G_\alpha$ is open. 
      \item Let $x \in \cap_{i=1}^n G_i$. Then, $x \in G_i$ for every $i$, and so for each $G_i$, there exists an $\epsilon_i > 0$ s.t. $B_{\epsilon_i} (x) \subset G_i$. Since the set $\{e_i\}$ is finite, we can take 
      \[\epsilon = \min_i \{\epsilon_i\}\]
      and see that $B_\epsilon (x) \subset G_i$ for all $i$, which implies that $B_\epsilon (x) \subset \cap_{i=1}^n G_i$. Since we have proved the existence of $\epsilon$, $\cap_{i=1}^n G_i$ is open. 
    \end{enumerate}
  \end{proof}

  \begin{definition}[Open Set]
    An \textbf{open set} is an element of $\mathscr{T}$. 
    \begin{enumerate}
      \item An \textbf{open neighborhood}, or sometimes just the \textbf{neighborhood}, of $x \in \mathbb{R}^n$ is an open set $U_x$ containing $x$. 
      \item A \textbf{punctured neighborhood} is $U_x^{\circ} = U_x \setminus \{x\}$. 
    \end{enumerate}
  \end{definition}

  \begin{theorem}[Equivalence to Open Ball Topology]
    $\mathscr{T}$ is equal to the topology $\mathscr{T}^\prime$ generated by the basis $\mathscr{B}$ of open balls 
    \begin{equation} 
      B(x, r) \coloneqq \{ y \in \mathbb{R}^n \mid ||x - y|| < r\}
    \end{equation}
  \end{theorem} 
  \begin{proof}
    Let $\mathscr{T}$ be the Euclidean topology and $\mathscr{T}^\prime$ be the open ball topology. 
    \begin{enumerate}
      \item We show $\mathscr{T} \subset \mathscr{T}^\prime$. 
      \item We show $\mathscr{T}^\prime \subset \mathscr{T}$. 
    \end{enumerate}
  \end{proof} 

  By defining the topology, we have automatically defined a bunch of topological objects and properties. For clarification, we will restate them. 
  
  \begin{corollary}
    An open ball is an open set. 
  \end{corollary}
  \begin{proof}
    Given $x \in B_r (p)$, we can imagine that $x$ will always have some space between it and the boundary. We want to show that there exists some $\epsilon >0$ s.t. $B_\epsilon (x) \subset B_r (p)$. That is, given any point $y \in B_\epsilon (x)$, we can show that $y \in B_r (p)$. Since $||x - p|| < r$, there exists some space $0 < r - ||x - p||$. There always exists a real number $0 < \epsilon < r - ||x - p||$, so given $y \in B_\epsilon (x)$, we can bound
    \begin{equation}
      ||y - p|| = ||y - x + x - p|| \leq ||y - x|| + ||x - p|| \leq \epsilon + ||x - p|| < r
    \end{equation}
  \end{proof}

  \begin{example}
    Here are some examples of sets which are open and not open. 
    \begin{enumerate}
      \item $U=\{(x,y)\in \mathbb{R}^2 : x^2+y^2 \neq 1\}$ is open since for every point $x \in U$, we just need to find a radius $\epsilon >0$ that is smaller than its distance to the unit circle. 
      \item $(a, b) \times (c, d) \subset \mathbb{R}^2$ is open since given a point $x$, we can take the minimum of its distance between the two sides of the rectangle and construct an open ball. 
      \item $S=\{(x,y)\in \mathbb{R}^2:xy\neq 0\}$ is open since given a point $x \in S$, we can take the minimum of the distance between it and the $x$ and $y$ axes. 
      \item The set of all complex $z$ such that $|z| \leq 1$ is not open since we cannot construct open balls at the boundary points that are fully contained in the set. 
      \item The set $S = \{1/n\}_{n \in \mathbb{N}}$ is not open since given any point $x = 1/n$, we can construct an open ball with radius $\epsilon < 1/(n+1)$, which contains irrationals that are not in $S$. 
    \end{enumerate}
  \end{example}

  \begin{definition}[Interior Point]
    A point $p \in S$ is an \textbf{interior point} if there exists an neighborhood $N$ of $p$ such that $N \subset S$. 
  \end{definition}

  An interior point means that we can always contain the point in $S$ with some ``breathing room." By definition an open set is a set where all of its points are interior points. A set is then said to be open if every point has this breathing room. This can be useful when defining differentiation at a point within an open set, since we can always find a neighborhood to take limits on. 

  Now that we have defined the Euclidean topology, we will prove that the features of topological objects can be reduced to features in $\mathbb{R}^n$. 

  \begin{theorem}[Convexity]
    An open ball is convex in a normed vector space. 
  \end{theorem}
  \begin{proof}
    The normed part is important here, as the properties of the metric is not sufficient. Given $B_r (p)$, $x, y \in B_r (p)$ implies that $||x - p|| < r$ and $||y - p ||<r$. Therefore, 
    \begin{align}
      ||t x + (1 - t)y - p|| & = ||t x - tp + (1 - t) y - (1 - t) p|| \\
      & \leq t ||x - p|| + (1 - t) ||y - p|| \\
      & = t r + (1 - t) r = r 
    \end{align}
  \end{proof}

  What happens if we weaken it to a metric? 

\subsection{Limit Points and Closure} 

  \begin{definition}[Limit Point]
    A point $p \in \mathbb{R}^n$ is a \textbf{limit point} of $S \subset \mathbb{R}^n$ if every punctured neighborhood of $p$ has a nontrivial intersection with $X$.\footnote{The definition just means that if we take a point and draw smaller and smaller circles around it, the circle itself should still overlap with $S$, no matter how small it gets. } The set of all limit points of $S$ is denoted $S^\prime$. 
  \end{definition}

  \begin{theorem}
    Let $A_1, \ldots, A_n$ be a finite collection of sets. Then 
    \[\bigcup_{i=1}^n A_i^\prime = \bigg( \bigcup_{i=1}^n A_i \bigg)^\prime\]
  \end{theorem}
  \begin{proof}
    Let the LHS be $W$ and the RHS be $V$. If $x \in W$, $x \in A_i^\prime$ for some $i$, and so for all $\epsilon > 0$, there exists a $B_\epsilon^\circ (x)$ s.t. 
    \[B_\epsilon^\circ (x) \cap A_i \neq \emptyset \implies B_\epsilon^\circ (x) \cap \bigg( \bigcup_{i=1}^n A_i \bigg) \neq \emptyset\]
    which means that $x \in V$. Now assume that $x \in V$. Then for all $\epsilon > 0$, there exists a $B_\epsilon^\circ (x)$ s.t. 
    \[B_\epsilon^\circ (x) \cap \bigg( \bigcup_{i=1}^n A_i \bigg) \neq \emptyset\]
    which implies that $B_\epsilon^\circ (x) \cap A_i \neq \emptyset$ for some $i$, which means that $x \in A_i^\prime \subset W$. 
  \end{proof}

  A closed set can be defined in many equivalent ways for arbitrary topological spaces. The more general proof is done in topology, but we still prove it in the context of analysis. 

  \begin{definition}[Closed Set]
    A \textbf{closed set} $S \in \mathbb{R}^n$ is a set that contains all of its limit points. 
  \end{definition}

  \begin{theorem}[Alternative Definition of Closed Set]
    A set $S$ is closed iff $S^c$ is open. 
  \end{theorem}
  \begin{proof}
    We prove both ways: 
    \begin{enumerate}
      \item ($\rightarrow$) Given that $S$ is closed, then let $x \in S^c$. $x$ is not a limit point of $S$ since if it were, then it would be in $S$, and so there exists a punctured open neighborhood $B_\epsilon^\circ (x)$ of $x$ s.t. $S \cap B_\epsilon^\circ (x) = \emptyset$. Since $x \not\in S$, we also have $S \cap B_\epsilon (x) = \emptyset$, which implies that $B_\epsilon (x) \subset S^c$. Since for every $x \in S^c$, there exists a $B_\epsilon (x) \subset S^c$, $S^c$ is open. 

      \item ($\leftarrow$) For simplicity, it suffices to prove if $S$ open, then $S^c$ is closed. Given that $S$ is open, we have for every $x \in S$, there exists $B_\epsilon (x) \subset S$, which implies that $B_\epsilon (x) \cap S^c = \emptyset$. Since there exists an $B_\epsilon (x)$ that does not contain points in $S^c$, $x$ cannot be a limit point of $S^c$, and so there exists no limit points of $S^c$ in $S$. Therefore, all limit points of $S^c$ are in $S^c$, proving that $S^c$ is closed.  
    \end{enumerate}
  \end{proof}

  \begin{theorem}
  We have the following topological properties: 
  \begin{enumerate}
      \item For any collection $\{F_\alpha\}_\alpha$ of closed sets, $\cap_\alpha F_\alpha$ is closed. 
      \item For any finite collection $F_1, \ldots, F_n$ of open sets, $\cup_{i=1}^n F_i$ is closed. 
  \end{enumerate}
  \end{theorem}
  \begin{proof}
  Listed. 
  \begin{enumerate}
      \item Let $x$ be a limit point of $\cap_\alpha F_\alpha$, and we want to show that $x \in \cap_\alpha F_\alpha$. By definition of limit points, for every $\epsilon > 0$, we have 
      \[B_\epsilon (x) \cap \bigg( \bigcap_\alpha F_\alpha \bigg) \]
      which means that $B_\epsilon (x) \cap F_\alpha \neq \emptyset$ for all $\alpha$. This means that $x$ is a limit point for every $F_\alpha$, and since they are all closed, $x \in F_\alpha$ for all $\alpha$, which implies that $x \in \cap_\alpha F_\alpha$. 
  \end{enumerate}
  \end{proof}

  We can intuitively see a few properties about this. First, a finite set $S$ of points does not have any limit points, since if we draw small enough circles around a $p \in S$, then at some point the circle will not contain any more points (remember that we're talking about deleted neighborhoods). Following this, we can deduce that a limit point must always have an infinite number of points close to it, as in no matter how small the circle gets, there are always an infinite number of points contained within that circle. This also means that if $p$ is a limit point, then we can construct a sequence of points in $S$ that converges to $p$, since every open ball with smaller and smaller radii will still have points in $S$.

  \begin{theorem}
    If $p$ is a limit point of $S$, then every neighborhood of $p$ contains infinitely many points of $S$. The converse is also true trivially. 
  \end{theorem}
  \begin{proof}
    Assume $p$ is a limit point and that there exists a finite number of points within a deleted neighborhood $B_r^\circ (p)$. Then, we can enumerate them $p_1, p_2, \ldots, p_n$ by their distances to $p$, with 
    \begin{equation}
      d(p_1, p) \leq d(p_2, p) \leq \ldots \leq d(p_n, p)
    \end{equation}
    Since $p_1 \neq p$, we have $d(p_1, p) > 0$ and so, we can choose an $0 < \epsilon < d(p_1, p)$ s.t. $B_\epsilon^\circ (p)$ does not contain any of the $p_i$'s. This neighborhood does not contain any elements of $S$ and so $p$ is not a limit point. 
  \end{proof}

  \begin{corollary}
    A finite set has no limit points. 
  \end{corollary}
  \begin{proof}
    If $S$ is a finite set, then every neighborhood of every point $p$ in $\mathbb{R}^n$ will have at most finite points, which, by the previous theorem, is not a limit point. 
  \end{proof}

  We show a very useful result that will make things much more convenient when proving the following theorems and exercises. This is quite intuitive, since it shows that the limit points of a finite union of sets is the same as the finite union of the limit points of each set. This is clearly not true for infinite unions: 
  \begin{enumerate}
    \item Look at the countable set $\mathbb{Q} \subset \mathbb{R}$. Each $\{q\}^\prime = \emptyset$, but $\mathbb{Q}^\prime = \mathbb{R}$. 
    \item Look at the uncountable set $\mathbb{R}$. Each $\{x \in \mathbb{R}\}^\prime = \emptyset$, but $\mathbb{R}^\prime = \mathbb{R}$. 
  \end{enumerate}

  Now, we give two more definitions for convenience of deriving open and closed sets from any arbitrary set. 

  \begin{definition}[Closure]
    Given a set $S$, let the set of all limit points of $S$ be denoted $S^\prime$. The \textbf{closure} of $S$ is the set $\overline{S} = S \cup S^\prime$. It is the smallest closed set that contains $S$. 
  \end{definition}

  \begin{definition}[Interior]
    Given a set $S$, the \textbf{interior} of $S$ is denoted $S^\circ$, the set of all interior points of $S$. It is the largest open set that is within $S$. 
  \end{definition}

  \begin{theorem}
    Let $E$ be a nonempty set of real numbers which is bounded above. Let $y = \sup{E}$. Then $y \in \overline{E}$. Hence $y \in E$ if $E$ is closed. 
  \end{theorem}
  \begin{proof}
    Assume that $y$ is not a limit point of $E$. Then, there exists some $\epsilon > 0$ s.t. $(y - \epsilon, y + \epsilon)$ does not intersect with $E$. This means that $y - \epsilon$ is an upper bound of $E$, and so $y$ is not the supremum. 
  \end{proof}

  \begin{theorem}
    If $X$ is a metric space and $E \subset X$, then 
    \begin{enumerate}
      \item $\overline{E}$ is closed. 
      \item $E = \overline{E}$ if and only if $E$ is closed. 
      \item $\overline{E} \subset F$ for every closed set $F \subset X$ such that $E \subset F$. That is, if $E \subset F$ closed, then ``increasing" the size of $E$ to its closure will not make it greater than $F$. 
    \end{enumerate}
  \end{theorem}
  \begin{proof}
    Listed. 
    \begin{enumerate}
      \item Let $x$ be a limit point of $\overline{E}$. Then, for every $\epsilon > 0$, we have $B_\epsilon (x) \cap \overline{E} \neq \emptyset$, which means that either $B_\epsilon (x) \cap E \neq \emptyset$ (in which case $x \in E^\prime \implies x \in \overline{E}$ and we are done) or $B_\epsilon (x) \cap E^\prime \neq \emptyset$. We wish to prove that in the latter case, $x$ being a limit point of $E^\prime$ still implies that $x$ is a limit point of $E$. Since $B_\epsilon (x) \cap E^\prime \neq \emptyset$, there must exist a $y \in B_\epsilon (x) \cap E^\prime$. Since $y \in E^\prime$, we can construct an open ball $B_\delta (y)$ containing elements of $E$, and since $B_\epsilon (x)$ is open, we can contain $B_\delta (y)$ entirely within $B_\epsilon (x)$. Therefore, 
      \[B_\delta (y) \cap E \neq \emptyset \implies B_\epsilon (x) \cap E \neq \emptyset\]
      therefore, $x \in E^\prime \implies x \in \overline{E}$. 

      \item If $E$ is closed, then $E^\prime \subset E \implies \overline{E} = E \cup E^\prime = E$. If $E = \overline{E} = E \cup E^\prime$, then $E^\prime \subset E \implies E$ is closed. 

      \item Since $E \subset F$, it suffices to prove that $E^\prime \subset F$. Consider a limit point $x$ of $E$. Then every punctured open neighborhood of $x$ satisfies $B_\epsilon^\circ (x) \cap E \neq \emptyset$. But since $E \subset F$, we have 
      \[B_\epsilon^\circ (x) \cap F \neq \emptyset\]
      and so $x$ is also a limit point of $F$. But since $F$ is closed, $x \in F$. Therefore, $\overline{E} = E \cup E^\prime \subset F$. 
    \end{enumerate}
  \end{proof}

  The first two statements (1) and (2) imply the following. 

  \begin{corollary}
    The closure of the closure of $E$ is equal to the closure of $E$. 
  \end{corollary}
  \begin{proof}
    We know that $\overline{\overline{E}} \supset \overline{E}$, so we must prove that $\overline{\overline{E}} \subset \overline{E}$, which is equivalent to proving that $\overline{E}^\prime \subset \overline{E}$. Let $x \in \overline{E}^\prime$, i.e. is a limit point of $\overline{E}$. Then, for every $\epsilon > 0$, we have $B_\epsilon (x) \cap \overline{E} \neq \emptyset$. Pick a point $y$ from this intersection, and since $B_\epsilon (x)$ is open, we can construct an open ball $B_\delta (y)$ fully contained in $B_\epsilon (x)$. Since $y \in \overline{E}$, $y$ is a limit point of $E$, which implies 
    \begin{equation}
      B_\delta (y) \cap E \neq \emptyset \implies B_\epsilon (x) \cap E \neq \emptyset
    \end{equation}
    and therefore $x$ is a limit point of $E$, $x \in \overline{E}$. 
  \end{proof}

\subsection{Compactness}

  \begin{definition}[Open Cover]
    An \textbf{open cover} of a set $E$ in a metric space $X$ is a collection $\{G_\alpha\}$ of open subsets of $X$ such that $E \subset \cup_\alpha G_\alpha$. 
  \end{definition}

  \begin{definition}[Compact Set]
    A subset $S$ of a metric space $X$ is said to be \textbf{compact} if every open cover of $S$ contains a finite subcover. 
  \end{definition}

  While openness behaves differently depending on its embedding space, compactness stays constant. Therefore, we don't have to worry about talking about which space a compact set is embedded in. 

  \begin{theorem}[Compactness is Preserved Under Subspace Topology]
    Suppose $K \subseteq Y \subseteq X$. Then $K$ is compact relative to $X$ if and only if $K$ is compact relative to $Y$. 
  \end{theorem}
  \begin{proof}
    We can prove bidirectionally. 
    \begin{enumerate}
      \item Suppose that $K$ is compact in $X$. Then given any open cover $\{U_\alpha\}_\alpha$ of $K$, there exists a finite subcover $\{U_i\}_{i}$. Now let there exist an open cover $\{V_\alpha\}$ in $Y$, but every $V_\alpha = U_\alpha \cap Y$ for some $U_\alpha$ open in $X$. Therefore, we can take the finite subcover $\{V_i = V_i \cap Y\}_i$. 

      \item Suppose that $K$ is compact in $Y$. Then given any open cover $\{V_\alpha\}$ of $K$, there exists a finite subcover $\{V_i\}_i$. Now let there exist an open cover $\{U_\alpha\}$ in $X$. Then we set $\{V_\alpha = U_\alpha \cap Y\}_\alpha$, which has a finite subcover $\{V_i = U_i \cap Y\}$, and therefore we can take $\{U_i\}$ as our finite subcover in $X$. 
    \end{enumerate}
  \end{proof}

  \begin{theorem}
    A finite union of compact sets is compact. 
  \end{theorem}
  \begin{proof}
    It suffices to prove for two sets $A, B$ by induction. Take an arbitrary cover $\mathscr{L}$ of $A \cup B$. Then $\mathscr{L}$ is a cover of $A$, so it has a finite subcover $\mathscr{F} \subset \mathscr{L}$. It is also a cover of $B$, so it has a finite subcover $\mathscr{G} \subset \mathscr{L}$. Therefore, $\mathscr{F} \cup \mathscr{G} \subset \mathscr{L}$ is a cover of $A \cup B$, and since it is the union of finite covers, it is finite. 
  \end{proof}

  As we will see in the following theorems, compact sets behave well with closed sets. In fact, compactness is in a form a stronger notion than closedness. 

  \begin{theorem}
    Compact subsets of metric spaces are closed. 
  \end{theorem}
  \begin{proof}
    We would like to show that if $A$ is compact in $X$, then $A^c$ is open. What we would like to do is if we have some $x \in A^c$, then we must prove that there exists some open set $B_\epsilon (x)$ that is disjoint with $A$. For every point $a \in A$, we can construct an open balls $V_a = B_{d(x, a)/2} (a)$ and $U_a = B_{d(x, a)/2} (x)$. We know that if $y \in B_{d(x, a)/2}(a)$, then assuming $y \in B_{d(x, a)/2} (x)$ will give
    \begin{equation}
      d(x, a) \leq d(x, y) + d(y, a) < \frac{d(x, a)}{2} + \frac{d(x, a)}{2} = d(x, a)
    \end{equation}
    which is absurd. 
    Since $\{V_a\}_{a \in A}$ forms an open covering of $A$, then by compactness we can take a finite subcover $V_{a_1}, \ldots, V_{a_n}$, along with the respective neighborhoods of $x$ $U_{a_1}, \ldots, U_{a_n}$. Since we have established 
    \begin{equation}
      V_{a_i} \cap U_{a_i} = \emptyset \implies \bigcap_{i=1}^n V_{a_i} \cap \bigg( \bigcup_{i=1}^n U_{a_i} \bigg) = \emptyset
    \end{equation}
    and since $\cap_{i=1}^n V_{a_i}$ is open (as it is the intersection of open sets) and disjoint from an open cover of $A$ and hence from $A$, we have proved that $A^c$ is open, and so $A$ is closed. 
  \end{proof}

  \begin{theorem}
    Closed subsets of compact sets are compact. 
  \end{theorem}
  \begin{proof}
    Let $C \subset K \subset X$ with $K$ compact and $C$ closed. Then let $\{U_\alpha\}$ be an open cover of $C$. Then $C^c$ is open in $X$, and so $\{U_\alpha\} \cup \{C^c\}$ is an open cover of $K$, so it has a finite subcover $S$. 
    \begin{enumerate}
      \item If $C^c \not\in S$, then we have a finite subcover of $C$. 
      \item If $C^c \in S$, then we can take the element out to get a finite subcover of $C$.  
    \end{enumerate}
    Therefore we have constructed a way to make a finite subcover. $C$ is compact. 
  \end{proof}

  \begin{corollary}
    If $F$ is closed and $K$ is compact, then $F \cap K$ is compact. 
  \end{corollary}

  The general notion of compactness\footnote{According to Terry Tao, a compact set is "small," in the sense that it is easy to deal with. While this may sound counterintuitive at first, since $[0,1]$ is considered compact while $(0,1)$, a subset of $[0,1]$, is considered noncompact. More generally, a set that is compact may be large in area and complicated, but the fact that it is compact means we can interact with it in a finite way using open sets, the building blocks of topology. That finite collection of open sets makes it possible to account for all the points in a set in a finite way. This is easily noticed, since functions defined over compact sets have more controlled behavior than those defined over noncompact sets. Similarly, classifying noncompact spaces are more difficult and less satisfying. } for topological spaces is not needed for analysis. Rather, we make use of the following theorem which allows us to focus on the compactness of subsets in Euclidean spaces $\mathbb{R}^n$. 

  \begin{theorem}[Heine-Borel]
    Let $E \subset \mathbb{R}^k$. The following are equivalent. 
    \begin{enumerate}
      \item $E$ is closed and bounded 
      \item $E$ is compact. 
      \item Every infinite subset of $E$ has a limit point in $E$. 
    \end{enumerate}
  \end{theorem}

  \begin{example}
    An open set in $\mathbb{R}^2$ is not compact. Take the open rectangle $ R = (0,1)^2 \subset \mathbb{R}^2$. There exists an infinite cover of $R$
    \[R = \bigcup_{n=0}^\infty \big(0,1\big) \times \bigg( 0, \frac{ 2^{n+1} - 1}{2^{n+1}} \bigg) \]
    that does not have a finite subcover. 
  \end{example}

  Clearly, the limit point of an open set is its boundary points. Note that a sequence of points can also have a limit point. 

  \begin{theorem}[Bolzano-Weierstrass Theorem]
    Every bounded sequence in $\mathbb{R}^n$ has a limit point. 
  \end{theorem}
  \begin{proof}
    The fact that the infinite sequence is bounded means that there exists some closed subset $I \in \mathbb{R}^n$ that contains all point of the sequence. By definition $I$ is compact, so by the Heine-Borel theorem, every cover of $I$ has a finite subcover. 

    Now, assume that there exists an infinite sequence in $I$ that is not convergent, i.e. has no limit point. Then, each point $x_i \in I$ would have a neighborhood $U(x_i)$ containing at most a finite number of points in the sequence. We can define $I$ such that the union of the neighborhoods is a cover of $I$. That is, 
    \[I \subset \bigcup_{i=1}^\infty U(x_i)\]
    However, since every $U(x_i)$ contains at most a finite number of points, we must have an infinite open neighborhoods to cover $I \implies$ we cannot have a finite subcover. This contradicts the fact that $I$ is compact. 
  \end{proof}

  In fact, compactness actually implies completeness. 

  \begin{theorem}
    Compact metric spaces are complete. 
  \end{theorem} 

  So far, we've been pretty abstract about compact sets. In general, it's pretty easy to prove that a set is not compact. We just need to find one example of an open cover that does not have a finite subcover. To prove that set \textit{is} compact, we must show that for \textit{every} open cover, we can get a finite subcover. This sounds quite daunting, but here is a special theorem that can start us off, and the theorems above allow us to construct more compact sets. We will need the third interpreation of completeness of the reals: nested intervals completeness.

  \begin{theorem}[Nested Intervals Theorem]
    If $\{I_n = [a_n, b_n]\}$ in $\mathbb{R}$ is a sequence of nested closed intervals, then 
    \begin{equation}
      \bigcap_{i=1}^\infty I_n \neq \emptyset
    \end{equation}
  \end{theorem}
  \begin{proof}
    Note that $\{a_n\}$ is bounded above by $b_1$. Therefore by LUB property it must have a supremum, call it $x = \sup_n \{a_n\}$. Then, we see that $a_n \leq x \leq b_n$ for all $n$, and so $x$ is in the intersection. 
  \end{proof}

  \begin{corollary}
    Every closed interval is compact. 
  \end{corollary}
  \begin{proof}
    Let $I = [a, b]$. Then if $x, y \in I$, $|x - y| \leq b - a = \delta$. Now by contradiction, suppose that there exists an open cover $\{U_\alpha\}$ of $I$ which contains no finite subcover of $I$. Then letting $c = (a + b)/2$, at least one of the two intervals $[a, c], [c, b]$ cannot have a finite subcovering (otherwise their finite union can be covered). WLOG let it be $[a, c]$. We keep subdividing and get the sequence of nested intervals. 
    \begin{equation}
      I \supset I_1 \supset I_2 \supset \ldots 
    \end{equation}
    We know that $I_n$ is not covered by any finite subcollection of $\{U_\alpha\}$ and if $x, y \in I_n$, then $|x - y| < 2^{-n} \delta$. From the nested intervals theorem, there exists a point $z$ lying in every $I_n$. There must then be an open neighborhood $U_z$ in the open cover, and by definition of openness there exists a $\epsilon > 0$ s.t. $z \in B_\epsilon (z) \subset U_z$. By the Archimidean property, we can set $n$ so large that $2^{-n} \delta < \epsilon$ and this means that $B_\epsilon (z) \supset I_n$, which contradicts the fact that $I_n$ is not covered by a finite subcollection. Therefore $I$ is compact. 
  \end{proof}

\subsection{Connectedness}

  \begin{definition}[Separate, Connected Sets]
    Two subsets $A$ and $B$ of a metric space $X$ are said to be \textbf{separated} if both $A \cap \overline{B}$ and $\overline{A} \cap B$ are empty, i.e. if no point of $A$ lies in the closure of $B$ and no point of $B$ lies in the closure of $A$. 
  \end{definition}

  \begin{example}
    It is clear that separate sets imply disjointness. However, this is not true for the other way around. 
    \begin{enumerate}
      \item $(0, 1)$ and $[1, 2)$ are disjoint but not separate. 
      \item The rationals and irrationals are disjoint, but not separate. 
    \end{enumerate}
  \end{example}

  \begin{theorem}
  A subset $E$ of the real line $\mathbb{R}$ is connected if and only if it has the following property: if $x \in E, y \in E$ and $x < z < y$, then $z \in E$. 
  \end{theorem}
  \begin{proof}

  \end{proof}

\subsection{Separability}

\subsection{Perfect Sets}

  \begin{definition}[Perfect Sets]
    A set $P$ is perfect if it is closed and all of its points are limit points of $P$. In other words, the limit points of $P$ and $P$ itself coincide. 
    \begin{equation}
      P^\prime = P
    \end{equation}
  \end{definition}

  \begin{theorem}
    Let $P$ be a nonempty perfect set in $\mathbb{R}^k$. Then $P$ is uncountable. 
  \end{theorem}


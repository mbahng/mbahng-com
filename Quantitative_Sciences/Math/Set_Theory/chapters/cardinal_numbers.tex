\section{Cardinal Numbers} 

  Now we would like to rigorously construct the intuitive concept of the ``size'' of a set. Before we can even label any set with such a number, which we call the \textit{cardinal number}, we can with our current tools compare the sizes, or \textit{cardinalities}, of sets. This is a bit counterintuitive since we're able to \textit{compare} sizes but not know what the sizes actually are! 

  \begin{definition}[Equipotence]
    Two sets $A$ and $B$ are \textbf{equipotent}, written $A \approx B$, if there exists a bijective map $f: A \rightarrow B$. This implies that their cardinalities are the same: $|A| = |B|$. It has the following properties: 
    \begin{enumerate}
      \item Reflexive: $A \approx A$
      \item Symmetric: $A \approx B$ implies $B \approx A$
      \item Transitive: $A \approx B$ and $B \approx C$ implies $A \approx C$
    \end{enumerate}
  \end{definition} 

  Now equipotence behaves like an equivalence relation, though we can't define an equivalence class on the nonexistent set of all sets. 

  \begin{theorem}
    The following hold for equipotence. 
    \begin{enumerate}
      \item $A \approx A$. 
      \item $A \approx B \implies B \approx A$. 
      \item $A \approx B, B \approx C \implies A \approx C$. 
    \end{enumerate}
  \end{theorem}
  \begin{proof}
    Listed. 
    \begin{enumerate}
      \item Take the identity map. 
      \item Take the inverse, which is well defined under bijection. 
      \item Take the composition of bijections which we proved is a bijection. 
    \end{enumerate}
  \end{proof}

  \begin{definition}[Comparison of Cardinality]
    The \textbf{cardinality} of $A$ is said to be less than or equal to the cardinality of $B$, denoted $|A| \leq |B|$) if there is a one-to-one mapping of $A$ into $B$. 
  \end{definition} 

  Just like how equipotence behaves like an equivalence relation, comparison of cardinality behaves like an ordering on the collection of equivalence classes. 

  \begin{theorem}
    The following hold. 
    \begin{enumerate}
      \item $|A| \leq |A|$. 
      \item $|A| \leq |B|, |A| = |C| \implies |C| \leq |B|$ 
      \item $|A| \leq |B|, |B| = |C| \implies |A| \leq |C|$. 
      \item $|A| \leq |B|, |B| \leq |C| \implies |A| \leq |C|$. 
    \end{enumerate}
  \end{theorem}

  However, we still have to establish antisymmetry, which---unlike the other properties---is a major theorem. 

  \begin{theorem}[Cantor-Bernstein]
    If $|A| \leq |B|$ and $|B| \leq |A|$, then $|A| = |B|$. 
  \end{theorem}
  \begin{proof}
    
  \end{proof} 

  So far, we have proved many properties of cardinality without actually defining what cardinality is. We don't actually need to define such a thing, but for convenience and convention we do. The following is really a theorem, which can be proved by the axiom of choice, but we introduce it as a definition. 

  \begin{definition}[Cardinal Numbers]
    There exists sets called \textbf{cardinal numbers} with the property that for every set $X$, there is a unique cardinal $|X|$, called the \textit{cardinality of $X$}, and sets $X$ and $Y$ are equipotent if and only if $|X|$ is equal to $Y$.\footnote{However, there does \textit{not} exist a set containing all the cardinals!}
  \end{definition} 

  We should intuitively see that the natural numbers can be treated as a subset of the cardinals, but it is not sufficient since by the axiom of infinity, there exists infinite sets $X$ in which $|X| \not\in \mathbb{N}$. We will start off with finite set and continue onto finite sets. 

\subsection{Finite Sets} 

  \begin{definition}[Finite Set]
    A set $S$ is \textbf{finite} if it is equipotent to some natural number $n \in \mathbb{N}$.\footnote{Note that $n$ is a set.} We define $|S| = n$ and say $S$ has $n$ elements. 
  \end{definition} 

  \begin{definition}[Infinite Set]
    A set $S$ that is not finite is called \textbf{infinite}. 
  \end{definition}

  It follows that the cardinal numbers of finite sets are the natural numbers, and natural numbers are themselves finite sets, meaning that $|n| = n$ for all $n \in \mathbb{N}$. It remains to prove that $|S|$ for a finite set is unique. 

  \begin{lemma}[No Proper Inclusion]
    If $n \in \mathbb{N}$, then there is no bijective mapping of $n$ onto a proper subset $X \subsetneq n$. 
  \end{lemma}
  \begin{proof}
    We do strong induction on $n$. 
  \end{proof}

  \begin{corollary}
    The following is immediate. 
    \begin{enumerate}
      \item If $n \neq m$, then there is no bijective mapping $f: n \to m$. 
      \item If $|S| = n$ and $|S| = m$, then $n = m$. 
      \item $\mathbb{N}$ is infinite. 
      \item 
    \end{enumerate}
  \end{corollary}

  \begin{theorem}[Preservation of Finite Cardinality]
    Finiteness is preserved under many set operations and maps. 
    \begin{enumerate}
      \item \textit{Subset}. If $X$ is finite and $Y \subset X$, then $Y$ is finite and $|Y| \leq |X|$. 
      \item \textit{Function}. If $X$ is finite, and $f$ is a function, then $f(X)$ is finite with $|f(X)| \leq |X|$. 
      \item \textit{Binary Union}. If $X$ and $Y$ are finite, $X \cup Y$ is finite with $|X \cup Y| \leq |X| + |Y|$.  If they are disjoint, then $|X \cup Y| = |X| + |Y|$. 
      \item \textit{Arbitrary Union}. If a set of sets $S$ is finite and every $X \in S$ is finite, then $\cup S$ is finite. 
      \item \textit{Power Set}. If $X$ is finite, then $\mathcal{P}(X)$ is finite. 
      \item \textit{Cartesian Product}. If $X_1, \ldots, X_n$ is finite, then $\prod X_i$ is finite. 
    \end{enumerate}
  \end{theorem}

\subsection{Countable Sets}

  Now we go into our first class of infinite sets, which we know are the naturals. 

  \begin{definition}[Aleph Null]
    The cardinal number of the naturals $\mathbb{N}$ is denoted $\aleph_0 = |\mathbb{N}|$. 
  \end{definition}

  \begin{definition}[Countable Set]
    \label{def:countable}
    A set $S$ is \textbf{countable} if $|S| = \aleph_0 = |\mathbb{N}|$. A set is \textbf{at most countable} if $|S| \leq |\mathbb{N}|$. 
  \end{definition} 

  At this point, we may already be familiar with the fact that $\mathbb{Q}$ is countable and $\mathbb{R}$ is uncountable. Let us formalize the statement that a countable infinity is the smallest type of infinity. We can show this by taking a countable set and showing that every infinite subset must be countable. If it was not countable (e.g. uncountable), then this would mean that a countable set contains some other class of infinite sets, which means that \textit{that} infinite set would be ``smaller'' than $\aleph_0$. 

  \begin{theorem}
    \label{countable smallest}
    Every infinite subset of a countable set $A$ is countable. 
  \end{theorem} 

  Now that we've established that $\aleph_0$ is the smallest infinity, let's try to see which set operations preserve this cardinality. 

  \begin{theorem}[Countable Union is Countable]
    An at most countable union of countable sets is countable. 
  \end{theorem}

  \begin{theorem}[Finite Product is Countable]
    A finite Cartesian product of countable sets is countable. 
  \end{theorem}
  \begin{proof}
    By induction it suffices to prove that if $X$ and $Y$ are countable, then $X \times Y$ is countable. We can find an enumeration by going across the diagonals. 
  \end{proof} 

  \begin{theorem}[Further Properties] 
    \begin{enumerate}
      \item The set of all finite sequences in countable $X$ is countable. 
      \item The set of all finite subsets of a countable set is countable. 
      \item An equivalence relation on a countable set has at most countably many equivalence classes. 
    \end{enumerate}
  \end{theorem} 

\subsection{Cardinal Arithmetic} 

  So far, the only sets we have to work with is the naturals, which are countable. To extend this to other infinities, we would want to use these cardinal numbers to hopefully create larger cardinals. Therefore, we need to define operations on cardinals. First, we need a lemma to prove that our definitions are consistent. 

  \begin{lemma} 
    If $A, B, A^\prime, B^\prime$ are sets such that $|A| = |A^\prime|$ and $|B| = |B^\prime|$, with $A \cap B = A^\prime \cap B^\prime = \emptyset$, then $|A \cup B| = |A^\prime \cup B^\prime|$. 
  \end{lemma}

  \begin{definition}[Addition of Cardinals]
    Given cardinals $\kappa, \lambda$ where $\kappa = |A|$ and $\lambda = |B|$ for some sets $A, B$ satisfying $A \cap B = \emptyset$, we can define 
    \begin{equation}
      \kappa + \lambda \coloneqq |A \cup B|
    \end{equation}
  \end{definition}

  Now to define multiplication as the cardinality of Cartesian products, we also prove consistency. 

  \begin{lemma} 
    If $A, B, A^\prime, B^\prime$ are sets such that $|A| = |A^\prime|$ and $|B| = |B^\prime|$, then $|A \times B| = |A^\prime \times B^\prime|$. 
  \end{lemma}

  \begin{definition}[Multiplication of Cardinals]
    Given cardinals $\kappa, \lambda$ where $\kappa = |A|$ and $\lambda = |B|$ for some sets $A, B$, we can define 
    \begin{equation}
      \kappa \cdot \lambda \coloneqq |A \times B| 
    \end{equation}
  \end{definition}

  These operations behave pretty nicely, as outlined below. 

  \begin{theorem}[Commutativity, Associativity]
    Given cardinals $\kappa, \lambda, \mu$, we have 
    \begin{enumerate}
      \item \textit{Commutativity of Addition}. $\kappa + \lambda = \lambda + \kappa$. 
      \item \textit{Associativity of Addition}. $\kappa + (\lambda + \mu) = (\kappa + \lambda) + \mu$. 
      \item \textit{Commutativity of Multiplication}. $\kappa \cdot \lambda = \lambda \cdot \kappa$
      \item \textit{Associativity of Multiplication}. $\kappa \cdot (\lambda \cdot \mu) = (\kappa \cdot \lambda) \cdot \mu$. 
      \item \textit{Distributivity}. $\kappa \cdot (\lambda + \mu) = \kappa \cdot \lambda + \kappa \cdot \mu$. 
    \end{enumerate}
  \end{theorem}

  \begin{theorem}[Inequalities]
    We have 
    \begin{enumerate}
      \item $\kappa \leq \kappa + \lambda$. 
      \item $\kappa_1 \leq \kappa_2, \lambda_1 \leq \lambda_2 \implies \kappa_1 + \lambda_1 \leq \kappa_2 + \lambda_2$. 
      \item $\kappa \leq \kappa \cdot \lambda$ if $\lambda > 0$. 
      \item $\kappa_1 \leq \kappa_2, \lambda_1 \leq \lambda_2 \implies \kappa_1 \cdot \lambda_1 \leq \kappa_2 \cdot \lambda_2$. 
      \item $\kappa + \kappa = 2 \cdot \kappa$. 
      \item $\kappa + \kappa \leq \kappa \cdot \kappa$ whenever $\kappa \geq 2$. 
    \end{enumerate}
  \end{theorem} 

  Now these don't actually help us create larger infinities, and to do this we define exponentiation of cardinal numbers. 

  \begin{lemma} 
    if $|A| = |A^\prime|$ and $|B| = |B^\prime|$, then $|A^B| = |(A^\prime)^{B^\prime}|$. 
  \end{lemma}

  Therefore the definition is consistent. 

  \begin{definition}[Exponentiation of Cardinals]
    Given cardinals $\kappa, \lambda$ where $\kappa = |A|$ and $\lambda = |B|$ for some sets $A, B$, we can define 
    \begin{equation}
      \kappa^{\lambda} = |A^B|
    \end{equation}
  \end{definition}

  \begin{theorem}[Properties of Exponentiation]
    The following holds. 
    \begin{enumerate}
      \item $\kappa \leq \kappa^\lambda$ if $\lambda > 0$. 
      \item $\lambda \leq \kappa^\lambda$ if $\kappa > 1$.  
      \item $\kappa_1 \leq \kappa_2, \lambda_1 \leq \lambda_2 \implies \kappa_1^{\lambda_1} \leq \kappa_2^{\lambda_2}$ 
      \item $\kappa \cdot \kappa = \kappa^2$. 
      \item $\kappa^{\lambda + \mu} = \kappa^\lambda \cdot \kappa^\mu$. 
      \item $(\kappa^{\lambda})^\mu = \kappa^{\lambda \cdot \mu}$. 
      \item $(\kappa \cdot \lambda)^\mu = \kappa^\mu \cdot \lambda^\mu$. 
    \end{enumerate}
  \end{theorem} 

  Now that we've established a useful collection of properties of cardinals, we are ready to extend beyond $\aleph_0$ and $2^{\aleph_0}$. 

  \begin{theorem}[Cantor's Theorem] 
    Actually the first part is known as Cantor's theorem, but the second is also used often together. 
    \begin{enumerate}
      \item For every set $X$, $|X| < |\mathcal{P}(X)|$. 
      \item For every set $X$, $|\mathcal{P}(X)| = 2^{|X|}$.\footnote{In terms of cardinal numbers, we have $\kappa < 2^\kappa$ for all cardinal $\kappa$.}
    \end{enumerate}
  \end{theorem} 

  Therefore, given any collection of sets, we can always find a set that has a strictly greater cardinality. Despite the existence of each cardinal number, you would be surprised to find that there exists \textit{no} set containing all the cardinal numbers! 

  \begin{theorem}[Set of Cardinal Numbers Does Not Exist]
    The set of all cardinal numbers does not exist. 
  \end{theorem}
  \begin{proof}
    Assume that $C$ was the set of all cardinals. Then $\cup C$ would be a cardinal exceeding all the cardinals in $C$, which is a contradiction. 
  \end{proof}

\subsection{Uncountable Sets} 

  \begin{definition}[Uncountable Set]
    A set $S$ is \textbf{uncountable} if it is infinite and not countable. 
  \end{definition}

  We know that $\mathbb{Q}$ is countable with cardinality $\aleph_0$ and $\mathbb{R}$ is uncountable with cardinality $2^{\aleph_0}$. The question of whether there exists an intermediate infinity $\kappa$ such that $\aleph_0 < \kappa < 2^{\aleph_0}$ is still not fully resolved today. 

  \begin{theorem}[Continuum Hypothesis]
    There is no uncountable cardinal number $\kappa$ such that $\kappa < 2^{\aleph_0}$. 
  \end{theorem}

  Now, how do we prove that a set is uncountable? We can't really use the contrapositive of Theorem $\ref{countable smallest}$, since to prove that an arbitrary set $A$ is uncountable, then we must find an infinite subset that is not countable. But now we must prove that this subset itself is not countable, too! Therefore, we can use this theorem. 

  \begin{theorem}
    Given an arbitrary set $A$, if every countable subset $B$ is a proper subset of $A$, then $A$ is uncountable. 
  \end{theorem}
  \begin{proof}
    Assume that $A$ is countable. Then $A$ itself is a countable subset of $A$, but by the assumption, $A$ should be a proper subset of $A$, which is absurd. Therefore, $A$ is uncountable. 
  \end{proof}

  \begin{theorem}
    The set of all functions $f: \mathbb{R} \to \mathbb{R}$ has cardinality $2^{2^{\aleph_0}} > 2^{\aleph_0}$. 
  \end{theorem}

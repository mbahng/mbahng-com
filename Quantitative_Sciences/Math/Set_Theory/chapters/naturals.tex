\section{The Naturals} 

\subsection{Axiom of Infinity}

  Now we attempt to try and construct the natural numbers. We start with existence of the empty set $S_0 = \emptyset = \{\}$. Now given $S_n$, we can define inductively $S_{n+1}$ through a \textit{successor function} $S$ that maps to the ``next number.''
  \begin{equation}
    S_{n+1} = S(W) \coloneqq S_n \cup \{S_n\}
  \end{equation}
  where 
  \begin{enumerate}
    \item $S_n$ exists either by induction on $n$ or from the base case where the axiom of empty set applies. 
    \item $\{S_n\}$ exists through the axiom of pairing. 
    \item $S_n \cup \{S_n\}$ exists through the axiom of union. 
  \end{enumerate} 
  Note that even though we did not talk about what $n+1$ means, we can just treat it as some elements of some indexing set. Great, so we can indeed prove that each element $S_n$ exists. This motivates the following definition. 

  \begin{definition}[Inductive Set]
    A set $I$ is called \textbf{inductive} if 
    \begin{enumerate}
      \item $0 = \{\} \in I$. 
      \item If $n \in I$, then $n + 1 = S(n) = n \cup \{n\} \in I$
    \end{enumerate} 
  \end{definition}

  Therefore, an inductive set contains $0$ and with each element, its sucessor. This makes us motivate the definition of the naturals as an inductive set which contains no other elements but the natural numbers, i.e. it is the smallest inductive set. We can express this minimal property by letting every natural number $x$ be contained in \textit{every} inductive set, leading us to the following definition. 
  \begin{equation}
    \mathbb{N} = \{x \mid x \in I \text{ for every inductive set } I \} 
  \end{equation}
  But note that this does not follow the schema of restricted specification, so we must first take some \textit{existing} inductive set $A$ and define 
  \begin{equation}
    \mathbb{N} = \{x \in A \mid x \in I \text{ for every inductive set } I \} 
  \end{equation} 
  Now the question remains of whether there are any inductive sets at all. The intuitive answer is yes, but with our axioms so far, the existence of infinite sets cannot be proven actually. The reason is that we have started out with a finite set $\emptyset$, and the construction of new sets with our axioms only allows us to create more finite sets. Therefore, while we can construct finite sets with an unbounded number of elements, we can never reach an infinite set. This is why we must axiomatically claim that a finite inductive set exists. 

  \begin{axiom}[Axiom of Infinity]
    An inductive set exists. 
  \end{axiom}

  This allows us to utilize the axiom of restricted specification to construct the primitive form of the naturals. 

  \begin{definition}[Von Neumann Ordinals] 
     The \textbf{Von Neumann ordinals} is the minimal set $X$ satisfying the axiom of infinity. It is the set containing 
    \begin{align*}
      0 & = \{\} = \emptyset \\
      1 & = \{0\} = \{\emptyset\} \\
      2 & = \{0,1\} = \{\emptyset,\{\emptyset\}\} \\
      3 & = \{0,1,2\} = \{\emptyset,\{\emptyset\},\{\emptyset,\{\emptyset\}\}\} \\
      4 & = \{0,1,2,3\} = \{\emptyset,\{\emptyset\},\{\emptyset,\{\emptyset\}\},\{\emptyset,\{\emptyset\},\{\emptyset,\{\emptyset\}\}\}\} \\
      \ldots & = \ldots 
    \end{align*} 
  \end{definition} 

  Now note that the von Neumann ordinals are not the only way to construct such a set. The \textbf{Zermelo ordinals} define the successor function to be $S(w) = \{w\}$ and state that the inductive set with this successor function exists. However, the von Neumann ordinals have the nice property that the cardinality $n$ of the set is precisely the natural number that we would like to identify it with, and the fact that a natural number $n$ contains all naturals $0, \ldots, n-1$ as elements of $n$. 

\subsection{Natural Numbers}

  From this, with a few more structures we can define the naturals as a commutative monoid with respect to both the addition and multiplication operations.

  \begin{definition}[Natural Numbers]
    \label{naturals}
    The \textbf{natural numbers} $\mathbb{N} = \{0, 1, 2, \ldots\}$ is the set of von Neumann ordinals with the following structure. 
    \begin{enumerate}
      \item \textit{Order}. The relation $\leq$ defined as 
      \begin{equation}
        m \leq n \iff m \in n
      \end{equation}
      is an order relation. 

      \item \textit{Addition} can be defined recursively with the successor function as\footnote{note that I am using $\coloneqq$ to denote that this is an \textit{identity}, not an equation to be solved.}
      \begin{align}
        n + 0 & \coloneqq n \\
        n + 1 & \coloneqq S(n) \\  
        n + S(m) & \coloneqq S(n + m)
      \end{align}

      \item \textit{Multiplication} is also defined recursively using the definition of addition.  
      \begin{align}
        n \times 0 & \coloneqq 0 \\
        n \times S(m) & \coloneqq (n \times m) + n
      \end{align}
      which is familiar to the process of adding $m$ to itself $n$ times. 

      \item \textit{Additive Identity} is $0$, following directly from the definition of addition. 
      \item \textit{Multiplicative Identity} is $1$, following directly from the definition of multiplication. 
    \end{enumerate}
  \end{definition} 

  \begin{example}[Order]
    I assert that $3 \leq 4$ since 
    \begin{equation}
      \{0, 1, 2\} = 3 \in \{0, 1, 2, 3 \} = 4
    \end{equation}
  \end{example}

  \begin{example}[Addition]
    To define $5 + 2$, we can see that 
    \begin{align}
      5 + 2 & = 5 + S(1) \\
            & \coloneqq S(5 + 1) \\
            & = S(5 + S(0)) \\
            & \coloneqq S(S(5 + 0)) \\
            & \coloneqq S(S(5)) \\
            & = S(6) \\
            & = 7
    \end{align}
  \end{example}

  \begin{example}[Multiplication]
    To define $4 \times 3$, we apply the recursive definitions 
    \begin{align}
      4 \times 3 & = 4 \times S(2) \\
                 & \coloneqq 4 \times 2 + 4 \\
                 & = 4 \times S(1) + 4 \\
                 & \coloneqq 4 \times 1 + 4 + 4 \\
                 & \coloneqq 4 \times S(0) + 4 + 4 \\
                 & = 4 \times 0 + 4 + 4 + 4 \\
                 & = 0 + 4 + 4 + 4 \\
                 & = 12
    \end{align}
  \end{example}

  \begin{theorem}[Commutative]
    $+$ and $\times$ are commutative in $\mathbb{N}$. 
  \end{theorem}

  \begin{theorem}[Associativity]
    $+$ and $\times$ are associative in $\mathbb{N}$. 
  \end{theorem}

\subsection{Induction}

  \begin{lemma}[Well Ordering Principle]
    Every nonempty subset of $\mathbb{N}$ has a minimal element. 
  \end{lemma} 
  \begin{proof}
    Take a subset $A \subset \mathbb{N}$. 
    \begin{enumerate}
      \item If $0 \in A$, the minimum is $0$. 
      \item Else if $1 \in A$, the minimum is $1$. 
      \item ...
    \end{enumerate} 
  \end{proof}

  We can use this inductive property of natural numbers to prove properties of them. Note that this can only be used to prove for finite (yet unbounded) numbers! 

  \begin{lemma}[Induction Principle]
    Given $P(n)$, a property depending on a natural number $n \in \mathbb{N}$, 
    \begin{enumerate}
      \item if $P(n_0)$ is true for some $n_0 \in \mathbb{N}$, and
      \item if for every $k \geq n_0$, $P(k)$ true implies $P(k+1)$ true, 
    \end{enumerate}
    then $P(n)$ is true for all $n \geq n_0$. 
  \end{lemma}

  \begin{lemma}[Strong Induction Principle]
    Given $P(n)$, a property depending on a positive integer $n$, 
    \begin{enumerate}
      \item if $P(n_0), P(n_0 + 1), \ldots, P(n_0 + m)$ are true for some positive integer $n_0$, and nonnegative integer $m$, and 
      \item if for every $k > n_0 + m, P(j)$ is true for all $n_0 \leq j \leq k$ implies $P(k)$ is true, 
    \end{enumerate}
    then $P(n)$ is true for all $n \geq n_0$. 
  \end{lemma} 

  \begin{theorem}[Equivalence of 3 Principles]
    The well-ordering principle, induction principle, and the strong induction principle are all equivalent. 
  \end{theorem}
  \begin{proof}
    We prove the steps. 
    \begin{enumerate}
      \item \textit{Well Ordering $\implies$ Strong Induction}. 
      \item \textit{Strong Induction $\implies$ Induction}.  
      \item \textit{Induction $\implies$ Well-Ordering}. 
    \end{enumerate}
  \end{proof}

  The idea behind the strong induction principle leads to the proof using infinite descent. Infinite descent combines strong induction with the fact that every subset of the positive integers has a smallest element, i.e. there is no strictly decreasing infinite sequence of positive integers. 

  \begin{theorem}[Infinite Descent]
    Given $P(n)$, a property depending on positive integer, assume that $P(n)$ is false for a set of integers $\mathcal{S}$. Let the smallest element of $\mathcal{S}$ be $n_0$. If $P(n_0)$ false implies $P(k)$ false, where $k < n_0$, then by contradiction $P(n)$ is true for all $n$. 
  \end{theorem} 

\subsection{Sequences and The Recursion Theorem}

  Since we have defined the naturals, we can construct a function that takes in a natural number and outputs to an arbitrary set. This is called a \textit{sequence}. 

  \begin{definition}[Sequence]
    \label{def:sequence}
    Given a nonempty set $X$, a function $f: \mathbb{N} \rightarrow X$ is called a \textit{sequence}. We usually denote it by 
    \begin{equation}
      x_1, x_2, x_3, \ldots
    \end{equation}
    It is often shorthand written as $(x_n)$.\footnote{Note that this is different from $\{x_n\}$, which is considered a \textit{set} and is unordered.} There are two ways to define such a sequence: 
    \begin{enumerate}
      \item \textit{Explicitly}. We denote $x_n = f(n)$ in some closed form. 
      \item \textit{Recursively}. We denote $x_{n+1} = g(x_n, n)$ for all $n \in \mathbb{N}$, with some base case $x_0$. 
    \end{enumerate}
  \end{definition}

  The explicit definition gives us a well-defined sequence, but the recursive definition requires a bit more care. Given an initial condition $x_0$ and some recursive condition (similar to differential equations), such a definition is only well-formulated if there exists such a sequence (or function) that satisfies these two conditions. The recursion theorem proves both the existence and uniqueness of such an infinite sequence. 

  \begin{theorem}[Recursion Theorem]
    For any set $A$, any $a \in A$, and function $g: A \times \mathbb{N} \to A$, there exists a unique infinite sequence $f: \mathbb{N} \to A$ such that 
    \begin{enumerate}
      \item $f(0) = a$. 
      \item $f(n+1) = g(f(n), n)$ for all $n \in \mathbb{N}$. 
    \end{enumerate}
  \end{theorem}

  Sequences are particularly important in topology and analysis, where they are used as a tool to analyze properties of topological spaces or metric spaces. Since we are working in an arbitrary set, we cannot do much more than this.   


\section{Structures} 

\subsection{Algebraic Structures}

  Now sets are very boring when studying by themselves. It is often the case that we \textit{endow} a set with some additional information, which we call \textit{structure}. Consider the familiar integers, which come with certain operations such as $+$ and $\times$, along with the concepts identity elements---namely $0$ and $1$---which allows us to define (additive and multiplicative) inverses. 

  \begin{definition}[Structure]
    A \textbf{structure} on a set is some additional information on the set, such as relations, constants, and operations associated with the set. 
  \end{definition} 

  A lot of these properties depend on what operations you can do with them.  

  \begin{definition}[Operation]
    A \textbf{p-ary operation}\footnote{or called an operation of arity $p$.} $\ast$ on a set $A$ is a map 
    \begin{equation}
      \ast : A^p \longrightarrow A
    \end{equation} 
    where $A^p$ is the $p$-fold Cartesian product of $A$. In specific cases, 
    \begin{enumerate}
      \item If $p = 1$, then $\ast$ is said to be \textbf{unary}. 
      \item If $p = 2$, then $\ast$ is \textbf{binary}. 
    \end{enumerate}
    We can consider for $p > 2$ and even if $p$ is infinite.  
  \end{definition}

  \begin{definition}[Monoid]
    A \textbf{monoid} is a set $S$ with an operation $+$. 
  \end{definition}

  \begin{definition}[Group]
    
  \end{definition}

  \begin{definition}[Ring]
    
  \end{definition}

  \begin{definition}[Field]
    A \textbf{field} is an algebraic structure $(\mathbb{F}, +, \cdot)$ where 
    \begin{enumerate}
      \item $\mathbb{F}$ is an abelian group under $+$, with $0$ being the \textit{additive identity}. 
      \item $\mathbb{F} \setminus \{0\}$ is an abelian group under $\cdot$, with $1$ being the \textit{multiplicative identity}. 
      \item It connects the two operations through the \textit{distributive property}.
      \begin{equation}
        x \cdot (y + z) = x \cdot y + x \cdot z
      \end{equation}
    \end{enumerate}
  \end{definition} 

  \begin{lemma}[Left = Right Distributivity]
    Left and right distributivity are equivalent. 
    \begin{equation}
      x \cdot (y + z) = (y + z) \cdot x
    \end{equation}
  \end{lemma} 
  \begin{proof}
    \begin{align}
      x \cdot (y + z) & = x \cdot y + x \cdot z && \tag{Distributive} \\
                      & = y \cdot x + z \cdot x && \tag{Commutative} \\
                      & = (y + z) \cdot x && \tag{Distributive} 
    \end{align}
  \end{proof} 

  \begin{lemma}[Properties of Addition]
    The properties of addition hold in a field. 
    \begin{enumerate}
      \item If $x + y = x + z$, then $y = z$. 
      \item If $x + y = x$, then $y = 0$. 
      \item If $x + y = 0$, then $y = -x$. 
      \item $(-(-x)) = x$. 
    \end{enumerate}
  \end{lemma}
  \begin{proof}
    For the first, we have 
    \begin{align}
      x + y = x + z & \implies -x + (x + y) = -x + (x + z) && \tag{addition is a function} \\
                    & \implies (-x + x) + y = (-x + x) + z && \tag{$+$ is associative} \\
                    & \implies 0 + y = 0 + z && \tag{definition of additive inverse} \\
                    & \implies y = z && \tag{definition of identity}
    \end{align} 
    For the second, we can set $z = 0$ and apply the first property. For the third, we have 
    \begin{align}
      x + y = 0 & \implies -x + (x + y) = -x + 0 && \tag{addition is a function} \\
                & \implies (-x + x) + y = -x + 0 && \tag{$+$ is associative} \\
                & \implies 0 + y = -x + 0 && \tag{definition of additive inverse} \\
                & \implies y = -x && \tag{definition of identity}
    \end{align}
    For the fourth, we simply follow that if $y$ is an inverse of $z$, then $z$ is an inverse of $y$. Therefore, $-x$ being an inverse of $x$ implies that $x$ is an inverse of $-x$. $-(-x)$ must also be an inverse of $-x$. Since inverses are unique\footnote{This is proved in algebra.}, $x = -(-x)$. 
  \end{proof}

  \begin{lemma}[Properties of Multiplication]
    The properties of multiplication hold in a field. 
    \begin{enumerate}
      \item If $x \neq 0$ and $xy = xz$, then $y = z$. 
      \item If $x \neq 0$ and $xy = x$, then $y = 1$. 
      \item If $x \neq 0$ and $xy = 1$, then $y = x^{-1}$. 
      \item If $x \neq 0$, then $(x^{-1})^{-1} = x$. 
    \end{enumerate}
  \end{lemma}
  \begin{proof}
    The proof is almost identical to the first. Since $x \neq 0$, we can always assume that $x^{-1}$ exists. For the first, we have
    \begin{align}
      x y = x z & \implies x^{-1} (x y) = x^{-1} (x z) && \tag{multiplication is a function} \\
                & \implies (x^{-1} x) y = (x^{-1} x) z && \tag{$\times$ is associative} \\
                & \implies 1 y = 1 z && \tag{definition of multiplicative inverse} \\  
                & \implies y = z && \tag{definition of identity}
    \end{align}
    For the second, we can set $z = 1$ and apply the first property. For the third, we have 
    \begin{align}
      xy = 1 & \implies x^{-1} (x y) = x^{-1} 1 && \tag{multiplication is a function} \\
             & \implies (x^{-1} x) y = x^{-1} 1 && \tag{$\times$ is associative} \\
             & \implies 1 y = x^{-1} 1 && \tag{definition of multiplicative inverse} \\
             & \implies y = x^{-1} && \tag{definition of identity}
    \end{align}
    For the fourth, we simply see that $x^{-1}$ is a multiplicative inverse of both $x$ and $(x^{-1})^{-1}$ in the group $(\mathbb{F} \setminus \{0\}, \times)$, and since inverses are unique, they must be equal. 
  \end{proof}

  \begin{lemma}[Properties of Distribution]
    For any $x, y, z \in \mathbb{F}$, the field axioms satisfy 
    \begin{enumerate}
      \item $0 \cdot x = 0$.
      \item If $x \neq 0$ and $y \neq 0$, then $x y \neq 0$.
      \item $-1 \cdot x = -x$. 
      \item $(-x) y = - (xy) = x (-y)$. 
      \item $(-x) (-y) = xy$. 
    \end{enumerate}
  \end{lemma} 
  \begin{proof}
    For the first, note that 
    \begin{align}
      0 x & = (0 + 0) \cdot x = 0 x + 0x 
    \end{align}
    and subtracting $0x$ from both sides gives $0 = 0x$. For the second, we can claim that $xy \neq 0$ equivalently claiming that it will have an identity. Since $x, y \neq 0$, their inverses exists, and we claim that $(xy)^{-1} = y^{-1} x^{-1}$ is an inverse. We can see that by associativity, 
    \begin{equation}
      (y^{-1} x^{-1}) (xy) = y^{-1} (x^{-1} x) y = y^{-1} y = 1
    \end{equation} 
    For the third, we see that 
    \begin{equation}
      0 = 0 \cdot x = (1 + (-1)) \cdot x = 1 \cdot x + (-1) \cdot x = x + (-1) \cdot x 
    \end{equation}
    which implies that $-1 \cdot x$ is the additive inverse. The fourth follows immediately from the third by the associative property. For the fifth we can see that 
    \begin{align}
      (-x) (-y) & = (-1) x (-1) y && \tag{property 3} \\
                & = (-1) (-1) x y && \tag{$\times$ is commutative} \\
                & = -1 \cdot (-xy) && \tag{property 3} \\
                & = -(-xy) && \tag{property 3} \\
                & = xy && \tag{addition property 4}
    \end{align}
  \end{proof}

  Note that given a set, we can really put whatever order we want on it. However, consider the field with the following order. 
  \begin{equation}
    \mathbb{F} = \{0, 1\}, \; 0 < 1
  \end{equation} 
  This does not behave well with respect to its operations because for example if we have $0 < 1$, then adding the same element to both sides should preserve the ordering. But this is not the case since $0 + 1 = 1 > 1 + 1 = 0$. While it may be easy to define an order, we would like it to be an ordered field. 

  \begin{definition}[Ordered Ring/Field]
    An \textbf{ordered ring} is a ring that has an order satisfying 
    \begin{enumerate}
      \item $y < z \implies x + y < x + z$ for all $x \in \mathbb{F}$. 
      \item $x > 0, y > 0 \implies xy > 0$. 
    \end{enumerate}
    An \textbf{ordered field} has the same definition, and an ordered field is an ordered ring. 
  \end{definition}

  \begin{theorem}[Properties]
    In an totally ordered ring, 
    \begin{enumerate}
      \item $x > 0 \implies -x < 0$. 
      \item $x \neq 0 \implies x^2 > 0$. 
      \item If $x > 0$, then $y < z \implies xy < xz$. 
    \end{enumerate}
  \end{theorem} 
  \begin{proof}
    The first property is a single-liner 
    \begin{equation}
      0 < x \implies 0 + -x < x + -x \implies -x < 0 
    \end{equation}
    For the second property, it must be the case that $x > 0$ or $x < 0$. If $x > 0$, then by definition $x^2 > 0$. If $x < 0$, then 
    \begin{equation}
      x^2 = 1 \cdot x^2 = (-1)^2 \cdot x^2 = (-1 \cdot x)^2 = (-x)^2
    \end{equation}
    and since $-x > 0$ from the first property, we have $x^2 = (-x)^2 > 0$. For the third, we use the distributive property. 
    \begin{align}
      y < z & \implies 0 < z - y \\ 
            & \implies 0 = x 0 < x(z - y) = xz - xy \\
            & \implies xy < xz
    \end{align}
  \end{proof}

\subsection{Topological and Metric Spaces} 

  \begin{definition}[Topology]
    
  \end{definition}

  Note that an order can be used to generate an order topology, which we will define below. 

  \begin{example}[Order Topology on $\mathbb{Q}$]
    The order topology on $\mathbb{Q}$ is the topology generated by the set $\mathscr{B}$ of all open intervals 
    \begin{equation}
      (a, b) \coloneqq \{ x \in \mathbb{Q} \mid a < x < b\}
    \end{equation}
  \end{example}

  \begin{definition}[Metric]
    Given an arbitrary set $X$, a \textbf{metric} on $X$ is a function 
    \begin{equation}
      d: X \times X \rightarrow \mathbb{R}
    \end{equation}
    satisfying
  \end{definition}

\subsection{Vector Spaces}

  \begin{definition}[Vector Space]
    A vector space $V$ over a field $\mathbb{F}$ is a. 
  \end{definition}

  \begin{example}[Norm]
    Given a vector space $V$ over subfield $\mathbb{F} \subset \mathbb{C}$, the norm 
    \begin{equation}
      ||\cdot|| : V \rightarrow \mathbb{R}
    \end{equation}
  \end{example}

  \begin{example}[Inner Product]
    Given a vector space $V$ over subfield $\mathbb{F} \subset \mathbb{C}$, the \textbf{inner product} 
    \begin{equation}
      \langle \cdot, \cdot \rangle : V \times V \rightarrow \mathbb{C}
    \end{equation} 
    is a map satisfying 
  \end{example}

  \begin{definition}[Convex Sets]
    A set $S$ is convex if for every point $x, y \in S$, the point 
    \begin{equation}
      z = t x + (1 - t) y \in S
    \end{equation}
    where $0 \leq t \leq 1$. 
  \end{definition}

\subsection{Measure Spaces}



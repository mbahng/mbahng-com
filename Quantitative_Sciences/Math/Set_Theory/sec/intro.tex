A few notes I have taken from reading Hrbacek and Jech's \textit{Introduction to Set Theory}, plus some other supplementary resources. 

Consider the set of all students at Duke University, or the set of all stars in our universe, or the set of all pink elephants. We can construct all kinds of sets, but they are not real objects of the world and are only created by our minds. When we say the number $2$, can we \textit{see} ``2'' with our own eyes, or \textit{hold} ``2'' on the palm of our hand? No. They are merely an abstraction of our minds, and so care must be taken to treat them as abstractions. Often, we may rely on our intuition in the real world to talk about sets, but as we will find our shortly, this may not work. Therefore, we will go with Cantor's definition of a set, created in his naive attempt  to construct set theory at the end of the 19th century. 

Unlike axiomatic set theories, which are defined using formal logic, naive set theory was defined informally at the end of the 19th century by Cantor, 
in natural language (like English). 

\begin{definition}[Cantor]
  A \textit{set} is a collection into a whole of definite (i.e. well-defined), distinct objects of our intuition or our thought. These objects are called \textit{elements}. 
\end{definition}

Okay, sounds simple enough, but there are a lot of holes in this definition. It all stems from the lack of specification of what ``well-defined'' means. According to Cantor, we can construct a set by specifying a defined property $P$ and talking about the set of all elements $x$ satisfying $P(x)$, denoted in \textit{set-builder notation} as 
\begin{equation}
  \{x \mid P(x)\}
\end{equation}

But the property $P$ cannot by itself guarantee the consistency and unambiguity of what exactly constitutes and what does not constitute a set, and therefore this is not a formal definition. Since such a formalism is absent, we cannot have any sets to work with in the first place. 

Perhaps this is all a bit too philosophical. If I talk about the set $\{1, 2, 3\}$ or the set of all prime numbers under $100$, this is pretty concrete and well-defined, and there is really no problem at first glance. This is to be expected, since Cantor relied on natural language (like English) to construct his \textit{naive set theory}. It describes the aspects of mathematical sets using words (e.g. \textit{satisfying, such as, ...}) and sufficed for the everyday use of set theory in mathematics back then. Such issues did not arise until Russell formulated such a ``well-defined'' set that lead to a paradox. 

\begin{definition}[Russell Set]
  Let $R = \{x \mid x \not\in x\}$, i.e. the set of all sets that do not contain themselves as elements. This is called the \textbf{Russell set}. 
\end{definition}

While the property may sound a bit confusing, this is still well-defined since for any set, we can check this property to be either true or false. 

\begin{example}[Russell Property is Verifiable]
  Let's consider some examples of sets. 
  \begin{enumerate}
    \item $\{1\} \in R$ since it does not contain $\{1\}$ as an element. 
    \item The set of \textit{everything}\footnote{everything is indeed well-defined... or is it?} is not in $R$ since it contains itself (as an element of everything). 
  \end{enumerate}
\end{example}

However, this leads to a paradox that violates the law of the excluded middle. 

\begin{theorem}[Russell's Paradox] 
  The Russell set exists and does not exist. 
\end{theorem}
\begin{proof}
  We will determine if $R$ is an element of itself. 
  \begin{enumerate}
    \item If $R \in R$, then by it does contain itself, so it does not satisfy the property and $R \not\in R$. 
    \item If $R \not\in R$, then it satisfies the property, so $R \in R$. 
  \end{enumerate}
  Therefore, it is both the case that $x \in R$ and $x \not\in R$, which contradicts the membership definition. Therefore, $R$ is both a set from set-builder construction and not a set due to the membership definition. 
\end{proof}

There are other paradoxes out there, which are similar to Russell's paradox. 

\begin{corollary}[Universal Set]
  The set of all sets $E$, called the \textbf{universal sets}, both exists and does not exist. Furthermore, the set of all singleton sets does not exist. 
\end{corollary}
\begin{proof}
  We can define $U^\prime = \{x \mid \{\} = \{\} \}$, which defines a set. Then the property $P$ that $\{\} = \{\}$ is always true, and $U^\prime$ would contain everything, and by the definition of equality $U = U^\prime$. Now since the Russell set $R$ is both a set and not a set from Russell's paradox, we have $R \in U$ and $R \not\in U$, which means that $U$ cannot exist. Therefore $U$ does not exist. 

  Alternatively, we can prove it as such. $E$ exists according to naive set theory due to set builder notation. Then by definition $E \in E$. Now we can construct Russell's paradox from this by defining 
  \begin{equation}
    \{ x \in E \mid x \not\in x \}
  \end{equation}
  To prove the second claim, note that if such a set $S$ existed, then $\cup S = E$, which leads to the same paradox. 
\end{proof}

So the sufficiency a well-defined property to be able to construct a set is \textit{too powerful} in that we can construct \textit{any} set we want. This leads us to construct the Russell set, which opens up a lot of paradoxes. This ability to take a set of all objects satisfying such an arbitrary property is known as the \textbf{schema of general comphrension}, and is the major flaw of naive set theory. Therefore, we would like to restrict the notion of well-defined in a way, which leads to axiomatic set theories. Attempting to achieve this will be done in axiomatic set theory, like ZFC. 



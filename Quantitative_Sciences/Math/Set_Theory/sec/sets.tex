\section{Sets} 

  So with these paradoxes in mind, we would like to construct an axiomatic formulation of sets. My take is to think that sets ``exist'' out there somewhere in the universe, and our job is to find them. Cantor with his naive set theory believed that for every meaningful property of things there is a set whose members are exactly all the things with that property. Russell shows this this cannot be the case. Nevertheless, \textit{some} sets exist, and we have intuitive experience thinking about finite sets. Therefore, the axioms of set theory are a limited list of \textit{assumptions} that we hope are true about that actually existing universe of sets. As long as they are true, then whatever we conclude from them by valid reasoning steps must also be true.\footnote{This idea is called naive Platonism.} Hence we have the following definition, which first requires the familiar property of acting like a collection of something, and then obeys the axioms we set. 
  
  \begin{definition}[Set]
    A \textbf{set} $X$ is anything 
    \begin{enumerate}
      \item that has the innate property of containing elements, and 
      \item obeys the axioms of ZFC. 
    \end{enumerate}
  \end{definition}  

  Let's first talk about the language, where they are defined formally using the axioms in the next subsection. From first-order logic, note that we have the following symbols in our alphabet $\mathcal{L}_{\mathrm{ZFC}}$. 
  \begin{enumerate}
    \item The logical connectives $\neg$, $\lor$, $\land$. 
    \item The quantifier symbols $\exists, \forall$ 
    \item Brackets $()$. 
  \end{enumerate}
  To represent sets, we also need symbols, and the membership property requires us to define a symbol for that too. 
  \begin{enumerate}
    \item A countably infinite amount of variables used for representing sets. 
    \item The set membership symbol $\in$. In fact, when we say $x \in A$, this is a proposition formed from the predicate $P(x)$. 
  \end{enumerate} 
  This is what we have to work with so far. We will construct the rest of the symbols ($=, \subset, \supset, \cup, \cap$) from the axioms. 

  Now we state the axioms, which is the foundation of ZF set theory. A question one might ask is: how do we even know for sure that these axioms aren't contradictory? The answer is that we don't, and that is why we take them as axioms rather than provable theorems. Fortunately, from the formulation in the early 20th century up until now, no contradictions have been found, and if there is one, then it would be very bad news for us.  
  
  As obvious as the axioms may seem, none of them can be rigorously proven since we need to start from a set of assumptions and some principles of logic to use some deductive reasoning. 

  \begin{axiom}[Axiom of Existence]
    There exists a set which has no elements. 
  \end{axiom}

  \begin{definition}[Empty Set]
    The \textbf{empty set} is denoted $\emptyset$.\footnote{Note that this is not determined to be unique... yet!} 
  \end{definition}

  Why do we need to axiomize what seems to be such an obvious thing? The rest of the axioms that follow talk about what we can or cannot do with sets, but this whole theory would be useless if we didn't know if there exists \textit{any} set that obeys the following axioms! Therefore, we would like to assert the existence of at least one set, namely the empty set. This asserts that our universe of discourse is not void, and so it gives us a starting set to work with, which we can build on to create more sets.   

\subsection{Axiom of Extensionality} 

  The empty set is simply described as the set containing nothing, but it can be constructed in various ways. I can describe it as the set of humans that have negative age, or the set of married bachelors. All examples of this kind describe one and the same set $\emptyset$, but we cannot yet prove this. This is why we need our second axiom, which states that a set is uniquely characterized simply by its elements. 

  \begin{axiom}[Axiom of Extensionality]
    Two sets are equal (are the same set) if they have the same elements. 
    \begin{equation}
      \forall A \forall B \big[ \forall x (x \in A \iff x \in B) \iff A = B\big]
    \end{equation}
  \end{axiom} 

  \begin{definition}[Equality]
    This axiom allows us to define the equality operator $=$, which we now add to our alphabet. 
  \end{definition} 

  \begin{theorem}[Uniqueness of Empty Set]
    The empty set is unique. 
  \end{theorem}
  \begin{proof}
    Assume that $A$ and $B$ are sets with no elements. Then it is vacuously true that every element of $A$ is an element of $B$, and vice versa. Therefore by the axiom of extensionality $A = B$. 
  \end{proof}

  This next theorem now shows that every set is uniquely characterized by its distinct elements, which aligns with our established notion that sets don't contain repeated elements. 

  \begin{theorem}[Sets Don't Contain Repeated Elements]
    Sets are unique up to distinct elements. That is, given 2 sets $A, B$, 
    \begin{equation}
      \forall x (x \in A \iff x \in B) \implies A = B
    \end{equation}
  \end{theorem} 

  As an example, we have the following: 
  \begin{equation}
    \{1, 1, 2\} = \{1, 2\} = \{1, 1, 2, 2\}
  \end{equation} 
  Though note that we don't even know if any of the sets above even exist with our axioms so far! 

\subsection{Axiom of Restricted Comprehension}

  The axiom assists us in regulating which sets are viable and which are not, preventing Russell's paradox. 

  \begin{axiom}[Axiom Schema of Restricted Comprehension]
    Given $P$ a formula with $P(x)$ specifying a property of $x$, for any set $A$, there exists a set $B$ such that $x \in B$ if and only if $x \in A$ and $P(x)$. That is, if $A$ exists, then the set, written in set-builder notation, 
    \begin{equation}
      B = \{x \in A \mid P(x) \}
    \end{equation}
    exists.\footnote{Note that this axiom does not allow the construction of entities of the more general form $\{x \mid P(x)\}$. This restriction is obviously needed to avoid Russell's paradox, hence the name \textit{restricted} comprehension. } 
  \end{axiom} 

  \begin{lemma}[Uniqueness of Restricted Commprehension]
    The set $B = \{x \in A \mid P(x) \}$ is unique, and therefore it makes sense to treat it as a unique object. 
  \end{lemma}

  The axiom of specification allows us to denote subsets. 

  \begin{definition}[Subset, Superset]
    Notationally, if $A$ is a subset of $B$, then we write $A \subset B$. Similarly, we say $A$ is a \textbf{superset} of $B$, written $A \supset B$, if $B \subset A$. 
  \end{definition} 

  We can extend this to the restriction of more sets. 

  \begin{theorem}[Existence of Binary Intersection]
    That is, if $A, B$ are sets, then there is a set $C$ such that $x \in C$ if and only if $x \in A$ and $x \in B$. This allows us to define intersection as 
    \begin{equation}
      A \cap B \coloneqq \{x \in A \mid x \in B \}
    \end{equation} 
  \end{theorem} 
  \begin{proof}
    
  \end{proof} 

  We can extend this proof to define the intersection of a set of sets. 
  
  \begin{definition}[Intersection]
    Given a nonempty\footnote{$\cap \emptyset$ is not defined since every $x$ belongs to all $A \in \emptyset$ vacuously, and so such a set would be the universal set, which does not exist.} set of sets $S$, the \textbf{intersection} $\cap S$ is the unique set satisfying $x \in \cap S$ if and only if $x \in A$ for all $A \in S$. In set builder notation, we can write 
    \begin{equation}
      \bigcap S \coloneqq \{x \in A \mid \forall B (B \in S \implies x \in B) \}
    \end{equation}
    We can also expand our notation by defining the following. 
    \begin{enumerate}
      \item Just another way of writing is 
        \begin{equation}
          \cup S = \bigcap_{A \in S} A
        \end{equation}

      \item By the axiom of extensionality, we can define 
        \begin{equation}
          A_1 \cap \ldots \cap A_n = \cap \{A_1, \ldots, A_n\}
        \end{equation}
    \end{enumerate}
  \end{definition}

  \begin{definition}[Disjoint Sets]
    Two sets $A$ and $B$ are \textbf{disjoint} if $A \cap B = \emptyset$. 
  \end{definition}

  Unfortunately, the union cannot be expressed in this specification schema, and we need a separate axiom for this. 

  \begin{definition}[Set Minus]
    We can however define set minus. Given sets $A, B$
    \begin{equation}
      A \setminus B \coloneqq \{ x \in A \mid x \not\in B \}
    \end{equation}
  \end{definition}

  \begin{definition}[Set Complement]
    Given $B$ and a subset $A \subset B$, the \textbf{complement} of $A$ with respect to $B$ is 
    \begin{equation}
      A^c \coloneqq \{ x \in B \mid x \not\in A \} = B \setminus A
    \end{equation}
  \end{definition}  

  \begin{definition}[Symmetric Difference]
    Given sets $A, B$, the \textbf{symmetric difference} between the two sets is defined 
    \begin{equation}
      A \triangle B \coloneqq (A \setminus B) \cup (B \setminus A)
    \end{equation}
  \end{definition} 

\subsection{Axiom of Pairing}

  Okay this is all great, but the only set that we have claimed the existence of is $\emptyset$, and the schema of restricted comprehension is useless in constructing any new sets since 
  \begin{equation}
    \{x \in \emptyset \mid P(x) \} = \emptyset
  \end{equation}
  for any property $P$. Starting from now, we provide more helpful axioms to construct new sets. 

  \begin{axiom}[Axiom of Pairing]
    If $A, B$ are sets, then there exists a set which contains $A$ and $B$ as elements.\footnote{For example, if $A = \{1, 2\}$ and $B = \{2, 3\}$,then $\{\{1, 2\}, \{2, 3\}\}$ exists.}
    \begin{equation}
      \forall A \forall B \exists C((A \in C) \land (B \in C))
    \end{equation}
    This allows us to construct sets from old ones. 
  \end{axiom}

  Note that clearly, this set is not necessarily unique, since there can be other elements in $C$ in addition to $A$ and $B$. 

  \begin{theorem}[Nested Sets]
    By the axiom of pairing, if we have a set $X$, then $\{X\}$ is also a set, since we can set $A = B = X$ which asserts the existence of $\{X, X\} = \{X\}$. 
  \end{theorem} 

  \begin{example}[More Sets]
    Now we have unlocked our first sets that are not the empty set! Consider the following. 
    \begin{enumerate}
      \item If $A = B = \emptyset$, then by the axiom of pairing we can get $C = \{\emptyset, \emptyset\}$, which is equal to $\{\emptyset\}$ by the axiom of extensionality. 
      \item Now we let $A = \emptyset, B = \{\emptyset\}$, and so $C = \{A, B \} = \{\emptyset, \{\emptyset\}\}$. 
    \end{enumerate}
  \end{example}

\subsection{Axiom of Union}

  \begin{axiom}[Axiom of Union] 
    For any set of sets $S$, there exists a set $U$ such that $x \in U$ if and only if $x \in A$ for some $A \in S$. 
    \begin{equation}
      \forall \mathcal{F} \exists U \forall X \forall x \big[ (x \in X \land X \in \mathcal{F}) \implies x \in U \big]
    \end{equation}
  \end{axiom}

  \begin{lemma}[Union is Unique]
    The union $U$ is unique. 
  \end{lemma}
  \begin{proof}
    Again use the axiom of extensionality. 
  \end{proof} 

  \begin{definition}[Union]
    The set $U$, is called the \textbf{union} of sets $A \in S$, denoted $\cup S$. We can also expand our notation by defining the following. 
    \begin{enumerate}
      \item Just another way of writing is 
        \begin{equation}
          \cup S = \bigcup_{A \in S} A
        \end{equation}

      \item By the axiom of extensionality, we can define 
        \begin{equation}
          A_1 \cup \ldots \cup A_n = \cup \{A_1, \ldots, A_n\}
        \end{equation}
    \end{enumerate}
  \end{definition}

  Sometimes, it is formulated alternatively as follows: For any set of sets $S$, there is a set $U$ containing every element that is a member of $S$. This does not necessarily mean that $U$ is unique, and so $\cup S$ is not defined yet. However, we can define it by using the axiom of restricted comprehension and defining 
  \begin{equation}
    \cup S \coloneqq \{ x \in A \mid \exists X (x \in X \land X \in S ) \}
  \end{equation}

  \begin{example}
    \begin{equation}
      \{\{\emptyset\}\} \cup \{\emptyset, \{\emptyset\}\} = \{\emptyset, \{\emptyset\}\}
    \end{equation}
  \end{example}

\subsection{Rules of Set Operations}

  Let's first talk about rules following the union, intersection, and set minus operators. 

  \begin{theorem}[Commutativity]
    Union and intersection are commutative. 
    \begin{align}
      A \cup B & = B \cup A \\
      A \cap B & = B \cap A
    \end{align}
  \end{theorem}

  \begin{theorem}[Associativity]
    Union and intersection are associative. 
    \begin{align}
      (A \cup B) \cup C & = A \cup (B \cup C) \\
      (A \cap B) \cap C & = A \cap (B \cap C) 
    \end{align}
  \end{theorem}

  \begin{theorem}[Distributivity]
    Given sets $A, B, C$, 
    \begin{align}
      A \cap (B \cup C) & = (A \cap B) \cup (A \cap C) \\
      A \cup (B \cap C) & = (A \cup B) \cap (A \cup C)
    \end{align}
  \end{theorem}
  \begin{proof}
    Listed. 
    \begin{enumerate}
      \item $A \cap (B \cup C) = (A \cap B) \cup (A \cap C)$. 
        \begin{enumerate}
          \item $A \cap (B \cup C) \subset (A \cap B) \cup (A \cap C)$. Assume $x \in A \cap (B \cup C)$. Then $x \in A$ and $x \in B \cup C$. If $x \in B$, then $x \in A \cap B$. If $x \in C$, then $x \in A \cap C$. Therefore, since $x \in B \cup C$, it must be the case that $x \in A \cap B$ or $x \in A \cap C$, which by definition implies $x \in (A \cap B) \cup (A \cap C)$. 

          \item $A \cap (B \cup C) \supset (A \cap B) \cup (A \cap C)$. Assume that $x \in (A \cap B) \cup (A \cap C)$. Then WLOG let $x \in A \cap B$. Then $x \in A$ and $x \in B \subset (B \cup C)$, so by definition $x \in A \cap (B \cup C)$. 
        \end{enumerate}

      \item $A \cup (B \cap C) = (A \cup B) \cap (A \cup C)$.
        \begin{enumerate}
          \item $A \cup (B \cap C) \subset (A \cup B) \cap (A \cup C)$. Assume $x \in A \cup(B \cap C)$. Then $x \in A$ or $x \in B \cap C$. If $x \in A$, then since $A \subset (A \cup B)$ and $A \subset (A \cup C)$, we have $x \in (A \cup B)$ and $x \in (A \cup C)$, which by definition means $x \in (A \cup B) \cap (A \cup C)$. If $x \not\in A$, then $x \in B \cap C \implies x \in B \subset (A \cup B)$ and $x \in C \subset (A \cup C)$, and so $x \in (A \cup B) \cap (A \cup C)$. 

          \item $A \cup (B \cap C) \supset (A \cup B) \cap (A \cup C)$. Assume $x \in (A \cup B) \cap (A \cup C)$. Then $x \in A \cup B$. If $x \in A$, then since $A \subset A \cup (B \cap C)$, $x \in A \cup (B \cap C)$. If $x \not\in A$, then $x \in B$. Since $x \in A \cup C$, $x \in C$ also. Therefore by definition $x \in (B \cap C) \subset A \cup (B \cap C) \implies x \in A \cup (B \cap C)$. 
        \end{enumerate}
    \end{enumerate}
  \end{proof}

  \begin{theorem}[DeMorgan's Laws]
    If $X$ is a set and $A, B \subset X$, then 
    \begin{align}
      X \setminus (A \cup B) & = (X \setminus A) \cap (X \setminus B) \\
      X \setminus (A \cap B) & = (X \setminus A) \cup (X \setminus B)
    \end{align}
  \end{theorem}
  \begin{proof}
    Listed. 
    \begin{enumerate}
      \item $X \setminus (A \cup B) = (A \setminus A) \cap (X \setminus B)$.
        \begin{enumerate}
          \item $X \setminus (A \cup B) \subset (A \setminus A) \cap (X \setminus B)$. Assume $x \in X \setminus (A \cup B) \iff x \in X$ and $x \not\in (A \cup B)$. Since $x \not\in (A \cup B$, $x \not\in A$ and $x \not\in B$. However, $x \in X$, so $x \not\in A \implies x \in X \setminus A$. Same goes for $B$, and so $x \in (X \setminus A) \cap (X \setminus B)$. 

          \item $X \setminus (A \cup B) \supset (A \setminus A) \cap (X \setminus B)$. Assume $x \in (X \setminus A) \cap (X \setminus B)$. Then $x \in X \setminus A \iff X \in X$ and $x \not\in A$, and $x \in X \setminus B \iff x \in X$ and $x \not\in B$. Since $x \not\in A$ and $x \not\in B$, $x \not\in A \cup B$. Combined with the fact that $x \in X$, we have $x \in X \setminus (A \cup B)$. 
        \end{enumerate}

      \item $X \setminus (A \cap B) = (A \setminus A) \cup (X \setminus B)$.
        \begin{enumerate}
          \item $X \setminus (A \cap B) \subset (A \setminus A) \cup (X \setminus B)$. Let $x \in X \setminus (A \cap B)$. Then $x \in X$ and $x \not\in A \cap B$. Since $x \not\in A \cap B$, it must be the case that at least $x \not\in A$ or $x \not\in B$. WLOG let $x \not\in A$. Then $x \in X$ and $x \not\in A \implies x \in (X \setminus A) \subset (X \setminus) \cup (X \setminus B) \implies x \in (X \setminus A) \cup (X \setminus B)$. 

          \item $X \setminus (A \cap B) \supset (A \setminus A) \cup (X \setminus B)$. WLOG let $x \in (X \setminus A)$. Then $x \in X$ and $x \not\in A$, and $x \not\in A \implies x \not\in (A \cap B) \subset A$ (contrapositive is trivial). Therefore, $x \in X$ and $x \not\in (A \cap B) \iff  x \in X \setminus (A \cap B)$. 
        \end{enumerate}
    \end{enumerate}
  \end{proof} 

  \begin{theorem}[Properties of Set Difference]
    We have the following. 
    \begin{enumerate}
      \item $A \cap (B \setminus C) = (A \cap B) \setminus C$ 
      \item $A \setminus B = \emptyset$ if and only if $A \subset B$. 
    \end{enumerate}
  \end{theorem}
  \begin{proof}
    
  \end{proof} 

  \begin{theorem}[Properties of Symmetric Difference]
    We have the following. 
    \begin{enumerate}
      \item $A \triangle A = \emptyset$
      \item $A \triangle B = B \triangle A$
      \item $(A \triangle B) \triangle C = A \triangle (B \triangle C)$
      \item $A \triangle B = (A \cup B) \setminus (A \cap B)$
    \end{enumerate}
  \end{theorem}
  \begin{proof}
    
  \end{proof} 

\subsection{Axiom of Regularity} 

  \begin{axiom}[Axiom of Regularity]
    Every non-empty set $A$ contains a member $x$ such that $A$ and $x$ are disjoint sets. 
    \begin{equation}
      \forall A \big[ A \neq \emptyset \implies \exists x (x \in A \land A \cap x = \emptyset) \big]
    \end{equation}
    This, along with the axioms of pairing and union, implies that no set is an element of itself and that every set has an ordinal rank. 
  \end{axiom}


\section{Factor Analysis} 

  As we have constantly seen, there are specific themes that run between models. In PCA, we have taken some data $x$ in high-dimension $d$ and reduced it to a lower-dimensional orthogonal representation in $\mathbb{R}^k$. In other words, for some sample $x \in \mathbb{R}^{1 \times d}$, the projection onto its component space $x V_k \in \mathbb{R}^{1 \times k}$ is a more parsimonious representation with respect to some other basis vectors. The $v_1, \ldots, v_k$ are new features that are linear combinations of the old vectors. Are they interpretable? In some cases yes and in most cases no, which is why we also call them \textit{latent variables} that live in a \textit{latent space}. 

  Another type of model that encodes covariates in a latent space are factor models, which was developed by Spearman in 1904 \cite{1904spearman}. The general idea was that we have some $d$-dimensional random vector $x$, and we would like to encode it in a $k$-dimensional random vector $f$, called the \textit{factors}. Since we are trying to compress the data, generally $k < d$. The first thing that comes to mind is to try and compare how the variables $x_i$ and $f_j$ correlate to each other, and this is exactly what Spearman did. 

  Before we get into factor models, let's step back and talk more about latent variable models. Colloquially, we would like to find the distribution of some data, whether it'd be $(x, y)$ supervised tasks or $x$ for unsupervised. For the unsupervised case, say that we have some covariates $x$ and we want to find its true distribution $p^\ast$. In density estimation so far, what we have done is define a family of distributions $\{p_\theta\}$ and optimize the loss by maximizing the MLE or something else. 
  \begin{equation}
    \min_\theta L(p_\theta, p^\ast) = \max_\theta \prod_{i} p_\theta(x^{(i)})
  \end{equation}
  In order to do this we work with explicitly parameterized distribution families (e.g. Gaussian, Gamma, multinomial, etc.), but this is too simple to model complex things in real like (e.g. the distribution of faces).

  Therefore, we consider \textit{implicitly parameterized} probability distributions by ``adding'' a latent distribution $z$, creating the joint distribution $(x, z)$. This may look more complicated, but it captures a much richer family of distributions. For example, we might try modeling $x$ as a function of $z$, and try to learn some function $x = f(z)$. If we have an accurate function $f$, we can do many things. 
  \begin{enumerate}
    \item Given an $x$, we might find the closest point on the image of $f$, perhaps some manifold, as low-rank approximation of $x$. This dimensionality reduction is essentially what PCA does with projections.\footnote{We will in fact extend PCA to probabilistic PCA soon to make it generative.} 
    \item If we can sample from $z$, then we can forward it through $f$ and can sample from $x$, making this is a generative model. 
  \end{enumerate}

  Like we do with everything else in math, we take a look at the simplest case when the class of functions are linear. This is known as \textit{linear latent variable modeling}. 
  \begin{equation}
    x = \mu + \Lambda z + \epsilon
  \end{equation}
  where the noise $\epsilon$ is typically Gaussian and diagonal (but not necessarily the same component-wise variances). 

\subsection{Probabilistic PCA}

  The main goal of PCA was to do dimensionality reduction by creating a botttleneck in the number of dimensions $k$. Our goal was to approximate the original random variable $x$ by first projecting onto a lower-dimensional space $z = V_k^T x$, and then embedding it through a linear injection. 
  \begin{equation}
    x \approx V_k V_k^T x = V_k z 
  \end{equation} 
  What if we don't restrict the dimensions at all, and just let $k = d$? Then we have the exact equation. 
  \begin{equation}
    x = V V^T x = V z
  \end{equation}
  This is trivial since $V$---as an orthogonal matrix---satisfies $V V^T = V^T V = I_d$. Essentially, this is just a rotation of the axes. We have the maximal restriction when $k = 1$, and as we increase  $k$ to $d$, our reconstruction loss will decrease. But almost always, if our data does not sit exactly on a subspace, we cannot get an exact reconstruction of our data with $k < d$. There are two goals of probabilistic PCA (PPCA), and each of them are solved by introducing a probabilistic term in the model. \cite{1999tipping} 
  \begin{enumerate}
    \item \textit{Generative}. We would like our model to be generative. In regular PCA, we saw that for some $z \in \mathbb{R}^k$ in the latent component space, our reconstructed sample is $\hat{x} = \Sigma V_k z$. Therefore, if we just change $z$ from a point to a distribution (e.g. Gaussian), we can sample $z \sim \mathcal{N}(0, I_k)$, and then transform it to get a random variable $x = \mu + \Sigma V_k z$, which will give a density. 
    \begin{equation}
      x \sim \mathcal{N} \big( \mu, (V_k \Sigma)(V_k \Sigma)^T \big) = \mathcal{N} \big( \mu, V_k \Sigma U^T U \Sigma V^T \big) = \mathcal{N} \big( \mu, X_k^T X_k)
    \end{equation} 
    However, this is a distribution. 

    \item \textit{Exact Reconstruction}. However, $X_k \in \mathbb{R}^{n \times d}$ with $d << n$, and so $X_k^T X_k \in \mathbb{R}^{d \times d}$ is not full rank, and so the distribution is restricted to strictly the $k$-dimensional subspace $L_k \subset \mathbb{R}^D$. We want a model that has the both of best worlds: it has a bottleneck so that $k < d$, but at the same time it can do an exact reconstruction of the data. This can be solved by introducing a probabilistic error term $\epsilon$ that accounts for the variability of the data around the principal subspace. So let's add an isotropic Gaussian $\epsilon \sim \mathcal{N}(0, \sigma^2 I)$\footnote{Why isotropic? No real reason. In factor models, we generalize this to arbitrary covariance matrices, and so PPCA is a specific case of factor analysis.}, which gives us 
    \begin{equation}
      x = \Sigma V_k z + \epsilon \implies X \sim \mathcal{N}(\mu, X_k^T X_k + \sigma^2 I)
    \end{equation} 
  \end{enumerate} 

  Note again that we are adding \textit{two} probabilistic terms, each of which serves a specific purpose. Great, and finally, let's clean up some notation and polish it up. 
  \begin{enumerate}
    \item First, let's just call $W = \Sigma V_k \in \mathbb{R}^{d \times k}$, keeping the $k$ implicit, and treat it as the parameter to estimate. 
    \item Second, let's remove the assumption that $x$ is $0$-mean and add back the mean term $\mu$. 
  \end{enumerate} 

  This gives us our model. 

  \begin{definition}[Probabilistic PCA] 
    The \textbf{probabilistic PCA} model is a latent factor model that summarizes the data generating distribution $x$ as  
    \begin{equation}
      x = \mu + W z + \epsilon, \qquad z \sim N(0, I), \epsilon \sim N(0, \sigma^2 I) 
    \end{equation} 
    with parameters $\theta = \{\mu, W, \sigma\}$. 
  \end{definition} 

  An immediate consequence is that the closed form of the distrbution of $x$ under this model can be solved. Calculating the pdf of $x$ requires us to marginalize out the $z$, but since marginals of Gaussians are Gaussians, this is quite easy. 

  \begin{lemma}[Likelihood of PPCA] 
    We claim that given $\theta = \{\mu, W, \sigma\}$, we have 
    \begin{equation}
      x \sim \mathcal{N}(\mu, W W^T + \sigma^2 I) 
    \end{equation} 
  \end{lemma}
  \begin{proof}
    By assuming it is Gaussian, you can just directly compute the expectation and covariance, but I will do the full density derivation. We start with the conditional and prior distributions from the probabilistic PCA model:
    \begin{align}
      p(x | z, W) &\propto \exp\left(-\frac{(x - W^T z)^T(x - W^T z)}{2\sigma^2}\right), \\
      p(z) &\propto \exp\left(-\frac{z^T z}{2}\right).
    \end{align}

    The joint distribution is given by:
    \begin{align}
      p(x, z | W) &= p(x | z, W)p(z) \quad \text{(since } z \perp W\text{)} \\
      &\propto \exp\left(-\frac{(x - W^T z)^T(x - W^T z)}{2\sigma^2} - \frac{z^T z}{2}\right).
    \end{align}

    Expanding the quadratic term $(x - W^T z)^T(x - W^T z)$:
    \begin{align}
      (x - W^T z)^T(x - W^T z) &= x^T x - x^T W^T z - z^T W x + z^T W W^T z \\
      &= x^T x - 2x^T W^T z + z^T W W^T z,
    \end{align}
    where we used the fact that $x^T W^T z = z^T W x$ (scalar quantities).

    Substituting back into the joint distribution:
    \begin{align}
      p(x, z | W) &\propto \exp\left(-\frac{x^T x - 2x^T W^T z + z^T W W^T z}{2\sigma^2} - \frac{z^T z}{2}\right) \\
      &= \exp\left(-\frac{x^T x - 2x^T W^T z + z^T W W^T z}{2\sigma^2} - \frac{z^T z}{2}\right).
    \end{align}

    Factoring out $-\frac{1}{2}$ and collecting terms:
    \begin{align}
      p(x, z | W) &\propto \exp\left(-\frac{1}{2}\left(x^T\left(\frac{1}{\sigma^2}I\right)x + 2x^T\left(-\frac{1}{\sigma^2}W^T\right)z + z^T\left(\frac{1}{\sigma^2}WW^T + I\right)z\right)\right).
    \end{align}

    We can rewrite this in quadratic form. Let $v = \begin{bmatrix} x \\ z \end{bmatrix}$. Then:
    \begin{align}
      p(x, z | W) &\propto \exp\left(-\frac{1}{2}\begin{bmatrix} x^T & z^T \end{bmatrix} \begin{bmatrix} \frac{1}{\sigma^2}I & -\frac{1}{\sigma^2}W^T \\ -\frac{1}{\sigma^2}W & \frac{1}{\sigma^2}WW^T + I \end{bmatrix} \begin{bmatrix} x \\ z \end{bmatrix}\right).
    \end{align}

    This has the form of a multivariate Gaussian distribution:
    \begin{equation}
      p(v | W) \propto \exp\left(-\frac{1}{2}(v - \mu)^T\Sigma^{-1}(v - \mu)\right),
    \end{equation}

    with $v = \begin{bmatrix} x \\ z \end{bmatrix}$, $\mu = 0$, and precision matrix:
    \begin{equation}
      \Sigma^{-1} = \begin{bmatrix} \frac{1}{\sigma^2}I & -\frac{1}{\sigma^2}W^T \\ -\frac{1}{\sigma^2}W & \frac{1}{\sigma^2}WW^T + I \end{bmatrix}.
    \end{equation}

    Remember that if we write a multivariate Gaussian in partitioned form,
    \begin{equation}
      \begin{bmatrix} x \\ z \end{bmatrix} \sim \mathcal{N}\left(\begin{bmatrix} \mu_x \\ \mu_z \end{bmatrix}, \begin{bmatrix} \Sigma_{xx} & \Sigma_{xz} \\ \Sigma_{zx} & \Sigma_{zz} \end{bmatrix}\right),
    \end{equation}

    then the marginal distribution $p(x)$ (integrating over $z$) is given by
    \begin{equation}
      x \sim \mathcal{N}(\mu_x, \Sigma_{xx}).
    \end{equation}

    For probabilistic PCA we assume $\mu_x = 0$, but we partitioned $\Sigma^{-1}$ instead of $\Sigma$. To get $\Sigma$ we can use a partitioned matrix inversion formula:
    \begin{equation}
      \Sigma = \begin{bmatrix} \frac{1}{\sigma^2}I & -\frac{1}{\sigma^2}W^T \\ -\frac{1}{\sigma^2}W & \frac{1}{\sigma^2}WW^T + I \end{bmatrix}^{-1} = \begin{bmatrix} W^TW + \sigma^2 I & W^T \\ W & I \end{bmatrix},
    \end{equation}

    which gives that the solution to integrating over $z$ is
    \begin{equation}
      x | W \sim \mathcal{N}(0, W^T W + \sigma^2 I).
    \end{equation}
  \end{proof}

  By taking the log-likelihood, the loss of a single sample is 
  \begin{equation}
    L(x \mid W, \mu, \sigma) = \frac{1}{2} (x - \mu)^T (W^T W + \sigma^2 I)^{-1} (x - \mu)
  \end{equation} 
  and the expected loss is 
  \begin{equation}
    \mathbb{E}_x [ L(x \mid W, \mu, \sigma)] = \int L(x \mid W, \mu, \sigma) \, p(x) \,dx
  \end{equation}
  Likewise, the empirical loss for a dataset of $n$ elements $\{x^{(i)}\}_{i=1}^n$ is 
  \begin{equation}
    \frac{1}{2} \sum_{i=1}^n (x^{(i)} - \mu)^T (W^T W + \sigma^2 I)^{-1} (x^{(i)} - \mu)
  \end{equation}

  Optimizing this model is actually quite easy. 


  \begin{theorem}[MLE of PPCA Model]
    Given $x^{(i)} \sim X$ iid, the MLEs for $W, \mu, \sigma$ are 
    \begin{align}
      \mu^\ast & = \frac{1}{n} \sum_{i=1}^n x^{(i)} \\
      \sigma^{2 \ast} & = \frac{1}{d-k} \sum_{j=k+1}^d \lambda_j \\
      W^\ast & = R (\Sigma - \sigma^{2 \ast} I_d )^{1/2} V_k
    \end{align}
    where $\lambda_j$ are the eigenvalues of $X^T X$ in decreasing order, $V_k$ is the truncated orthogonal matrix consisting of the first $k$ columns of $V$ for SVD $X = U \Sigma V^T$, and $R$ is any unitary matrix.\footnote{Note that $W^{\ast}$ is not unique. Say that $W^\ast$ is an MLE, then, for any unitary $R \in \mathbb{R}^{d \times d}$, we have $W^{\ast T} W^\ast = (R W^\ast)^T (R W^\ast)$.} 
  \end{theorem}
  \begin{proof}
    We could also just differentiate the expected risk directly, but for no particular reason I will differentiate the empirical risk. 
    \begin{enumerate}
      \item For $\mu^\ast$, we use the matrix derivative $\frac{\partial}{\partial x} x^T A x = 2 A x$ and get 
      \begin{align}
        0 & = \frac{1}{2} \sum_{i=1}^n \frac{\partial}{\partial \mu} \left\{ (x^{(i)} - \mu)^T (W^T W + \sigma^2 I) (x^{(i)} - \mu) \right\} \\ 
          & = \frac{1}{2} \sum_{i=1}^n 2 (W^T W + \sigma^2 I) (x^{(i)} - \mu) \\ 
          & = (W^T W + \sigma^2 I) \left( \sum_{i=1}^n x^{(i)} - \mu \right)
      \end{align} 
      and since $W^T W + \sigma^2 I$ is positive definite, its inverse is positive definite and so it can only be $0$ when it is mapping the $0$ vector. So $\mu = \sum_{i=1}^n x^{(i)}$. 

      \item For $\sigma$, we first take a look at $C = \mathrm{Var}[X] = W W^T + \sigma^2 I$. It is the sum of positive semidefinite matrices that are also symmetric, so by the spectral theorem it is diagonalizable and has full rank $d$. But $W W^T$ is rank $k$, so $d - k$ of the eigenvalues of $W W^T$ is $0$, indicating that the same $d-k$ smallest eigenvalues of $C$ is $\sigma^2$. Therefore, we can take the smallest $d-k$ eigenvalues of our MLE estimator of $C$, which is $S$, and average them to get our MLE for $\sigma$. 
      \begin{equation}
        \hat{\sigma}^{2\ast} = \frac{1}{d-k} \sum_{j=k+1}^d \lambda_j
      \end{equation}
      \item TBD: Justify this again. For $W$, we can set $\mu^\ast$ first and then compute 
      \begin{equation}
        \widehat{\mathrm{Var}}(\mu^{\ast}) = S = \frac{1}{n} \sum_{i=1}^n (x^{(i)} - \mu^{\ast}) (x^{(i)} - \mu^{\ast})^T
      \end{equation}
      We can approximate $W W^T = C - \sigma^2 I \approx S - \hat{\sigma}^{2\ast} I$, and by further taking the eigendecomposition $C = U \Sigma U^T \implies W W^T = U (\Sigma - \sigma^2 I) U^T$ and cutting off the last $d-k$ smallest eigenvalues and their corresponding eigenvectors, we can get 
      \begin{equation}
        W^{\ast} = R (\Sigma - \sigma^{2 \ast} I_d )^{1/2} V_k
      \end{equation}
      where the $R$ just accounts for any unitary matrix. 
    \end{enumerate}
  \end{proof}

  The final fact is intuitive. We have introduced the error term $\epsilon$ that allows us to extend beyond the principal subspace. If we let $\epsilon$ vanish, the density model defined by PPCA becomes very sharp around these $d$ dimensions spanned by the columns of $W$. At $0$, we are reduced to regular PCA. 

  \begin{theorem}[PPCA as $\sigma \to 0$] 
    As $\sigma \rightarrow 0$, the MLE estimates of $W$ is equivalent to that of PCA. That is, when $W \in \mathbb{R}^{d \times k}$, 
    \begin{equation}
      W^\ast = \Sigma V_k
    \end{equation} 
    where $X = U \Sigma V^T$, and $V_k$ is the matrix formed by the first $k$ columns of $V$. That is, the conditional expected value of $z$ given $X$ becomes an orthogonal projection of $X - \mu$ onto the subspace spanned by the columns of $W$. 
  \end{theorem}
  \begin{proof}
    At $0$, our MLE of $W$ is simplified and we have 
    \begin{equation}
      x = W^\ast z + \mu^\ast + \epsilon = \Sigma V_k z + \mu^\ast
    \end{equation}
    which essentially reduces to regular PCA. 
  \end{proof}

  Intuitively, we can see that we are estimating the Gaussian, which corresponds to the mean squared distance from each $x^{(i)}$ to $\ell_k$. 

\subsection{Linear Factor Models} 

  A linear factor model is pretty much the same thing as PPCA. In fact, I don't even know why these two models are distinguished. The only two differences is notation and that linear factor models loosen the restriction that the covariance matrix of $\epsilon$ must be isotopic. 

  \begin{definition}[Factor Analysis] 
    Let $x$ be a random variable in $\mathbb{R}^d$. We would like to model it as 
    \begin{equation}
      x - \mu = \Lambda \eta + \epsilon
    \end{equation}
    where $\mu = \mathbb{E}[x]$ simply normalizes the distribution, $\eta \in \mathbb{R}^k$ are the \textbf{factors} that act as latent variables, $\Lambda \in \mathbb{R}^{d \times k}$ is the \textbf{loading matrix} that maps the factors to the samples, and $\epsilon$ is an error term.\footnote{$\eta$ is analogous to $V^T x \in \mathbb{R}^k$. $\Lambda$ is analogous to the embedding $V^T: \mathbb{R}^k \to \mathbb{R}^d$. It is also common notation to use $L$ and $f$ as the loading matrix and factors. } We have the following assumptions. 
    \begin{enumerate}
      \item $\mathbb{E}[\eta] = 0$. 
      \item $\Var[\eta] = I_k$ so that the factors are uncorrelated. 
      \item $\Lambda$ and $\epsilon$ are independent. 
      \item Sometimes, $\epsilon$ is assumed to be an isotropic Gaussian. 
    \end{enumerate}
  \end{definition} 

  Remember that given data matrix $X \in \mathbb{R}^{n \times d}$, we can take the SVD of it $X = U \Sigma V^T$ where the columns of $V$ represent the principal axes. Since $V$ is orthogonal, $V^T V = V V^T = I_d$, and so $X = X V V^T$. This may seem pretty trivial, but in PCA, we did data compression by removing the last columns of $V$ to get $X \approx X V_k V_k^T$. In a sense we can think of $XV_k \in \mathbb{R}^{n \times k}$ as the projection into the component space and then $(X V_k) V_k^T \in \mathbb{R}^{n \times d}$ as the embedding back into the original space. 

  If we were working with the data matrix $X \in \mathbb{R}^{n \times d}$, then we would be using right matrix multiplication, and so our model will look like 
  \begin{equation}
    X = \eta \Lambda + \epsilon
  \end{equation}
  where $\eta \in \mathbb{R}^{n \times k}$, $\Lambda \in \mathbb{R}^{k \times d}, \epsilon \in \mathbb{R}^{n \times d}$. 

  \begin{lemma}[Likelihood of Linear Factor Models]
    The likelihood of the linear factor model is 
    \begin{equation}
      p(x \mid \Lambda, \mu, \Sigma_\epsilon) 
    \end{equation}
  \end{lemma}
  \begin{proof}
    It should be clear to us that $X$ should be Gaussian since linear transformations of Gaussians are Gaussian. Therefore it suffices to compute its mean and variance. 
    \begin{enumerate}
      \item The mean can be computed exactly the same way for PPCA, which gives $\mathbb{E}[X] = \mu$. 
      \item The variance can be computed as 
      \begin{align} 
        \mathrm{Var}[X] & = \mathbb{E}[ (X - \mu)(X - \mu)^T ] \\
                        & = \mathbb{E}[ (W Z + \epsilon) (Z^T W^T + \epsilon^T)] \\
                        & = \mathbb{E}[W z z^T W^T] + \mathbb{E}[ \epsilon \epsilon^T] \\
                        & = W \mathbb{E}[ z z^T] W^T + \mathbb{E}[ \epsilon \epsilon^T] \\
                        & = W W^T + \mathrm{diag}(\sigma_1^2, \ldots, \sigma_d^2) 
      \end{align} 
    \end{enumerate}
  \end{proof}

  The $W, \mu$, and $\sigma_i$'s can be estimated using MLE methods. 

  \begin{theorem}[MLE of Factor Model]
    
  \end{theorem}

\subsection{Degrees of Freedom}

  
\subsection{Differences from Linear Regression and PCA} 

  

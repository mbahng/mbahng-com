\documentclass{article}

  % packages
    % basic stuff for rendering math
    \usepackage[letterpaper, top=1in, bottom=1in, left=1in, right=1in]{geometry}
    \usepackage[utf8]{inputenc}
    \usepackage[english]{babel}
    \usepackage{amsmath} 
    \usepackage{amssymb}
    \usepackage{bookmark}

    % extra math symbols and utilities
    \usepackage{mathtools}        % for extra stuff like \coloneqq
    \usepackage{mathrsfs}         % for extra stuff like \mathsrc{}
    \usepackage{centernot}        % for the centernot arrow 
    \usepackage{bm}               % for better boldsymbol/mathbf 
    \usepackage{enumitem}         % better control over enumerate, itemize
    \usepackage{hyperref}         % for hypertext linking
    \usepackage{xr-hyper}
    \usepackage{fancyvrb}          % for better verbatim environments
    \usepackage{newverbs}         % for texttt{}
    \usepackage{xcolor}           % for colored text 
    \usepackage{listings}         % to include code
    \usepackage{lstautogobble}    % helper package for code
    \usepackage{parcolumns}       % for side by side columns for two column code
    \usepackage{algorithm}
    \usepackage{algpseudocode}
    \usepackage{bbm}
    \algblock{Class}{EndClass}

    % page layout
    \usepackage{fancyhdr}         % for headers and footers 
    \usepackage{uniquecounter}
    \usepackage{lastpage}         % to include last page number in footer 
    \usepackage{parskip}          % for no indentation and space between paragraphs   
    \usepackage[T1]{fontenc}      % to include \textbackslash
    \usepackage{footnote}
    \usepackage{etoolbox}

    % for custom environments
    \usepackage{tcolorbox}        % for better colored boxes in custom environments
    \tcbuselibrary{breakable}     % to allow tcolorboxes to break across pages

    % figures
    \usepackage{pgfplots}
    \pgfplotsset{compat=1.18}
    \usepackage{float}            % for [H] figure placement
    \usepackage{tikz}
    \usepackage{tikz-cd}
    \usepackage{circuitikz}
    \usetikzlibrary{arrows, arrows.meta}
    \usetikzlibrary{positioning}
    \usetikzlibrary{calc}
    \usepackage{graphicx}
    \usepackage{caption} 
    \usepackage{subcaption}
    \captionsetup{font=small} 

    % for tabular stuff 
    \usepackage{dcolumn}

    \usepackage[nottoc]{tocbibind}
    \pdfsuppresswarningpagegroup=1
    \hfuzz=5.002pt                % ignore overfull hbox badness warnings below this limit

  % New and replaced operators
    \DeclareMathOperator{\Tr}{Tr}
    \DeclareMathOperator{\Sym}{Sym}
    \DeclareMathOperator{\Span}{span}
    \DeclareMathOperator{\elbo}{ELBO}
    \DeclareMathOperator{\std}{std}
    \DeclareMathOperator{\Cov}{Cov}
    \DeclareMathOperator{\Var}{Var}
    \DeclareMathOperator{\Corr}{Corr}
    \DeclareMathOperator{\pos}{pos}
    \DeclareMathOperator*{\argmin}{\arg\!\min}
    \DeclareMathOperator*{\argmax}{\arg\!\max}
    \newcommand{\qed}{\hfill$\blacksquare$}     % I like QED squares to be black

  % Custom Environments
    \newtcolorbox[auto counter, number within=section]{question}[1][]
    {
      colframe = orange!25,
      colback  = orange!10,
      coltitle = orange!20!black,  
      breakable, 
      title = \textbf{Question \thetcbcounter ~(#1)}
    }

    \newtcolorbox[auto counter, number within=section]{exercise}[1][]
    {
      colframe = teal!25,
      colback  = teal!10,
      coltitle = teal!20!black,  
      breakable, 
      title = \textbf{Exercise \thetcbcounter ~(#1)}
    }
    \newtcolorbox[auto counter, number within=section]{solution}[1][]
    {
      colframe = violet!25,
      colback  = violet!10,
      coltitle = violet!20!black,  
      breakable, 
      title = \textbf{Solution \thetcbcounter}
    }
    \newtcolorbox[auto counter, number within=section]{lemma}[1][]
    {
      colframe = red!25,
      colback  = red!10,
      coltitle = red!20!black,  
      breakable, 
      title = \textbf{Lemma \thetcbcounter ~(#1)}
    }
    \newtcolorbox[auto counter, number within=section]{theorem}[1][]
    {
      colframe = red!25,
      colback  = red!10,
      coltitle = red!20!black,  
      breakable, 
      title = \textbf{Theorem \thetcbcounter ~(#1)}
    } 
    \newtcolorbox[auto counter, number within=section]{corollary}[1][]
    {
      colframe = red!25,
      colback  = red!10,
      coltitle = red!20!black,  
      breakable, 
      title = \textbf{Corollary \thetcbcounter ~(#1)}
    } 
    \newtcolorbox[auto counter, number within=section]{proof}[1][]
    {
      colframe = orange!25,
      colback  = orange!10,
      coltitle = orange!20!black,  
      breakable, 
      title = \textbf{Proof. }
    } 
    \newtcolorbox[auto counter, number within=section]{definition}[1][]
    {
      colframe = yellow!25,
      colback  = yellow!10,
      coltitle = yellow!20!black,  
      breakable, 
      title = \textbf{Definition \thetcbcounter ~(#1)}
    } 
    \newtcolorbox[auto counter, number within=section]{example}[1][]
    {
      colframe = blue!25,
      colback  = blue!10,
      coltitle = blue!20!black,  
      breakable, 
      title = \textbf{Example \thetcbcounter ~(#1)}
    } 
    \newtcolorbox[auto counter, number within=section]{code}[1][]
    {
      colframe = green!25,
      colback  = green!10,
      coltitle = green!20!black,  
      breakable, 
      title = \textbf{Code \thetcbcounter ~(#1)}
    } 
    \newtcolorbox[auto counter, number within=section]{algo}[1][]
    {
      colframe = green!25,
      colback  = green!10,
      coltitle = green!20!black,  
      breakable, 
      title = \textbf{Algorithm \thetcbcounter ~(#1)}
    } 
    \definecolor{cverbbg}{gray}{0.93}
    \newenvironment{cverbatim}
      {\SaveVerbatim{cverb}}
      {\endSaveVerbatim
        \flushleft\fboxrule=0pt\fboxsep=.5em
        \colorbox{cverbbg}{%
          \makebox[\dimexpr\linewidth-2\fboxsep][l]{\BUseVerbatim{cverb}}%
        }
        \endflushleft
    }

    \definecolor{dkgreen}{rgb}{0,0.6,0}
    \definecolor{gray}{rgb}{0.5,0.5,0.5}
    \definecolor{mauve}{rgb}{0.58,0,0.82}
    \definecolor{lightgray}{gray}{0.93}
    \renewcommand{\algorithmiccomment}[1]{\hfill$\triangleright$\textcolor{blue}{#1}}

    % default options for listings (for code)
    \lstset{
      autogobble,
      frame=ltbr,
      language=Python,                           % the language of the code
      aboveskip=3mm,
      belowskip=3mm,
      showstringspaces=false,
      columns=fullflexible,
      keepspaces=true,
      basicstyle={\small\ttfamily},
      numbers=left,
      firstnumber=1,                        % start line number at 1
      numberstyle=\tiny\color{gray},
      keywordstyle=\color{blue},
      commentstyle=\color{dkgreen},
      stringstyle=\color{mauve},
      backgroundcolor=\color{lightgray}, 
      breaklines=true,                      % break lines
      breakatwhitespace=true,
      tabsize=3, 
      xleftmargin=2em, 
      framexleftmargin=1.5em, 
      stepnumber=1
    }

  % Page style
    \pagestyle{fancy}
    \fancyhead[L]{GAN}
    \fancyhead[C]{Muchang Bahng}
    \fancyhead[R]{Spring 2023} 
    \fancyfoot[C]{\thepage / \pageref{LastPage}}
    \renewcommand{\footrulewidth}{0.4pt}          % the footer line should be 0.4pt wide
    \renewcommand{\thispagestyle}[1]{}  % needed to include headers in title page
%    \renewcommand{\thefootnote}{\arabic{footnote}}

  % external documents 
    \externaldocument[algo-]{../../Computer_Science/DSA/paper}[../../Computer_Science/DSA/paper.pdf]

\begin{document}

\tikzset{every picture/.style={line width=0.75pt}} 

\title{Generative Adversarial Networks} 
\author{Muchang Bahng}
\date{Spring 2023}

\maketitle
\tableofcontents
\pagebreak 


Unlike my machine learning notes, which focuses mostly on the theoretical soundness of classical (e.g. pre-deep and interpretable) models, the theory of deep neural networks have not been developed as well yet. Furthermore, the recency of developments, especially in the post-2010s, results in a pretty nonlinear\footnote{No pun intended.} timeline that is difficult to categorize effectively. Before I give my motivation, the direct prerequisites for deep learning are basic knowledge of my notes in probability theory, machine learning, and real analysis. 

Therefore, after much thought, I think organizing my notes in chronological order would be best. It turned out that in the early days of deep learning, most researchers like Andrew Ng stated that he focused on supervised models,\footnote{He states this in his podcast with Lex Fridman.} and it wasn't until the 2010s that the development of unsupervised models burgeoned. 

\begin{enumerate}
  \item We start by introducing \textit{multilayered perceptrons} (MLPs), which build upon generalized linear models (GLMs) that we have went over in machine learning. This is pretty much the ``foundational'' model that we will build on. We then talk a about practical methodologies regarding training and control. 

  \item Then we introduce the other two architectures. The \textit{convolutional neural network} (CNN) gives us a scalable way to perform sparse matrix multiplication efficiently, by taking advantage of locality. Then in \textit{recurrent neural networks} (RNNs), we are not limited to a single $n$-dimensional vector input and are able to process a time series of inputs.  
    
  \item With these building blocks, we are able take and two neural networks together to create an encoder-decoder model. The two main applications of this was dimensionality reduction with \textit{autoencoders}, followed by \textit{seq2seq} for machine translation between sequences of words.  

  \item The practical success of autoencoders led to the development of not just density estimation models, but \textit{generative} ones that seek to sample from the learned distribution. These generative models were ``deep'' extensions of the classical linear factor models resulting in \textit{restricted Boltzmann machines (RBMs)} and \textit{variational autoencoders (VAEs)}, which attempted to explicitly model an approximation of the true data generating distribution. More specifically, RBMs were \textit{energy models} that borrowed ideas from physics to learn a distribution of the form $p_X (x) = \frac{1}{Z} e^{-f_\theta (x)}$. To sample from this, researchers already had access to the established Markov Chain Monte Carlo (MCMC) algorithms designed for this exact problem. As for VAEs, these were trained and sampled through \textit{variational inference} methods. 

  \item In 2014, a completely new architecture composed of 2 neural networks competing each other gave rise to \textit{generative adversarial networks} (GANs) which blew RBMs and VAEs away. A similar architecture followed with \textit{generative stochastic networks} (GSNs).

  \item In 2016, \textit{normalizing flow models}, which attempted to model a smooth function that transformed simple latent variables to the true data generating distribution, were invented by Google DeepMind and surpassed GANs. 

  \item In 2017, \textit{attention} took the world by storm, which had drastically improved machine translation in seq2seq, and the resulting \textit{self-attention} plus the \textit{transformer} architectures led to the success of OpenAI's ChatGPT. 
\end{enumerate} 

Again, I emphasize that the math in these notes are not very advanced. However, implementing these simple models and training algorithms from scratch is a challenge in itself. I will go through implementing everything from scratch as if we were building a mini-version of PyTorch as we learn new topics. My implementations can be found \href{https://github.com/mbahng/pyember}{here}. 


\section{Generative Adversarial Networks} 

  The year after Kigma published his paper on VAEs, Ian Goodfellow in 2014 published \cite{gans} on \textit{generative adversarial networks}. Unlike the current generative models, which relied on ``fancy'' methods like variational Bayes or contrastive divergence, Goodfellow claimed that we can just do this with vanilla MLPs. The general idea is to approximate the true density $p^\ast (x)$, you are trying to build a function (a neural network) $G_\alpha$ that is a transformation of a simple latent variable, e.g. $Z \sim \mathcal{N}(0, 1)$, such that the pdf of $G_\alpha (Z)$ is similar to $p^\ast (x)$. 

  Remember that for the generative models that we have so far, we were able to \textit{explicitly model} the true pdf. 
  \begin{enumerate}
    \item \textit{Explicit and Analytical}. The simplest types of densities can be analytically written and can be sampled from directly (e.g. Gaussian with inverse CDF, Box-Muller transform, and even GMMs). 

    \item \textit{Explicit and Approximate}. More complicated models such as the RBM and VAE cannot be analytically written, but can still be directly sampled from with probabilities. Boltzmann machines approximate the explicit density as $p_\theta (x) = \frac{1}{Z} e^{-E(x, z)}$ for some quadratic term $-E(x, z)$. VAEs construct another family of posterior distributions $\{q_\lambda\}$ where each $\lambda$ is the output of an encoder neural network $E_\alpha$. Later on, we will see that normalizing flows and autoregressive models also fall in this category. 

    \item \textit{Implicit}. GANs approximate the true pdf through a transformation of random variables and are implicit in the way that you can't estimate the probabilities but you can still sample from them.\footnote{Given $Z \sim \mathcal{N}(0, 1)$ and the sine function $f(z) = \sin(z)$, it's hard to analytically compute the pdf of $X = f(Z)$. However, to sample from this, we can just sample $z$ from $Z$ and then transform it through $f$. }

    \begin{figure}[H]
      \centering 
      \includegraphics[scale=0.4]{img/implicit_gen_models.png}
      \caption{ You essentially want to take some simple latent distribution and construct a differentiable a \textbf{generator model} that maps it to a more complicated distribution in the observable space. } 
      \label{fig:implicit_gen_models}
    \end{figure}
  \end{enumerate}

  The problem with VAEs is that they tend to generate blurry images, which is the result of optimizing a variational lower bound rather than the true objective. This makes it easy to identify whether a given image is from the true dataset or has been artificially generated. In contrast, GANs generate high-resolution images because we are directly optimizing the true objective. 

  \begin{definition}[Generative Adversarial Networks]
    A \textbf{generative adversarial network (GAN)} is a nonlinear latent-variable model consisting of 
    by optimizing a pair of generator and disciminator neural networks, which play a game where one tries to bea the other. 
    \begin{enumerate} 
      \item A latent random variable $Z$ with pdf $p(z)$. 
      \item The \textbf{generator} tries to generate fake samples to fool the discriminator. We sample from a latent space $\mathbf{z}$ and run that through the neural network to get $\mathbf{x} = \mathcal{G}_{\theta_g} (\mathbf{z})$.i It should be differentiable, but does not have to be invertible.   
      \item The \textbf{discriminator} tries to distinguish between real and fake samples, like a critic which can tell from real from fake. It should also be differentiable, and its output is essentially $0 \leq \mathcal{D}_{\theta_d}(\mathbf{x}) \leq 1$, with a value of $1$ if real, $0$ if fake.  
    \end{enumerate}
    We want to train these two models against each other, and in the end, we throw $\mathcal{D}$ away, since it's only role is to force $\mathcal{G}_{\theta_g}$ to work harder, which leaves us with a really good generative $\mathcal{D}_{\theta_g}$.  
    
    \begin{figure}[H]
      \centering 
      \includegraphics[scale=0.4]{img/general_gan_architecture.png}
      \caption{} 
      \label{fig:gan_arch}
    \end{figure}
  \end{definition}

  \begin{definition}[GAN Loss]
    Now each of these networks will have its own set of parameters which we have to optimize. We want to optimize them by maximizing the likelihood such that the model says ``real" to the samples from the world and ``fake" to the generated samples. This leads to 
    \begin{equation}
      \mathcal{L}(\theta_d, \theta_g) = V(\mathcal{D}, \mathcal{G}) = \underbrace{\mathbb{E}_{x \sim real} \big[ \log \mathcal{D}_{\theta_d}(\mathbf{x}) \big]}_{\substack{\text{log-prob that } \mathcal{D} \text{ correctly} \\ \text{predicts real data as real}}} + \underbrace{\mathbb{E}_{\mathbf{z}} \big[ \log \big( 1 - \mathcal{D}_{\theta_d} ( \mathcal{G}_{\theta_g} (\mathbf{z}))\big)\big]}_{\substack{{log-prob that } \mathcal{D} \text{ correctly} \\ \text{predicts generated data as fake} }}
    \end{equation}
    Therefore, the discrimiator is trying to maximize its reward (to get max value of $0$), and the generator is trying to minimize the disciminator's reward (pulling this log probability down to $-\infty$). This is known as a \textbf{minimax} optimization problem, and we have to find 
    \begin{equation}
      \min_{\theta_g} \max_{\theta_d} V(\mathcal{D}_{\theta_d}, \mathcal{G}_{\theta_g})
    \end{equation}
    which is some saddle point. 
  \end{definition}

  Before we even talk about evaluating the gradients for this, let's prove a result on the properties of such an optimum.   

  \begin{theorem}[Global Optimality] 
    Given the minimax loss above, for a fixed $\mathcal{G}$, the optimal discriminator $\mathcal{D}_G^\ast$ is given by 
    \begin{equation}
      \mathcal{D}_G^\ast (\mathbf{x}) = \frac{p(\mathbf{x} \mid \text{real})}{p(\mathbf{x} \mid \text{real}) + p(\mathbf{x} \mid \text{synthesized})}
    \end{equation}
    Therefore, the global minimum of the training criterion, $\max_{\mathcal{D}} V(\mathcal{D}, \mathcal{G})$ is achieved if and only if 
    \begin{equation}
      p(\mathbf{x} \mid \text{real}) = p(\mathbf{x} \mid \text{synthesized})
    \end{equation}
  \end{theorem}
  \begin{proof} 
    We first have 
    \begin{equation}
      V(\mathcal{D}, \mathcal{G}) = \mathbb{E}_{\mathbf{x} \sim real} \log \mathcal{D}_{\theta_d} (\mathbf{x}) + \mathbb{E}_{\mathbf{z}} \log \big( 1 - \mathcal{D}_{\theta_d} (\mathcal{G}_{\theta_g} (\mathcal{z}))\big) 
    \end{equation}
    Since $\mathcal{G}$ is fixed (i.e. $\theta_g$ is fixed) and acting, we can write the second expectation with respect to the probability measure induced by $\mathcal{G}$.
    \begin{align*} 
      V(\mathcal{D}, \mathcal{G}) & = \mathbb{E}_{\mathbf{x} \sim real} \log \mathcal{D}_{\theta_d} (\mathbf{x}) + \mathbb{E}_{\mathbf{w} \sim \text{fake}} \log \big( 1 - \mathcal{D}_{\theta_d} (\mathbf{w}) \big) \\   
                                      & = \int p(\mathbf{x} \mid \text{real}) \log \mathcal{D}_{\theta_d}(\mathbf{x}) \,d\mathbf{x}+ \int p(\mathbf{x} \mid \text{fake}) \, \log \big( 1 - \mathcal{D}_{\theta_d} (\mathbf{x}) \big) \,d\mathbf{x} 
    \end{align*}
    where the $\mathbf{x}$ in the second integral is a dummy variable. Taking the derivative w.r.t. $\mathbf{x}$ and setting equal to $0$ gives 
    \begin{equation}
      p(\mathbf{x} \mid \text{real}) \frac{1}{\mathcal{D}_{\theta_d} (\mathbf{x})} + p(\mathbf{x} \mid \text{fake}) \cdot \frac{-1}{1 - \mathcal{D}_{\theta_d}(\mathbf{x})} = 0
    \end{equation}
    implies that 
    \begin{equation}
      \mathcal{D}_{\theta_d}( \mathbf{x}) = \frac{p(\mathbf{x} \mid \text{real})}{p(\mathbf{x} \mid \text{real}) + p(\mathbf{x} \mid \text{fake})}
    \end{equation}
  \end{proof} 

  If the discriminator $\mathcal{D}$ is optimal, then the generator is minimizing the \textbf{Jensen-Shannon divergence} between the real and generated (model distributions). However, $\mathcal{D}$ is not optimal in practice since we have limited computational resources, the loss is non-convex, etc. 

  \begin{figure}[H]
    \centering 
    \includegraphics[scale=0.3]{img/equil.png}
    \caption{} 
    \label{fig:nash_equilibrium}
  \end{figure}

  Now we can talk about evaluating the gradients and optimizing it. What is the gradient of such a loss function? This is quite easy since the expectations are over the true data distribution and the known prior $z$, and so we can swap gradients and expectations accordingly. 

  \begin{lemma}[Gradients of GAN Loss]
    
  \end{lemma} 
  \begin{proof}
    
  \end{proof}

  Note that the generator has no effect on the probability of $\mathcal{D}$ correctly identifying real images, so it only focuses on the latter term. 


  \begin{algo}[Alternative Gradient Descent]
    The idea is to train both models simultaneously via SGD using mini-batches consisting of some generated samples and some real-world samples, which is called the \textbf{alternative gradient descent algorithm}. It is usually the case that the generator does better than the disciminator, so we sometimes make $k > 1$ to allow more steps for training. 
    \begin{algorithm}[H]
      \caption{Minibatch stochastic gradient descent training of generative adversarial nets. The number of steps to apply to the discriminator, $k$, is a hyperparameter. We used $k = 1$, the least expensive option, in our experiments.}
      \label{alg:alt_gd}
      \begin{algorithmic}[1] 
        \For{number of training iterations}
            \For{$k$ steps}
                \State Sample minibatch of $m$ noise samples $\{z^{(1)}, \ldots, z^{(m)}\}$ from noise prior $p_g(z)$.
                \State Sample minibatch of $m$ examples $\{x^{(1)}, \ldots, x^{(m)}\}$ from data generating distribution $p_{\text{data}}(x)$.
                \State Update the discriminator by ascending its stochastic gradient:
                \[
                    \nabla_{\theta_d} \frac{1}{m} \sum_{i=1}^m \left[\log D\left(x^{(i)}\right) + \log\left(1-D\left(G\left(z^{(i)}\right)\right)\right)\right].
                \]
            \EndFor
            \State Sample minibatch of $m$ noise samples $\{z^{(1)}, \ldots, z^{(m)}\}$ from noise prior $p_g(z)$.
            \State Update the generator by descending its stochastic gradient:
            \[
                \nabla_{\theta_g} \frac{1}{m} \sum_{i=1}^m \log\left(1-D\left(G\left(z^{(i)}\right)\right)\right).
            \]
        \EndFor
      \end{algorithmic}
    \end{algorithm}
    The gradient-based updates can use any standard gradient-based learning rule. We used momentum in our experiments.
  \end{algo} 

  There is a vanishing gradient problem in GANs. For instance, assume that the model $\mathcal{G}_{\theta_g}$ has very bad parameters, and it generates very bad samples that the disciminator can detect very well. Then, $\mathbb{E}_{\mathbf{z}} \big[ \log \big( 1 - \mathcal{D}_{\theta_d} (\mathcal{G}_{\theta_g}))\big) \big]$ will be very close to $0$, and the generator's cost will be very flat. Therefore, the gradient would die out and the generator can't improve! For example, if $\mathcal{D}$ was just a sigmoid function, then we can approximate the gradient of the expectation with a sample of the gradient, which would die out. 

  \begin{align*} 
    \nabla_{\theta_g} V(\mathcal{D}, \mathcal{G}) & = \nabla_{\theta_g} \mathbb{E}_{z \sim q(\mathbf(\mathbf{z})} \big[ \log \big( 1 - \mathcal{D}(\mathcal{G}(\mathbf{z}))\big) \big] \\
                                                  & \approx \nabla_a \log \big( 1 - \sigma(a)\big) \\
                                                  & = - \frac{\sigma(a) \, \big( 1 - \sigma(a)\big)}{1 - \sigma(a)} = - \sigma(a) = - \mathcal{D}(\mathcal{G}(\mathbf{z})) 
  \end{align*} 

  and so the gradient goes to $0$ if $\mathcal{D}$ is confident, i.e. $\mathcal{D}(\mathcal{G}(\mathbf{z})) \rightarrow 0$. Therefore, we can modify the cost for the generator term by changing the cost to 
  \begin{equation}
    \mathbb{E}_{\mathbf{z}} \log \big( 1 - \mathcal{D}_{\theta_d} ( \mathcal{G}_{\theta_g} (\mathbf{z}))\big)
  \end{equation}
  and trying to minimize it. 


  \begin{theorem}[Convergence]
    
  \end{theorem}

  \begin{algo}
    An implementation of a GAN in PyTorch is \href{code/gan.html}{here}. 
  \end{algo}

\subsection{Conditional GAN} 

\subsection{CycleGAN}

\subsection{Deep Convolution GAN (DCGAN)} 


\bibliographystyle{alpha}
\bibliography{./bibfile}
\end{document}


Unlike unsupervised learning, which comes in many different shapes and forms (anomaly detection, feature extraction, density estimation, dimensionality reduction, etc.), supervised learning comes in a much cleaner format. In supervised learning, we consider an input space $\mathcal{X}$ and an output space $\mathcal{Y}$. We assume that there exists some unknown measure $\mathbb{P}$ over $\mathcal{X} \times \mathcal{Y}$, making this some probability space. We then assume that some data $\mathcal{D} = \{(x^{(i)}, y^{(i)})\}$ is generated sampled \textit{independently and identically (iid)} from $\mathbb{P}$. Now this assumption is quite strong and is almost always not the case, as different data can be correlated, but we will relax this assumption later. Let's formally construct this from the bottom up. 

\begin{enumerate}
  \item We start off with a general probability space $(\Omega, \mathcal{F}, \mathbb{P})$. This is our model of the world and everything that we are interested in. 

  \item A measurable function $X: \Omega \rightarrow \mathcal{X}$ extracts a set of features, which we call the \textbf{covariates} and induces a probability measure on $\mathcal{X}$, say $\mathbb{P}_X$. 

  \item Another measurable function $Y: \Omega \rightarrow \mathcal{Y}$ extracts another set of features called the \textbf{labels} and induces another probability measure on $\mathcal{Y}$, the \textbf{label set}, say $\mathbb{P}_Y$. 

  \item At this point the function $X \times Y$ is all we are interested in, and we throw away $\Omega$ since we only care about the distribution over $\mathcal{X} \times \mathcal{Y}$. 

  \item We model the generation of data from $\Omega$ by sampling $N$ samples from $\mathbb{P}_{X \times Y}$, which we assume to be iid (this assumption will be relaxed later). This gives us the \textbf{dataset} 
    \[\mathcal{D} = \{(\mathbf{x}^{(i)}, \mathbf{y}^{(i)}) \}_{i=1}^N\]
\end{enumerate}

Now our goal is to construct a function $f: \mathcal{X} \rightarrow \mathcal{Y}$ that predicts $Y$ from $X$, but we want to define some measure of how good our function is. We can use a loss function $L$ to talk about this. 

\begin{definition}[Risk]
  The \textbf{risk}, or \textbf{expected risk}, of function $f$ is defined as 
  \begin{equation}
    R(f) = \mathbb{E}_{X \times Y} [ L(Y, f(X))] = \int_{\mathcal{X} \times \mathcal{Y}} L(y, f(x)) \,d\mathbb{P}(x, y)
  \end{equation}
\end{definition}

Clearly, we don't know what this risk is since we don't know the true measure $\mathbb{P}$, so we try to approximate it with the \textit{empirical risk}. 

\begin{definition}[Empirical Risk]
  The \textbf{empirical risk} of function $f$ is defined as 
  \begin{equation}
    \hat{R}_n(f) = \frac{1}{n} \sum_{i=1}^n L(y^{(i)}, f(x^{(i)}))
  \end{equation}
\end{definition}

\begin{definition}[Generlize]
  A function $f$ is said to \textbf{generalize} if 
  \begin{equation}
    \lim_{n \rightarrow +\infty} \hat{R}_n (f) = R(f)
  \end{equation}
\end{definition}

This gives us a way of computing with the actual data. Now two questions arise from this. First, how do we even choose the loss function $L$? Second, how do we know that the empirical risk is a good approximation of the true risk? The first question can be quite convoluted, but we introduce it with decision theory. The second has a simple answer with concentration of measure.  


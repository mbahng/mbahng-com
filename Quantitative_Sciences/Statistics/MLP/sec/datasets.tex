\section{Popular Benchmark Datasets} 

  For here, we will go over some of the main datasets that are used in deep learning. 

  \begin{definition}[MNIST and Fashion MNIST]
    The MNIST dataset consists of 60k training images and 10k test images of handwritten digits. The Fashion MNIST dataset consists of 60k training images and 10k test images of clothing items. These are considered quite easy with the basic benchmarks: 
    \begin{enumerate} 
      \item Linear classifiers can reach past 90\% accuracy. 
      \item A 2 layer MLP can reach up to 97\% accuracy. 
      \item A CNN can reach up to 99\% accuracy. 
    \end{enumerate}
  \end{definition}

  \begin{definition}[CIFAR10 and CIFAR 100]
    The CIFAR10 dataset consists of 60k 32x32 color images in 10 classes, with 6k images per class. The CIFAR100 dataset consists of 60k 32x32 color images in 100 classes, with 600 images per class. These are considered quite hard with the basic benchmarks: 
    \begin{enumerate} 
      \item Linear classifiers can reach past 40\% accuracy. 
      \item A 2 layer MLP can reach up to 60\% accuracy. 
      \item A CNN can reach up to 80\% accuracy. 
    \end{enumerate}
  \end{definition}

  \begin{definition}[ImageNet]
    The ImageNet dataset, created at Stanford by Fei-Fei Li \cite{ImageNet}, consists of 1.2 million training images and 50k validation images in 1000 classes. This is considered very hard with the basic benchmarks. 
  \end{definition}

  Creating your own custom dataset with spreadsheets or images is easy.\footnote{https://pytorch.org/tutorials/beginner/data\_loading\_tutorial.html} Loading it to a dataloader that shuffles and outputs minibatches of data is trivial. However, when doing so, you should pay attention to a couple things. 
  \begin{enumerate} 
    \item Batch size: The dataloader stores the dataset (which can be several hundred GBs) in the drive, and extracts batches into memory for processing. You should set your batch sizes so that they can fit into the GPU memory, which is often smaller than the CPU memory. 
  \end{enumerate}


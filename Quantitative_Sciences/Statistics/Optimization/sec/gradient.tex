\section{Gradient Methods} 

  The first thing you learn about gradients in multivariate calculus is that they point in the step of steepest ascent. Generally, you can think of the function you're trying to minimize as a ``landscape.'' This inspires a greedy approach to simply walk in this direction $\nabla f(x)$ to maximize a function (or walk in the opposite direction $-\nabla f(x)$). This gives the following. 

  \begin{algo}[Gradient Descent]
    Generally speaking, at every point you should point in the direction of steepest descent, and move in that direction. The only question that remains is: how far? This is manually adjusted by the \textit{learning rate}. 

    \begin{algorithm}[H]
    \label{alg:gradient-descent}
    \begin{algorithmic}[1]  % The [1] adds line numbers
      % Input/Output documentation
      \Require{Function $f(x)$, initialization $x_0$, learning rate}
      \Procedure{GradientDescent}{$f, x_0$}
        \State $x \gets x_0$ 
        \While{not converged} 
          \State $x \gets x - \eta \nabla f(x)$
        \EndWhile
        \State \Return $x$ 
      \EndProcedure
    \end{algorithmic}
    \end{algorithm} 
  \end{algo}

\subsection{Newton-Raphson Method for Root Finding} 

  Before we talk more about gradient descent methods, let's take a look at one of the simplest numerical root-finders.\footnote{Optimization and root finding are very similar, since to optimize $f$ you should first find a root of $f^\prime$.}

  \begin{theorem}[Convergence]
    Given a differentiable function $f: \mathbb{R} \to \mathbb{R}$, the sequence $(x_n)$ defined 
    \begin{equation}
      x_n = x_{n-1} - \frac{f(x_{n-1})}{f^\prime (x_{n-1})}
    \end{equation}
    for any $x_0 \in \mathbb{R}$, converges to a root of $f$. 
  \end{theorem}
  \begin{proof}
    Need to verify this. 
  \end{proof}

  \begin{algo}[Newton-Raphson Method]
    \begin{algorithm}[H]
    \label{alg:newton-raphson}
    \begin{algorithmic}[1]  % The [1] adds line numbers
      % Input/Output documentation
      \Require{Function $f(x)$, initial guess $x_0$}
      \Procedure{NewtonRaphson}{$f, f', x_0, \epsilon, N_{max}$}

      \State $x \gets x_0$ 
      \While{ not converged} 
        \If{$f^\prime (x) = 0$} 
          \State Retry with another $x_0$.  
        \EndIf
        \State $x \gets \frac{f(x)}{f^\prime (x)}$
      \EndWhile
          
      \State \Return $x$ 
      \EndProcedure
    \end{algorithmic}
    \end{algorithm}
  \end{algo}

\subsection{Stochastic Gradient Descent} 

  Now let's go back to gradient descent. Usually, in machine learning, we are trying to optimize a parameteric model $\mathcal{P} = \{ \mathbb{P}_\theta \mid \theta \in \Theta\}$ over a dataset $\mathcal{D} = \{x_i\}_{i=1}^n$., For example, our estimator in the maximum-likelihood approach will be 
  \begin{equation}
    \delta(\mathcal{D}) = \argmax_{\theta} L(\theta \mid \mathcal{D}) = \log p(\mathcal{D} \mid \theta)
  \end{equation}
  for some loss function $L$. 

  If we assume that the samples are iid, then we can decompose the gradient as 
  \begin{equation}
    \nabla_\theta \log{p(\mathcal{D} \mid \theta)} = \sum_i \nabla_\theta \log{p(d_i \mid \theta)}
  \end{equation}
  which scales linearly with the size of a dataset. This is not scalable with extremely large datasets, and so we must remove this $O(n)$ term. Intuitively, we can think of approximating this gradient by taking a minibatch $b$ of $\mathcal{D}$ and computing the gradient only across that minibatch. 

  \begin{theorem}[Minibatch Gradient is an Unbiased Estimator]
    Let us take a minibatch of $b$ samples $\mathcal{B} \subset \mathcal{D}$ without replacement, where $b << D$. Then, our approximation of the gradient of the log likelihood
    \begin{equation}
       \nabla_\theta \log{p (\mathcal{B} \mid \theta)} \coloneqq \frac{1}{b} \sum_{x \in \mathcal{B}} \nabla_\theta \log{p(x \mid \theta)}
    \end{equation}
    is an unbiased estimator of the true gradient $\nabla_\theta \log{p(\mathcal{D} \mid \theta)}$. That is, setting $\mathcal{M}$ as a random variable of samples over $\mathcal{D}$, we have 
    \begin{equation}
      \mathbb{E}_{\mathcal{M}} [\nabla \mathcal{L}_{\mathcal{M}} (\mathbf{w})] = \nabla \mathcal{L} (\mathbf{w})
    \end{equation}
  \end{theorem}
  \begin{proof}
    We use linearity of expectation for all $\mathcal{M} \subset \mathcal{D}$ of size $M$. 
  \end{proof} 

  This also has the additional advantage of saving memory. You don't have to load in the gradients for the whole dataset (which may be a few TB), and can allocate just enough memory for each batch (perhaps a few GB). 

  \begin{algo}[Stochastic Gradient Descent]

    \begin{algorithm}[H]
    \label{alg:stochacistic-gradient-descent}
    \begin{algorithmic}[1]  % The [1] adds line numbers
      % Input/Output documentation
      \Require{Function $L(\theta)$, initialization $\theta_0$, learning rate, batch size $b$, dataset $\mathcal{D}$}
      \Procedure{StochasticGradientDescent}{$f, \theta_0, b$}
        \State $\theta \gets \theta_0$ 
        \While{not converged} 
          \State Sample minibatch $\mathcal{B} \subset \mathcal{D}$. 
          \State $\theta \gets \theta - \eta \nabla_\theta \log p(\mathcal{B} \mid \theta)$ 
        \EndWhile
        \State \Return $\theta$
      \EndProcedure
    \end{algorithmic}
    \end{algorithm} 
  \end{algo} 

  \begin{example}[Linear Regression]
    We have assumed knowledge of gradient descent in the back propagation step in the previous section, but let's revisit this by looking at linear regression. Given our dataset $\mathcal{D} = \{\mathbf{x}^(n), y^{(n)}\}$, we are fitting a linear model of the form 
    \begin{equation}
      f(\mathbf{x}; \mathbf{w}, b) = \mathbf{w}^T \mathbf{x} + b
    \end{equation} 
    The squared loss function is 
    \begin{equation}
      \mathcal{L}(\mathbf{w}, b) = \frac{1}{2} \sum_{n=1}^N \big( y - f(\mathbf{x}; \mathbf{w}, b) \big)^2 = \frac{1}{2} \sum_{n=1}^N \big( y - (\mathbf{w}^T \mathbf{x} + b) \big)^2  
    \end{equation}
    If we want to minimize this function, we can visualize it as a $d$-dimensional surface that we have to traverse. Recall from multivariate calculus that the gradient of an arbitrary function $\mathcal{L}$ points in the steepest direction in which $\mathcal{L}$ increases. Therefore, if we can compute the gradient of $\mathcal{L}$ and step in the \textit{opposite direction}, then we would make the more efficient progress towards minimizing this function (at least locally). The gradient can be solved using chain rule. Let us solve it with respect to $\mathbf{w}$ and $b$ separately first. Beginners might find it simpler to compute the gradient element-wise. 
    \begin{align}
      \frac{\partial}{\partial w_j} \mathcal{L}(\mathbf{w}, b) 
      & = \frac{\partial}{\partial w_j} \bigg(\frac{1}{2} \sum_{n=1}^N \Big( f (\mathbf{x}^{(n)}; \mathbf{w}, b) - y^{(n)} \Big)^2 \bigg) \\
      & = \frac{1}{2} \sum_{n=1}^N \frac{\partial}{\partial w_j} \Big( f(\mathbf{x}^{(n)}; \mathbf{w}, b) - y^{(n)}\Big)^2 \\
      & = \frac{1}{2} \sum_{n=1}^N 2 \Big( f(\mathbf{x}^{(n)}) - y^{(n)}\Big) \cdot \frac{\partial}{\partial w_j} \big( f(\mathbf{x}^{(n)}; \mathbf{w}, b) - y^{(n)} \big) \\
      & = \frac{1}{2} \sum_{n=1}^N 2 \Big( f(\mathbf{x}^{(n)}) - y^{(n)}\Big) \cdot \frac{\partial}{\partial w_j} \big( \mathbf{w}^T \mathbf{x}^{(n)} + b - y^{(n)} \big) \\
      & = \sum_{n=1}^N \big( f(\mathbf{x}^{(n)}; \mathbf{w}, b) - y^{(n)}\big) \cdot x_j^{(n)} \;\;\;\;\;(\text{for } j = 0, 1, \ldots, d)
    \end{align}
    As for getting the derivative w.r.t. $b$, we can redo the computation and get 
    \begin{equation}
      \frac{\partial}{\partial w_j}\mathcal{L}(\mathbf{w}, b) = \sum_{n=1}^N \big( f (\mathbf{x}^{(n)}; \mathbf{w}, b) - y^{(n)}\big) 
    \end{equation}
    and in the vector form, setting $\boldsymbol{\theta} = (\mathbf{w}, b)$, we can set 
    \begin{align}
      \nabla \mathcal{L} (\mathbf{w}) & = \mathbf{X}^T (\hat{\mathbf{y}} - \mathbf{y}) \\
      \nabla \mathcal{L} (b) & = (\hat{\mathbf{y}} - \mathbf{y}) \cdot \mathbf{1}
    \end{align}
    where $\hat{\mathbf{y}}_n = f(\mathbf{x}^{(n)}; \mathbf{w}, b)$ are the predictions under our current linear model and $\mathbf{X} \in \mathbb{R}^{n \times d}$ is our design matrix. This can easily be done on a computer using a package like \texttt{numpy}. 
    
    Rather than updating the vector $\boldsymbol{\theta}$ in batches, we can apply \textbf{stochastic gradient descent} that works incrementally by updating $\boldsymbol{\theta}$ with each term in the summation. That is, rather than updating as a batch by performing the entire matrix computation by multiplying over $N$ dimensions,
    \begin{equation}
      \nabla \mathcal{L} (\mathbf{w}) = \underbrace{\mathbf{X}^T}_{D \times N} \underbrace{(\hat{\mathbf{y}} - \mathbf{y})}_{N \times 1}
    \end{equation}
    we can reduce this load by choosing a smaller subset $\mathcal{M} \subset \mathcal{D}$ of $M < N$ elements, which gives 
    \begin{equation}
      \nabla \mathcal{L}_{\mathcal{M}} (\mathbf{w}) = \underbrace{\mathbf{X}_{\mathcal{M}}^T}_{D \times M} \underbrace{(\hat{\mathbf{y}_{\mathcal{M}}} - \mathbf{y}}_{\mathcal{M}})_{M \times 1}
    \end{equation}
  \end{example} 

  Even though these estimators are noisy, we get to do much more iterations and therefore have a faster net rate of convergence. But now we have an additional choice to make. What should our minibatch size be? 

  \begin{heuristic}[Choosing Batch Size] 
    It really depends on your hardware, but generally, 
    \begin{enumerate}
      \item A smaller batch size might mean more noisy estimates, and therefore may not converge. However, it tends to escape local minima better. 
      \item A high batch size means more exact estimates, but it tends to get stuck in local minima.  
    \end{enumerate}
    You want to make sure that the batches fit into memory. 
  \end{heuristic}

\subsection{Learning Rates and Schedulers}

  The algorithm may not converge if $\alpha$ (the step size) is too high, since it may overshoot. This can be solved by reducing the $\alpha$ with each step, using \textit{schedulers}. 

  Ideally, we would want to have a variable step size $h(t)$ so that $h \rightarrow 0$ as $t \rightarrow + \infty$. 

  \begin{algo}[Decay on Plateau Learning Rate]
    Basically, this says that if the loss doesn't decrease for the past $p$ epochs, then decrease the learning rate $\eta \gets \gamma \cdot \eta$. 
    \begin{algorithm}[H]
    \label{alg:lr_decay_plateau}
    \begin{algorithmic}[1]  % The [1] adds line numbers
      \Require{Patience $p$, decay rate $0 < \gamma < 1$, initial $\theta_0$, loss $L$}

      \Procedure{LRDecayOnPlateau}{$p, \gamma, \theta, \eta$}
        \State $\text{best\_loss} \gets L(\theta)$ 
        \State $\text{bad\_epochs} \gets 0$
        \While{not converged} 
          \State $\theta \gets \theta - \eta \nabla L(\theta)$ 
          \If{$L(\theta) < \text{best\_loss}$}  
            \State $\text{best\_loss} \gets L(\theta)$ 
            \State $\text{bad\_epochs} \gets 0$
          \Else 
            \State $\text{bad\_epochs} \gets \text{bad\_epochs} + 1$
          \EndIf 
          \If{$\text{bad\_epochs} \geq p$} 
            \State $\eta \gets \eta \cdot \gamma$ 
            \State $\text{bad\_epochs} \gets 0$
          \EndIf 
        \EndWhile
        \State \Return $\theta, \eta$
      \EndProcedure
    \end{algorithmic}
    \end{algorithm}
  \end{algo}

\subsection{Momentum and Nesterov}

  \begin{algo}[Stochastic Gradient Descent with Momentum]
    This modifies vanilla SGD by keeping a running velocity term that accumulates past gradients, which smooths updates and helps escape sharp local minima. 
    \begin{algorithm}[H]
    \label{alg:sgd_momentum}
    \begin{algorithmic}[1]  % The [1] adds line numbers
      \Require{Learning rate $\eta$, momentum $0 \leq \mu < 1$, initial parameters $\theta_0$, loss $L$}

      \Procedure{SGDMomentum}{$\eta, \mu, \theta$}
        \State $v \gets 0$ \Comment{Initialize velocity}
        \While{not converged}
          \State $v \gets \mu v + \nabla L(\theta)$        \Comment{Decay prev. velocity and add in gradient (acceleration)}
          \State $\theta \gets \theta - \eta v$
        \EndWhile
        \State \Return $\theta$
      \EndProcedure
    \end{algorithmic}
    \end{algorithm}
  \end{algo}

  \begin{algo}[Stochastic Gradient Descent with Nesterov Momentum]
    Nesterov momentum modifies standard momentum by computing the gradient at the \emph{lookahead} position $\theta - \eta \mu v$, leading to faster convergence in practice. 
    \begin{algorithm}[H]
    \label{alg:sgd_nesterov}
    \begin{algorithmic}[1]  % The [1] adds line numbers
      \Require{Learning rate $\eta$, momentum $0 \leq \mu < 1$, initial parameters $\theta_0$, loss $L$}

      \Procedure{SGDNesterov}{$\eta, \mu, \theta$}
        \State $v \gets 0$ \Comment{Initialize velocity}
        \While{not converged}
          \State $g \gets \nabla L(\theta - \eta \mu v)$ \Comment{Lookahead gradient}
          \State $v \gets \mu v + g$
          \State $\theta \gets \theta - \eta v$
        \EndWhile
        \State \Return $\theta$
      \EndProcedure
    \end{algorithmic}
    \end{algorithm}
  \end{algo}

\subsection{Block Coordinate Descent} 

  \begin{algo}[Block Gradient Descent]
    Block Gradient Descent partitions the parameter vector $\theta$ into $m$ disjoint blocks. At each iteration, it selects one block (cyclically or randomly) and updates only that block’s parameters using the gradient, while keeping the other blocks fixed. 
    \begin{algorithm}[H]
    \label{alg:block_gd}
    \begin{algorithmic}[1]  % The [1] adds line numbers
      \Require{Learning rate $\eta$, number of blocks $m$, partition $\theta = (\theta^{(1)}, \dots, \theta^{(m)})$, loss $L$}

      \Procedure{BlockGradientDescent}{$\eta, \{\theta^{(j)}\}_{j=1}^m$}
        \State $t \gets 0$
        \While{not converged}
          \State $j \gets \text{SelectBlock}(t, m)$ \Comment{e.g., cyclic: $j = (t \bmod m) + 1$}
          \State $\theta^{(j)} \gets \theta^{(j)} - \eta \nabla_{\theta^{(j)}} L(\theta^{(1)}, \dots, \theta^{(m)})$
          \State $t \gets t + 1$
        \EndWhile
        \State \Return $\theta$
      \EndProcedure
    \end{algorithmic}
    \end{algorithm}
  \end{algo}


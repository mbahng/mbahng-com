\section{Frechet Regression} 

  Frechet regression generalizes regression on manifolds to general metric spaces (with some probability measure) \cite{2017petersen}. In linear regression, we try to estimate the conditional distribution $g(x) = \mathbb{E}[Y \mid X = x]$ with some linear function. 
  \begin{equation}
    y = \beta^T x = \beta_0 + \beta_1 x_1 + \ldots + \beta_d x_d
  \end{equation}

  Note that this requires 

  \begin{definition}[Frechet Mean]
    The \textbf{Frechet mean} of a set of points $y^{(1)}, \ldots, y^{(n)} \in (M, d)$ in a metric space is 
    \begin{equation}
      \mu = \argmin_{m \in M} \sum_{i=1}^n d^2 (y^{(i)}, m)
    \end{equation} 
    if a unique value exists. 
  \end{definition}

  Now we can define the conditional Frechet mean. Since $M$ is a probability space, we can define the integral with respect to the conditional distribution $\mathbb{P}(Y \mid X = x)$. The problem is that to compute the integral $\int f(y) \,d \mathbb{P}(Y \mid x)$, we need $f$ to be in some vector space---where addition and scalar multiplication are defined. We can take the Frechet mean to be this function, where the argmin is taken outside the integral. 

  \begin{definition}[Conditional Frechet Mean]
    The \textbf{conditional Frechet mean} is defined 
    \begin{equation}
      \mu(x) = \argmin_{m \in M} \mathbb{E}_{y} \left[ d^2 (y, m) \mid X = x \right]
    \end{equation}
  \end{definition}

  All that is left to do is try to represent this function $\mu$ with some parametric model. 

\documentclass{article}

  % packages
    % basic stuff for rendering math
    \usepackage[letterpaper, top=1in, bottom=1in, left=1in, right=1in]{geometry}
    \usepackage[utf8]{inputenc}
    \usepackage[english]{babel}
    \usepackage{amsmath} 
    \usepackage{amssymb}

    % extra math symbols and utilities
    \usepackage{mathtools}        % for extra stuff like \coloneqq
    \usepackage{mathrsfs}         % for extra stuff like \mathsrc{}
    \usepackage{centernot}        % for the centernot arrow 
    \usepackage{bm}               % for better boldsymbol/mathbf 
    \usepackage{enumitem}         % better control over enumerate, itemize
    \usepackage{hyperref}         % for hypertext linking
    \usepackage{xr-hyper}
    \usepackage{fancyvrb}          % for better verbatim environments
    \usepackage{newverbs}         % for texttt{}
    \usepackage{xcolor}           % for colored text 
    \usepackage{listings}         % to include code
    \usepackage{lstautogobble}    % helper package for code
    \usepackage{parcolumns}       % for side by side columns for two column code
    

    % page layout
    \usepackage{fancyhdr}         % for headers and footers 
    \usepackage{lastpage}         % to include last page number in footer 
    \usepackage{parskip}          % for no indentation and space between paragraphs    
    \usepackage[T1]{fontenc}      % to include \textbackslash
    \usepackage{footnote}
    \usepackage{etoolbox}

    % for custom environments
    \usepackage{tcolorbox}        % for better colored boxes in custom environments
    \tcbuselibrary{breakable}     % to allow tcolorboxes to break across pages

    % figures
    \usepackage{pgfplots}
    \pgfplotsset{compat=1.18}
    \usepackage{float}            % for [H] figure placement
    \usepackage{tikz}
    \usepackage{tikz-cd}
    \usepackage{circuitikz}
    \usetikzlibrary{arrows}
    \usetikzlibrary{positioning}
    \usetikzlibrary{calc}
    \usepackage{graphicx}
    \usepackage{algorithmic}
    \usepackage{caption} 
    \usepackage{subcaption}
    \captionsetup{font=small}

    % for tabular stuff 
    \usepackage{dcolumn}

    \usepackage[nottoc]{tocbibind}
    \pdfsuppresswarningpagegroup=1
    \hfuzz=5.002pt                % ignore overfull hbox badness warnings below this limit

  % New and replaced operators
    \DeclareMathOperator{\Tr}{Tr}
    \DeclareMathOperator{\Sym}{Sym}
    \DeclareMathOperator{\Span}{span}
    \DeclareMathOperator{\std}{std}
    \DeclareMathOperator{\Cov}{Cov}
    \DeclareMathOperator{\Var}{Var}
    \DeclareMathOperator{\Corr}{Corr}
    \DeclareMathOperator{\pos}{pos}
    \DeclareMathOperator*{\argmin}{\arg\!\min}
    \DeclareMathOperator*{\argmax}{\arg\!\max}
    \newcommand{\ket}[1]{\ensuremath{\left|#1\right\rangle}}
    \newcommand{\bra}[1]{\ensuremath{\left\langle#1\right|}}
    \newcommand{\braket}[2]{\langle #1 | #2 \rangle}
    \newcommand{\qed}{\hfill$\blacksquare$}     % I like QED squares to be black

  % Custom Environments
    \newtcolorbox[auto counter, number within=section]{question}[1][]
    {
      colframe = orange!25,
      colback  = orange!10,
      coltitle = orange!20!black,  
      breakable, 
      title = \textbf{Question \thetcbcounter ~(#1)}
    }

    \newtcolorbox[auto counter, number within=section]{exercise}[1][]
    {
      colframe = teal!25,
      colback  = teal!10,
      coltitle = teal!20!black,  
      breakable, 
      title = \textbf{Exercise \thetcbcounter ~(#1)}
    }
    \newtcolorbox[auto counter, number within=section]{solution}[1][]
    {
      colframe = violet!25,
      colback  = violet!10,
      coltitle = violet!20!black,  
      breakable, 
      title = \textbf{Solution \thetcbcounter}
    }
    \newtcolorbox[auto counter, number within=section]{lemma}[1][]
    {
      colframe = red!25,
      colback  = red!10,
      coltitle = red!20!black,  
      breakable, 
      title = \textbf{Lemma \thetcbcounter ~(#1)}
    }
    \newtcolorbox[auto counter, number within=section]{theorem}[1][]
    {
      colframe = red!25,
      colback  = red!10,
      coltitle = red!20!black,  
      breakable, 
      title = \textbf{Theorem \thetcbcounter ~(#1)}
    } 
    \newtcolorbox[auto counter, number within=section]{proposition}[1][]
    {
      colframe = red!25,
      colback  = red!10,
      coltitle = red!20!black,  
      breakable, 
      title = \textbf{Proposition \thetcbcounter ~(#1)}
    } 
    \newtcolorbox[auto counter, number within=section]{corollary}[1][]
    {
      colframe = red!25,
      colback  = red!10,
      coltitle = red!20!black,  
      breakable, 
      title = \textbf{Corollary \thetcbcounter ~(#1)}
    } 
    \newtcolorbox[auto counter, number within=section]{proof}[1][]
    {
      colframe = orange!25,
      colback  = orange!10,
      coltitle = orange!20!black,  
      breakable, 
      title = \textbf{Proof. }
    } 
    \newtcolorbox[auto counter, number within=section]{definition}[1][]
    {
      colframe = yellow!25,
      colback  = yellow!10,
      coltitle = yellow!20!black,  
      breakable, 
      title = \textbf{Definition \thetcbcounter ~(#1)}
    } 
    \newtcolorbox[auto counter, number within=section]{example}[1][]
    {
      colframe = blue!25,
      colback  = blue!10,
      coltitle = blue!20!black,  
      breakable, 
      title = \textbf{Example \thetcbcounter ~(#1)}
    } 
    \newtcolorbox[auto counter, number within=section]{code}[1][]
    {
      colframe = green!25,
      colback  = green!10,
      coltitle = green!20!black,  
      breakable, 
      title = \textbf{Code \thetcbcounter ~(#1)}
    } 
    \newtcolorbox[auto counter, number within=section]{algo}[1][]
    {
      colframe = green!25,
      colback  = green!10,
      coltitle = green!20!black,  
      breakable, 
      title = \textbf{Algorithm \thetcbcounter ~(#1)}
    } 

    \definecolor{dkgreen}{rgb}{0,0.6,0}
    \definecolor{gray}{rgb}{0.5,0.5,0.5}
    \definecolor{mauve}{rgb}{0.58,0,0.82}
    \definecolor{darkblue}{rgb}{0,0,139}
    \definecolor{lightgray}{gray}{0.93}
    \renewcommand{\algorithmiccomment}[1]{\hfill$\triangleright$\textcolor{blue}{#1}}

    % default options for listings (for code)
    \lstset{
      autogobble,
      frame=ltbr,
      language=Python,
      aboveskip=3mm,
      belowskip=3mm,
      showstringspaces=false,
      columns=fullflexible,
      keepspaces=true,
      basicstyle={\small\ttfamily},
      numbers=left,
      firstnumber=1,                        % start line number at 1
      numberstyle=\tiny\color{gray},
      keywordstyle=\color{blue},
      commentstyle=\color{dkgreen},
      stringstyle=\color{mauve},
      backgroundcolor=\color{lightgray}, 
      breaklines=true,                      % break lines
      breakatwhitespace=true,
      tabsize=3, 
      xleftmargin=2em, 
      framexleftmargin=1.5em, 
      stepnumber=1
    }

  % Page style
    \pagestyle{fancy}
    \fancyhead[L]{Singal Processing}
    \fancyhead[C]{Muchang Bahng}
    \fancyhead[R]{Spring 2025} 
    \fancyfoot[C]{\thepage / \pageref{LastPage}}
    \renewcommand{\footrulewidth}{0.4pt}          % the footer line should be 0.4pt wide
    \renewcommand{\thispagestyle}[1]{}  % needed to include headers in title page

  % external documents 
  %  \externaldocument[place-]{../Machine_Learning/paper}[../Machine_Learning/paper.pdf] 

\begin{document}

\title{Singal Processing}
\author{Muchang Bahng}
\date{Spring 2025}

\maketitle
\tableofcontents
\pagebreak

Unlike my machine learning notes, which focuses mostly on the theoretical soundness of classical (e.g. pre-deep and interpretable) models, the theory of deep neural networks have not been developed as well yet. Furthermore, the recency of developments, especially in the post-2010s, results in a pretty nonlinear\footnote{No pun intended.} timeline that is difficult to categorize effectively. Before I give my motivation, the direct prerequisites for deep learning are basic knowledge of my notes in probability theory, machine learning, and real analysis. 

Therefore, after much thought, I think organizing my notes in chronological order would be best. It turned out that in the early days of deep learning, most researchers like Andrew Ng stated that he focused on supervised models,\footnote{He states this in his podcast with Lex Fridman.} and it wasn't until the 2010s that the development of unsupervised models burgeoned. 

\begin{enumerate}
  \item We start by introducing \textit{multilayered perceptrons} (MLPs), which build upon generalized linear models (GLMs) that we have went over in machine learning. This is pretty much the ``foundational'' model that we will build on. We then talk a about practical methodologies regarding training and control. 

  \item Then we introduce the other two architectures. The \textit{convolutional neural network} (CNN) gives us a scalable way to perform sparse matrix multiplication efficiently, by taking advantage of locality. Then in \textit{recurrent neural networks} (RNNs), we are not limited to a single $n$-dimensional vector input and are able to process a time series of inputs.  
    
  \item With these building blocks, we are able take and two neural networks together to create an encoder-decoder model. The two main applications of this was dimensionality reduction with \textit{autoencoders}, followed by \textit{seq2seq} for machine translation between sequences of words.  

  \item The practical success of autoencoders led to the development of not just density estimation models, but \textit{generative} ones that seek to sample from the learned distribution. These generative models were ``deep'' extensions of the classical linear factor models resulting in \textit{restricted Boltzmann machines (RBMs)} and \textit{variational autoencoders (VAEs)}, which attempted to explicitly model an approximation of the true data generating distribution. More specifically, RBMs were \textit{energy models} that borrowed ideas from physics to learn a distribution of the form $p_X (x) = \frac{1}{Z} e^{-f_\theta (x)}$. To sample from this, researchers already had access to the established Markov Chain Monte Carlo (MCMC) algorithms designed for this exact problem. As for VAEs, these were trained and sampled through \textit{variational inference} methods. 

  \item In 2014, a completely new architecture composed of 2 neural networks competing each other gave rise to \textit{generative adversarial networks} (GANs) which blew RBMs and VAEs away. A similar architecture followed with \textit{generative stochastic networks} (GSNs).

  \item In 2016, \textit{normalizing flow models}, which attempted to model a smooth function that transformed simple latent variables to the true data generating distribution, were invented by Google DeepMind and surpassed GANs. 

  \item In 2017, \textit{attention} took the world by storm, which had drastically improved machine translation in seq2seq, and the resulting \textit{self-attention} plus the \textit{transformer} architectures led to the success of OpenAI's ChatGPT. 
\end{enumerate} 

Again, I emphasize that the math in these notes are not very advanced. However, implementing these simple models and training algorithms from scratch is a challenge in itself. I will go through implementing everything from scratch as if we were building a mini-version of PyTorch as we learn new topics. My implementations can be found \href{https://github.com/mbahng/pyember}{here}. 


\bibliography{./bibfile}
\bibliographystyle{alpha}

\end{document}

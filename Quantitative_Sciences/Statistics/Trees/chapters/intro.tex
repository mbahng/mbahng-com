Say that you were looking at a picture and were told to classify it as a bird or a dog. You would probably look for some features and have the following thought process: if it has wings, then it is a bird, and if not, then it is a dog. 

Now let's try a slightly harder classification. We have four classes consisting of two species of dogs (golden retriever, husky) and two species of birds (pigeon, hummingbird). Then you might work something like this. 
\begin{enumerate}
  \item If it has wings, and 
    \begin{enumerate}
      \item it has a long beak, then it is a hummingbird. 
      \item it doesn't have a long beak, then it is a pigeon. 
    \end{enumerate}
  \item If it doesn't have wings, and 
    \begin{enumerate}
      \item its fur color is yellow, then it is a golden retriever. 
      \item its fur color is not yellow, then it is a husky. 
    \end{enumerate}
\end{enumerate}

Decision trees attempt to model this method of thinking by using a tree structure, and hopefully this example should convince you that this type of model is worth studying. It is a discriminative model that learns to classify data by first identifying the relevant feature to look at (e.g. wings, beak length, fur color) and then deciding how to split it. 

Surprisingly, the origin of tree models is not clear, though there have been some papers as early as 1959 that mentions a decision tree-like structure.\footnote{See \href{https://stats.stackexchange.com/questions/257537/who-invented-the-decision-tree}{https://stats.stackexchange.com/questions/257537/who-invented-the-decision-tree}.} I personally would have thought it to be older given the simplicity of the idea. 

A final note. In a sense, trees and K-nearest neighbors are similar in that they try to find neighborhoods of a certain point to classify it. For vanilla KNN, we use a circular region consisting of the $k$ nearest samples in our dataset. In decision trees, we are splitting across variables, so it's more like rectangular regions, which may or may not be more adaptive. This is the motivation behind \textit{adaptive KNN}, which adapts the shape in the region you're doing KNN. 

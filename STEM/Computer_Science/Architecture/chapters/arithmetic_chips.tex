\section{Arithmetic Logical Unit}

  This is basically the ALU. 

  Talk about how to construct arithmetic operations with these gates such as adding two integers or multiplying them, and not just that, but other operations that we may need in a programming language. 

\subsection{Addition and Subtraction} 

\subsection{Multiplication} 

  \begin{theorem}[Implementation of Moving Data in Circuits]
    
  \end{theorem}

  \begin{theorem}[Implementation of Addition, Subtraction in Circuits]
    
  \end{theorem}

  \begin{theorem}[Implementation of Multiplication in Circuits]
    
  \end{theorem}

  \begin{theorem}[Implementation of Bitwise Operations in Circuits]
    
  \end{theorem}

  \begin{theorem}[Implementation of Bitshift Operations]
    
  \end{theorem}

\subsection{Conditionals}

  We also want some sort of conditionals. This then can be used to implement loops by checking some conditional. 

  \begin{theorem}[Implementation of Conditionals in Circuits]
    
  \end{theorem}



\section{Input Output}

  Obviously, a computer consists more than just a CPU and memory. There are other external devices, most notably the keyboard, the mouse, and the monitor. These devices fit into our model as our \textit{input-output devices}. 

  \begin{definition}[Input-Output]
    \textbf{Input} refers to data/signals that a computer receives from an external source, while \textbf{output} refers to data/signals that a computer produces. We fer to this as \textbf{input-output (IO)}. 
    \begin{enumerate}
      \item The physical devices that handle IO are called \textbf{IO devices}. 
      \item The data that flows in and out of the computer through IO devices is called the \textbf{IO stream}. 
    \end{enumerate}
  \end{definition}

\subsection{IO Devices}

  IO devices are generally categorized into: 
  \begin{enumerate}
    \item \textit{User Interface}: keyboard, monitor, mouse, speakers, USB ports, card/CD readers. 
    \item \textit{Disk}: such as HDD and SSDs.\footnote{We will elaborate on this in the next section. }
    \item \textit{Network Communications}: Wifi chips/drivers, ethernet ports.\footnote{This will be covered in \textit{Computer Networks}. }
  \end{enumerate}

\subsection{IO Buses and Interconnects}
  
  Hardware/physical layer. How devices physically connect (USB, PCI, SATA). Electrical signaling and timing. 

\subsection{Control Strategies}

  Hardware interface layer. Why we need controllers (protocol translation, timing management). MMIO, PMIO, along with Polling, Interrupts, and DMA. 

  Within the context of computer architecture, we want to integrate these devices without modifying the von Neumann architecture. The way that we do this is to treat these IO devices as memory. 

  \begin{definition}[Memory Mapped IO]
    \textbf{Memory mapped IO} is an adaption of the von Neumann architecture where for every IO device and corresponding IO stream is designated a specific subset of memory addresses that it can write to. As data is fed through the stream, it continuously changes the values of the bits at the memory address. The CPU can then access this memory, which simulates the computer interacting with the outside environment. 
  \end{definition}

  \begin{example}[Webcams]
    Suppose that in a webcam, each frame is 128 bits. The webcam's memory mapped IO has a set location in memory where you can read from that address as if it were reading from your RAM. But there is an extra layer of signal that tells you its not from the RAM but the memory. 
  \end{example}

  Port mapped IO. 

  \begin{definition}[Port Mapped IO]
    \textbf{Port Mapped IO} uses a special class of CPU instructions designed specifically for performing IO. 
  \end{definition}

  \begin{example}[Port Mapped IO on x86]
    The \texttt{in} and \texttt{out} instructions found on microprocessors based on the x86 architecture are specific for performing IO. Different forms of these two instructions can copy one, two or four bytes (\texttt{outb}, \texttt{outw} and \texttt{outl}, respectively) between the \texttt{eax} register and a specified IO port address which is assigned to an IO device.
  \end{example}

  One merit of memory-mapped IO is that, by discarding the extra complexity that port IO brings, a CPU requires less internal logic and is thus cheaper, faster, easier to build, consumes less power and can be physically smaller. 

  To access memory in real-time, there are two methods. 

  \begin{definition}[Polling]
    Every once in a while (usually every few clock cycles), the CPU will ask the memory mapped IO to retrieve the data. That is, the CPU initiates the processing of the IO stream, and this is useful for continuous IO streams. 
  \end{definition}

  \begin{definition}[Interrupts]
    Whenever there is a new reading, a sensor sends a signal to the CPU. That is, the device initiates the processing of the IO stream, and this is useful for data you don't get very often. 
  \end{definition}

  Or we can bypass the CPU for data transfer. 

  \begin{definition}[Direct Memory Access]
    A \textbf{direct memory access (DMA) controller} is a piece of hardware that allows peripherals (like hard drives, network cards, graphics cards) to directly access memory without involving the CPU. 
  \end{definition}

  \begin{example}
    
  \end{example}

\subsection{Device Controllers}

  Now that we know the general strategies for control, we look at the implementations for each device. Controller architecture (e.g. keyboard controller, disk controller). 
  \begin{enumerate}
    \item Keyboard controller: Uses memory-mapped registers + interrupts
    \item Disk controller: Uses memory-mapped registers + interrupts + DMA
    \item Network controller: Uses memory-mapped registers + interrupts + DMA
    \item Display controller: Uses memory-mapped framebuffer + polling/interrupts
  \end{enumerate}


\section{Logic Gates}

\subsection{Elementary Logic Gates} 

  The most common elementary operations for logical operations. 

  \begin{definition}[NOT Gate]
    A \textbf{NOT gate} is defined as follows.  
    \begin{equation}
      \NOT: \{0,1\} \longrightarrow \{0,1\}, \qquad 
      \NOT (a) = \lnot a = \begin{cases}
        0 & a = 1 \\
        1 & a = 0
      \end{cases}
    \end{equation}

    \begin{figure}[H]
      \centering 
      \begin{circuitikz}[scale=0.9]
        \draw
        (0,2) node[not port] (not) {};
      \end{circuitikz}    
      \caption{A helpful hint to remember that this is NOT. Think of the triangle shape as ``doing nothing'' and pay attention to the circle at the tip, which represents negation. } 
      \label{fig:not_gate}
    \end{figure}
  \end{definition}

  \begin{definition}[AND Gate]
    A \textbf{AND gate} is defined as follows.  
    \begin{equation}
      \AND: \{0,1\}^2 \longrightarrow \{0,1\}, \qquad 
      \AND (a,b) = a \wedge b = \begin{cases}
        1 & a = b = 1 \\
        0 & \text{else}
      \end{cases}
    \end{equation}
    \begin{figure}[H]
      \centering 
      \begin{circuitikz}[scale=0.9]
        \draw
        (0,2) node[and port] (and) {};
      \end{circuitikz}    
      \caption{A helpful hint to remember that this is AND. Think of the D-shape as requiring both inputs to be 1 for the output to be 1.} 
      \label{fig:and_gate}
    \end{figure}
  \end{definition}

  \begin{definition}[OR Gate]
    A \textbf{OR gate} is defined as follows.  
    \begin{equation}
      \OR: \{0,1\}^2 \longrightarrow \{0,1\}, \qquad 
      \OR (a,b) = a \vee b = \begin{cases}
        0 & a = b = 0 \\
        1 & \text{else}
      \end{cases}
    \end{equation}
    \begin{figure}[H]
      \centering 
      \begin{circuitikz}[scale=0.9]
        \draw
        (0,2) node[or port] (or) {};
      \end{circuitikz}    
      \caption{A helpful hint to remember that this is OR. Think of the rounded shape as requiring at least one input to be 1 for the output to be 1.} 
      \label{fig:or_gate}
    \end{figure}
  \end{definition}

  Many functions can be created when composing these extremely simple functions. 

  \begin{example}[Majority Function]
    Consider the function that outputs whatever the majority bit value is amongst its 3 inputs. 
    \begin{equation}
      \mathrm{MAJ}: \{0,1\}^3 \longrightarrow \{0,1\}, \qquad 
      \mathrm{MAJ} (x) = \begin{cases}
        1 & x_0 + x_1 + x_2 \geq 2 \\
        0 & \text{else}
      \end{cases}
    \end{equation}
    Since the OR of three conditions $c_0, c_1, c_2$ can be written as $\OR(c_0, \OR(c_1, c_2))$, we can now translate this function into a formula as follows: 
    \begin{align}
      \mathrm{MAJ}(x_0, x_1, x_2) & = \OR\big( \AND (x_0, x_1), \OR(\AND(x_1, x_2), \AND(x_0, x_2)) \big) \\
      & = \big((x_0 \wedge x_1) \vee (x_1 \wedge x_2)\big) \vee (x_0 \wedge x_2)
    \end{align}
  \end{example}

  \begin{definition}[Universal Gate Sets]
    A set of gates $S$ is said to be a \textbf{universal gate set} if one can construct any finite function $f: \{0, 1\}^n \to \{0, 1\}^m$ with the gates in $S$. 
  \end{definition}

  \begin{theorem}[Universality of NOT/AND/OR]
    $\{ \NOT, \AND, \OR \}$ is a universal gate set. 
  \end{theorem}
  \begin{proof}
    It suffices to prove for any
  \end{proof}

  \begin{definition}[NAND Gate]
    A \textbf{NAND gate} is defined as follows.\footnote{$\NAND$ is really the composition of the NOT and AND functions; that is, $\NAND(a, b) = (\NOT \circ \AND) (a, b)$.}  
    \begin{equation}
      \NAND: \{0,1\}^2 \longrightarrow \{0,1\}, \qquad 
      \NAND (a,b) = \overline{a \wedge b} = \begin{cases}
        0 & a = b = 1 \\
        1 & \text{else}
      \end{cases}
    \end{equation}

    \begin{figure}[H]
      \centering 
      \begin{circuitikz}[scale=0.9]
        \draw
        (0,2) node[nand port] (nand) {};
      \end{circuitikz}    
      \caption{A helpful hint to remember that this is NAND. Notice the D-shape of the AND gate with the circle at the output representing negation (NOT).} 
      \label{fig:nand_gate}
    \end{figure}
  \end{definition}

  \begin{theorem}[Universality of NAND]
    NAND is universal gate set. 
  \end{theorem}
  \begin{proof}
    We can see that, using double negation, 
    \begin{align}
      \NOT(a) & = \NOT(\AND(a, a)) \\
      & = \NAND(a, a)\\
      \AND(a, b) & = \NOT(\NOT(\AND(a, b))) \\
      & = \NOT(\NAND(a, b)) \\
      & = \NAND(\NAND(a, b), \NAND(a, b)) \\
      \OR(a, b) & = \NOT(\AND(\NOT(a), \NOT(b))) \\
      & = \NOT(\AND(\NAND(a,a), \NAND(b,b))) \\
      & = \NAND(\NAND(a, a), \NAND(b, b)) 
    \end{align}
  \end{proof}

  For every Boolean circuit $C$ of $s$ gates, there exists a NAND circuit $C^\prime$ of at most $3s$ gates that computes the same function as $C$. Replace every AND, OR, and NOT gate with their NAND equivalents. 

  \begin{definition}[NOR Gate]
    A \textbf{NOR gate} is defined as follows.\footnote{NOR is really the composition of the NOT and OR functions; that is, $\NOR(a, b) = (\NOT \circ \OR) (a, b)$. }
    \begin{equation}
      \NOR: \{0,1\}^2 \longrightarrow \{0,1\}, \qquad 
      \NOR (a,b) = \overline{a \vee b} = \begin{cases}
        1 & a = b = 0 \\
        0 & \text{else}
      \end{cases}
    \end{equation}

    \begin{figure}[H]
      \centering 
      \begin{circuitikz}[scale=0.9]
        \draw
        (0,2) node[nor port] (nor) {};
      \end{circuitikz}    
      \caption{A helpful hint to remember that this is NOR. Notice the rounded shape of the OR gate with the circle at the output representing negation (NOT).} 
      \label{fig:nor_gate}
    \end{figure}
  \end{definition}

  \begin{definition}[XOR Gate]
    A \textbf{XOR gate} is defined as follows.  
    \begin{equation}
      \XOR: \{0,1\}^2 \longrightarrow \{0,1\}, \qquad 
      \XOR (a,b) = a \oplus b = \begin{cases}
        0 & a = b \\
        1 & a \neq b
      \end{cases}
    \end{equation}

    \begin{figure}[H]
      \centering 
      \begin{circuitikz}[scale=0.9]
        \draw
        (0,2) node[xor port] (xor) {};
      \end{circuitikz}    
      \caption{A helpful hint to remember that this is XOR. Notice the additional curve on the OR gate shape, indicating that exactly one input must be 1 for the output to be 1.} 
      \label{fig:xor_gate}
    \end{figure}
  \end{definition}

  \begin{example}[XOR from NAND]
    XOR can be expressed in terms of other logic gates as follows:
    \begin{equation}
      \XOR(a, b) = (a \wedge \neg b) \vee (\neg a \wedge b)
    \end{equation}
    We can create a NAND circuit of the XOR function that maps $x_0, x_1 \in \{0,1\}$ to $x_0 + x_1 \pmod{2}$. 
    \begin{center}
      \begin{circuitikz}[scale=0.4]
        \draw
        (0,0) node[nand port] (nand1) {}
        (5, 2) node[nand port] (nand2) {}
        (5, -2) node[nand port] (nand3) {}
        (10,0) node[nand port] (nand4) {}
        (nand1.out)--(nand2.in 2)
        (nand1.out)--(nand3.in 1)
        (nand2.out)--(nand4.in 1)
        (nand3.out)--(nand4.in 2) 
        (-6, 3) node(x1) {X[0]}
        (-6, -3) node(x2) {X[1]}
        (x1)--(nand1.in 1)
        (x2)--(nand1.in 2)
        (x1)--(nand2.in 1)
        (x2)--(nand3.in 2)
        (13,0) node(y) {Y[0]}
        (nand4.out)--(y);
      \end{circuitikz}
    \end{center}
  \end{example}

  \begin{example}[All Equals Function]
    Let us define the function $\mathrm{ALLEQ}: \{0,1\}^4 \longrightarrow \{0,1\}$ to be the function that on input $x \in \{0,1\}^4$ outputs $1$ if and only if $x_0 = x_1 = x_2 = x_3$. 

    \begin{figure}[H]
      \centering 
      \begin{circuitikz}[scale=0.4]
        \draw
        (2, 2) node[and port] (and1) {}
        (2, 6) node[not port] (not1) {}
        (2, 10) node[not port] (not2) {}
        (8, 2) node[and port] (and2) {}
        (8, 6) node[and port] (and3) {}
        (8, -2) node[not port] (not3) {}
        (14, 6) node[and port] (and4) {}
        (14, 10) node[and port] (and5) {}
        (14, 14) node[not port] (not4) {}
        (19.5, 12) node[and port] (and6) {}
        (24, 11) node[or port] (or1) {}
        (-4,-1) node(d) {X[3]}
        (-4, 3) node(c) {X[2]}
        (-4, 8) node(b) {X[1]}
        (-4, 12) node(a) {X[0]}
        (a)--(not4.in)
        (a)--(and5.in 1)
        (b)--(not2.in)
        (b)--(and1.in 1)
        (c)--(not1.in) 
        (c)--(and1.in 2)
        (d)--(and2.in 2)
        (d)--(not3.in) 
        (and1.out)--(and2.in 1)
        (not1.out)--(and3.in 2)
        (not2.out)--(and3.in 1)
        (and2.out)--(and5.in 2) 
        (and3.out)--(and4.in 1) 
        (and4.out)--(and6.in 2) 
        (not3.out)--(and4.in 2) 
        (and5.out)--(or1.in 2) 
        (not4.out)--(and6.in 1)
        (and6.out)--(or1.in 1)
        (26, 11) node(y) {Y[0]}
        (or1.out)--(y);
      \end{circuitikz}
      \caption{The Boolean circuit for computing $\mathrm{ALLEQ}$.} 
      \label{fig:alleq}
    \end{figure}
  \end{example}

  \begin{definition}[Multiplexor Gate]
    A \textbf{Multiplexor gate} (MUX) is defined as follows.  
    \begin{equation}
      \MUX: \{0,1\}^3 \longrightarrow \{0,1\}, \qquad 
      \MUX (a, b, s) = \begin{cases}
        a & s = 0 \\
        b & s = 1
      \end{cases}
    \end{equation}
    A multiplexor can be expressed using basic logic gates as follows:
    \begin{equation}
      \MUX(a, b, s) = (a \wedge \neg s) \vee (b \wedge s)
    \end{equation}
  \end{definition}

  \begin{definition}[Demultiplexor Gate]
    A \textbf{Demultiplexor gate} (DEMUX) is defined as follows.  
    \begin{equation}
      \DEMUX: \{0,1\}^2 \longrightarrow \{0,1\}^2, \qquad 
      \DEMUX (d, s) = (y_0, y_1) \text{ where } \begin{cases}
        y_0 = d \wedge \neg s \\
        y_1 = d \wedge s
      \end{cases}
    \end{equation}
    A demultiplexor routes the input $d$ to one of two outputs based on the select signal $s$.
  \end{definition}

  \begin{theorem}[Constant Plus Multiplexor is Universal]
    
  \end{theorem}

\subsection{Hardware Description Languages (HDL)}

  Clearly, engineers don't probably spend their days endlessly drawing little gates on paper. We use specific software---called hardware description language (HDL)---to model and design these digital systems. Historically, \textit{Verilog} was the most dominant, but it has been largely replaced by \textit{SystemVerilog} and \textit{VHDL}. 

\subsection{Multi-Bit Logic Gates}

  \begin{definition}[Multi-Bit NOT Gate]
    
  \end{definition}

  \begin{definition}[Multi-Bit AND Gate]
    
  \end{definition}

  \begin{definition}[Multi-Bit OR Gate]
    
  \end{definition}

  \begin{definition}[Multi-Bit NAND Gate]
    
  \end{definition}

  \begin{definition}[Multi-Bit XOR Gate]
    
  \end{definition}

  \begin{definition}[Multi-Bit Multiplexor Gate]
    
  \end{definition}

  \begin{definition}[Multi-Bit Demultiplexor Gate]
    
  \end{definition}


  boolean operations


\documentclass{article}

% packages
  % basic stuff for rendering math
  \usepackage[letterpaper, top=1in, bottom=1in, left=1in, right=1in]{geometry}
  \usepackage[utf8]{inputenc}
  \usepackage[english]{babel}
  \usepackage{amsmath} 
  \usepackage{amssymb}
  % \usepackage{amsthm}

  % extra math symbols and utilities
  \usepackage{mathtools}        % for extra stuff like \coloneqq
  \usepackage{mathrsfs}         % for extra stuff like \mathsrc{}
  \usepackage{centernot}        % for the centernot arrow 
  \usepackage{bm}               % for better boldsymbol/mathbf 
  \usepackage{enumitem}         % better control over enumerate, itemize
  \usepackage{hyperref}         % for hypertext linking
  \usepackage{fancyvrb}          % for better verbatim environments
  \usepackage{newverbs}         % for texttt{}
  \usepackage{xcolor}           % for colored text 
  \usepackage{listings}         % to include code
  \usepackage{lstautogobble}    % helper package for code
  \usepackage{parcolumns}       % for side by side columns for two column code
  

  % page layout
  \usepackage{fancyhdr}         % for headers and footers 
  \usepackage{lastpage}         % to include last page number in footer 
  \usepackage{parskip}          % for no indentation and space between paragraphs    
  \usepackage[T1]{fontenc}      % to include \textbackslash
  \usepackage{footnote}
  \usepackage{etoolbox}

  % for custom environments
  \usepackage{tcolorbox}        % for better colored boxes in custom environments
  \tcbuselibrary{breakable}     % to allow tcolorboxes to break across pages

  % figures
  \usepackage{pgfplots}
  \pgfplotsset{compat=1.18}
  \usepackage{float}            % for [H] figure placement
  \usepackage{tikz}
  \usepackage{tikz-cd}
  \usepackage{circuitikz}
  \usetikzlibrary{arrows}
  \usetikzlibrary{positioning}
  \usetikzlibrary{calc}
  \usepackage{graphicx}
  \usepackage{algorithmic}
  \usepackage{caption} 
  \usepackage{subcaption}
  \captionsetup{font=small}

  % for tabular stuff 
  \usepackage{dcolumn}

  \usepackage[nottoc]{tocbibind}
  \pdfsuppresswarningpagegroup=1
  \hfuzz=5.002pt                % ignore overfull hbox badness warnings below this limit

% New and replaced operators
  \DeclareMathOperator{\Tr}{Tr}
  \DeclareMathOperator{\Sym}{Sym}
  \DeclareMathOperator{\Span}{span}
  \DeclareMathOperator{\std}{std}
  \DeclareMathOperator{\Cov}{Cov}
  \DeclareMathOperator{\Var}{Var}
  \DeclareMathOperator{\Corr}{Corr}
  \DeclareMathOperator{\pos}{pos}
  \DeclareMathOperator*{\argmin}{\arg\!\min}
  \DeclareMathOperator*{\argmax}{\arg\!\max}
  \newcommand{\ket}[1]{\ensuremath{\left|#1\right\rangle}}
  \newcommand{\bra}[1]{\ensuremath{\left\langle#1\right|}}
  \newcommand{\braket}[2]{\langle #1 | #2 \rangle}
  \newcommand{\qed}{\hfill$\blacksquare$}     % I like QED squares to be black

% Custom Environments
  \newtcolorbox[auto counter, number within=section]{question}[1][]
  {
    colframe = orange!25,
    colback  = orange!10,
    coltitle = orange!20!black,  
    breakable, 
    title = \textbf{Question \thetcbcounter ~(#1)}
  }

  \newtcolorbox[auto counter, number within=section]{exercise}[1][]
  {
    colframe = teal!25,
    colback  = teal!10,
    coltitle = teal!20!black,  
    breakable, 
    title = \textbf{Exercise \thetcbcounter ~(#1)}
  }
  \newtcolorbox[auto counter, number within=section]{solution}[1][]
  {
    colframe = violet!25,
    colback  = violet!10,
    coltitle = violet!20!black,  
    breakable, 
    title = \textbf{Solution \thetcbcounter}
  }
  \newtcolorbox[auto counter, number within=section]{lemma}[1][]
  {
    colframe = red!25,
    colback  = red!10,
    coltitle = red!20!black,  
    breakable, 
    title = \textbf{Lemma \thetcbcounter ~(#1)}
  }
  \newtcolorbox[auto counter, number within=section]{theorem}[1][]
  {
    colframe = red!25,
    colback  = red!10,
    coltitle = red!20!black,  
    breakable, 
    title = \textbf{Theorem \thetcbcounter ~(#1)}
  } 
  \newtcolorbox[auto counter, number within=section]{proposition}[1][]
  {
    colframe = red!25,
    colback  = red!10,
    coltitle = red!20!black,  
    breakable, 
    title = \textbf{Proposition \thetcbcounter ~(#1)}
  } 
  \newtcolorbox[auto counter, number within=section]{corollary}[1][]
  {
    colframe = red!25,
    colback  = red!10,
    coltitle = red!20!black,  
    breakable, 
    title = \textbf{Corollary \thetcbcounter ~(#1)}
  } 
  \newtcolorbox[auto counter, number within=section]{proof}[1][]
  {
    colframe = orange!25,
    colback  = orange!10,
    coltitle = orange!20!black,  
    breakable, 
    title = \textbf{Proof. }
  } 
  \newtcolorbox[auto counter, number within=section]{definition}[1][]
  {
    colframe = yellow!25,
    colback  = yellow!10,
    coltitle = yellow!20!black,  
    breakable, 
    title = \textbf{Definition \thetcbcounter ~(#1)}
  } 
  \newtcolorbox[auto counter, number within=section]{example}[1][]
  {
    colframe = blue!25,
    colback  = blue!10,
    coltitle = blue!20!black,  
    breakable, 
    title = \textbf{Example \thetcbcounter ~(#1)}
  } 
  \newtcolorbox[auto counter, number within=section]{code}[1][]
  {
    colframe = green!25,
    colback  = green!10,
    coltitle = green!20!black,  
    breakable, 
    title = \textbf{Code \thetcbcounter ~(#1)}
  } 
  \newtcolorbox[auto counter, number within=section]{algo}[1][]
  {
    colframe = green!25,
    colback  = green!10,
    coltitle = green!20!black,  
    breakable, 
    title = \textbf{Algorithm \thetcbcounter ~(#1)}
  } 

  \BeforeBeginEnvironment{example}{\savenotes}
  \AfterEndEnvironment{example}{\spewnotes}
  \BeforeBeginEnvironment{lemma}{\savenotes}
  \AfterEndEnvironment{lemma}{\spewnotes}
  \BeforeBeginEnvironment{theorem}{\savenotes}
  \AfterEndEnvironment{theorem}{\spewnotes}
  \BeforeBeginEnvironment{corollary}{\savenotes}
  \AfterEndEnvironment{corollary}{\spewnotes}
  \BeforeBeginEnvironment{proposition}{\savenotes}
  \AfterEndEnvironment{proposition}{\spewnotes}
  \BeforeBeginEnvironment{definition}{\savenotes}
  \AfterEndEnvironment{definition}{\spewnotes}
  \BeforeBeginEnvironment{exercise}{\savenotes}
  \AfterEndEnvironment{exercise}{\spewnotes}
  \BeforeBeginEnvironment{proof}{\savenotes}
  \AfterEndEnvironment{proof}{\spewnotes}
  \BeforeBeginEnvironment{solution}{\savenotes}
  \AfterEndEnvironment{solution}{\spewnotes}
  \BeforeBeginEnvironment{question}{\savenotes}
  \AfterEndEnvironment{question}{\spewnotes}
  \BeforeBeginEnvironment{code}{\savenotes}
  \AfterEndEnvironment{code}{\spewnotes}
  \BeforeBeginEnvironment{algo}{\savenotes}
  \AfterEndEnvironment{algo}{\spewnotes}

  \definecolor{dkgreen}{rgb}{0,0.6,0}
  \definecolor{gray}{rgb}{0.5,0.5,0.5}
  \definecolor{mauve}{rgb}{0.58,0,0.82}
  \definecolor{darkblue}{rgb}{0,0,139}
  \definecolor{lightgray}{gray}{0.93}
  \renewcommand{\algorithmiccomment}[1]{\hfill$\triangleright$\textcolor{blue}{#1}}

  % default options for listings (for code)
  \lstset{
    autogobble,
    frame=ltbr,
    language=C++,
    aboveskip=3mm,
    belowskip=3mm,
    showstringspaces=false,
    columns=fullflexible,
    keepspaces=true,
    basicstyle={\small\ttfamily},
    numbers=left,
    firstnumber=1,                        % start line number at 1
    numberstyle=\tiny\color{gray},
    keywordstyle=\color{blue},
    commentstyle=\color{dkgreen},
    stringstyle=\color{mauve},
    backgroundcolor=\color{lightgray}, 
    breaklines=true,                      % break lines
    breakatwhitespace=true,
    tabsize=3, 
    xleftmargin=2em, 
    framexleftmargin=1.5em, 
    stepnumber=1
  }

% Page style
  \pagestyle{fancy}
  \fancyhead[L]{C++}
  \fancyhead[C]{Muchang Bahng}
  \fancyhead[R]{Winter 2024} 
  \fancyfoot[C]{\thepage / \pageref{LastPage}}
  \renewcommand{\footrulewidth}{0.4pt}          % the footer line should be 0.4pt wide
  \renewcommand{\thispagestyle}[1]{}  % needed to include headers in title page

\begin{document}

\title{C++}
\author{Muchang Bahng}
\date{Winter 2024}

\maketitle
\tableofcontents
\pagebreak

\section{Translation} 

    \textbf{Translating} C++ code to a binary consists of multiple steps: 
    \begin{enumerate}
      \item Preprocessing the code. 
      \item Compiling each file independently. 
      \item Linking all the files. 
    \end{enumerate}
    Conventionally, all of these are called \textit{compiling}, but it really isn't. 

  \subsection{Preprocessing} 

    When preprocessing, we do some boring stuff like removing comments. However, the main job is to take care of \textbf{preprocessing directives}, which are expressions with the \texttt{\#} symbol. The most obvious is the \texttt{\#include} directives, which \textbf{replaces the include directive with the contents of the included file}. That is, \texttt{\#include} is really just a way to substitute code.  
    \begin{enumerate}
      \item including with angle brackets, e.g. \texttt{\#include <iostream>}, means that the compiler is looking for this file in the standard library files. 
      \item including with double quotes, e.g. \texttt{\#include "tensor.h"}, means that the compiler is looking for this file locally in your project directory. It means you've written it. 
    \end{enumerate}
    
    Other directives is the \texttt{\#define} directive. 
    \begin{enumerate}
      \item You can define it to substitute text. It is conventionally in all upper-case.  
        \begin{lstlisting}
          #define NAME "Muchang"  // all instances of NAME will be replaced with "Muchang"
        \end{lstlisting}
      \item Or you can define it without substitution text, where further occurrences of \texttt{NAME} will be replaced by nothing. 
        \begin{lstlisting}
          #define NAME 
        \end{lstlisting}
    \end{enumerate}

    The second isn't used for substitution, but rather for \textbf{conditional compilation}, which can be useful. You just wrap C++ statements around as such.

    \noindent\begin{minipage}{.5\textwidth}
    \begin{lstlisting}[]{Code}
      #ifdef NAME 
      ... 
      #endif
    \end{lstlisting}
    \end{minipage}
    \hfill
    \begin{minipage}{.49\textwidth}
    \begin{lstlisting}[]{Output}
      #ifndef NAME 
      ... 
      #endif
      
    \end{lstlisting}
    \end{minipage} 

    To see the output after preprocessing, use the \texttt{-E} flag. 
    \begin{lstlisting}
      g++ main.cpp -E
    \end{lstlisting}

  \subsection{Compilation} 

    We only compile files one at a time and independently. When the compiler compiles a file, it goes through each line sequentially. Therefore, we must ensure that all functions/variables/classes are \textit{declared} first before they are called. \textit{Forward declaration} makes this a lot easier. 
    
    If we 

  \subsection{Linking} 

    Remember, declaration is not the same thing as definition. When we do the linking, we go through all the source files in our project and match all the declarations with our definitions. The source files must all be written in the compile command. 

    \begin{lstlisting}
      g++ main.cpp add.cpp
      g++ add.cpp main.cpp
    \end{lstlisting}

    This should not be order dependent. The source files can be 

    \noindent\begin{minipage}{.5\textwidth}
    \begin{lstlisting}[]{Code}
      // main.cpp 
      int add(int x, int y); // declaration

      int main() {  
        int z = add(2, 3); 
        return 0; 
      }
    \end{lstlisting}
    \end{minipage}
    \hfill
    \begin{minipage}{.49\textwidth}
    \begin{lstlisting}[]{Output}
      // add.cpp
      // definition
      int add(int x, int y) { 
        return x + y;
      }
      .
      .
    \end{lstlisting}
    \end{minipage}

\section{Multiple Files}


\end{document}

\section{Documentation with LaTeX} 

  Latex is a great way to take notes. One can go to Overleaf and have everything preconfigured, but in here I set it up on my local desktop. I will already assume you have a PDF viewer installed. I use zathura, which is lightweight and also comes with vim motions for navigation. 

  First install the VimTex plugin in \texttt{plugins.lua} with \texttt{use lervag/vimtex}. Then, you want to install TexLive, which is needed to compile tex files and to manage packages. The directions for TexLive installation is available [here](https://tug.org/texlive/quickinstall.html). Once I downloaded the install files, I like to run \texttt{sudo perl ./install-tl --scheme=small}. Be careful with the server location (which can be set with the \texttt{--location} parameter), as I have gotten some errors. I set \texttt{--scheme=small}, which installs about 350 packages compared to the default scheme, which installs about 5000 packages (~7GB). I also did not set \texttt{--no-interaction} since I want to slightly modify the \texttt{--texuserdir} to some other path rather than just my home directory. 

  Once you installed everything, make sure to add the binaries to PATH, which will allow you to access the \textbf{tlmgr} package manager, which pulls from the CTAN (Comprehensive TeX Archive Network) and gives VimTex access to these executables. Unfortunately, the small scheme installation does not also install the \textbf{latexmk} compiler, which is recommended by VimTex. We can simply install this by running 
  ```
  sudo tlmgr install latexmk
  ```
  Now run `:checkhealth` in Neovim and make sure that everything is OK, and install whatever else is needed. 


  To install other Latex packages (and even document classes), we can use tlmgr. All the binaries and packages are located in \texttt{/usr/loca/texlive/202*/} and since we're modifying this, we should run it with root privileges. The binaries can also be found here. Let's go through some basic commands: 
  \begin{enumerate}
    \item List all available packages: \texttt{tlmgr list}
    \item List installed packages: \texttt{tlmgr list --only-installed} (the packages with the `i` next to them are installed)
    \item Install a package and dependencies: \texttt{sudo tlmgr install amsmath tikz} 
    \item Reinstall a package: \texttt{sudo tlmgr install amsmath --reinstall}
    \item Remove a package: \texttt{sudo tlmgr remove amsmath} 
  More commands can be found \href{http://tug.ctan.org/info/tlmgrbasics/doc/tlmgr.pdf}{here} for future reference.  
  \end{enumerate}

  After this, you can install Inkscape, which is free vector-based graphics editor (like Adobe Illustrator). It is great for drawing diagrams, and you can generate custom keymaps that automatically open Inkscape for drawing diagrams within LaTeX, allowing for an seamless note-taking experience.  


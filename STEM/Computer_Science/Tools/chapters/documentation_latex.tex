\section{Documentation with LaTeX} 

  Latex is a great way to take notes, and while it may be tedious to write it at first, it's a skill that you build up just like when you first wrote the alphabet in kindergarten. If you think you'll be slow at latex forever, don't worry. In a few months I was faster at taking notes with latex than by hand or google docs. 

  I still handwrite notes though during class. Some people handwrite because it helps with retention, but I do it because I often need to reorganize the structure and content of my notes multiple times before I have a clear picture of the entire course. My general system is to handwrite notes for about a month's worth of classes, and then spend a few days thinking about them and adding them to my website.  

  Now let's get back to latex. Most users write on Overleaf, which is a platform that makes latex easier by having all the packages you'll ever need on the cloud, along with a user-friendly GUI. Everything is preconfigured, but that means that your work environment can't really be tailored to the way you like it. I've used Overleaf for about 3 years before I started using Vim, and the lack of Vim keybindings on Overleaf just made me write latex on my local desktop. This allows me to have all my files locally, which I can then store in some remote Github repository.  

  I use the Neovim plugin \textbf{VimTex}, which is installed in my \texttt{plugins.lua} with \texttt{use lervag/vimtex}. Then, you want to install TexLive, which is needed to compile tex files and to manage packages. The directions for TexLive installation is available [here](https://tug.org/texlive/quickinstall.html). Once I downloaded the install files, I like to run \texttt{sudo perl ./install-tl --scheme=small}. Be careful with the server location (which can be set with the \texttt{--location} parameter), as I have gotten some errors. I set \texttt{--scheme=small}, which installs about 350 packages compared to the default scheme, which installs about 5000 packages (~7GB). I also did not set \texttt{--no-interaction} since I want to slightly modify the \texttt{--texuserdir} to some other path rather than just my home directory. 

\subsection{TLMGR}

  Once you installed everything, make sure to add the binaries to PATH, which will allow you to access the \textbf{tlmgr} package manager, which pulls from the CTAN (Comprehensive TeX Archive Network) and gives VimTex access to these executables. Unfortunately, the small scheme installation does not also install the \textbf{latexmk} compiler, which is recommended by VimTex. We can simply install this by running 
  ```
  sudo tlmgr install latexmk
  ```
  Now run `:checkhealth` in Neovim and make sure that everything is OK, and install whatever else is needed. 


  To install other Latex packages (and even document classes), we can use tlmgr. All the binaries and packages are located in \texttt{/usr/loca/texlive/202*/} and since we're modifying this, we should run it with root privileges. The binaries can also be found here. Let's go through some basic commands: 
  \begin{enumerate}
    \item List all available packages: \texttt{tlmgr list}
    \item List installed packages: \texttt{tlmgr list --only-installed} (the packages with the `i` next to them are installed)
    \item Install a package and dependencies: \texttt{sudo tlmgr install amsmath tikz} 
    \item Reinstall a package: \texttt{sudo tlmgr install amsmath --reinstall}
    \item Remove a package: \texttt{sudo tlmgr remove amsmath} 
  More commands can be found \href{http://tug.ctan.org/info/tlmgrbasics/doc/tlmgr.pdf}{here} for future reference.  
  \end{enumerate}

\subsection{PDF Viewers}

I will already assume you have a PDF viewer installed. On Arch Linux I use \textbf{zathura}, which is lightweight and also comes with vim motions for navigation. On Mac the most similar is \textbf{Skim}, which also has keybindings and supports a dark mode\footnote{Under Skim, Settings, PDF Display, Invert Colors for Dark Mode} for PDF, which is a lot better on the eyes. 

\subsection{Text Mode}

  The default mode is text mode. To be honest most of this is Google-able, so I write some settings that are moderately complicated that I might need to reference for the future. 

\subsection{Math Mode} 

  You can write math equations. 

\subsection{Macros}

\subsection{Figures and Tikz}

  After this, you can install Inkscape, which is free vector-based graphics editor (like Adobe Illustrator). It is great for drawing diagrams, and you can generate custom keymaps that automatically open Inkscape for drawing diagrams within LaTeX, allowing for an seamless note-taking experience.  



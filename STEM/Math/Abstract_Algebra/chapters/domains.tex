\section{Domains}

  We can see that domains behave similarly to the integers, but with the missing property that $\times$ is commutative. This motivates the following definition of an integral domain, which can be seen as a generalization of the integers. 

  \begin{definition}[Domain, Integral Domain]
    A ring $R$ with no zero divisors for every element is called a \textbf{domain}. An \textbf{integral domain} is a commutative domain $R$.\footnote{Almost always, we work with integral domains so we will default to this.} 
  \end{definition} 

  \begin{example}[Domains vs Integral Domains]
    We show some examples of domains and integral domains. 
    \begin{enumerate}
      \item $(\mathbb{Z}, +, \times)$ is an integral domain
      \item $(\mathbb{Q}, +, \times)$ is an integral domain. 
      \item $(\mathbb{R}, +, \times)$ is an integral domain. 
      \item Quaternions $\mathbb{H}$ are not commutative but are a domain. 
    \end{enumerate}
  \end{example} 

  \begin{example}[Non-Domains]
    Here are some examples of non-domains. 
    \begin{enumerate}
      \item The ring of $n \times n$ matrices over any nonzero ring when $ n \geq 2$ is not a domain. Given matrices $A, B$, if the image of $B$ is in the kernel of $A$, then $A B = 0$.
      \item The ring of continuous functions on the interval is not a domain. To see why, notice that given the piecewise functions 
      \begin{equation}
        f (x) = \begin{cases}
          1 - 2x & x \in [0, \frac{1}{2}] \\
          0 & x \in [\frac{1}{2}, 1] 
        \end{cases}, \qquad 
        g (x) = \begin{cases}
          0 & x \in [0, \frac{1}{2}] \\
          2x - 1 & x \in [\frac{1}{2}, 1] 
        \end{cases}
      \end{equation}
      $f, g \neq 0$, but $f g = g f = 0$. 

      \item A product of two nonzero commutative rings with unity $R \times S$ is not an integral domain since $(1,0) \cdot (0, 1) = (0, 0) \in R \times S$. 
    \end{enumerate}
  \end{example}

  The nice thing about not having zero divisors is that they give a nice cancellation property in rings. Therefore, a nonzero element that is not a zero divisor enjoys some of the properties of a unit without necessarily possessing a multiplicative inverse. 

  \begin{lemma}[Cancellation Property of Elements]
    In an integral domain $R$, if $ab = ac$, then either $a = 0$ or $b = c$. 
  \end{lemma}
  \begin{proof}
    We have 
    \begin{equation}
      ab = ac \implies ab - ac = a(b - c) = 0
    \end{equation}
    and since $R$ is a domain, one of the claims must follow. 
  \end{proof}

  This extends into ideals as well. 

  \begin{theorem}[Cancellation Property of Ideals]
    In an integral domain $R$ and two ideals $I, J \subset R$, if $aI = aJ$, then either $a = 0$ or $I = J$. 
  \end{theorem}
  \begin{proof}
    
  \end{proof}

  Another nice property is that the characteristic of an integral domain must be prime and that gcd's---while not yet unique---are now guaranteed to be associated. 

  \begin{corollary}[Characteristic of Integral Domain]
    If $R$ is an integral domain, then $\Char(R)$ is either $0$ or a prime number. 
  \end{corollary}
  \begin{proof}
    Let $m \in \mathbb{Z}$ be such that $\langle m \rangle = \ker{f}$. If $m = ab$, then $f(a) f(b) = f(m) = 0$. Since $R$ is an integral domain, $f(a) = 0$ or $f(b) = 0$. Thus $d \in \ker{f} = \langle m \rangle$ or $e \in \ker{f} = \langle m \rangle \implies m$ is prime or $0$. 
  \end{proof}

  However, the converse is not true, as there exist rings with prime characteristic that are not integral domains, e.g. $\mathbb{Z}_p \times \mathbb{Z}_p$. 

  \begin{theorem}[Ideals of an Integral Domain]
    $R$ is an integral domain if and only if the only ideal $\langle 0 \rangle$ of $R$ is a prime ideal. 
  \end{theorem}
  \begin{proof}
    
  \end{proof}

  \begin{theorem}[Primes are Irreducible]
    In an integral domain $R$, a prime $p \in R$ is always irreducible. 
  \end{theorem}

  Here is an alternative equivalent characterization of an integral domain. 

  \begin{definition}[Regular Elements]
     An element $r$ of a ring $R$ is \textbf{regular} if the mapping 
     \begin{equation}
       \rho: R \longrightarrow R, \qquad x \mapsto x r
     \end{equation}
    is injective for all $x \in R$. 
  \end{definition}

  \begin{theorem}[Integral Domains w.r.t. Regularity]
    An integral domain is a commutative ring where every element is regular. 
  \end{theorem} 

\subsection{Unique Factorization Domains}

  \begin{definition}[Unique Factorization Domain]
    A \textbf{unique factorization domain (UFD)} is an integral domain $R$ in which every nonzero element $r \in R$ which is not a unit 
    can be written as a finite product of irreducible elements $p_i \in R$ (not necessarily distinct). 
    \begin{equation}
      r = p_1 p_2 \ldots p_n
    \end{equation}
    The decomposition is unique up to associates and permutations. 
  \end{definition}

  We have shown that in an integral domain, a prime is always irreducible. What about the converse? 

  \begin{theorem}[Primes and Irreducibility are Equivalent in UFDs]
    Given a UFD $R$, a nonzero element $p \in R$ is prime if and only if it is irreducible. 
  \end{theorem}

  The next part is that GCD's are unique.\footnote{Though we can make a slightly more general statement about uniquness in \textit{GCD domains}.} Furthermore, the following theorem gives us an algorithmic method of computing GCDs.

  \begin{theorem}[GCD in a UFD]
    Let $a, b \in R$ be two nonzero elements of a UFD $R$ and suppose 
    \begin{equation}
      a = u p_1^{e_1} p_2^{e_2} \ldots  p_n^{e_n}, \qquad b = v p_1^{f_1} p_2^{f_2} \ldots p_n^{f_n}
    \end{equation}
    are prime factorizations for $a$ and $b$, where $u, v$ are units, the primes $p_1, \ldots, p_n$ are distinct, and the exponents $e_i, f_i \geq 0$. Then the element 
    \begin{equation}
      d = p_1^{\min(e_1, f_1)} p_2^{\min(e_2, f_2)} \ldots p_n^{\min(e_n, f_n)}
    \end{equation}
    is the greatest common divisor of $a$ and $b$. 
  \end{theorem}
  \begin{proof}
    Since the exponents of each of the primes occurring in $d$ are no larger than the exponents occurring in the factorizations of both $a$ and $b$, $d$ divides $a$ and $b$. To show that $d$ is a greatest common divisor, let $c$ be any common divisor of $a$ and $b$, and we can factor it as 
    \begin{equation}
      c = q_1^{g_1} q_m^{g_m} \ldots q_m^{g_m}
    \end{equation}
    Since each $q_i$ divides $c$, it hence divides $a$ and $b$, and we see that $q_i$ must divide one of the primes $p_j$. In particular, up to associates the primes occurring in $c$ must be a subset of the primes occurring in $a$ and $b$. 
    \begin{equation}
      \{q_1, \ldots, q_m\} \subset \{p_1, \ldots, p_n\}
    \end{equation}
    Similarly, the exponents for the primes occurring in $c$ must be no larger than those occurring in $d$. This implies that $c \mid d$. 
  \end{proof}

\subsection{Principal Ideal Domains}

  \begin{definition}[Principal Ideal Domain]
    A \textbf{principal ideal domain}, also called a \textbf{PID}, is an integral domain in which every ideal is principal.  
  \end{definition}

  So a principal ideal domain is an integral domain by definition. It may seem that PIDs are an oddly specific structure to be studying separately, but this actually turns out to unlock a lot more nice properties that we are familiar with. The first is that GCDs are now unique, which is great. Second, we have Bezout's identity, saying that if $x$ and $y$ are elements of a PID without common divisors, then every element of the PID can be written in the form $a x + b y$. Finally, and most importantly, any element of a PID has a unique decomposition into irreducible factors. We now introduce some examples of PIDs, which are not as trivial and should be introduced as theorems. 

  \begin{example}[Integers and Polynomials over Fields are PIDs]
    The following are all examples of principal ideal domains. 
    \begin{enumerate}
      \item It is quite easy to see that any field $\mathbb{F}$ is a PID since the only two possible ideals are $\{0\}$ and $\mathbb{F}$, both of which are principal. 

      \item The ring of integers $\mathbb{Z}$ is a PID. If $I \subset \mathbb{Z}$ is an ideal, then if $I = \langle 0 \rangle$, then we're done. Otherwise, let $a \in I$ be the smallest positive integer in $I$. It is clear that $\langle a \rangle \subset I$. Now given an element $b \in I$, by the Euclidean algorithm we have $b = aq + r$ with $r < a$. Since $a, b \in I$, it follows that $r \in I$. But since $0 \leq r < a$ and $a$ is the smallest positive integer, $r = 0$, and so $b = aq \implies b \in \langle a \rangle$. 
    \end{enumerate}
  \end{example}

  Note that we have established the existence and uniqueness of the gcd in UFDs. In PIDs, we can say something stronger through \textit{Bezout's lemma}.\footnote{Actually, this holds for a slightly more general structure called \textit{Bezout rings}. } It allows us to represent the gcd as a linear combination of ring elements. 

  \begin{lemma}[Bezout's Lemma in PIDs]
    Let $R$ be a PID and $x, y \in R$. Then, there exists $a, b \in R$ s.t. 
    \begin{equation}
      ax + by = d
    \end{equation}
  \end{lemma}
  \begin{proof}
    We can prove this by showing that $\langle x \rangle + \langle y \rangle = \langle d \rangle$. We know that sums of ideals are ideals, and since $R$ is a PID, $I = \langle x \rangle + \langle y \rangle$ must be principal, i.e. $I = \langle d \rangle$ for $d = a_0 x + b_0 y$ for some $a_0, b_0 \in R$. We claim that $d$ is the gcd. 
    \begin{enumerate}
      \item \textit{$d$ is a divisor}. We have $x \in I = \langle d \rangle$, which implies that $d \mid x$. Similarly for $y$, we have $d \mid y$. 
      \item \textit{$d$ is greatest}. If $c \mid x, c \mid y$, then $x = rc$ and $y = sc$, and substituting this in for $d$, we have 
        \begin{equation}
          d = a_0 r c + b_0 s c \implies c \mid d 
        \end{equation} 
    \end{enumerate}
  \end{proof}

  Note that in general, a UFD does not have to satisfy Bezout's lemma. 

  \begin{example}[UFDs Doesn't Necessarily Satisfy Bezout's Lemma]
    Take the UFD $\mathbb{Z}[x]$, with $2, x \in \mathbb{Z}[x]$. It clearly has gcd of $1$, but there are no solutions to the equation 
    \begin{equation}
      2 f(x) + x g(x) = 1
    \end{equation}
  \end{example}

  \begin{corollary}[Ideals Generated by Irreducible Elements]
    \label{thm:irreducible-maximal}
    Let $R$ be a PID and $I \subsetneq R$ be a proper ideal. 
    \begin{enumerate}
      \item If $a \in I$ is irreducible, then $I = \langle a \rangle$.
      \item $a$ is irreducible iff $\langle a \rangle$ is maximal. 
    \end{enumerate}
  \end{corollary}
  \begin{proof}
    For the first claim, it is clear that $a \in I \implies \langle a \rangle \subset I$. Now we show that $I$ cannot be strictly bigger. Assume that it was, i.e. take $b \in I \setminus \langle a \rangle$. Then, there exists $x, y \in R$ s.t. 
      \begin{equation}
        ax + by = 1
      \end{equation}
    and so $1 \in I \implies I = R$, which contradicts the fact that $I$ is proper. Therefore $I = \langle a \rangle$. This also proves the forward direction of the second claim. For the reverse direction, we prove the contrapositive by assuming that $\langle a \rangle$ is not maximal. Then there exists an ideal $I$ s.t. $\langle a \rangle \subsetneq I \subsetneq R$. Since $R$ is a PID, $I = \langle b \rangle$, and so $\langle a \rangle \subset \langle b \rangle \implies b \mid a$. Therefore, $a$ is not irreducible. 
  \end{proof}

  We finally establish the hierarchy of PIDs. 

  \begin{theorem}[PIDs are UFDs]
    Every principal ideal domain is a unique factorization domain. 
  \end{theorem}
  \begin{proof}
    We show that it is impossible to find an infinite sequence $a_1, a_2, \ldots$ s.t. $a_i$ is divisible by $a_{i+1}$ but is not an associate. Once done we can iteratively factor an element as we are guaranteed this process terminates. Suppose such a sequence exists. Then the $a_i$ generate the sequence of distinct principal ideals $\langle a_1 \rangle \subset \langle a_2 \rangle \subset \ldots$. Then union of these ideals is some principal ideal $\langle a \rangle$. So $a \in \langle a_n \rangle$ for some $n$ by definition of containment in the intersection. But then for all $i \geq n$, this must mean that $\langle a_i \rangle = \langle a_n \rangle$, which implies that $\langle a \rangle$ can be obtained through a finite intersection, a contradiction. 

    Now we prove uniqueness. Each irreducible $p$ generates a maximal deal $\langle p \rangle$ from \ref{thm:irreducible-maximal}. Next suppose an element of $R$ has two factorizations. 
    \begin{equation}
      p_1 p_2 \ldots p_r = q_1 q_2 \ldots q_s 
    \end{equation}
    Consider the ideals $\langle p_i \rangle, \langle q_i \rangle$. Relabel so that $p_1$ generates a minimal ideal amongst these (i.e. does not strictly contain another one of these ideals). Now we show that $\langle p_1 \rangle = \langle q_i \rangle$ for some $i$. Suppose not. Then $\langle p_1 \rangle$ does not contain any $q_i$, thus $q_i$ is nonzero modulo $\langle p_1 \rangle$  for all $i$, which is a contradiction because the LHS of the above equation is zero modulo $\langle p_1 \rangle$. 

    Relabel so that $\langle p_1 \rangle = \langle q_1 \rangle$. Then $p_1 = w q_1$ for some unit $u$. Cancelling gives $u p_2 \ldots p_r = q_2 \ldots q_s$. The element $u p_2$ is also irreducible, so by induction we have that the factorization is unique. 
  \end{proof}

  \begin{example}[UFD that is not a PID]
    $\mathbb{Z}[x]$ is a UFD, but not a PID. We can see this in two ways. 
    \begin{enumerate}
      \item The ideal $I = \langle 2, x \rangle \subset \mathbb{Z}[x]$ is not principal. Suppose it was. Then we have $\langle a \rangle = \langle 2, x \rangle$ for some $a \in \mathbb{Z}[x]$. So $a \mid 2$ and $a \mid x$. Note that $a \mid 2 \implies a = 1$ or $a = 2$, but $a = 2$ means that $a \nmid x$. So $a = 1$, and so $I = R$. However, $x + 1 \not\in I$ since the multiples of $x$ cannot affect the constant term, and multiples of $2$ must be even. So $I \neq R$. 
      \item If it was a PID, then from \ref{thm:pid-field} this would imply that $\mathbb{Z}$ is a field, which we know is not. 
    \end{enumerate}
  \end{example}

\subsection{Euclidean Domains}

  We have seen that PIDs unlock a lot of familiar properties that we see in integers. In fact, pretty much everything holds except for the existence of Euclidean algorithm for factorization, which turns out to be extremely powerful. First, we define the notion of a \textit{norm} on an integral domain $R$, which is no more than a measure of size. Note that it is different from the usual sense of norm in vector spaces, which satisfies the additional axioms of scalar multiplication and triangle inequality. 

  \begin{definition}[Norm on an Integral Domain]
    Given an integral domain $R$, a function $N: R \to \mathbb{N} \cup \{0\}$ with $N(0) = 0$ is called a \textbf{(Euclidean) norm} on the integral domain $R$. 
  \end{definition}

  \begin{definition}[Euclidean Domain]
    An integral domain $R$ is a \textbf{Euclidean domain} if there is a norm $N$ on $R$ such that for any two elements $a, b \in R$ with $b \neq 0$, there exists elements $q, r \in R$, with 
    \begin{equation}
      a = qb + r, \qquad r = 0 \text{ or } N(r) < N(b) 
    \end{equation}
    $q$ is called the \textbf{quotient} and $r$ is called the \textbf{remainder}. 
  \end{definition}

  The two prime examples are the integers and polynomials. 

  \begin{example}[Integers]
    $\mathbb{Z}$ is a Euclidean domain with Euclidean division, also called long division, defined 

    \begin{center}
      \intlongdivision{521}{13}
    \end{center}
  \end{example}

  \begin{example}[Gaussian Integers]
    The subring of $\mathbb{C}$, defined
    \begin{equation}
      \mathbb{Z}[i] \equiv \{ a + b i \mid a, b \in \mathbb{Z} \}
    \end{equation}
    is a Euclidean integral domain with respect to the norm 
    \begin{equation}
      N(c) \equiv a^2 + b^2
    \end{equation}
    since $N(c d) = N(c) N(d)$ and the invertible elements of $\mathbb{Z}[i]$ are $\pm 1, \pm i$. 
  \end{example}

  \begin{example}[Dyadic Rationals]
    The ring of rational numbers of the form $2^{-n} m, \; n \in \mathbb{Z}_+, m \in \mathbb{Z}$, is a Euclidean domain. To define the norm, we can first assume that $m$ can be prime factorized into the form 
    \begin{equation}
      m = \pm \prod_{i} p_{i}^{k_i}, \; p \text{ prime}
    \end{equation}
    and the norm is defined 
    \begin{equation}
      N(\frac{m}{2^n}) \equiv 1 + \sum_i k_i
    \end{equation}
    We must further show that division with remainder is possible, but we will not show it here. 
  \end{example}

  The first implication of a division algorithm for an integral domain is that it forces every ideal of $R$ to be principal. 

  \begin{theorem}[Euclidean Domains are PIDs]
    Let $R$ be a Euclidean domain. 
    \begin{enumerate}
      \item $R$ is a principal ideal domain. 
      \item Every nonzero ideal $I \subset R$ is of the form $\langle d \rangle$, where $d \in I$ is an element of minimum norm. 
    \end{enumerate}
  \end{theorem}
  \begin{proof}
    Let $I$ be an ideal. Then if $I$ is the zero ideal, there is nothing to prove. Otherwise let $d$ be a nonzero element $I$ of minimum norm, which exists since the set $\{N(a) \mid a \in I \} \subset \mathbb{N}$ has minimum element by the Well Ordering Principl. Clearly $\langle d \rangle \subset I$. To show the reverse inclusion, let $a \in I$. Then we use the division algorithm to see that $a = qd + r$ with $r = 0$ or $N(r) < N(d)$. Then $r = a - qd$ and both $a, qd \in I$, which means $r \in I$. By the minimality of the norm of $d$, $r = 0$. Therefore $a = qd \in \langle d \rangle$, implying that $I \subset \langle d \rangle$. 
  \end{proof}

  \begin{corollary}[Fundamental Theorem of Arithmetic]
    $\mathbb{Z}$ is a unique factorization domain. 
  \end{corollary}
  \begin{proof}
    $\mathbb{Z}$ is a Euclidean domain, hence a PID, hence a UFD. 
  \end{proof}

  A useful fact that we will use later is to verify whether a quotient ring is a Euclidean domain. There is an analogous statement for fields. 

  \begin{theorem}[Quotient Rings as Euclidean Domains]
    Let $R$ be a nontrivial commutative ring and $I \subset R$ an ideal. $R/I$ is a Euclidean domain iff $I$ is a prime ideal. 
  \end{theorem}
  \begin{proof}
    TBD
  \end{proof}

  Finally we show a condition that a ring is not a Euclidean domain. 

  \begin{definition}[Universal Side Divisor]
    Given integral domain $R$, let $\Tilde{R} = R^\ast \cup \{0\}$ be the set of units of $R$ together with $0$. An element $u \in R \setminus \Tilde{R}$ is called a \textbf{universal side divisor} if for every $x \in R$ there is some $z \in \Tilde{R}$ s.t. $u \mid x - z$ in $R$. 
  \end{definition}

  In other words, there is a type of division algorithm for $u$: Every $x$ may be written $x = qu + z$, where $z$ is either $0$ or a unit. The existence of universal side divisors is a weakening of the Euclidean condition. 

  \begin{theorem}[Euclidean Domains Contain Universal Side Divisors]
    Let $R$ be an integral domain that is not a field. If $R$ is a Euclidean domain, then there are universal side divisors in $R$. 
  \end{theorem}
  \begin{proof}
    TBD
  \end{proof}

  \begin{example}[]
    We claim that the quadratic integer ring $R = \mathbb{Z}[\frac{1 + \sqrt{19}}{2}]$ is not a Euclidean domain w.r.t. any norm. 
  \end{example}

\section{Field Theory}  

  Now that we have established the theory of general rings and for polynomials, we will delve deeper into the theory of fields by talking about \textit{field extensions}, which allows us to model fields as vector spaces. This will help us in classifying fields along with providing the foundations of Galois theory. 

\subsection{Field Extensions as Vector Spaces}

  This method in which we have taken higher powers of $\alpha$ to reveal elements in $\mathbb{Q}$ reveals a deeper structure of a finite-dimensional vector space, which will be useful for analyzing certain fields in the examples below. Note that a vector space is only well-defined over a field $F$, so we will consider \textit{field extensions} from now on. First, recall that a field is trivially a vector space. 

  \begin{lemma}[Fields are a Vector Space]
    A field $F$ is a 1-dimensional vector space over itself. 
  \end{lemma} 

  It turns out that we can generalize this a bit more. 

  \begin{theorem}[Fields are a Vector Space over Subfields]
    \label{thm:fields_vector_space}
    Let $F$ be a subfield of $K$. Then $K$ is a $F$-vector space. 
  \end{theorem}
  \begin{proof}
    A $F$-vector space has $0$, addition, and multiplication by $F$. $K$ indeed has $0$, addition, and we can multiply any element of $K$ by an element of $F$. The extra axioms follow but are too verbose to write a full proof. 
  \end{proof}

  \begin{corollary} 
    $\mathbb{R}$ is an infinite-dimensional vector space over $\mathbb{Q}$. 
  \end{corollary}
  \begin{proof}
    The fact that it is a vector space immediately follows from $\mathbb{Q} \hookrightarrow \mathbb{R}$. For dimensionality, the outline is to show that $\{\sqrt{p} \mid p \text{ prime }\}$ are linearly independent. This takes work to prove and won't do it. 
  \end{proof}

  Therefore, by constructing a subfield, we can model the original field as a vector space. This additional structure warrants a name. 

  \begin{definition}[Field Extension]
    If $F \subset K$ are fields, then this is called a \textbf{field extension}. Its \textbf{degree} is the $F$-dimension of $K$, denoted
    \begin{equation}
      [K:F] \coloneqq \dim_F (K) 
    \end{equation}
  \end{definition} 

  So when we are given a field, we can automatically treat it as a vector space. Furthermore, if we are given a field extension, we can treat the larger field as a vector space over the smaller field (though it may be finite or infinite-dimensional). As we create concatenated field extensions, sometimes called a \textit{tower}, the dimensions behave nicely as well. 

  \begin{theorem}[Tower Rule]
    $E \hookrightarrow F \hookrightarrow K$ are field extensions. Then $E \hookrightarrow K$ is a field extension with degree 
    \begin{equation}
      [K:E] = [K:E] [E:F]
    \end{equation}
  \end{theorem}
  \begin{proof}
    Let $\alpha_1, \ldots \alpha_m$ be a basis for $E$ over $F$ and $\beta_1, \ldots, \beta_n$ be a basis for $K$ over $E$. We claim that $\{\alpha_i \beta_j\}$ is a basis for $K$ over $F$, with multiplication done in the field $K$. We check linear indepdence. Let $\beta \in K$ be arbitrary. Then by the $E$-basis, we have 
    \begin{equation}
      \beta = \sum_{j=1}^n x_j \beta_j
    \end{equation} 
    But since $x_j \in E$, there are elements 
    \begin{equation}
      x_j = \sum_{i=1}^m y_{ij} \alpha_i
    \end{equation}
    and so combining we get 
    \begin{equation}
      \beta = \sum_{j=1}^n \sum_{i=1}^m y_{ij} (\alpha_i \beta_j) 
    \end{equation}
    To prove linear independence, suppose $\beta = 0$. Then we have 
    \begin{equation}
      0 = \sum_{j=1}^n \sum_{i=1}^m y_{ij} (\alpha_i \beta_j) = \sum_{j=1}^n \bigg( \sum_{i=1}^m y_{ij} \alpha_i \bigg) \beta_j 
    \end{equation}
    Since $\beta_1, \ldots, \beta_n$ are linearly indpendent, we must have $\sum_{i=1}^m y_{ij} \alpha_i = 0$ for all $j$. But since $\alpha_i$'s are linear independent, this means $y_{ij} = 0$ for all $i, j$. 
  \end{proof} 

\subsection{Finite Fields} 

  We know that a field---as an integral domain---has characteristic $0$ or prime $p$. We also know that a field is a vector space, at least over itself. But now that we have shown that a field can be modeled as a vector space, we can use this to say something more specific about finite fields, which will lead to a complete classification of finite fields. 

  \begin{theorem}[Characteristic Determines Base Field of Vector Space]
    \label{thm:char_field}
    Given a field $F$, 
    \begin{enumerate}
      \item If $\Char(F) = p$, then $F$ is a vector space over $\mathbb{Z}_p$. 
      \item If $\Char(F) = 0$, then $F$ is a vector space over $\mathbb{Q}$. 
    \end{enumerate}
  \end{theorem}
  \begin{proof}
    
  \end{proof} 

  Therefore, just from the characteristic we can classify all fields as vector spaces over either $\mathbb{Q}$ or $\mathbb{Z}_p$. Now if we focus on finite fields, we can do a reverse classification. 

  \begin{theorem}[Finite Fields Have Cardinality $p^d$]
    Let $F$ be a finite field. Then $|F| = p^n$ for some $n \in \mathbb{N}$. 
  \end{theorem}
  \begin{proof}
    $F$ is a vector space over $\mathbb{Z}_p$ from \ref{thm:char_field}. Since $F$ has finitely many elements, $F$ has a finite spanning set, which implies $\dim_{\mathbb{Z}_p} F \leq + \infty$. Let $d$ be the dimension and $\{b_1, \ldots, b_d\}$ be the basis. The elements of $F$ are 
    \begin{equation}
      a_1 b_1 + \ldots + a_d b_d
    \end{equation}
    with $a_1, \ldots a_d \in \mathbb{Z}_p$. Thus there are $p^d$ elements of $F$, so $F \simeq \mathbb{Z}_p^d$. 
  \end{proof}

  In fact, for \textit{every} prime power there exists a unique field. Therefore we can create a bijection by proving the converse. 

  \begin{theorem}[Field for Every $p^d$]
    For every prime $p$ and $n \in \mathbb{N}$, there exists a field with $q = p^d$ elements, unique up to isomorphism.  
  \end{theorem} 
  \begin{proof} 
    Let $f(x) = x^q - x \in \mathbb{Z}_p [x]$. Then this polynomial has a splitting field $K \supset \mathbb{Z}$. Now we claim the roots of $f(x)$ in $K$ are distinct and form a subfield $F_q \subset K$. This will complete the proof since $F_q \subset K$ and $K \subset F_q \implies K = F_q$. Assume $\alpha, \beta \in K$ are roots of $f(x)$, and so $\alpha^p = \alpha$ and $\beta^p = \beta$
    \begin{enumerate}
      \item $\alpha + \beta \in K$ since by a modification of Freshman's dream, $(\alpha + \beta)^p = \alpha^p + \beta^p = \alpha + \beta$.\footnote{We induct on $n$ for $q = p^n$. For $n=1$, this is trivial by Freshmans dream. Now assume it holds for some $n \in \mathbb{N}$. Then $(x + y)^{p^{n+1}} = ( (x + y)^{p^n} )^p = (x^{p^n} + y^{p^n})^p = (x^{p^n})^p + (y^{p^n})^p = x^{p^{n+1}} + y^{p^{n+1}}$. }
      \item $(-\alpha)^q = (-1)^q \alpha^q = (-1)^q \alpha = -\alpha$ since $-1 = 1$ or $q$ is odd. 
      \item $\alpha \beta \in K$ since $\mathbb{Z}_p$ is a field and so $(\alpha \beta)^p = \alpha^p \beta^p = \alpha \beta$. 
      \item For multiplicative inverses, let $\alpha \neq 0$. Then $(\alpha^{-1})^p = (\alpha^{p})^{-1} = \alpha^{-1}$. 
      \item For all $p$, $0$ and $1$ are roots so $0, 1 \in K$. 
    \end{enumerate}
    Now we show that $K$ consists of distinct roots. Certainly $0 \in K$ with multiplicity $1$ since $f(x) = x (x^{q-1} - x)$. Now suppose nonzero $r \in K$ is a root with multiplicity $m$. The multiplicity of $r$ is the multiplicity of $0$ of 
    \begin{equation}
      f(x + r) = (x + r)^q - (x + r) = x^q + r^q - x - r = x^q - x
    \end{equation}
    where the final step follows from $0 = r^q - r$ since $r \in K$. Therefore $r$ has multiplicity $1$. Since $K[x]$ has unique factorization property, it follows that $m=1$ and every $r$ is a simple root. 

    To show that every field with $p^n$ elements is unique, let $F$ be such a field. We claim that $\Char(F) = p \implies \mathbb{Z}_p \subset F$. We claim that every element of $F$ is a root of $f(x) = x^q - x \in \mathbb{Z}_p [x]$, where $F$ is the splitting field. Let $G = F^\ast$ be the multiplicative group of units. Since $F$ is a field, then $|F^\ast| = |F| - 1 = p^d - 1$, and by constructing the cyclic group $\langle g \rangle \subset G$ for any $g \in G$, we know by Lagrange's theorem that $g^{|G|} = 1_G$, which implies that for all $x \in F$, 
    \begin{enumerate}
      \item If $x \neq 0$ then $x^{p^d - 1} = x \implies x^{p^d} = x$ and so $x \in K$. 
      \item If $x = 0$ then $x^{p^d} - x = 0$ and so $x \in K$. 
    \end{enumerate}
    Therefore $F \subset K$ with $|F| = |K|$ both finite, and so $F = K$. 
  \end{proof} 

  From this, we can write for every prime $p$ and natural $n$ the finite field of order $p^n$ as $\mathbb{F}_{p^n}$. It is clear that if $n = 1$ then $\mathbb{F}_p \simeq \mathbb{Z}_p$. The final result we will show is a hierarchy of subfields. 

  \begin{theorem}[Hierarchy of Fields]
    For a given prime $p$, if $p^m < p^n$, then 
    \begin{equation}
      F_{q^m} \subset F_{q^n} \iff m \mid n
    \end{equation}
  \end{theorem}


\section{Fields}

  \begin{definition}[Field]
    A \textbf{field} $(F, +, \times)$ is a commutative, associative ring with unity where every nonzero element is invertible (with respect to $\times$). It is usually denoted as $\mathbb{F}$. Note that $F$ is now an abelian group with respect to $\times$. 
  \end{definition}

  \begin{proposition}
    Every field is a domain. 
  \end{proposition}
  \begin{proof}
    Given $x, y \in \mathbb{F}$, assume $x y = 0$ with $x \neq 0$. Since $x$ is invertible,
    \begin{equation}
      0 = x^{-1} 0 = x^{-1} (x y) = y
    \end{equation}
    Now assuming that $y \neq 0$, since $y$ is invertible, 
    \begin{equation}
      0 = 0 y^{-1} = (x y) y^{-1} = x
    \end{equation}
  \end{proof}

  While the converse is not true, we can state the following result. 

  \begin{theorem}[Wedderburn's little theorem]
    Every finite domain is a field. 
  \end{theorem} 

  \begin{definition}[Field Extension]
    Given two fields $F \subset K$ with $\alpha \in K$. Then 
    \begin{equation}
      F[\alpha] \coloneqq \{p(\alpha) \in K \mid p \in F[x]\}
    \end{equation}
    That is, we take $\alpha \in K$, map it through all polynomials in $F[x]$, which will be contained in $K$. 
  \end{definition}

  \begin{example}[$\mathbb{Q} \subset \mathbb{R}$]
    Given $\mathbb{Q} \subset \mathbb{R}$, we can see that 
    \begin{equation}
      \mathbb{Q}[\sqrt{2}] = \{a + b \sqrt{2} \mid a, b \in \mathbb{Q} \}
    \end{equation}
    This is the case because we take $\sqrt{2} \in \mathbb{R}$, map it through all polynomials $p \in \mathbb{Q}[x]$, which will result in 
    \begin{equation}
      p(\sqrt{2}) = a_n (\sqrt{2})^n + a_{n-1} (\sqrt{2})^{n-1} + \ldots + a_2 (\sqrt{2})^2 + a_1 (\sqrt{2}) + a_0 
    \end{equation}
    which can be rearranged to the form $a + b\sqrt{2}$. Denoting the RHS as $S$, this proves that $\mathbb{Q}[\sqrt{2}] \subset S$, and it is clear that the other way is true since given $a + b \sqrt{2} \in S$, we can see that the polynomial $p(x) = a + b x$ maps $\sqrt{2}$ to it. 
  \end{example}

\subsection{Algebraically Closed Fields}

  \subsubsection{Roots of Polynomials}

    \begin{definition}
      A root $c$ of polynomial $f$ is called \textbf{simple} if $f$ is not divisible by $(x-c)^2$ and \textbf{multiple} otherwise. The \textbf{multiplicity} of a root $c$ is the maximum $k$ such that $(x-c)^k$ divides $f$. 
    \end{definition}

    \begin{theorem}
      The number of roots of a polynomial, counted with multiplicity, does not exceed the degree of this polynomial. Furthermore, these numbers are equal if and only if the polynomial is a product of linear factors.
    \end{theorem}

    \begin{theorem}[Viete's Formulas]
      Given that a polynomial $f$ factors into linear terms, that is 
      \begin{equation}
        f(x) = a_0 \prod_{i = 1}^{n} (x - c_i), c_i \text{ roots of } f
      \end{equation}
      Then the coefficients of $f$ can be presented with the formulas
      \begin{align*}
        & \sum_{i=1}^n c_i = - \frac{a_1}{a_0} \\
        & \sum_{i_1 < i_2} c_{i_1} c_{i_2} = \frac{a_2}{a_0} \\
        & \sum_{i_1< ...< i_k} \prod_{j = 1}^{k} c_{i_j} = (-1)^k \frac{a_k}{a_0} \\
        & c_1 c_2 c_3 ... c_n = (-1)^n \frac{a_n}{a_0}
      \end{align*}
    \end{theorem}

  \subsubsection{Fundamental Theorem of Algebra of Complex Numbers}

    While we have defined an upper bound for the number of roots for a polynomial, we have not determined whether a polynomial has any roots at all. Fortunately, it is sufficient to extend the field to $\mathbb{C}$ in order to strongly define a lower limit, too. 

    \begin{definition}
      A field $F$ is \textbf{algebraically closed} if every polynomial of positive degree (i.e. non-constant) in $F[x]$ has at least one root in $F$. This is equivalent to saying that every polynomial can be expressed as a product of first degree polynomials.
    \end{definition}

    \begin{proposition}
      A field $F$ is algebraically closed if and only if for each natural number $n$, every endomorphism of $F^n$ (that is, ever linear map from $F^n$ to itself) has at least one eigenvector. 
    \end{proposition}
    \begin{proof}
      An endomorphism of $F^n$ has an eigenvector if and only if its characteristic polynomial has some root. $(\rightarrow)$ So, when $F$ is algebraically closed, every characteristic polynomial, which is an element of $F[x]$, must have a root. $(\leftarrow)$ Assume that every characteristic polynomial has some root, and let $p \in F[x]$. Dividing the polynomial by a scalar doesn't change its roots, so we can assume $p$ to have leading coefficient $1$. If $p(x) = a_0 + a_1 x + ... + x^n$, then we can identify matrix 
      \begin{equation}
        A = \begin{pmatrix}
        0 & 0 & ... & 0 & -a_0 \\
        1 & 0 & ... & 0 & -a_1 \\
        0 & 1 & ... & 0 & -a_2 \\
        ... & ... & ... & ... & ... \\
        0 & 0 & ... & 1 & -a_{n-1}
        \end{pmatrix}
      \end{equation}
      such that the characteristic polynomial of $A$ is $p$. 
    \end{proof}

    \begin{proposition}
      $\mathbb{R}$ is not algebraically closed. 
    \end{proposition}
    \begin{proof}
      $x^2 + 1$ doesn't have any roots in $\mathbb{R}$. 
    \end{proof}

    \begin{theorem}
      Every polynomial of positive degree over field $\mathbb{C}$ has a root. 
    \end{theorem}

    \begin{corollary}
      In the algebra $\mathbb{C}[x]$, every polynomial splits into a product of linear factors. 
    \end{corollary}

    \begin{corollary}
      Every polynomial of degree $n$ over $\mathbb{C}$ has $n$ roots, counted with multiplicities. 
    \end{corollary}

    \begin{corollary}
      $\mathbb{C}$ is algebraically closed. 
    \end{corollary}

  \subsubsection{Roots of Polynomials with Real Coefficients}

    \begin{theorem}
      If $c$ is a complex root of polynomial $f \in \mathbb{R}[x]$, then $\bar{c}$ is also a root of the polynomial. Moreover, $\bar{c}$ has the same multiplicity as $c$. 
    \end{theorem}

    \begin{corollary}
      Every nonzero polynomial in $\mathbb{R}[x]$ factors into a product of linear terms and quadratic terms with negative discriminants. 
    \end{corollary}

    \begin{example}
    \begin{align*}
      x^5 - 1 & = (x-1) \bigg( x - \Big( \cos{\frac{2\pi}{5}} + i \sin{\frac{2\pi}{5}}\Big) \bigg) \bigg( x - \Big( \cos{\frac{2\pi}{5}} - i \sin{\frac{2\pi}{5}}\Big) \bigg) \\
      & \times \bigg( x - \Big( \cos{\frac{4\pi}{5}} + i \sin{\frac{4\pi}{5}}\Big) \bigg) \bigg( x - \Big( \cos{\frac{4\pi}{5}} - i \sin{\frac{4\pi}{5}}\Big) \bigg) \\
      & = (x-1) \bigg( x^2 - \frac{\sqrt{5} - 1}{2} x + 1\bigg) \bigg( x^2 + \frac{\sqrt{5} + 1}{2} x + 1\bigg) 
    \end{align*}
    \end{example}

    \begin{corollary}
      Every polynomial $f \in \mathbb{R}[x]$ of odd degree has at least one real root. 
    \end{corollary}
    \begin{proof}
      This is a direct result of Theorem **. Alternatively, without loss of generality we can assume that the leading coefficient of $f$ is positive. Then
      \begin{equation}
        \lim_{x \rightarrow + \infty} f(x) = + \infty, \; \lim_{x \rightarrow -\infty} f(x) = -\infty
      \end{equation}
      By the intermediate value theorem, there must be some point where $f$ equals $0$. 
    \end{proof}

    \begin{theorem}[Descartes' Theorem]
      The number of positive roots (counted with multiplicities) of a polynomial $f \in \mathbb{R}[x]$ (denote this $N(f)$) does not exceed the number of changes of sign in the sequence of its coefficients (denote this $L(f)$). Additionally, $L(f) \equiv N(f) \pmod{2}$. If all the complex roots of $f$ are real, then $L(f) = N(f)$. 
    \end{theorem}

    Note that if a polynomial has a multiple root but its coefficients are known only approximately (but with any degree of precision), then it is impossible to prove that the multiple roots exists because under any perturbation of the coefficients, however small, it may separate into simple roots or simply cease to exist. This fact leads to the "instability" of the Jordan Normal form because under any perturbation of the elements of a matrix $A$, the change may drastically affect the characteristic polynomial, hence affecting the geometric multiplicities of its eigenvectors. 

\subsection{Field of Complex Numbers}

  The impossibility of defining division on the ring of integers motivates its extension into the field of rational numbers. Similarly, the inability to take square roots of negative real numbers forces us to extend the field of real numbers to the bigger field of complex numbers. 



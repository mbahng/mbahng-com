\section{Fields}

\subsection{Ring Extensions}

  We will introduce this in a slightly different way, but by building up some theorems, we will unify these two soon enough. 
  
  \begin{definition}[Ring of Univariate Polynomial Elements] 
    Let $F \subset K$ be fields, $F[x]$ a polynomial ring, and a constant $\alpha \in K$, 
    \begin{equation}
      F[\alpha] \coloneqq \{ f(\alpha) \in F \mid f \in F[x]\} \subset K
    \end{equation}
  \end{definition} 

  \begin{lemma}[Ring Extension] 
    We have the following subring structure. 
    \begin{equation}
      F \subset F[\alpha] \subset K
    \end{equation}
    Furthermore, if $\alpha \not\in F$, then $F \subsetneq F[\alpha]$. 
  \end{lemma}
  \begin{proof}
    Note that $F \subset F[\alpha]$ since we can just take the constant polynomials, so this is not very interesting. Given two elements $\phi, \gamma \in F[\alpha]$, there exists polynomials $f, g \in F[x]$ s.t. $\phi = f(\alpha), \gamma = g(\alpha)$. Since $F[x]$ is a ring, we see that 
    \begin{align}
      \phi + \gamma & = f(\alpha) + g(\alpha) = (f + g)(\alpha) \\
      \phi \cdot \gamma & = f(\alpha) \cdot g(\alpha) = (fg)(\alpha)
    \end{align} 
    Furthermore, it is easy to check that $0$ and $1$ are the images of $\alpha$ through the $0$ and $1$ polynomials. What allows us to make this inclusion proper is that the $\alpha \in K$, which does not necessarily have to be in $F$, \textit{extends} this field a bit further, but since we can only map the one element $\alpha$, it may not cover all of $K$. 
  \end{proof} 

  Let's go through some examples. 

  \begin{example}[Radical Extensions of $\sqrt{2}$]
    Let $F = \mathbb{Q}$ and $K = \mathbb{C}$. We claim $\mathbb{Q}[\sqrt{2}] = \{a + b \sqrt{2} \mid a, b \in \mathbb{Q} \}$.
    \begin{enumerate}
      \item $\mathbb{Q}[\sqrt{2}] \subset \{a + b \sqrt{2} \mid a, b \in \mathbb{Q} \}$. $\mathbb{Q}[\sqrt{2}]$ are elements of the form
      \begin{equation}
        f(\sqrt{2}) = a_n (\sqrt{2})^n + a_{n-1} (\sqrt{2})^{n-1} + \ldots + a_2 (\sqrt{2})^2 + a_1 \sqrt{2} + a_0
      \end{equation} 
      This can be written by collecting terms, of the form $a + b \sqrt{2}$. 

      \item $\mathbb{Q}[\sqrt{2}] \supset \{a + b \sqrt{2} \mid a, b \in \mathbb{Q} \}$. Given an element $a + b \sqrt{2}$, this is clearly in $\mathbb{Q}[\sqrt{2}]$ since it is the image of $\sqrt{2}$ under the polynomial $f(x) = a + bx$. 
    \end{enumerate}
  \end{example} 

  Given this, we may extrapolate this pattern and claim that $\mathbb{Q}[\sqrt{2} + \sqrt{3}]$ consists of all numbers of form $a + (\sqrt{2} + \sqrt{3}) b$. However, this is \textit{not} the case. 

  \begin{example}
    Given any element $\beta \in \mathbb{Q}[\sqrt{2} + \sqrt{3}]$, it is by definition of the form 
    \begin{equation}
      \beta = \sum_{k=0}^n a_k (\sqrt{2} + \sqrt{3})^k 
    \end{equation} 
    Clearly $1, \sqrt{2} + \sqrt{3} \in \mathbb{Q}[\sqrt{2} + \sqrt{3}]$ by mapping $\sqrt{2} + \sqrt{3}$ through the polynomials $f(x) = 1$ and $f(x) = $. However, we can see that $(\sqrt{2} + \sqrt{3})^2 = 5 + \sqrt{6}$,\footnote{where we use $\sqrt{6}$ as notation for $\sqrt{2} \cdot \sqrt{3}$} and so $\sqrt{6} \in \mathbb{Q}[\sqrt{2} + \sqrt{3}]$. Furthermore, we have $(\sqrt{2} + \sqrt{3})^3 = 11 \sqrt{2} + 9 \sqrt{3}$, and so with the ring properties we can conclude that 
    \begin{align}
      \frac{1}{2} \big[ (11 \sqrt{2} + 9 \sqrt{3}) - 9 (\sqrt{2} + \sqrt{3})\big] = \sqrt{2} & \in \mathbb{Q}[\sqrt{2} + \sqrt{3}] \\
      -\frac{1}{2} \big[ (11 \sqrt{2} + 9 \sqrt{3}) - 11 (\sqrt{2} + \sqrt{3})\big] = \sqrt{3} & \in \mathbb{Q}[\sqrt{2} + \sqrt{3}] \\
    \end{align} 
    If we go a bit further, we can show that 
    \begin{equation}
      \mathbb{Q}[\sqrt{2} + \sqrt{3}] = \{a + b \sqrt{2} + c \sqrt{3} + d\sqrt{6} \mid a, b, c, d \in \mathbb{Q} \}
    \end{equation}
  \end{example}

  This method in which we have taken higher powers of $\alpha$ to reveal elements in $\mathbb{Q}$ reveals a deeper structure of a finite-dimensional vector space, which will be useful for analyzing certain fields in the examples below. 

  \begin{lemma}[Vector Space Structure]
    $F[\alpha]$ is a finite-dimensional vector space over $F$. If $f(x) = a_n x^n + \ldots a_0$, then $S = \{1, \alpha, \ldots, \alpha^{n-1}\}$ spans $F[\alpha]$.\footnote{Note that this does not mean that it is a basis.} 
  \end{lemma}
  \begin{proof}
    An element of $F[\alpha]$ is of the form 
    \begin{equation}
      f(\alpha) = \sum_{k=0}^n a_k \alpha^k
    \end{equation} 
    for some $f \in F[x]$, and so it is immediate that $\{\alpha^k\}_{k \in \mathbb{N}_0}$ spans $F[\alpha]$. We claim that $\alpha^{n-1+i}$ is in $S$ for all $i > 0$. By induction, if $i = 1$, then 
    \begin{equation}
      \alpha^n = -\frac{1}{a_n} \big( a_{n-1} \alpha^{n-1} + \ldots + a_0 \big)
    \end{equation}
    which proves the claim. Now assume that $\alpha^n, \alpha^{n+1}, \ldots, \alpha^{n-1+i} \in \Span\{1, \ldots, \alpha^{n-1}\}$. Then 
    \begin{equation}
      \alpha^i f(\alpha) = 0 \implies a_n \alpha^{n+i} + \alpha_{n-1} \alpha^{n+i-1} + \ldots + a_0 \alpha^i = 0 
    \end{equation}
    and so 
    \begin{equation}
      \alpha^{n+i} = -\frac{1}{a_n} \big(a_{n-1} \alpha^{n+i-1} + \ldots + a_0 \alpha^i)
    \end{equation}
    which means that $\alpha^{n+i} \in \Span\{1, \ldots, \alpha^{n-1}\}$, completing the proof. 
  \end{proof} 

\subsection{Field Extensions} 
  
  Great, so we automatically have the ring and vector space structures on $F[\alpha]$. However, what we would really like is a field structure since that was our original goal. Remember that $F[\alpha]$ is a ring that contains both $F$ and $\alpha$. With one more assumption, we can claim that it is a field. 

  \begin{theorem}[Adjoining Fields]
    Given fields $F \subset K$, if there exists a $f \in F[x]$ s.t. $\alpha \in K$ is a root of $f$, then $F[\alpha] \subset K$ is a field. To emphasize that it is a field, we usually denote it as $F(\alpha)$ and refer it as the field obtained by \textbf{adjoining} $\alpha$ to $F$. 
  \end{theorem}
  \begin{proof}
    It is clear that $F[\alpha]$ is a commutative ring since $F$ is a field. So it remains to show that every nonzero element of $\beta \in F[\alpha]$ is a unit. By definition $\beta = p(\alpha)$ for some polynomial $p \in F[x]$.  Factor $f \in F[x]$ as the product of irreducible polynomials. Then $\alpha$ must be a root of one of those irreducible factors, say $g(x)$. Note that $g(x) \nmid p(x)$ since $p(\alpha) \neq 0$. Since $g$ is irreducible, we know that $\gcd(g, p) = 1$ and so $\exists s, t \in F[x]$ s.t. 
    \begin{equation}
      1 = s p + t g \implies 1 = s(\alpha) p(\alpha) + t(\alpha) g(\alpha) = s(\alpha) p(\alpha)
    \end{equation}  
    Therefore we have found a multiplicative inverse $s = p^{-1} \in F[\alpha]$. 
  \end{proof} 
  \begin{proof}
    We can prove it using the vector space structure. Treating $F[\alpha]$as a finite-dimensional vector space over $F$, let us define the $F$-linear function\footnote{linearity is easy to check}
    \begin{equation}
      m_b: F[\alpha] \rightarrow F[\alpha], \qquad m_b (\beta) = b\beta
    \end{equation} 
    Since $F[\alpha] \subset K$, $F[\alpha]$ is an integral domain. Thus $\not\exists \beta \in F[\alpha] \setminus \{0\}$ s.t. $b \beta = 0$. This means that the kernel of $m_b$ is $0$, and so $m_b$ is injective. By the rank-nullity theorem, it is bijective, and so there exists a $\beta \in F[\alpha]$ s.t. $b \beta = 1 \implies b$ is a unit. 
  \end{proof}

  \begin{corollary}[Adjoining Field is Minimal]
    $F[\alpha]$ is the smallest field containing $F$ and $\alpha$. 
  \end{corollary}

  \begin{example}[$\mathbb{Q}\lbrack \sqrt{3} i\rbrack$ is a Field]
    $\mathbb{Q}[\sqrt{3} i]$ is a field, hence denoted $\mathbb{Q}(\sqrt{3} i)$ since $\sqrt{3}i$ is a root of the polynomial $f(x) = x^2 + 3$. 
  \end{example}

  \begin{example}[$\mathbb{Q}\lbrack \pi \rbrack$ not a Field]
    However, $\mathbb{Q}[\pi]$ is not a field. 
  \end{example} 

  \begin{example}[Finding Multiplicative Inverses of elements in $\mathbb{Q}\lbrack \alpha \rbrack$]
    Given $\beta = p(\alpha) = \alpha^2 + \alpha - 1 \in \mathbb{Q}[\alpha]$, where $\alpha$ is a root of $f(\alpha) = \alpha^3 + \alpha + 1$, we first know that $\beta$ must have a multiplicative inverse since $\mathbb{Q}[\alpha]$ is a field. Applying the Euclidean algorithm, we have 
    \begin{equation}
      1 = \frac{1}{3} \big\{ (x+1) f(x) - (x^2 + 2) p(x)\big\} = -\frac{1}{3} (\alpha^2 + 2) p(\alpha)
    \end{equation}
    and so $\beta^{-1} = (\alpha^2 + \alpha - 1)^{-1} = -\frac{1}{3} (\alpha^2 + 2)$. We can check that 
    \begin{align}
      -\frac{1}{3} (\alpha^2 + 2) (\alpha^2 + \alpha - 1) & = -\frac{1}{3} (\alpha^4 + \alpha^3 + \alpha^2 + 2 \alpha - 2) \\
                                                          & = -\frac{1}{3} (\alpha^3 + \alpha - 2) \\
                                                          & = -\frac{1}{3} (-3) = 1
    \end{align}
  \end{example}

  Intuitively, the extra $\alpha \in K$ allows us to ``expand'' our field $F$ into a bigger field of $K$. We can also define this for multivariate polynomials.  

  \begin{definition}[Ring of Multivariate Polynomial Elements]
    Given a polynomial ring $F[x, y]$ over a field $F$ and constants $\alpha, \beta \in F$, the following definitions are equivalent. 
    \begin{align}
      F[\alpha, \beta] & \coloneqq \{ f(\alpha, \beta) \in F \mid f \in F[x, y] \} \\ 
                       & = (F[\alpha])[\beta] \\
                       & = (F[\beta])[\alpha]
    \end{align}
  \end{definition}
  \begin{proof}
    
  \end{proof} 
  
  \begin{example}[Extensions of $\sqrt{2}$ and $i$]
    We claim that 
    \begin{equation}
      \mathbb{Q}[\sqrt{2}, i] = \{ a + b \sqrt{2} + ci + d(\sqrt{2} i) \mid a, b, c, d \in \mathbb{Q}\}
    \end{equation}
    From the previous example, we know that $\mathbb{Q}[\sqrt{2}]$ are all numbers of the form $a + b\sqrt{2}$. Now we take $i \in \mathbb{C}$ and map it through all polynomials with coefficients in $\mathbb{Z}[\sqrt{2}]$, which will be of form 
    \begin{equation}
      f(i) = (a_n + b_n \sqrt{2}) i^n + (a_{n-1} + b_{n-1}\sqrt{2}) i^{n-1} + \ldots + (a_2 + b_2 \sqrt{2}) i^2 + (a_1 + b_1 \sqrt{2}) i + (a_0 + b_0 \sqrt{2})
    \end{equation} 
    However, we can see that since $i^2 = -1$, we only need to consider up to degree 1 polynomials of form 
    \begin{equation}
      (a + b \sqrt{2}) + (c + d \sqrt{2}) i 
    \end{equation}
    which is clearly of the desired form. For the other way around, this is trivial since we can construct a linear polynomial as before. 
  \end{example} 

  \begin{example}
    We claim $\mathbb{Q}[\sqrt{3} + i] = \mathbb{Q}[\sqrt{3}, i]$. 
    \begin{enumerate}
      \item $\mathbb{Q}[\sqrt{3} + i] \subset \mathbb{Q}[\sqrt{3}, i]$
      \item $\mathbb{Q}[\sqrt{3} + i] \supset \mathbb{Q}[\sqrt{3}, i]$. Note that 
        \begin{align}
          (\sqrt{3} + i)^3 = 8i & \implies i \in \mathbb{Q}[\sqrt{3} + i] \\
                                & \implies (\sqrt{3} + i) - i = \sqrt{3} \in \mathbb{Q}[\sqrt{3} + i] 
        \end{align}
        Therefore, $\mathbb{Q}[\sqrt{3} + i]$ contains the elements $1, \sqrt{3}, i$, which form the basis of $\mathbb{Q}[\sqrt{3}, i]$. 
    \end{enumerate}
  \end{example}

  \begin{example}[Extensions of $\sqrt{3}i$ and $\sqrt{3}, i$]
    We claim that $\mathbb{Q}[\sqrt{3} i] \subsetneq \mathbb{Q}[\sqrt{3}, i]$. 
    \begin{enumerate}
      \item We can see that $\{1, \sqrt{3}i \}$ span $\mathbb{Q}[\sqrt{3}i ]$ as a $\mathbb{Q}$-vector space. Therefore, 
      \begin{equation}
        \sqrt{3}, i \in \mathbb{Q}[\sqrt{3}, i] \implies \sqrt{3} i \in \mathbb{Q}[\sqrt{3}, i]
      \end{equation} 
      implies that $\mathbb{Q}[\sqrt{3} i] \subset \mathbb{Q}[\sqrt{3}, i]$. 

      \item To prove proper inclusion, we claim that $i \not\in \mathbb{Q}[\sqrt{3}i]$. Assuming that it can, we represent it in the basis $i = b_0 + b_1 \sqrt{3} i$, and so
      \begin{equation}
        -1 = (b_0 + b_1 \sqrt{3} i)^2 = (b_0^2 - 3b_1^2) + 2b_0 b_1 \sqrt{3} i
      \end{equation}
      Therefore we must have $2b_0 b_1 \sqrt{3} = 0 \implies b_0$ or $b_1$ should be $0$. If $b_0 = 0$, then $b_0^2 - 3b_1^2 = -3 b_1^2 \implies b_1^2 = 1/3$, which is not possible since $b_1^2 \in \mathbb{Q}$. If $b_1 = 0$, then $b_0 - 3 b_1^2 = b_0^2 > 0$, and so it cannot be $-1$. 
    \end{enumerate}
  \end{example}

\subsection{Splitting Fields}

  Now we return to the problem of taking a polynomial $f \in \mathbb{Q}[x]$ and finding the \textit{smallest} possible field $K \subset \mathbb{C}$ s.t. $f$ can be factored as a product of linear polynomials in $K[x]$. 

  \begin{example}[Simple Splitting Fields]
    We provide some simple examples to gain intuition. 
    \begin{enumerate}
      \item Let $f(x) = x^2 + 2x + 2 \in \mathbb{Q}[x]$. Then the roots of $f(x)$ are $-1 \pm i$, so 
      \begin{equation}
        f(x) = (x - (-1 + i)) (x - (-1 - i)) 
      \end{equation}
      and we can show that $\mathbb{Q}[-1 - i, -1+i] = \mathbb{Q}[i]$ is the splitting field of $f$. 

      \item Let $f(x) = x^2 - 2x - 1 \in \mathbb{Q}[x]$. The roots are $1 \pm \sqrt{2}$, and so 
      \begin{equation}
        f(x) = (x - (1 + \sqrt{2})) (x - (1 - \sqrt{2}))
      \end{equation}
      and so $\mathbb{Q}[\sqrt{2}]$ is the splitting field of $f$. 

      \item Let $f(x) = x^6 - 1 \in \mathbb{Q}[x]$. We can factor 
        \begin{equation}
          f(x) = (x-1) (x + 1) (x^2 + x + 1) (x^2 - x + 1)
        \end{equation} 
        and the non-rational roots are $\frac{\pm 1 \pm \sqrt{3} i}{2}$. Thus the splitting field of $f$ is $\mathbb{Q}[\sqrt{3} i]$. 
    \end{enumerate}
  \end{example}

  \begin{example}
    Let $f(x) = x^4 - 2 \in \mathbb{Q}[x]$. It follows that the roots are 
    \begin{equation}
      \{ \sqrt[4]{2}, \sqrt[4]{2}, -\sqrt[4]{2}, - \sqrt[4]{2} i \} = \Big\{ \sqrt[4]{2}, \sqrt[4]{2} e^{\frac{2\pi i}{4}}, \sqrt[4]{2} e^{\frac{4\pi i}{4}}, \sqrt[4]{2} e^{\frac{6\pi i}{4}} \Big\}
    \end{equation}
    thus the splitting field of $f$ is 
    \begin{equation}
      \mathbb{Q} \big( \sqrt[4]{2}, \sqrt[4]{2} e^{\frac{2\pi i}{4}}, \sqrt[4]{2} e^{\frac{4\pi i}{4}}, \sqrt[4]{2} e^{\frac{6\pi i}{4}} \big) \subset \mathbb{Q}(\sqrt[4]{2}, e^{\frac{2\pi i}{4}})
    \end{equation}
    since $\sqrt[4]{2} e^{\frac{m \pi i}{4}} \in \mathbb{Q}(\sqrt[4]{2}, e^{\frac{2\pi i}{4}})$. In fact, the two are equal, and to prove this we can see that since we are working in a field, 
    \begin{equation}
      e^{2 \pi i / 4} = \frac{\sqrt[4]{2} e^{2\pi i/4}}{\sqrt[4]{2}} \in \mathbb{Q} \big( \sqrt[4]{2}, \sqrt[4]{2} e^{\frac{2\pi i}{4}}, \sqrt[4]{2} e^{\frac{4\pi i}{4}}, \sqrt[4]{2} e^{\frac{6\pi i}{4}} \big) 
    \end{equation}
    which implies that $\sqrt[4]{2} \in \mathbb{Q} \big( \sqrt[4]{2}, \sqrt[4]{2} e^{\frac{2\pi i}{4}}, \sqrt[4]{2} e^{\frac{4\pi i}{4}}, \sqrt[4]{2} e^{\frac{6\pi i}{4}} \big)$. Therefore we can conclude that the splitting field is 
    \begin{equation}
      \mathbb{Q} \big( \sqrt[4]{2}, \sqrt[4]{2} e^{\frac{2\pi i}{4}}, \sqrt[4]{2} e^{\frac{4\pi i}{4}}, \sqrt[4]{2} e^{\frac{6\pi i}{4}} \big) = \mathbb{Q}(\sqrt[4]{2}, e^{\frac{2\pi i}{4}})
    \end{equation}
  \end{example} 

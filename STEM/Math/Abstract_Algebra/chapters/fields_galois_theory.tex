\section{Field and Galois Theory}

\subsection{Extensions and Splitting Fields} 

  Great, so by establishing the fact that $\mathbb{C}$ is algebraically closed, this gives us a ``safe space'' to work in, in the sense that if we take any subfield $F \subset \mathbb{C}$ and find a polynomial $f(x) \in F[x]$, we are \textit{guaranteed} to find a linear factorization of $f$ in $\mathbb{C}[x]$. Let's define this a bit more generally for arbitrary fields $F \subset K$. 

  \begin{definition}[Field Extension]
    The pair of fields $F \subset K$ is called a \textbf{field extension}. 
  \end{definition}

  Therefore, if $K$ is algebraically closed and $F \subset K$ is a field extension, $f(x) \in F[x]$ is guaranteed to \textit{split} completely into linear factors. This is true for \textit{all} $f(x) \in F[x]$, but now if we \textit{fix} $f(x) \in F[x]$, perhaps we don't need the entire field $K$ to split $f(x)$. Maybe we can work in a slightly larger field $E$---such that $F \subset E \subset K$---where $f(x)$ splits in $E$. This process of finding such a minimal field is important to understand the behavior of roots of such polynomials. 

  \begin{definition}[Splitting Field]
    Given a field extension $F \subset K$ and a polynomial $f \in F[x]$, 
    \begin{enumerate}
      \item $f$ \textbf{splits} in $K$ if $f$ can be written as the product of linear polynomials in $K[x]$. 
      \item If $f$ splits in $K$ and there exists no field $E$ s.t. $F \subsetneq E \subsetneq K$, then $K$ is called a \textbf{splitting field} of $f$.\footnote{i.e. the splitting field is the smallest field that splits $f$.} 
    \end{enumerate}
  \end{definition}

  \begin{example}[Don't Need(?) Complex]
    Consider the following. 
    \begin{enumerate}
      \item Let $f(x) = x^2 - 1$. If $f(x) \in \mathbb{R}[x]$, it does split in $\mathbb{R}$. In fact, even if we consider it as an element of $\mathbb{Z}_2 [x]$, it still splits into $(x + 1)(x - 1)$. 
      \item Let $f(x) = x^2 - 2$. If $f(x) \in \mathbb{Q}[x]$, it doesn't split in $\mathbb{Q}$ since the roots $\pm \sqrt{2} \not\in \mathbb{Q}$, but $\pm \sqrt{2}$ are real numbers, so $f(x)$ does in fact split in $\mathbb{R}$ since it splits into $(x + \sqrt{2}) (x - \sqrt{2})$. However, maybe it is not the (smallest) splitting field. 
      \item Let $f(x) = x^2 + 1$. We can see that if we consider it as an element of $\mathbb{Q}[x]$ or $\mathbb{R}[x]$, neither fields split $f(x)$ since $\pm i$ are its roots and therefore are contained in the coefficients of its linear factors. We know that it definitely splits in $\mathbb{C}$, but can we find a smaller field that splits $f(x)$? Perhaps.  
    \end{enumerate}
  \end{example}

  So how does one find a splitting field? Note that in the example above, we have found that there were some roots $\alpha$ of certain polynomials $f(x) \in F[x]$ are not contained in $F$. Therefore, what we want to do is find the smallest field $F$ containing both $F$ and $\alpha$ (plus any other $\alpha$'s). This smallest such field is called an \textit{adjoining field}. 

\subsection{Finite Fields}


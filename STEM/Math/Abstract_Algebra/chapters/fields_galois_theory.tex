\section{Fields and Galois Theory}

\subsection{The Rational Numbers}

  \subsubsection{Field Properties} 

    Now that we've reviewed some fields, let's construct $\mathbb{Q}$ from $\mathbb{Z}$ and verify it's a field. 

    \begin{definition}[Rationals]
      Given the ordered ring of integers $(\mathbb{Z}, +_{\mathbb{Z}}, \times_{\mathbb{Z}}, \leq_{\mathbb{Z}})$ the \textbf{rational numbers} $(\mathbb{Q}, +_{\mathbb{Q}}, \times_{\mathbb{Q}})$ are defined as such. 
      \begin{enumerate}
        \item $\mathbb{Q}$ is the quotient space on $\mathbb{Z} \times \mathbb{Z} \setminus \{0\}$ with the equivalence relation $\sim$ 
        \begin{equation}
          (a, b) \sim (c, d) \iff a \times_{\mathbb{Z}} d = b \times_{\mathbb{Z}} c
        \end{equation} 
        We denote this class as $(a, b)$, where $b > 0$, since if $b < 0$, we know that $(-a, -b)$ are also in this order. 

        \item The additive and multiplicative identities are 
        \begin{equation}
          0_{\mathbb{Q}} \coloneqq (0_{\mathbb{Z}}, a), \;\;\; 1_{\mathbb{Q}} \coloneqq (a, a)
        \end{equation}

        \item Addition on $\mathbb{Q}$ is defined 
        \begin{equation}
          (a, b) +_{\mathbb{Q}} (c, d) \coloneqq \big( (a \times_{\mathbb{Z}} d) +_{\mathbb{Z}} (b \times_{\mathbb{Z}} c), b \times_{\mathbb{Z}} d \big) 
        \end{equation}

        \item The additive inverse is defined 
        \begin{equation}
          -(a, b) \coloneqq (-a, b)
        \end{equation}

        \item Multiplication on $\mathbb{Q}$ is defined 
        \begin{equation}
          (a, b) \times_{\mathbb{Q}} (c, d) \coloneqq \big( a \times_{\mathbb{Z}} c, b \times_{\mathbb{Z}} d \big)
        \end{equation} 

        \item The multiplicative inverse is defined 
        \begin{equation}
          (a, b)^{-1} \coloneqq (b, a)
        \end{equation}
      \end{enumerate}
    \end{definition}

    \begin{theorem}[Rationals are a Field]
      $\mathbb{Q}$ is a field. 
    \end{theorem} 
    \begin{proof}
      We do a few things. 
      \begin{enumerate}
        \item Verify the additive identity. 
        \begin{equation}
          (a, b) + (0, c) = (ac + 0b, bc) = (ac, bc) \sim (a, b)
        \end{equation}
        \item Verify the multiplicative identity. 
        \begin{equation}
          (a, b) \times (c, c) = (ac, bc) \sim (a, b)
        \end{equation}
        \item Additive inverse is actually an inverse. 
        \begin{equation}
          (a, b) + (-a, b) = (ab + (-ba), bb) = (0, bb) \sim (0, 1)
        \end{equation}
        \item Multiplicative inverse is actually an inverse. 
        \begin{equation}
          (a, b) \times (b, a) = (ab, ba) = (ab, ab) \sim (1, 1)
        \end{equation}
        \item Addition is commutative. 
        \begin{equation}
          (a, b) + (c, d) = (ad + bc, bd) = (cb + ad, bd) = (c, d) + (a, b)
        \end{equation}
        \item Addition is associative. 
        \begin{align}
          (a, b) + ((c, d) + (e, f)) & = (a, b) + (cf + de, df) \\
                                     & = (adf + bcf + bde, bdf) \\
                                     & = (ad + bc, bd) + (e, f) \\
                                     & = ((a, b) + (c, d)) + (e, f)
        \end{align}
        \item Multiplication is commutative. 
        \begin{equation}
          (a, b) \times (c, d) = (ac, bd) = (ca, db) = (c, d) \times (a, b)
        \end{equation}
        \item Multiplication is associative. 
        \begin{align}
          (a, b) \times ((c, d) \times (e, f)) & = (a, b) \times (ce, df) \\ 
                                               & = (ace, bdf) \\
                                               & = (ac, bd) \times (e, f) \\
                                               & = ((a, b) \times (c, d)) \times (e, f)
        \end{align}
        \item Multiplication distributes over addition. 
          \begin{align}
            (a, b) \times ((c, d) + (e, f)) & = (a, b) \times (c, d) + (a, b) \times (e, f) \\
                                            & = (ac, bd) + (ae, bf) \\
                                            & = (abcf + abde, b^2 df) \\
                                            & = (acf + ade, bdf)  
                                            & = (a, b) \times (cf + de, df)
          \end{align}
      \end{enumerate}
    \end{proof} 

    We have successfully defined the rationals, but now these are almost completely separate elements. We know that all integers are rational numbers, and so to show that the rationals are an extension of $\mathbb{Z}$ we want to identify a \textit{canonical injection} $\iota: \mathbb{Z} \rightarrow \mathbb{Q}$. This can't just be any canonical injection; it must preserve the algebraic structure between the two sets and must therefore be a \textit{ring homomorphism}. 

    \begin{theorem}[Canonical Injection of $\mathbb{Z}$ to $\mathbb{Q}$ is a Ring Homomorphism]
      Let us define the canonical injection $\iota: \mathbb{Z} \rightarrow \mathbb{Q}$ to be $\iota(a) = (a, 1)$. This is a ring homomorphism. 
    \end{theorem}
    \begin{proof} 
      We show a few things. 
      \begin{enumerate}
        \item Preservation of addition. 
          \begin{align}
            \iota(a) +_{\mathbb{Q}} \iota(b) & = (a, 1) +_{\mathbb{Q}} (b, 1) \\
                                             & = (1a +_{\mathbb{Z}} 1b, 1^2) \\
                                             & = (a +_{\mathbb{Z}} b, 1) \\
                                             & = \iota(a +_{\mathbb{Z}} b) 
          \end{align}
        \item Preservation of multiplication. 
          \begin{align}
            \iota(a) \times_{\mathbb{Q}} \iota(b) & = (a, 1) \times_{\mathbb{Q}} (b, 1) \\
                                                  & = (a \times_{\mathbb{Z}} b, 1^2) \\
                                                  & = (a \times_{\mathbb{Z}} b, 1) \\
                                                  & = \iota(a \times_{\mathbb{Z}} b, 1)
          \end{align}
        \item Preservation of multiplicative identity. 
          \begin{equation}
            \iota(1_{\mathbb{Z}}) = (1, 1) = 1_{\mathbb{Q}}
          \end{equation}
      \end{enumerate}
    \end{proof} 

    \begin{example}[Numbers]
      The rationals, reals, and complex numbers are all fields.\footnote{Quaternions are not!}
    \end{example}

    Note the subfield structure $\mathbb{Q} \subset \mathbb{R} \subset \mathbb{C}$. However, we will find that there are tons of other fields lurking in between $\mathbb{Q}$ and $\mathbb{C}$ other than $\mathbb{R}$. We can actually say that there are no subfields of $\mathbb{Q}$. 

    \begin{lemma}[Rationals are a Minimal Field]
      Every subfield of $\mathbb{C}$ contains $\mathbb{Q}$. 
    \end{lemma}
    \begin{proof}
      Must contain $0$ and $1$. Keep adding $1$ and inverting it to get $\mathbb{Z}$. Now $\mathbb{Z}$ must contain units so $1/n$ also contained. Then multiply the elements to get $\mathbb{Q}$. 
    \end{proof} 

  \subsubsection{Ordered Field Properties} 

    Great, so we have established that $\mathbb{Q}$ is a field. The next property we want to formalize is order. There are countless ways to do it, but I just take the difference and claim that it is greater than $0$. 

    \begin{theorem}[Order on Rationals]
      The order $\leq_{\mathbb{Q}}$ defined on the rationals as 
      \begin{equation}
        (a, b) \leq_{\mathbb{Q}} (c, d) \iff ad \leq_{\mathbb{Z}} bc
      \end{equation}
      is a total order. Remember that we have defined $b, d > 0$. 
    \end{theorem}
    \begin{proof}
      We prove the three properties. 
      \begin{enumerate}
        \item Reflexive. 
        \begin{equation}
          (a, b) \leq_{\mathbb{Q}} (a, b) \iff ab \leq_{\mathbb{Z}} ab
        \end{equation} 

        \item Antisymmetric. 
        \begin{align}
          (a, b) \leq_{\mathbb{Q}} (c, d) & \implies ad \leq_{\mathbb{Z}} bc
          (c, d) \leq_{\mathbb{Q}} (a, b) & \implies bc \leq_{\mathbb{Z}} ad
        \end{align} 
        This implies that both $ad = bc$, which by definition means that they are in the same equivalence class. 

        \item Transitivity. Assume that $(a, b) \leq (c, d)$ and $(c, d) \leq (e, f)$. Then, we notice that $b, d, f > 0$ and therefore by the ordered ring property\footnote{If $a \leq b$ and $0 \leq c$, then $ac \leq bc$.} of $\mathbb{Z}$, we have 
        \begin{align}
          (a, b) \leq_{\mathbb{Q}} (c, d) & \implies ad \leq_{\mathbb{Z}} bc \implies adf \leq_{\mathbb{Z}} bcf \\ 
          (c, d) \leq_{\mathbb{Q}} (e, f) & \implies cf \leq_{\mathbb{Z}} de \implies bcf \leq_{\mathbb{Z}} bde
        \end{align}
        Therefore from transitivity of the ordering on $\mathbb{Z}$ we have $adf \leq bde$. By the ordered ring property\footnote{If $a \leq b$, then $a + c \leq b + c$.}  we have $0 \leq bde - adf = d(be - af)$. But notice that $d > 0$ from our definition of rationals, and therefore it must be the case that $0 \leq be - af \implies af \leq_{\mathbb{Z}} be$, which by definition means $(a, b) \leq_{\mathbb{Q}} (e, f)$. 
      \end{enumerate}
    \end{proof} 

    As soon as we define an order the concept of extrema and bounds are well defined. Let's define them too. 

    Note that given a set, we can really put whatever order we want on it. However, consider the field with the following order. 
    \begin{equation}
      \mathbb{F} = \{0, 1\}, \; 0 < 1
    \end{equation} 
    This does not behave well with respect to its operations because for example if we have $0 < 1$, then adding the same element to both sides should preserve the ordering. But this is not the case since $0 + 1 = 1 > 1 + 1 = 0$. While it may be easy to define an order, we would like it to be an ordered field. 

    \begin{definition}[Ordered Field]
      An \textbf{ordered field} is a field that has an order satisfying 
      \begin{enumerate}
        \item $y < z \implies x + y < x + z$ for all $x \in \mathbb{F}$. 
        \item $x > 0, y > 0 \implies xy > 0$. 
      \end{enumerate}
    \end{definition}

    \begin{theorem}[Properties]
      In an totally ordered field, 
      \begin{enumerate}
        \item $x > 0 \implies -x < 0$. 
        \item $x \neq 0 \implies x^2 > 0$. 
        \item If $x > 0$, then $y < z \implies xy < xz$. 
      \end{enumerate}
    \end{theorem} 
    \begin{proof}
      The first property is a single-liner 
      \begin{equation}
        0 < x \implies 0 + -x < x + -x \implies -x < 0 
      \end{equation}
      For the second property, it must be the case that $x > 0$ or $x < 0$. If $x > 0$, then by definition $x^2 > 0$. If $x < 0$, then 
      \begin{equation}
        x^2 = 1 \cdot x^2 = (-1)^2 \cdot x^2 = (-1 \cdot x)^2 = (-x)^2
      \end{equation}
      and since $-x > 0$ from the first property, we have $x^2 = (-x)^2 > 0$. For the third, we use the distributive property. 
      \begin{align}
        y < z & \implies 0 < z - y \\ 
              & \implies 0 = x 0 < x(z - y) = xz - xy \\
              & \implies xy < xz
      \end{align}
    \end{proof}

    As we have hinted, the rationals is an ordered field. 

    \begin{theorem}[Rationals are an Ordered Field]
      $\mathbb{Q}$ is an ordered field. 
    \end{theorem} 
    \begin{proof}
      We show that our defined order satisfies the definition. 
      \begin{enumerate}
        \item Assume that $y = (a, b) \leq (c, d) = z$. Let $x = (e, f)$. Then $x + y = (af + be, bf)$, $x + z = (cf + de, df)$. Therefore 
        \begin{align}
          (af + be) df & = adf^2 + bedf \\ 
                       & \leq bcf^2 + bedf \\
                       & = (cf + de) bf
        \end{align} 
        But $(af + be) df = (cf + de) bf$ is equivalent to saying $(af + be, bf) \leq_{\mathbb{Q}} (cf + de, df)$, i.e. $x + y \leq x + z$!  

        \item Let $x = (a, b), y = (c, d)$. Since $0 < x, 0 < y$, by construction this means that $0 < a, 0 < c$ (since $b, d > 0$ in the canonical rational form). By the ordered ring property of the integers, $0 < ac$. So 
        \begin{equation}
          0 < ac \iff 0 \cdot bd < ac \cdot 1 \iff (0, 1) < (ac, bd)  \iff 0_{\mathbb{Q}} < (a, c) \times_{\mathbb{Q}} (b, d) = x y
        \end{equation}
      \end{enumerate}
    \end{proof} 

    Not only is it an ordered field, but it also is consistent with the ordering on $\mathbb{Z}$! It's nice how all these properties seem to fit together. 

    \begin{theorem}[Preservation of Order]
      The canonical injection $\iota$ is an \textit{order homomorphism}. That is, for $a, b \in \mathbb{Z}$, 
      \begin{equation}
        a \leq_{\mathbb{Z}} b \iff \iota(a) \leq_{\mathbb{Q}} \iota(b)
      \end{equation}
    \end{theorem}
    \begin{proof} 
      \begin{align}
        a \leq_{\mathbb{Z}} b & \iff a \cdot 1 \leq_{\mathbb{Z}} b \cdot 1 \\
                              & \iff (a, 1) \leq_{\mathbb{Q}} (b, 1) \\
                              & \iff \iota(a) \leq_{\mathbb{Q}} \iota(b)
      \end{align}
    \end{proof}

    Note that an order can be used to generate an order topology, which we will define below. 

    \begin{definition}[Order Topology on $\mathbb{Q}$]
      The order topology on $\mathbb{Q}$ is the topology generated by the set $\mathscr{B}$ of all open intervals 
      \begin{equation}
        (a, b) \coloneqq \{ x \in \mathbb{Q} \mid a < x < b\}
      \end{equation}
    \end{definition}

    \begin{theorem}[Finite Fields]
      There are no finite ordered fields. 
    \end{theorem} 
    \begin{proof}
      Assume $\mathbb{F}$ is such an ordered field. It must be the case that $0, 1 \in \mathbb{F}$, with $0 < 1$. Therefore, we also have $0 + 1 < 1 + 1 \implies 1 < 1 + 1$. Repeating this we get 
      \begin{equation}
        0 < 1 < 1 + 1 < 1 + 1 + 1 < \ldots
      \end{equation}
      where these elements must be distinct (since only one of $>, <, =$ must be true for a totally ordered set). Since this can be done for a countably infinite number of times, $\mathbb{F}$ cannot be finite. 
    \end{proof}

  \subsubsection{Norm} 

    Note that we can also define a norm on the rationals with just the order and algebraic properties. 

    \begin{theorem}[Norm on $\mathbb{Q}$] 
      The following is indeed a norm on $\mathbb{Q}$. 
      \begin{equation}
        |x| \coloneqq \begin{cases} x & \text{ if } x \geq 0 \\ -x & \text{ if } x < 0 \end{cases}
      \end{equation} 
    \end{theorem} 

    It is well known that the metric induced by any norm is indeed a metric. Therefore we state the metric as a definition. 

    \begin{definition}[Metric on $\mathbb{Q}$]
      The Euclidean metric on $\mathbb{Q}$ is defined 
      \begin{equation}
        d(x, y) \coloneqq |x - y| = \begin{cases} x - y & \text{ if } x \geq y \\ y - x & \text{ if } x < y \end{cases}
      \end{equation}
    \end{definition}

    Thus we get to what we want: the induced topology of open balls. 
    Again, since we know from point-set topology that metric topologies are indeed topologies, we will state this as a definition rather than a theorem.  

    \begin{definition}[Open-Ball Topology on $\mathbb{Q}$]
      The Euclidean topology on $\mathbb{Q}$ is the topology generated by the set $\mathscr{B}$ of all open balls
      \begin{equation}
        B(x, r) \coloneqq \{ y \in \mathbb{Q} \mid |x - y| < r \}
      \end{equation} 
    \end{definition}

    Note that this is the same topology as the order topology. This should however be proved. 

    \begin{theorem}[Metric and Order Topologies on $\mathbb{Q}$]
      The metric and order topologies on $\mathbb{Q}$ are the same topologies. 
    \end{theorem}
    \begin{proof}
      
    \end{proof}

\subsection{Ring Extensions}

  We will introduce this in a slightly different way, but by building up some theorems, we will unify these two soon enough. 
  
  \begin{definition}[Ring of Univariate Polynomial Elements] 
    Let $F \subset K$ be fields, $F[x]$ a polynomial ring, and a constant $\alpha \in K$, 
    \begin{equation}
      F[\alpha] \coloneqq \{ f(\alpha) \in F \mid f \in F[x]\} \subset K
    \end{equation}
  \end{definition} 

  \begin{lemma}[Ring Extension] 
    We have the following subring structure. 
    \begin{equation}
      F \subset F[\alpha] \subset K
    \end{equation}
    Furthermore, if $\alpha \not\in F$, then $F \subsetneq F[\alpha]$. 
  \end{lemma}
  \begin{proof}
    Note that $F \subset F[\alpha]$ since we can just take the constant polynomials, so this is not very interesting. Given two elements $\phi, \gamma \in F[\alpha]$, there exists polynomials $f, g \in F[x]$ s.t. $\phi = f(\alpha), \gamma = g(\alpha)$. Since $F[x]$ is a ring, we see that 
    \begin{align}
      \phi + \gamma & = f(\alpha) + g(\alpha) = (f + g)(\alpha) \\
      \phi \cdot \gamma & = f(\alpha) \cdot g(\alpha) = (fg)(\alpha)
    \end{align} 
    Furthermore, it is easy to check that $0$ and $1$ are the images of $\alpha$ through the $0$ and $1$ polynomials. What allows us to make this inclusion proper is that the $\alpha \in K$, which does not necessarily have to be in $F$, \textit{extends} this field a bit further, but since we can only map the one element $\alpha$, it may not cover all of $K$. 
  \end{proof} 

  Let's go through some examples. 

  \begin{example}[Radical Extensions of $\sqrt{2}$]
    Let $F = \mathbb{Q}$ and $K = \mathbb{C}$. We claim $\mathbb{Q}[\sqrt{2}] = \{a + b \sqrt{2} \mid a, b \in \mathbb{Q} \}$.
    \begin{enumerate}
      \item $\mathbb{Q}[\sqrt{2}] \subset \{a + b \sqrt{2} \mid a, b \in \mathbb{Q} \}$. $\mathbb{Q}[\sqrt{2}]$ are elements of the form
      \begin{equation}
        f(\sqrt{2}) = a_n (\sqrt{2})^n + a_{n-1} (\sqrt{2})^{n-1} + \ldots + a_2 (\sqrt{2})^2 + a_1 \sqrt{2} + a_0
      \end{equation} 
      This can be written by collecting terms, of the form $a + b \sqrt{2}$. 

      \item $\mathbb{Q}[\sqrt{2}] \supset \{a + b \sqrt{2} \mid a, b \in \mathbb{Q} \}$. Given an element $a + b \sqrt{2}$, this is clearly in $\mathbb{Q}[\sqrt{2}]$ since it is the image of $\sqrt{2}$ under the polynomial $f(x) = a + bx$. 
    \end{enumerate}
  \end{example} 

  Given this, we may extrapolate this pattern and claim that $\mathbb{Q}[\sqrt{2} + \sqrt{3}]$ consists of all numbers of form $a + (\sqrt{2} + \sqrt{3}) b$. However, this is \textit{not} the case. 

  \begin{example}
    Given any element $\beta \in \mathbb{Q}[\sqrt{2} + \sqrt{3}]$, it is by definition of the form 
    \begin{equation}
      \beta = \sum_{k=0}^n a_k (\sqrt{2} + \sqrt{3})^k 
    \end{equation} 
    Clearly $1, \sqrt{2} + \sqrt{3} \in \mathbb{Q}[\sqrt{2} + \sqrt{3}]$ by mapping $\sqrt{2} + \sqrt{3}$ through the polynomials $f(x) = 1$ and $f(x) = $. However, we can see that $(\sqrt{2} + \sqrt{3})^2 = 5 + \sqrt{6}$,\footnote{where we use $\sqrt{6}$ as notation for $\sqrt{2} \cdot \sqrt{3}$} and so $\sqrt{6} \in \mathbb{Q}[\sqrt{2} + \sqrt{3}]$. Furthermore, we have $(\sqrt{2} + \sqrt{3})^3 = 11 \sqrt{2} + 9 \sqrt{3}$, and so with the ring properties we can conclude that 
    \begin{align}
      \frac{1}{2} \big[ (11 \sqrt{2} + 9 \sqrt{3}) - 9 (\sqrt{2} + \sqrt{3})\big] = \sqrt{2} & \in \mathbb{Q}[\sqrt{2} + \sqrt{3}] \\
      -\frac{1}{2} \big[ (11 \sqrt{2} + 9 \sqrt{3}) - 11 (\sqrt{2} + \sqrt{3})\big] = \sqrt{3} & \in \mathbb{Q}[\sqrt{2} + \sqrt{3}] \\
    \end{align} 
    If we go a bit further, we can show that 
    \begin{equation}
      \mathbb{Q}[\sqrt{2} + \sqrt{3}] = \{a + b \sqrt{2} + c \sqrt{3} + d\sqrt{6} \mid a, b, c, d \in \mathbb{Q} \}
    \end{equation}
  \end{example}

  This method in which we have taken higher powers of $\alpha$ to reveal elements in $\mathbb{Q}$ reveals a deeper structure of a finite-dimensional vector space, which will be useful for analyzing certain fields in the examples below. 

  \begin{lemma}[Vector Space Structure]
    $F[\alpha]$ is a finite-dimensional vector space over $F$. If $f(x) = a_n x^n + \ldots a_0$, then $S = \{1, \alpha, \ldots, \alpha^{n-1}\}$ spans $F[\alpha]$.\footnote{Note that this does not mean that it is a basis.} 
  \end{lemma}
  \begin{proof}
    An element of $F[\alpha]$ is of the form 
    \begin{equation}
      f(\alpha) = \sum_{k=0}^n a_k \alpha^k
    \end{equation} 
    for some $f \in F[x]$, and so it is immediate that $\{\alpha^k\}_{k \in \mathbb{N}_0}$ spans $F[\alpha]$. We claim that $\alpha^{n-1+i}$ is in $S$ for all $i > 0$. By induction, if $i = 1$, then 
    \begin{equation}
      \alpha^n = -\frac{1}{a_n} \big( a_{n-1} \alpha^{n-1} + \ldots + a_0 \big)
    \end{equation}
    which proves the claim. Now assume that $\alpha^n, \alpha^{n+1}, \ldots, \alpha^{n-1+i} \in \Span\{1, \ldots, \alpha^{n-1}\}$. Then 
    \begin{equation}
      \alpha^i f(\alpha) = 0 \implies a_n \alpha^{n+i} + \alpha_{n-1} \alpha^{n+i-1} + \ldots + a_0 \alpha^i = 0 
    \end{equation}
    and so 
    \begin{equation}
      \alpha^{n+i} = -\frac{1}{a_n} \big(a_{n-1} \alpha^{n+i-1} + \ldots + a_0 \alpha^i)
    \end{equation}
    which means that $\alpha^{n+i} \in \Span\{1, \ldots, \alpha^{n-1}\}$, completing the proof. 
  \end{proof} 

\subsection{Field Extensions} 
  
  Great, so we automatically have the ring and vector space structures on $F[\alpha]$. However, what we would really like is a field structure since that was our original goal. Remember that $F[\alpha]$ is a ring that contains both $F$ and $\alpha$. With one more assumption, we can claim that it is a field. 

  \begin{theorem}[Adjoining Fields]
    Given fields $F \subset K$, if there exists a $f \in F[x]$ s.t. $\alpha \in K$ is a root of $f$, then $F[\alpha] \subset K$ is a field. To emphasize that it is a field, we usually denote it as $F(\alpha)$ and refer it as the field obtained by \textbf{adjoining} $\alpha$ to $F$. 
  \end{theorem}
  \begin{proof}
    It is clear that $F[\alpha]$ is a commutative ring since $F$ is a field. So it remains to show that every nonzero element of $\beta \in F[\alpha]$ is a unit. By definition $\beta = p(\alpha)$ for some polynomial $p \in F[x]$.  Factor $f \in F[x]$ as the product of irreducible polynomials. Then $\alpha$ must be a root of one of those irreducible factors, say $g(x)$. Note that $g(x) \nmid p(x)$ since $p(\alpha) \neq 0$. Since $g$ is irreducible, we know that $\gcd(g, p) = 1$ and so $\exists s, t \in F[x]$ s.t. 
    \begin{equation}
      1 = s p + t g \implies 1 = s(\alpha) p(\alpha) + t(\alpha) g(\alpha) = s(\alpha) p(\alpha)
    \end{equation}  
    Therefore we have found a multiplicative inverse $s = p^{-1} \in F[\alpha]$. 
  \end{proof} 
  \begin{proof}
    We can prove it using the vector space structure. Treating $F[\alpha]$as a finite-dimensional vector space over $F$, let us define the $F$-linear function\footnote{linearity is easy to check}
    \begin{equation}
      m_b: F[\alpha] \rightarrow F[\alpha], \qquad m_b (\beta) = b\beta
    \end{equation} 
    Since $F[\alpha] \subset K$, $F[\alpha]$ is an integral domain. Thus $\not\exists \beta \in F[\alpha] \setminus \{0\}$ s.t. $b \beta = 0$. This means that the kernel of $m_b$ is $0$, and so $m_b$ is injective. By the rank-nullity theorem, it is bijective, and so there exists a $\beta \in F[\alpha]$ s.t. $b \beta = 1 \implies b$ is a unit. 
  \end{proof}

  \begin{corollary}[Adjoining Field is Minimal]
    $F[\alpha]$ is the smallest field containing $F$ and $\alpha$. 
  \end{corollary}

  \begin{example}[$\mathbb{Q}\lbrack \sqrt{3} i\rbrack$ is a Field]
    $\mathbb{Q}[\sqrt{3} i]$ is a field, hence denoted $\mathbb{Q}(\sqrt{3} i)$ since $\sqrt{3}i$ is a root of the polynomial $f(x) = x^2 + 3$. 
  \end{example}

  \begin{example}[$\mathbb{Q}\lbrack \pi \rbrack$ not a Field]
    However, $\mathbb{Q}[\pi]$ is not a field. 
  \end{example} 

  \begin{example}[Finding Multiplicative Inverses of elements in $\mathbb{Q}\lbrack \alpha \rbrack$]
    Given $\beta = p(\alpha) = \alpha^2 + \alpha - 1 \in \mathbb{Q}[\alpha]$, where $\alpha$ is a root of $f(\alpha) = \alpha^3 + \alpha + 1$, we first know that $\beta$ must have a multiplicative inverse since $\mathbb{Q}[\alpha]$ is a field. Applying the Euclidean algorithm, we have 
    \begin{equation}
      1 = \frac{1}{3} \big\{ (x+1) f(x) - (x^2 + 2) p(x)\big\} = -\frac{1}{3} (\alpha^2 + 2) p(\alpha)
    \end{equation}
    and so $\beta^{-1} = (\alpha^2 + \alpha - 1)^{-1} = -\frac{1}{3} (\alpha^2 + 2)$. We can check that 
    \begin{align}
      -\frac{1}{3} (\alpha^2 + 2) (\alpha^2 + \alpha - 1) & = -\frac{1}{3} (\alpha^4 + \alpha^3 + \alpha^2 + 2 \alpha - 2) \\
                                                          & = -\frac{1}{3} (\alpha^3 + \alpha - 2) \\
                                                          & = -\frac{1}{3} (-3) = 1
    \end{align}
  \end{example}

  Intuitively, the extra $\alpha \in K$ allows us to ``expand'' our field $F$ into a bigger field of $K$. We can also define this for multivariate polynomials.  

  \begin{definition}[Ring of Multivariate Polynomial Elements]
    Given a polynomial ring $F[x, y]$ over a field $F$ and constants $\alpha, \beta \in F$, the following definitions are equivalent. 
    \begin{align}
      F[\alpha, \beta] & \coloneqq \{ f(\alpha, \beta) \in F \mid f \in F[x, y] \} \\ 
                       & = (F[\alpha])[\beta] \\
                       & = (F[\beta])[\alpha]
    \end{align}
  \end{definition}
  \begin{proof}
    
  \end{proof} 
  
  \begin{example}[Extensions of $\sqrt{2}$ and $i$]
    We claim that 
    \begin{equation}
      \mathbb{Q}[\sqrt{2}, i] = \{ a + b \sqrt{2} + ci + d(\sqrt{2} i) \mid a, b, c, d \in \mathbb{Q}\}
    \end{equation}
    From the previous example, we know that $\mathbb{Q}[\sqrt{2}]$ are all numbers of the form $a + b\sqrt{2}$. Now we take $i \in \mathbb{C}$ and map it through all polynomials with coefficients in $\mathbb{Z}[\sqrt{2}]$, which will be of form 
    \begin{equation}
      f(i) = (a_n + b_n \sqrt{2}) i^n + (a_{n-1} + b_{n-1}\sqrt{2}) i^{n-1} + \ldots + (a_2 + b_2 \sqrt{2}) i^2 + (a_1 + b_1 \sqrt{2}) i + (a_0 + b_0 \sqrt{2})
    \end{equation} 
    However, we can see that since $i^2 = -1$, we only need to consider up to degree 1 polynomials of form 
    \begin{equation}
      (a + b \sqrt{2}) + (c + d \sqrt{2}) i 
    \end{equation}
    which is clearly of the desired form. For the other way around, this is trivial since we can construct a linear polynomial as before. 
  \end{example} 

  \begin{example}
    We claim $\mathbb{Q}[\sqrt{3} + i] = \mathbb{Q}[\sqrt{3}, i]$. 
    \begin{enumerate}
      \item $\mathbb{Q}[\sqrt{3} + i] \subset \mathbb{Q}[\sqrt{3}, i]$
      \item $\mathbb{Q}[\sqrt{3} + i] \supset \mathbb{Q}[\sqrt{3}, i]$. Note that 
        \begin{align}
          (\sqrt{3} + i)^3 = 8i & \implies i \in \mathbb{Q}[\sqrt{3} + i] \\
                                & \implies (\sqrt{3} + i) - i = \sqrt{3} \in \mathbb{Q}[\sqrt{3} + i] 
        \end{align}
        Therefore, $\mathbb{Q}[\sqrt{3} + i]$ contains the elements $1, \sqrt{3}, i$, which form the basis of $\mathbb{Q}[\sqrt{3}, i]$. 
    \end{enumerate}
  \end{example}

  \begin{example}[Extensions of $\sqrt{3}i$ and $\sqrt{3}, i$]
    We claim that $\mathbb{Q}[\sqrt{3} i] \subsetneq \mathbb{Q}[\sqrt{3}, i]$. 
    \begin{enumerate}
      \item We can see that $\{1, \sqrt{3}i \}$ span $\mathbb{Q}[\sqrt{3}i ]$ as a $\mathbb{Q}$-vector space. Therefore, 
      \begin{equation}
        \sqrt{3}, i \in \mathbb{Q}[\sqrt{3}, i] \implies \sqrt{3} i \in \mathbb{Q}[\sqrt{3}, i]
      \end{equation} 
      implies that $\mathbb{Q}[\sqrt{3} i] \subset \mathbb{Q}[\sqrt{3}, i]$. 

      \item To prove proper inclusion, we claim that $i \not\in \mathbb{Q}[\sqrt{3}i]$. Assuming that it can, we represent it in the basis $i = b_0 + b_1 \sqrt{3} i$, and so
      \begin{equation}
        -1 = (b_0 + b_1 \sqrt{3} i)^2 = (b_0^2 - 3b_1^2) + 2b_0 b_1 \sqrt{3} i
      \end{equation}
      Therefore we must have $2b_0 b_1 \sqrt{3} = 0 \implies b_0$ or $b_1$ should be $0$. If $b_0 = 0$, then $b_0^2 - 3b_1^2 = -3 b_1^2 \implies b_1^2 = 1/3$, which is not possible since $b_1^2 \in \mathbb{Q}$. If $b_1 = 0$, then $b_0 - 3 b_1^2 = b_0^2 > 0$, and so it cannot be $-1$. 
    \end{enumerate}
  \end{example}

\subsection{Splitting Fields}

  Now we return to the problem of taking a polynomial $f \in \mathbb{Q}[x]$ and finding the \textit{smallest} possible field $K \subset \mathbb{C}$ s.t. $f$ can be factored as a product of linear polynomials in $K[x]$. 

  \begin{example}[Simple Splitting Fields]
    We provide some simple examples to gain intuition. 
    \begin{enumerate}
      \item Let $f(x) = x^2 + 2x + 2 \in \mathbb{Q}[x]$. Then the roots of $f(x)$ are $-1 \pm i$, so 
      \begin{equation}
        f(x) = (x - (-1 + i)) (x - (-1 - i)) 
      \end{equation}
      and we can show that $\mathbb{Q}[-1 - i, -1+i] = \mathbb{Q}[i]$ is the splitting field of $f$. 

      \item Let $f(x) = x^2 - 2x - 1 \in \mathbb{Q}[x]$. The roots are $1 \pm \sqrt{2}$, and so 
      \begin{equation}
        f(x) = (x - (1 + \sqrt{2})) (x - (1 - \sqrt{2}))
      \end{equation}
      and so $\mathbb{Q}[\sqrt{2}]$ is the splitting field of $f$. 

      \item Let $f(x) = x^6 - 1 \in \mathbb{Q}[x]$. We can factor 
        \begin{equation}
          f(x) = (x-1) (x + 1) (x^2 + x + 1) (x^2 - x + 1)
        \end{equation} 
        and the non-rational roots are $\frac{\pm 1 \pm \sqrt{3} i}{2}$. Thus the splitting field of $f$ is $\mathbb{Q}[\sqrt{3} i]$. 
    \end{enumerate}
  \end{example}

  \begin{example}
    Let $f(x) = x^4 - 2 \in \mathbb{Q}[x]$. It follows that the roots are 
    \begin{equation}
      \{ \sqrt[4]{2}, \sqrt[4]{2}, -\sqrt[4]{2}, - \sqrt[4]{2} i \} = \Big\{ \sqrt[4]{2}, \sqrt[4]{2} e^{\frac{2\pi i}{4}}, \sqrt[4]{2} e^{\frac{4\pi i}{4}}, \sqrt[4]{2} e^{\frac{6\pi i}{4}} \Big\}
    \end{equation}
    thus the splitting field of $f$ is 
    \begin{equation}
      \mathbb{Q} \big( \sqrt[4]{2}, \sqrt[4]{2} e^{\frac{2\pi i}{4}}, \sqrt[4]{2} e^{\frac{4\pi i}{4}}, \sqrt[4]{2} e^{\frac{6\pi i}{4}} \big) \subset \mathbb{Q}(\sqrt[4]{2}, e^{\frac{2\pi i}{4}})
    \end{equation}
    since $\sqrt[4]{2} e^{\frac{m \pi i}{4}} \in \mathbb{Q}(\sqrt[4]{2}, e^{\frac{2\pi i}{4}})$. In fact, the two are equal, and to prove this we can see that since we are working in a field, 
    \begin{equation}
      e^{2 \pi i / 4} = \frac{\sqrt[4]{2} e^{2\pi i/4}}{\sqrt[4]{2}} \in \mathbb{Q} \big( \sqrt[4]{2}, \sqrt[4]{2} e^{\frac{2\pi i}{4}}, \sqrt[4]{2} e^{\frac{4\pi i}{4}}, \sqrt[4]{2} e^{\frac{6\pi i}{4}} \big) 
    \end{equation}
    which implies that $\sqrt[4]{2} \in \mathbb{Q} \big( \sqrt[4]{2}, \sqrt[4]{2} e^{\frac{2\pi i}{4}}, \sqrt[4]{2} e^{\frac{4\pi i}{4}}, \sqrt[4]{2} e^{\frac{6\pi i}{4}} \big)$. Therefore we can conclude that the splitting field is 
    \begin{equation}
      \mathbb{Q} \big( \sqrt[4]{2}, \sqrt[4]{2} e^{\frac{2\pi i}{4}}, \sqrt[4]{2} e^{\frac{4\pi i}{4}}, \sqrt[4]{2} e^{\frac{6\pi i}{4}} \big) = \mathbb{Q}(\sqrt[4]{2}, e^{\frac{2\pi i}{4}})
    \end{equation}
  \end{example} 

\subsection{Ring Homomorphisms in Polynomials} 
  
  Obviously, we can prove that things like the identity map are homomorphisms. However, the following will be used quite often. 

  \begin{example}[Evaluation Homomorphism of Polynomials]
    Given fields $F \subset K$, the \textbf{evaluation function} 
    \begin{equation}
      \ev_\alpha: F[x] \rightarrow K
    \end{equation}
    mapping $f(x) \mapsto f(\alpha)$ is a homomorphism. 
  \end{example}

  Overall, we must use this theorem cleverly in order to prove that two rings are isomorphic to each other. 

  \begin{example}
    The evaluation map 
    \begin{equation}
      \phi: \frac{\mathbb{R}[x]}{\langle x^2 + 1 \rangle} \rightarrow \mathbb{C}, \qquad \phi\big( f(x) \pmod{\langle x^2 + 1 \rangle} \big) = f(i)
    \end{equation}
    is an isomorphism.\footnote{Intuitively, we can see that the quotient ring can only consist up to linear polynomials since $x^2 \equiv -1$. This is a real vector space of dimension $2$, and so is $\mathbb{C}$, so it makes sense that they may be isomorphic. } This is because we can think of the evaluation homomorphism $\ev_i : f(x) \in \mathbb{R}[x] \mapsto f(i) \in \mathbb{R}[i]$. We know that $\mathbb{R}$ a field implies $\mathbb{R}[x]$ is a PID. Now take $\ker(\ev_i)$. We can see that it contains the polynomial $x^2 + 1$, and since it is irreducible in $\mathbb{R}[x]$, it must be the case that $\ker(\ev_i) = \langle x^2 + 1 \rangle$. Now it follows by the fundamental ring homomorphism theorem that 
    \begin{equation}
      \frac{\mathbb{R}[x]}{\ker(\ev_i)} = \frac{\mathbb{R}[x]}{\langle x^2 + 1 \rangle} \simeq \mathbb{R}[i] = \mathbb{C}
    \end{equation}
  \end{example} 

  \begin{example}
    The evaluation map 
    \begin{equation}
      \ev_{\sqrt{2}}: \mathbb{Q}[x] \mapsto \mathbb{Q}[\sqrt{2}], \qquad \ev_{\sqrt{2}} (f) = f(\sqrt{2}) 
    \end{equation}
    is a homomorphism. Furthermore, it has a kernel $\langle x^2 - 2 \rangle$ since $(x^2 - 2)$ is an irreducible polynomial in $\mathbb{Q}[x]$ containing the root $\sqrt{2}$. Therefore by the fundamental ring homomorphism theorem we have 
    \begin{equation}
      \frac{\mathbb{Q}[x]}{\langle x^2 - 2 \rangle} \simeq \mathbb{Q}[\sqrt{2}]
    \end{equation}
  \end{example} 

  \begin{theorem}[Quotient Polynomial Ring Can be Splitting Field]
    Let $F$ be a field with $f(x) \in F[x]$. 
    \begin{enumerate}
      \item Then $K = F[x] / \langle f(x) \rangle$ is a field iff $f(x)$ is irreducible in $F[x]$. 
      \item If $f(x)$ is irreducible, then $K$ contains a root $\alpha$ of $f(x)$, and $K \simeq F[\alpha]$. 
    \end{enumerate}
  \end{theorem}

  \begin{corollary}
    Any polynomial $f(x) \in F[x]$ has a splitting field. 
  \end{corollary}

  \begin{corollary}
    Let $c \in \mathbb{C}$. Then $\mathbb{Q}[c] \subset \mathbb{C}$ is a field if and only if $c$ is an algebraic number. 
  \end{corollary}

\subsection{Extensions and Splitting Fields} 

  Great, so by establishing the fact that $\mathbb{C}$ is algebraically closed, this gives us a ``safe space'' to work in, in the sense that if we take any subfield $F \subset \mathbb{C}$ and find a polynomial $f(x) \in F[x]$, we are \textit{guaranteed} to find a linear factorization of $f$ in $\mathbb{C}[x]$. Let's define this a bit more generally for arbitrary fields $F \subset K$. 

  \begin{definition}[Field Extension]
    The pair of fields $F \subset K$ is called a \textbf{field extension}. 
  \end{definition}

  Therefore, if $K$ is algebraically closed and $F \subset K$ is a field extension, $f(x) \in F[x]$ is guaranteed to \textit{split} completely into linear factors. This is true for \textit{all} $f(x) \in F[x]$, but now if we \textit{fix} $f(x) \in F[x]$, perhaps we don't need the entire field $K$ to split $f(x)$. Maybe we can work in a slightly larger field $E$---such that $F \subset E \subset K$---where $f(x)$ splits in $E$. This process of finding such a minimal field is important to understand the behavior of roots of such polynomials. 

  \begin{definition}[Splitting Field]
    Given a field extension $F \subset K$ and a polynomial $f \in F[x]$, 
    \begin{enumerate}
      \item $f$ \textbf{splits} in $K$ if $f$ can be written as the product of linear polynomials in $K[x]$. 
      \item If $f$ splits in $K$ and there exists no field $E$ s.t. $F \subsetneq E \subsetneq K$, then $K$ is called a \textbf{splitting field} of $f$.\footnote{i.e. the splitting field is the smallest field that splits $f$.} 
    \end{enumerate}
  \end{definition}

  \begin{example}[Don't Need(?) Complex]
    Consider the following. 
    \begin{enumerate}
      \item Let $f(x) = x^2 - 1$. If $f(x) \in \mathbb{R}[x]$, it does split in $\mathbb{R}$. In fact, even if we consider it as an element of $\mathbb{Z}_2 [x]$, it still splits into $(x + 1)(x - 1)$. 
      \item Let $f(x) = x^2 - 2$. If $f(x) \in \mathbb{Q}[x]$, it doesn't split in $\mathbb{Q}$ since the roots $\pm \sqrt{2} \not\in \mathbb{Q}$, but $\pm \sqrt{2}$ are real numbers, so $f(x)$ does in fact split in $\mathbb{R}$ since it splits into $(x + \sqrt{2}) (x - \sqrt{2})$. However, maybe it is not the (smallest) splitting field. 
      \item Let $f(x) = x^2 + 1$. We can see that if we consider it as an element of $\mathbb{Q}[x]$ or $\mathbb{R}[x]$, neither fields split $f(x)$ since $\pm i$ are its roots and therefore are contained in the coefficients of its linear factors. We know that it definitely splits in $\mathbb{C}$, but can we find a smaller field that splits $f(x)$? Perhaps.  
    \end{enumerate}
  \end{example}

  So how does one find a splitting field? Note that in the example above, we have found that there were some roots $\alpha$ of certain polynomials $f(x) \in F[x]$ are not contained in $F$. Therefore, what we want to do is find the smallest field $F$ containing both $F$ and $\alpha$ (plus any other $\alpha$'s). This smallest such field is called an \textit{adjoining field}. 

\subsection{Finite Fields}

\subsection{April 16} 

  Let $K \subset E \subset F$ be field extensions. We denote $[E:K]$ as the dimension of $E$ as a $K$-vector space. 

  \begin{lemma}[Embeddings of Fields]
    Suppose $K \subset U \subset L$ be two field extensions (also called a tower). Suppose that they are both finite, i.e. $[E:K]$ and $[L:E]$ are finite. Then $[L:K]$ is too and 
    \begin{equation}
      [L:K] = [L:E]\, [E:K]
    \end{equation}
  \end{lemma}
  \begin{proof} 
    Set up the following. 
    \begin{enumerate}
      \item Let $n = [E:K]$ Let $b_1, \ldots, b_n$ be a $K$-basis for $E$. 
      \item Let $m = [L:E]$ with $v_1, \ldots, v_m$ be an $L$-basis for $L$. 
    \end{enumerate}

    We claim that $B = \{ b_i v_j \mid 1 \leq i \leq n, 1 \leq j \leq m\}$ is a basis for $L$ over $K$. It suffices show that $B$ spans. Take $l \in L$. Since $\{v_1, \ldots v_m\}$ is an $E$-basis of $L$, $\exists  e_1, \ldots, e_m \in E$ s.t. $l = \sum e_i v_i$. Since $\{b_1, \ldots b_n\}$ is a $K$-basis. For $E$, there exists $a_{ij}\in K$, $j = 1, \ldots, n$ such that 
    \begin{equation}
      e_i = \sum_{j=1}^n a_{ij} b_j, \text{ for } i = 1, 2, \ldots, m
    \end{equation}
    which implies that 
    \begin{align}
      l & = \sum_{i=1}^m \bigg( \sum_{j=1}^n a_{ij} b_J \bigg) v_i \\ 
        & = \sum_{ij} a_{ij} b_j v_i
    \end{align} 
    where $a_{ij} \in K$, and so $B$ spans. Now we show that $B$ is linearly independent. We do this the usual way. Suppose that 
    \begin{equation}
      0 = \sum_{ij} a_{ij} b_j v_i
    \end{equation}
    Then we can group the sums as before. 
    \begin{equation}
      0 = \sum_{i=1}^n \bigg( \sum_{j=1}^n a_{ij} b_j \bigg) v_i 
    \end{equation}
    Note that since $a_{ij} \in K$ and $b_j \in E$, $\sum_{j=1}^n a_{ij} b_j \in E$. Since $\{v_1, \ldots, v_m\}$ is $E$-linearly independent, we have $\forall i = 1, \ldots, m$, 
    \begin{equation}
      \sum_{j=1}^n a_{ij} b_j = 0
    \end{equation}
    Since the $a_{ij} \in K$ and $\{b_1, \ldots, b_n\}$ are $K$-linearly independent, it follows that $a_{ij} = 0$ for all $i, j$. This show that $B$ is linearly independent. 
  \end{proof} 

  \begin{definition}[Field Embedding]
    An $F$-embedding is a ring homomorphism such that $\phi(a) = a$ for all $a \in F$. 
  \end{definition}

  \begin{definition}[Minimal Polynomial]
    
  \end{definition}

  We were interested in the Galois group of the symmetries of the field. Now we will extend this to bigger fields. 

  \begin{theorem}[Shifrin 7.6, 6.4]
    Let $F \subset K_1, F \subset K_2$ be field extensions. Then there are at most $[K_1:F]$ $F$-embeddings $\phi: K_1 \to K_2$. Moreoever, if there are exactly $[K_1:F]$ $F$-embeddings, then for all $F \subset E \subset K$, there are $[E:F]$ $F$-embeddings $\phi: E \to K_2$. 
  \end{theorem}
  \begin{proof}
    By induction on $[K_1:F]$. When $[K_1:F] = 1$. Now suppose the proposition holds for all $F, K_1, K_2$ with $[K_1:F] \leq n-1$. Now choose $F, K_1, K_2$, as in the statement with $[K_1 : F] = n > 1$. Since $[K_1: F] > 1$, we can choose $\alpha \in K_1$, $\alpha \not\in F$. Let $f(x) \in F[x]$ be the minimal polynomial of $\alpha$. We've seen that given the smallest subfield of $K_1$ containing $\alpha$, i.e. $F[\alpha]$, we can use the evaluation homomorphism to state 
    \begin{equation}
      F[\alpha] \simeq \frac{F[x]}{\langle f(x)\rangle}, \qquad F[\alpha] \xleftarrow{\mathrm{ev}_\alpha} \frac{F[x]}{\langle f(x)\rangle} 
    \end{equation}
    Any $F$-embedding of $K_1$ restricst to an $F$-embedding of $F[\alpha]$. By induction, there are at most $[F[\alpha]: F] = m$ $F$-embeddings 
    \begin{equation}
      \phi_1, \ldots, \phi_m : F[\alpha] \to K_2
    \end{equation}
    But by induction, we know that there are at most $[K_1: F[\alpha]]$ $F[\alpha]$-embeddings $K_1 \to K_2$ where $F[\alpha] \subset K_2$ (which we can defined through multiple ways for each injection $\phi_i$). This implies that there are at most 
    \begin{align}
      m \, [K_1: F[\alpha]] & = [F[\alpha]: F]\, [K_1: F[\alpha] \\
                            & = [K_1 : F]
    \end{align}
    $F$-embeddings. Suppose equality holds. Let $F \subset E \subset K_1$. We know 
    \begin{enumerate}
      \item There are at most $[E:F]$ $F$-embeddings $\phi_1, \ldots, \phi_{[E:F]}: E \to K_2$ 
      \item There are at most $[K_1:E]$ $E$-embeddings $K_1 \to K_2$ where here $E \subset K_2$. 
    \end{enumerate}
    This gives at most $[K_2:E] [E:F]$ $F$-embeddings since $[K_2 :E] [E:F] = [K_2 :F]$, and there are exactly $[K_2 : F]$ $F$-embeddings $K_1 \to K_2$, all the $\leq$'s must be $=$'s (otherwise it would be a strict inequality) in that there must be exactly $[E:F]$ $F$-embeddings $E \to K_2$ and exactly $[K_1: E]$ $E$-embeddings $K_1 \to K_2$. 

    Correction. We need to see that there are at most $[F[\alpha]:F]$ $F$-embeddings $F[\alpha] \to K_2$. It's okay to use induction if $[F[\alpha]: F] \subset [K_1:F]$. In general, we showed $F$-embeddings maps 
  \end{proof}

  \begin{definition}[Galois]
    A field extension $F \subset K$ is \textbf{Glaois} if the number of $F$-embeddings $K \to K$ is $[K:F]$. 
  \end{definition}

  \begin{example}
    Let $\alpha = \sqrt[7]{2}$. $\mathbb{Q} \subset \mathbb{Q}[\alpha]$ is not a Galois extension. $x^7 - 2$ is a polynomial with root $\alpha$, and by Eisenstein $x^7 - 2$ is irreducible. So $f(x) x^7 - 2$ is the minimal polynomial of $\alpha$. This means that the number of $\mathbb{Q}$-embeddings $\mathbb{Q}[\alpha] \to \mathbb{Q}[\alpha]$ is in bijection with the number of roots of $f(x)$ in $\mathbb{Q}[\alpha]$. All $7$ roots of $f(x)$ are $\sqrt[7]{2} e^{2\pi i j/7}$ for $j = 0, \ldots, 6$, which has one real root. So there is $1$ $\mathbb{Q}$-embedding $\mathbb{Q}[\alpha] \to \mathbb{Q}[\alpha]$. But $[\mathbb{Q}[\sqrt[7]{2}] : \mathbb{Q}] = \deg{f(x)} = 7$. Since $\mathbb{Q}[\alpha] \simeq \mathbb{Q}[x]/{\langle x^7 - 2 \rangle}$, which are polynomials of degree at most $6$. 
  \end{example}

  \begin{theorem}[Shifrin 6.8]
    Suppose $F \subset K$ is the splitting field of a polynomial $f(x) \in F[x]$ such that no irreducible factor of $f(x)$ has repeated roots. Then $F \subset K$ is Galois. 
  \end{theorem}
  \begin{proof}
    Since $K$ is the splitting field of $f(x)$ over $F$, we have $F[r_1, \ldots, r_m]$ where the $r_i$ are the distinct roots of $f(x)$. We show by induction that there are $[F[r_1, \ldots, r_j]:F]$ $F$-embeddings of $F[r_1, \ldots, r_j] \to K$. For $j = 1$, $r_1$ is a root of some irreducible factor $f_1 (x)$ of $f(x)$. 
    \begin{equation}
      F[r_1] \simeq \frac{F[x]}{\langle f_1 (x) \rangle}
    \end{equation} 
    and the set of $F$-embeddings $F[r_1] \to K$ is in bijection with the set of roots $\alpha \in K$ of $f_1 (x)$. By hypothesis , $f(x)$ has no repeated roots, which implies that the number of $F$-embeddings $F[r_1] \to K$ is $\deg{f_1 (x)} = [F[r_1]: F]$ which gives the base case. For the inductive step, we know there are 
    \begin{equation}
      [F[r_1, \ldots, r_{j-1}]: F] 
    \end{equation}
    $F$-embeddings $F[r_1, \ldots, r_{j-1}] \xrightarrow{\phi} K$. For each, we will show that there are exactly $[F[r_1, \ldots, r_j]: F[r_1, \ldots, r_{j-1}]]$ extensions of $\phi$ which completes the proof. Because 
    \begin{align}
      F[r_1, \ldots, r_j]: F] & = F[r_1, \ldots, r_j]: F[r_1, \ldots, r_{j-1}]] F[r_1, \ldots, r_{j-1}]: F]
    \end{align} 
    Let $g(x)$ be the minimal polynomial of $r_j$ over $F[r_1, \ldots, r_{j-1}]$. Since $g(r_j) = 0$, $g$ divides one of the irreducible factors of $f(x)$ in $F[r_1, \ldots, r_{j-1}] [x]$ which implies it has no repeated roots. Then 
    \begin{equation}
      F[r_1, \ldots, r_j] = \frac{E[x]}{\langle g(x)\rangle} 
    \end{equation}
    The number of $E$-embeddings is equal to the number of roots in in $g$ which is $\deg{g} = F[r_1, \ldots, r_j: F]$.
  \end{proof}

\subsection{Cubic Equations}

  The well known discriminant of a quadratic equation 
  \begin{equation}
    f(x) = ax^2 + bx + c
  \end{equation}
  is known in the form $\nabla = b^2 - 4ac$. However, we will present it in a slightly different manner. 

  \begin{definition}
    The \textbf{discriminant} $D(\varphi)$ of a quadratic polynomial
    \begin{equation}
      \varphi = a_0 x^2 + a_1 x + a_2 \in \mathbb{C}[x]
    \end{equation}
    with $c_1, c_2 \in \mathbb{C}$ as its roots is defined
    \begin{equation}
      D(\varphi) = a_1^2 - 4 a_0 a_2 = a_0^2 \bigg( \Big(\frac{a_1}{a_0} \Big)^2 - \frac{4 a_2}{a_0} \bigg) = a_0^2 \big( (c_1 + c_2)^2 - 4 c_1 c_2 \big) = a_0^2 (c_1 - c_2)^2
    \end{equation}
    Clearly, the value of $D(\varphi)$ can tell us three things
    \begin{enumerate}
      \item $c_1, c_2 \in \mathbb{R}, c_1 \neq c_2$. Then $c_1 - c_2$ is a nonzero real number and $D(\varphi) > 0$. 
      \item $c_1 = c_2 \in \mathbb{R}$. Then $c_1 - c_2 = 0$ and $D(\varphi) = 0$. 
      \item $c_1, c_2 \in \mathbb{C}, c_1 = \bar{c}_2$. Then, $c_1 - c_2$ is a nonzero strictly imaginary number and $D(\varphi) < 0$. 
    \end{enumerate}
  \end{definition}

  \begin{definition}
    We can generalize this notion of the discriminant to arbitrary polynomials
    \begin{equation}
      \varphi = a_0 x^n + a_1 x^{n-1} + ... + a_{n-1} x + a_n \in \mathbb{F}[x], \; a_0 \neq 0
    \end{equation}
    The discriminant $D(\varphi)$ of the polynomial above is defined
    \begin{equation}
      D(\varphi) \equiv a_0^{2n-2} \prod_{i>j} (c_i - c_j)^2
    \end{equation}
    The $a_0$ term isn't very important in this formula, since it does not affect whether $D(\varphi)$ is positive, negative, or zero. 
  \end{definition}

  \begin{definition}
    A polynomial 
    \begin{equation}
      \varphi = a_0 x^n + a_1 x^{n-1} + ... + a_{n-1} x + a_n \in \mathbb{F}[x], \; a_0 \neq 0
    \end{equation}
    where $a_1 = 0$ is called \textbf{depressed}. A depressed cubic polynomial is of form
    \begin{equation}
      \varphi = x^3 + p x + q
    \end{equation}
  \end{definition}

  \begin{proposition}
    Every monic (leading coefficeint $=1$) polynomial (and non-monic ones) 
    \begin{equation}
      \varphi = x^n + a_1 x^{n-1} + ... + a_{n-1} x + a_n \in \mathbb{F}[x], \; a_0 \neq 0
    \end{equation}
    can be turned into a depressed polynomial with the change of variable
    \begin{equation}
      x = y - \frac{a_1}{n}
    \end{equation}
    to get the polynomial 
    \begin{equation}
      \psi = y^n + b_2 y^{n-2} + ... + b_{n-1} y + b_n
    \end{equation}
  \end{proposition}

  \begin{lemma}
    A cubic polynomial 
    \begin{equation}
      \varphi = a_0 x^3 + a_1 x^2 + a_2 x + a_3 \in \mathbb{R}[x]
    \end{equation}
    with roots $c_1, c_2, c_3 \in \mathbb{C}$ has discriminant
    \begin{equation}
      D(\varphi) \equiv a_0^4 (c_1 - c_2)^2 (c_1 - c_3)^2 (c_2 - c_3)^2
    \end{equation}
    With a bit of evaluation, it can also be expressed in terms of its coefficients as
    \begin{equation}
      D(\varphi) = a_1^2 a_2^2 - 4a_1^3 a_3 - 4a_0 a_2^3 + 18 a_0 a_1 a_2 a_3 - 27 a_0^2 a_3^2
    \end{equation}
    Again, three possibilities can occur (up to reordering of its roots). 
    \begin{enumerate}
        \item $c_1, c_2, c_3$ are distinct real numbers. Then $D(\varphi) > 0$. 
        \item $c_1, c_2, c_3 \in \mathbb{R}, c_1 = c_2$. Then $D(\varphi) = 0$. 
        \item $c_1 \in \mathbb{R}, c_2 = \bar{c}_3 \not\in \mathbb{R}$. Then $D(\varphi) < 0$. 
    \end{enumerate}
    Furthermore, the cubic formula used to find the roots of the polynomial is 
    \begin{equation}
      c_{1, 2, 3} = \sqrt[3]{-\frac{q}{2} + \sqrt{\frac{p^3}{27} + \frac{q^2}{4}}} + \sqrt[3]{-\frac{q}{2} - \sqrt{\frac{p^3}{27} + \frac{q^2}{4}}}
    \end{equation}
    known as \textbf{Cardano's formula}, after the mathematician Gerolamo Cardano. 
  \end{lemma}





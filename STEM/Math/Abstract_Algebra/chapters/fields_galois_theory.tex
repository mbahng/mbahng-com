\section{Fields and Galois Theory} 

  Now that we have established the theory of general rings and for polynomials, we will delve deeper into the theory of fields by talking about \textit{field extensions}, which allows us to consider minimal fields in which polynomials can be decomposed into linear factors. This is called a \textit{splitting field} and also turns out to have a vector space structure as well. This is similar to how we have constructed ring extensions, and we would like to find conditions in which an adjoining ring is a field. 

\subsection{Field Extensions and Vector Spaces}

  This method in which we have taken higher powers of $\alpha$ to reveal elements in $\mathbb{Q}$ reveals a deeper structure of a finite-dimensional vector space, which will be useful for analyzing certain fields in the examples below. Note that a vector space is only well-defined over a field $F$, so we will consider \textit{field extensions} from now on. First, recall that a field is trivially a vector space. 

  \begin{lemma}[Fields are a Vector Space]
    A field $F$ is a 1-dimensional vector space over itself. 
  \end{lemma} 

  It turns out that we can generalize this a bit more. 

  \begin{theorem}[Fields are a Vector Space over Subfields]
    \label{thm:fields_vector_space}
    Let $F$ be a subfield of $K$. Then $K$ is a $F$-vector space. 
  \end{theorem}
  \begin{proof}
    A $F$-vector space has $0$, addition, and multiplication by $F$. $K$ indeed has $0$, addition, and we can multiply any element of $K$ by an element of $F$. The extra axioms follow but are too verbose to write a full proof. 
  \end{proof}

  \begin{corollary} 
    $\mathbb{R}$ is an infinite-dimensional vector space over $\mathbb{Q}$. 
  \end{corollary}
  \begin{proof}
    The fact that it is a vector space immediately follows from $\mathbb{Q} \hookrightarrow \mathbb{R}$. For dimensionality, the outline is to show that $\{\sqrt{p} \mid p \text{ prime }\}$ are linearly independent. This takes work to prove and won't do it. 
  \end{proof}

  Therefore, by constructing a subfield, we can model the original field as a vector space. This additional structure warrants a name. 

  \begin{definition}[Field Extension]
    If $F, K$ are fields, then this is called a \textbf{field extension}. Its \textbf{degree} is the $F$-dimension of $K$, denoted
    \begin{equation}
      [K:F] \coloneqq \dim_F (K) 
    \end{equation}
  \end{definition} 

  So when we are given a field, we can automatically treat it as a vector space. Furthermore, if we are given a field extension, we can treat the larger field as a vector space over the smaller field (though it may be finite or infinite-dimensional). As we create concatenated field extensions, sometimes called a \textit{tower}, the dimensions behave nicely as well. 

  \begin{theorem}[]
    $F \hookrightarrow E$ and $F \hookrightarrow K$ are field extensions. Then $E \hookrightarrow K$ is a field extension with degree 
    \begin{equation}
      [K:E] = [K:E] [E:F]
    \end{equation}
  \end{theorem}
  \begin{proof}
    Let $\alpha_1, \ldots \alpha_m$ be a basis for $E$ over $F$ and $\beta_1, \ldots, \beta_n$ be a basis for $K$ over $E$. We claim that $\{\alpha_i \beta_j\}$ is a basis for $K$ over $F$, with multiplication done in the field $K$. We check linear indepdence. Let $\beta \in K$ be arbitrary. Then by the $E$-basis, we have 
    \begin{equation}
      \beta = \sum_{j=1}^n x_j \beta_j
    \end{equation} 
    But since $x_j \in E$, there are elements 
    \begin{equation}
      x_j = \sum_{i=1}^m y_{ij} \alpha_i
    \end{equation}
    and so combining we get 
    \begin{equation}
      \beta = \sum_{j=1}^n \sum_{i=1}^m y_{ij} (\alpha_i \beta_j) 
    \end{equation}
    To prove linear independence, suppose $\beta = 0$. Then we have 
    \begin{equation}
      0 = \sum_{j=1}^n \sum_{i=1}^m y_{ij} (\alpha_i \beta_j) = \sum_{j=1}^n \bigg( \sum_{i=1}^m y_{ij} \alpha_i \bigg) \beta_j 
    \end{equation}
    Since $\beta_1, \ldots, \beta_n$ are linearly indpendent, we must have $\sum_{i=1}^m y_{ij} \alpha_i = 0$ for all $j$. But since $\alpha_i$'s are linear independent, this means $y_{ij} = 0$ for all $i, j$. 
  \end{proof} 

  In fact, there is a nice classification algorithm that connects fields to vector spaces. 

  \begin{theorem}
    
  \end{theorem}

  Once we know that field extensions are vector spaces, what consistutes linear maps? A first guess would be a ring homomorphism, but this may not be true. 

  \begin{example}[Ring Homomorphisms are Not Always Linear Maps]
    
  \end{example}

  Therefore, we need some additional constraint. 

  \begin{theorem}[Ring Homomorphisms as Linear Maps]
    Let $F \subset K_1$ and $F \subset K_2$ be field extensions and $f: K_1 \to K_2$ be a ring-homomorphism. If $f(x) = x$ for all $x \in F$, then $f$ is a linear map of $F$-vector spaces. 
  \end{theorem}

\subsection{Adjoining Fields and Quotient Rings} 

  We have examined the properties of field extensions in general. Now we consider the specific extension $F \subset F[\alpha]$. What we should think is that $F[\alpha]$ ends up becoming both a ring and a vector space, but not yet a field. It is only when $\alpha$ is the root of an irreducible polynomial in $F(x)$ that $F[\alpha]$ is a field.  

  \begin{lemma}[Vector Space Structure]
    If $F$ is a field, 
    \begin{enumerate}
      \item $F[\alpha]$ is a finite-dimensional vector space over $F$. 
      \item If $f(x) = a_n x^n + \ldots + a_0$ is any polynomial with root $\alpha$, then $\{1, \alpha, \ldots, \alpha^{n-1}\}$ spans $F[\alpha]$.\footnote{Note that this does not mean that it is a basis.} 
    \end{enumerate}
  \end{lemma}
  \begin{proof}
    An element of $F[\alpha]$ is of the form 
    \begin{equation}
      f(\alpha) = \sum_{k=0}^n a_k \alpha^k
    \end{equation} 
    for some $f \in F[x]$, and so it is immediate that $\{\alpha^k\}_{k \in \mathbb{N}_0}$ spans $F[\alpha]$. We claim that $\alpha^{n-1+i}$ is in $S$ for all $i > 0$. By induction, if $i = 1$, then 
    \begin{equation}
      \alpha^n = -\frac{1}{a_n} \big( a_{n-1} \alpha^{n-1} + \ldots + a_0 \big)
    \end{equation}
    which proves the claim. Now assume that $\alpha^n, \alpha^{n+1}, \ldots, \alpha^{n-1+i} \in \Span\{1, \ldots, \alpha^{n-1}\}$. Then 
    \begin{equation}
      \alpha^i f(\alpha) = 0 \implies a_n \alpha^{n+i} + \alpha_{n-1} \alpha^{n+i-1} + \ldots + a_0 \alpha^i = 0 
    \end{equation}
    and so 
    \begin{equation}
      \alpha^{n+i} = -\frac{1}{a_n} \big(a_{n-1} \alpha^{n+i-1} + \ldots + a_0 \alpha^i)
    \end{equation}
    which means that $\alpha^{n+i} \in \Span\{1, \ldots, \alpha^{n-1}\}$, completing the proof. 
  \end{proof} 

  Great, so we know that $F[\alpha]$ is a finite-dimensional vector space. What additional constrains do we need for it to be a field? With one more assumption, we can claim that it is a field, giving a subfield substructure $F \subset F[\alpha] \subset K$

  \begin{theorem}[Adjoining Fields]
    Let $F \hookrightarrow K$ be a field extension. 
    \begin{enumerate}
      \item If there exists a $f(x) \in F[x]$ s.t. $\alpha \in K$ is a root of $f$, then $F[\alpha] \subset K$ is a field. To emphasize that it is a field, we usually denote it as $F(\alpha)$ and refer it as the \textbf{adjoining field}. 
      \item If there exists a $f(x) \in F[x]$ is irreducible of degree $n$, then $[F[\alpha]:F] = n$.
    \end{enumerate}
  \end{theorem}
  \begin{proof}
    It is clear that $F[\alpha]$ is a commutative ring since $F$ is a field. So it remains to show that every nonzero element of $\beta \in F[\alpha]$ is a unit. By definition $\beta = p(\alpha)$ for some polynomial $p \in F[x]$.  Factor $f \in F[x]$ as the product of irreducible polynomials. Then $\alpha$ must be a root of one of those irreducible factors, say $g(x)$. Note that $g(x) \nmid p(x)$ since $p(\alpha) \neq 0$. Since $g$ is irreducible, we know that $\gcd(g, p) = 1$ and so $\exists s, t \in F[x]$ s.t. 
    \begin{equation}
      1 = s p + t g \implies 1 = s(\alpha) p(\alpha) + t(\alpha) g(\alpha) = s(\alpha) p(\alpha)
    \end{equation}  
    Therefore we have found a multiplicative inverse $s = p^{-1} \in F[\alpha]$. 
  \end{proof} 
  \begin{proof}
    We can also prove it using the vector space structure, which requires the lemma below. Treating $F[\alpha]$as a finite-dimensional vector space over $F$, let us define the $F$-linear function\footnote{linearity is easy to check}
    \begin{equation}
      m_b: F[\alpha] \rightarrow F[\alpha], \qquad m_b (\beta) = b\beta
    \end{equation} 
    Since $F[\alpha] \subset K$, $F[\alpha]$ is an integral domain. Thus $\not\exists \beta \in F[\alpha] \setminus \{0\}$ s.t. $b \beta = 0$. This means that the kernel of $m_b$ is $0$, and so $m_b$ is injective. By the rank-nullity theorem, it is bijective, and so there exists a $\beta \in F[\alpha]$ s.t. $b \beta = 1 \implies b$ is a unit. 
  \end{proof} 

  Intuitively, the extra $\alpha \in K$ allows us to ``expand'' our field $F$ into a bigger field of $K$. Since $F[\alpha]$ is the smallest ring containing both $F$ and $\alpha$, it immediately follow that it is the smallest \textit{field} containing $F$ and $\alpha$. If we examine subfields of $\mathbb{C}$, this is equivalent to $\alpha$ being an \textit{algebraic number} (i.e. $\alpha$ must be a root of some polynomial with rational---or equivalently, integer---coefficients). 

  \begin{corollary}[Adjoining with Algebraic Numbers Creates Fields]
    Let $\alpha \in \mathbb{C}$. Then $\mathbb{Q}[\alpha] \subset \mathbb{C}$ is a field if and only if $\alpha$ is an algebraic number. 
  \end{corollary} 

  \begin{corollary} 
    Suppose $[K:F] = n$ and $\alpha \in K$ is the root of an irreducible polynomial $f(x) \in F[x]$. Then $\deg{f(x)} \mid n$. 
  \end{corollary}

  \begin{example}[$\mathbb{Q}\lbrack \sqrt{3} i\rbrack$ is a Field]
    $\mathbb{Q}[\sqrt{3} i]$ is a field, hence denoted $\mathbb{Q}(\sqrt{3} i)$ since $\sqrt{3}i$ is a root of the polynomial $f(x) = x^2 + 3$. 
  \end{example}

  \begin{example}[$\mathbb{Q}\lbrack \pi \rbrack$ not a Field]
    However, $\mathbb{Q}[\pi]$ is not a field. 
  \end{example} 

  \begin{example}[Finding Multiplicative Inverses of elements in $\mathbb{Q}\lbrack \alpha \rbrack$]
    Given $\beta = p(\alpha) = \alpha^2 + \alpha - 1 \in \mathbb{Q}[\alpha]$, where $\alpha$ is a root of $f(\alpha) = \alpha^3 + \alpha + 1$, we first know that $\beta$ must have a multiplicative inverse since $\mathbb{Q}[\alpha]$ is a field. Applying the Euclidean algorithm, we have 
    \begin{equation}
      1 = \frac{1}{3} \big\{ (x+1) f(x) - (x^2 + 2) p(x)\big\} = -\frac{1}{3} (\alpha^2 + 2) p(\alpha)
    \end{equation}
    and so $\beta^{-1} = (\alpha^2 + \alpha - 1)^{-1} = -\frac{1}{3} (\alpha^2 + 2)$. We can check that 
    \begin{align}
      -\frac{1}{3} (\alpha^2 + 2) (\alpha^2 + \alpha - 1) & = -\frac{1}{3} (\alpha^4 + \alpha^3 + \alpha^2 + 2 \alpha - 2) \\
                                                          & = -\frac{1}{3} (\alpha^3 + \alpha - 2) \\
                                                          & = -\frac{1}{3} (-3) = 1
    \end{align}
  \end{example}

  Great, so we have seen examples of both ring and field extensions and seen a condition that makes $F[\alpha]$ a field as well. Now recall quotient rings, which do not necessarily preserve the properties of the original ring. That is, if $F$ is a field, then $F/I$ may not be a field. Using the fundamental ring homomorphism theorem, we can precisely correlate certain quotient maps with adjoining fields. 

  \begin{lemma}[Evaluation Homomorphism of Polynomials]
    Given fields $F \subset K$, the \textbf{evaluation function} is a homomorphism. 
    \begin{equation}
      \ev_\alpha: F[x] \rightarrow K, \qquad f(x) \mapsto f(\alpha)
    \end{equation}
  \end{lemma}
  \begin{proof}
    
  \end{proof} 

  \begin{theorem}[Quotient Polynomial Ring Can be Splitting Field]
    Let $F$ be a field and $f(x) \in F[x]$, then 
    \begin{equation}
      f(x) \text{ is irreducible in } F[x] \iff K = \frac{F[x]}{\langle f(x) \rangle} \text{ is a field}
    \end{equation} 
    and if either condition is satisfied, then $K$ must contain  a root $\alpha$ of $f(x)$, and 
    \begin{equation}
      \frac{F[x]}{\langle f(x) \rangle} \simeq F[\alpha]
    \end{equation}
    making $F \subset K$ a field extension. It immediately follows then as vector spaces, 
    \begin{equation}
      \dim_F \frac{F[x]}{\langle f(x) \rangle} = \deg(f)
    \end{equation}
  \end{theorem}
  \begin{proof} 
    Prove that a root $\alpha$ of $f(x)$ is contained in $K$. 

    If $K$ is a field, then $F \subset K$ is a field extension. Then let $\alpha \in K$ be a root of $f(x)$. Then $F[\alpha]$ is a field.  

    We can see that $F \subset F[x]$. Therefore if $f(x)$ is irreducible, then $\alpha \in K$
    By fundamental homomorphism theorem. 

    For dimension, we know that $\{1, \ldots, x^{n-1}\}$ is a basis. 
  \end{proof}
  
  \begin{example}[Quotient Rings as Fields]
    Since $x^2 + 1$ is irreducible in $\mathbb{Z}_7 [x]$ and $\mathbb{R}$, the following are fields. 
    \begin{equation}
      \frac{\mathbb{Z}_7 [x]}{\langle x^2 + 1 \rangle}, \quad \frac{\mathbb{R}[x]}{\langle x^2 + 1 \rangle}
    \end{equation}
  \end{example}

  With this theorem, we can use it cleverly to prove that two rings are isomorphic to each other. 

  \begin{example}
    The evaluation map 
    \begin{equation}
      \phi: \frac{\mathbb{R}[x]}{\langle x^2 + 1 \rangle} \rightarrow \mathbb{C}, \qquad \phi\big( f(x) \pmod{\langle x^2 + 1 \rangle} \big) = f(i)
    \end{equation}
    is an isomorphism.\footnote{Intuitively, we can see that the quotient ring can only consist up to linear polynomials since $x^2 \equiv -1$. This is a real vector space of dimension $2$, and so is $\mathbb{C}$, so it makes sense that they may be isomorphic. } This is because we can think of the evaluation homomorphism $\ev_i : f(x) \in \mathbb{R}[x] \mapsto f(i) \in \mathbb{R}[i]$. We know that $\mathbb{R}$ a field implies $\mathbb{R}[x]$ is a PID. Now take $\ker(\ev_i)$. We can see that it contains the polynomial $x^2 + 1$, and since it is irreducible in $\mathbb{R}[x]$, it must be the case that $\ker(\ev_i) = \langle x^2 + 1 \rangle$. Now it follows by the fundamental ring homomorphism theorem that 
    \begin{equation}
      \frac{\mathbb{R}[x]}{\ker(\ev_i)} = \frac{\mathbb{R}[x]}{\langle x^2 + 1 \rangle} \simeq \mathbb{R}[i] = \mathbb{C}
    \end{equation}
  \end{example} 

  \begin{example}
    The evaluation map 
    \begin{equation}
      \ev_{\sqrt{2}}: \mathbb{Q}[x] \mapsto \mathbb{Q}[\sqrt{2}], \qquad \ev_{\sqrt{2}} (f) = f(\sqrt{2}) 
    \end{equation}
    is a homomorphism. Furthermore, it has a kernel $\langle x^2 - 2 \rangle$ since $(x^2 - 2)$ is an irreducible polynomial in $\mathbb{Q}[x]$ containing the root $\sqrt{2}$. Therefore by the fundamental ring homomorphism theorem we have 
    \begin{equation}
      \frac{\mathbb{Q}[x]}{\langle x^2 - 2 \rangle} \simeq \mathbb{Q}[\sqrt{2}]
    \end{equation}
  \end{example} 

  \begin{example}[Extensions of $\sqrt{2}$ and $i$]
    We claim that 
    \begin{equation}
      \mathbb{Q}[\sqrt{2}, i] = \{ a + b \sqrt{2} + ci + d(\sqrt{2} i) \mid a, b, c, d \in \mathbb{Q}\}
    \end{equation}
    From the previous example, we know that $\mathbb{Q}[\sqrt{2}]$ are all numbers of the form $a + b\sqrt{2}$. Now we take $i \in \mathbb{C}$ and map it through all polynomials with coefficients in $\mathbb{Z}[\sqrt{2}]$, which will be of form 
    \begin{equation}
      f(i) = (a_n + b_n \sqrt{2}) i^n + (a_{n-1} + b_{n-1}\sqrt{2}) i^{n-1} + \ldots + (a_2 + b_2 \sqrt{2}) i^2 + (a_1 + b_1 \sqrt{2}) i + (a_0 + b_0 \sqrt{2})
    \end{equation} 
    However, we can see that since $i^2 = -1$, we only need to consider up to degree 1 polynomials of form 
    \begin{equation}
      (a + b \sqrt{2}) + (c + d \sqrt{2}) i 
    \end{equation}
    which is clearly of the desired form. For the other way around, this is trivial since we can construct a linear polynomial as before. 
  \end{example} 

  \begin{example}
    We claim $\mathbb{Q}[\sqrt{3} + i] = \mathbb{Q}[\sqrt{3}, i]$. 
    \begin{enumerate}
      \item $\mathbb{Q}[\sqrt{3} + i] \subset \mathbb{Q}[\sqrt{3}, i]$
      \item $\mathbb{Q}[\sqrt{3} + i] \supset \mathbb{Q}[\sqrt{3}, i]$. Note that 
        \begin{align}
          (\sqrt{3} + i)^3 = 8i & \implies i \in \mathbb{Q}[\sqrt{3} + i] \\
                                & \implies (\sqrt{3} + i) - i = \sqrt{3} \in \mathbb{Q}[\sqrt{3} + i] 
        \end{align}
        Therefore, $\mathbb{Q}[\sqrt{3} + i]$ contains the elements $1, \sqrt{3}, i$, which form the basis of $\mathbb{Q}[\sqrt{3}, i]$. 
    \end{enumerate}
  \end{example}

  \begin{example}[Extensions of $\sqrt{3}i$ and $\sqrt{3}, i$]
    We claim that $\mathbb{Q}[\sqrt{3} i] \subsetneq \mathbb{Q}[\sqrt{3}, i]$. 
    \begin{enumerate}
      \item We can see that $\{1, \sqrt{3}i \}$ span $\mathbb{Q}[\sqrt{3}i ]$ as a $\mathbb{Q}$-vector space. Therefore, 
      \begin{equation}
        \sqrt{3}, i \in \mathbb{Q}[\sqrt{3}, i] \implies \sqrt{3} i \in \mathbb{Q}[\sqrt{3}, i]
      \end{equation} 
      implies that $\mathbb{Q}[\sqrt{3} i] \subset \mathbb{Q}[\sqrt{3}, i]$. 

      \item To prove proper inclusion, we claim that $i \not\in \mathbb{Q}[\sqrt{3}i]$. Assuming that it can, we represent it in the basis $i = b_0 + b_1 \sqrt{3} i$, and so
      \begin{equation}
        -1 = (b_0 + b_1 \sqrt{3} i)^2 = (b_0^2 - 3b_1^2) + 2b_0 b_1 \sqrt{3} i
      \end{equation}
      Therefore we must have $2b_0 b_1 \sqrt{3} = 0 \implies b_0$ or $b_1$ should be $0$. If $b_0 = 0$, then $b_0^2 - 3b_1^2 = -3 b_1^2 \implies b_1^2 = 1/3$, which is not possible since $b_1^2 \in \mathbb{Q}$. If $b_1 = 0$, then $b_0 - 3 b_1^2 = b_0^2 > 0$, and so it cannot be $-1$. 
    \end{enumerate}
  \end{example}

  \begin{theorem}
    Given field $F = \mathbb{R}[x]/\langle x^2 + 1 \rangle$\footnote{This is a field since $x^2 + 1$ is irreducible in $\mathbb{R}[x]$. }
    \begin{enumerate}
      \item $F$ is a finite dimensional vector space over $\mathbb{R}$. 
      \item $F$ is an infinite dimensional vector space over $\mathbb{Q}$. 
    \end{enumerate}
  \end{theorem}
  \begin{proof} 
    Since $F \simeq \mathbb{R}[i] \supset \mathbb{R} \supset \mathbb{Q}$, $F$ is a vector space over its subfields $\mathbb{R}$ and $\mathbb{Q}$. It suffices to prove dimensionality. 
    \begin{enumerate}
      \item For (1), we claim that $F \simeq \mathbb{R}^2$. 
      \item For (2), we note that $F \supset \mathbb{R}$ and $\mathbb{R}$ is infinite dimensional over $\mathbb{Q}$, so it follows for $F$. 
    \end{enumerate}
  \end{proof}

\subsection{Splitting Fields} 

  Remember that by previously establishing that $\mathbb{C}$ is algebraically closed, this gives us a ``safe space'' to work in, in the sense that if we take any subfield $F \subset \mathbb{C}$ and find a polynomial $f(x) \in F[x]$, we are \textit{guaranteed} to find a linear factorization of $f$ in $\mathbb{C}[x]$. 

  Therefore, if $K$ is algebraically closed and $F \subset K$ is a field extension, $f(x) \in F[x]$ is guaranteed to \textit{split} completely into linear factors. This is true for \textit{all} $f(x) \in F[x]$, but now if we \textit{fix} $f(x) \in F[x]$, perhaps we don't need the entire field $K$ to split $f(x)$. Maybe we can work in a slightly larger field $E$---such that $F \subset E \subset K$---where $f(x)$ splits in $E$. This process of finding such a minimal field is important to understand the behavior of roots of such polynomials. 

  \begin{definition}[Splitting Field]
    Given a field extension $F \subset K$ and a polynomial $f \in F[x]$, 
    \begin{enumerate}
      \item $f$ \textbf{splits} in $K$ if $f$ can be written as the product of linear polynomials in $K[x]$. 
      \item If $f$ splits in $K$ and there exists no field $E$ s.t. $F \subsetneq E \subsetneq K$, then $K$ is called a \textbf{splitting field} of $f$.\footnote{i.e. the splitting field is the smallest field that splits $f$.} 
    \end{enumerate}
  \end{definition}

  \begin{example}[Don't Need Necessarily Complex Numbers to Split]
    Consider the following. 
    \begin{enumerate}
      \item Let $f(x) = x^2 - 1$. If $f(x) \in \mathbb{R}[x]$, it does split in $\mathbb{R}$. In fact, even if we consider it as an element of $\mathbb{Z}_2 [x]$, it still splits into $(x + 1)(x - 1)$. 
      \item Let $f(x) = x^2 - 2$. If $f(x) \in \mathbb{Q}[x]$, it doesn't split in $\mathbb{Q}$ since the roots $\pm \sqrt{2} \not\in \mathbb{Q}$, but $\pm \sqrt{2}$ are real numbers, so $f(x)$ does in fact split in $\mathbb{R}$ since it splits into $(x + \sqrt{2}) (x - \sqrt{2})$. However, maybe it is not the (smallest) splitting field. 
      \item Let $f(x) = x^2 + 1$. We can see that if we consider it as an element of $\mathbb{Q}[x]$ or $\mathbb{R}[x]$, neither fields split $f(x)$ since $\pm i$ are its roots and therefore are contained in the coefficients of its linear factors. We know that it definitely splits in $\mathbb{C}$, but can we find a smaller field that splits $f(x)$? Perhaps.  
    \end{enumerate}
  \end{example}

  So how does one find a splitting field? Note that in the example above, we have found that there were some roots $\alpha$ of certain polynomials $f(x) \in F[x]$ are not contained in $F$. Therefore, what we want to do is find the smallest field $F$ containing both $F$ and $\alpha$ (plus any other $\alpha$'s). This is precisely the adjoining field $F[\alpha]$, which guarantees unique factorization since $F[\alpha]$ is a Euclidean domain. 

  \begin{corollary}
    Any polynomial $f(x) \in F[x]$ has a unique splitting field. 
  \end{corollary}

  \begin{example}[Simple Splitting Fields]
    We provide some simple examples to gain intuition. 
    \begin{enumerate}
      \item Let $f(x) = x^2 + 2x + 2 \in \mathbb{Q}[x]$. Then the roots of $f(x)$ are $-1 \pm i$, so 
      \begin{equation}
        f(x) = (x - (-1 + i)) (x - (-1 - i)) 
      \end{equation}
      and we can show that $\mathbb{Q}[-1 - i, -1+i] = \mathbb{Q}[i]$ is the splitting field of $f$. 

      \item Let $f(x) = x^2 - 2x - 1 \in \mathbb{Q}[x]$. The roots are $1 \pm \sqrt{2}$, and so 
      \begin{equation}
        f(x) = (x - (1 + \sqrt{2})) (x - (1 - \sqrt{2}))
      \end{equation}
      and so $\mathbb{Q}[\sqrt{2}]$ is the splitting field of $f$. 

      \item Let $f(x) = x^6 - 1 \in \mathbb{Q}[x]$. We can factor 
        \begin{equation}
          f(x) = (x-1) (x + 1) (x^2 + x + 1) (x^2 - x + 1)
        \end{equation} 
        and the non-rational roots are $\frac{\pm 1 \pm \sqrt{3} i}{2}$. Thus the splitting field of $f$ is $\mathbb{Q}[\sqrt{3} i]$. 
    \end{enumerate}
  \end{example}

  \begin{example}
    Let $f(x) = x^4 - 2 \in \mathbb{Q}[x]$. It follows that the roots are 
    \begin{equation}
      \{ \sqrt[4]{2}, \sqrt[4]{2}, -\sqrt[4]{2}, - \sqrt[4]{2} i \} = \Big\{ \sqrt[4]{2}, \sqrt[4]{2} e^{\frac{2\pi i}{4}}, \sqrt[4]{2} e^{\frac{4\pi i}{4}}, \sqrt[4]{2} e^{\frac{6\pi i}{4}} \Big\}
    \end{equation}
    thus the splitting field of $f$ is 
    \begin{equation}
      \mathbb{Q} \big( \sqrt[4]{2}, \sqrt[4]{2} e^{\frac{2\pi i}{4}}, \sqrt[4]{2} e^{\frac{4\pi i}{4}}, \sqrt[4]{2} e^{\frac{6\pi i}{4}} \big) \subset \mathbb{Q}(\sqrt[4]{2}, e^{\frac{2\pi i}{4}})
    \end{equation}
    since $\sqrt[4]{2} e^{\frac{m \pi i}{4}} \in \mathbb{Q}(\sqrt[4]{2}, e^{\frac{2\pi i}{4}})$. In fact, the two are equal, and to prove this we can see that since we are working in a field, 
    \begin{equation}
      e^{2 \pi i / 4} = \frac{\sqrt[4]{2} e^{2\pi i/4}}{\sqrt[4]{2}} \in \mathbb{Q} \big( \sqrt[4]{2}, \sqrt[4]{2} e^{\frac{2\pi i}{4}}, \sqrt[4]{2} e^{\frac{4\pi i}{4}}, \sqrt[4]{2} e^{\frac{6\pi i}{4}} \big) 
    \end{equation}
    which implies that $\sqrt[4]{2} \in \mathbb{Q} \big( \sqrt[4]{2}, \sqrt[4]{2} e^{\frac{2\pi i}{4}}, \sqrt[4]{2} e^{\frac{4\pi i}{4}}, \sqrt[4]{2} e^{\frac{6\pi i}{4}} \big)$. Therefore we can conclude that the splitting field is 
    \begin{equation}
      \mathbb{Q} \big( \sqrt[4]{2}, \sqrt[4]{2} e^{\frac{2\pi i}{4}}, \sqrt[4]{2} e^{\frac{4\pi i}{4}}, \sqrt[4]{2} e^{\frac{6\pi i}{4}} \big) = \mathbb{Q}(\sqrt[4]{2}, e^{\frac{2\pi i}{4}})
    \end{equation}
  \end{example} 

  \begin{example}[Multivariate Splitting Fields]
    $x^3 - 2 \in \mathbb{Q}[x]$ has a splitting field. By Eisenstein, $x^3 - 2$ is irreducible, and so 
    \begin{equation}
      F = \frac{\mathbb{Q}[x]}{\langle x^3 - 2 \rangle} 
    \end{equation}
    is a field. Furthermore, there must be a root $\alpha \in F$. We factor this in the field $F$ to get 
    \begin{equation}
      x^3 - 2 = (x - \alpha) (x^2 + \alpha x + \alpha^2) 
    \end{equation}
    Then either $x^2 + \alpha x + \alpha^2$ has a root in $F$ and $F$ is the splitting field, or it is irreducible. It turns out that $x^2 + \alpha x + \alpha^2$ is irreducible in $F$, and so the splitting field is 
    \begin{equation}
      E = \frac{F[y]}{\langle y^2 + \alpha y + \alpha^2 \langle} \simeq \frac{\mathbb{Q}[x, y]}{\langle x^3 - 2, y^2 + xy + x^2 \rangle}
    \end{equation}
  \end{example}

\subsection{Finite Fields} 

  We know that a field---as an integral domain---has characteristic $0$ or prime $p$. We also know that a field is a vector space, at least over itself. But given the characteristic of a field, we can model it as a vector space over two very specific fields. 

  \begin{theorem}[Characteristic Determines Base Field of Vector Space]
    \label{thm:char_field}
    Given a field $F$, 
    \begin{enumerate}
      \item If $\Char(F) = p$, then $F$ is a vector space over $\mathbb{Z}_p$. 
      \item If $\Char(F) = 0$, then $F$ is a vector space over $\mathbb{Q}$. 
    \end{enumerate}
  \end{theorem}
  \begin{proof}
    
  \end{proof} 

  Therefore, just from the characteristic we can classify all fields as vector spaces over either $\mathbb{Q}$ or $\mathbb{Z}_p$. Now if we focus on finite fields, we can do a reverse classification. 

  \begin{theorem}[Finite Fields Have Cardinality $p^d$]
    Let $F$ be a finite field. Then $|F| = p^n$ for some $n \in \mathbb{N}$. 
  \end{theorem}
  \begin{proof}
    $F$ is a vector space over $\mathbb{Z}_p$ from \ref{thm:char_field}. Since $F$ has finitely many elements, $F$ has a finite spanning set, which implies $\dim_{\mathbb{Z}_p} F \leq + \infty$. Let $d$ be the dimension and $\{b_1, \ldots, b_d\}$ be the basis. The elements of $F$ are 
    \begin{equation}
      a_1 b_1 + \ldots + a_d b_d
    \end{equation}
    with $a_1, \ldots a_d \in \mathbb{Z}_p$. Thus there are $p^d$ elements of $F$, so $F \simeq \mathbb{Z}_p^d$. 
  \end{proof}

  In fact, for \textit{every} prime power there exists a unique field. Therefore we can create a bijection by proving the converse. 

  \begin{theorem}[Field for Every $p^d$]
    For every prime $p$ and $n \in \mathbb{N}$, there exists a field with $q = p^d$ elements, unique up to isomorphism.  
  \end{theorem} 
  \begin{proof} 
    Let $f(x) = x^q - x \in \mathbb{Z}_p [x]$. Then this polynomial has a splitting field $K \supset \mathbb{Z}$. Now we claim the roots of $f(x)$ in $K$ are distinct and form a subfield $F_q \subset K$. This will complete the proof since $F_q \subset K$ and $K \subset F_q \implies K = F_q$. Assume $\alpha, \beta \in K$ are roots of $f(x)$, and so $\alpha^p = \alpha$ and $\beta^p = \beta$
    \begin{enumerate}
      \item $\alpha + \beta \in K$ since by a modification of Freshman's dream, $(\alpha + \beta)^p = \alpha^p + \beta^p = \alpha + \beta$.\footnote{We induct on $n$ for $q = p^n$. For $n=1$, this is trivial by Freshmans dream. Now assume it holds for some $n \in \mathbb{N}$. Then $(x + y)^{p^{n+1}} = ( (x + y)^{p^n} )^p = (x^{p^n} + y^{p^n})^p = (x^{p^n})^p + (y^{p^n})^p = x^{p^{n+1}} + y^{p^{n+1}}$. }
      \item $(-\alpha)^q = (-1)^q \alpha^q = (-1)^q \alpha = -\alpha$ since $-1 = 1$ or $q$ is odd. 
      \item $\alpha \beta \in K$ since $\mathbb{Z}_p$ is a field and so $(\alpha \beta)^p = \alpha^p \beta^p = \alpha \beta$. 
      \item For multiplicative inverses, let $\alpha \neq 0$. Then $(\alpha^{-1})^p = (\alpha^{p})^{-1} = \alpha^{-1}$. 
      \item For all $p$, $0$ and $1$ are roots so $0, 1 \in K$. 
    \end{enumerate}
    Now we show that $K$ consists of distinct roots. Certainly $0 \in K$ with multiplicity $1$ since $f(x) = x (x^{q-1} - x)$. Now suppose nonzero $r \in K$ is a root with multiplicity $m$. The multiplicity of $r$ is the multiplicity of $0$ of 
    \begin{equation}
      f(x + r) = (x + r)^q - (x + r) = x^q + r^q - x - r = x^q - x
    \end{equation}
    where the final step follows from $0 = r^q - r$ since $r \in K$. Therefore $r$ has multiplicity $1$. Since $K[x]$ has unique factorization property, it follows that $m=1$ and every $r$ is a simple root. 

    To show that every field with $p^n$ elements is unique, let $F$ be such a field. We claim that $\Char(F) = p \implies \mathbb{Z}_p \subset F$. We claim that every element of $F$ is a root of $f(x) = x^q - x \in \mathbb{Z}_p [x]$, where $F$ is the splitting field. Let $G = F^\ast$ be the multiplicative group of units. Since $F$ is a field, then $|F^\ast| = |F| - 1 = p^d - 1$, and by constructing the cyclic group $\langle g \rangle \subset G$ for any $g \in G$, we know by Lagrange's theorem that $g^{|G|} = 1_G$, which implies that for all $x \in F$, 
    \begin{enumerate}
      \item If $x \neq 0$ then $x^{p^d - 1} = x \implies x^{p^d} = x$ and so $x \in K$. 
      \item If $x = 0$ then $x^{p^d} - x = 0$ and so $x \in K$. 
    \end{enumerate}
    Therefore $F \subset K$ with $|F| = |K|$ both finite, and so $F = K$. 
  \end{proof} 

  From this, we can write for every prime $p$ and natural $n$ the finite field of order $p^n$ as $\mathbb{F}_{p^n}$. It is clear that if $n = 1$ then $\mathbb{F}_p \simeq \mathbb{Z}_p$. The final result we will show is a hierarchy of subfields. 

  \begin{theorem}[Hierarchy of Fields]
    For a given prime $p$, if $p^m < p^n$, then 
    \begin{equation}
      F_{q^m} \subset F_{q^n} \iff m \mid n
    \end{equation}
  \end{theorem}

\subsection{Galois Theory} 

  \begin{definition}[Minimal Polynomial]
    
  \end{definition}
  
  Now we unify the three ideas of groups, polynomials, and fields. 

  \begin{definition}[Field Embedding]
    Let $F \subset K_1, K_2$ be two field extensions. An \textbf{$F$-embedding} is a ring homomorphism $\phi: K_1 \to K_2$ such that $\phi(a) = a$ for all $a \in F$. An \textbf{$F$-automorphism of $K$} is an $F$-embedding $\phi: K \to K$. 
  \end{definition}

  \begin{definition}[Galois Group]
    The group of all $F$-automorphisms of $K$ with composition is called the \textbf{Galois group of $K$ over $F$}, denoted $G(K/F)$. 
  \end{definition}
  \begin{proof}
    This is indeed a group under composition. The identity map $\iota \in G(K/F)$. The composition is clearly closed. Now given that $\phi \in G(K/F)$, $\phi^{-1}$ is also an automorphism that is constant on $F$, so it is also in $G(K/F)$. 
  \end{proof} 

  Essentially, the Galois group is a subgroup of the ring automorphism group of $K$ that doesn't vary $F \subset K$. Let's provide a few examples to derive some of the Galois groups. 

  \begin{example}[Galois Group of $\mathbb{Q}(\sqrt{2})$ over $\mathbb{Q}$]
    Given the field extension $\mathbb{Q} \subset \mathbb{Q}(\sqrt{2})$, we have for any $a, b \in \mathbb{Q}$ and $\phi \in G(\mathbb{Q}(\sqrt{2})/\mathbb{Q})$, we have 
    \begin{equation}
      \phi(a + b \sqrt{2}) = \phi(a) + \phi(b \sqrt{2}) = a + b \phi(\sqrt{2}) 
    \end{equation}
    So $\phi$ is completely determined by the value of $\phi(\sqrt{2})$. Now let $\phi(\sqrt{2}) = \alpha + \beta \sqrt{2}$ for some $\alpha, \beta \in \mathbb{Q}$. We have 
    \begin{align}
      2 = \phi(2) = \phi(\sqrt{2} \sqrt{2}) = \phi(\sqrt{2})^2 = (\alpha + \beta\sqrt{2})^2 = (\alpha^2 + 2 \beta^2) + 2 \alpha \beta \sqrt{2} 
    \end{align}
    Therefore $\alpha \beta = 0$ and $\alpha^2 + 2 \beta^2 = 2$. With further casework, we must have $\alpha = 0, \beta = \pm 1$. In conclusion, there are exactly two $\mathbb{Q}$-automorphisms of $\mathbb{Q}(\sqrt{2})$. 
    \begin{enumerate}
      \item The identity map $\iota(a + b \sqrt{2}) = a + b \sqrt{2}$, and 
      \item The conjugation map $\phi(a + b \sqrt{2}) = a - b \sqrt{2}$. 
    \end{enumerate}
  \end{example}

  \begin{example}[Galois Group of $\mathbb{Q}(2^{1/3})$ over $\mathbb{Q}$]
    Given the field extension $\mathbb{Q} \subset \mathbb{Q}(\sqrt[3]{2})$, let us write $\xi = \sqrt[3]{2}$. Then since $\mathbb{Q}(\xi)$ is a vector space with basis $1, \xi, \xi^2$, we can write any element as $a + b \xi + c \xi^2$, and so by definition an element $\phi \in G(\mathbb{Q}(\xi)/\mathbb{Q})$ must satisfy 
    \begin{equation}
      \phi(a + b \xi + c \xi^2) = a + b \phi(\xi) + c \phi(\xi)^2
    \end{equation}
    So the action is completely determined by the value of $\phi(\xi)$. Suppose $\phi(\xi) = \alpha + \beta \xi + \gamma \xi^2$ for some $\alpha, \beta, \gamma \in \mathbb{Q}$. Through some derivation we have 
    \begin{align}
      2 & = \phi(2) = \phi(\xi^3) = (\phi(\xi))^3 = (\alpha + \beta \xi + \gamma \xi^2)^3 \\ 
        & = (\alpha^3 + 2\beta^3 + 4\gamma^3) + 3(\alpha^2\beta + \beta^2\gamma + \gamma^2\alpha)\xi + 3(\alpha\beta^2 + \beta\gamma^2 + \gamma\alpha^2)\xi^2
    \end{align} 
    From the linear independence of $1, \xi, \xi^2$ we can see that 
    \begin{align}
      \alpha^3 + 2\beta^3 + 4\gamma^3 &= 2 \\
      \alpha^2\beta + \beta^2\gamma + \gamma^2\alpha &= 0 \\
      \alpha\beta^2 + \beta\gamma^2 + \gamma\alpha^2 &= 0
    \end{align}
    which turns out to have the only solution $\alpha = \gamma = 0, \beta = 1$. Therefore, the only $\mathbb{Q}$-automorphism of $\mathbb{Q}(\sqrt[3]{2})$ is the idnetity map. 
  \end{example} 

  Now let's look at how an $F$-automorphism acts on the roots of a polynomial. Let $f(x) = x^n + a_{n-1} x^{n-1} + \ldots + a_1 x + a_0 \in F[x]$, $F \subset K$ a field extension, and suppose $f(x)$ has roots $\alpha_1, \ldots, \alpha_m$ lying in $K$ (and perhaps other roots lying in a further extension). It turns out that any $F$-automorphism of $K$ must permute $\alpha_1, \ldots, \alpha_m$, so the roots stay ``within'' the polynomial. 

  \begin{lemma}[$F$-Automorphims Permute Roots]
    Let $F \subset K$ and let $\alpha \in K$ be a root $f(x) \in F[x]$.
    \begin{enumerate}
      \item For any $\phi \in G(K/F)$, $\phi(\alpha)$ is also a root of $f(x)$ in $K$.\footnote{However it may not be a permutation!}
      \item If $K$ is the splitting field of $f(x)$ and $f(x)$ has distinct roots $\alpha_1, \ldots, \alpha_n \in K$, then $G(F/K)$ is a subgroup of $\mathrm{Perm}\{\alpha_1, \ldots, \alpha_n\}$. 
    \end{enumerate}
  \end{lemma}
  \begin{proof}
    We know that $f(\alpha) = 0$. We claim first that $f(\phi(\alpha)) = 0$ and so $\phi(\alpha)$ is also a root. Now assume that $S = \{\alpha_1, \ldots, \alpha_n\}$ is contained in $K$. We know that $\phi(\alpha_j) \in S$. The map $\phi \mapsto \phi(\alpha_j)$ is indeed a group homomorphism $G(K/F) \to \mathrm{Perm}(S)$. If $K$ is the splitting field, then $K = F[S]$, and so if $\phi \in G(K/F)$ fixes all the $\alpha_j$'s, then it must fix all of $K$. 
  \end{proof} 

  \begin{example}
    Therefore, we can get a much simply solution of the Galois group of $\mathbb{Q}(\sqrt[3]{2})$ over $\mathbb{Q}$. Since $\xi = \sqrt[3]{2}$ is a root of the irreducible polynomial $x^3 - 2$, any $\phi \in G(\mathbb{Q}[\xi]/\mathbb{Q})$ must carry $\xi$ to some other root of $x^3 - 2$. The other roots are in the complex plane, and so $\phi$ must carry $\xi \mapsto \xi$. Thus, $\phi$ must be the identity. 
  \end{example}

  \begin{example}
    Let $f(x) = x^2 - 2x - 1 \in \mathbb{Q}[x]$. One root of $f(x)$ is $1 + \sqrt{2} \in \mathbb{Q}(\sqrt{2})$. But we know that there exists a $\phi \in G(\mathbb{Q}(\sqrt{2})/\mathbb{Q})$ that conjugates, and so $1 - \sqrt{2}$ must also be a root. 
  \end{example} 

  \begin{example}
    It follows that $G(\mathbb{C}/\mathbb{R})$ is a group of order $2$ generated by complex conjugation. 
  \end{example} 

  By studying how $F$-automorphisms behave, we are able to understand a bit more about the Galois group. Now we attempt to know more about the order of a Galois group by finding out how many possible $F$-automorphisms over $K$ can exist. If we let $F \subset K$ be a field extension of degree $n >1$ (i.e. $\dim_F K = n > 1$) and $\alpha \in K$, the set of $n+1$ vectors $\{1, \alpha, \alpha^2, \ldots, \alpha^n\}$ must be linearly dependent over a vector space of dimension $n$. Therefore, there exists some nontrivial linear combination 
  \begin{equation}
    0 = a_n \alpha^n + a_{n-1} \alpha^{n-1} + \ldots + a_1 \alpha + a_0
  \end{equation} 
  which corresponds to some nonzero $f(x) = a_n x^n + \ldots a_0$. Therefore $\alpha$ is a root of a polynomial $f(x) \in F[x]$ of degree at most $n$---therefore a root of an irreducible polynomial of degree at most $n$ (since there could be a smaller linearly dependent set). 

  Now if we assume that $K = F[\alpha]$, let's try to see how many elements $G(K/F)$ could have. 
  \begin{enumerate}
    \item $f(x)$ must have degree exactly $n$ (why?). Since $1, \alpha, \ldots, \alpha^{n-1}$ form a basis of $K$ over $F$, if $\phi \in G(K/F)$, the action of $\phi$ on $K = F[\alpha]$ is completely determined by $\phi(\alpha)$, which is what we saw for the first two examples above. 
    \item On the other hand, $\phi$ must map $\alpha$ to some root of $f(x)$ in $K$, and there are at most $n = [K:F]$ of these. 
  \end{enumerate}
  Thus we have $|G(F[\alpha]/F)| \leq [F[\alpha]:F]$, and it follows that if $F[\alpha] \subset K$, then $|G(F[\alpha]/F)| \leq [K:F]$. But our goal is to do so for arbitrary field extensions, i.e. 
  \begin{equation}
    |G(K/F)| \leq [K:F]
  \end{equation} 
  To do this, we will need to talk about extensions of field isomorphisms. Recall that a field (i.e. a ring) isomorphism $\phi: F \to F^\prime$ induces a ring isomorphism $\Tilde{\phi}: F[x] \to F^\prime [x]$. 


  \begin{theorem}[Extending Isomorphisms of Fields]
    Let $\phi: F \to F^\prime$ be an isomorphism of fields.
    \begin{enumerate}
      \item Let $f(x) \in F[x]$ be an irreducible polynomial with root $\alpha$ in some field extension $K \supset F$.
      \item Let $g(x) = \phi(f(x)) \in F^\prime[x]$ and let $\beta$ be a root of $g(x)$ in some field extension $K^\prime \supset F^\prime$. 
    \end{enumerate}
    Then there is a unique isomorphism  
    \begin{equation}
      \bar{\phi} : F[\alpha] \to F^\prime [\alpha^\prime]
    \end{equation}
    that is an extension of $\phi$ (i.e. behaves the same under $F$) and carries $\alpha$ to $\alpha$. 
  \end{theorem}

  We were interested in the Galois group of the symmetries of the field. Now we will extend this to bigger fields. 

  \begin{theorem}[Bound on Number of Field Isomorphism Extensions]
    Let $F \subset K_1, F \subset K_2$ be field extensions. Then there are at most $[K_1:F]$ $F$-embeddings $\phi: K_1 \to K_2$. Moreoever, if there are exactly $[K_1:F]$ $F$-embeddings, then for all $F \subset E \subset K$, there are $[E:F]$ $F$-embeddings $\phi: E \to K_2$. 
  \end{theorem}
  \begin{proof}
    By induction on $[K_1:F]$. When $[K_1:F] = 1$. Now suppose the proposition holds for all $F, K_1, K_2$ with $[K_1:F] \leq n-1$. Now choose $F, K_1, K_2$, as in the statement with $[K_1 : F] = n > 1$. Since $[K_1: F] > 1$, we can choose $\alpha \in K_1$, $\alpha \not\in F$. Let $f(x) \in F[x]$ be the minimal polynomial of $\alpha$. We've seen that given the smallest subfield of $K_1$ containing $\alpha$, i.e. $F[\alpha]$, we can use the evaluation homomorphism to state 
    \begin{equation}
      F[\alpha] \simeq \frac{F[x]}{\langle f(x)\rangle}, \qquad F[\alpha] \xleftarrow{\mathrm{ev}_\alpha} \frac{F[x]}{\langle f(x)\rangle} 
    \end{equation}
    Any $F$-embedding of $K_1$ restricst to an $F$-embedding of $F[\alpha]$. By induction, there are at most $[F[\alpha]: F] = m$ $F$-embeddings 
    \begin{equation}
      \phi_1, \ldots, \phi_m : F[\alpha] \to K_2
    \end{equation}
    But by induction, we know that there are at most $[K_1: F[\alpha]]$ $F[\alpha]$-embeddings $K_1 \to K_2$ where $F[\alpha] \subset K_2$ (which we can defined through multiple ways for each injection $\phi_i$). This implies that there are at most 
    \begin{align}
      m \, [K_1: F[\alpha]] & = [F[\alpha]: F]\, [K_1: F[\alpha] \\
                            & = [K_1 : F]
    \end{align}
    $F$-embeddings. Suppose equality holds. Let $F \subset E \subset K_1$. We know 
    \begin{enumerate}
      \item There are at most $[E:F]$ $F$-embeddings $\phi_1, \ldots, \phi_{[E:F]}: E \to K_2$ 
      \item There are at most $[K_1:E]$ $E$-embeddings $K_1 \to K_2$ where here $E \subset K_2$. 
    \end{enumerate}
    This gives at most $[K_2:E] [E:F]$ $F$-embeddings since $[K_2 :E] [E:F] = [K_2 :F]$, and there are exactly $[K_2 : F]$ $F$-embeddings $K_1 \to K_2$, all the $\leq$'s must be $=$'s (otherwise it would be a strict inequality) in that there must be exactly $[E:F]$ $F$-embeddings $E \to K_2$ and exactly $[K_1: E]$ $E$-embeddings $K_1 \to K_2$. 

    Correction. We need to see that there are at most $[F[\alpha]:F]$ $F$-embeddings $F[\alpha] \to K_2$. It's okay to use induction if $[F[\alpha]: F] \subset [K_1:F]$. In general, we showed $F$-embeddings maps 
  \end{proof}

  If we reach this bound, then there are some special properties. 

  \begin{definition}[Galois]
    A field extension $F \subset K$ is \textbf{Glaois} if the number of $F$-embeddings $K \to K$ is $[K:F]$. 
  \end{definition}

  \begin{example}
    We review the derived Galois groups above and see if the field extensions are Galois. 
    \begin{enumerate}
      \item $\mathbb{Q}(\sqrt{2})$ is a Galois extension of $\mathbb{Q}$. 
      \item $\mathbb{Q}(\sqrt[3]{2})$ is not a Galois extension of $\mathbb{Q}$ since $|G(\mathbb{Q}[\sqrt[3]{2}]/\mathbb{Q})| = 1$ but $[\mathbb{Q}[\sqrt[3]{2}] : \mathbb{Q}] = 3$. 
      \item Let $K = \mathbb{Q}[\sqrt[3]{2}, i \sqrt{3}]$ is the splitting field of $f(x) = x^3 - 2$. Then $[K:\mathbb{Q}] = 6$. $G(K/\mathbb{Q}) \simeq S_3$ has order $6$ since we showed 6 $\mathbb{Q}$-automorphisms of $K$, but now we know that there can be no more. 
    \end{enumerate}
  \end{example}

  \begin{example}
    Let $\alpha = \sqrt[7]{2}$. $\mathbb{Q} \subset \mathbb{Q}[\alpha]$ is not a Galois extension. $x^7 - 2$ is a polynomial with root $\alpha$, and by Eisenstein $x^7 - 2$ is irreducible. So $f(x) = x^7 - 2$ is the minimal polynomial of $\alpha$. This means that the number of $\mathbb{Q}$-embeddings $\mathbb{Q}[\alpha] \to \mathbb{Q}[\alpha]$ is in bijection with the number of roots of $f(x)$ in $\mathbb{Q}[\alpha]$. All $7$ roots of $f(x)$ are $\sqrt[7]{2} e^{2\pi i j/7}$ for $j = 0, \ldots, 6$, which has one real root. So there is $1$ $\mathbb{Q}$-embedding $\mathbb{Q}[\alpha] \to \mathbb{Q}[\alpha]$. But $[\mathbb{Q}[\sqrt[7]{2}] : \mathbb{Q}] = \deg{f(x)} = 7$. Since $\mathbb{Q}[\alpha] \simeq \mathbb{Q}[x]/{\langle x^7 - 2 \rangle}$, which are polynomials of degree at most $6$. 
  \end{example}

  \begin{theorem}[Splitting Fields of a Square Free Polynomial are Galois]
    Suppose $F \subset K$ is the splitting field of a polynomial $f(x) \in F[x]$ such that no irreducible factor of $f(x)$ has repeated roots. Then $F \subset K$ is Galois. 
  \end{theorem}
  \begin{proof}
    Since $K$ is the splitting field of $f(x)$ over $F$, we have $F[r_1, \ldots, r_m]$ where the $r_i$ are the distinct roots of $f(x)$. We show by induction that there are $[F[r_1, \ldots, r_j]:F]$ $F$-embeddings of $F[r_1, \ldots, r_j] \to K$. For $j = 1$, $r_1$ is a root of some irreducible factor $f_1 (x)$ of $f(x)$. 
    \begin{equation}
      F[r_1] \simeq \frac{F[x]}{\langle f_1 (x) \rangle}
    \end{equation} 
    and the set of $F$-embeddings $F[r_1] \to K$ is in bijection with the set of roots $\alpha \in K$ of $f_1 (x)$. By hypothesis , $f(x)$ has no repeated roots, which implies that the number of $F$-embeddings $F[r_1] \to K$ is $\deg{f_1 (x)} = [F[r_1]: F]$ which gives the base case. For the inductive step, we know there are 
    \begin{equation}
      [F[r_1, \ldots, r_{j-1}]: F] 
    \end{equation}
    $F$-embeddings $F[r_1, \ldots, r_{j-1}] \xrightarrow{\phi} K$. For each, we will show that there are exactly $[F[r_1, \ldots, r_j]: F[r_1, \ldots, r_{j-1}]]$ extensions of $\phi$ which completes the proof. Because 
    \begin{align}
      F[r_1, \ldots, r_j]: F] & = F[r_1, \ldots, r_j]: F[r_1, \ldots, r_{j-1}]] F[r_1, \ldots, r_{j-1}]: F]
    \end{align} 
    Let $g(x)$ be the minimal polynomial of $r_j$ over $F[r_1, \ldots, r_{j-1}]$. Since $g(r_j) = 0$, $g$ divides one of the irreducible factors of $f(x)$ in $F[r_1, \ldots, r_{j-1}] [x]$ which implies it has no repeated roots. Then 
    \begin{equation}
      F[r_1, \ldots, r_j] = \frac{E[x]}{\langle g(x)\rangle} 
    \end{equation}
    The number of $E$-embeddings is equal to the number of roots in in $g$ which is $\deg{g} = F[r_1, \ldots, r_j: F]$.
  \end{proof} 

  Now we restrict our scope to $\mathbb{Q}$. Note the following. 

  \begin{lemma}
    Any irreducible polynomial in $\mathbb{Q}$ has no repeated roots. 
  \end{lemma}

  With this, the following is immediately. 

  \begin{theorem} 
    % Shifrin 6.8 
    Let $\mathbb{Q} \subset K$ be the splitting field of $f(x) \in Q[x]$. Then $\mathbb{Q} \subset K$ is Galois. 
  \end{theorem}

  What about the converse? There are two steps to proving that for any Galois field extension $F \subset K$, there exists a polynomial $f(x) \in F[x]$ that splits in $K$. 

  \begin{lemma}[Irreducible Polynomial with Root in Galois Extension Splits] 
    % Shifrin 6.9
    Let $F \subset K$ be a Galois extension of fields. Let $f(x) \in F[x]$ be an irreducible polynomial with a root $\alpha \in K$. Then $f(x)$ splits in $K$, i.e. all other roots must be in $K$.  
  \end{lemma} 

  \begin{theorem}[Every Galois Extension is a Splitting Field]
    Let $F \subset K$ be a Galois extension. Then $K$ is the splitting field of a polynomial $f(x) \in F[x]$. 
  \end{theorem}

  Therefore, we have made a bijection of sets between set of $F$-embeddings and.  


  \begin{theorem}[Fundamental Theorem of Galois Theory]
    
  \end{theorem}


\subsection{Cubic Equations}

  The well known discriminant of a quadratic equation 
  \begin{equation}
    f(x) = ax^2 + bx + c
  \end{equation}
  is known in the form $\nabla = b^2 - 4ac$. However, we will present it in a slightly different manner. 

  \begin{definition}
    The \textbf{discriminant} $D(\varphi)$ of a quadratic polynomial
    \begin{equation}
      \varphi = a_0 x^2 + a_1 x + a_2 \in \mathbb{C}[x]
    \end{equation}
    with $c_1, c_2 \in \mathbb{C}$ as its roots is defined
    \begin{equation}
      D(\varphi) = a_1^2 - 4 a_0 a_2 = a_0^2 \bigg( \Big(\frac{a_1}{a_0} \Big)^2 - \frac{4 a_2}{a_0} \bigg) = a_0^2 \big( (c_1 + c_2)^2 - 4 c_1 c_2 \big) = a_0^2 (c_1 - c_2)^2
    \end{equation}
    Clearly, the value of $D(\varphi)$ can tell us three things
    \begin{enumerate}
      \item $c_1, c_2 \in \mathbb{R}, c_1 \neq c_2$. Then $c_1 - c_2$ is a nonzero real number and $D(\varphi) > 0$. 
      \item $c_1 = c_2 \in \mathbb{R}$. Then $c_1 - c_2 = 0$ and $D(\varphi) = 0$. 
      \item $c_1, c_2 \in \mathbb{C}, c_1 = \bar{c}_2$. Then, $c_1 - c_2$ is a nonzero strictly imaginary number and $D(\varphi) < 0$. 
    \end{enumerate}
  \end{definition}

  \begin{definition}
    We can generalize this notion of the discriminant to arbitrary polynomials
    \begin{equation}
      \varphi = a_0 x^n + a_1 x^{n-1} + ... + a_{n-1} x + a_n \in \mathbb{F}[x], \; a_0 \neq 0
    \end{equation}
    The discriminant $D(\varphi)$ of the polynomial above is defined
    \begin{equation}
      D(\varphi) \equiv a_0^{2n-2} \prod_{i>j} (c_i - c_j)^2
    \end{equation}
    The $a_0$ term isn't very important in this formula, since it does not affect whether $D(\varphi)$ is positive, negative, or zero. 
  \end{definition}

  \begin{definition}
    A polynomial 
    \begin{equation}
      \varphi = a_0 x^n + a_1 x^{n-1} + ... + a_{n-1} x + a_n \in \mathbb{F}[x], \; a_0 \neq 0
    \end{equation}
    where $a_1 = 0$ is called \textbf{depressed}. A depressed cubic polynomial is of form
    \begin{equation}
      \varphi = x^3 + p x + q
    \end{equation}
  \end{definition}

  \begin{proposition}
    Every monic (leading coefficeint $=1$) polynomial (and non-monic ones) 
    \begin{equation}
      \varphi = x^n + a_1 x^{n-1} + ... + a_{n-1} x + a_n \in \mathbb{F}[x], \; a_0 \neq 0
    \end{equation}
    can be turned into a depressed polynomial with the change of variable
    \begin{equation}
      x = y - \frac{a_1}{n}
    \end{equation}
    to get the polynomial 
    \begin{equation}
      \psi = y^n + b_2 y^{n-2} + ... + b_{n-1} y + b_n
    \end{equation}
  \end{proposition}

  \begin{lemma}
    A cubic polynomial 
    \begin{equation}
      \varphi = a_0 x^3 + a_1 x^2 + a_2 x + a_3 \in \mathbb{R}[x]
    \end{equation}
    with roots $c_1, c_2, c_3 \in \mathbb{C}$ has discriminant
    \begin{equation}
      D(\varphi) \equiv a_0^4 (c_1 - c_2)^2 (c_1 - c_3)^2 (c_2 - c_3)^2
    \end{equation}
    With a bit of evaluation, it can also be expressed in terms of its coefficients as
    \begin{equation}
      D(\varphi) = a_1^2 a_2^2 - 4a_1^3 a_3 - 4a_0 a_2^3 + 18 a_0 a_1 a_2 a_3 - 27 a_0^2 a_3^2
    \end{equation}
    Again, three possibilities can occur (up to reordering of its roots). 
    \begin{enumerate}
        \item $c_1, c_2, c_3$ are distinct real numbers. Then $D(\varphi) > 0$. 
        \item $c_1, c_2, c_3 \in \mathbb{R}, c_1 = c_2$. Then $D(\varphi) = 0$. 
        \item $c_1 \in \mathbb{R}, c_2 = \bar{c}_3 \not\in \mathbb{R}$. Then $D(\varphi) < 0$. 
    \end{enumerate}
    Furthermore, the cubic formula used to find the roots of the polynomial is 
    \begin{equation}
      c_{1, 2, 3} = \sqrt[3]{-\frac{q}{2} + \sqrt{\frac{p^3}{27} + \frac{q^2}{4}}} + \sqrt[3]{-\frac{q}{2} - \sqrt{\frac{p^3}{27} + \frac{q^2}{4}}}
    \end{equation}
    known as \textbf{Cardano's formula}, after the mathematician Gerolamo Cardano. 
  \end{lemma}





\section{Galois Theory}

Just a placeholder don't judge me plz. 

\subsection{Symmetric Polynomials}

  \begin{definition}
    A polynomial $f \in \mathbb{F}[x_1, ..., x_n]$ is called \textbf{symmetric} if it is invariant under any permutation of the variables $x_i$. 
  \end{definition}

  \begin{example}
    Power sums are symmetric polynomials. 
    \begin{equation}
      p(x_1, x_2, ..., x_n) = \sum_{i=1}^n x_i^k
    \end{equation}
  \end{example}

  \begin{definition}
    An \textbf{elementary symmetric polynomial} is a symmetric polynomial of one of these forms: 
    \begin{align*}
      \sigma_1 & = x_1 + x_2 + ... + x_n \\
      \sigma_2 & = x_1 x_2 + x_1 x_3 + ... + x_{n-1} x_n \\
      ... & = ... \\
      \sigma_k & = \sum_{i_1 < ... < i_k} x_{i_1} x_{i_2} ... x_{i_k} \\
      ... & = ... \\
      \sigma_n & = x_1 x_2 ... x_n
    \end{align*}
  \end{definition}

  The following theorem presents an extremely useful result about the decomposition of symmetric polynomials. 

  \begin{theorem}
    Every symmetric polynomial can be written as a polynomial of elementary symmetric polynomials $\sigma_i$. 
  \end{theorem}

  \begin{example}
    The polynomial 
    \begin{equation}
      f \equiv \sum_{i=1}^n x_i^3
    \end{equation}
    can be expressed as 
    \begin{equation}
      f = \sigma_1^3 - 3 \sigma_1 \sigma 2 + 3 \sigma_3
    \end{equation}
  \end{example}

\subsection{Cubic Equations}

  The well known discriminant of a quadratic equation 
  \begin{equation}
    f(x) = ax^2 + bx + c
  \end{equation}
  is known in the form $\nabla = b^2 - 4ac$. However, we will present it in a slightly different manner. 

  \begin{definition}
    The \textbf{discriminant} $D(\varphi)$ of a quadratic polynomial
    \begin{equation}
      \varphi = a_0 x^2 + a_1 x + a_2 \in \mathbb{C}[x]
    \end{equation}
    with $c_1, c_2 \in \mathbb{C}$ as its roots is defined
    \begin{equation}
      D(\varphi) = a_1^2 - 4 a_0 a_2 = a_0^2 \bigg( \Big(\frac{a_1}{a_0} \Big)^2 - \frac{4 a_2}{a_0} \bigg) = a_0^2 \big( (c_1 + c_2)^2 - 4 c_1 c_2 \big) = a_0^2 (c_1 - c_2)^2
    \end{equation}
    Clearly, the value of $D(\varphi)$ can tell us three things
    \begin{enumerate}
      \item $c_1, c_2 \in \mathbb{R}, c_1 \neq c_2$. Then $c_1 - c_2$ is a nonzero real number and $D(\varphi) > 0$. 
      \item $c_1 = c_2 \in \mathbb{R}$. Then $c_1 - c_2 = 0$ and $D(\varphi) = 0$. 
      \item $c_1, c_2 \in \mathbb{C}, c_1 = \bar{c}_2$. Then, $c_1 - c_2$ is a nonzero strictly imaginary number and $D(\varphi) < 0$. 
    \end{enumerate}
  \end{definition}

  \begin{definition}
    We can generalize this notion of the discriminant to arbitrary polynomials
    \begin{equation}
      \varphi = a_0 x^n + a_1 x^{n-1} + ... + a_{n-1} x + a_n \in \mathbb{F}[x], \; a_0 \neq 0
    \end{equation}
    The discriminant $D(\varphi)$ of the polynomial above is defined
    \begin{equation}
      D(\varphi) \equiv a_0^{2n-2} \prod_{i>j} (c_i - c_j)^2
    \end{equation}
    The $a_0$ term isn't very important in this formula, since it does not affect whether $D(\varphi)$ is positive, negative, or zero. 
  \end{definition}

  \begin{definition}
    A polynomial 
    \begin{equation}
      \varphi = a_0 x^n + a_1 x^{n-1} + ... + a_{n-1} x + a_n \in \mathbb{F}[x], \; a_0 \neq 0
    \end{equation}
    where $a_1 = 0$ is called \textbf{depressed}. A depressed cubic polynomial is of form
    \begin{equation}
      \varphi = x^3 + p x + q
    \end{equation}
  \end{definition}

  \begin{proposition}
    Every monic (leading coefficeint $=1$) polynomial (and non-monic ones) 
    \begin{equation}
      \varphi = x^n + a_1 x^{n-1} + ... + a_{n-1} x + a_n \in \mathbb{F}[x], \; a_0 \neq 0
    \end{equation}
    can be turned into a depressed polynomial with the change of variable
    \begin{equation}
      x = y - \frac{a_1}{n}
    \end{equation}
    to get the polynomial 
    \begin{equation}
      \psi = y^n + b_2 y^{n-2} + ... + b_{n-1} y + b_n
    \end{equation}
  \end{proposition}

  \begin{lemma}
    A cubic polynomial 
    \begin{equation}
      \varphi = a_0 x^3 + a_1 x^2 + a_2 x + a_3 \in \mathbb{R}[x]
    \end{equation}
    with roots $c_1, c_2, c_3 \in \mathbb{C}$ has discriminant
    \begin{equation}
      D(\varphi) \equiv a_0^4 (c_1 - c_2)^2 (c_1 - c_3)^2 (c_2 - c_3)^2
    \end{equation}
    With a bit of evaluation, it can also be expressed in terms of its coefficients as
    \begin{equation}
      D(\varphi) = a_1^2 a_2^2 - 4a_1^3 a_3 - 4a_0 a_2^3 + 18 a_0 a_1 a_2 a_3 - 27 a_0^2 a_3^2
    \end{equation}
    Again, three possibilities can occur (up to reordering of its roots). 
    \begin{enumerate}
        \item $c_1, c_2, c_3$ are distinct real numbers. Then $D(\varphi) > 0$. 
        \item $c_1, c_2, c_3 \in \mathbb{R}, c_1 = c_2$. Then $D(\varphi) = 0$. 
        \item $c_1 \in \mathbb{R}, c_2 = \bar{c}_3 \not\in \mathbb{R}$. Then $D(\varphi) < 0$. 
    \end{enumerate}
    Furthermore, the cubic formula used to find the roots of the polynomial is 
    \begin{equation}
      c_{1, 2, 3} = \sqrt[3]{-\frac{q}{2} + \sqrt{\frac{p^3}{27} + \frac{q^2}{4}}} + \sqrt[3]{-\frac{q}{2} - \sqrt{\frac{p^3}{27} + \frac{q^2}{4}}}
    \end{equation}
    known as \textbf{Cardano's formula}, after the mathematician Gerolamo Cardano. 
  \end{lemma}



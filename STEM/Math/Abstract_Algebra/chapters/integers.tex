\section{Integers} 

\subsection{Exercises} 

  \begin{exercise}[Shifrin 1.2.1]
    For each of the following pairs of numbers $a$ and $b$, find $d = \gcd(a,b)$ and express $d$ in the form $ma+nb$ for suitable integers $m$ and $n$.
    \begin{enumerate}
      \item[(a)] $14, 35$
      \item[(b)] $56, 77$
      \item[(c)] $618, 336$
      \item[(d)] $2873, 6643$
      \item[(e)] $512, 360$
      \item[(f)] $4432, 1080$
    \end{enumerate}
  \end{exercise}
  \begin{solution}
    Listed. 
    \begin{enumerate}
      \item $d = 7 = (-2) \cdot 14 + (1) \cdot 35$. 
      \item $d = 7 = (-4) \cdot 56 + 3 \cdot 77$. 
      \item $d = 6 = -25 \cdot 618 + 46 \cdot 336$ 
      \item $d = 13 = 37 \cdot 2873 + (-16) \cdot 6643$. 
      \item $d = 8 = 19 \cdot 512 + (-27) \cdot 360$. 
      \item $d = 8 = 29 \cdot 4432 + (-119) \cdot 1080$. 
    \end{enumerate}
  \end{solution}

  \begin{exercise}[Shifrin 1.2.2]
    You have at your disposal arbitrarily many 4-cent stamps and 7-cent stamps. What are the postages you can pay? Show in particular that you can pay all postages greater than 17 cents.
  \end{exercise}

  \begin{exercise}[Shifrin 1.2.3]
    Prove that whenever $m \neq 0$, $\gcd(0, m) = |m|$.
  \end{exercise}

  \begin{exercise}[Shifrin 1.2.4]
    \begin{enumerate}
      \item[(a)] Prove that if $a|x$ and $b|y$, then $ab|xy$.
      \item[(b)] Prove that if $d = \gcd(a, b)$, then $\gcd(\frac{a}{d}, \frac{b}{d}) = 1$.
    \end{enumerate}
  \end{exercise}

  \begin{exercise}[Shifrin 1.2.5]
    Prove or give a counterexample: the integers $q$ and $r$ guaranteed by the division algorithm, Theorem 2.2, are unique.
  \end{exercise}

  \begin{exercise}[Shifrin 1.2.6]
     Prove or give a counterexample. Let $a, b \in \mathbb{Z}$. If there are integers $m$ and $n$ so that $d = am + bn$, then $d = \gcd(a, b)$.
  \end{exercise}

  \begin{exercise}[Shifrin 1.2.7]
    Generalize Proposition 2.5: if $\gcd(m, c) = 1$ and $m|cz$, then prove $m|z$.
  \end{exercise}
  \begin{solution}
    Let $\mathrm{gcd}(m, c) = 1$ and $m | cz$. Then there exists $a, b \in \mathbb{Z}$ such that $am + bc = 1$. Multiply both sides of the equation by $z$ to get by the distributive property 
    \begin{equation}
      (am + bc) z = amz + bcz = z
    \end{equation} 
    $m | amz$ and $m | cz \implies m | bcz$. Therefore, the sum of the two, which is equal to $z$, must be divisible by $m$. Therefore $m | z$. 
  \end{solution}

  \begin{exercise}[Shifrin 1.2.8]
    Suppose $a, b, n \in \mathbb{N}$, $\gcd(a, n) = 1$, and $\gcd(b, n) = 1$. Prove or give a counterexample: $\gcd(ab, n) = 1$.
  \end{exercise}

  \begin{exercise}[Shifrin 1.2.9]
    Prove that if $p$ is prime and $p|(a_1 a_2 \ldots a_n)$, then $p|a_j$ for some $j$, $1 \leq j \leq n$. (Hint: Use Proposition 2.5 and induction.)
  \end{exercise}

  \begin{exercise}[Shifrin 1.2.10]
    Given a positive integer $n$, find $n$ consecutive composite numbers.
  \end{exercise}

  \begin{exercise}[Shifrin 1.2.11]
    Prove that there are no integers $m, n$ so that $(\frac{m}{n})^2 = 2$. (Hint: You may start by assuming $m$ and $n$ are relatively prime. Why? Then use Exercise 1.1.3.)
  \end{exercise}

  \begin{exercise}[Shifrin 1.2.12]
    Find all rectangles whose sides have integral lengths and whose area and perimeter are equal.
  \end{exercise}

  \begin{exercise}[Shifrin 1.2.13]
    Given two nonzero integers $a, b$, in analogy with the definition of $\gcd(a, b)$, we define the \textbf{least common multiple} $\operatorname{lcm}(a, b)$ to be the positive number $\mu$ with the properties:
    \begin{enumerate}
      \item[(i)] $a|\mu$ and $b|\mu$, and
      \item[(ii)] if $s \in \mathbb{Z}$, $a|s$ and $b|s \Rightarrow \mu|s$.
    \end{enumerate}
    Prove that
    \begin{enumerate}
      \item[(a)] if $\gcd(a, b) = 1$, then $\mu = ab$. (Hint: If $\gcd(a, b) = 1$, then there are integers $m$ and $n$ so that $1 = ma + nb$; therefore, $s = mas + nbs$.)
      \item[(b)] more generally, if $\gcd(a, b) = d$, then $\mu = ab/d$.
    \end{enumerate}
  \end{exercise}
  \begin{solution}
    Listed. 
    \begin{enumerate}
      \item We can simply verify the two properties. Since $\mu = ab$, $a | \mu$ and $b | \mu$ trivially by the existence of $b$ and $a$, respectively. As for the second property, let $s \in \mathbb{Z}$ exist such that $a | s$ and $b | s$. Since $a | s$, $s = xa$ for some $x \in \mathbb{Z}$. But since $b | s$, $b | xa$. Since $\mathrm{gcd}(a, b) = 1$ by assumption, the result in [Shifrin 1.2.7] tells us that $b | x$, i.e. there exists some $k \in \mathbb{Z}$ such that $x = kb$. Therefore $s = xa = kba = kab = k \mu$. By existence of $k$, $\mu | s$, and we are done. 
      \item Given $a, b$ with $\mathrm{gcd}(a, b) = d$, there exists some $a^\prime, b^\prime \in \mathbb{Z}$ s.t. $a = da^\prime, b = db^\prime$. We claim that $\mu = ab/d \coloneqq d a^\prime b^\prime$ is the lcm.\footnote{Since division isn't generally closed in the integers, I prefer to define $ab/d$ this way.} It is clear that $a | \mu$ and $b | \mu$ by the existence of integers $b^\prime$ and $a^\prime$, respectively. To prove the second property, let $s \in \mathbb{Z}$ with $a | s$ and $b | s$. Since $a | s \iff d a^\prime | s$, there must exist some $x \in \mathbb{Z}$ s.t. $s = d a^\prime x$. But since $b | s$, this means that $d b^\prime | s \iff d b^\prime | d a^\prime x \iff b^\prime | a^\prime x$. But $\mathrm{gcd}(a^\prime, b^\prime) = 1$ which follows from the definition of gcd, and so by [Shifrin 1.2.7] it must be the case that $b^\prime | x$, i.e. there exists some $k \in \mathbb{Z}$ s.t. $x = b^\prime k$. Substituting this back we have $s = d a^\prime b^\prime k = \mu k$, and by existence of $k$ it follows that $\mu | s$. Since it satisfies these 2 properties $\mu$ is the lcm. 
    \end{enumerate}
  \end{solution} 

  \begin{exercise}[Shifrin 1.2.14]
    See Exercise 13 for the definition of $\operatorname{lcm}(a, b)$. Given prime factorizations $a = p_1^{\mu_1} \cdots p_m^{\mu_m}$ and $b = p_1^{\nu_1} \cdots p_m^{\nu_m}$, with $\mu_i, \nu_i \geq 0$, express $\gcd(a, b)$ and $\operatorname{lcm}(a, b)$ in terms of $p_1,\ldots,p_m$. Prove that your answers are correct.
  \end{exercise}

  \begin{exercise}[Shifrin 1.3.8] 
    We see that in $\bmod{10}$, 
    \begin{align}
      3^{400} \equiv 9^{200} \equiv (-1)^{200} \equiv 1^{100} \equiv 1
    \end{align} 
    so the last digit is $1$. To get the last 2 digits, we use the binomial expansion and focus on the last 2 terms. 
    \begin{equation}
      3^{400} = 9^{200} = (10 - 1)^{200} = \ldots + \binom{200}{199} 10^1 (-1)^{199} + \binom{200}{200} (-1)^{200} 
    \end{equation}
    since every combination of the form $\binom{n}{k}$ is an integer and all the other terms have a factor of $10^2$, the expansion $\bmod{100}$ becomes 
    \begin{equation}
      3^{400} \equiv \binom{200}{199} 10^1 (-1)^{199} + \binom{200}{200} (-1)^{200} = 200 \cdot 10 \cdot (-1)^{199} + 1 \equiv 1 \pmod{100}
    \end{equation}
    and so the last two digits is $01$. To get the last digit of $7^{99}$, we see that in $\bmod{10}$, 
    \begin{equation}
      7^{99} \equiv 7^{96} \cdot 7^3 \equiv (7^4)^{24} \cdot 343 \equiv 2401^{24} \cdot 343 \equiv 1^{24} \cdot 3 \equiv 3
    \end{equation}
  \end{exercise}

  \begin{exercise}[Shifrin 1.3.10]
    We must show that 
    \begin{equation}
      n \equiv 0 \pmod{13} \iff n^\prime = \sum_{i=1}^k a_i 10^{i-1} + 4a_0 \equiv 0 \pmod{13}
    \end{equation} 
    We see that $n \equiv n + 39 a_0 \equiv 0 \pmod{13}$, and 
    \begin{align}
      n + 39 a_0 & = \sum_{i=0}^k 10^i a_i + 39 a_0 \\
                 & = \sum_{i=1}^k 10^i a_i + 40 a_0 \\
                 & = 10 \bigg( \sum_{i=1}^k 10^{i-1} a_i + 4 a_0 \bigg) \\
                 & = 10 n^\prime
    \end{align} 
    and so we have $n \equiv 10 n^\prime \pmod{13}$, and so $n^\prime \equiv 0 \pmod{13} \implies n \equiv 0 \pmod{13}$. Conversely, if $n \equiv 0 \pmod{13}$, then $4n \equiv 0 \pmod{13}$, but $4n \equiv 40 n^\prime$ and so $n^\prime \equiv 40 n^\prime \equiv 4n \equiv 0 \pmod{13}$. Therefore both implications are proven. 
  \end{exercise}

  \begin{exercise}[Shifrin 1.3.12]
    Suppose that $p$ is prime. Prove that if $a^2 \equiv b^2 \pmod{p}$, then $a \equiv b \pmod{p}$ or $a \equiv -b \pmod{p}$. 
  \end{exercise}
  \begin{solution}
    We have 
    \begin{align}
      a^2 \equiv b^2 \pmod{p} & \implies a^2 - b^2 \equiv 0 \pmod{p} \\
                              & \implies (a + b) (a - b) \equiv 0 \pmod{p}
    \end{align} 
    We claim that there are no zero divisors in $\mathbb{Z}_p$. If $mn \equiv 0 \pmod{p}$, then by definition this means $p | mn$, which implies that in the integers this must mean that $p | m$ or $p | n$.\footnote{Proposition 2.5} But since $m, n \not\equiv 0$, $p \not| n$ and $p \not| m$, arriving at a contradiction. Going back to our main argument, it must be the case that $a + b \equiv 0 \implies a \equiv -b$ or $a - b \equiv 0 \implies a \equiv b$.  
  \end{solution}

  \begin{exercise}[Shifrin 1.3.15]
    Let us assume that $n = a^2 + b^2 + c^2$ for some $a, b, c \in \mathbb{Z}$. Let us consider for each integer $z$, all the possible values of $z^2 \pmod{8}$. 
    \begin{align}
      z \equiv 0 & \implies z^2 \equiv 0 \pmod{8} \\
      z \equiv 1 & \implies z^2 \equiv 1 \pmod{8} \\
      z \equiv 2 & \implies z^2 \equiv 4 \pmod{8} \\
      z \equiv 3 & \implies z^2 \equiv 1 \pmod{8} \\
      z \equiv 4 & \implies z^2 \equiv 0 \pmod{8} \\
      z \equiv 5 & \implies z^2 \equiv 1 \pmod{8} \\
      z \equiv 6 & \implies z^2 \equiv 4 \pmod{8} \\
      z \equiv 7 & \implies z^2 \equiv 1 \pmod{8} 
    \end{align}
    Therefore, $a^2 + b^2 + c^2 \pmod{8}$ can take any values of the form 
    \begin{equation}
      x + y + z \pmod{8} \text{ for } x, y, z \in \{0, 1, 4\}
    \end{equation}
    Since addition is commutative, WLOG let $x \leq y \leq z$. We can just brute force search this. 
    \begin{enumerate}
      \item If $z = 0$, then $x = y = z = 0$ and $x + y + z = 0 \not\equiv 7$. 
      \item If $z = 1$, then we see 
      \begin{align}
        0 + 0 + 1 \equiv 1 \\ 
        0 + 1 + 1 \equiv 2 \\ 
        1 + 0 + 1 \equiv 2 \\ 
        1 + 1 + 1 \equiv 3 
      \end{align}
      \item If $z = 4$, then we see that 
        \begin{align}
          0 + 0 + 4 & \equiv 4 \\
          0 + 1 + 4 & \equiv 5 \\
          0 + 4 + 4 & \equiv 0 \\
          1 + 1 + 4 & \equiv 6 \\
          1 + 4 + 4 & \equiv 1 \\
          4 + 4 + 4 & \equiv 4
        \end{align}
    \end{enumerate}
    And so $a^2 + b^2 + c^2 \not\equiv 7 \pmod{8}$ for any $a, b, c \in \mathbb{Z}$. 
  \end{exercise}

  \begin{exercise}[Shifrin 1.3.20.a/b/g]
    For (a), 
    \begin{equation}
      3x \equiv 2 \pmod{5} \implies 6x \equiv 4 \pmod{5} \implies x \equiv 4 \pmod{5} 
    \end{equation}
    For (b), 
    \begin{align}
      6x + 3 \equiv 1 \pmod{10} & \implies 6x \equiv -2 \equiv 8 \pmod{10} \\
                                & \implies 10 | (6x - 8) \\
                                & \implies 5 | (3x - 4) \\
                                & \implies 3x \equiv 4 \pmod{5} \\
                                & \implies 3x \equiv 9 \pmod{5} \\
                                & \implies x \equiv 3 \pmod{5}
    \end{align}
    For (g), 
    \begin{align}
      15x \equiv 25 \pmod{35} & \implies 35 | (15x - 25) \\
                              & \implies 7 | (3x - 5) \\
                              & \implies 3x \equiv 5 \pmod{7} \\
                              & \implies 3x \equiv 12 \pmod{7} \\ 
                              & \implies x \equiv 4 \pmod{7}
    \end{align}
  \end{exercise}

  \begin{exercise}[Shifrin 1.3.21.b/c]
    For (b), we see that $4$ and $13$ are coprime with $-3 \cdot 4 + 1 \cdot 13 = 1$. Therefore, by the Chinese remainder theorem 
    \begin{equation}
      x \equiv 1 \cdot 1 \cdot 12 + (-3) \cdot 7 \cdot 4 \pmod{52} \implies x \equiv 33 \pmod{52}
    \end{equation}
    For (c), we solve the first two congruences $x \equiv 3 \pmod{4}$ and $x \equiv 4 \pmod{5}$. $4$ and $5$ are coprime with $-1 \cdot 4 + 1 \cdot 5 = 1$. Therefore, by CRT 
    \begin{equation}
      x \equiv -1 \cdot 4 \cdot 4 + 1 \cdot 5 \cdot 3 \pmod{20} \implies x \equiv -1 \pmod{20}
    \end{equation}
    Then we solve $x \equiv -1 \pmod{20}$ with the final congruence $x \equiv 3 \pmod{7}$. We see that $20$ and $7$ are coprime with $-1 \cdot 20 + 3 \cdot 7 = 1$. Therefore by CRT 
    \begin{equation}
      x \equiv -1 \cdot 20 \cdot 3 + 3 \cdot 7 \cdot -1 \pmod{140} \implies x \equiv 59 \pmod{140}
    \end{equation}
  \end{exercise}

  \begin{exercise}[Shifrin 1.3.25]
    We prove bidirectionally. 
    \begin{enumerate}
      \item Assume a solution exists for $cx \equiv b \pmod{m}$. Then $m | (cx - b)$, which means that there exists a $y \in \mathbb{Z}$ s.t. $my = cx - b \iff b = cx - my$. Since $d = \mathrm{gcd}(c, m)$, there exists $c^\prime, m^\prime \in \mathbb{Z}$ s.t. $c = d c^\prime$ and $m = d m^\prime$. So 
      \begin{equation}
        b = cx - my = d (c^\prime x - m^\prime y) \implies d | b
      \end{equation} 

    \item Assume that $d | b$. Then there exists a $b^\prime \in \mathbb{Z}$ s.t. $b = d b^\prime$, and we have 
    \begin{align}
      cx \equiv b \pmod{m} & \iff m | (cx - b) \\
                           & \iff d m^\prime | d (c^\prime x - b^\prime) \\
                           & \iff m^\prime | (c^\prime x - b^\prime) \\
                           & \iff c^\prime x \equiv b^\prime \pmod{m^\prime} 
    \end{align}
    Since $\mathrm{gcd}(c^\prime, m^\prime) = 1$\footnote{Since $\mathrm{gcd}(c, m) = d \implies$ that there exists a $y, z \in \mathbb{Z}$ s.t. $c y + m z = d$, and dividing both sides by $d$ guarantees the existence of $y, z$ satisfying $c^\prime y + m^\prime z = 1$, meaning that $\mathrm{gcd}(c^\prime, m^\prime) = 1$.}, by Shifrin Proposition 3.5 the equation $c^\prime x \equiv b^\prime \pmod{m^\prime}$ is guaranteed to have a solution, and working backwards in the iff statements gives us the solution for $cx \equiv b \pmod{m}$. 
    \end{enumerate}

    We have proved existence of a solution in $\bmod{(m/d) = m^\prime}$. Now we show uniqueness. Assume that there are two solutions $x \equiv \alpha$, $x \equiv \beta \pmod{m^\prime}$ with $\alpha \not\equiv \beta \pmod{m^\prime}$. Then, $x$ can be written as $x = k_\alpha m^\prime + \alpha$ and $x = k_\beta m^\prime + \beta$. But we see that 
    \begin{align}
      0 = x - x & = (k_\alpha m^\prime + \alpha) - (k_\beta m^\prime + \beta) \\
                & = m^\prime (k_\alpha - k_\beta) + (\alpha - \beta) \\
                & \equiv \alpha - \beta \pmod{m^\prime}
    \end{align}
    which implies that $\alpha \equiv \beta \pmod{m^\prime}$, contradicting our assumption that they are different in modulo. Therefore the solution must be unique. 
  \end{exercise}

  \begin{exercise}[Shifrin 1.4.1]
    For $\mathbb{Z}_7$. There are no zero divisors and the units are all elements. 
    \begin{equation}
      \begin{array}{c|ccccccc}
        \times & 0 & 1 & 2 & 3 & 4 & 5 & 6 \\
        \hline
        0 & 0 & 0 & 0 & 0 & 0 & 0 & 0 \\
        1 & 0 & 1 & 2 & 3 & 4 & 5 & 6 \\
        2 & 0 & 2 & 4 & 6 & 1 & 3 & 5 \\
        3 & 0 & 3 & 6 & 2 & 5 & 1 & 4 \\
        4 & 0 & 4 & 1 & 5 & 2 & 6 & 3 \\
        5 & 0 & 5 & 3 & 1 & 6 & 4 & 2 \\
        6 & 0 & 6 & 5 & 4 & 3 & 2 & 1
      \end{array}
    \end{equation}
    For $\mathbb{Z}_8$. The zero divisors are $2, 4, 6$. The units are $1, 3, 5, 7$. 
    \begin{equation}
      \begin{array}{c|cccccccc}
        \times & 0 & 1 & 2 & 3 & 4 & 5 & 6 & 7 \\
        \hline
        0 & 0 & 0 & 0 & 0 & 0 & 0 & 0 & 0 \\
        1 & 0 & 1 & 2 & 3 & 4 & 5 & 6 & 7 \\
        2 & 0 & 2 & 4 & 6 & 0 & 2 & 4 & 6 \\
        3 & 0 & 3 & 6 & 1 & 4 & 7 & 2 & 5 \\
        4 & 0 & 4 & 0 & 4 & 0 & 4 & 0 & 4 \\
        5 & 0 & 5 & 2 & 7 & 4 & 1 & 6 & 3 \\
        6 & 0 & 6 & 4 & 2 & 0 & 6 & 4 & 2 \\
        7 & 0 & 7 & 6 & 5 & 4 & 3 & 2 & 1
      \end{array} 
    \end{equation}
    For $\mathbb{Z}_{12}$. The zero divisors are $2, 3, 4, 6, 8, 9, 10$. The units are $1, 5, 7, 11$. 
    \begin{equation}
      \begin{array}{c|cccccccccccc}
        \times & 0 & 1 & 2 & 3 & 4 & 5 & 6 & 7 & 8 & 9 & 10 & 11 \\
        \hline
        0 & 0 & 0 & 0 & 0 & 0 & 0 & 0 & 0 & 0 & 0 & 0 & 0 \\
        1 & 0 & 1 & 2 & 3 & 4 & 5 & 6 & 7 & 8 & 9 & 10 & 11 \\
        2 & 0 & 2 & 4 & 6 & 8 & 10 & 0 & 2 & 4 & 6 & 8 & 10 \\
        3 & 0 & 3 & 6 & 9 & 0 & 3 & 6 & 9 & 0 & 3 & 6 & 9 \\
        4 & 0 & 4 & 8 & 0 & 4 & 8 & 0 & 4 & 8 & 0 & 4 & 8 \\
        5 & 0 & 5 & 10 & 3 & 8 & 1 & 6 & 11 & 4 & 9 & 2 & 7 \\
        6 & 0 & 6 & 0 & 6 & 0 & 6 & 0 & 6 & 0 & 6 & 0 & 6 \\
        7 & 0 & 7 & 2 & 9 & 4 & 11 & 6 & 1 & 8 & 3 & 10 & 5 \\
        8 & 0 & 8 & 4 & 0 & 8 & 4 & 0 & 8 & 4 & 0 & 8 & 4 \\
        9 & 0 & 9 & 6 & 3 & 0 & 9 & 6 & 3 & 0 & 9 & 6 & 3 \\
        10 & 0 & 10 & 8 & 6 & 4 & 2 & 0 & 10 & 8 & 6 & 4 & 2 \\
        11 & 0 & 11 & 10 & 9 & 8 & 7 & 6 & 5 & 4 & 3 & 2 & 1
      \end{array} 
    \end{equation}
  \end{exercise}

  \begin{exercise}[Shifrin 1.4.5.a/b/c]
    \begin{enumerate}
      \item Prove that $\gcd(a, m) = 1 \iff \bar{a} \in \mathbb{Z}_m$ is a unit.
      \item Prove that if $\bar{a} \in \mathbb{Z}_m$ is a zero-divisor, then $\gcd(a, m) > 1$, and conversely, provided $m \nmid a$.
      \item Prove that every nonzero element of $\mathbb{Z}_m$ is either a unit or a zero-divisor.
      \item Prove that in any commutative ring $R$, a zero-divisor cannot be a unit, and a unit cannot be a zero-divisor. Do you think c.\ holds in general?
    \end{enumerate}
  \end{exercise}
  \begin{solution}
    For (a), 
    \begin{enumerate}
      \item $(\rightarrow)$. If $\mathrm{gcd}(a, m) = 1$, then there exists $x, y \in \mathbb{Z}$ such that $ax + my = 1$. Taking the modulo on both sides gives $ax \equiv 1 \pmod{m}$, and therefore we have established the existence of $x \in \mathbb{Z}$, which implies the existence of $\bar{x} \in \mathbb{Z}_m$. 

      \item $(\leftarrow)$. If we have $a \in \mathbb{Z}$ and $\bar{a}$ is a unit, then there exists a $\bar{x} \in \mathbb{Z}_m$ s.t. $\bar{a} \bar{x} = \bar{1} \iff ax \equiv 1 \pmod{m}$, which means that $m | (1 - ax)$. So there exists an integer $y \in \mathbb{Z}$ s.t. $my = 1 - ax \iff ax + my = 1$. By Shifrin corollary 2.4 $a, m$ must be coprime. 
    \end{enumerate}

    For (b), 
    \begin{enumerate}
      \item ($\rightarrow$) Let $\bar{a} \in \mathbb{Z}_m$ be a zero-divisor. Then there exists $\bar{x} \neq \bar{0}$ in $\mathbb{Z}_m$ such that $\bar{a}\bar{x} = \bar{0}$. This means: $ax \equiv 0 \pmod{m}$, so $m \mid ax$, and  $m \nmid x$ (since $\bar{x} \neq \bar{0}$). Since $m \mid ax$ but $m \nmid x$, some prime factor of $m$ must divide $a$. This prime factor is then a common divisor of $a$ and $m$ greater than 1, so $\gcd(a,m) > 1$.

      \item ($\leftarrow$) Let $a \in \mathbb{Z}$, $m \in \mathbb{N}$ where $\gcd(a, m) = d > 1$ and $m \nmid a$. Then $a = a'd$ and $m = m'd$ for some $a', m' \in \mathbb{Z}$. Therefore, 
      \begin{equation}
        \bar{a} \bar{m'} = \overline{am'} = \overline{a'd m'} = \overline{a'm} = \bar{0}
      \end{equation}
      Also since $m \nmid a$, we have $\bar{a} \neq \bar{0}$, and since $m = m'd$, we have $m \nmid m'$ (since $m \nmid a \implies d \neq m$), so $\bar{m'} \neq \bar{0}$. Therefore $\bar{a}$ is a zero-divisor in $\mathbb{Z}_m$.
    \end{enumerate}

    For (c), let $a \in \mathbb{Z}_m$ be a nonzero element. Then it must be the case that $\mathrm{gcd}(a, m) = 1$ or $\mathrm{gcd}(a, m)  > 1$. In the former case, $a$ is a unit by (a), and in the latter case, $a \not\equiv 0 \implies m \nmid a$\footnote{By contrapositive $m \mid a \implies a \equiv 0 \pmod{m}$ is trivial.}, and so by (b) $a$ is a zero divisor. 
  \end{solution}

  \begin{exercise}[Shifrin 1.4.6.b/c/d]
    Prove that in any ring $R$:
    \begin{enumerate}
      \item $0 \cdot a = 0$ for all $a \in R$ (cf.\ Lemma 1.1);
      \item $(-1)a = -a$ for all $a \in R$ (cf.\ Lemma 1.2);
      \item $(-a)(-b) = ab$ for all $a,b \in R$;
      \item the multiplicative identity $1 \in R$ is unique.
    \end{enumerate}
  \end{exercise}
  \begin{solution} 
    For (a), note that $0 a = (0 + 0) \cdot a = 0a + 0a$ and by subtracting $0a$ from both sides, we have $0 = 0a$. Similarly, $a0 = a (0 + 0) = a0 + a0 \implies 0 = a0$. 
    For (b), 
    \begin{align}
      a + (-1) \cdot a & = 1 \cdot a + (-1) \cdot a && \tag{definition of $1$} \\
                       & = (1 + -1) \cdot a && \tag{left distributivity} \\
                       & = 0 \cdot a && \tag{definition of add inverse}\\
                       & = 0 && \tag{From (a)}
    \end{align}
    For (c), note that by right distributivity, 
    \begin{align}
      (-1) \cdot a + a & = (-1) \cdot a + 1 \cdot a && \tag{definition of $1$} \\
                       & = (-1 + 1) \cdot a && \tag{right distributivity} \\
                       & = a \cdot 0 && \tag{definition of add inverse}\\
                       & = 0 && \tag{From (a)}
    \end{align}
    Therefore, 
    \begin{align}
      (-a)(-b) & = (-1 \cdot a) (-1 \cdot b) && \tag{from (b)}\\
               & = -1 \cdot (a \cdot -1) \cdot b && \tag{associativity} \\
               & = -1 \cdot -a \cdot b && \tag{from (b)} \\
               & = -1 \cdot -1 \cdot a \cdot b && \tag{from (b)} \\
               & = (-1 \cdot -1) \cdot ab && \tag{associativity} \\
               & = 1ab && \tag{shown below}\\
               & = ab && \tag{definition of identity}
    \end{align} 
    where $(-1)(-1) = 1$ since by (b), $(-1)(-1) = -(-1)$. We know that $-(-1)$ is an additive inverse for $-1$ and so is $1$. Since the multiplicative identity is unique in a ring, $-(-1) = 1$.  We show uniqueness for (d). Let us have $1 \neq 1^\prime$. Then by definition of identity, 
    \begin{equation}
      1 = 1 1^\prime = 1^\prime 1 = 1^\prime
    \end{equation}
    which is a contradiction. 
  \end{solution}

  \begin{exercise}[Shifrin 1.4.10]
    \begin{enumerate}
      \item Prove that the multiplicative inverse of a unit $a$ in a ring $R$ is unique. That is, if $ab = ba = 1$ and $ac = ca = 1$, then $b = c$. (You will need to use associativity of multiplication in $R$.)
      
      \item Indeed, more is true. If $a \in R$ and there exist $b,c \in R$ so that $ab = 1$ and $ca = 1$, prove that $b = c$ and thus that $a$ is a unit.
    \end{enumerate}
  \end{exercise}
  \begin{solution}
    For (a), we see that 
    \begin{equation}
      c = 1c = (ab)c = (ba)c = b(ac) = b(ca) = b1 = b
    \end{equation} 
    For (b), we have  
    \begin{equation}
      b = 1b = (ca)b = c(ab) = c1 = c
    \end{equation}
  \end{solution}

  \begin{exercise}[Shifrin 1.4.13]
    Let $p$ be a prime number. Use the fact that $\mathbb{Z}_p$ is a field to prove that $(p-1)! \equiv -1 \pmod{p}$. (Hint: Pair elements of $\mathbb{Z}_p$ with their multiplicative inverses; cf. Exercise 1.3.12.). 
  \end{exercise}
  \begin{solution}
    For $p = 2$, the result is trivial. Now let $p > 2$ be a prime. Then since $\mathbb{F}$ is a field, every element $a \in \mathbb{F}$ contains a multiplicative inverse $a^{-1}$. We claim that the only values for which $a = a^{-1}$ is $1, p-1$. Assume that $a = a^{-1}$. Then 
    \begin{equation}
      a^2 = 1 \implies p|(a^2 - 1) \implies p | (a+1)(a-1)
    \end{equation}
    and since $p$ is prime, it must be the case that $p|a+1 \iff a \equiv -1 \pmod{p}$ or $p|a-1 \iff a \equiv 1 \pmod{p}$. Therefore, we are left to consider the $(p-3)$ elements: $2, \ldots, p-2$. Since inverses are unique and the inverses of inverses is the original element, we can partition these $p-2$ elements into $(p-3)/2$ pairs.\footnote{Since $p \neq 2$, $p$ is odd and therefore $p-3$ is even.} Let's call the set of pairs $K = \{(a, b)\}$ where $b = a^{-1}$. Therefore, by commutativity and associativity we have 
    \begin{equation}
      (p-1)! \equiv (1)(p-1) \prod_{(a, b) \in K} ab \equiv -1 \cdot \prod_{(a, b) \in K} 1 \equiv -1 \pmod{p}. 
    \end{equation}
  \end{solution} 

  \begin{exercise}[Shifrin 2.3.2.a/b/c]
    Recall that the conjugate of the complex number $z = a + bi$ is defined to be $\bar{z} = a - bi$. Prove the following properties of the conjugate:
    \begin{enumerate}
      \item $\overline{z + w} = \bar{z} + \bar{w}$
      \item $\overline{zw} = \bar{z}\bar{w}$
      \item $\bar{z} = z \iff z \in \mathbb{R}$ and $\bar{z} = -z \iff iz \in \mathbb{R}$
      \item If $z = r(\cos\theta + i\sin\theta)$, then $\bar{z} = r(\cos\theta - i\sin\theta)$
    \end{enumerate}
  \end{exercise}
  \begin{solution}
    Let $z = a + bi, w = c + di$. For (a), 
    \begin{equation}
      \overline{z + w} = \overline{(a + c) + (b + d)i} = (a + c) - (b + d)i = a + c - bi - di = (a - bi) + (c - di) = \overline{z} + \overline{w}
    \end{equation} 
    For (b), 
    \begin{equation}
      \overline{zw} = \overline{(ac - bd) + (ad + bc)i} = (ac - bd) - (ad + bc)i = ac - bd - adi - bci = (a - bi)(c - di) = \bar{z}\bar{w}
    \end{equation}
    For (c), consider 
    \begin{align}
      \overline{z} = z & \iff a + bi = a - bi \\
                       & \iff bi = -bi \\
                       & \iff 2bi = 0 \\
                       & \iff b = 0 && \tag{field has no 0 divisors}
    \end{align}
    Therefore, $z = a \in \mathbb{R}$. 
    \begin{align}
      \overline{z} = -z & \iff a - bi = -a - bi \\
                        & \iff a = -a \\
                        & \iff 2a = 0 \\
                        & \iff a = 0 && \tag{field has no 0 divisors.}
    \end{align}
    Therefore, $z = bi \implies iz = -b \in \mathbb{R}$. 
  \end{solution}

  \begin{exercise}[Shifrin 2.3.3.a/b/c]
    Recall that the modulus of the complex number $z = a + bi$ is defined to be $|z| = \sqrt{a^2 + b^2}$. Prove the following properties of the modulus:
    \begin{enumerate}
      \item $|zw| = |z||w|$
      \item $|\bar{z}| = |z|$
      \item $|z|^2 = z\bar{z}$
      \item $|z + w| \leq |z| + |w|$ (This is called the triangle inequality; why?)
    \end{enumerate}
  \end{exercise}
  \begin{solution}
    Let $z = a + bi$ and $w = c + di$. For (a),
    \begin{align*}
      |zw| &= |(ac - bd) + (ad + bc)i| \\
      &= \sqrt{(ac - bd)^2 + (ad + bc)^2} \\
      &= \sqrt{a^2c^2 - 2abcd + b^2d^2 + a^2d^2 + 2abcd + b^2c^2} \\
      &= \sqrt{(a^2 + b^2)(c^2 + d^2)} \\
      &= \sqrt{a^2 + b^2}\sqrt{c^2 + d^2} \\
      &= |z||w|
    \end{align*}

    For (b), if $z = a + bi$, then $\bar{z} = a - bi$, so:
    \begin{equation}
      |\bar{z}| = \sqrt{a^2 + (-b)^2} = \sqrt{a^2 + b^2} = |z|
    \end{equation}

    For (c),
    \begin{align*}
      z\bar{z} &= (a + bi)(a - bi) \\
      &= a^2 + b^2 \\
      &= |z|^2
    \end{align*}
  \end{solution}


With set theory, we have established what sets, along with functions and relations are. Abstract algebra extends on this by studying \textit{algebraic structures}, which are sets $S$ with specific \textit{operations} acting on their elements. This is a very natural extension and to be honest does not require much motivation. Let's precisely define what operations are. 

\begin{definition}[Operation]
  A \textbf{p-ary operation}\footnote{or called an operation of arity $p$.} $\ast$ on a set $A$ is a map 
  \begin{equation}
    \ast : A^p \longrightarrow A
  \end{equation} 
  where $A^p$ is the $p$-fold Cartesian product of $A$. In specific cases, 
  \begin{enumerate}
    \item If $p = 1$, then $\ast$ is said to be \textbf{unary}. 
    \item If $p = 2$, then $\ast$ is \textbf{binary}. 
  \end{enumerate}
  We can consider for $p > 2$ and even if $p$ is infinite.  
\end{definition}

\begin{definition}[Algebraic Structure]
  An \textbf{algebraic structure} is a nonempty set $A$ with a finite set of operations $\ast_1, \ldots, \ast_n$ and satisfying a finite set of axioms. It is written as $(A, \ast_1, \ldots, \ast_n)$. 
\end{definition} 

If we consider functions between algebraic structures $f: A \rightarrow B$, there are some natural properties that we would like $f$ to have. 

\begin{definition}[Preservation of Operation]
  Given algebraic structures $(A, \mu_A)$, $(B, \mu_B)$, where $\mu_A$ and $\mu_B$ have the same arity $p$, a function $f: A \rightarrow B$ is said to \textbf{preserve the operation} if for all $x_1, \ldots, x_p \in A$, 
  \begin{equation}
    f(\mu_A(x_1, \ldots, x_p)) = \mu_B (f(x_1), f(x_2), \ldots, f(x_p))
  \end{equation}
\end{definition} 

Functions that preserve operations are generally called \textit{homomorphisms}. However, given that preservation is defined with respect to each operation, a map may preserve one operation but not the other. Therefore, we will formally define homomorphisms for each class of algebraic structures we encounter. 


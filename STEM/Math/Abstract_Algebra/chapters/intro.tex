With set theory, we have established what sets, along with functions and relations are. Abstract algebra extends on this by studying \textit{algebraic structures}, which are sets $S$ with specific \textit{operations} acting on their elements. This is a very natural extension and to be honest does not require much motivation. Let's precisely define what operations are. 

\begin{definition}[Operation]
  A \textbf{p-ary operation}\footnote{or called an operation of arity $p$.} $\ast$ on a set $A$ is a map 
  \begin{equation}
    \ast : A^p \longrightarrow A
  \end{equation} 
  where $A^p$ is the $p$-fold Cartesian product of $A$. In specific cases, 
  \begin{enumerate}
    \item If $p = 1$, then $\ast$ is said to be \textbf{unary}. 
    \item If $p = 2$, then $\ast$ is \textbf{binary}. 
  \end{enumerate}
  We can consider for $p > 2$ and even if $p$ is infinite.  
\end{definition}

\begin{definition}[Algebraic Structure]
  An \textbf{algebraic structure} is a nonempty set $A$ with a finite set of operations $\ast_1, \ldots, \ast_n$ and satisfying a finite set of axioms. It is written as $(A, \ast_1, \ldots, \ast_n)$. 
\end{definition} 

If we consider functions between algebraic structures $f: A \rightarrow B$, there are some natural properties that we would like $f$ to have. 

\begin{definition}[Preservation of Operation]
  Given algebraic structures $(A, \mu_A)$, $(B, \mu_B)$, where $\mu_A$ and $\mu_B$ have the same arity $p$, a function $f: A \rightarrow B$ is said to \textbf{preserve the operation} if for all $x_1, \ldots, x_p \in A$, 
  \begin{equation}
    f(\mu_A(x_1, \ldots, x_p)) = \mu_B (f(x_1), f(x_2), \ldots, f(x_p))
  \end{equation}
\end{definition} 

Functions that preserve operations are generally called \textit{homomorphisms}. However, given that preservation is defined with respect to each operation, a map may preserve one operation but not the other. Therefore, we will formally define homomorphisms for each class of algebraic structures we encounter. 

\begin{definition}[Commutative, Associative Operations]
  A binary operation $\cdot : A \times A \to A$ is said to be 
  \begin{enumerate}
    \item \textbf{associative} if for all $a, b, c \in A$, $(ab)c = a(bc)$. 
    \item \textbf{commutative} if for all $a, b \in A$, $ab = ba$. 
  \end{enumerate}
\end{definition}

Associativity is a particularly important property that we would like to have, and it is quite rare to work with algebraic structures that don't have associativity. It basically states that when doing an operation sequentially over 3 elements, it doesn't matter if we evaluate $ab$ or $bc$ first. Therefore, associativity allows us to throw the parentheses away since the evaluated result does not change. 

Commutativity on the other hand is not as prevalent. It simply tells us that we can ``swap'' terms when evaluating. This usually is a another nice convenience, and in the theory of rings commutativity is very prevalent. Either way, in both of these scenarios we can extend to any finite sequence of operations.  

\begin{theorem}[Generalized Associativity]
  Given that a binary operation $\cdot$ is associative on a set $S$, it is always the case that for any finite collection $a_1, \ldots, a_n$, the value $a_1 \ldots a_n$ is unique. 
\end{theorem}
\begin{proof}
  We prove by strong induction on $n$ from $n = 3$. Clearly $(a_1 a_2) a_3 = a_1 (a_2 a_3)$ by definition of associativity. The rest is a bit tedious but is mentioned in Jacobson's \textit{Basic Algebra 1}. 
\end{proof}

\begin{theorem}[Generalized Commutativity]
  Given that a binary operation $\cdot$ is commutative and associative on a set $S$, with $\alpha = a_1 + \ldots + a_n$, we have 
  \begin{equation}
    \alpha = a_{i_1} + \ldots + a_{i_n}
  \end{equation}
  for any permutation $(i_1, \ldots, i_n)$ of $(1, \ldots, n)$. 
\end{theorem}

Now that we've gotten these out of the way, we can start talking about algebraic structures. I've went through 4 main textbooks, plus Google and talking to friends/professors in creating these notes. 
\begin{enumerate}
  \item Vinberg's \textit{A Course in Algebra}. 
  \item Nathan Jacobson's \textit{Basic Algebra 1}, given to me by Marty. 
  \item Ted Shifrin's \textit{Abstract Algebra, A Geometric Approach}, used in Duke Math 401. 
  \item Dummit and Foote's \textit{Abstract Algebra, 3rd Edition}, used in Duke Math 501. 
\end{enumerate}



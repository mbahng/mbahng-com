\section{Modules} 

  Now we give a generalization of vector spaces in which the underlying field is replaced by a general ring. 

  \begin{definition}[Module]
    Given a ring $R$, a \textbf{left $R$-module} $M$ is an abelian group $(M, +)$ and an operation $\cdot: R \times M \to M$---called \textit{scalar multiplication}---such that for all $r, s \in R$ and $x, y \in M$, we have 
    \begin{enumerate}
      \item $r (x + y) = rx + ry$. 
      \item $(r + s) x = rx + sx$. 
      \item $(rs) \cdot x = r(sx)$ 
      \item $1 \cdot x = x$. 
    \end{enumerate}
    Note that the ``left'' refers to the ring elements appearing on the left $R \times M$, and the analogous definition for right modules can be made. 
  \end{definition}

  Before going into any examples, we introduce submodules, which are subsets of modules $M$ that are themselves modules under the restricted operations. 

  \begin{definition}[Submodule]
    Given a $R$-module $M$, a \textbf{submodule} $N$ of $M$ is a subgroup which is closed under the action of ring elements, i.e. $rn \in N$ for all $r \in R, n \in N$. 
  \end{definition}

  In particular, if $R = F$ is a field, then modules and submodules are the same as vector spaces and subspaces---though we haven't formally defined them yet. 

  \begin{example}[Modules]
    Let $R$ be a ring. 
    \begin{enumerate}
      \item $R$ is a submodule, where scalar multiplication $\cdot : R \times R$ is the same as the ring multiplication in $R$. 
      \item Given $n \in \mathbb{N}$, the set 
        \begin{equation}
          R^n \coloneqq \{(a_1, \ldots, a_n) \mid \forall i, a_i \in R \} 
        \end{equation}
        is an $R$-module with addition and scalar multiplication defined component-wise. This is called the \textbf{free-module of rank $n$ over $R$}. 
    \end{enumerate}
  \end{example}

  \begin{theorem}[Submodule Criterion]
    Let $R$ be a ring and $M$ an $R$-module. A subset $N \subset M$ is a submodule of $M$ if and only if 
    \begin{enumerate}
      \item $N \neq \emptyset$, and 
      \item $x + ry \in N$ for all $r \in R$ and $x, y \in N$. 
    \end{enumerate}
  \end{theorem}

\subsection{Modules over a PID} 

\subsection{Rational Canonical Form}

\subsection{Jordan Canonical Form}


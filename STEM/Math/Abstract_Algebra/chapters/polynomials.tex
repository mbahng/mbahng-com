\section{Polynomial Rings} 

  One of the most widely studied rings are the ring of polynomials. Let's reintroduce them. 

  \begin{definition}[Univariate Polynomials]
    For a ring $R$, the \textbf{univariate polynomial ring over $R$}, denoted $R[x]$ consists of elements called \textbf{polynomials} which are formal expressions of the form 
    \begin{equation}
      f(x) = a_nx^n + a_{n-1}x^{n-1} + \dots + a_1x + a_0 \text{ where } a_i \in R
    \end{equation}
    with coefficients $a_i \in R$ and $x$ is called a \textbf{variable}, or \textbf{indeterminant}.\footnote{Note that $x$ is just a formal symbol, whose powers $x^i$ are just placeholders for the corresponding coefficients $a_i$ so that the given formal expression is a way to encode the finitary sequence. $(a_0, a_1, a_2, ..., a_n)$.} Two polynomials are equal if and only if the sequences of their corresponding coefficients are equal. We can also see a polynomial as a function $f: R \rightarrow R$ as well. 

    Furthermore, $R[x]$ is a ring, with addition and multiplication defined
    \begin{equation}
      a_i x^i + b_i x^i = (a_i + b_i) x^i, \qquad x^ix^j = x^{i+j}
    \end{equation}
    along with $0$ as the additive identity and $1$ as the multiplicative identity.   
  \end{definition}

  While we will mainly deal with univariate polynomials, we can also define multivariate polynomials similarly. 

  \begin{definition}[Multivariate Polynomials] 
    For a ring $R$, the \textbf{multivariate polynomial ring over $R$}, denoted $R[x_1, \ldots, x_n]$ consists of elements called \textbf{polynomials} which are formal expressions of the form 
    \begin{equation}
      f(x_1, \ldots, x_n) = \sum_{0 \leq k_i \leq n} a_{k_1 \ldots k_n} x_1^{k_1} x_2^{k_2} \ldots x_n^{k_n}
    \end{equation}
    with coefficients $a \in R$ and $x_i$'s the \textbf{variables}. We can treat an element $f \in R[x_1, \ldots, x_n]$ as a function $f: R^n \rightarrow R$.  

    Furthermore, $R[x_1, \ldots, x_n]$ is a ring, with addition and multiplication defined 
    \begin{align}
      a_{k_1 \ldots k_n} x_1^{k_1} x_2^{k_2} \ldots x_n^{k_n} + b_{k_1 \ldots k_n} x_1^{k_1} x_2^{k_2} \ldots x_n^{k_n} & = (a_{k_1 \ldots k_n} + b_{k_1 \ldots k_n}) x_1^{k_1} x_2^{k_2} \ldots x_n^{k_n} \\
      x^{k_1 \ldots k_n} x^{l_1 \ldots l_n} & = x^{k_1 + l_1, k_2 + l_2, \ldots, k_n + l_n}
    \end{align}
  \end{definition}

  Usually, the properties of the base ring $R$ determines the properties on $R$. 

  \begin{lemma}[Commutativity Extends to Polynomials]
    We have the following. 
    \begin{enumerate}
      \item $R$ is a commutative ring iff $R[x]$ is a commutative ring. 
      \item $R$ is an integral domain iff $R[x]$ is an integral domain. 
      \item $F$ is a field iff $F[x]$ is a Euclidean domain.  
    \end{enumerate}
  \end{lemma}
  \begin{proof}
    TBD. 
  \end{proof} 

  With this theorem we unlock all the properties that we have studied in general for the subclasses of rings. Almost always we will assume that $R$ is at least commutative, so let's get that out of the way. Before we move on, let's get some terms out of the way. 

  \begin{definition}[Some Terms for Polynomials]
    Given a univariate polynomial $f(x)$. 
    \begin{enumerate} 
      \item The \textbf{leading coefficient} is the last nonzero coefficient  
      \item The \textbf{degree} of $f$---denoted $\deg f$---is the index of the leading coefficient.
      \item A \textbf{monomial} is a polynomial of a single term $a_j x^j$. 
      \item A \textbf{linear} polynomial is a polynomial of degree 1. 
      \item A \textbf{quadratic} polynomial is a polynomial of degree 2. 
      \item A \textbf{cubic} polynomial is a polynomial of degree 3. 
    \end{enumerate}
  \end{definition}

\subsection{Basic Properties of Polynomials}

  We need to be very careful about the properties that hold for polynomials, as they may not be intuitive. For example, for certain finite fields (which are rings), some formally different polynomials may be indistinguishable in terms of mappings.\footnote{$x$ and $x^2$ are equivalent in the polynomial algebra defined on the domain $\mathbb{Z}_2$. } Second, a polynomial may have more roots than its degree. Therefore, we will work in different rings $R$ and provide conditions where our intuition is true in $R[x]$. It is clear that if you have two polynomials of degree $n$ and $m$, their sum may be degree $k < n, m$. This is not always true for multiplication. 

  \begin{example}[Product of Two Linear Polynomials is $0$]
    Given $f, g \in \mathbb{Z}_6 [x]$ with $f(x) = 2x + 4$ and $g(x) = 3x + 3$, we have 
    \begin{equation}
      f(x) \cdot g(x) = (2x + 4)(3x + 3) = 6x^2 + 18 x + 12 = 0
    \end{equation}
  \end{example}

  There is a simple condition in which the degree is additive, however. 

  \begin{theorem}[Bounds on Degrees From Operations]
    Given that $R$ is a ring and $f, g \in R[x]$, 
    \begin{equation}
      \deg(f+g) \leq \max\{\deg f, \deg g\} \\
    \end{equation}
    If $R$ is a domain, then 
    \begin{equation}
      \deg (f g) = \deg f + \deg g
    \end{equation}
    Note that this automatically implies that $R[x]$ is a domain. Combined with the lemma above, we have: $R$ is an integral domain $\implies R[x]$ is an integral domain. 
  \end{theorem}
  \begin{proof}
    The second may not be true if $R$ has zero divisors. 
  \end{proof}

  Just working in domains do not make things all better. Sometimes, we may have two different polynomials but they may define the same function from $R$ to $R$! 

  \begin{example}[Polynomials as Same Function]
    Given $f, g \in \mathbb{Z}_2 [x]$, 
    \begin{equation}
      f(x) = x \sim g(x) = x^2
    \end{equation} 
  \end{example}

  As shown in the example above, it is not so simple as to restrict which underlying set you are working on. Some rings $R$ may or may not assert uniqueness of functions in $F[x]$, and vice versa. Therefore, here are some special theorems. 

  \begin{theorem}[Uniqueness of Polynomials over Field]
    If the field $\mathbb{F}$ is infinite, then different polynomials in $\mathbb{F}[x]$ determine different functions. 
  \end{theorem}

\subsubsection{Euclidean Division} 

  Just like how we can do Euclidean division with integers, there is an analogous result for polynomials. However, we require to work with a \textit{field} $F$ rather than an arbitrary ring $R$. 

  \begin{theorem}[Polynomials as Euclidean Domain]
    Given a field $F$, $F[x]$ is a Euclidean domain. That is, given polynomials $f(x), g(x) \in F[x]$, there are unique polyomials $q(x), r(x) \in F[x]$ s.t. 
    \begin{equation}
      f(x) = q(x) g(x) + r(x), \qquad \deg(r(x)) < \deg(g(x))
    \end{equation} 
  \end{theorem} 
  \begin{proof}
    We first prove existence. If $\deg(f(x)) < \deg(g(x))$, then we can trivially set $q(x) = 0, r(x) = f(x)$. Therefore we can assume that $\deg(f(x)) \geq \deg(g(x))$. We can prove this by strong induction on $k = \deg(f(x))$. Assume that $\deg(f(x)) = 1$. Then if $\deg(g(x)) > 1$ it is trivial as before, so we show for $\deg(g(x)) = 1$. So let 
    \begin{equation}
      f(x) = a_1 x + a_0, \qquad g(x) = b_1 x + b_0 
    \end{equation}
    and we can find the solutions 
    \begin{equation}
      f(x) = \frac{a_1}{b_1} g(x) + \bigg( a_0 - \frac{a_1 b_0}{b_1} \bigg)
    \end{equation} 
    Now suppose that the results is known for whenver $\deg(f(x)) \leq k$ and we have a polynomial $F(x) = a_{k+1} x^{k+1} + \ldots a_0$ of degree $k+1$. Then we must check that there exists a quotient and remainder for $0 \leq \deg(g(x)) = m \leq k + 1$. Note that the coefficients of $x^{k+1}$ in $F(x)$ and in the polynomial $\frac{a_{k+1}}{b_m} x^{k+1-m} g(x)$ are the same, so the polynomial 
    \begin{equation}
      f(x) = F(x) - \frac{a_{k+1}}{b_m} x^{k+1-m} g(x) 
    \end{equation}
    has degree at most $k$. Thus by our induction hypothesis we can write $f(x) = q(x) g(x) + r(x)$, and so 
    \begin{align}
      F(x) & = f(x) + \frac{a_{k+1}}{b_m} x^{k+1-m} g(x) \\
           & = q(x) g(x) + r(x) + \frac{a_{k+1}}{b_m} x^{k+1-m} g(x) \\ 
           & = \bigg( q(x) + \frac{a_{k+1}}{b_m} x^{k+1-m} \bigg) g(x) + r(x)
    \end{align} 
    which is indeed a decomposition. Now to prove uniqueness, suppose we had two different decompositions 
    \begin{equation}
      f(x) = q(x) g(x) + r(x) = q^\prime (x) g(x) + r^\prime (x) \implies \big( q(x) - q^\prime (x) \big) g(x) = r(x) - r^\prime (x)  
    \end{equation}  
    IF $q(x) \neq q^\prime (x)$, then the degree of the LHS is at least $\deg(g(x))$, while the degree of the RHS must be strictly less, a contradiction. 
  \end{proof}

  \begin{example}[Polynomials over Fields] 
    The algorithimc way to get such $q(x), r(x)$ is through \textit{polynomial long division}. 
    
    \begin{center}
      \polylongdiv{x^3 + 4x^2 - x + 7}{x - 2}
    \end{center}

    Given field $\mathbb{Z}_5$, $\mathbb{Z}_5[x]$ is a Euclidean domain, with Euclidean division.  
  \end{example} 

  In fact, it turns out that you don't necessarily require a polynomial to always come from a field in order to do long division. You can do polynomial long division over \textit{any} commutative rings, as long as the leading coefficient of the divisor is a unit (and since all elements of a field are units, we can do so). This is because at each step, you only need to divide the leading coefficient of the divisor into the leading coefficient of the polynomial you have left. An immediate consequence of this theorem is the following. 

  \begin{corollary}[Remainder Theorem]
    Let $c \in F$ and $f(x) \in F[x]$. When we divide $f(x)$ by $g(x) = x - c$, the remainder is $f(c)$. 
  \end{corollary}
  \begin{proof}
    By the Euclidean algorithm, 
    \begin{equation}
      f(x) = (x - c) q(x) + r(x) \implies f(c) = (c - c) q(c) + r(c) = r(c)
    \end{equation}
  \end{proof} 

\subsubsection{Roots and Factorization}

  Next, we can define the all too familiar factors and roots of a polynomial. 

  \begin{definition}[Factor]
    Given a ring $R$ and a polynomial $f(x) \in R[x]$, if there exists $g(x), h(x)$ of degree at least 1 such that 
    \begin{equation}
      f(x) = g(x) h(x)
    \end{equation}
    then $g, h$ are said to be \textbf{factors}, or \textbf{divisors}, of $f$. If there are no such factors of $f$, then $f(x)$ is said to be \textbf{irreducible}. 
  \end{definition}

  Irreducible polynomials are analogous to prime numbers in $\mathbb{Z}$. 

  \begin{definition}[Polynomial Root]
    An element $r \in R$ is a \textbf{root} of polynomial $f \in R[x]$ if and only if 
    \begin{equation}
      f(r) = 0
    \end{equation}
  \end{definition} 

  Note that both factors and roots are intimately tied to Euclidean division, so the two are closely related. 

  \begin{theorem}[Root-Factor Theorem]
    Given a commutative ring $R$ (usually $R$ is a field) and $f(x) \in R[x]$, $(x - c)$ is a factor of $f(x)$, i.e. can be factored into 
    \begin{equation}
      f(x) = (x - c) q(x) 
    \end{equation}
    for some $q(x) \in R[x]$ of degree $\deg(f) - 1$ if and only if $f(c) = 0$.\footnote{Note that this is not true for an arbitrary ring. $R$ must be commutative at least.}
  \end{theorem} 
  \begin{proof}
    We prove for when $R$ is a field $F$, but it turns out that the theorem also holds for commutative rings $R$. 
    \begin{enumerate}
      \item $(\rightarrow)$. Given that $(x - c)$ is a factor of $f(x)$, this means that by the Euclidean algorithm $f(x) = (x - c) q(x)$ for some $q(x)$, and so $f(c) = (c - c) q(c) = 0$. 
      \item $(\leftarrow)$. Given that $f(c) = 0$. By the remainder theorem this means that when we divide $f(x)$ by $(x - c)$, the remainder is $f(c) = 0$, and so $f(x) = (x - c) q(x) + 0 = (x - c) q(x) \implies (x - c)$ is a factor of $f(x)$. 
    \end{enumerate}
  \end{proof}

  Notice how these polynomials mimick integers, and to drive this point even further, let's introduce the greatest common divisor. 

  \begin{theorem}[GCD of Two Polynomials Exist]
    Given nonzero polynomials $f(x), g(x) \in F[x]$, let 
    \begin{equation}
      S = \{h(x) \in F[x] \mid h(x) = a(x) f(x) + b(x) g(x) \text{ for some } a(x), b(x) \in F[x] \}
    \end{equation} 
    Then there exists some polynomial $d(x) \in S$ of smallest degree, and every $h(x) \in S$ is divisible by $d(x)$. 
  \end{theorem}
  \begin{proof}
    The existence is trivial since by the well-ordering principle on the degrees of polynomials in $S$, such a minimal degree must exist. Now we prove the second claim by proving $d(x) \mid f(x)$. We apply the division algorithm to write 
    \begin{equation}
      f(x) = q(x) d(x) + r(x)
    \end{equation}
    If $r(x) = 0$, then by root factor theorem we are done. If $r(x) \neq 0$, we then write 
    \begin{align}
      r(x) & = f(x) - q(x) d(x) \\ 
           & = f(x) - \big( s(x) f(x) + t(x) g(x) \big) q(x) \\ 
           & = \big( 1 - s(x) q(x) \big) f(x) - \big( t(x) q(x) \big) g(x) \in S
    \end{align}
    Since $r(x) \in S$ due to its form, the fact that $\deg(r(x)) < \deg(d(x))$ contradicts the way that $d(x)$ was chosen. Therefore $r(x) = 0$. It turns out that $d(x)$ is unique up to a constant factor. 
  \end{proof}

  \begin{definition}[GCD]
    $d(x)$ as above is called the \textbf{greatest common divisor} of $f(x), g(x)$, denoted $d(x) = \gcd(f(x), g(x))$ satisfying 
    \begin{enumerate}
      \item $d(x) \mid f(x)$, $d(x) \mid g(x)$, and 
      \item $\forall e(x) \in F[x]$, if $e(x) \mid f(x)$ and $e(x) \mid g(x)$, then $e(x) \mid d(x)$. 
    \end{enumerate}
    $f(x), g(x)$ are said to be \textbf{relatively prime} if $\gcd(f(x), g(x)) = 1$. 
  \end{definition}

  The algorithmic way for computing the GCD is done the same way by performing Euclidean algorithm on two polynomials: dividing on by the other, taking the remainder, and dividing the lesser degree by the remainder again, until the remainder is $0$. 

  \begin{lemma}
    Suppose $f(x)$ is irreducible and $f(x) \mid g(x) h(x)$. Then $f(x) \mid g(x)$ or $f(x) \mid h(x)$. 
  \end{lemma}

  Now, we show an extremely important theorem. This should be intuitive since $F$ a field implies $F[x]$ a Euclidean domain, which is a PID, which has the unique factorization theorem. 

  \begin{theorem}[Unique Factorization of Polynomials over Fields]
    Given field $F$ and nonconstant polynomial $f(x) \in F[x]$ of degree $n$, we can always write $f(x)$ as a unique\footnote{up to constant factors and rearrangement} product of at most $n$ irreducible polynomials in $F[x]$. 
  \end{theorem}
  \begin{proof}
    To prove the bound, the general idea is that by the root factor theorem, each root gives rise to a linear factor, and so inductively we cannot have more than $n$ linear factors.  
    Strong induction on degree of $f(x)$ by starting with linear. 
  \end{proof}

  Note that this is \textit{not} true in arbitrary rings. 

  \begin{example}[Linear Polynomial with 3 Roots]
    Consider $f(x) = x^2 - 1 \in \mathbb{Z}_8 [x]$, a commutative ring. Then $1, 3, 5, 7$ are all roots of $f(x)$, which is greater than its degree. Furthermore, it has two different factorizations 
    \begin{equation}
      x^2 - 1 = (x + 1)(x - 1) = (x + 3)(x - 3)
    \end{equation}
  \end{example} 

  \begin{theorem}[Interpolation]
    For any collection of given field values $y_1, y_2, ..., y_n \in F$ at given distinct points $x_1, x_2, ..., x_n \in F$, there exists a unique polynomial $f \in F[x]$ with deg$\, f < n$ such that
    \begin{equation}
      f(x_i) = y_i, \quad i = 1, 2, ..., n
    \end{equation}
    This is commonly known as the \textbf{interpolation problem}, and when $n = 2$, this is called \textbf{linear interpolation}. 
  \end{theorem} 

\subsection{Exercises}

  \begin{exercise}[Shifrin 3.1.2.c/d]
    Find the greatest common divisors $d(x)$ of the following polynomials $f(x), g(x) \in F[x]$, and express $d(x)$ as $s(x)f(x) + t(x)g(x)$ for appropriate $s(x), t(x) \in F[x]$:
    \begin{enumerate}
      \item $f(x) = x^3 - 1$, $g(x) = x^4 + x^3 - x^2 - 2x - 2$, $F = \mathbb{Q}$
      \item $f(x) = x^2 + (1 - \sqrt{2})x - \sqrt{2}$, $g(x) = x^2 - 2$, $F = \mathbb{R}$
      \item $f(x) = x^2 + 1$, $g(x) = x^2 - i + 2$, $F = \mathbb{C}$
      \item $f(x) = x^2 + 2x + 2$, $g(x) = x^2 + 1$, $F = \mathbb{Q}$
      \item $f(x) = x^2 + 2x + 2$, $g(x) = x^2 + 1$, $F = \mathbb{C}$
    \end{enumerate}
  \end{exercise}
  \begin{solution}
    For (c), the gcd is $1$, with 
    \begin{equation} 
      -\frac{1}{1 - i} (x^2 + 1) + \frac{1}{1 - i} (x^2 - i + 2) = \frac{1}{1-i} (x^2 - i + 2 - x^2 - 1) = \frac{1}{1-i} (1 - i) = 1
    \end{equation}
    where $1/(1-i) = (1 + i)/2$. For (d), the gcd is $1$, with 
    \begin{align}
      \frac{1}{5} (2x + 3) (x^2 + 1) & + \frac{1}{5} (1 - 2x) (x^2 + 2x + 2) \\
                                          & = \frac{1}{5} (2x^3 + 3x^2 + 2x + 3) + \frac{1}{5} (-2x^3 - 3x^2 - 2x + 2) = 1
    \end{align}
  \end{solution}

  \begin{exercise}[Shifrin 3.1.6]
    Prove that if $F$ is a field, $f(x) \in F[x]$, and $\mathrm{deg}(f(x)) = n$, then $f(x)$ has at most $n$ roots in $F$. 
  \end{exercise}
  \begin{solution}
    We start when $n=1$. Then $f(x) = mx + b$ and we claim that the only root is $x = -b/m$ since we can solve for $0 = mx + b$ with the field operations, which leads to a unique solution. This implies by corr 1.5 that $(x + b/m)$ is the only factor of $f$. Now suppose this holds true for some degree $n-1$ and let us have a degree $n$ polynomial $f$. Assume that some $c$ is a root of $f$ (if there exists no $c$, then we are trivially done), which means $(x - c)$ is a factor of $f$, and we can write 
    \begin{equation}
      f(x) = (x - c) \, g(x)
    \end{equation}
    for some polynomial $g(x)$ of degree $n-1$. By our inductive hypothesis, $g(x)$ must have at most $n-1$ roots, and so $f$ has at most $n$ roots. 
  \end{solution}

  \begin{exercise}[Shifrin 3.1.8]
    Let $F$ be a field. Prove that if $f(x) \in F[x]$ is a polynomial of degree $2$ or $3$, then $f(x)$ is irreducible in $F[x]$ if and only if $f(x)$ has no root in $F$.
  \end{exercise}
  \begin{solution}
    We prove bidirectionally. 
    \begin{enumerate}
      \item $(\rightarrow)$. Let $f$ be irreducible. Then it cannot be factored into polynomials $p(x) q(x)$ where $\mathrm{deg}(p) + \mathrm{deg}(q) = n$. Note that two positive integers adding up to $2$ or $3$ means that at least one of the integers must be $1$, by the pigeonhole principle. This means that $f$ irreducible is equivalent to saying that $f$ does not have linear factors of form $(x-c)$, which by corollary 1.5 implies that there exists no root $c$ for $f(x)$. 
      \item $(\leftarrow)$. Let $f$ have no root in $F$. Then by corollary 1.5 there exists no linear factors $(x-c)$. By the same pigeonhole principle argument, we know that having a linear factor for degree 2 or 3 polynomials is equivalent to having (general) factors, and so $f$ has no factors. Therefore $f$ is irreducible. 
    \end{enumerate}
  \end{solution}

  \begin{exercise}[Shifrin 3.1.13]
    List all the irreducible polynomials in $\mathbb{Z}_2[x]$ of degree $\leq 4$. Factor $f(x) = x^7 + 1$ as a product of irreducible polynomials in $\mathbb{Z}_2[x]$.
  \end{exercise}
  \begin{solution}
    Listed by degree. 
    \begin{enumerate}
      \item $1$: $x, x + 1$. 
      \item $2$: $x^2 + x + 1$. 
      \item $3$: $x^3 + x^2 + 1, x^3 + x + 1$. 
      \item $4$: $x^4 + x + 1, x^4 + x^3 + 1, x^4 + x^3 + x^2 + x + 1$. 
    \end{enumerate}
    We have 
    \begin{align}
      x^7 + 1 & = (x + 1)(x^6 + x^5 + x^4 + x^3 + x^2 + x + 1) \\
              & = (x + 1) (x^3 + x + 1) (x^3 + x^2 + 1)
    \end{align}
  \end{solution}


  \begin{exercise}[Shifrin 3.2.2.b/c]
    Prove that
    \begin{enumerate}
      \item $\mathbb{Q}[\sqrt{2}, i] = \mathbb{Q}[\sqrt{2} + i]$, but $\mathbb{Q}[\sqrt{2}i] \subsetneq \mathbb{Q}[\sqrt{2}, i]$
      \item $\mathbb{Q}[\sqrt{2}, \sqrt{3}] = \mathbb{Q}[\sqrt{2} + \sqrt{3}]$, but $\mathbb{Q}[\sqrt{6}] \subsetneq \mathbb{Q}[\sqrt{2}, \sqrt{3}]$
      \item $\mathbb{Q}[\sqrt[3]{2} + i] = \mathbb{Q}[\sqrt[3]{2}, i]$; what about $\mathbb{Q}[\sqrt[3]{2}i] \subset \mathbb{Q}[\sqrt[3]{2}, i]$?
    \end{enumerate}
  \end{exercise}
  \begin{solution}[Shifrin 3.2.2.b]
    From Shifrin, I use the fact that $\mathbb{Q}[\sqrt{2}] = \{ a + b \sqrt{2} \mid a, b \in \mathbb{Q}\}$, and the same proof immediately shows that $\mathbb{Q}[\sqrt{3}] = \{ a + b \sqrt{3} \mid a, b \in \mathbb{Q}\}$ along with that for $\mathbb{Q}[\sqrt{6}]$. As for $\mathbb{Q}[\sqrt{2}, \sqrt{3}]$, I also follow the same logic to show 
    \begin{align}
      \mathbb{Q}[\sqrt{2}, \sqrt{3}] & = \mathbb{Q}[\sqrt{2}][\sqrt{3}] \\
                                     & = \{\alpha + \beta \sqrt{3} \mid a, b \in \mathbb{Q}[\sqrt{2}]\} \\
                                     & = \{ (a + b\sqrt{2}) + (c + d \sqrt{2}) \sqrt{3} \mid a, b, c, d \in \mathbb{Q} \} \\
                                     & = \{ a + b\sqrt{2} + c \sqrt{3} + d \sqrt{6} \mid a, b, c, d \in \mathbb{Q} \} 
    \end{align}
    Where $\sqrt{2} \times \sqrt{3} = \sqrt{2 \times 3} = \sqrt{6}$ follows from the definition of $n$th roots plus associativity on the reals. For (b), we prove bidirectionally.
    \begin{enumerate}
      \item $\mathbb{Q}[ \sqrt{2} + \sqrt{3}] \subset \mathbb{Q}[\sqrt{2}, \sqrt{3}]$. Consider $y \in \mathbb{Q}[\sqrt{2} + \sqrt{3}]$. Then there exists $p \in \mathbb{Q}[x]$ s.t. 
      \begin{equation}
        y = p(\sqrt{2} + \sqrt{3}) = a_n (\sqrt{2} + \sqrt{3})^n + \ldots + a_1 (\sqrt{2} + \sqrt{3}) + a_0
      \end{equation}
      where the terms can be expanded an rearranged to the form $a + b \sqrt{2} + c \sqrt{3} + d \sqrt{6} \in \mathbb{Q}[\sqrt{2}, \sqrt{3}]$. 

    \item $\mathbb{Q}[\sqrt{2}, \sqrt{3}] \subset \mathbb{Q}[ \sqrt{2} + \sqrt{3}]$. Consider $\sqrt{2} + \sqrt{3} \in \mathbb{Q}[\sqrt{2} + \sqrt{3}]$. Since it is a field and $\sqrt{2} + \sqrt{3}$ is a unit, by rationalizing the denominator, we can get 
      \begin{equation}
        (\sqrt{2} + \sqrt{3})^{-1} = \frac{\sqrt{2} - \sqrt{3}}{2 - 3} = \sqrt{3} - \sqrt{2} \in \mathbb{Q}[\sqrt{2} + \sqrt{3}]
      \end{equation}
      Therefore by adding and subtracting the two elements, we have $\sqrt{2}, \sqrt{3} \in \mathbb{Q}[\sqrt{2} + \sqrt{3}] \implies \sqrt{6} \in \mathbb{Q}[\sqrt{2} + \sqrt{3}]$. Since $\mathbb{Q} \subset \mathbb{Q}[\sqrt{2} + \sqrt{3}]$, from the ring properties all elements of the form $a + b \sqrt{2} + c \sqrt{3} + d \sqrt{6} \in \mathbb{Q}[\sqrt{2} + \sqrt{3}]$. 
    \end{enumerate}

    For the second part, I claim that $\sqrt{2} \not\in \mathbb{Q}[\sqrt{6}]$. Assuming it is, we have $\sqrt{2} = a + b \sqrt{6} \implies 2 = a^2 + 6b^2 + 2ab \sqrt{6}$. So $a = 0$ or $b = 0$. If $a = 0$, then $b^2 = 1/3 \implies b = 1/\sqrt{3}$ which contradicts that $b$ is rational. If $b = 0$, then $a^2 = 2 \implies a = \sqrt{2}$ which contradicts that $a$ is rational. 
  \end{solution}

  \begin{solution}[Shifrin 3.2.2.c]
    Note that $\mathbb{Q}[\sqrt[3]{2}] = \{a + b \sqrt[3]{2} + c \sqrt[3]{4}\}$, and so 
    \begin{align}
      \mathbb{Q}[\sqrt[3]{2}, i] & = \mathbb{Q}[\sqrt[3]{2}][i] \\
                                 & = \{\alpha + \beta i \mid \alpha, \beta \in \mathbb{Q}[\sqrt[3]{2}]\} \\
                                 & = \{ (a + b \sqrt[3]{2} + c \sqrt[3]{4}) + (d + e \sqrt[3]{2} + f \sqrt[3]{4}) i \mid a, b, c, d, e, f \in \mathbb{Q}\} \\
                                 & = \{ a + b \sqrt[3]{2} + c \sqrt[3]{4} + d i + e \sqrt[3]{2} i + f \sqrt[3]{4} i \mid a, b, c, d, e, f \in \mathbb{Q}\}
    \end{align}
    We prove bidirectionally. 
    \begin{enumerate}
      \item $\mathbb{Q}[\sqrt[3]{2} + i] \subset \mathbb{Q}[\sqrt[3]{2}, i]$. Consider $y \in \mathbb{Q}[\sqrt[3]{2} + i]$. Then there exists a $p \in \mathbb{Q}[x]$ s.t. 
      \begin{equation}
        y = p(\sqrt[3]{2} + i) = a_n (\sqrt[3]{2} + i)^n + \ldots + a_1 (\sqrt[3]{2} + i) + a_0
      \end{equation}
      Then we can expand and rearrange the terms to be of the form 
      \begin{equation}
        a + b \sqrt[3]{2} + c \sqrt[3]{4} + d i + e i \sqrt[3]{2} + f i \sqrt[3]{4} \in \mathbb{Q}[\sqrt[3]{2}, i]
      \end{equation}

      \item $\mathbb{Q}[\sqrt[3]{2}, i] \subset \mathbb{Q}[\sqrt[3]{2} + i]$. Consider $\alpha = \sqrt[3]{2} + i \in \mathbb{Q}[\sqrt[3]{2} + i]$. Then $(\alpha - i)^3 = 2$. Therefore 
      \begin{align}
        \alpha^3 - 3 \alpha^2 i - 3 \alpha + i = 2 & \implies i(1 - 3 \alpha^2) = 2 + 3 \alpha - \alpha^3 \\ 
                                                   & \implies i = \frac{2 + 3 \alpha - \alpha^3}{1 - 3 \alpha^2} \in \mathbb{Q}[\sqrt[3]{2} + i]
      \end{align}
      Therefore $\sqrt[3]{2} = \alpha - i \in \mathbb{Q}[\sqrt[3]{2} + i]$, which allows us add all combinations $\{1, \sqrt[3]{2}, \sqrt[3]{4}, i, \sqrt[3]{2} i, \sqrt[3]{4} i\}$ into our basis. 
    \end{enumerate}
  \end{solution}

  \begin{exercise}[Shifrin 3.2.6.b/c/d/g]
    Suppose $\alpha \in \mathbb{C}$ is a root of the given irreducible polynomial $f(x) \in \mathbb{Q}[x]$. Find the multiplicative inverse of $\beta \in \mathbb{Q}[\alpha]$.
    \begin{enumerate}
      \item $f(x) = x^2 + 3x - 3$, $\beta = \alpha - 1$ 
      \item $f(x) = x^3 + x^2 - 2x - 1$, $\beta = \alpha + 1$
      \item $f(x) = x^3 + x^2 + 2x + 1$, $\beta = \alpha^2 + 1$
      \item $f(x) = x^3 - 2$, $\beta = \alpha + 1$
      \item $f(x) = x^3 + x^2 - x + 1$, $\beta = \alpha + 2$
      \item $f(x) = x^3 - 2$, $\beta = r + s\alpha + t\alpha^2$
      \item $f(x) = x^4 + x^2 - 1$, $\beta = \alpha^3 + \alpha - 1$
    \end{enumerate}
  \end{exercise}
  \begin{solution}
    For (b), using the Euclidean algorithm gives 
    \begin{equation}
      (1) (x^3 + x^2 - 2x - 1) + (-x^2 + 2) (x + 1) = 1 
    \end{equation}
    and substituting the root $\alpha$ gives $(-\alpha^2 + 2)(\alpha + 1) = 1$. So we have $\beta^{-1} = -\alpha^2 + 2$.  
    For (c), doing the same thing gives 
    \begin{equation}
      (-x) (x^3 + x^2 + 2x + 1) + (x^2 + x + 1)(x^2 + 1) = 1
    \end{equation}
    and substituting $\alpha$ gives $(\alpha^2 + \alpha + 1)(\alpha^2 + 1) = 1$, so $\beta^{-1} = \alpha^2 + \alpha + 1$. 
    For (d), we have 
    \begin{equation}
      (-\frac{1}{3}) (x^3 - 2) + (\frac{1}{3} x^2 - \frac{1}{3} x + \frac{1}{3}) (x + 1) = 1 
    \end{equation}
    and so substituting $\alpha$ gives $(\frac{1}{3} \alpha^2 - \frac{1}{3} \alpha + \frac{1}{3}) (\alpha + 1) = 1$, so $\beta^{-1} = \frac{1}{3} \alpha^2 - \frac{1}{3} \alpha + \frac{1}{3}$. For (g), we have 
    \begin{equation}
      (-x^2 - x - 2) (x^4 + x^2 - 1) + (x^3 + x^2 + 2x + 1) (x^3 + x - 1) = 1
    \end{equation}
    and so substituting $\alpha$ gives $(\alpha^3 + \alpha^2 + 2\alpha + 1) (\alpha^3 + \alpha - 1) = 1$, and so $\beta^{-1} = \alpha^3 + \alpha^2 + 2\alpha + 1$. 
  \end{solution}

  \begin{exercise}[Shifrin 3.2.7]
    Let $f(x) \in \mathbb{R}[x]$.
    \begin{enumerate}
      \item Prove that the complex roots of $f(x)$ come in ``conjugate pairs''; i.e., $\alpha \in \mathbb{C}$ is a root of $f(x)$ if and only if $\overline{\alpha}$ is also a root.
      \item Prove that the only irreducible polynomials in $\mathbb{R}[x]$ are linear polynomials and quadratic polynomials $ax^2 + bx + c$ with $b^2 - 4ac < 0$.
    \end{enumerate}
  \end{exercise}
  \begin{solution}
    Listed. 
    \begin{enumerate}
      \item If $\alpha \in \mathbb{C}$ is a root of $f$, then 
      \begin{equation}
        0 = f(\alpha) = a_n \alpha^n + \ldots + a_1 \alpha + a_0
      \end{equation}
      for $a_i \in \mathbb{R}$. Since 
      \begin{align}
        0 = \overline{0} & = \overline{f(\alpha)} \\
                         & = \overline{a_n \alpha^n + \ldots + a_1 \alpha + a_0} \\
                         & = \overline{a_n} \overline{\alpha^n} + \ldots + \overline{a_1} \overline{\alpha} + \overline{a_0} \\
                         & = a_n \overline{\alpha}^n + \ldots + a_1 \overline{\alpha} + a_0 \\
                         & = p(\overline{\alpha})
      \end{align} 
      we can see that $\overline{\alpha} \in \mathbb{C}$ is immediately a root as well. Since $\overline{\overline{\alpha}} = \alpha$, the converse is immediately proven. 

      \item Linear polynomials in $F[x]$ for a given field are trivially irreducible (since multiplying polynomials increases the degree of the product as there are no zero divisors in a field). Perhaps without Theorem 4.1, we can assume that a real quadratic polynomial $p(x) = ax^2 + bx + c$ is reducible, which is equivalent to 
      \begin{equation}
        p(x) = (dx + e)(fx + g) = dfx^2 + (dg + ef) x + eg 
      \end{equation}
      For $d, e, f, g \in \mathbb{R}$, and evaluating $b^2 - 4ac = (dg + ef)^2 - 4dfeg = (dg - ef)^2 \geq 0$ since this is a squared term of a real number. So we have proved that if it is quadratic and reducible, then the discriminant $\geq 0$. To prove the other way, we assume that it is not reducible, i.e. there exists some complex root $\alpha$ from the fundamental theorem of algebra. Then from (1), we know that $\overline{\alpha}$ must also be a complex conjugate. Then this is reducible in $\mathbb{C}$ as 
      \begin{equation}
        p(x) = a (x - \alpha) (x - \overline{\alpha}) 
      \end{equation}
      for some constant factor $a$. Letting $\alpha = d + ei$ for $d, e \in \mathbb{R}$, expanding it gives us 
      \begin{align}
        p(x) & = a \big( x^2 - (\alpha + \overline{\alpha}) x + \alpha \overline{\alpha} \big) \\
             & = a x^2 + - 2 a d x + a(d^2 + e^2)
      \end{align}
      and evaluating the discriminant gives  
      \begin{equation}
        4a^2 d^2 - 4 a^2 (d^2 + e^2) = -4 a^2 e^2 < 0
      \end{equation}
      and we are done. For higher degree polynomials, we can proceed by taking a complex root (which is guaranteed to exist by fundamental theorem of algebra). If it contains an imaginary term, then its conjugate is also a root, and we factor out the quadratic. If it is real, then we can factor out the linear term. We can keep going this until we hit our base cases of a quadratic or linear term. 
    \end{enumerate}
  \end{solution}

  \begin{exercise}[Shifrin 3.2.13]
    Let $K$ be a field extension of $F$, and suppose $\alpha, \beta \in K$. Show that $(F[\alpha])[\beta] = (F[\beta])[\alpha]$, so that $F[\alpha, \beta]$ makes good sense.
    
    (Remark: One way to do this is to think about the ring of polynomials in two variables. The other way is just to show directly that every element of one ring belongs to the other.)
  \end{exercise}
  \begin{solution}
    Let $y \in (F[\alpha])[\beta]$. Then there exists a polynomial $p \in (F[\alpha])[x]$ s.t. 
    \begin{equation}
      y = p(\beta) = b_n \beta^n + \ldots + b_1 \beta + b_0 = \sum_{i=0}^n b_i \beta^i 
    \end{equation}
    for $b_i \in F[\alpha]$. But since $b_i \in F[\alpha]$, there exists a polynomial $q_i \in F[x]$ s.t. (omitting the subscript $i$ for clarity)
    \begin{equation}
      b_i = q_i (\alpha) = a_{n_i} \alpha^n + \ldots + a_1 \alpha + a_0 = \sum_{j=0}^{n_i} a_{j} \alpha^j 
    \end{equation}
    for $a_j \in F$. Substituting each $b_i$ in gives   
    \begin{equation}
      y = \sum_{i=0}^n \bigg( \sum_{j=0}^{n_i} a_j \alpha^j \bigg) \beta^i = \sum_{i=0}^n \sum_{j=0}^{n_i} a_j \alpha^j \beta^i
    \end{equation}
    With the same logic, every element of $(F[\beta])[\alpha]$ can be written as 
    \begin{equation}
      y = \sum_{i=0}^n \bigg( \sum_{j=0}^{n_i} a_j \beta^j \bigg) \alpha^i = \sum_{i=0}^n \sum_{j=0}^{n_i} a_j \alpha^i \beta^j
    \end{equation}
    Note that since $F[\alpha]$ is a vector space spanned by $\{1, \ldots, \alpha^{n-1}\}$, and $F[\beta]$ is a also a vector space spanned by $\{1, \ldots, \beta^{m-1}\}$ for some $m$, the two spaces above are spanned by all products $\{\alpha^i \beta^j\}_{i < n, j < m}$, and they are the same set. 
  \end{solution}

  \begin{exercise}[Shifrin 3.3.2.a/d/e/g]
    Decide which of the following polynomials are irreducible in
    $\mathbb{Q}[x]$.
    \begin{enumerate}
      \item[a] $f(x) = x^3 + 4x^2 - 3x + 5$
      \item $f(x) = 4x^4 - 6x^2 + 6x - 12$
      \item $f(x) = x^3 + x^2 + x + 1$
      \item[d] $f(x) = x^4 - 180$
      \item[e] $f(x) = x^4 + x^2 - 6$
      \item $f(x) = x^4 - 2x^3 + x^2 + 1$
      \item[g] $f(x) = x^3 + 17x + 36$
      \item $f(x) = x^4 + x + 1$
      \item $f(x) = x^5 + x^3 + x^2 + 1$
      \item $f(x) = x^5 + x^3 + x + 1$
    \end{enumerate}
  \end{exercise}
  \begin{solution}
    For (a), by the rational root theorem the rational roots, if any, must be in the set $\{\pm 1, \pm 5\}$. Calculating them gives $f(x) = 7, 11, 215, -5$. Since this is third degree, no linear factors means that it is irreducible, so $f$ is irreducible. 

    For (d), by the Eisenstein's criterion with $p = 5$ this polynomial is irreducible. 

    For (e), the rational root theorem states that the rational roots must be in $\{\pm 1, \pm 2, \pm 3, \pm 6\}$. This polynomial is clearly even, so it suffices to check the positive candidates. This gives $-4, 14, 84, 1326$. Therefore if it is reducible, by Gauss's lemma it must be of the form 
    \begin{equation}
      (ax^2 + bx + c)(dx^2 + ex + f)
    \end{equation} 
    for integer coefficients. $a = d = 1$ is trivial ($-1, -1$ is also possible but constant factors don't matter). Expanding this gives 
    \begin{equation}
      x^4 + (b + e) x^3 + (c + f + be) x^2 + (bf + ce) x + cf = x^4 + x^2 - 6
    \end{equation}
    The coefficients of $x^3$ tell us that $e = -b$, which means that for the coefficents of $x$, $bf + ce = bf - bc = 0 \implies f = c$. So $c^2 = -6$, which has no solution. Therefore $f$ is irreducible. 

    For (g), we must check rational roots of $\{\pm1, \pm2, \pm3, \pm4, \pm6, \pm9, \pm12, \pm18, \pm36\}$. Since this polynomial is monotonically increasing, with $f(-2) = -6$ and $f(0) = 36$. It only suffices to check $x = -1$, which gives $f(-1) = 18$. Therefore there are no linear factors. Since this is third degree, no linear factors means that it is irreducible, so $f$ is irreducible. 
  \end{solution}

  \begin{exercise}[Shifrin 3.3.4]
    Show that each of the following polynomials has no rational root:
    \begin{enumerate}
      \item $x^{200} - x^{41} + 4x + 1$
      \item $x^8 - 54$
      \item $x^{2k} + 3x^{k+1} - 12$, $k \geq 1$
    \end{enumerate}
  \end{exercise}
  \begin{solution}
    Listed. 
    \begin{enumerate}
      \item By the rational root theorem, the only possible rational roots are $\pm1$. Solving for both of these values gives 
      \begin{align}
        f(1) & = 1 - 1 + 4 + 1 = 5 \\ 
        f(-1)& = 1 + 1 - 4 + 1 = -1
      \end{align}
      Therefore there are no rational roots. 

      \item The only possible rational roots are $\pm 1, \pm 2, \pm 3, \pm 6, \pm 9, \pm 18, \pm 27, \pm 54$. But this polynomial is even, so it suffices to check the positive roots. $f(1) = -53$, $f(2) = 256 - 54 = 202$, and any greater inputs will increase the output since $f$ is monotonic in $\mathbb{Z}^+$. Therefore $f$ has no rational roots. 

      \item By Eisenstein's criterion with $p = 3$, this polynomial is irreducible and therefore has no rational roots. 
    \end{enumerate}
  \end{solution}

  \begin{exercise}[Shifrin 3.3.6]
    Listed. 
    \begin{enumerate}
      \item Prove that $f(x) \in \mathbb{Z}_2[x]$ has $x + 1$ as a factor if and only if it has an even number of nonzero coefficients.
      \item List the irreducible polynomials in $\mathbb{Z}_2[x]$ of degrees $2, 3, 4$, and $5$.
    \end{enumerate}
  \end{exercise}
  \begin{solution}
    Listed. 
    Since $f(x)$ has $x + 1$ as a factor iff 
    \begin{equation}
      f(1) = a_n 1^n + \ldots + a_1 1^1 + a_0 = a_n + \ldots + a_1 + a_0 = 0
    \end{equation}
    where each $a_i \in \{0, 1\}$. Therefore, this is equivalent to saying that there are an even number of $1$'s (nonzero coefficients), which sum to $0$ mod 2. Therefore, the irreducible polynomials should at least have a constant coefficient of $1$ (so we can't factor $x$) and should have odd number of terms (so that we can't factor $x+1$). This will guarantee that $f(0) = f(1) = 1$. 
    \begin{enumerate}
      \item Degree 2: $x^2 + x + 1$ is the only candidate and indeed is an irreducible polynomial. 

      \item Degree 3: $x^3 + x^2 + 1$, $x^3 + x + 1$ and indeed $f(0) = f(1) = 1$. Since it's only degree 3 we don't need to check irreducibility into 2 terms of both degree at least 2. 

      \item Degree 4: $x^4 + x^3 + x^2 + x + 1$, $x^4 + x^3 + 1$, $x^4 + x^2 + 1$, $x^4 + x + 1$ are candidates. However we need to check that they cannot be factored into two irreducible quadratic polynomials. The only possible such factorization is 
      \begin{equation}
        (x^2 + x + 1) (x^2 + x + 1) = x^4 + x^2 + 1 
      \end{equation}
      and so the irreducible polynomials are $x^4 + x^3 + x^2 + x + 1$, $x^4 + x^3 + 1$, $x^4 + x + 1$. 

      \item Degree 5: $x^5 + x^4 + 1$, $x^5 + x^3 + 1$, $x^5 + x^2 + 1$, $x^5 + x + 1$, $x^5 + x^4 + x^3 + x^2 + 1$, $x^5 + x^4 + x^3 + x + 1$, $x^5 + x^4 + x^2 + x + 1$, $x^5 + x^3 + x^2 + x + 1$ are the possible candidates. But we need to check that it is not factorable into an irreducible quadratic and cubic. The three candidates are 
      \begin{align}
        (x^2 + x + 1)(x^3 + x^2 + 1) & = x^5 + x + 1 \\
        (x^2 + x + 1)(x^3 + x + 1) & = x^5 + x^4 + 1
      \end{align}
      and so the irreducible polynomials are $x^5 + x^3 + 1$, $x^5 + x^2 + 1$, $x^5 + x^4 + x^3 + x^2 + 1$, $x^5 + x^4 + x^3 + x + 1$, $x^5 + x^4 + x^2 + x + 1$, $x^5 + x^3 + x^2 + x + 1$. 
    \end{enumerate}
  \end{solution}

  \begin{exercise}[Shifrin 3.3.7]
    Prove that for any prime number $p$, $f(x) = x^{p-1} + x^{p-2} + \cdots + x + 1$ is irreducible in $\mathbb{Q}[x]$.
  \end{exercise}
  \begin{solution}
    We can use the identity 
    \begin{equation}
      f(x) = x^{p-1} + x^{p-2} + \cdots + x + 1 = \frac{x^p - 1}{x - 1} 
    \end{equation}
    Therefore, 
    \begin{align}
      f(x+1) = \frac{(x+1)^p - 1}{(x + 1) - 1} & = \frac{1}{x}\bigg\{ \bigg( \sum_{k=0}^p \binom{p}{k} x^k \bigg) - 1 \bigg\} \\
                                               & = \frac{1}{x} \sum_{k=1}^p \binom{p}{k} x^k =  \sum_{k=1}^p \binom{p}{k} x^{k-1}
    \end{align}
    Focusing on the coefficients, the leading coefficient is $\binom{p}{p} = 1$, and the rest of the coefficients are divisible by $p$. The constant coefficient is $\binom{p}{1} = p$, which is not divisible by $p^2$. By Eisenstein's criterion, $f(x+1)$ is irreducible $\implies f(x)$ is irreducible. To justify the final step, assume that $f(x)$ is reducible. Then $f(x) = g(x) h(x)$ for positive degree polynomials $g, h$. Then by substituting $x + 1$, we have that $f(x+1) = g(x+1) h(x+1)$, which means that $f(x+1)$ is irreducible. 
  \end{solution}

  \begin{exercise}[Shifrin 4.1.3]
    \begin{enumerate}
      \item[(a)] Prove that if $I \subset R$ is an ideal and $1 \in I$, then $I = R$.
      \item[(b)] Prove that $a \in R$ is a unit if and only if $\langle a \rangle = R$.
      \item[(c)] Prove that the only ideals in a (commutative) ring $R$ are $\langle 0 \rangle$ and $R$ if and only if $R$ is a field.
    \end{enumerate}
  \end{exercise}
  \begin{solution}
    Listed. 
    \begin{enumerate}
      \item[(a)] If $1 \in I$, then for every $r \in R$, we must have $r1 = r \in I$. Therefore $I = R$. 
      \item[(b)] If $a \in R$ is a unit, then $a^{-1} \in R$, and so for every $r \in R$, $r a^{-1} \in R$. Therefore, $\langle a \rangle$ must contain all elements of form $ra^{-1} a = r$, which is precisely $R$. Now assume that $a$ is not a unit, and so there exists no $a^{-1} \in R$. Therefore, $\langle a \rangle$, which consists of all $ra$ for $r \in R$, cannot contain $1$ since $r \neq a^{-1}$, and so $\langle a \rangle \neq R$. 
      \item[(c)] For the forwards implication, assume that $R$ is not a field. Then there exists some $a \neq 0$ that is not a unit, and taking $\langle a \rangle$ gives us an ideal that---from (b)---is not $R$. For the backward implication we know that $\langle 0 \rangle$ is an ideal. Now assume that there exists another ideal $I$ containing $a \neq 0$. Since $R$ is a field, $a$ is a unit, and so by (b) $R = \langle a \rangle \subset I \subset R \implies I = R$. 
    \end{enumerate}
  \end{solution}

  \begin{exercise}[Shifrin 4.1.4.a/b/c]
    Find all the ideals in the following rings:
    \begin{enumerate}
      \item[(a)] $\mathbb{Z}$
      \item[(b)] $\mathbb{Z}_7$
      \item[(c)] $\mathbb{Z}_6$
      \item[(d)] $\mathbb{Z}_{12}$
      \item[(e)] $\mathbb{Z}_{36}$
      \item[(f)] $\mathbb{Q}$
      \item[(g)] $\mathbb{Z}[i]$ (see Exercise 2.3.18)
    \end{enumerate}
  \end{exercise}
  \begin{solution}
    Listed. 
    \begin{enumerate}
      \item[(a)] All sets of form $\{k z \in \mathbb{Z} \mid z \in \mathbb{Z}\}$ for all $k \in \mathbb{Z}$. 
      \item[(b)] Only $\{0\}$ and $\mathbb{Z}_7$ is an ideal. 
      \item[(c)] We have $\{0\}, \{0, 2, 4\}, \{0, 3\}, \mathbb{Z}_6$. 
    \end{enumerate}
  \end{solution}

  \begin{exercise}[Shifrin 4.1.5]
    \begin{enumerate}
      \item[(a)] Let $I = \langle f(x) \rangle$, $J = \langle g(x) \rangle$ be ideals in $F[x]$. Prove that $I \subset J \Leftrightarrow g(x)|f(x)$.
      \item[(b)] List all the ideals of $\mathbb{Q}[x]$ containing the element 
      $f(x) = (x^2 + x - 1)^3(x - 3)^2$.
    \end{enumerate}
  \end{exercise}
  \begin{solution}
    For (a), we prove bidirectionally. 
    \begin{enumerate}
      \item $(\rightarrow)$. Since $f (x) \in \langle f(x) \rangle \implies f(x) \in \langle g(x) \rangle$, this means that $f(x) = r(x) g(x)$ for some $r(x) \in F[x$. Therefore $g(x) \mid f(x)$. 

      \item $(\leftarrow)$. Given that $g(x) \mid f(x)$, let us take some $f_1 (x) \in I$. Then it is of the form $f_1(x) = r(x) f(x)$ for some $r(x) \in F[x]$. But since $g(x) \mid f(x)$, $f(x) = h(x) g(x)$ for some $h(x) \in F[x]$. Therefore $f_1 (x) = r(x) h(x) g(x) = (rh)(x) g(x)$, where $(rh)(x) \in F[x]$, and so $f_1 (x) \in J$. 
    \end{enumerate}

    For (b), we can use the logic from (a) to find all the factors of $f(x)$, which generate all sup-ideals of $\langle f(x) \rangle$, which is the minimal ideal containing $f(x)$. 
    \begin{enumerate}
      \item $g(x) = 1 \implies \langle 1 \rangle = F[x]$  
      \item $g(x) = x^2 + x - 1 \implies \langle x^2 + x - 1 \rangle$
      \item $g(x) = (x^2 + x - 1)^2 \implies \langle (x^2 + x - 1)^2 \rangle$
      \item $g(x) = (x^2 + x - 1)^3 \implies \langle (x^2 + x - 1)^3 \rangle$
      \item $g(x) = x - 3 \implies \langle x - 3 \rangle$
      \item $g(x) = (x^2 + x - 1)(x - 3) \implies \langle (x^2 + x - 1)(x - 3) \rangle$
      \item $g(x) = (x^2 + x - 1)^2 (x - 3) \implies \langle (x^2 + x - 1)^2 (x - 3) \rangle$
      \item $g(x) = (x^2 + x - 1)^3 (x - 3) \implies \langle (x^2 + x - 1)^3 (x - 3) \rangle$
      \item $g(x) = (x - 3)^2 \implies \langle (x - 3)^2 \rangle$
      \item $g(x) = (x^2 + x - 1)(x - 3)^2 \implies \langle (x^2 + x - 1)(x - 3)^2 \rangle$
      \item $g(x) = (x^2 + x - 1)^2 (x - 3)^2 \implies \langle (x^2 + x - 1)^2 (x - 3)^2 \rangle$
      \item $g(x) = (x^2 + x - 1)^3 (x - 3)^2 \implies \langle (x^2 + x - 1)^3 (x - 3)^2 \rangle$
    \end{enumerate}
  \end{solution}

  \begin{exercise}[Shifrin 4.1.14.a/b]
    Mimicking Example 5(c), give the addition and multiplication tables of
    \begin{enumerate}
      \item[(a)] $\mathbb{Z}_2[x]/\langle x^2 + x \rangle$
      \item[(b)] $\mathbb{Z}_3[x]/\langle x^2 + x - 1 \rangle$
      \item[(c)] $\mathbb{Z}_2[x]/\langle x^3 + x + 1 \rangle$
    \end{enumerate}
    In each case, is the quotient ring an integral domain? a field?
  \end{exercise}
  \begin{solution}
    For (a), note that the quotient allows us to state that $x^2 \equiv x \pmod{I}$, and therefore every polynomial in $\mathbb{Z}_2 [x]/ \langle x^2 + x \rangle$ is equivalent to a linear polynomial. Therefore, the elements in this quotient are $0, 1, x, x + 1$. As you can see, this is not an integral domain (and hence not a field) since $x, x + 1$ are zero divisors. 

    \begin{figure}[H]
      \centering
      \begin{subfigure}[b]{0.48\textwidth}
        \centering
        \begin{tabular}{c|cccc}
          $+$ & $0$ & $1$ & $x$ & $x+1$ \\
          \hline
          $0$ & $0$ & $1$ & $x$ & $x+1$ \\
          $1$ & $1$ & $0$ & $x+1$ & $x$ \\
          $x$ & $x$ & $x+1$ & $0$ & $1$ \\
          $x+1$ & $x+1$ & $x$ & $1$ & $0$ \\
        \end{tabular}
      \end{subfigure}
      \hfill 
      \begin{subfigure}[b]{0.48\textwidth}
        \centering
        \begin{tabular}{c|cccc}
          $\times$ & $0$ & $1$ & $x$ & $x+1$ \\
          \hline
          $0$ & $0$ & $0$ & $0$ & $0$ \\
          $1$ & $0$ & $1$ & $x$ & $x+1$ \\
          $x$ & $0$ & $x$ & $x$ & $0$ \\
          $x+1$ & $0$ & $x+1$ & $0$ & $x+1$ \\
        \end{tabular}
      \end{subfigure}
      \caption{Addition and multiplication tables for $\mathbb{Z}_2 [x]/ \langle x^2 + x \rangle$. }
    \end{figure}

    For (b), note that the quotient allows us to state that $x^2 \equiv 2x + 1 \pmod{I}$, and therefore every polynomial in $\mathbb{Z}_3 [x] / \langle x^2 + x - 1 \rangle$ is equivalent to a linear polynomial. Therefore, the elements in this quotient are $0, 1, 2, x, x + 1, x + 2, 2x, 2x + 1, 2x + 2$. This is indeed an integral domain since there are no zero divisors, and it is a field since every nonzero element is a unit (all rows/columns are filled with all elements of the set). 

    \begin{figure}[H]
      \centering
      \begin{tabular}{c|ccccccccc}
        $+$ & $0$ & $1$ & $2$ & $x$ & $x+1$ & $x+2$ & $2x$ & $2x+1$ & $2x+2$ \\
        \hline
        $0$ & $0$ & $1$ & $2$ & $x$ & $x+1$ & $x+2$ & $2x$ & $2x+1$ & $2x+2$ \\
        $1$ & $1$ & $2$ & $0$ & $x+1$ & $x+2$ & $x$ & $2x+1$ & $2x+2$ & $2x$ \\
        $2$ & $2$ & $0$ & $1$ & $x+2$ & $x$ & $x+1$ & $2x+2$ & $2x$ & $2x+1$ \\
        $x$ & $x$ & $x+1$ & $x+2$ & $2x$ & $2x+1$ & $2x+2$ & $0$ & $1$ & $2$ \\
        $x+1$ & $x+1$ & $x+2$ & $x$ & $2x+1$ & $2x+2$ & $2x$ & $1$ & $2$ & $0$ \\
        $x+2$ & $x+2$ & $x$ & $x+1$ & $2x+2$ & $2x$ & $2x+1$ & $2$ & $0$ & $1$ \\
        $2x$ & $2x$ & $2x+1$ & $2x+2$ & $0$ & $1$ & $2$ & $x$ & $x+1$ & $x+2$ \\
        $2x+1$ & $2x+1$ & $2x+2$ & $2x$ & $1$ & $2$ & $0$ & $x+1$ & $x+2$ & $x$ \\
        $2x+2$ & $2x+2$ & $2x$ & $2x+1$ & $2$ & $0$ & $1$ & $x+2$ & $x$ & $x+1$ \\
      \end{tabular}
      \caption{Addition table for $\mathbb{Z}_3[x]/ \langle x^2 + x - 1\rangle$.}
    \end{figure}

    \begin{figure}[H]
      \centering
      \begin{tabular}{c|ccccccccc}
        $\times$ & $0$ & $1$ & $2$ & $x$ & $x+1$ & $x+2$ & $2x$ & $2x+1$ & $2x+2$ \\
        \hline
        $0$ & $0$ & $0$ & $0$ & $0$ & $0$ & $0$ & $0$ & $0$ & $0$ \\
        $1$ & $0$ & $1$ & $2$ & $x$ & $x+1$ & $x+2$ & $2x$ & $2x+1$ & $2x+2$ \\
        $2$ & $0$ & $2$ & $1$ & $2x$ & $2x+2$ & $2x+1$ & $x$ & $x+2$ & $x+1$ \\
        $x$ & $0$ & $x$ & $2x$ & $2x + 1$ & $1$ & $x+1$ & $x+2$ & $2x+2$ & $2$ \\
        $x+1$ & $0$ & $x+1$ & $2x+2$ & $1$ & $x+2$ & $2x$ & $2$ & $x$ & $2x+1$ \\
        $x+2$ & $0$ & $x+2$ & $2x+1$ & $x+1$ & $2x$ & $2$ & $2x+2$ & $1$ & $x$ \\
        $2x$ & $0$ & $2x$ & $x$ & $x+2$ & $2$ & $2x+2$ & $2x+1$ & $x+1$ & $1$ \\
        $2x+1$ & $0$ & $2x+1$ & $x+2$ & $2x+2$ & $x$ & $1$ & $x+1$ & $2$ & $2x$ \\
        $2x+2$ & $0$ & $2x+2$ & $x+1$ & $2$ & $2x+1$ & $x$ & $1$ & $2x$ & $x+2$ \\
      \end{tabular}
      \caption{Multiplication table for $\mathbb{Z}_3[x]/\langle x^2 + x - 1\rangle$.}
    \end{figure}

  \end{solution}

  \begin{exercise}[Shifrin 4.1.17]
    Let $R$ be a commutative ring and let $I,J \subset R$ be ideals. Define
    \begin{align*}
      I \cap J &= \{a \in R : a \in I \text{ and } a \in J\}\\
      I + J &= \{a + b \in R : a \in I, b \in J\}.
    \end{align*}
    \begin{enumerate}
      \item[(a)] Prove that $I \cap J$ and $I + J$ are ideals.
      \item[(b)] Suppose $R = \mathbb{Z}$ or $F[x]$, $I = \langle a \rangle$, and $J = \langle b \rangle$. Identify $I \cap J$ and $I + J$.
      \item[(c)] Let $a_1,\ldots,a_n \in R$. Prove that $\langle a_1,\ldots,a_n \rangle = \langle a_1 \rangle + \cdots + \langle a_n \rangle$.
    \end{enumerate}
  \end{exercise}
  \begin{solution}
    For (a), we have the following. 
    \begin{enumerate}
      \item $I \cap J$ is an ideal. Given $a, b \in I \cap J$, then $a, b \in I \implies a + b \in I$, and $a, b \in J \implies a + b \in J$. So $a + b \in I \cap J$. Furthermore, for every $r \in R$, $a \in I \implies r a \in I$ and $a \in J \implies r a \in J$, so $a \in I \cap J \implies ra \in I \cap J$. 

      \item $I + J$ is an ideal. Given $x, y \in I + J$, then $x = a_x + b_x$ and $y = a_y + b_y$ for $a_x, a_y \in I, b_x, b_y \in J$. So 
      \begin{equation}
        x + y = (a_x + b_x) + (a_y + b_y) = (a_x + a_y) + (b_x + b_y)
      \end{equation}
      where $a_x + a_y \in I, b_x + b_y \in J$ by definition of an ideal, and so $x + y \in I + J$. Noe let $x = a_x + b_x \in I + J$. Then given $r \in R$,
      \begin{equation}
        rx = r(a_x + b_x) = r a_x + r b_x
      \end{equation}
      where $r a_x \in I$ and $r b_x \in J$ since $I, J$ are ideals. Therefore $rx \in I + J$.  
    \end{enumerate}
    For (b), the argument is equivalent for $\mathbb{Z}$ and $F[x]$. $I \cap J$ consists of all elements that are divisible by both $a$ and $b$, so $I \cap J = \langle \mathrm{lcm}(a, b) \rangle$. $I + J$ consists of all elements that are of form $r a + s b$, but this are all multiples of $\mathrm{gcd}(a, b)$ and so $I + J = \langle \mathrm{gcd}(a, b) \rangle$. 

    For (c), it suffices to prove $\langle a, b \rangle = \langle a \rangle + \langle b \rangle$. 
    \begin{enumerate}
      \item $\langle a, b \rangle \subset \langle a \rangle + \langle b \rangle$. $x \in \langle a, b \rangle \implies x = r_a a + r_b b$ for $r_a, r_b \in R$. But $a \in \langle a \rangle, b \in \langle b \rangle \implies r_a a \in \langle a \rangle, r_b b \in \langle b \rangle$, and so $x \in \langle a \rangle + \langle b \rangle$. 

    \item $\langle a, b \rangle \supset \langle a \rangle + \langle b \rangle$. $x \in \langle a \rangle + \langle b \rangle \implies x = a_x + b_x$ for $a_x \in \langle a \rangle, b_x \in \langle b \rangle$. But $a_x \in \langle a \rangle \implies a_x = r_a a$ for some $r_a \in R$, and $b_x \in \langle b \rangle \implies b_x = r_b b$ for some $r_b \in R$. So $x = r_a a + r_b b \iff x \in \langle a, b \rangle$. 
    \end{enumerate}
    We know that for $\langle a_1 \rangle = \langle a_1 \rangle$, and so by making this argument $n-1$ times we can build up by induction that $\langle a_1, \ldots a_{n-1}, a_n \rangle = \langle a_1, \ldots, a_{n-1} \rangle + \langle a_n \rangle$. 
  \end{solution}

  \begin{exercise}[Shifrin 4.2.1]
    \begin{enumerate}
      \item[(a)] Prove that the function $\phi: \mathbb{Q}[\sqrt{2}] \to \mathbb{Q}[\sqrt{2}]$ defined by $\phi(a + b\sqrt{2}) = a - b\sqrt{2}$ is an isomorphism.
      \item[(b)] Define $\phi: \mathbb{Q}[\sqrt{3}] \to \mathbb{Q}[\sqrt{7}]$ by $\phi(a + b\sqrt{3}) = a + b\sqrt{7}$. Is $\phi$ an isomorphism? Is there any isomorphism?
    \end{enumerate}
  \end{exercise}
  \begin{solution}
    For (a), we first prove that it is a homomorphism. 
    \begin{align}
      \phi((a + b \sqrt{2}) + (c + d \sqrt{2})) & = \phi((a + c) + (b + d) \sqrt{2}) \\
                                                & = (a + c) - (b + d) \sqrt{2} \\
                                                & = (a - b \sqrt{2}) + (c - d \sqrt{2}) \\
                                                & = \phi(a + b \sqrt{2}) + \phi(c + d \sqrt{2}) \\
      \phi((a + b \sqrt{2}) (c + d \sqrt{2})) & = \phi((ac + 2bd) + (ad + bc) \sqrt{2}) \\
                                              & = (ac + 2bd) - (ad + bc) \sqrt{2} \\
                                              & =  (a - b \sqrt{2}) (c - d \sqrt{2}) \\
                                              & = \phi(a + b \sqrt{2}) \times \phi(c + d \sqrt{2}) \\ 
                                      \phi(1) & = 1
    \end{align}
    This is injective since given that $a + b \sqrt{2} \neq c + d \sqrt{2}$, then at least $a \neq b$ or $c \neq d$, in which case $a - b \sqrt{2} \neq c - d \sqrt{2}$. Alternatively, we can see that the kernel is $0$, so it must be injective. It is onto since given any $c + d\sqrt{2}$, the preimage is $c - d \sqrt{2}$. Therefore $\phi$ is an isomorphism.  

    For (b), no it is not an isomorphism since 
    \begin{align}
      \phi ((a + b \sqrt{3}) (c + d \sqrt{3})) & = \phi ((ac + 3bd) + (ad + bc) \sqrt{3}) \\
                                               & = (ac + 3bd) + (ad + bc) \sqrt{7} \\
                                               & \neq (ac + 7bd) + (ad + bc) \sqrt{7} \\ 
                                               & = (a + b \sqrt{7}) (c + d \sqrt{7}) \\
                                               & = \phi(a + b \sqrt{3}) \phi(c + d  \sqrt{3}) 
    \end{align} 
    We claim that there is no isomorphism. Assume that such $\phi$ exists. Then $\phi(1) = 1$, and so $\phi(3) = \phi(1 + 1 + 1) = \phi(1) + \phi(1) + \phi(1) = 1 + 1 + 1 = 3$. Now given $\sqrt{3} \in \mathbb{Q}[\sqrt{3}]$, we follows that 
    \begin{equation}
      \phi(\sqrt{3})^2 = \phi(3) = 3
    \end{equation}
    and so $\phi(\sqrt{3})$ must map to the square root of $3$ which must live in $\mathbb{Q}[\sqrt{7}]$. Assume such a number is $a + b \sqrt{7} \implies (a^2 + 7b^2) + (2ab) \sqrt{7} = \sqrt{3}$. This implies that $2ab = 0$, leaving the rational term, but we know that $\sqrt{3}$ does not exist in the rationals, and so $\sqrt{3}$ does not exist.  
  \end{solution}

  \begin{exercise}[Shifrin 4.2.12]
    Let $R$ be a commutative ring, $I \subset R$ an ideal. Suppose $a \in R$, $a \notin I$, and $I + \langle a \rangle = R$ (see Exercise 4.1.17 for the notion of the sum of two ideals). Prove that $\bar{a} \in R/I$ is a unit.
  \end{exercise}
  \begin{solution}
    Since $R = I + \langle a \rangle$, $1 \in R = I + \langle a \rangle$. So there exists $i \in I, ra \in \langle a \rangle$ s.t. $1 = i + ra \implies ra = 1 - i$. Therefore, in the quotient ring, $\bar{i} = 0$ and we have 
    \begin{equation}
      \bar{r} \bar{a} = \bar{1} - \bar{0} = \bar{1}
    \end{equation}
    and so $\bar{r}$ is a multiplicative inverse of $\bar{a}$. So $\bar{a}$ is a unit. 
  \end{solution}


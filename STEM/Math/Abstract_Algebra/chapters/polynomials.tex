\section{Polynomial Rings} 

  One of the most widely studied rings are the ring of polynomials. Let's reintroduce them. 

  \begin{definition}[Univariate Polynomials]
    For a ring $R$, the \textbf{univariate polynomial ring over $R$}, denoted $R[x]$ consists of elements called \textbf{polynomials} which are formal expressions of the form 
    \begin{equation}
      f(x) = a_nx^n + a_{n-1}x^{n-1} + \dots + a_1x + a_0 \text{ where } a_i \in R
    \end{equation}
    with coefficients $a_i \in R$ and $x$ is called a \textbf{variable}, or \textbf{indeterminant}.\footnote{Note that $x$ is just a formal symbol, whose powers $x^i$ are just placeholders for the corresponding coefficients $a_i$ so that the given formal expression is a way to encode the finitary sequence. $(a_0, a_1, a_2, ..., a_n)$.} Two polynomials are equal if and only if the sequences of their corresponding coefficients are equal. We can also see a polynomial as a function $f: R \rightarrow R$ as well. 

    Furthermore, $R[x]$ is a ring, with addition and multiplication defined
    \begin{equation}
      a_i x^i + b_i x^i = (a_i + b_i) x^i, \qquad x^ix^j = x^{i+j}
    \end{equation}
    along with $0$ as the additive identity and $1$ as the multiplicative identity. The last nonzero coefficient is called the \textbf{leading coefficient}, and the degree of the polynomial $f$, denoted $\deg f$, is the index of the leading coefficient.
  \end{definition} 

  While we will mainly deal with univariate polynomials, we can also define multivariate polynomials similarly. 

  \begin{definition}[Multivariate Polynomials] 
    For a ring $R$, the \textbf{multivariate polynomial ring over $R$}, denoted $R[x_1, \ldots, x_n]$ consists of elements called \textbf{polynomials} which are formal expressions of the form 
    \begin{equation}
      f(x_1, \ldots, x_n) = \sum_{0 \leq k_i \leq n} a_{k_1 \ldots k_n} x_1^{k_1} x_2^{k_2} \ldots x_n^{k_n}
    \end{equation}
    with coefficients $a \in R$ and $x_i$'s the \textbf{variables}. We can treat an element $f \in R[x_1, \ldots, x_n]$ as a function $f: R^n \rightarrow R$.  

    Furthermore, $R[x_1, \ldots, x_n]$ is a ring, with addition and multiplication defined 
    \begin{align}
      a_{k_1 \ldots k_n} x_1^{k_1} x_2^{k_2} \ldots x_n^{k_n} + b_{k_1 \ldots k_n} x_1^{k_1} x_2^{k_2} \ldots x_n^{k_n} & = (a_{k_1 \ldots k_n} + b_{k_1 \ldots k_n}) x_1^{k_1} x_2^{k_2} \ldots x_n^{k_n} \\
      x^{k_1 \ldots k_n} x^{l_1 \ldots l_n} & = x^{k_1 + l_1, k_2 + l_2, \ldots, k_n + l_n}
    \end{align}
  \end{definition}

  Usually, we almost always deal with commutative rings $R$, so we will assume this unless otherwise stated. This has a nice property on $R[x]$. 

  \begin{lemma}[Commutativity Extends to Polynomials]
    $R$ is a commutative ring $\implies R[x]$ is a commutative ring. 
  \end{lemma}
  \begin{proof}
    
  \end{proof}

  We need to be very careful about the properties that hold for polynomials, as they may not be intuitive. For example, for certain finite fields (which are rings), some formally different polynomials may be indistinguishable in terms of mappings.\footnote{$x$ and $x^2$ are equivalent in the polynomial algebra defined on the domain $\mathbb{Z}_2$. } Second, a polynomial may have more roots than its degree. Therefore, we will work in different rings $R$ and provide conditions where our intuition is true in $R[x]$. It is clear that if you have two polynomials of degree $n$ and $m$, their sum may be degree $k < n, m$. This is not always true for multiplication. 

  \begin{example}[Product of Two Linear Polynomials is $0$]
    Given $f, g \in \mathbb{Z}_6 [x]$ with $f(x) = 2x + 4$ and $g(x) = 3x + 3$, we have 
    \begin{equation}
      f(x) \cdot g(x) = (2x + 4)(3x + 3) = 6x^2 + 18 x + 12 = 0
    \end{equation}
  \end{example}

  There is a simple condition in which the degree is additive, however. 

  \begin{theorem}[Bounds on Degrees From Operations]
    Given that $R$ is a ring and $f, g \in R[x]$, 
    \begin{equation}
      \deg(f+g) \leq \max\{\deg f, \deg g\} \\
    \end{equation}
    If $R$ is a domain, then 
    \begin{equation}
      \deg (f g) = \deg f + \deg g
    \end{equation}
    Note that this automatically implies that $R[x]$ is a domain. Combined with the lemma above, we have: $R$ is an integral domain $\implies R[x]$ is an integral domain. 
  \end{theorem}
  \begin{proof}
    The second may not be true if $R$ has zero divisors. 
  \end{proof}

  Just working in domains do not make things all better. Sometimes, we may have two different polynomials but they may define the same function from $R$ to $R$! 

  \begin{example}[Polynomials as Same Function]
    Given $f, g \in \mathbb{Z}_2 [x]$, 
    \begin{equation}
      f(x) = x \sim g(x) = x^2
    \end{equation} 
  \end{example}

  As shown in the example above, it is not so simple as to restrict which underlying set you are working on. Some rings $R$ may or may not assert uniqueness of functions in $F[x]$, and vice versa. Therefore, here are some special theorems. 

  \begin{theorem}[Uniqueness of Polynomials over Field]
    If the field $\mathbb{F}$ is infinite, then different polynomials in $\mathbb{F}[x]$ determine different functions. 
  \end{theorem}

\subsection{Euclidean Division} 

  Just like how we can do Euclidean division with integers, there is an analogous result for polynomials. However, we require to work with a \textit{field} $F$ rather than an arbitrary ring $R$. 

  \begin{theorem}[Polynomials as Euclidean Domain]
    Given a field $F$, $F[x]$ is a Euclidean domain. 
  \end{theorem}

  \begin{example}[Polynomials over Fields]
    These are also Euclidean domains. 
    
    \begin{center}
      \polylongdiv{x^3 + 4x^2 - x + 7}{x - 2}
    \end{center}

    Given field $\mathbb{Z}_5$, $\mathbb{Z}_5[x]$ is a Euclidean domain, with Euclidean division.  
  \end{example} 

  \begin{definition}[GCD]
    
  \end{definition}

\subsection{Roots and Factorization}

  Next, we can define the all too familiar root of a polynomial.  

  \begin{definition}[Polynomial Root]
    An element $r \in R$ is a \textbf{root} of polynomial $f \in R[x]$ if and only if 
    \begin{equation}
      f(r) = 0
    \end{equation}
  \end{definition}

  \begin{theorem}[Interpolation]
    For any collection of given field values $y_1, y_2, ..., y_n \in \mathbb{F}$ at given distinct points $x_1, x_2, ..., x_n \in \mathbb{F}$, there exists a unique polynomial $f \in F[x]$ with deg$\, f < n$ such that
    \begin{equation}
      f(x_i) = y_i, \; i = 1, 2, ..., n
    \end{equation}
    This is commonly known as the \textbf{interpolation problem}, and when $n = 2$, this is called \textbf{linear interpolation}. 
  \end{theorem} 

  Now we can introduce the bread and butter of polynomial rings: the fundamental theorem of algebra. Ironically, this theorem cannot be proven with algebra alone. We need complex analysis.\footnote{Gauss proved this for the first time in 1799.} 

  \begin{theorem}[Fundamental Theorem of Algebra]
    Suppose $f \in \mathbb{C}[x]$ is a polynomial of degree $n \geq 1$. Then $f(x)$ has a root in $\mathbb{C}$. It immediately follows from induction that it can be factored as a product of linear polynomials in $\mathbb{C}[x]$. 
  \end{theorem}
  \begin{proof}
    WLOG we can assume that $f$ is monic: $f(z) = z^n + a_{n-1} z^{n-1} + \ldots + a_1 z + a_0$. Since $\mathbb{C}$ is a field, we can set 
    \begin{equation}
      f(z) = z^n \bigg( 1 + \frac{a_{n-1}}{z} + \frac{a_{n-2}}{z^2} + \ldots + \frac{a_0}{z_n} \bigg)
    \end{equation} 
    Since 
    \begin{equation}
      \lim_{|z| \rightarrow \infty} \bigg( 1 + \frac{a_{n-1}}{z} + \frac{a_{n-2}}{z^2} + \ldots + \frac{a_0}{z_n} \bigg) = 0
    \end{equation}
    there exists a $R > 0$ s.t. 
    \begin{equation}
      |z| > R \implies \bigg| 1 + \frac{a_{n-1}}{z} + \frac{a_{n-2}}{z^2} + \ldots + \frac{a_0}{z_n} \bigg| < \frac{1}{2}
    \end{equation}
    and hence 
    \begin{equation}
      |z| > R \implies |f(z)| > |z|^n \cdot \bigg( 1 - \frac{1}{2} \bigg) > \frac{R^n}{2}
    \end{equation}
    So $z$ cannot be a root if $|z| > R$. On the other hand, $f(z)$ is continuous (under the Euclidean topology) and so on the compact set $\{z \in \mathbb{C} \mid |z| \leq R\}$, $|f(z)|$ achieves a minimum value say at the point $z_0$. We claim that $\min_z f(z) = 0$. 

    For convenience, we let $z_0 = 0$ (we can do a change of basis on the polynomial) and assume that the minimum is some positive number, i.e. $f(0) = a_0 \neq 0$. Let $j$ be the smallest positive integer such that $a_j = 0$. Let 
    \begin{equation}
      g(z) = \frac{a_{j+1}}{a_j} z + \ldots + \frac{a_n}{a_j} z^{n-j} \implies f(z) = a_0 + a_j z^j \big( 1 + g(z) \big) 
    \end{equation}
    We set $\gamma = \sqrt[j]{-a_0/a_j}$ and consider the values of 
    \begin{align}
      f(t \gamma) & = a_0 + a_j (t\gamma)^j \big( 1 + g(t\gamma) \big) \\
                  & = a_0 - a_0 t^j \big(1 + g(t \gamma) \big) \\
                  & = a_0 \big\{ 1 - t^j \big(1 + g(t \gamma) \big) \big\}
    \end{align} 
    for $t > 0$. For $t$ sufficiently small, we have 
    \begin{equation}
      |g(t \gamma)| = \bigg| \frac{a_{j+1}}{a_j} (t \gamma) + \ldots + \frac{a_n}{a_j} (t \gamma)^{n-j} \bigg| < \frac{1}{2} 
    \end{equation}
    and for such $t$, this implies 
    \begin{equation}
      |f(t \gamma)| = |a_0| |1 - t^j (1 + g(t \gamma))| \leq |a_0| |1 - t^j/2| < |a_0|
    \end{equation}
    and so $z_0$ cannot have been the minimum of $|f(z)|$. Therefore, the minimum value must be $0$.  
  \end{proof}

\subsection{Rational Polynomials} 

  \subsubsection{Field Extensions}

    Given a polynomial $f \in \mathbb{Q}[x]$, the fundamental theorem of algebra guarantees that it will have all of its roots in $\mathbb{C}$. This notion of taking a field $F$ and creating a sup-field $K \supset F$ will be done many times. 

    \begin{definition}[Field Extension]
      The pair of fields $F \subset K$ is called a \textbf{field extension}. 
    \end{definition}

    Our immediate goal is to find the \textit{smallest} possible field $K \subset \mathbb{C}$ containing them all. What can we determine about $K$? 
    
    \begin{lemma}
      Every subfield of $\mathbb{C}$ contains $\mathbb{Q}$. 
    \end{lemma}
    \begin{proof}
      
    \end{proof} 

    \begin{definition}[Ring of Univariate Polynomial Elements] 
      Let $F \subset K$ be fields, $F[x]$ a polynomial ring, and a constant $\alpha \in K$, 
      \begin{equation}
        F[\alpha] \coloneqq \{ f(\alpha) \in F \mid f \in F[x]\} \subset K
      \end{equation}
    \end{definition} 

    The most obvious structure is that of a ring. 

    \begin{lemma}[Rings]
      $F[\alpha]$ is a subring of $K$. 
    \end{lemma}
    \begin{proof}
      Given two elements $\phi, \gamma \in F[\alpha]$, there exists polynomials $f, g \in F[x]$ s.t. $\phi = f(\alpha), \gamma = g(\alpha)$. Since $F[x]$ is a ring, we see that 
      \begin{align}
        \phi + \gamma & = f(\alpha) + g(\alpha) = (f + g)(\alpha) \\
        \phi \cdot \gamma & = f(\alpha) \cdot g(\alpha) = (fg)(\alpha)
      \end{align} 
      Furthermore, it is easy to check that $0$ and $1$ are the images of $\alpha$ through the $0$ and $1$ polynomials. 
    \end{proof} 

    Let's go through some examples. In here, we will always assume that $F = \mathbb{Q}$ and $K = \mathbb{C}$. 

    \begin{example}[Radical Extensions of $\sqrt{2}$]
      We claim $\mathbb{Q}[\sqrt{2}] = \{a + b \sqrt{2} \mid a, b \in \mathbb{Q} \}$.
      \begin{enumerate}
        \item $\mathbb{Q}[\sqrt{2}] \subset \{a + b \sqrt{2} \mid a, b \in \mathbb{Q} \}$. $\mathbb{Q}[\sqrt{2}]$ are elements of the form
        \begin{equation}
          f(\sqrt{2}) = a_n (\sqrt{2})^n + a_{n-1} (\sqrt{2})^{n-1} + \ldots + a_2 (\sqrt{2})^2 + a_1 \sqrt{2} + a_0
        \end{equation} 
        This can be written by collecting terms, of the form $a + b \sqrt{2}$. 

        \item $\mathbb{Q}[\sqrt{2}] \supset \{a + b \sqrt{2} \mid a, b \in \mathbb{Q} \}$. Given an element $a + b \sqrt{2}$, this is clearly in $\mathbb{Q}[\sqrt{2}]$ since it is the image of $\sqrt{2}$ under the polynomial $f(x) = a + bx$. 
      \end{enumerate}
    \end{example} 

    Given this, we may extrapolate this pattern and claim that $\mathbb{Q}[\sqrt{2} + \sqrt{3}]$ consists of all numbers of form $a + (\sqrt{2} + \sqrt{3}) b$. However, this is \textit{not} the case. 

    \begin{example}
      Given any element $\beta \in \mathbb{Q}[\sqrt{2} + \sqrt{3}]$, it is by definition of the form 
      \begin{equation}
        \beta = \sum_{k=0}^n a_k (\sqrt{2} + \sqrt{3})^k 
      \end{equation} 
      Clearly $1, \sqrt{2} + \sqrt{3} \in \mathbb{Q}[\sqrt{2} + \sqrt{3}]$ by mapping $\sqrt{2} + \sqrt{3}$ through the polynomials $f(x) = 1$ and $f(x) = $. However, we can see that $(\sqrt{2} + \sqrt{3})^2 = 5 + \sqrt{6}$,\footnote{where we use $\sqrt{6}$ as notation for $\sqrt{2} \cdot \sqrt{3}$} and so $\sqrt{6} \in \mathbb{Q}[\sqrt{2} + \sqrt{3}]$. Furthermore, we have $(\sqrt{2} + \sqrt{3})^3 = 11 \sqrt{2} + 9 \sqrt{3}$, and so with the ring properties we can conclude that 
      \begin{align}
        \frac{1}{2} \big[ (11 \sqrt{2} + 9 \sqrt{3}) - 9 (\sqrt{2} + \sqrt{3})\big] = \sqrt{2} & \in \mathbb{Q}[\sqrt{2} + \sqrt{3}] \\
        -\frac{1}{2} \big[ (11 \sqrt{2} + 9 \sqrt{3}) - 11 (\sqrt{2} + \sqrt{3})\big] = \sqrt{3} & \in \mathbb{Q}[\sqrt{2} + \sqrt{3}] \\
      \end{align} 
      If we go a bit further, we can show that 
      \begin{equation}
        \mathbb{Q}[\sqrt{2} + \sqrt{3}] = \{a + b \sqrt{2} + c \sqrt{3} + d\sqrt{6} \mid a, b, c, d \in \mathbb{Q} \}
      \end{equation}
    \end{example}

    This method in which we have taken higher powers of $\alpha$ to reveal elements in $\mathbb{Q}$ reveals a deeper structure of a finite-dimensional vector space, which will be useful for analyzing certain fields in the examples below. 

    \begin{lemma}[Vector Space Structure]
      $F[\alpha]$ is a finite-dimensional vector space over $F$. If $f(x) = a_n x^n + \ldots a_0$, then $S = \{1, \alpha, \ldots, \alpha^{n-1}\}$ spans $F[\alpha]$.\footnote{Note that this does not mean that it is a basis.} 
    \end{lemma}
    \begin{proof}
      An element of $F[\alpha]$ is of the form 
      \begin{equation}
        f(\alpha) = \sum_{k=0}^n a_k \alpha^k
      \end{equation} 
      for some $f \in F[x]$, and so it is immediate that $\{\alpha^k\}_{k \in \mathbb{N}_0}$ spans $F[\alpha]$. We claim that $\alpha^{n-1+i}$ is in $S$ for all $i > 0$. By induction, if $i = 1$, then 
      \begin{equation}
        \alpha^n = -\frac{1}{a_n} \big( a_{n-1} \alpha^{n-1} + \ldots + a_0 \big)
      \end{equation}
      which proves the claim. Now assume that $\alpha^n, \alpha^{n+1}, \ldots, \alpha^{n-1+i} \in \Span\{1, \ldots, \alpha^{n-1}\}$. Then 
      \begin{equation}
        \alpha^i f(\alpha) = 0 \implies a_n \alpha^{n+i} + \alpha_{n-1} \alpha^{n+i-1} + \ldots + a_0 \alpha^i = 0 
      \end{equation}
      and so 
      \begin{equation}
        \alpha^{n+i} = -\frac{1}{a_n} \big(a_{n-1} \alpha^{n+i-1} + \ldots + a_0 \alpha^i)
      \end{equation}
      which means that $\alpha^{n+i} \in \Span\{1, \ldots, \alpha^{n-1}\}$, completing the proof. 
    \end{proof} 

    Great, so we automatically have the ring and vector space structures on $F[\alpha]$. However, what we would really like is a field structure since that was our original goal. There is a sufficient condition for it to be a field. 

    \begin{theorem}[Adjoining Fields]
      Given fields $F \subset K$, if there exists a $f \in F[x]$ s.t. $\alpha \in K$ is a root of $f$, then $F[\alpha] \subset K$ is a field. To emphasize that it is a field, we usually denote it as $F(\alpha)$ and refer it as the field obtained by \textbf{adjoining} $\alpha$ to $F$. 
    \end{theorem}
    \begin{proof}
      It is clear that $F[\alpha]$ is a commutative ring since $F$ is a field. So it remains to show that every nonzero element of $\beta \in F[\alpha]$ is a unit. By definition $\beta = p(\alpha)$ for some polynomial $p \in F[x]$.  Factor $f \in F[x]$ as the product of irreducible polynomials. Then $\alpha$ must be a root of one of those irreducible factors, say $g(x)$. Note that $g(x) \nmid p(x)$ since $p(\alpha) \neq 0$. Since $g$ is irreducible, we know that $\gcd(g, p) = 1$ and so $\exists s, t \in F[x]$ s.t. 
      \begin{equation}
        1 = s p + t g \implies 1 = s(\alpha) p(\alpha) + t(\alpha) g(\alpha) = s(\alpha) p(\alpha)
      \end{equation}  
      Therefore we have found a multiplicative inverse $s = p^{-1} \in F[\alpha]$. 
    \end{proof} 
    \begin{proof}
      We can prove it using the vector space structure. Treating $F[\alpha]$as a finite-dimensional vector space over $F$, let us define the $F$-linear function\footnote{linearity is easy to check}
      \begin{equation}
        m_b: F[\alpha] \rightarrow F[\alpha], \qquad m_b (\beta) = b\beta
      \end{equation} 
      Since $F[\alpha] \subset K$, $F[\alpha]$ is an integral domain. Thus $\not\exists \beta \in F[\alpha] \setminus \{0\}$ s.t. $b \beta = 0$. This means that the kernel of $m_b$ is $0$, and so $m_b$ is injective. By the rank-nullity theorem, it is bijective, and so there exists a $\beta \in F[\alpha]$ s.t. $b \beta = 1 \implies b$ is a unit. 
    \end{proof}

    \begin{example}[$\mathbb{Q}\lbrack \sqrt{3} i\rbrack$ is a Field]
      $\mathbb{Q}[\sqrt{3} i]$ is a field, hence denoted $\mathbb{Q}(\sqrt{3} i)$ since $\sqrt{3}i$ is a root of the polynomial $f(x) = x^2 + 3$. 
    \end{example}

    \begin{example}[$\mathbb{Q}\lbrack \pi \rbrack$ not a Field]
      However, $\mathbb{Q}[\pi]$ is not a field. 
    \end{example} 

    \begin{example}[Finding Multiplicative Inverses of elements in $\mathbb{Q}\lbrack \alpha \rbrack$]
      Given $\beta = p(\alpha) = \alpha^2 + \alpha - 1 \in \mathbb{Q}[\alpha]$, where $\alpha$ is a root of $f(\alpha) = \alpha^3 + \alpha + 1$, we first know that $\beta$ must have a multiplicative inverse since $\mathbb{Q}[\alpha]$ is a field. Applying the Euclidean algorithm, we have 
      \begin{equation}
        1 = \frac{1}{3} \big\{ (x+1) f(x) - (x^2 + 2) p(x)\big\} = -\frac{1}{3} (\alpha^2 + 2) p(\alpha)
      \end{equation}
      and so $\beta^{-1} = (\alpha^2 + \alpha - 1)^{-1} = -\frac{1}{3} (\alpha^2 + 2)$. We can check that 
      \begin{align}
        -\frac{1}{3} (\alpha^2 + 2) (\alpha^2 + \alpha - 1) & = -\frac{1}{3} (\alpha^4 + \alpha^3 + \alpha^2 + 2 \alpha - 2) \\
                                                            & = -\frac{1}{3} (\alpha^3 + \alpha - 2) \\
                                                            & = -\frac{1}{3} (-3) = 1
      \end{align}
    \end{example}

    Intuitively, the extra $\alpha \in K$ allows us to ``expand'' our field $F$ into a bigger field of $K$. We can also define this for multivariate polynomials.  

    \begin{definition}[Ring of Multivariate Polynomial Elements]
      Given a polynomial ring $F[x, y]$ over a field $F$ and constants $\alpha, \beta \in F$, the following definitions are equivalent. 
      \begin{align}
        F[\alpha, \beta] & \coloneqq \{ f(\alpha, \beta) \in F \mid f \in F[x, y] \} \\ 
                         & = (F[\alpha])[\beta] \\
                         & = (F[\beta])[\alpha]
      \end{align}
    \end{definition}
    \begin{proof}
      
    \end{proof} 
    
    \begin{example}[Extensions of $\sqrt{2}$ and $i$]
      We claim that 
      \begin{equation}
        \mathbb{Q}[\sqrt{2}, i] = \{ a + b \sqrt{2} + ci + d(\sqrt{2} i) \mid a, b, c, d \in \mathbb{Q}\}
      \end{equation}
      From the previous example, we know that $\mathbb{Q}[\sqrt{2}]$ are all numbers of the form $a + b\sqrt{2}$. Now we take $i \in \mathbb{C}$ and map it through all polynomials with coefficients in $\mathbb{Z}[\sqrt{2}]$, which will be of form 
      \begin{equation}
        f(i) = (a_n + b_n \sqrt{2}) i^n + (a_{n-1} + b_{n-1}\sqrt{2}) i^{n-1} + \ldots + (a_2 + b_2 \sqrt{2}) i^2 + (a_1 + b_1 \sqrt{2}) i + (a_0 + b_0 \sqrt{2})
      \end{equation} 
      However, we can see that since $i^2 = -1$, we only need to consider up to degree 1 polynomials of form 
      \begin{equation}
        (a + b \sqrt{2}) + (c + d \sqrt{2}) i 
      \end{equation}
      which is clearly of the desired form. For the other way around, this is trivial since we can construct a linear polynomial as before. 
    \end{example} 

    \begin{example}
      We claim $\mathbb{Q}[\sqrt{3} + i] = \mathbb{Q}[\sqrt{3}, i]$. 
      \begin{enumerate}
        \item $\mathbb{Q}[\sqrt{3} + i] \subset \mathbb{Q}[\sqrt{3}, i]$
        \item $\mathbb{Q}[\sqrt{3} + i] \supset \mathbb{Q}[\sqrt{3}, i]$. Note that 
          \begin{align}
            (\sqrt{3} + i)^3 = 8i & \implies i \in \mathbb{Q}[\sqrt{3} + i] \\
                                  & \implies (\sqrt{3} + i) - i = \sqrt{3} \in \mathbb{Q}[\sqrt{3} + i] 
          \end{align}
          Therefore, $\mathbb{Q}[\sqrt{3} + i]$ contains the elements $1, \sqrt{3}, i$, which form the basis of $\mathbb{Q}[\sqrt{3}, i]$. 
      \end{enumerate}
    \end{example}

    \begin{example}[Extensions of $\sqrt{3}i$ and $\sqrt{3}, i$]
      We claim that $\mathbb{Q}[\sqrt{3} i] \subsetneq \mathbb{Q}[\sqrt{3}, i]$. 
      \begin{enumerate}
        \item We can see that $\{1, \sqrt{3}i \}$ span $\mathbb{Q}[\sqrt{3}i ]$ as a $\mathbb{Q}$-vector space. Therefore, 
        \begin{equation}
          \sqrt{3}, i \in \mathbb{Q}[\sqrt{3}, i] \implies \sqrt{3} i \in \mathbb{Q}[\sqrt{3}, i]
        \end{equation} 
        implies that $\mathbb{Q}[\sqrt{3} i] \subset \mathbb{Q}[\sqrt{3}, i]$. 

        \item To prove proper inclusion, we claim that $i \not\in \mathbb{Q}[\sqrt{3}i]$. Assuming that it can, we represent it in the basis $i = b_0 + b_1 \sqrt{3} i$, and so
        \begin{equation}
          -1 = (b_0 + b_1 \sqrt{3} i)^2 = (b_0^2 - 3b_1^2) + 2b0 b_1 \sqrt{3} i
        \end{equation}
        Therefore we must have $2b_0 b_1 \sqrt{3} = 0 \implies b_0$ or $b_1$ should be $0$. If $b_0 = 0$, then $b_0^2 - 3b_1^2 = -3 b_1^2 \implies b_1^2 = 1/3$, which is not possible since $b_1^2 \in \mathbb{Q}$. If $b_1 = 0$, then $b_0 - 3 b_1^2 = b_0^2 > 0$, and so it cannot be $-1$. 
      \end{enumerate}
    \end{example}

  \subsubsection{Splitting Fields}

  Now we return to the problem of taking a polynomial $f \in \mathbb{Q}[x]$ and finding the \textit{smallest} possible field $K \subset \mathbb{C}$ s.t. $f$ can be factored as a product of linear polynomials in $K[x]$. 

  \begin{definition}[Splitting Field]
    Given a field extension $F \subset K$ and a polynomial $f \in F[x]$, we say $f$ \textbf{splits} in $K$ if $f$ can be written as the product of linear polynomials in $K[x]$. If $f$ splits in $K$ and there exists no field $E$ s.t. $F \subsetneq E \subsetneq K$, then $K$ is called a \textbf{splitting field} of $f$.\footnote{i.e. the splitting field is the smallest field that splits $f$.} 
  \end{definition}

  \begin{example}[Simple Splitting Fields]
    We provide some simple examples to gain intuition. 
    \begin{enumerate}
      \item Let $f(x) = x^2 + 2x + 2 \in \mathbb{Q}[x]$. Then the roots of $f(x)$ are $-1 \pm i$, so 
      \begin{equation}
        f(x) = (x - (-1 + i)) (x - (-1 - i)) 
      \end{equation}
      and we can show that $\mathbb{Q}[-1 - i, -1+i] = \mathbb{Q}[i]$ is the splitting field of $f$. 

      \item Let $f(x) = x^2 - 2x - 1 \in \mathbb{Q}[x]$. The roots are $1 \pm \sqrt{2}$, and so 
      \begin{equation}
        f(x) = (x - (1 + \sqrt{2})) (x - (1 - \sqrt{2}))
      \end{equation}
      and so $\mathbb{Q}[\sqrt{2}]$ is the splitting field of $f$. 

      \item Let $f(x) = x^6 - 1 \in \mathbb{Q}[x]$. We can factor 
        \begin{equation}
          f(x) = (x-1) (x + 1) (x^2 + x + 1) (x^2 - x + 1)
        \end{equation} 
        and the non-rational roots are $\frac{\pm 1 \pm \sqrt{3} i}{2}$. Thus the splitting field of $f$ is $\mathbb{Q}[\sqrt{3} i]$. 
    \end{enumerate}
  \end{example}

  \begin{example}
    Let $f(x) = x^4 - 2 \in \mathbb{Q}[x]$. It follows that the roots are 
    \begin{equation}
      \{ \sqrt[4]{2}, \sqrt[4]{2}, -\sqrt[4]{2}, - \sqrt[4]{2} i \} = \Big\{ \sqrt[4]{2}, \sqrt[4]{2} e^{\frac{2\pi i}{4}}, \sqrt[4]{2} e^{\frac{4\pi i}{4}}, \sqrt[4]{2} e^{\frac{6\pi i}{4}} \Big\}
    \end{equation}
    thus the splitting field of $f$ is 
    \begin{equation}
      \mathbb{Q} \big( \sqrt[4]{2}, \sqrt[4]{2} e^{\frac{2\pi i}{4}}, \sqrt[4]{2} e^{\frac{4\pi i}{4}}, \sqrt[4]{2} e^{\frac{6\pi i}{4}} \big) \subset \mathbb{Q}(\sqrt[4]{2}, e^{\frac{2\pi i}{4}})
    \end{equation}
    since $\sqrt[4]{2} e^{\frac{m \pi i}{4}} \in \mathbb{Q}(\sqrt[4]{2}, e^{\frac{2\pi i}{4}})$. In fact, the two are equal, and to prove this we can see that since we are working in a field, 
    \begin{equation}
      e^{2 \pi i / 4} = \frac{\sqrt[4]{2} e^{2\pi i/4}}{\sqrt[4]{2}} \in \mathbb{Q} \big( \sqrt[4]{2}, \sqrt[4]{2} e^{\frac{2\pi i}{4}}, \sqrt[4]{2} e^{\frac{4\pi i}{4}}, \sqrt[4]{2} e^{\frac{6\pi i}{4}} \big) 
    \end{equation}
    which implies that $\sqrt[4]{2} \in \mathbb{Q} \big( \sqrt[4]{2}, \sqrt[4]{2} e^{\frac{2\pi i}{4}}, \sqrt[4]{2} e^{\frac{4\pi i}{4}}, \sqrt[4]{2} e^{\frac{6\pi i}{4}} \big)$. Therefore we can conclude that the splitting field is 
    \begin{equation}
      \mathbb{Q} \big( \sqrt[4]{2}, \sqrt[4]{2} e^{\frac{2\pi i}{4}}, \sqrt[4]{2} e^{\frac{4\pi i}{4}}, \sqrt[4]{2} e^{\frac{6\pi i}{4}} \big) = \mathbb{Q}(\sqrt[4]{2}, e^{\frac{2\pi i}{4}})
    \end{equation}
  \end{example} 

  \begin{theorem}[Descartes' Rule of Signs] 
    \label{thm:descartes}
    Let $f(x) = x^n + a_{n-1}x^{n-1} + \cdots + a_1x + a_0 \in \mathbb{R}[x]$. Let $C_+$ be the number of times the coefficients of $f(x)$ change signs (here we ignore the zero coefficients); let $Z_+$ be the number of positive roots of $f(x)$, counting multiplicities. Then $Z_+ \leq C_+$ and $Z_+ \equiv C_+ \pmod{2}$. Moreover, if we set $g(x) = f(-x)$, let $C_-$ be the number of times the coefficients of $g(x)$ change signs, and $Z_-$ the number of negative roots of $f(x)$. Then $Z_- \leq C_-$ and $Z_- \equiv C_- \pmod{2}$.
  \end{theorem}

  \begin{theorem}
    The number of positive roots of $f(x)$ is the same as the number of negative roots of $f(-x)$.
  \end{theorem}

  \begin{example}[Easy Way to Find Number of Positive Roots]
    Given $f(x) = x^5 + x^4 - x^2 - 1$, 
    \begin{enumerate}
      \item We have $C_+ = 1$. By Descartes' rule of signs, it must be the case that $Z_+ \leq 1$ and $Z_+ \equiv 1 \pmod{2} \implies Z_+ = 1$. 
      \item Since $f(-x) = -x^5 + x^4 - x^2 - 1$, we have $C_- = 2$, so $Z_- = 0$ or $2$. This is the best that we can do, though it turns out that it actually has $0$ negative roots.\footnote{On the other hand, $x^5 + 3x^3 - x^2 - 1$ has 2 negative roots.} 
    \end{enumerate}
  \end{example}

\subsection{Integer Polynomials} 

  Even though we have covered a more general theory of polynomials with rational coefficients, it is worthwhile to visit integer polynomials for two reasons. First, there are a few specialized theorems that allow us to easily determine reducibility in $\mathbb{Z}[x]$. Second, Gauss's lemma allows us to check for reducibility in $\mathbb{Q}[x]$ by checking for reducibility in $\mathbb{Z}[x]$, at which point we can abuse the specialized theorems we have developed. 

  \begin{theorem}[Rational Root Theorem]
    Let $a_n x^n + \ldots + a_0 \in \mathbb{Z}[x]$. If $r/s \in \mathbb{Q}$ with $\gcd(r, s) = 1$, then $r \mid a_0$ and $s \mid a_n$. 
  \end{theorem}
  \begin{proof}
    Given that $r/s$ is a root, we have 
    \begin{equation}
      a_n (r/s)^n + \ldots + a_0 = 0
    \end{equation}
    Multiplying by $s^n$, we get 
    \begin{equation}
      a_n r^n + a_{n-1} r^{n-1} s + \ldots + a_1 s^{n-1} r + a_0 s^n = 0
    \end{equation}
    and putting this equation on mod $r$ and mod $s$ implies that $r | a_0 s^n$ and $s | a_n r^n$, respectively. But since we assumed that $\gcd (r, s) = 1$, $r | a_0$ and $s | a_n$. 
  \end{proof}

  The next is quite a remarkable result, since it says that decompositions in $\mathbb{Q}[x]$ imply decompositions in $\mathbb{Z}[x]$! Therefore, to check irreducibility in $\mathbb{Q}[x]$, it suffices to check irreducibility in $\mathbb{Z}[x]$. 

  \begin{lemma}[Gauss's Lemma]
    Let $f \in \mathbb{Z}[x]$. If $\exists g, h \in \mathbb{Q}[x]$ s.t. $f(x) = g(x) h(x)$, then $\exists \bar{g}, \bar{h} \in \mathbb{Z}[x]$ s.t. $f(x) = \bar{g}(x) \bar{h}(x)$. 
  \end{lemma}
  \begin{proof}
    We can find $k, l \in \mathbb{Z}$ s.t. $g_1 (x) = k g(x)$ and $h_1 (x) = l h(x)$ have integer coefficients, i.e. $g_1, h_1 \in \mathbb{Z}[x]$. Then, $k l f(x) = g_1 (x) h_1 (x) \in \mathbb{Z}[x]$. Let $p$ be a prime factor of $kl$. We have 
    \begin{equation}
      0 \equiv \bar{k} \bar{l} \bar{f} (x) \equiv \bar{g}_1 (x) \bar{h}_1 (x) \text{ in } \mathbb{Z}_p [x]
    \end{equation}
    Since $\mathbb{Z}_p$ is an integral domain, $\mathbb{Z}_p [x]$ is an integral domain, and so $\bar{g}_1$ or $\bar{h}_1$ must be $0$. WLOG let it be $\bar{g}_1$. Then every coefficient of $g_1 (x)$ is divisible by $p$, and we can write it in the form $g_2(x) = p g_1 (x)$. Therefore, 
    \begin{equation}
      p(x) \cdot \frac{kl}{p} = \underbrace{\frac{g_1 (x)}{p}}_{g_2 (x)} \cdot \underbrace{h_1 (x)}_{h_2 (x)} \iff f(x) \frac{kl}{p} = g_2 (x) h_2 (x)
    \end{equation}
    Since there are only finitely many prime divisors, we do this for all prime factors of $kl$, and we have 
    \begin{equation}
      f(x) = g_n (x) h_n (x), \qquad g_n, h_n \in \mathbb{Z}[x]
    \end{equation}
  \end{proof}

  \begin{example}[Reducibility of Integer Polynomials]
    Let $f(x) = x^4 - x^3 + 2$. The rational roots are in the set $S = \{\pm 1, \pm2 \}$, but none of them work since $f(\pm1), f(\pm2) \neq 0$. By degree considerations and Gauss's lemma, if $f(x)$ is reducible, then 
    \begin{equation}
      f(x) = (x^2 + ax + b) (x^2 + cx + d), \qquad a, b, c, d \in \mathbb{Z}
    \end{equation}
    We know that $bd \in S$, with $a + c = -1$, $d + b + ac = 0$, and so on for each coefficients. We can brute force this finite set of possibilities. 
  \end{example}

  A great way to check irreducibility is to check in mod $p$. 

  \begin{theorem}
    Let $f(x) = a_n x^n + \ldots + a_0 \in \mathbb{Z}[x]$. If $p \nmid a_n$ and $f \in \mathbb{Z}_p [x]$ is irreducible, then $f$ is irreducible in $\mathbb{Q}[x]$.\footnote{May need to verify this again.}
  \end{theorem}
  \begin{proof}
    Suppose that $f(x) = g(x) h(x) \in \mathbb{Z}[x]$ with $\deg(g), \deg(h) > 0$. Then 
    \begin{equation}
      f(x) \equiv g(x) h(x) \text{ in } \mathbb{Z}_p [x]
    \end{equation}
    Since $f(x)$ is irreducible in $\mathbb{Z}_p [x]$, we must have that one of $g(x)$ or $h(x)$ has degree $0$ in $\mathbb{Z}_p [x]$. WLOG let it be $g(x)$, but this means that the leading coefficient of $g(x)$ must be divisible by $p \implies$ leading coefficient of $f(x)$ is divisible by $p \iff p \mid a_n$. 
  \end{proof}

  \begin{example}
    $x^4 + x + 1$ is irreducible in $\mathbb{Z}_2 [x]$. So we can extend this to $\mathbb{Z}[x]$ to see that \textit{all} fourth degree polynomials of form $a x^4 + b x^3 + c x^2 + dx + e$, which $a, d, e$ odd and $b, c$ even is irreducible in $\mathbb{Q}[x]$. 
  \end{example}

  This is a powerful theorem to quickly find a large class of polynomials that are irreducible. However, being reducible in $\mathbb{Z}_p [x]$ does not imply reducibility in $\mathbb{Q}$. In fact, there are polynomials $f(x) \in \mathbb{Z}[x]$ which are irreducible but reducible in $\mathbb{Z}_p$ for \textit{every} prime $p$. 

  \begin{theorem}[Eisenstein's Criterion]
    Let $f(x) = a_n x^n + \ldots + a_0 \in \mathbb{Z}[x]$ and $p \in \mathbb{Z}$ a prime s.t. $p \nmid a_n$, $p \mid a_i$ for $i = 0, \ldots, a_{n-1}$, and $p^2 \nmid a_0$. Then $f(x)$ is irreducible in $\mathbb{Q}[x]$. 
  \end{theorem}
  \begin{proof}
    Suppose that $f(x) = g(x) h(x) \in \mathbb{Q}[x]$ with $\deg(g), \deg(h) > 0$. Then, by Gauss's lemma, $g, h \in \mathbb{Z}[x]$. Reducing the equations mod $p$, 
    \begin{equation}
      f(x) = g(x) h(x) \text{ in } \mathbb{Z}_p [x]
    \end{equation}
    But $f(x) = a_n x^n$. By unique factorization theorem in $\mathbb{Z}_p [x]$, $g, h \in \mathbb{Z}_p [x]$ must be products of units and prime factors of $a_n x^n$, which are $\{x\}$. Therefore, let 
    \begin{equation}
      g(x) = b_m x^m, h(x) = \frac{a_n}{b_m} x^{n-m} \in \mathbb{Z}_p [x]
    \end{equation}
    with $\deg(g) = m > 0$ and $\deg(h) = n - m > 0$ in $\mathbb{Z}[x]$. This implies that the constant coefficients of $g(x), h(x)$ are divisible by $p$, which implies that the constant coefficients of $f(x) = g(x) h(x)$ are divisible by $p^2$, a contradiction. 
  \end{proof}

  \begin{example}
    Listed. 
    \begin{enumerate}
      \item $x^{13} + 2x^{10} + 4x + 6$ is divisible by Eisenstein for $p = 2$. 
      \item $x^3 + 9x^2 + 12x + 3$ is divisible by Eisenstein for $p = 3$. 
    \end{enumerate}
  \end{example}

  \begin{example}
    Let $f(x) = x^4 + x^3 + x^2 + x + 1$. Then, we know that $f(x) = \frac{x^5 - 1}{x-1}$ and so 
    \begin{align}
      f(x + 1) & = \frac{(x + 1)^5 - 1}{(x + 1) - 1} \\
               & = \frac{1}{x} \bigg( x^5 + \binom{5}{1} x^4 + \binom{5}{2} x^3 + \binom{5}{3} x^2 + \binom{5}{4} x + \binom{5}{5} - 1 \bigg) 
               & = x^4 + 5x^3 + 10 x^2 + 10x + 5
    \end{align}
    So all nonleading coefficients are divisible by $5$ exactly once, which by Eisenstein implies that $f(x+1)$ is irreducible which implies that $f(x)$ is irreducible. 
  \end{example}

  We have prod that for $\alpha \in \mathbb{C}$, subfield $F \subset \mathbb{C}$, and $f(x) \in F[x]$, with $f(\alpha) = 0$, then $B = \{1, \alpha, \ldots, \alpha^{\deg(f) - 1}\}$ spans $F[\alpha]$ as a $F$-vector space. If $f(x)$ is irreducible then $B$ is a basis. 

\subsection{Exercises}

  \begin{exercise}[Shifrin 3.1.2.c/d]
    Find the greatest common divisors $d(x)$ of the following polynomials $f(x), g(x) \in F[x]$, and express $d(x)$ as $s(x)f(x) + t(x)g(x)$ for appropriate $s(x), t(x) \in F[x]$:
    \begin{enumerate}
      \item $f(x) = x^3 - 1$, $g(x) = x^4 + x^3 - x^2 - 2x - 2$, $F = \mathbb{Q}$
      \item $f(x) = x^2 + (1 - \sqrt{2})x - \sqrt{2}$, $g(x) = x^2 - 2$, $F = \mathbb{R}$
      \item $f(x) = x^2 + 1$, $g(x) = x^2 - i + 2$, $F = \mathbb{C}$
      \item $f(x) = x^2 + 2x + 2$, $g(x) = x^2 + 1$, $F = \mathbb{Q}$
      \item $f(x) = x^2 + 2x + 2$, $g(x) = x^2 + 1$, $F = \mathbb{C}$
    \end{enumerate}
  \end{exercise}
  \begin{solution}
    For (c), the gcd is $1$, with 
    \begin{equation} 
      -\frac{1}{1 - i} (x^2 + 1) + \frac{1}{1 - i} (x^2 - i + 2) = \frac{1}{1-i} (x^2 - i + 2 - x^2 - 1) = \frac{1}{1-i} (1 - i) = 1
    \end{equation}
    where $1/(1-i) = (1 + i)/2$. For (d), the gcd is $1$, with 
    \begin{align}
      \frac{1}{5} (2x + 3) (x^2 + 1) & + \frac{1}{5} (1 - 2x) (x^2 + 2x + 2) \\
                                          & = \frac{1}{5} (2x^3 + 3x^2 + 2x + 3) + \frac{1}{5} (-2x^3 - 3x^2 - 2x + 2) = 1
    \end{align}
  \end{solution}

  \begin{exercise}[Shifrin 3.1.6]
    Prove that if $F$ is a field, $f(x) \in F[x]$, and $\mathrm{deg}(f(x)) = n$, then $f(x)$ has at most $n$ roots in $F$. 
  \end{exercise}
  \begin{solution}
    We start when $n=1$. Then $f(x) = mx + b$ and we claim that the only root is $x = -b/m$ since we can solve for $0 = mx + b$ with the field operations, which leads to a unique solution. This implies by corr 1.5 that $(x + b/m)$ is the only factor of $f$. Now suppose this holds true for some degree $n-1$ and let us have a degree $n$ polynomial $f$. Assume that some $c$ is a root of $f$ (if there exists no $c$, then we are trivially done), which means $(x - c)$ is a factor of $f$, and we can write 
    \begin{equation}
      f(x) = (x - c) \, g(x)
    \end{equation}
    for some polynomial $g(x)$ of degree $n-1$. By our inductive hypothesis, $g(x)$ must have at most $n-1$ roots, and so $f$ has at most $n$ roots. 
  \end{solution}

  \begin{exercise}[Shifrin 3.1.8]
    Let $F$ be a field. Prove that if $f(x) \in F[x]$ is a polynomial of degree $2$ or $3$, then $f(x)$ is irreducible in $F[x]$ if and only if $f(x)$ has no root in $F$.
  \end{exercise}
  \begin{solution}
    We prove bidirectionally. 
    \begin{enumerate}
      \item $(\rightarrow)$. Let $f$ be irreducible. Then it cannot be factored into polynomials $p(x) q(x)$ where $\mathrm{deg}(p) + \mathrm{deg}(q) = n$. Note that two positive integers adding up to $2$ or $3$ means that at least one of the integers must be $1$, by the pigeonhole principle. This means that $f$ irreducible is equivalent to saying that $f$ does not have linear factors of form $(x-c)$, which by corollary 1.5 implies that there exists no root $c$ for $f(x)$. 
      \item $(\leftarrow)$. Let $f$ have no root in $F$. Then by corollary 1.5 there exists no linear factors $(x-c)$. By the same pigeonhole principle argument, we know that having a linear factor for degree 2 or 3 polynomials is equivalent to having (general) factors, and so $f$ has no factors. Therefore $f$ is irreducible. 
    \end{enumerate}
  \end{solution}

  \begin{exercise}[Shifrin 3.1.13]
    List all the irreducible polynomials in $\mathbb{Z}_2[x]$ of degree $\leq 4$. Factor $f(x) = x^7 + 1$ as a product of irreducible polynomials in $\mathbb{Z}_2[x]$.
  \end{exercise}
  \begin{solution}
    Listed by degree. 
    \begin{enumerate}
      \item $1$: $x, x + 1$. 
      \item $2$: $x^2 + x + 1$. 
      \item $3$: $x^3 + x^2 + 1, x^3 + x + 1$. 
      \item $4$: $x^4 + x + 1, x^4 + x^3 + 1, x^4 + x^3 + x^2 + x + 1$. 
    \end{enumerate}
    We have 
    \begin{align}
      x^7 + 1 & = (x + 1)(x^6 + x^5 + x^4 + x^3 + x^2 + x + 1) \\
              & = (x + 1) (x^3 + x + 1) (x^3 + x^2 + 1)
    \end{align}
  \end{solution}


  \begin{exercise}[Shifrin 3.2.2.b/c]
    Prove that
    \begin{enumerate}
      \item $\mathbb{Q}[\sqrt{2}, i] = \mathbb{Q}[\sqrt{2} + i]$, but $\mathbb{Q}[\sqrt{2}i] \subsetneq \mathbb{Q}[\sqrt{2}, i]$
      \item $\mathbb{Q}[\sqrt{2}, \sqrt{3}] = \mathbb{Q}[\sqrt{2} + \sqrt{3}]$, but $\mathbb{Q}[\sqrt{6}] \subsetneq \mathbb{Q}[\sqrt{2}, \sqrt{3}]$
      \item $\mathbb{Q}[\sqrt[3]{2} + i] = \mathbb{Q}[\sqrt[3]{2}, i]$; what about $\mathbb{Q}[\sqrt[3]{2}i] \subset \mathbb{Q}[\sqrt[3]{2}, i]$?
    \end{enumerate}
  \end{exercise}
  \begin{solution}[Shifrin 3.2.2.b]
    From Shifrin, I use the fact that $\mathbb{Q}[\sqrt{2}] = \{ a + b \sqrt{2} \mid a, b \in \mathbb{Q}\}$, and the same proof immediately shows that $\mathbb{Q}[\sqrt{3}] = \{ a + b \sqrt{3} \mid a, b \in \mathbb{Q}\}$ along with that for $\mathbb{Q}[\sqrt{6}]$. As for $\mathbb{Q}[\sqrt{2}, \sqrt{3}]$, I also follow the same logic to show 
    \begin{align}
      \mathbb{Q}[\sqrt{2}, \sqrt{3}] & = \mathbb{Q}[\sqrt{2}][\sqrt{3}] \\
                                     & = \{\alpha + \beta \sqrt{3} \mid a, b \in \mathbb{Q}[\sqrt{2}]\} \\
                                     & = \{ (a + b\sqrt{2}) + (c + d \sqrt{2}) \sqrt{3} \mid a, b, c, d \in \mathbb{Q} \} \\
                                     & = \{ a + b\sqrt{2} + c \sqrt{3} + d \sqrt{6} \mid a, b, c, d \in \mathbb{Q} \} 
    \end{align}
    Where $\sqrt{2} \times \sqrt{3} = \sqrt{2 \times 3} = \sqrt{6}$ follows from the definition of $n$th roots plus associativity on the reals. For (b), we prove bidirectionally.
    \begin{enumerate}
      \item $\mathbb{Q}[ \sqrt{2} + \sqrt{3}] \subset \mathbb{Q}[\sqrt{2}, \sqrt{3}]$. Consider $y \in \mathbb{Q}[\sqrt{2} + \sqrt{3}]$. Then there exists $p \in \mathbb{Q}[x]$ s.t. 
      \begin{equation}
        y = p(\sqrt{2} + \sqrt{3}) = a_n (\sqrt{2} + \sqrt{3})^n + \ldots + a_1 (\sqrt{2} + \sqrt{3}) + a_0
      \end{equation}
      where the terms can be expanded an rearranged to the form $a + b \sqrt{2} + c \sqrt{3} + d \sqrt{6} \in \mathbb{Q}[\sqrt{2}, \sqrt{3}]$. 

    \item $\mathbb{Q}[\sqrt{2}, \sqrt{3}] \subset \mathbb{Q}[ \sqrt{2} + \sqrt{3}]$. Consider $\sqrt{2} + \sqrt{3} \in \mathbb{Q}[\sqrt{2} + \sqrt{3}]$. Since it is a field and $\sqrt{2} + \sqrt{3}$ is a unit, by rationalizing the denominator, we can get 
      \begin{equation}
        (\sqrt{2} + \sqrt{3})^{-1} = \frac{\sqrt{2} - \sqrt{3}}{2 - 3} = \sqrt{3} - \sqrt{2} \in \mathbb{Q}[\sqrt{2} + \sqrt{3}]
      \end{equation}
      Therefore by adding and subtracting the two elements, we have $\sqrt{2}, \sqrt{3} \in \mathbb{Q}[\sqrt{2} + \sqrt{3}] \implies \sqrt{6} \in \mathbb{Q}[\sqrt{2} + \sqrt{3}]$. Since $\mathbb{Q} \subset \mathbb{Q}[\sqrt{2} + \sqrt{3}]$, from the ring properties all elements of the form $a + b \sqrt{2} + c \sqrt{3} + d \sqrt{6} \in \mathbb{Q}[\sqrt{2} + \sqrt{3}]$. 
    \end{enumerate}

    For the second part, I claim that $\sqrt{2} \not\in \mathbb{Q}[\sqrt{6}]$. Assuming it is, we have $\sqrt{2} = a + b \sqrt{6} \implies 2 = a^2 + 6b^2 + 2ab \sqrt{6}$. So $a = 0$ or $b = 0$. If $a = 0$, then $b^2 = 1/3 \implies b = 1/\sqrt{3}$ which contradicts that $b$ is rational. If $b = 0$, then $a^2 = 2 \implies a = \sqrt{2}$ which contradicts that $a$ is rational. 
  \end{solution}

  \begin{solution}[Shifrin 3.2.2.c]
    Note that $\mathbb{Q}[\sqrt[3]{2}] = \{a + b \sqrt[3]{2} + c \sqrt[3]{4}\}$, and so 
    \begin{align}
      \mathbb{Q}[\sqrt[3]{2}, i] & = \mathbb{Q}[\sqrt[3]{2}][i] \\
                                 & = \{\alpha + \beta i \mid \alpha, \beta \in \mathbb{Q}[\sqrt[3]{2}]\} \\
                                 & = \{ (a + b \sqrt[3]{2} + c \sqrt[3]{4}) + (d + e \sqrt[3]{2} + f \sqrt[3]{4}) i \mid a, b, c, d, e, f \in \mathbb{Q}\} \\
                                 & = \{ a + b \sqrt[3]{2} + c \sqrt[3]{4} + d i + e \sqrt[3]{2} i + f \sqrt[3]{4} i \mid a, b, c, d, e, f \in \mathbb{Q}\}
    \end{align}
    We prove bidirectionally. 
    \begin{enumerate}
      \item $\mathbb{Q}[\sqrt[3]{2} + i] \subset \mathbb{Q}[\sqrt[3]{2}, i]$. Consider $y \in \mathbb{Q}[\sqrt[3]{2} + i]$. Then there exists a $p \in \mathbb{Q}[x]$ s.t. 
      \begin{equation}
        y = p(\sqrt[3]{2} + i) = a_n (\sqrt[3]{2} + i)^n + \ldots + a_1 (\sqrt[3]{2} + i) + a_0
      \end{equation}
      Then we can expand and rearrange the terms to be of the form 
      \begin{equation}
        a + b \sqrt[3]{2} + c \sqrt[3]{4} + d i + e i \sqrt[3]{2} + f i \sqrt[3]{4} \in \mathbb{Q}[\sqrt[3]{2}, i]
      \end{equation}

      \item $\mathbb{Q}[\sqrt[3]{2}, i] \subset \mathbb{Q}[\sqrt[3]{2} + i]$. Consider $\alpha = \sqrt[3]{2} + i \in \mathbb{Q}[\sqrt[3]{2} + i]$. Then $(\alpha - i)^3 = 2$. Therefore 
      \begin{align}
        \alpha^3 - 3 \alpha^2 i - 3 \alpha + i = 2 & \implies i(1 - 3 \alpha^2) = 2 + 3 \alpha - \alpha^3 \\ 
                                                   & \implies i = \frac{2 + 3 \alpha - \alpha^3}{1 - 3 \alpha^2} \in \mathbb{Q}[\sqrt[3]{2} + i]
      \end{align}
      Therefore $\sqrt[3]{2} = \alpha - i \in \mathbb{Q}[\sqrt[3]{2} + i]$, which allows us add all combinations $\{1, \sqrt[3]{2}, \sqrt[3]{4}, i, \sqrt[3]{2} i, \sqrt[3]{4} i\}$ into our basis. 
    \end{enumerate}
  \end{solution}

  \begin{exercise}[Shifrin 3.2.6.b/c/d/g]
    Suppose $\alpha \in \mathbb{C}$ is a root of the given irreducible polynomial $f(x) \in \mathbb{Q}[x]$. Find the multiplicative inverse of $\beta \in \mathbb{Q}[\alpha]$.
    \begin{enumerate}
      \item $f(x) = x^2 + 3x - 3$, $\beta = \alpha - 1$ 
      \item $f(x) = x^3 + x^2 - 2x - 1$, $\beta = \alpha + 1$
      \item $f(x) = x^3 + x^2 + 2x + 1$, $\beta = \alpha^2 + 1$
      \item $f(x) = x^3 - 2$, $\beta = \alpha + 1$
      \item $f(x) = x^3 + x^2 - x + 1$, $\beta = \alpha + 2$
      \item $f(x) = x^3 - 2$, $\beta = r + s\alpha + t\alpha^2$
      \item $f(x) = x^4 + x^2 - 1$, $\beta = \alpha^3 + \alpha - 1$
    \end{enumerate}
  \end{exercise}
  \begin{solution}
    For (b), using the Euclidean algorithm gives 
    \begin{equation}
      (1) (x^3 + x^2 - 2x - 1) + (-x^2 + 2) (x + 1) = 1 
    \end{equation}
    and substituting the root $\alpha$ gives $(-\alpha^2 + 2)(\alpha + 1) = 1$. So we have $\beta^{-1} = -\alpha^2 + 2$.  
    For (c), doing the same thing gives 
    \begin{equation}
      (-x) (x^3 + x^2 + 2x + 1) + (x^2 + x + 1)(x^2 + 1) = 1
    \end{equation}
    and substituting $\alpha$ gives $(\alpha^2 + \alpha + 1)(\alpha^2 + 1) = 1$, so $\beta^{-1} = \alpha^2 + \alpha + 1$. 
    For (d), we have 
    \begin{equation}
      (-\frac{1}{3}) (x^3 - 2) + (\frac{1}{3} x^2 - \frac{1}{3} x + \frac{1}{3}) (x + 1) = 1 
    \end{equation}
    and so substituting $\alpha$ gives $(\frac{1}{3} \alpha^2 - \frac{1}{3} \alpha + \frac{1}{3}) (\alpha + 1) = 1$, so $\beta^{-1} = \frac{1}{3} \alpha^2 - \frac{1}{3} \alpha + \frac{1}{3}$. For (g), we have 
    \begin{equation}
      (-x^2 - x - 2) (x^4 + x^2 - 1) + (x^3 + x^2 + 2x + 1) (x^3 + x - 1) = 1
    \end{equation}
    and so substituting $\alpha$ gives $(\alpha^3 + \alpha^2 + 2\alpha + 1) (\alpha^3 + \alpha - 1) = 1$, and so $\beta^{-1} = \alpha^3 + \alpha^2 + 2\alpha + 1$. 
  \end{solution}

  \begin{exercise}[Shifrin 3.2.7]
    Let $f(x) \in \mathbb{R}[x]$.
    \begin{enumerate}
      \item Prove that the complex roots of $f(x)$ come in ``conjugate pairs''; i.e., $\alpha \in \mathbb{C}$ is a root of $f(x)$ if and only if $\overline{\alpha}$ is also a root.
      \item Prove that the only irreducible polynomials in $\mathbb{R}[x]$ are linear polynomials and quadratic polynomials $ax^2 + bx + c$ with $b^2 - 4ac < 0$.
    \end{enumerate}
  \end{exercise}
  \begin{solution}
    Listed. 
    \begin{enumerate}
      \item If $\alpha \in \mathbb{C}$ is a root of $f$, then 
      \begin{equation}
        0 = f(\alpha) = a_n \alpha^n + \ldots + a_1 \alpha + a_0
      \end{equation}
      for $a_i \in \mathbb{R}$. Since 
      \begin{align}
        0 = \overline{0} & = \overline{f(\alpha)} \\
                         & = \overline{a_n \alpha^n + \ldots + a_1 \alpha + a_0} \\
                         & = \overline{a_n} \overline{\alpha^n} + \ldots + \overline{a_1} \overline{\alpha} + \overline{a_0} \\
                         & = a_n \overline{\alpha}^n + \ldots + a_1 \overline{\alpha} + a_0 \\
                         & = p(\overline{\alpha})
      \end{align} 
      we can see that $\overline{\alpha} \in \mathbb{C}$ is immediately a root as well. Since $\overline{\overline{\alpha}} = \alpha$, the converse is immediately proven. 

      \item Linear polynomials in $F[x]$ for a given field are trivially irreducible (since multiplying polynomials increases the degree of the product as there are no zero divisors in a field). Perhaps without Theorem 4.1, we can assume that a real quadratic polynomial $p(x) = ax^2 + bx + c$ is reducible, which is equivalent to 
      \begin{equation}
        p(x) = (dx + e)(fx + g) = dfx^2 + (dg + ef) x + eg 
      \end{equation}
      For $d, e, f, g \in \mathbb{R}$, and evaluating $b^2 - 4ac = (dg + ef)^2 - 4dfeg = (dg - ef)^2 \geq 0$ since this is a squared term of a real number. So we have proved that if it is quadratic and reducible, then the discriminant $\geq 0$. To prove the other way, we assume that it is not reducible, i.e. there exists some complex root $\alpha$ from the fundamental theorem of algebra. Then from (1), we know that $\overline{\alpha}$ must also be a complex conjugate. Then this is reducible in $\mathbb{C}$ as 
      \begin{equation}
        p(x) = a (x - \alpha) (x - \overline{\alpha}) 
      \end{equation}
      for some constant factor $a$. Letting $\alpha = d + ei$ for $d, e \in \mathbb{R}$, expanding it gives us 
      \begin{align}
        p(x) & = a \big( x^2 - (\alpha + \overline{\alpha}) x + \alpha \overline{\alpha} \big) \\
             & = a x^2 + - 2 a d x + a(d^2 + e^2)
      \end{align}
      and evaluating the discriminant gives  
      \begin{equation}
        4a^2 d^2 - 4 a^2 (d^2 + e^2) = -4 a^2 e^2 < 0
      \end{equation}
      and we are done. For higher degree polynomials, we can proceed by taking a complex root (which is guaranteed to exist by fundamental theorem of algebra). If it contains an imaginary term, then its conjugate is also a root, and we factor out the quadratic. If it is real, then we can factor out the linear term. We can keep going this until we hit our base cases of a quadratic or linear term. 
    \end{enumerate}
  \end{solution}

  \begin{exercise}[Shifrin 3.2.13]
    Let $K$ be a field extension of $F$, and suppose $\alpha, \beta \in K$. Show that $(F[\alpha])[\beta] = (F[\beta])[\alpha]$, so that $F[\alpha, \beta]$ makes good sense.
    
    (Remark: One way to do this is to think about the ring of polynomials in two variables. The other way is just to show directly that every element of one ring belongs to the other.)
  \end{exercise}
  \begin{solution}
    Let $y \in (F[\alpha])[\beta]$. Then there exists a polynomial $p \in (F[\alpha])[x]$ s.t. 
    \begin{equation}
      y = p(\beta) = b_n \beta^n + \ldots + b_1 \beta + b_0 = \sum_{i=0}^n b_i \beta^i 
    \end{equation}
    for $b_i \in F[\alpha]$. But since $b_i \in F[\alpha]$, there exists a polynomial $q_i \in F[x]$ s.t. (omitting the subscript $i$ for clarity)
    \begin{equation}
      b_i = q_i (\alpha) = a_{n_i} \alpha^n + \ldots + a_1 \alpha + a_0 = \sum_{j=0}^{n_i} a_{j} \alpha^j 
    \end{equation}
    for $a_j \in F$. Substituting each $b_i$ in gives   
    \begin{equation}
      y = \sum_{i=0}^n \bigg( \sum_{j=0}^{n_i} a_j \alpha^j \bigg) \beta^i = \sum_{i=0}^n \sum_{j=0}^{n_i} a_j \alpha^j \beta^i
    \end{equation}
    With the same logic, every element of $(F[\beta])[\alpha]$ can be written as 
    \begin{equation}
      y = \sum_{i=0}^n \bigg( \sum_{j=0}^{n_i} a_j \beta^j \bigg) \alpha^i = \sum_{i=0}^n \sum_{j=0}^{n_i} a_j \alpha^i \beta^j
    \end{equation}
    Note that since $F[\alpha]$ is a vector space spanned by $\{1, \ldots, \alpha^{n-1}\}$, and $F[\beta]$ is a also a vector space spanned by $\{1, \ldots, \beta^{m-1}\}$ for some $m$, the two spaces above are spanned by all products $\{\alpha^i \beta^j\}_{i < n, j < m}$, and they are the same set. 
  \end{solution}

  \begin{exercise}[Shifrin 3.3.2.a/d/e/g]
    Decide which of the following polynomials are irreducible in
    $\mathbb{Q}[x]$.
    \begin{enumerate}
      \item[a] $f(x) = x^3 + 4x^2 - 3x + 5$
      \item $f(x) = 4x^4 - 6x^2 + 6x - 12$
      \item $f(x) = x^3 + x^2 + x + 1$
      \item[d] $f(x) = x^4 - 180$
      \item[e] $f(x) = x^4 + x^2 - 6$
      \item $f(x) = x^4 - 2x^3 + x^2 + 1$
      \item[g] $f(x) = x^3 + 17x + 36$
      \item $f(x) = x^4 + x + 1$
      \item $f(x) = x^5 + x^3 + x^2 + 1$
      \item $f(x) = x^5 + x^3 + x + 1$
    \end{enumerate}
  \end{exercise}
  \begin{solution}
    For (a), by the rational root theorem the rational roots, if any, must be in the set $\{\pm 1, \pm 5\}$. Calculating them gives $f(x) = 7, 11, 215, -5$. Since this is third degree, no linear factors means that it is irreducible, so $f$ is irreducible. 

    For (d), by the Eisenstein's criterion with $p = 5$ this polynomial is irreducible. 

    For (e), the rational root theorem states that the rational roots must be in $\{\pm 1, \pm 2, \pm 3, \pm 6\}$. This polynomial is clearly even, so it suffices to check the positive candidates. This gives $-4, 14, 84, 1326$. Therefore if it is reducible, by Gauss's lemma it must be of the form 
    \begin{equation}
      (ax^2 + bx + c)(dx^2 + ex + f)
    \end{equation} 
    for integer coefficients. $a = d = 1$ is trivial ($-1, -1$ is also possible but constant factors don't matter). Expanding this gives 
    \begin{equation}
      x^4 + (b + e) x^3 + (c + f + be) x^2 + (bf + ce) x + cf = x^4 + x^2 - 6
    \end{equation}
    The coefficients of $x^3$ tell us that $e = -b$, which means that for the coefficents of $x$, $bf + ce = bf - bc = 0 \implies f = c$. So $c^2 = -6$, which has no solution. Therefore $f$ is irreducible. 

    For (g), we must check rational roots of $\{\pm1, \pm2, \pm3, \pm4, \pm6, \pm9, \pm12, \pm18, \pm36\}$. Since this polynomial is monotonically increasing, with $f(-2) = -6$ and $f(0) = 36$. It only suffices to check $x = -1$, which gives $f(-1) = 18$. Therefore there are no linear factors. Since this is third degree, no linear factors means that it is irreducible, so $f$ is irreducible. 
  \end{solution}

  \begin{exercise}[Shifrin 3.3.4]
    Show that each of the following polynomials has no rational root:
    \begin{enumerate}
      \item $x^{200} - x^{41} + 4x + 1$
      \item $x^8 - 54$
      \item $x^{2k} + 3x^{k+1} - 12$, $k \geq 1$
    \end{enumerate}
  \end{exercise}
  \begin{solution}
    Listed. 
    \begin{enumerate}
      \item By the rational root theorem, the only possible rational roots are $\pm1$. Solving for both of these values gives 
      \begin{align}
        f(1) & = 1 - 1 + 4 + 1 = 5 \\ 
        f(-1)& = 1 + 1 - 4 + 1 = -1
      \end{align}
      Therefore there are no rational roots. 

      \item The only possible rational roots are $\pm 1, \pm 2, \pm 3, \pm 6, \pm 9, \pm 18, \pm 27, \pm 54$. But this polynomial is even, so it suffices to check the positive roots. $f(1) = -53$, $f(2) = 256 - 54 = 202$, and any greater inputs will increase the output since $f$ is monotonic in $\mathbb{Z}^+$. Therefore $f$ has no rational roots. 

      \item By Eisenstein's criterion with $p = 3$, this polynomial is irreducible and therefore has no rational roots. 
    \end{enumerate}
  \end{solution}

  \begin{exercise}[Shifrin 3.3.6]
    Listed. 
    \begin{enumerate}
      \item Prove that $f(x) \in \mathbb{Z}_2[x]$ has $x + 1$ as a factor if and only if it has an even number of nonzero coefficients.
      \item List the irreducible polynomials in $\mathbb{Z}_2[x]$ of degrees $2, 3, 4$, and $5$.
    \end{enumerate}
  \end{exercise}
  \begin{solution}
    Listed. 
    Since $f(x)$ has $x + 1$ as a factor iff 
    \begin{equation}
      f(1) = a_n 1^n + \ldots + a_1 1^1 + a_0 = a_n + \ldots + a_1 + a_0 = 0
    \end{equation}
    where each $a_i \in \{0, 1\}$. Therefore, this is equivalent to saying that there are an even number of $1$'s (nonzero coefficients), which sum to $0$ mod 2. Therefore, the irreducible polynomials should at least have a constant coefficient of $1$ (so we can't factor $x$) and should have odd number of terms (so that we can't factor $x+1$). This will guarantee that $f(0) = f(1) = 1$. 
    \begin{enumerate}
      \item Degree 2: $x^2 + x + 1$ is the only candidate and indeed is an irreducible polynomial. 

      \item Degree 3: $x^3 + x^2 + 1$, $x^3 + x + 1$ and indeed $f(0) = f(1) = 1$. Since it's only degree 3 we don't need to check irreducibility into 2 terms of both degree at least 2. 

      \item Degree 4: $x^4 + x^3 + x^2 + x + 1$, $x^4 + x^3 + 1$, $x^4 + x^2 + 1$, $x^4 + x + 1$ are candidates. However we need to check that they cannot be factored into two irreducible quadratic polynomials. The only possible such factorization is 
      \begin{equation}
        (x^2 + x + 1) (x^2 + x + 1) = x^4 + x^2 + 1 
      \end{equation}
      and so the irreducible polynomials are $x^4 + x^3 + x^2 + x + 1$, $x^4 + x^3 + 1$, $x^4 + x + 1$. 

      \item Degree 5: $x^5 + x^4 + 1$, $x^5 + x^3 + 1$, $x^5 + x^2 + 1$, $x^5 + x + 1$, $x^5 + x^4 + x^3 + x^2 + 1$, $x^5 + x^4 + x^3 + x + 1$, $x^5 + x^4 + x^2 + x + 1$, $x^5 + x^3 + x^2 + x + 1$ are the possible candidates. But we need to check that it is not factorable into an irreducible quadratic and cubic. The three candidates are 
      \begin{align}
        (x^2 + x + 1)(x^3 + x^2 + 1) & = x^5 + x + 1 \\
        (x^2 + x + 1)(x^3 + x + 1) & = x^5 + x^4 + 1
      \end{align}
      and so the irreducible polynomials are $x^5 + x^3 + 1$, $x^5 + x^2 + 1$, $x^5 + x^4 + x^3 + x^2 + 1$, $x^5 + x^4 + x^3 + x + 1$, $x^5 + x^4 + x^2 + x + 1$, $x^5 + x^3 + x^2 + x + 1$. 
    \end{enumerate}
  \end{solution}

  \begin{exercise}[Shifrin 3.3.7]
    Prove that for any prime number $p$, $f(x) = x^{p-1} + x^{p-2} + \cdots + x + 1$ is irreducible in $\mathbb{Q}[x]$.
  \end{exercise}
  \begin{solution}
    We can use the identity 
    \begin{equation}
      f(x) = x^{p-1} + x^{p-2} + \cdots + x + 1 = \frac{x^p - 1}{x - 1} 
    \end{equation}
    Therefore, 
    \begin{align}
      f(x+1) = \frac{(x+1)^p - 1}{(x + 1) - 1} & = \frac{1}{x}\bigg\{ \bigg( \sum_{k=0}^p \binom{p}{k} x^k \bigg) - 1 \bigg\} \\
                                               & = \frac{1}{x} \sum_{k=1}^p \binom{p}{k} x^k =  \sum_{k=1}^p \binom{p}{k} x^{k-1}
    \end{align}
    Focusing on the coefficients, the leading coefficient is $\binom{p}{p} = 1$, and the rest of the coefficients are divisible by $p$. The constant coefficient is $\binom{p}{1} = p$, which is not divisible by $p^2$. By Eisenstein's criterion, $f(x+1)$ is irreducible $\implies f(x)$ is irreducible. To justify the final step, assume that $f(x)$ is reducible. Then $f(x) = g(x) h(x)$ for positive degree polynomials $g, h$. Then by substituting $x + 1$, we have that $f(x+1) = g(x+1) h(x+1)$, which means that $f(x+1)$ is irreducible. 
  \end{solution}

  \begin{exercise}[Shifrin 4.1.3]
    \begin{enumerate}
      \item[(a)] Prove that if $I \subset R$ is an ideal and $1 \in I$, then $I = R$.
      \item[(b)] Prove that $a \in R$ is a unit if and only if $\langle a \rangle = R$.
      \item[(c)] Prove that the only ideals in a (commutative) ring $R$ are $\langle 0 \rangle$ and $R$ if and only if $R$ is a field.
    \end{enumerate}
  \end{exercise}
  \begin{solution}
    Listed. 
    \begin{enumerate}
      \item[(a)] If $1 \in I$, then for every $r \in R$, we must have $r1 = r \in I$. Therefore $I = R$. 
      \item[(b)] If $a \in R$ is a unit, then $a^{-1} \in R$, and so for every $r \in R$, $r a^{-1} \in R$. Therefore, $\langle a \rangle$ must contain all elements of form $ra^{-1} a = r$, which is precisely $R$. Now assume that $a$ is not a unit, and so there exists no $a^{-1} \in R$. Therefore, $\langle a \rangle$, which consists of all $ra$ for $r \in R$, cannot contain $1$ since $r \neq a^{-1}$, and so $\langle a \rangle \neq R$. 
      \item[(c)] For the forwards implication, assume that $R$ is not a field. Then there exists some $a \neq 0$ that is not a unit, and taking $\langle a \rangle$ gives us an ideal that---from (b)---is not $R$. For the backward implication we know that $\langle 0 \rangle$ is an ideal. Now assume that there exists another ideal $I$ containing $a \neq 0$. Since $R$ is a field, $a$ is a unit, and so by (b) $R = \langle a \rangle \subset I \subset R \implies I = R$. 
    \end{enumerate}
  \end{solution}

  \begin{exercise}[Shifrin 4.1.4.a/b/c]
    Find all the ideals in the following rings:
    \begin{enumerate}
      \item[(a)] $\mathbb{Z}$
      \item[(b)] $\mathbb{Z}_7$
      \item[(c)] $\mathbb{Z}_6$
      \item[(d)] $\mathbb{Z}_{12}$
      \item[(e)] $\mathbb{Z}_{36}$
      \item[(f)] $\mathbb{Q}$
      \item[(g)] $\mathbb{Z}[i]$ (see Exercise 2.3.18)
    \end{enumerate}
  \end{exercise}
  \begin{solution}
    Listed. 
    \begin{enumerate}
      \item[(a)] All sets of form $\{k z \in \mathbb{Z} \mid z \in \mathbb{Z}\}$ for all $k \in \mathbb{Z}$. 
      \item[(b)] Only $\{0\}$ and $\mathbb{Z}_7$ is an ideal. 
      \item[(c)] We have $\{0\}, \{0, 2, 4\}, \{0, 3\}, \mathbb{Z}_6$. 
    \end{enumerate}
  \end{solution}

  \begin{exercise}[Shifrin 4.1.5]
    \begin{enumerate}
      \item[(a)] Let $I = \langle f(x) \rangle$, $J = \langle g(x) \rangle$ be ideals in $F[x]$. Prove that $I \subset J \Leftrightarrow g(x)|f(x)$.
      \item[(b)] List all the ideals of $\mathbb{Q}[x]$ containing the element 
      $f(x) = (x^2 + x - 1)^3(x - 3)^2$.
    \end{enumerate}
  \end{exercise}
  \begin{solution}
    For (a), we prove bidirectionally. 
    \begin{enumerate}
      \item $(\rightarrow)$. Since $f (x) \in \langle f(x) \rangle \implies f(x) \in \langle g(x) \rangle$, this means that $f(x) = r(x) g(x)$ for some $r(x) \in F[x$. Therefore $g(x) \mid f(x)$. 

      \item $(\leftarrow)$. Given that $g(x) \mid f(x)$, let us take some $f_1 (x) \in I$. Then it is of the form $f_1(x) = r(x) f(x)$ for some $r(x) \in F[x]$. But since $g(x) \mid f(x)$, $f(x) = h(x) g(x)$ for some $h(x) \in F[x]$. Therefore $f_1 (x) = r(x) h(x) g(x) = (rh)(x) g(x)$, where $(rh)(x) \in F[x]$, and so $f_1 (x) \in J$. 
    \end{enumerate}

    For (b), we can use the logic from (a) to find all the factors of $f(x)$, which generate all sup-ideals of $\langle f(x) \rangle$, which is the minimal ideal containing $f(x)$. 
    \begin{enumerate}
      \item $g(x) = 1 \implies \langle 1 \rangle = F[x]$  
      \item $g(x) = x^2 + x - 1 \implies \langle x^2 + x - 1 \rangle$
      \item $g(x) = (x^2 + x - 1)^2 \implies \langle (x^2 + x - 1)^2 \rangle$
      \item $g(x) = (x^2 + x - 1)^3 \implies \langle (x^2 + x - 1)^3 \rangle$
      \item $g(x) = x - 3 \implies \langle x - 3 \rangle$
      \item $g(x) = (x^2 + x - 1)(x - 3) \implies \langle (x^2 + x - 1)(x - 3) \rangle$
      \item $g(x) = (x^2 + x - 1)^2 (x - 3) \implies \langle (x^2 + x - 1)^2 (x - 3) \rangle$
      \item $g(x) = (x^2 + x - 1)^3 (x - 3) \implies \langle (x^2 + x - 1)^3 (x - 3) \rangle$
      \item $g(x) = (x - 3)^2 \implies \langle (x - 3)^2 \rangle$
      \item $g(x) = (x^2 + x - 1)(x - 3)^2 \implies \langle (x^2 + x - 1)(x - 3)^2 \rangle$
      \item $g(x) = (x^2 + x - 1)^2 (x - 3)^2 \implies \langle (x^2 + x - 1)^2 (x - 3)^2 \rangle$
      \item $g(x) = (x^2 + x - 1)^3 (x - 3)^2 \implies \langle (x^2 + x - 1)^3 (x - 3)^2 \rangle$
    \end{enumerate}
  \end{solution}

  \begin{exercise}[Shifrin 4.1.14.a/b]
    Mimicking Example 5(c), give the addition and multiplication tables of
    \begin{enumerate}
      \item[(a)] $\mathbb{Z}_2[x]/\langle x^2 + x \rangle$
      \item[(b)] $\mathbb{Z}_3[x]/\langle x^2 + x - 1 \rangle$
      \item[(c)] $\mathbb{Z}_2[x]/\langle x^3 + x + 1 \rangle$
    \end{enumerate}
    In each case, is the quotient ring an integral domain? a field?
  \end{exercise}
  \begin{solution}
    For (a), note that the quotient allows us to state that $x^2 \equiv x \pmod{I}$, and therefore every polynomial in $\mathbb{Z}_2 [x]/ \langle x^2 + x \rangle$ is equivalent to a linear polynomial. Therefore, the elements in this quotient are $0, 1, x, x + 1$. As you can see, this is not an integral domain (and hence not a field) since $x, x + 1$ are zero divisors. 

    \begin{figure}[H]
      \centering
      \begin{subfigure}[b]{0.48\textwidth}
        \centering
        \begin{tabular}{c|cccc}
          $+$ & $0$ & $1$ & $x$ & $x+1$ \\
          \hline
          $0$ & $0$ & $1$ & $x$ & $x+1$ \\
          $1$ & $1$ & $0$ & $x+1$ & $x$ \\
          $x$ & $x$ & $x+1$ & $0$ & $1$ \\
          $x+1$ & $x+1$ & $x$ & $1$ & $0$ \\
        \end{tabular}
      \end{subfigure}
      \hfill 
      \begin{subfigure}[b]{0.48\textwidth}
        \centering
        \begin{tabular}{c|cccc}
          $\times$ & $0$ & $1$ & $x$ & $x+1$ \\
          \hline
          $0$ & $0$ & $0$ & $0$ & $0$ \\
          $1$ & $0$ & $1$ & $x$ & $x+1$ \\
          $x$ & $0$ & $x$ & $x$ & $0$ \\
          $x+1$ & $0$ & $x+1$ & $0$ & $x+1$ \\
        \end{tabular}
      \end{subfigure}
      \caption{Addition and multiplication tables for $\mathbb{Z}_2 [x]/ \langle x^2 + x \rangle$. }
    \end{figure}

    For (b), note that the quotient allows us to state that $x^2 \equiv 2x + 1 \pmod{I}$, and therefore every polynomial in $\mathbb{Z}_3 [x] / \langle x^2 + x - 1 \rangle$ is equivalent to a linear polynomial. Therefore, the elements in this quotient are $0, 1, 2, x, x + 1, x + 2, 2x, 2x + 1, 2x + 2$. This is indeed an integral domain since there are no zero divisors, and it is a field since every nonzero element is a unit (all rows/columns are filled with all elements of the set). 

    \begin{figure}[H]
      \centering
      \begin{tabular}{c|ccccccccc}
        $+$ & $0$ & $1$ & $2$ & $x$ & $x+1$ & $x+2$ & $2x$ & $2x+1$ & $2x+2$ \\
        \hline
        $0$ & $0$ & $1$ & $2$ & $x$ & $x+1$ & $x+2$ & $2x$ & $2x+1$ & $2x+2$ \\
        $1$ & $1$ & $2$ & $0$ & $x+1$ & $x+2$ & $x$ & $2x+1$ & $2x+2$ & $2x$ \\
        $2$ & $2$ & $0$ & $1$ & $x+2$ & $x$ & $x+1$ & $2x+2$ & $2x$ & $2x+1$ \\
        $x$ & $x$ & $x+1$ & $x+2$ & $2x$ & $2x+1$ & $2x+2$ & $0$ & $1$ & $2$ \\
        $x+1$ & $x+1$ & $x+2$ & $x$ & $2x+1$ & $2x+2$ & $2x$ & $1$ & $2$ & $0$ \\
        $x+2$ & $x+2$ & $x$ & $x+1$ & $2x+2$ & $2x$ & $2x+1$ & $2$ & $0$ & $1$ \\
        $2x$ & $2x$ & $2x+1$ & $2x+2$ & $0$ & $1$ & $2$ & $x$ & $x+1$ & $x+2$ \\
        $2x+1$ & $2x+1$ & $2x+2$ & $2x$ & $1$ & $2$ & $0$ & $x+1$ & $x+2$ & $x$ \\
        $2x+2$ & $2x+2$ & $2x$ & $2x+1$ & $2$ & $0$ & $1$ & $x+2$ & $x$ & $x+1$ \\
      \end{tabular}
      \caption{Addition table for $\mathbb{Z}_3[x]/ \langle x^2 + x - 1\rangle$.}
    \end{figure}

    \begin{figure}[H]
      \centering
      \begin{tabular}{c|ccccccccc}
        $\times$ & $0$ & $1$ & $2$ & $x$ & $x+1$ & $x+2$ & $2x$ & $2x+1$ & $2x+2$ \\
        \hline
        $0$ & $0$ & $0$ & $0$ & $0$ & $0$ & $0$ & $0$ & $0$ & $0$ \\
        $1$ & $0$ & $1$ & $2$ & $x$ & $x+1$ & $x+2$ & $2x$ & $2x+1$ & $2x+2$ \\
        $2$ & $0$ & $2$ & $1$ & $2x$ & $2x+2$ & $2x+1$ & $x$ & $x+2$ & $x+1$ \\
        $x$ & $0$ & $x$ & $2x$ & $2x + 1$ & $1$ & $x+1$ & $x+2$ & $2x+2$ & $2$ \\
        $x+1$ & $0$ & $x+1$ & $2x+2$ & $1$ & $x+2$ & $2x$ & $2$ & $x$ & $2x+1$ \\
        $x+2$ & $0$ & $x+2$ & $2x+1$ & $x+1$ & $2x$ & $2$ & $2x+2$ & $1$ & $x$ \\
        $2x$ & $0$ & $2x$ & $x$ & $x+2$ & $2$ & $2x+2$ & $2x+1$ & $x+1$ & $1$ \\
        $2x+1$ & $0$ & $2x+1$ & $x+2$ & $2x+2$ & $x$ & $1$ & $x+1$ & $2$ & $2x$ \\
        $2x+2$ & $0$ & $2x+2$ & $x+1$ & $2$ & $2x+1$ & $x$ & $1$ & $2x$ & $x+2$ \\
      \end{tabular}
      \caption{Multiplication table for $\mathbb{Z}_3[x]/\langle x^2 + x - 1\rangle$.}
    \end{figure}

  \end{solution}

  \begin{exercise}[Shifrin 4.1.17]
    Let $R$ be a commutative ring and let $I,J \subset R$ be ideals. Define
    \begin{align*}
      I \cap J &= \{a \in R : a \in I \text{ and } a \in J\}\\
      I + J &= \{a + b \in R : a \in I, b \in J\}.
    \end{align*}
    \begin{enumerate}
      \item[(a)] Prove that $I \cap J$ and $I + J$ are ideals.
      \item[(b)] Suppose $R = \mathbb{Z}$ or $F[x]$, $I = \langle a \rangle$, and $J = \langle b \rangle$. Identify $I \cap J$ and $I + J$.
      \item[(c)] Let $a_1,\ldots,a_n \in R$. Prove that $\langle a_1,\ldots,a_n \rangle = \langle a_1 \rangle + \cdots + \langle a_n \rangle$.
    \end{enumerate}
  \end{exercise}
  \begin{solution}
    For (a), we have the following. 
    \begin{enumerate}
      \item $I \cap J$ is an ideal. Given $a, b \in I \cap J$, then $a, b \in I \implies a + b \in I$, and $a, b \in J \implies a + b \in J$. So $a + b \in I \cap J$. Furthermore, for every $r \in R$, $a \in I \implies r a \in I$ and $a \in J \implies r a \in J$, so $a \in I \cap J \implies ra \in I \cap J$. 

      \item $I + J$ is an ideal. Given $x, y \in I + J$, then $x = a_x + b_x$ and $y = a_y + b_y$ for $a_x, a_y \in I, b_x, b_y \in J$. So 
      \begin{equation}
        x + y = (a_x + b_x) + (a_y + b_y) = (a_x + a_y) + (b_x + b_y)
      \end{equation}
      where $a_x + a_y \in I, b_x + b_y \in J$ by definition of an ideal, and so $x + y \in I + J$. Noe let $x = a_x + b_x \in I + J$. Then given $r \in R$,
      \begin{equation}
        rx = r(a_x + b_x) = r a_x + r b_x
      \end{equation}
      where $r a_x \in I$ and $r b_x \in J$ since $I, J$ are ideals. Therefore $rx \in I + J$.  
    \end{enumerate}
    For (b), the argument is equivalent for $\mathbb{Z}$ and $F[x]$. $I \cap J$ consists of all elements that are divisible by both $a$ and $b$, so $I \cap J = \langle \mathrm{lcm}(a, b) \rangle$. $I + J$ consists of all elements that are of form $r a + s b$, but this are all multiples of $\mathrm{gcd}(a, b)$ and so $I + J = \langle \mathrm{gcd}(a, b) \rangle$. 

    For (c), it suffices to prove $\langle a, b \rangle = \langle a \rangle + \langle b \rangle$. 
    \begin{enumerate}
      \item $\langle a, b \rangle \subset \langle a \rangle + \langle b \rangle$. $x \in \langle a, b \rangle \implies x = r_a a + r_b b$ for $r_a, r_b \in R$. But $a \in \langle a \rangle, b \in \langle b \rangle \implies r_a a \in \langle a \rangle, r_b b \in \langle b \rangle$, and so $x \in \langle a \rangle + \langle b \rangle$. 

    \item $\langle a, b \rangle \supset \langle a \rangle + \langle b \rangle$. $x \in \langle a \rangle + \langle b \rangle \implies x = a_x + b_x$ for $a_x \in \langle a \rangle, b_x \in \langle b \rangle$. But $a_x \in \langle a \rangle \implies a_x = r_a a$ for some $r_a \in R$, and $b_x \in \langle b \rangle \implies b_x = r_b b$ for some $r_b \in R$. So $x = r_a a + r_b b \iff x \in \langle a, b \rangle$. 
    \end{enumerate}
    We know that for $\langle a_1 \rangle = \langle a_1 \rangle$, and so by making this argument $n-1$ times we can build up by induction that $\langle a_1, \ldots a_{n-1}, a_n \rangle = \langle a_1, \ldots, a_{n-1} \rangle + \langle a_n \rangle$. 
  \end{solution}

  \begin{exercise}[Shifrin 4.2.1]
    \begin{enumerate}
      \item[(a)] Prove that the function $\phi: \mathbb{Q}[\sqrt{2}] \to \mathbb{Q}[\sqrt{2}]$ defined by $\phi(a + b\sqrt{2}) = a - b\sqrt{2}$ is an isomorphism.
      \item[(b)] Define $\phi: \mathbb{Q}[\sqrt{3}] \to \mathbb{Q}[\sqrt{7}]$ by $\phi(a + b\sqrt{3}) = a + b\sqrt{7}$. Is $\phi$ an isomorphism? Is there any isomorphism?
    \end{enumerate}
  \end{exercise}
  \begin{solution}
    For (a), we first prove that it is a homomorphism. 
    \begin{align}
      \phi((a + b \sqrt{2}) + (c + d \sqrt{2})) & = \phi((a + c) + (b + d) \sqrt{2}) \\
                                                & = (a + c) - (b + d) \sqrt{2} \\
                                                & = (a - b \sqrt{2}) + (c - d \sqrt{2}) \\
                                                & = \phi(a + b \sqrt{2}) + \phi(c + d \sqrt{2}) \\
      \phi((a + b \sqrt{2}) (c + d \sqrt{2})) & = \phi((ac + 2bd) + (ad + bc) \sqrt{2}) \\
                                              & = (ac + 2bd) - (ad + bc) \sqrt{2} \\
                                              & =  (a - b \sqrt{2}) (c - d \sqrt{2}) \\
                                              & = \phi(a + b \sqrt{2}) \times \phi(c + d \sqrt{2}) \\ 
                                      \phi(1) & = 1
    \end{align}
    This is injective since given that $a + b \sqrt{2} \neq c + d \sqrt{2}$, then at least $a \neq b$ or $c \neq d$, in which case $a - b \sqrt{2} \neq c - d \sqrt{2}$. Alternatively, we can see that the kernel is $0$, so it must be injective. It is onto since given any $c + d\sqrt{2}$, the preimage is $c - d \sqrt{2}$. Therefore $\phi$ is an isomorphism.  

    For (b), no it is not an isomorphism since 
    \begin{align}
      \phi ((a + b \sqrt{3}) (c + d \sqrt{3})) & = \phi ((ac + 3bd) + (ad + bc) \sqrt{3}) \\
                                               & = (ac + 3bd) + (ad + bc) \sqrt{7} \\
                                               & \neq (ac + 7bd) + (ad + bc) \sqrt{7} \\ 
                                               & = (a + b \sqrt{7}) (c + d \sqrt{7}) \\
                                               & = \phi(a + b \sqrt{3}) \phi(c + d  \sqrt{3}) 
    \end{align} 
    We claim that there is no isomorphism. Assume that such $\phi$ exists. Then $\phi(1) = 1$, and so $\phi(3) = \phi(1 + 1 + 1) = \phi(1) + \phi(1) + \phi(1) = 1 + 1 + 1 = 3$. Now given $\sqrt{3} \in \mathbb{Q}[\sqrt{3}]$, we follows that 
    \begin{equation}
      \phi(\sqrt{3})^2 = \phi(3) = 3
    \end{equation}
    and so $\phi(\sqrt{3})$ must map to the square root of $3$ which must live in $\mathbb{Q}[\sqrt{7}]$. Assume such a number is $a + b \sqrt{7} \implies (a^2 + 7b^2) + (2ab) \sqrt{7} = \sqrt{3}$. This implies that $2ab = 0$, leaving the rational term, but we know that $\sqrt{3}$ does not exist in the rationals, and so $\sqrt{3}$ does not exist.  
  \end{solution}

  \begin{exercise}[Shifrin 4.2.12]
    Let $R$ be a commutative ring, $I \subset R$ an ideal. Suppose $a \in R$, $a \notin I$, and $I + \langle a \rangle = R$ (see Exercise 4.1.17 for the notion of the sum of two ideals). Prove that $\bar{a} \in R/I$ is a unit.
  \end{exercise}
  \begin{solution}
    Since $R = I + \langle a \rangle$, $1 \in R = I + \langle a \rangle$. So there exists $i \in I, ra \in \langle a \rangle$ s.t. $1 = i + ra \implies ra = 1 - i$. Therefore, in the quotient ring, $\bar{i} = 0$ and we have 
    \begin{equation}
      \bar{r} \bar{a} = \bar{1} - \bar{0} = \bar{1}
    \end{equation}
    and so $\bar{r}$ is a multiplicative inverse of $\bar{a}$. So $\bar{a}$ is a unit. 
  \end{solution}



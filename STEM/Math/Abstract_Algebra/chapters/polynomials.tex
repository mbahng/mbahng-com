\section{Polynomial Rings} 

\subsection{Stuff}

\begin{theorem}[Rational Root Theorem]
  Let $a_n x^n + \ldots + a_0 \in \mathbb{Z}[x]$. If $r/s \in \mathbb{Q}$ with $\gcd(r, s) = 1$, then $r \mid a_0$ and $s \mid a_n$. 
\end{theorem}
\begin{proof}
  Given that $r/s$ is a root, we have 
  \begin{equation}
    a_n (r/s)^n + \ldots + a_0 = 0
  \end{equation}
  Multiplying by $s^n$, we get 
  \begin{equation}
    a_n r^n + a_{n-1} r^{n-1} s + \ldots + a_1 s^{n-1} r + a_0 s^n = 0
  \end{equation}
  and putting this equation on mod $r$ and mod $s$ implies that $r | a_0 s^n$ and $s | a_n r^n$, respectively. But since we assumed that $\gcd (r, s) = 1$, $r | a_0$ and $s | a_n$. 
\end{proof}

The next is quite a remarkable result, since it says that decompositions in $\mathbb{Q}[x]$ imply decompositions in $\mathbb{Z}[x]$! Therefore, to check irreducibility in $\mathbb{Q}[x]$, it suffices to check irreducibility in $\mathbb{Z}[x]$. 

\begin{lemma}[Gauss's Lemma]
  Let $f \in \mathbb{Z}[x]$. If $\exists g, h \in \mathbb{Q}[x]$ s.t. $f(x) = g(x) h(x)$, then $\exists \bar{g}, \bar{h} \in \mathbb{Z}[x]$ s.t. $f(x) = \bar{g}(x) \bar{h}(x)$. 
\end{lemma}
\begin{proof}
  We can find $k, l \in \mathbb{Z}$ s.t. $g_1 (x) = k g(x)$ and $h_1 (x) = l h(x)$ have integer coefficients, i.e. $g_1, h_1 \in \mathbb{Z}[x]$. Then, $k l f(x) = g_1 (x) h_1 (x) \in \mathbb{Z}[x]$. Let $p$ be a prime factor of $kl$. We have 
  \begin{equation}
    0 \equiv \bar{k} \bar{l} \bar{f} (x) \equiv \bar{g}_1 (x) \bar{h}_1 (x) \text{ in } \mathbb{Z}_p [x]
  \end{equation}
  Since $\mathbb{Z}_p$ is an integral domain, $\mathbb{Z}_p [x]$ is an integral domain, and so $\bar{g}_1$ or $\bar{h}_1$ must be $0$. WLOG let it be $\bar{g}_1$. Then every coefficient of $g_1 (x)$ is divisible by $p$, and we can write it in the form $g_2(x) = p g_1 (x)$. Therefore, 
  \begin{equation}
    p(x) \cdot \frac{kl}{p} = \underbrace{\frac{g_1 (x)}{p}}_{g_2 (x)} \cdot \underbrace{h_1 (x)}_{h_2 (x)} \iff f(x) \frac{kl}{p} = g_2 (x) h_2 (x)
  \end{equation}
  Since there are only finitely many prime divisors, we do this for all prime factors of $kl$, and we have 
  \begin{equation}
    f(x) = g_n (x) h_n (x), \qquad g_n, h_n \in \mathbb{Z}[x]
  \end{equation}
\end{proof}

\begin{example}[Reducibility of Integer Polynomials]
  Let $f(x) = x^4 - x^3 + 2$. The rational roots are in the set $S = \{\pm 1, \pm2 \}$, but none of them work since $f(\pm1), f(\pm2) \neq 0$. By degree considerations and Gauss's lemma, if $f(x)$ is reducible, then 
  \begin{equation}
    f(x) = (x^2 + ax + b) (x^2 + cx + d), \qquad a, b, c, d \in \mathbb{Z}
  \end{equation}
  We know that $bd \in S$, with $a + c = -1$, $d + b + ac = 0$, and so on for each coefficients. We can brute force this finite set of possibilities. 
\end{example}

A great way to check irreducibility is to check in mod $p$. 

\begin{theorem}
  Let $f(x) = a_n x^n + \ldots + a_0 \in \mathbb{Z}[x]$. If $p \nmid a_n$ and $f \in \mathbb{Z}_p [x]$ is irreducible, then $f$ is irreducible in $\mathbb{Q}[x]$.\footnote{May need to verify this again.}
\end{theorem}
\begin{proof}
  Suppose that $f(x) = g(x) h(x) \in \mathbb{Z}[x]$ with $\deg(g), \deg(h) > 0$. Then 
  \begin{equation}
    f(x) \equiv g(x) h(x) \text{ in } \mathbb{Z}_p [x]
  \end{equation}
  Since $f(x)$ is irreducible in $\mathbb{Z}_p [x]$, we must have that one of $g(x)$ or $h(x)$ has degree $0$ in $\mathbb{Z}_p [x]$. WLOG let it be $g(x)$, but this means that the leading coefficient of $g(x)$ must be divisible by $p \implies$ leading coefficient of $f(x)$ is divisible by $p \iff p \mid a_n$. 
\end{proof}

\begin{example}
  $x^4 + x + 1$ is irreducible in $\mathbb{Z}_2 [x]$. So we can extend this to $\mathbb{Z}[x]$ to see that \textit{all} fourth degree polynomials of form $a x^4 + b x^3 + c x^2 + dx + e$, which $a, d, e$ odd and $b, c$ even is irreducible in $\mathbb{Q}[x]$. 
\end{example}

This is a powerful theorem to quickly find a large class of polynomials that are irreducible. However, being reducible in $\mathbb{Z}_p [x]$ does not imply reducibility in $\mathbb{Q}$. In fact, there are polynomials $f(x) \in \mathbb{Z}[x]$ which are irreducible but reducible in $\mathbb{Z}_p$ for \textit{every} prime $p$. 

\begin{theorem}[Eisenstein's Criterion]
  Let $f(x) = a_n x^n + \ldots + a_0 \in \mathbb{Z}[x]$ and $p \in \mathbb{Z}$ a prime s.t. $p \nmid a_n$, $p \mid a_i$ for $i = 0, \ldots, a_{n-1}$, and $p^2 \nmid a_0$. Then $f(x)$ is irreducible in $\mathbb{Q}[x]$. 
\end{theorem}
\begin{proof}
  Suppose that $f(x) = g(x) h(x) \in \mathbb{Q}[x]$ with $\deg(g), \deg(h) > 0$. Then, by Gauss's lemma, $g, h \in \mathbb{Z}[x]$. Reducing the equations mod $p$, 
  \begin{equation}
    f(x) = g(x) h(x) \text{ in } \mathbb{Z}_p [x]
  \end{equation}
  But $f(x) = a_n x^n$. By unique factorization theorem in $\mathbb{Z}_p [x]$, $g, h \in \mathbb{Z}_p [x]$ must be products of units and prime factors of $a_n x^n$, which are $\{x\}$. Therefore, let 
  \begin{equation}
    g(x) = b_m x^m, h(x) = \frac{a_n}{b_m} x^{n-m} \in \mathbb{Z}_p [x]
  \end{equation}
  with $\deg(g) = m > 0$ and $\deg(h) = n - m > 0$ in $\mathbb{Z}[x]$. This implies that the constant coefficients of $g(x), h(x)$ are divisible by $p$, which implies that the constant coefficients of $f(x) = g(x) h(x)$ are divisible by $p^2$, a contradiction. 
\end{proof}

\begin{example}
  Listed. 
  \begin{enumerate}
    \item $x^{13} + 2x^{10} + 4x + 6$ is divisible by Eisenstein for $p = 2$. 
    \item $x^3 + 9x^2 + 12x + 3$ is divisible by Eisenstein for $p = 3$. 
  \end{enumerate}
\end{example}

\begin{example}
  Let $f(x) = x^4 + x^3 + x^2 + x + 1$. Then, we know that $f(x) = \frac{x^5 - 1}{x-1}$ and so 
  \begin{align}
    f(x + 1) & = \frac{(x + 1)^5 - 1}{(x + 1) - 1} \\
             & = \frac{1}{x} \bigg( x^5 + \binom{5}{1} x^4 + \binom{5}{2} x^3 + \binom{5}{3} x^2 + \binom{5}{4} x + \binom{5}{5} - 1 \bigg) 
             & = x^4 + 5x^3 + 10 x^2 + 10x + 5
  \end{align}
  So all nonleading coefficients are divisible by $5$ exactly once, which by Eisenstein implies that $f(x+1)$ is irreducible which implies that $f(x)$ is irreducible. 
\end{example}

We have prod that for $\alpha \in \mathbb{C}$, subfield $F \subset \mathbb{C}$, and $f(x) \in F[x]$, with $f(\alpha) = 0$, then $B = \{1, \alpha, \ldots, \alpha^{\deg(f) - 1}\}$ spans $F[\alpha]$ as a $F$-vector space. If $f(x)$ is irreducible then $B$ is a basis. 




\section{Representations}

  We will assume that $V$ is a finite-dimensional vector space over field $\mathbb{C}$. 

  \begin{definition}
    The \textbf{general linear group} of vector space $V$, denoted $\GL(V)$, is the group of all automorphisms of $V$ to itself. The \textbf{special linear group} of vector space $V$, denoted $\SL(V)$ is the subgroup of automorphisms of $V$ with determinant $1$. 
  \end{definition}

  When studying an abstract set, it is often useful to consider the set of all maps from this abstract set to a well known set (e.g. $\GL(V)$). 

  \begin{definition}
    A \textbf{representation} of an (algebraic) group $\mathcal{G}$ is a homomorphism 
    \begin{equation}
      \rho: G \longrightarrow \GL(V)
    \end{equation}
    for some vector space $V$. That is, given an element $g \in \mathcal{G}$, $\rho(g) \in \GL (V)$, meaning that $\rho(g)(v) \in V$. Additionally, since it is a homomorphism, the algebraic structure is preserved. 
    \begin{equation}
      \rho(g_1 \cdot g_2) = \rho(g_1) \cdot \rho(g_2)
    \end{equation}
    where $\cdot$ on the left hand side is the abstract group multiplication while the $\cdot$ on the right hand side is matrix multiplication. To shorten the notation, we will denote 
    \begin{equation}
      g v = \rho(g) v, \; v \in V
    \end{equation}
    Since $\rho$ is a group morphism, we have 
    \begin{equation}
      g_2 (g_1 v) = (g_2 g_1) v \; \iff \rho(g_2) \big( \rho(g_1) (v) \big) = \big( \rho(g_2) \rho(g_1) \big) (v)
    \end{equation}
    Additionally, since $g$ (that is, $\rho(g)$) is a linear map, 
    \begin{equation}
      g(\lambda_1 v_1 + \lambda_2 v_2) = \lambda_1 g v_1 + \lambda_2 g v_2
    \end{equation}
    Usually, we refer to the map as the representation, but if the map is well-understood, we just call the vector space $V$ the representation and say that the group acts on this vector space. 
  \end{definition}

  \begin{example}
    The group $\GL(2, \mathbb{C})$ can be represented a by the vector space $\mathbb{C}^2$, or explicitly, by the group of $2 \times 2$ matrices over $\mathbb{C}$ with nonzero determinant.
    \begin{equation}
      \GL(2, \mathbb{C}) \xmapsto{id} \text{Mat}(2, \mathbb{C})
    \end{equation}
    This is a trivial representation. 
  \end{example}

  We now show a nontrivial representation of $\GL(2, \mathbb{C})$. 

  \begin{example}
    We take Sym$^2 \mathbb{C}^2$, the second symmetric power of $\mathbb{C}^2$. Note that given a basis $x_1, x_2 \in \mathbb{C}^2$, the set
    \begin{equation}
      \{x_1 \odot x_1, x_1 \odot x_2, x_2 \odot x_2\}
    \end{equation}
    forms a basis of Sym$^2 \mathbb{C}^2 \implies \dim\,$Sym$^2 \mathbb{C}^2 = 3$. So, we want to represent $\GL(2, \mathbb{C})$ by associating its element with elements of $\GL(Sym^2 \mathbb{C}^2)$. More concretely, we are choosing to represent a $2 \times 2$ matrix over $\mathbb{C}$ with a $3 \times 3$ matrix group (since $\GL(Sym^2 \mathbb{C}^2) \simeq \GL(3, \mathbb{C})$. Clearly,
    \begin{align*}
      & \rho(g) (x_1 \odot x_1) = g(x_1) \odot g(x_1) \in Sym^2 \mathbb{C}^2 \\
      & \rho(g) (x_1 \odot x_2) = g(x_1) \odot g(x_2) \\
      & \rho(g) (x_2 \odot x_2) = g(x_2) \odot g(x_2)
    \end{align*}
    To present this in matrix form, let us have an element in $\GL (2, \mathbb{C})$
    \begin{equation}
      \mathcal{A} \equiv \begin{pmatrix}
      a & b \\
      c & d
      \end{pmatrix}
    \end{equation}
    We evaluate the corresponding representation in $\GL( Sym^2 \mathbb{C}^2)$. Using the identities above, we have 
    \begin{align*}
      \rho(g) (x_1 \odot x_1) & = g(x_1) \odot g(x_1) \\
      & = (a x_1 + c x_2) \odot (a x_1 + c x_2) \\
      & = a^2 x_1 \odot x_1 + 2ac x_1 \odot x_2 + c^2 x_2 \odot x_2 \\
      \rho(g) (x_1 \odot x_2) & = g(x_1) \odot g(x_2) \\
      & = (a x_1 + c x_2) \odot (b x_1 + d x_2) \\
      & = ab x_1 \odot x_1 + (ad + bc) x_1 \odot x_2 + cd x_2 \odot x_2 \\
      \rho(g) (x_2 \odot x_2) & = g(x_2) \odot g(x_2) \\
      & = (b x_1 + d x_2) \odot (b x_1 + d x_2) \\
      & = b^2 x_1 \odot x_1 + 2bd x_1 \odot x_2 + d^2 x_2 \odot x_2
    \end{align*}
    And this completely determines the matrix. So, 
    \begin{equation}
      \rho \begin{pmatrix}
      a&b\\c&d
      \end{pmatrix} = \begin{pmatrix}
      a^2&ab&b^2\\2ac&ad+bc&2bd\\c^2&cd&d^2
      \end{pmatrix}
    \end{equation}
    is the $3 \times 3$ representation of $\mathcal{A}$ in $\GL(Sym^2 \mathbb{C}^2)$. 
  \end{example}

  We continue to define maps between two representations of $\mathcal{G}$. 

  \begin{definition}
    A \textbf{morphism} between 2 representations 
    \begin{align*}
      & \rho_1: \mathcal{G} \longrightarrow \GL(V_1) \\
      & \rho_2: \mathcal{G} \longrightarrow \GL(V_2) 
    \end{align*}
    of some group but not necessarily the same vector space is a linear map $f: V_1 \longrightarrow V_2$ that is \textbf{compatible} with the group action. That is, $f$ satisfies the property that for all $g \in \mathcal{G}$
    \begin{equation}
      f \circ g = g \circ f
    \end{equation}
    Again, we use the shorthand notation that $g = \rho(g)$, meaning that the statement above really translates to $ f \circ \rho(g) = \rho(g) \circ f$. This is equivalent to saying that the following diagram commutes. 
    \[\begin{tikzcd}
    V_1 \arrow{r}{\rho_1(g)} \arrow{d}{f} & V_1 \arrow{d}{f} \\
    V_2 \arrow{r}{\rho_2 (g)} & V_2
    \end{tikzcd}\]
  \end{definition}

  \begin{definition}
    Let $V$ be a representation of $\mathcal{G}$. A \textbf{subrepresentation} is a subspace $W \subset V$ such that for all $g \in \mathcal{G}$ and for all $w \in W$, 
    \begin{equation}
      \rho(g)(w) \in W
    \end{equation}
  \end{definition}

  \begin{example}
    $V$ and $\{0\}$ are always subrepresentations of $V$. 
  \end{example}

  We now introduce the "building blocks" of all representations. 
  \begin{definition}
    A representation $W$ is \textbf{irreducible representation} if $\{0\}$ and $W$ are the only subrepresentations of $W$. 
  \end{definition}

  \begin{lemma}[Schur's Lemma]
    Let $V_1, V_2$ be irreducible representations and let $f: V_1 \longrightarrow V_2$ be a morphism (of representations). Then, either
    \begin{enumerate}
      \item $f$ is an isomorphism. 
      \item $f = 0$
    \end{enumerate}
    Furthermore, any 2 isomorphisms differ by a constant. That is, 
    \begin{equation}
      f_1 = \lambda f_2
    \end{equation}
  \end{lemma}
  \begin{proof}
    $\ker{f}$ is clearly a vector space. Furthermore, it is a subrepresentation (since it is a subspace of $V_1$) $\implies \ker{f} = V$ or $\ker{f} = 0$. If $\ker{f} = V$, then $f = 0$ and the theorem is satisfied. If $\ker{f} = 0$, then $f$ is injective, and $\im{f}$ is a subrepresentation of $V_2 \implies \im{f} = 0$ or $\im{f} = V_2$. But $\im{f} \neq 0$ since $f$ is injective, so $\im{f} = V_2 \implies f$ is surjective $\implies f$ is bijective, that is, $f$ is an isomorphism of vector spaces. So, the inverse $f^{-1}$ exists, and this map $f^{-1}$ satisfies
    \begin{equation}
      f^{-1} \circ \rho_2(g) = \rho_1 (g) \circ f^{-1}
    \end{equation}
    To prove the second part, without loss of generality, assume that the first isomorphism is the identity mapping. That is, 
    \begin{equation}
      f_1 = id
    \end{equation}
    Since we are working over the field $\mathbb{C}$, we can find an eigenvector of $f_2$. That is, there exists a $v \in V_1$ such that 
    \begin{equation}
      f_2 (v) = \lambda v
    \end{equation}
    Now, we define the map
    \begin{equation}
      f: V_1 \longrightarrow V_2, \; f \equiv f_2 - \lambda f_1
    \end{equation}
    Clearly, $\ker{f} \neq 0$, since $v \in \ker{f}$. That is, we have a map $f$ between 2 irreducible representations that has a nontrivial kernel. This means that $f = 0 \implies f_2 = \lambda f_1$.  
  \end{proof}

  \begin{theorem}[Mache's Theorem]
    Let $V$ be finite dimensional, with $\mathcal{G}$ a finite group. Then, $V$ can be decomposed as 
    \begin{equation}
      V = \bigoplus_{i} V_i
    \end{equation}
    where each $V_i$ is an irreducible representation of $\mathcal{G}$. 
  \end{theorem}
  \begin{proof}
    By induction on dimension, it suffices to prove that if $W$ is a subrepresentation of $V$, then there exists a subrepresentation $W^\prime \subset V$ such that $W \oplus W^\prime = V$. So, if $V$ isn't an irreducible representation, it can always be decomposed into smaller subrepresentations $W$ and $W^\prime$ that direct sum to $V$. Now, we define the canonical (linear) projection 
    \begin{equation}
      \pi: V \longrightarrow W
    \end{equation}
    Then, we define the new map 
    \begin{equation}
      \Tilde{\pi}: V \longrightarrow W, \; \Tilde{\pi}(v) \equiv \frac{1}{|\mathcal{G}|} \sum_{g \in \mathcal{G}} \rho(g)\big|_W \circ \pi \circ \rho(g)^{-1}
    \end{equation}
    This "averaging" of the group elements are done so that this mapping is a map of representations. This implies that 
    \begin{equation}
      V = W \oplus \ker{\Tilde{\pi}}
    \end{equation}
    meaning that $V$ can indeed be decomposed into direct sums of subrepresentations. 
  \end{proof}


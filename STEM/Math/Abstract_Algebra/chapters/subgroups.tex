\section{Subgroups} 

  We have seen a few examples of subgroups, but we will heavily elaborate on here. We know that given a set, we can define an equivalence relation on it to get a quotient set. Now if we have a group, defining any such relation may not be compatible with the group structure. Therefore, it would be nice to have some principles in which we can construct such compatible equivalence classes. Fortunately, we can do such a thing by taking a subgroup $H \subset G$ and ``shifting'' it to form the cosets of $G$, which are the equivalence classes. 
  
  \begin{definition}[Coset]
    Given a group $G$, $g \in G$, and subgroup $H$, 
    \begin{enumerate}
      \item A \textbf{left coset} is $g H \coloneqq \{g h \mid h \in H \}$. 
      \item A \textbf{right coset} is $H g \coloneqq \{h g \mid h \in H \}$. 
      \item If $G$ is abelian, then the \textbf{coset} is $gH \coloneqq \{g + h \mid h \in H\}$. 
    \end{enumerate}
    This divides the group into equivalence classes $g \mapsto [g] = gH$, and we write (for left cosets)
    \begin{equation}
      a \equiv b \pmod{H} \iff a = b h \text{ for some } h \in H
    \end{equation}
  \end{definition}
  \begin{proof}
    We show that this indeed forms an equivalence class. 
  \end{proof}

  With this partitioning scheme in mind, the following theorem on the order of such groups becomes very intuitive, and has a lot of consequences. 

  \begin{theorem}[Lagrange's Theorem]
    Let $G$ be a finite group and $H$ its subgroup. Then 
    \begin{equation}
      |G| = [G:H] |H|
    \end{equation}
    where $[G:H]$, called the \textbf{index of $H$}, is the number of cosets in $G$. Therefore, the order of a subgroup of a finite group divides the order of the group. 
  \end{theorem}
  \begin{proof}
    The union of the $[G:H]$ disjoint cosets is all of $G$. On the other hand, every $H$ is in one-to-one correspondence with each coset $aH$, so every coset has $|H|$ elements. Therefore, there are $[G:H] |H|$ elements altogether. 
  \end{proof}

  However, the converse is usually false, as there is a group of order 12 having no subgroup of order 6. 

  \begin{corollary}
    The order of any element of a finite group divides the order of the group. 
  \end{corollary}
  \begin{proof}
    Take any $a \in G$ and construct the cyclic subgroup $\langle a \rangle \subset G$. Then by Lagrange's theorem, $|a| = |\langle a \rangle|$ divides $|G|$. 
  \end{proof}

  \begin{corollary}
    Every finite group of a prime order is cyclic. 
  \end{corollary}
  \begin{proof}
    Let $a \in G$ be any element other than the identity $e$, and consider $\langle a \rangle \subset G$. The order must divide $|G|$ which is prime, so $|a| = 1$ or $|G|$. But $|a| \neq 1$ since we did not choose the identity, so $|a| = |G| \implies \langle a \rangle = G$. 
  \end{proof}

  \begin{corollary}
    If $|G| = n$ and $a \in G$ is arbitrary, then $a^n = e$. 
  \end{corollary}
  \begin{proof}
    Let $|a| = k$. Then $k \mid n$, and so $a^n = a^{kl} = (a^k)^l = e^l = e$. 
  \end{proof}

  \begin{corollary}[Fermant's Little Theorem]
    Let $p$ be a prime number. The multiplicative group $\mathbb{Z}_{p} \setminus \{0\}$ of the field $\mathbb{Z}_{p}$ is an abelian group of order $p-1 \implies g^{p-1} = 1$ for all $g \in \mathbb{Z}_{p} \setminus \{0\}$. So,
    \begin{equation}
      a^{p-1} \equiv 1 \iff a^{p} \equiv a \pmod{p}
    \end{equation}
  \end{corollary} 

  \begin{definition}[Normal Subgroups]
    A subgroup $N \subset G$ is a \textbf{normal subgroup} iff the left cosets equal the right cosets. That is, $\forall b \in G, h \in H$. 
    \begin{equation}
      b^{-1} h b \in H
    \end{equation}
    Every subgroup of an abelian group is normal. 
  \end{definition} 

  The concept of normal subgroups allow us to endow on the quotient set a group structure. 

  \begin{definition}[Quotient Group]
    Given a group $G$ and a normal subgroup $H$, the \textbf{quotient group} $G/H$ is the set of left cosets $aH$ with the operation 
    \begin{equation}
      aH \, bH = abH
    \end{equation}
  \end{definition}

  \begin{lemma} 
    A subgroup $H \subset G$ is normal if and only if there exists a group homomorphism $\phi: G \rightarrow G^\prime$ with $\ker{\phi} = H$. 
  \end{lemma}
  \begin{proof}
    We prove bidirectionally. 
    \begin{enumerate}
      \item $(\rightarrow)$. Since $H$ is normal, we can form the quotient group $G/H$. Let $\phi: G \rightarrow G/H$ be defined $\phi(a) = aH$. Then, 
      \begin{align}
        \ker{\phi} = \phi^{-1}(eH) & = \{a \in G \mid aH = eH = H \} \\
                                   & = \{a \in G \mid a \in H \}
      \end{align}
      Therefore, $\phi$ is a homomorphism because $\phi(ab) = abH = (aH)(bH)$. 
    \end{enumerate}
  \end{proof}

  \begin{theorem}[Quotient Maps are Homomorphisms]
    The map $\pi: G \rightarrow G/H$ is a group homomorphism, and the \textbf{quotient group} is the set of left cosets with 
  \end{theorem}
  \begin{proof}
    
  \end{proof} 

  \begin{corollary}
    If $|G| = n$, then $g^{n} = e$ for all $g \in G$. 
  \end{corollary}

  \begin{definition}[Euler's Totient Function]
    \textbf{Euler's Totient Function}, denoted $\varphi(n)$, consists of all the numbers less than or equal to $n$ that are coprime to $n$. 
  \end{definition}

  \begin{theorem}[Euler's Theorem]
    For any $n$, the order of the group $\mathbb{Z}_{n} \setminus \{0\}$ of invertible elements of the ring $\mathbb{Z}_{n}$ equals $\varphi(n)$, where $\varphi$ is Euler's totient function. In other words with $G = \mathbb{Z}_{n} \setminus \{0\}$, 
    \begin{equation}
      a^{\varphi(n)} \equiv 1 \pmod{n}, \; \text{ where $a$ is coprime to $n$}
    \end{equation}
  \end{theorem}

  \begin{example}
    In $\mathbb{Z}_{125} \setminus \{0\}$, $\varphi(125) = 125 - 25 = 100 \implies 2^{100} \equiv 1 \pmod{125}$
  \end{example}

  \begin{definition}
    Let $G$ be a transformation group on set $X$. Points $x, y \in X$ are equivalent with respect to $G$ if there exists an element $g \in G$ such that $y = g x$. This has already been defined through the equivalence of figures before. This relation splits $X$ into equivalence classes, called \textbf{orbits}. Note that cosets are the equivalence classes of the transformation group $G$; oribits are those of $X$. We denote it as
    \begin{equation}
      Gx \equiv \{ g x \;|\;g \in G \}
    \end{equation}
  \end{definition}

  By definition, transitive transformation groups have only one orbit.

  \begin{definition}
    The subgroup $G_{x} \subset G$, where $G_{x} \equiv \{ g \in G | g x = x\}$ is called the \textbf{stabilizer} of $x$.
  \end{definition}

  \begin{example}
    The orbits of $O(2)$ are concentric circles around the origin, as well as the origin itself. The stabilizer of the point $p \neq 0$ is the identity and the reflection across the line $??$. The stabilizer of $0$ is the entire $O(2)$.
  \end{example}

  \begin{example}
    The group $S_n$ is transitive on the set $\{1, 2, ..., n\}$. The stabilizer of $k, (1 \leq k \leq n)$ is the subgroup $H_{k} \simeq S_{n-1}$, where $H_k$ is the permutation group that does not move $k$ at all. 
  \end{example}

  \begin{theorem}
    There exists a 1-to-1 injective correspondence between an orbit $G_x$ and the set $G / G_{x}$ of cosets, which maps a point $y = g x \in G x $ to the coset $g G_x$. 
  \end{theorem}

  \begin{definition}
    The \textbf{length of an orbit} is the number of elements in it. 
  \end{definition}

  \begin{corollary}
    If $G$ is a finite group, then 
    \begin{equation}
      |G| = |G_x| |G x|
    \end{equation}
    In fact, there exists a precise relation between the stabilizers of points of the same orbit, regardless of $G$ being finite or infinite: 
    \begin{equation}
      G_{g x} = g G_{x} g^{-1}
    \end{equation}
  \end{corollary}

\subsection{Centralizers and Normalizers} 

\subsection{Stabilizers and Orbits}

\subsection{Subgroups Generated by Subsets of a Group} 

  Subgroups of cyclic group

\subsection{Lattice of Subgroups} 


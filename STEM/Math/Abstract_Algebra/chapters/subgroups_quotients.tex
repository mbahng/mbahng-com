\section{Subgroups} 

  We have seen a few examples of subgroups, but we will heavily elaborate on here. We know that given a set, we can define an equivalence relation on it to get a quotient set. Now if we have a group, defining any such equivalence relation may not be compatible with the group structure. Therefore, it would be nice to have some principles in which we can construct such compatible equivalence classes, i.e. through a \textbf{congruence relation} that preserves the operations. 

\subsection{Cosets}

  Fortunately, we can do such a thing by taking a subgroup $H \subset G$ and ``shifting'' it to form the cosets of $G$, which are the equivalence classes. 
  
  \begin{definition}[Coset]
    Given a group $G$, $a \in G$, and subgroup $H$, 
    \begin{enumerate}
      \item A \textbf{left coset} is $a H \coloneqq \{a h \mid h \in H \}$. 
      \item A \textbf{right coset} is $H a \coloneqq \{h a \mid h \in H \}$. 
      \item When $G$ is abelian, the \textbf{coset} is denoted $a + H$. 
    \end{enumerate}
    With this, we can take arbitrary elements $a, b \in G$ and determine if they are in the same coset as such. Since $a \in aH$, $b \in aH$ iff $b = ah$ for some $h \in H$. Therefore, we have the equivalence relation. 
    \begin{equation}
      a \equiv b \pmod{H} \iff a = b h \text{ for some } h \in H
    \end{equation}
  \end{definition}
  \begin{proof}
    We show that this indeed forms an equivalence class. 
    \begin{enumerate}
      \item \textit{Reflexive}. $a \equiv a \pmod{H}$ since $e \in H \implies a = a e$. 
      \item \textit{Symmetric}. Let $a \equiv b \pmod{H}$. Then $a = bh$ for some $h \in H$, but since $H$ is a group, $h^{-1} \in H \implies a h^{-1} = b \implies b \equiv a \pmod{H}$. 
      \item \textit{Transitive}. Let $a \equiv b \pmod{H}$ and $b \equiv c \pmod{H}$. Then $a = bh$ and $b = ch^\prime$ for some $h, h^\prime \in H$. But then 
      \begin{equation}
        a = bh = (ch^\prime) h = c(h^\prime h)
      \end{equation}
      where $h^\prime h \in H$ due to closure. 
    \end{enumerate}
  \end{proof} 

  Note that a coset is \textit{not} a subgroup. It is only the case that $eH = H$ is a subgroup, but for $a \neq e$, $aH$ does not even contain the identity. We should think of a coset as a \textit{translation} of the subgroup $H$. 

  \begin{example}[Familiar Cosets]
    Here are some examples. Note that all it takes is to find \textit{some} subgroup, and the cosets will naturally pop up. 
    \begin{enumerate}
      \item Let $H = 2 \mathbb{Z} \subset (\mathbb{Z}, +)$ be the even integers. Then $0 + H$ and $1 + H$ are the even and odd integers, respectively. 
      \item Let $H = \{e, f\} \subset \Dih(3)$. Then 
      \begin{equation}
        H = \{e, f\}, rH = \{r, rf\}, r^2 H = \{r^2, r^2 f\} 
      \end{equation}
      are the cosets. 
    \end{enumerate}
  \end{example}

  With this partitioning scheme in mind, the following theorem on the order of such groups becomes very intuitive, and has a lot of consequences. 

  \begin{theorem}[Lagrange's Theorem]
    Let $G$ be a finite group and $H$ its subgroup. Then 
    \begin{equation}
      |G| = [G:H] |H|
    \end{equation}
    where $[G:H]$, called the \textbf{index of $H$}, is the number of cosets in $G$. Therefore, the order of a subgroup of a finite group divides the order of the group. 
  \end{theorem}
  \begin{proof}
    The union of the $[G:H]$ disjoint cosets is all of $G$. On the other hand, every $H$ is in one-to-one correspondence with each coset $aH$, so every coset has $|H|$ elements. Therefore, there are $[G:H] |H|$ elements altogether. 
  \end{proof}

  Therefore, Lagrange's theorem says that \textit{given} that you find a subgroup, the order of the subgroup must divide the order of $G$. However, that doesn't mean that such a subgroup may even exist. For example, there is a group of order 12 having no subgroup of order 6. 

  \begin{corollary}
    The order of any element of a finite group divides the order of the group. 
  \end{corollary}
  \begin{proof}
    Take any $a \in G$ and construct the cyclic subgroup $\langle a \rangle \subset G$. Then by Lagrange's theorem, $|a| = |\langle a \rangle|$ divides $|G|$. 
  \end{proof}

  \begin{corollary}
    Every finite group of a prime order is cyclic. 
  \end{corollary}
  \begin{proof}
    Let $a \in G$ be any element other than the identity $e$, and consider $\langle a \rangle \subset G$. The order must divide $|G|$ which is prime, so $|a| = 1$ or $|G|$. But $|a| \neq 1$ since we did not choose the identity, so $|a| = |G| \implies \langle a \rangle = G$. 
  \end{proof}

  \begin{corollary}
    If $|G| = n$, then for every $a \in G$ $a^n = e$. 
  \end{corollary}
  \begin{proof}
    Let $|a| = k$. Then $k \mid n$, and so $a^n = a^{kl} = (a^k)^l = e^l = e$. 
  \end{proof}

  \begin{corollary}[Fermant's Little Theorem]
    Let $p$ be a prime number. The multiplicative group $\mathbb{Z}_{p} \setminus \{0\}$ of the field $\mathbb{Z}_{p}$ is an abelian group of order $p-1 \implies g^{p-1} = 1$ for all $g \in \mathbb{Z}_{p} \setminus \{0\}$. So,
    \begin{equation}
      a^{p-1} \equiv 1 \iff a^{p} \equiv a \pmod{p}
    \end{equation}
  \end{corollary} 

  We can generalize this. 

  \begin{definition}[Euler's Totient Function]
    \textbf{Euler's Totient Function}, denoted $\varphi(n)$, consists of all the numbers less than or equal to $n$ that are coprime to $n$. 
  \end{definition}

  \begin{theorem}[Euler's Theorem]
    For any $n$, the order of the group $\mathbb{Z}_{n} \setminus \{0\}$ of invertible elements of the ring $\mathbb{Z}_{n}$ equals $\varphi(n)$, where $\varphi$ is Euler's totient function. In other words with $G = \mathbb{Z}_{n} \setminus \{0\}$, 
    \begin{equation}
      a^{\varphi(n)} \equiv 1 \pmod{n}, \; \text{ where $a$ is coprime to $n$}
    \end{equation}
  \end{theorem}

  \begin{example}
    In $\mathbb{Z}_{125} \setminus \{0\}$, $\varphi(125) = 125 - 25 = 100 \implies 2^{100} \equiv 1 \pmod{125}$
  \end{example}

\subsection{Normal Subgroups}

  By introducing cosets, we have successfully constructed an equivalence relation on $G$. This set of cosets is indeed a partition of $G$, but we would like to endow it with a group structure that respects that of $G$. That is, let $a, b \in G$ and its corresponding cosets be $aH, bH$. Then, we would like to define an operation $\cdot$ on the cosets such that 
  \begin{equation}
    (aH) \cdot (bH) \coloneqq (ab)H
  \end{equation} 
  That is, we would like to upgrade the equivalence relation to a \textit{congruence relation}. If we try to show that this is indeed a well-defined operation, we run into some trouble. Suppose $aH = a^\prime H$ and $bH = b^\prime H$. Then with our definition, we should be able to derive that $(aH)(bH) = (a^\prime H) (b^\prime H)$ through the equation 
  \begin{equation}
     (aH) (bH) = (ab)H = (a^\prime b^\prime) H = (a^\prime H) (b^\prime H) 
  \end{equation}
  We have $a^\prime = a h_1$, $b^\prime = b h_2$, and $a^\prime b^\prime = ab h$. Then, 
  \begin{align}
    (ab) H = (a^\prime b^\prime) H & \implies a^\prime b^\prime = abh \text{ for some } h \in H \\
                                   & \implies a h_1 b h_2 = abh \text{ for some } h_1, h_2, h \in H
  \end{align}
  But the final statement is not true in general. In an abelian group, we could just swap $h_1$ and $b$ to derive it completely, but perhaps there is a weaker condition on just the subgroup $H$ that allows us to ``swap'' the two. 

  \begin{definition}[Normal Subgroups]
    A subgroup $N \subset G$ is a \textbf{normal subgroup} iff the left cosets equal the right cosets. That is, $\forall g \in G, h \in H$. 
    \begin{equation}
      g^{-1} h g \in H
    \end{equation}
    We call $g^{-1} h g$ the \textbf{conjugate} of $h$ by $g$. 
  \end{definition} 

  As stated above, it is easy to see that every subgroup of an abelian group is normal. In fact, we can do better. 

  \begin{lemma} 
    A subgroup $H \subset G$ is normal if and only if there exists a group homomorphism $\phi: G \rightarrow G^\prime$ with $\ker{\phi} = H$. 
  \end{lemma}
  \begin{proof}
    We prove bidirectionally. 
    \begin{enumerate}
      \item $(\rightarrow)$. Since $H$ is normal, we can form the quotient group $G/H$. Let $\phi: G \rightarrow G/H$ be defined $\phi(a) = aH$. Then, 
      \begin{align}
        \ker{\phi} = \phi^{-1}(eH) & = \{a \in G \mid aH = eH = H \} \\
                                   & = \{a \in G \mid a \in H \}
      \end{align}
      Therefore, $\phi$ is a homomorphism because $\phi(ab) = abH = (aH)(bH)$. 
    \end{enumerate}
  \end{proof}

  \begin{example}[Normal Subgroups]
    
  \end{example}

  \begin{example}[Subgroups that are Not Normal]
    
  \end{example}

\subsection{Quotient Groups}

  Now that we know about normal subgroups, this allows us to endow on the quotient set a group structure. 

  \begin{definition}[Quotient Group]
    Given a group $G$ and a normal subgroup $H$, the \textbf{quotient group} $G/H$ is the set of left cosets $aH$ with 
    \begin{enumerate}
      \item the operation $(aH) \cdot (bH) \coloneqq (ab)H$ 
      \item the identity element $eH$. 
      \item inverses $(aH)^{-1}) = (a^{-1})H$. 
    \end{enumerate}
    and order $|G/H| = |G| / |H|$.  
  \end{definition}
  \begin{proof}
    It suffices to check that multiplication is well defined. Suppose as above that $aH = a^\prime H$ and $bH = b^\prime H$. Then $a^\prime = ah$ and $b^\prime = bk$ for some $h, k \in H$. Since $H$ is normal, $b^{-1} h b = h^\prime$ for some $h^\prime \in H$. Therefore, 
    \begin{equation}
      a^\prime b^\prime = (ah) (bk) = a(hb) k = (ab h^\prime) k = (ab)(h^\prime k) \in (ab) H
    \end{equation}
    and so $(ab)H = (a^\prime b^\prime)H$. 
  \end{proof}

  \begin{theorem}[Quotient Maps are Homomorphisms]
    The map $\pi: G \rightarrow G/H$ is a group homomorphism, and the \textbf{quotient group} is the set of left cosets with 
  \end{theorem}
  \begin{proof}
    
  \end{proof} 

  \begin{theorem}[Fundamental Group Homomorphism Theorem]
    Let $f: G \to G^\prime$ be a surjective homomorphism. Then $G/{\ker{f}} \simeq G^\prime$.\footnote{Note that if $f$ is not surjective, we can just have it be surjective by restricting $G^\prime$ to be the image of $f$. }

    \begin{figure}[H]
      \centering 
      \begin{tikzcd}
        G \arrow[r, "f"] \arrow[d, "p"] & G' \\
        G/\ker{f} \arrow[ur, "\bar{f}"'] &
      \end{tikzcd}
      \caption{Given $f$ and the projection map $p: G \to G/{\ker{f}}$, this induces an isomorphism $\bar{f}$ such that $f = \bar{f} \circ p$.} 
      \label{fig:group_fund_homo_theorem}
    \end{figure}
  \end{theorem}

\subsection{Orbits and Stabilizers}

  \begin{definition}[Orbits]
    Let $G$ be a transformation group on set $X$. Points $x, y \in X$ are equivalent with respect to $G$ if there exists an element $g \in G$ such that $y = g x$. This has already been defined through the equivalence of figures before. This relation splits $X$ into equivalence classes, called \textbf{orbits}. Note that cosets are the equivalence classes of the transformation group $G$; oribits are those of $X$. We denote it as
    \begin{equation}
      Gx \equiv \{ g x \;|\;g \in G \}
    \end{equation}
  \end{definition}

  By definition, transitive transformation groups have only one orbit.

  \begin{definition}
    The subgroup $G_{x} \subset G$, where $G_{x} \equiv \{ g \in G | g x = x\}$ is called the \textbf{stabilizer} of $x$.
  \end{definition}

  \begin{example}
    The orbits of $O(2)$ are concentric circles around the origin, as well as the origin itself. The stabilizer of $0$ is the entire $O(2)$.
  \end{example}

  \begin{example}
    The group $S_n$ is transitive on the set $\{1, 2, ..., n\}$. The stabilizer of $k, (1 \leq k \leq n)$ is the subgroup $H_{k} \simeq S_{n-1}$, where $H_k$ is the permutation group that does not move $k$ at all. 
  \end{example}

  \begin{theorem}
    There exists a 1-to-1 injective correspondence between an orbit $G_x$ and the set $G / G_{x}$ of cosets, which maps a point $y = g x \in G x $ to the coset $g G_x$. 
  \end{theorem}

  \begin{corollary}
    If $G$ is a finite group, then 
    \begin{equation}
      |G| = |G_x| |G x|
    \end{equation}
    In fact, there exists a precise relation between the stabilizers of points of the same orbit, regardless of $G$ being finite or infinite: 
    \begin{equation}
      G_{g x} = g G_{x} g^{-1}
    \end{equation}
  \end{corollary}

\subsection{Centralizers and Normalizers} 

\subsection{Lattice of Subgroups} 


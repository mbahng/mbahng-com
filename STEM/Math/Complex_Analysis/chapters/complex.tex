\section{The Complex Numbers} 

  The next field that will be particularly important is the complex numbers. It is straightforward to construct $\mathbb{C}$, but let's motivate this for a minute. 

  \begin{example}[Polynomial Roots]
    The roots of the polynomial 
    \begin{equation}
      f(x) = x^2 + 1
    \end{equation}
    does not exist in $\mathbb{R}$. 
  \end{example} 

  Therefore, we would like to construct a new space that contains all possible roots for all possible polynomials with real coefficients. We call this $\mathbb{C}$. Clearly, by constructing polynomials of the form $x^2 - r^2$ for some $r \in \mathbb{R}$, we know that $\mathbb{R} \subset \mathbb{C}$. Therefore, we want to create a further extension of $\mathbb{R}$, along with some canonical injection $\iota: \mathbb{R} \rightarrow \mathbb{C}$ that is also a field homomorphism. It turns out that once we construct this field, there is no possible way that we can make it an ordered field. However, the norm extends naturally into $\mathbb{C}$ such that $\iota$ is isometric. Finally, we can define a new operator called \textit{conjugation} that gives us additional structure. 

  This is not the only way to construct the complex plane however. Rather than defining all these from scratch, we could just define the addition operations with an isometric vector space isomorphism from $\mathbb{R}^2$ to $\mathbb{C}$ actually, and then define multiplication. Another way is to start again with $\mathbb{Q} \times \mathbb{Q}$, define a norm on it, complete it, and finally define the addition and multiplication operations that satisfy the field property.   

\subsection{Construction}

  \begin{theorem}[Construction of the Complex Numbers]
    Let $\mathbb{C}$ be defined as the space $\mathbb{R} \times \mathbb{R}$ with the following operations. 
    \begin{enumerate}
      \item \textit{Addition}. $x = (a, b), y = (c, d) \implies x +_{\mathbb{C}} y = (a + c, b + d)$. 
      \item \textit{Additive Identity}. $0_{\mathbb{C}} = (0, 0)$. 
      \item \textit{Additive Inverse}. $x = (a, b) \implies -x = (-a, -b)$. 
      \item \textit{Multiplication}. $x = (a, b), y = (c, d) \implies x \times_{\mathbb{C}} y = (ac - bd, ad + bc)$. 
      \item \textit{Multiplicative Identity}. $1_{\mathbb{C}} = (1, 0)$. 
      \item \textit{Multiplicative Inverse}. 
      \begin{equation}
        x = (a, b) \implies x^{-1} = \bigg( \frac{a}{a^2 + b^2}, \frac{-b}{a^2 + b^2} \bigg)
      \end{equation}
    \end{enumerate}
    Our first claim is that $(\mathbb{C}, +_{\mathbb{C}}, \times_{\mathbb{C}})$ is a field. Furthermore, we define the additional structures
    \begin{enumerate}
      \item \textit{Conjugate}. $x = (a, b) \implies \overline{x} = (a, -b)$. 
      \item \textit{Norm}. $|x|_{\mathbb{C}} = x \times_{\mathbb{C}} \overline{x} = a^2 + b^2$. 
      \item \textit{Metric}. This is the norm-induced metric. $d_{\mathbb{C}}(x, y) = |x - y|_{\mathbb{C}}$. 
      \item \textit{Topology}. This is the metric-induced topology generated by the open balls $B(x, r) \coloneqq \{y \in \mathbb{C} | d(x, y) < r\}$, where $x \in \mathbb{C}, r \in \mathbb{R}$. 
    \end{enumerate} 
    Our second claim is that the canonical injection $\iota: \mathbb{R} \rightarrow \mathbb{C}$ defined 
    \begin{equation}
      \iota(r) = (r, 0)
    \end{equation}
    is an isometric field isomorphism. Our third claim is that $\mathbb{C}$ is Cauchy-complete with respect to this metric. 
  \end{theorem} 

  Note that we do not talk about order $\mathbb{C}$, and so the concepts of Dedekind completeness, least upper bound properties, or Archimedean principle is meaningless in the complex plane. 

  \begin{definition}[Imaginary Number] 
    Let us denote $i = (0, 1)$ which we call the \textbf{imaginary number}, which has the property that $i^2 = 1$. With this notation, we can see through abuse of notation that 
    \begin{equation}
      (a, b) = (a, 0) + (0, b) = (a, 0) + (b, 0) (0, 1) = a + bi
    \end{equation} 
    Therefore, we generally write complex numbers as $z = a + bi$, and we define the real and imaginary components as $\re(z)$ and $\im(z)$, respectively. 
  \end{definition}

  Note that the identity $x^2 + 1 \equiv (x + i) (x - i)$ implies that the equation $x^2 = -1$ has exactly two solutions in $\mathbb{C}$, $i$ and $-i$. Therefore, if a subfield of $\mathbb{C}$ contains one of these solutions, it must contain the other (since $i$ and $-i$ are additive and multiplicative inverses). 

  Furthermore, since $i$ is defined to be $\sqrt{-1}$, we could replace $i$ with $-i$ and our calculations would still be consistent throughout the rest of mathematics. In fact, $i$ and $-i$ behave \textbf{exactly} identically and cannot be distinguished in an abstract sense. Visually, the complex plane "flipped" across the real number axis produces the same complex plane. 

  \begin{theorem}[Uniqueness of $\mathbb{C}$]
    $\mathbb{C}$ is unique up to an isomorphism that maps all real numbers to themselves. Every complex number can be uniquely written as $a + bi$, where $a, b \in \mathbb{R}$ and $i$ is a fixed element such that $i^2 = -1$. 
  \end{theorem}
  \begin{proof}
    Consider the subset of $\mathbb{C}$
    \begin{equation}
      K \equiv \{ a + bi \; | \; a, b \in \mathbb{R}\}
    \end{equation}
    By evaluating its operations, we can check for closure, identity, and invertibility of nonzero elements to conclude that $K$ is a subfield of $\mathbb{C} \implies$ by prop. (iii), $K = \mathbb{C} \implies$ every element in $\mathbb{C}$ can be written in form $a + bi$. To prove uniqueness, we assume that $p \in \mathbb{C}$ can be written in distinct forms $p = a + bi = a^{\prime} + b^\prime i$. Then
    \begin{align*}
       a + bi = a^{\prime} + b^\prime i & \implies (a - a^\prime)^2 = (b^\prime i - b i)^2 = - (b^\prime - b)^2 \\
       & \implies a - a^\prime = b^\prime - b = 0
    \end{align*}
    To prove uniqueness of $\mathbb{C}$ up to ismorphism, we assume that $\mathbb{C}^\prime$ exists with $i^\prime$ such that $i^{\prime 2}$ containing elements $a + b i'$. Let $f: \mathbb{C} \longrightarrow \mathbb{C}^\prime$ defined 
    \begin{equation}
      f( a + bi) = a + bi^\prime
    \end{equation}
    Then, 
    \begin{align*}
      f\big((a + b i) + (c + d i) \big) & = f\big( (a + c) + (b + d)i \big) \\
      & = (a + c) + (b + d) i^\prime \\
      & = (a + b i^\prime) + (c + d i^\prime) \\
      & = f(a + b i) + f( c + d i) \\
      f\big( \kappa (a + b i)\big) & = f\big( \kappa a + \kappa b i\big) \\
      & = \kappa a + \kappa b i^\prime \\
      & = \kappa (a + b i^\prime) \\
      & = \kappa f(a + b i)
    \end{align*}
    So, $f$ is an isomorphism, and $\mathbb{C} \simeq \mathbb{C}^\prime$. From analysis, we can construct and prove the existence of $\mathbb{R}$. We then define the map
    \begin{equation}
      \rho: \mathbb{R}^2 \longrightarrow \mathbb{C}, \; \rho(a, b) \equiv a + bi
    \end{equation}
    with $\rho(1, 0)$ as the multiplicative identity and $\rho(0,1) \equiv i$. Therefore, every element of $\mathbb{C}$ can be uniquely represented as an element of $\mathbb{R}^2$. 
  \end{proof}

  Unfortunately, we lose the ordering. 

  \begin{theorem}[Order on Complex Plane]
    There exists no order on $\mathbb{C}$ that makes it a totally ordered field.
  \end{theorem}
  \begin{proof}
    We attempt to construct an order on $i$ and $0$ in $\mathbb{C}$. 
    \begin{enumerate}
      \item If $i = 0$, then $i^4 = 0 \cdot i^3 \implies 1 = 0$, which contradicts that $0 < 1$. 
      \item If $i \neq 0$, then $i^2 > 0$ from the field axioms, and so $-1 > 0$. But this also means that $1 = i^4 > 0$. This contradicts the ordered field property that $x > 0 \iff -x < 0$. 
    \end{enumerate}
    Therefore $\mathbb{C}$ cannot be turned into an ordered field. 
  \end{proof}

\subsection{Properties of the Complex Plane}

  \begin{theorem}[Conjugation is an Isomorphism]
    Conjugation is an isometric field automorphism of $\mathbb{C}$. 
    \begin{equation}
      c = a + b i \mapsto \bar{c} = a - b i
    \end{equation}
    This is identically defined by replacing $i$ with $-i$. Clearly, $\bar{\bar{c}} = c$. 
  \end{theorem}
  \begin{proof}
    
  \end{proof}

  \begin{proposition}[Properties of Conjugation]
    For any $c \in \mathbb{C}$, $c + \bar{c}$ and $c \bar{c}$ are real. 
  \end{proposition}
  \begin{proof}
    Using the fact that the complex conjugate is an isomorphism, 
    \begin{align*}
      & \bar{c + \bar{c}} = \bar{c} + \bar{\bar{c}} = \bar{c} + c = c + \bar{c} \\
      & \bar{ c \bar{c}} = \bar{c} \bar{\bar{c}} = \bar{c} c = c \bar{c}
    \end{align*}
  \end{proof}
  Note that we proved this abstractly using only the properties given above, and did not decompose $c$ to its \textbf{algebraic form} $a + b i$. 

  If $c = a + b i, \; a, b \in \mathbb{R}$, then 
  \begin{equation}
    c + \bar{c} = 2a, \; c \bar{c} = a^2 + b^2
  \end{equation}

\subsection{Polar Coordinates}

  In case the reader is unaware, it is common to interpret complex numbers $c = a + b i$ as points or vectors $(a, b)$ on the complex plane. 

  \begin{definition}[Polar Form of Complex Numbers]
    The \textbf{polar representation}, or \textbf{trigonometric representation}, of a complex number $c = a + b i$ is defined using the equations 
    \begin{equation}
      a = r \cos{\varphi}, \; b = r\sin{\varphi} \implies c = r (\cos{\varphi} + i \sin{\varphi})
    \end{equation}
    where $r = |c|$ and $\varphi$ is the \textbf{argument} of $c$, which is 
    the angle formed by the corresponding vector with the polar axis defined within the interval $[0, 2\pi)$. 
    \begin{equation}
      \text{arg}(c) \equiv \tan^{-1}{\frac{b}{a}}
    \end{equation}
    This mapping can be defined 
    \begin{equation}
      \rho: \mathbb{R} \times \frac{\mathbb{R}}{2 \pi} \longrightarrow \mathbb{C}, \; \rho(r, \varphi) = r (\cos{\varphi} + i \sin{\varphi})
    \end{equation}
  \end{definition}

  \begin{theorem}
    $\rho$ is "similar" to a homomorphism in the following way. By defining the domain and codomain as groups, 
    \begin{equation}
      \rho: \big( \mathbb{R}, \times \big) \times \Big( \frac{\mathbb{R}}{2 \pi} \Big) \longrightarrow \big( \mathbb{C}, \times \big)
    \end{equation}
    we can see that
    \begin{equation}
      \rho (r_1, \varphi_1) \times \rho(r_2, \varphi_2) = \rho(r_1 \times r_2, \varphi_1 + \varphi_2) 
    \end{equation}
    or equivalently, 
    \begin{equation}
      r_1 (\cos{\varphi_1} + i \sin{\varphi_1}) \cdot r_2 (\cos{\varphi_2} + i \sin{\varphi_2}) = r_1 r_2 (\cos{(\varphi_1 + \varphi_2)} + i \sin{(\varphi_1 + \varphi_2)})
    \end{equation}
  \end{theorem}

  \begin{corollary}
    The formula for the ratio of complex numbers is defined
    \begin{equation}
      \frac{r_1 (\cos{\varphi_1} + i \sin{\varphi_1})}{r_2 (\cos{\varphi_2} + i \sin{\varphi_2})} = \frac{r_1}{r_2} (\cos{(\varphi_1 - \varphi_2)} + i \sin{(\varphi_1 - \varphi_2)})
    \end{equation}
  \end{corollary}

  \begin{corollary}
    The positive integer power of a complex number can be written using \textbf{De Moivre's formula}. 
    \begin{equation}
      \big(r(\cos{\varphi} + i \sin{\varphi})\big)^n = r^n (\cos{n \varphi} + i \sin{n \varphi})
    \end{equation}
  \end{corollary}

\subsection{Roots, Exponentials, Logarithms}

  We can use this formula to extract a root of $n$th degree from a complex number $c = r(\cos{\varphi} + i \sin{\varphi})$, which means to solve the equation $z^n = c$. Let $z = s (\cos{\psi} + i \sin{\psi})$. Then by De Moivre's formula, 
  \begin{align*}
    z^n & = s^n (\cos{n \psi} + i \sin{n \psi}) = r(\cos{\varphi} + i \sin{\varphi}) \\
    & \implies s = \sqrt[n]{r}, \; \psi = \frac{\varphi + 2\pi k}{n} \\
    & \implies z = \sqrt[n]{r} \bigg( \cos{\frac{\varphi + 2\pi k}{n}} + i \sin{\frac{\varphi + 2\pi k}{n}}\bigg) \text{ for } k = 0, 1, ..., n-1
  \end{align*}
  Geometrically, the $n$ solutions lie at the vertices of a regular $n$-gon centered at the origin. When $c = 1$, the solutions are the $n$th roots of unity.

\subsection{Trigonometric Functions}

  Now with complex numbers, we have a yet another way of defining trigonometric functions that generalizes that of the reals. We can use the series representation. 

\subsection{Dual Numbers}

  Another similar number system. 


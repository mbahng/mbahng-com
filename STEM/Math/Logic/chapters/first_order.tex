\section{First-Order Logic} 

  In propositional logic, we deal with simple declarative propositions. \textbf{First-order logic} extends this by covering predicates and quantification. Let's motivate them. 

  We can think of predicates as properties. If we say \textit{Socrates is a philosopher} and \textit{Plato is a philosopher}, in propositional logic both these statements, represented as $P$ and $Q$, as utterances that are either true or false, and they are completely independent from one another. However, we may want to view them as an application of a predicate \textit{ $\ast$ is a philosopher} on the entities \textit{Socrates} and \textit{Plato}. This motives the formalism of the domain of discourse and the predicate. 

  \begin{definition}[Domain of Discourse]
    Given an individual $x$, its \textbf{domain of discourse} is the set over which certain variables of interest in some formal treatment may range. 
  \end{definition}

  \begin{definition}[Predicate]
    A \textbf{predicate} $P$ is a symbol that represents a property or a relation of a certain individual $x$ in a domain of discourse. Using predicates, $P(x)$ can be viewed as a proposition about the individual $x$. 
  \end{definition} 

  Note that a predicate itself is not a proposition, since saying \textit{$\ast$ is a philosopher} doesn't have any truth or false meaning to it, akin to a sentence fragment. But it is a placeholder $P(\cdot)$ upon which if an individual $x$ is put in, it makes sense to ask whether $P(x)$ is true. 

  \begin{definition}[Formula]
    A \textbf{formula} is a string of propositions, connectives, predicates, and variables $\phi$ that turns into a proposition once all free variables have been instantiated. 
  \end{definition}

  With predicates alone, all we have really done is add notational convenience. However, if we want to state a proposition not just about $x$, but its domain of discourse, then we can use quantifiers. 

  \begin{definition}[Quantifier]
    A \textbf{quantifier} is an operator that specifies how many individuals in the domain of source satisfy a proposition. The two most used quantifiers are 
    \begin{enumerate}
      \item \textit{Universal Quantification}. $\forall$, which means \textit{for every}. 
      \item \textit{Existential Quantification}. $\exists$, which means \textit{there exists}. 
    \end{enumerate}
  \end{definition} 

  These quantifiers are additional symbols in our language $\mathcal{L}$. If we add the equality symbol, we get first-order logic with equality. 

  \begin{axiom}[Equality]
    \textbf{Equality} is a primitive logical symbol which is always interpreted as the real equality relation between members of the domain of discourse. These equality axioms are: 
    \begin{enumerate}
      \item \textit{Reflexivity}. For each variable $x$, $x = x$. 
      \item \textit{Substitution for Functions}. For all variables $x$ and $y$, and any function symbol $f$, 
        \begin{equation}
          x = y \implies f(x) = f(y)
        \end{equation}
      \item \textit{Substitution for Formulas}. For any variables $x$ and $y$, and any formula $\phi(z)$ with free variable $z$, then 
        \begin{equation}
          x = y \implies (\phi(x) \implies \phi(y))
        \end{equation}
    \end{enumerate}
    Symmetry and transitivity follow from the axioms above. 
  \end{axiom} 

  Ordinary first-order interpretations have a single domain of discourse over which all quantifiers range. \textbf{Many-sorted first-order logic}, or \textbf{typed first-order logic} allows variables to have different \textbf{sorts} or \textbf{types}, i.e. coming from different domains.  

\subsection{Exercises}

  \begin{exercise}[Shifrin Abstract Algebra Appendix 1.1] 
    Negate the following sentences; in each case, indicate whether the original sentence or its negation is a true statement. Be sure to move the ``not" through all the quantifiers.
    \begin{enumerate}
      \item For every integer $n \geq 2$, the number $2^n - 1$ is prime.
      \item There exists a real number $M$ so that for all real numbers $t$, $|\sin t| \leq M$.
      \item For every real number $x > 0$, there exists a real number $y > 0$ so that $xy > 1$.
    \end{enumerate}
  \end{exercise}
  \begin{solution}
    Listed. 
    \begin{enumerate}
      \item \textit{Negation}. For at least one $n \geq 2$, the number $2^n - 1$ is composite (not prime). The negation is true. Consider $n = 4 \implies 2^4 - 1 = 15 = 3 \cdot 5$. 

      \item \textit{Negation}. There exists no real number $M$ such that for all real numbers $t$, $|\sin{t}| \leq M$. The original is true. Pick $M=1$, and by definition $|\sin{t}| \leq 1$. 

      \item \textit{Negation}. For at least one real number $x > 0$, there exists no real number $y > 0$ so that $xy > 1$. The original is true. Given a real number $x > 0$, choose $y = \frac{1}{x} + 1$. Then, 
      \begin{equation}
        xy = x \bigg( \frac{1}{x} + 1 \bigg) = 1 + x > 1
      \end{equation} 
      where the steps follow from the ordered field properties of $\mathbb{R}$. 
    \end{enumerate}
  \end{solution} 

  \begin{exercise}[Shifrin Abstract Algebra Appendix 1.4]
    Suppose $n$ is an odd integer. Prove:
    \begin{enumerate}
      \item The equation $x^2 + x - n = 0$ has no solution $x \in \mathbb{Z}$.
      \item Prove that for any $m \in \mathbb{Z}$, the equation $x^2 + 2mx + 2n = 0$ has no solution $x \in \mathbb{Z}$.
    \end{enumerate}
  \end{exercise}
  \begin{solution}
    We prove by contradiction. Assume such a solution $x$ exists for odd $n$. We consider the two cases where $x$ 
    \begin{enumerate}
      \item is even. 
      \begin{align}
        x \text{ is even} & \implies x \equiv 0 \; (\mathrm{mod } 2) \\
                          & \implies x^2 + x \equiv 0 \; (\mathrm{mod } 2) \\
                          & \implies x^2 + x - n \equiv 1 \; (\mathrm{mod } 2)
      \end{align} 

      \item is odd. 
      \begin{align}
        x \text{ is odd} & \implies x \equiv 1 \; (\mathrm{mod } 2) \\
                         & \implies x^2 + x \equiv 1 + 1 \equiv 0 (\mathrm{mod } 2) \\
                         & \implies x^2 + x - n \equiv 1 (\mathrm{mod } 2) 
      \end{align}
    \end{enumerate}
    Both cases result in the quadratic expression lying in the equivalence class $[1]$ and thus cannot be $0$. This contradicts our assumption that it is a solution. 
    We prove by contradiction. Assume a solution $x$ exists for odd $n$. Note that since $x^2 + 2mx + 2n \equiv x^2 \equiv 0 \; (\mathrm{mod } 2)$, this implies that $x \equiv 0 \; (\mathrm{mod } 2)$.\footnote{This is true if we look at the contrapositive: $x \equiv 1 \implies x^2 \equiv 1$. } Therefore, we can write $x = 2x^\prime$ for some $x^\prime \in \mathbb{Z}$, our assumption is equivalent to the existence of $x^\prime$. Substituting this gives 
    \begin{equation}
      4 x^{\prime 2} + 4 m x^\prime + 2n = 0 \iff 2x^{\prime 2} + 2m x^\prime + n = 0
    \end{equation} 
    Since $2x^{\prime 2} + 2 m x^\prime$ is even, $n$ must be even as well, which contradicts our assumption that $n$ is odd. 
  \end{solution}


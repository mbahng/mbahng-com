\documentclass{article}

% packages
  % basic stuff for rendering math
  \usepackage[letterpaper, top=1in, bottom=1in, left=1in, right=1in]{geometry}
  \usepackage[utf8]{inputenc}
  \usepackage[english]{babel}
  \usepackage{amsmath} 
  \usepackage{amssymb}
  % \usepackage{amsthm}

  % extra math symbols and utilities
  \usepackage{mathtools}        % for extra stuff like \coloneqq
  \usepackage{mathrsfs}         % for extra stuff like \mathsrc{}
  \usepackage{centernot}        % for the centernot arrow 
  \usepackage{bm}               % for better boldsymbol/mathbf 
  \usepackage{enumitem}         % better control over enumerate, itemize
  \usepackage{hyperref}         % for hypertext linking
  \usepackage{fancyvrb}          % for better verbatim environments
  \usepackage{newverbs}         % for texttt{}
  \usepackage{xcolor}           % for colored text 
  \usepackage{listings}         % to include code
  \usepackage{lstautogobble}    % helper package for code
  \usepackage{parcolumns}       % for side by side columns for two column code
  

  % page layout
  \usepackage{fancyhdr}         % for headers and footers 
  \usepackage{lastpage}         % to include last page number in footer 
  \usepackage{parskip}          % for no indentation and space between paragraphs    
  \usepackage[T1]{fontenc}      % to include \textbackslash
  \usepackage{footnote}
  \usepackage{etoolbox}

  % for custom environments
  \usepackage{tcolorbox}        % for better colored boxes in custom environments
  \tcbuselibrary{breakable}     % to allow tcolorboxes to break across pages

  % figures
  \usepackage{pgfplots}
  \pgfplotsset{compat=1.18}
  \usepackage{float}            % for [H] figure placement
  \usepackage{tikz}
  \usepackage{tikz-cd}
  \usepackage{circuitikz}
  \usetikzlibrary{arrows}
  \usetikzlibrary{positioning}
  \usetikzlibrary{calc}
  \usepackage{graphicx}
  \usepackage{algorithmic}
  \usepackage{caption} 
  \usepackage{subcaption}
  \captionsetup{font=small}

  % for tabular stuff 
  \usepackage{dcolumn}

  \usepackage[nottoc]{tocbibind}
  \pdfsuppresswarningpagegroup=1
  \hfuzz=5.002pt                % ignore overfull hbox badness warnings below this limit

% New and replaced operators
  \DeclareMathOperator*{\card}{card}
  \DeclareMathOperator*{\argmin}{\arg\!\min}
  \DeclareMathOperator*{\argmax}{\arg\!\max}
  \newcommand{\qed}{\hfill$\blacksquare$}     % I like QED squares to be black

% Custom Environments
  \newtcolorbox[auto counter, number within=section]{question}[1][]
  {
    colframe = orange!25,
    colback  = orange!10,
    coltitle = orange!20!black,  
    breakable, 
    title = \textbf{Question \thetcbcounter ~(#1)}
  }

  \newtcolorbox[auto counter, number within=section]{exercise}[1][]
  {
    colframe = teal!25,
    colback  = teal!10,
    coltitle = teal!20!black,  
    breakable, 
    title = \textbf{Exercise \thetcbcounter ~(#1)}
  }
  \newtcolorbox[auto counter, number within=section]{solution}[1][]
  {
    colframe = violet!25,
    colback  = violet!10,
    coltitle = violet!20!black,  
    breakable, 
    title = \textbf{Solution \thetcbcounter}
  }
  \newtcolorbox[auto counter, number within=section]{lemma}[1][]
  {
    colframe = red!25,
    colback  = red!10,
    coltitle = red!20!black,  
    breakable, 
    title = \textbf{Lemma \thetcbcounter ~(#1)}
  }
  \newtcolorbox[auto counter, number within=section]{theorem}[1][]
  {
    colframe = red!25,
    colback  = red!10,
    coltitle = red!20!black,  
    breakable, 
    title = \textbf{Theorem \thetcbcounter ~(#1)}
  } 
  \newtcolorbox[auto counter, number within=section]{proposition}[1][]
  {
    colframe = red!25,
    colback  = red!10,
    coltitle = red!20!black,  
    breakable, 
    title = \textbf{Proposition \thetcbcounter ~(#1)}
  } 
  \newtcolorbox[auto counter, number within=section]{corollary}[1][]
  {
    colframe = red!25,
    colback  = red!10,
    coltitle = red!20!black,  
    breakable, 
    title = \textbf{Corollary \thetcbcounter ~(#1)}
  } 
  \newtcolorbox[auto counter, number within=section]{proof}[1][]
  {
    colframe = orange!25,
    colback  = orange!10,
    coltitle = orange!20!black,  
    breakable, 
    title = \textbf{Proof. }
  } 
  \newtcolorbox[auto counter, number within=section]{definition}[1][]
  {
    colframe = yellow!25,
    colback  = yellow!10,
    coltitle = yellow!20!black,  
    breakable, 
    title = \textbf{Definition \thetcbcounter ~(#1)}
  } 
  \newtcolorbox[auto counter, number within=section]{example}[1][]
  {
    colframe = blue!25,
    colback  = blue!10,
    coltitle = blue!20!black,  
    breakable, 
    title = \textbf{Example \thetcbcounter ~(#1)}
  } 
  \newtcolorbox[auto counter, number within=section]{axiom}[1][]
  {
    colframe = green!25,
    colback  = green!10,
    coltitle = green!20!black,  
    breakable, 
    title = \textbf{Axiom \thetcbcounter ~(#1)}
  } 
  \newtcolorbox[auto counter, number within=section]{algo}[1][]
  {
    colframe = green!25,
    colback  = green!10,
    coltitle = green!20!black,  
    breakable, 
    title = \textbf{Algorithm \thetcbcounter ~(#1)}
  } 

  \BeforeBeginEnvironment{example}{\savenotes}
  \AfterEndEnvironment{example}{\spewnotes}
  \BeforeBeginEnvironment{lemma}{\savenotes}
  \AfterEndEnvironment{lemma}{\spewnotes}
  \BeforeBeginEnvironment{theorem}{\savenotes}
  \AfterEndEnvironment{theorem}{\spewnotes}
  \BeforeBeginEnvironment{corollary}{\savenotes}
  \AfterEndEnvironment{corollary}{\spewnotes}
  \BeforeBeginEnvironment{proposition}{\savenotes}
  \AfterEndEnvironment{proposition}{\spewnotes}
  \BeforeBeginEnvironment{definition}{\savenotes}
  \AfterEndEnvironment{definition}{\spewnotes}
  \BeforeBeginEnvironment{exercise}{\savenotes}
  \AfterEndEnvironment{exercise}{\spewnotes}
  \BeforeBeginEnvironment{proof}{\savenotes}
  \AfterEndEnvironment{proof}{\spewnotes}
  \BeforeBeginEnvironment{solution}{\savenotes}
  \AfterEndEnvironment{solution}{\spewnotes}
  \BeforeBeginEnvironment{question}{\savenotes}
  \AfterEndEnvironment{question}{\spewnotes}
  \BeforeBeginEnvironment{axiom}{\savenotes}
  \AfterEndEnvironment{axiom}{\spewnotes}
  \BeforeBeginEnvironment{algo}{\savenotes}
  \AfterEndEnvironment{algo}{\spewnotes}

  \definecolor{dkgreen}{rgb}{0,0.6,0}
  \definecolor{gray}{rgb}{0.5,0.5,0.5}
  \definecolor{mauve}{rgb}{0.58,0,0.82}
  \definecolor{darkblue}{rgb}{0,0,139}
  \definecolor{lightgray}{gray}{0.93}
  \renewcommand{\algorithmiccomment}[1]{\hfill$\triangleright$\textcolor{blue}{#1}}

  % default options for listings (for code)
  \lstset{
    autogobble,
    frame=ltbr,
    language=Python,
    aboveskip=3mm,
    belowskip=3mm,
    showstringspaces=false,
    columns=fullflexible,
    keepspaces=true,
    basicstyle={\small\ttfamily},
    numbers=left,
    firstnumber=1,                        % start line number at 1
    numberstyle=\tiny\color{gray},
    keywordstyle=\color{blue},
    commentstyle=\color{dkgreen},
    stringstyle=\color{mauve},
    backgroundcolor=\color{lightgray}, 
    breaklines=true,                      % break lines
    breakatwhitespace=true,
    tabsize=3, 
    xleftmargin=2em, 
    framexleftmargin=1.5em, 
    stepnumber=1
  }

% Page style
  \pagestyle{fancy}
  \fancyhead[L]{Set Theory}
  \fancyhead[C]{Muchang Bahng}
  \fancyhead[R]{Spring 2025} 
  \fancyfoot[C]{\thepage / \pageref{LastPage}}
  \renewcommand{\footrulewidth}{0.4pt}          % the footer line should be 0.4pt wide
  \renewcommand{\thispagestyle}[1]{}  % needed to include headers in title page

\begin{document}

\title{Logic and Set Theory}
\author{Muchang Bahng}
\date{Spring 2025}

\maketitle
\tableofcontents
\pagebreak

\section{Propositional Logic} 

    Philosophers still debate about what a proposition really means. As a complete beginner, I mention some interpretations of it, but I by no means claim that this is the definitive definition. 

    \begin{definition}[Possible World]
      A \textbf{possible world} is a complete and consistent way the world is or could have been. 
    \end{definition} 

    The \textbf{language} of propositional logic consists of just two things: propositions and connectives. 

    \begin{definition}[Proposition]
      A \textbf{proposition} does not have a formal definition, but we can describe it in the following ways.  
      \begin{enumerate}
        \item They can be understood as an indicator function $f: W \rightarrow \{T, F\}$\footnote{$T, F$ stands for True, False.} that takes in a possible world and returns a truth value. We can also model it with the preimage of $f$ under $T$, i.e. the characteristic set of $f$. 
        \item They deal with \textbf{statements}, which are defined as declarative sentences having a truth value. 
      \end{enumerate} 
      Propositions are either true or false. 
    \end{definition}

    \begin{example}
      The proposition that \textit{the sky is blue} is represented as the function that returns $T$ for every possible world where the sky is blue. 
    \end{example}

    These declarative sentences are contrasted with questions, such as \textit{how are you doing?} and imperative statements such as \textit{please run my models}. Such non-declarative sentences have no truth value. 

    A statement can contain one or more other statements as parts. For example, compound sentences form simpler sentences. 

    \begin{definition}[Connectives]
      Statements are combined with \textbf{logical connectives}. 
      \begin{table}[H]
        \centering
        \begin{tabular}{|l|l|}
        \hline
        \textbf{Connective} & \textbf{Symbols} \\
        \hline
        AND & $A \wedge B$, $A \cdot B$, $AB$, $A \& B$, $A \&\& B$ \\
        \hline
        OR & $A \vee B$, $A + B$, $A \mid B$, $A \parallel B$ \\
        \hline
        NOT & $\neg A$, $-A$, $\overline{A}$, $\sim A$ \\
        \hline
        NAND & $\overline{A \wedge B}$, $A \mid B$, $\overline{A \cdot B}$ \\
        \hline
        NOR & $\overline{A \vee B}$, $A \downarrow B$, $\overline{A + B}$ \\
        \hline
        XOR & $A \veebar B$, $A \oplus B$ \\
        \hline
        XNOR & $A \odot B$ \\
        \hline
        IMPLIES & $A \Rightarrow B$, $A \supset B$, $A \rightarrow B$ \\
        \hline
        EQUIVALENT & $A \equiv B$, $A \Leftrightarrow B$, $A \leftrightarrow B$ \\
        \hline
        NONEQUIVALENT & $A \not\equiv B$, $A \not\Leftrightarrow B$, $A \not\leftrightarrow B$ \\
        \hline
        \end{tabular}
        \caption{Logical Connectives and Their Symbols}
        \label{tab:logical-connectives}
      \end{table} 
    \end{definition} 

    \begin{definition}[Propositional Formula]
      Propositions, represented by letters and denoted \textbf{propositional variables}, along with these symbols for connectives, combine to make a \textbf{propositional formula}. 
    \end{definition}

    Propositional logic is not concerned with the structures of propositions beyond the point where they cannot be decomposed any more by logical connectives. 

  \subsection{Arguments}

    At this point we may look at a set of propositions $P_1, \ldots, P_n$ and try come to a logical conclusion $Q$. This is called an argument. 

    \begin{definition}[Argument]
      Let $P$ be a set of propositions, called the \textbf{premises}. Let $Q$ be a proposition, called the \textbf{conclusion}. Then an \textbf{argument} is an attempt to deduce $Q$ from $P$. It is written in the forms 
      \begin{enumerate}
        \item If $P$, then $Q$.  
        \item $P \implies Q$
      \end{enumerate}
      An argument is \textbf{valid} if and only if  
      \begin{enumerate}
        \item It is necessary that if $P$ is true, $Q$ is true. 
        \item It is impossible for $P$ to be true, while $Q$ is false. 
      \end{enumerate}
    \end{definition}

    \begin{example}
      The following is an argument. 
      \begin{center} 
        If \textit{it is raining}, then \textit{it is cloudy}. 
      \end{center} 
    \end{example} 

    Logic in general aims to specify valid arguments. This is done by defining a valid argument as one in which its conclusion is a logical consequence of its premises. Determining whether a proposition is a a logical consequence of another proposition is the process of \textbf{deductive argument}, which has rules. These rules, called \textbf{rules of inference}, determines the ``legal moves'' from one or more premises to the conclusion. We give 2 familiar ones. 

    \begin{definition}[Modus Ponens]
      \textbf{Modus ponens} is a deductive argument form and rule of inference.\footnote{In some literature it is treated as an axiom, though most people think of it as a rule.} The argument states that given the premises
      \begin{enumerate}
        \item $P \implies Q$ 
        \item $P$
      \end{enumerate}
      Then our conclusion is $Q$. 
    \end{definition} 

    The next one is the familiar statement that a statement is equivalent to its contrapositive. 

    \begin{definition}[Modus Tollens]
      \textbf{Modus tollens} is a deductive argument form and a rule of inference. The argument states that given the premises 
      \begin{enumerate}
        \item $P \implies Q$ 
        \item $Q$ 
      \end{enumerate}
      Then our conclusion is not $P$. 
    \end{definition}

\section{First-Order Logic} 

  In propositional logic, we deal with simple declarative propositions. \textbf{First-order logic} extends this by covering predicates and quantification. Let's motivate them. 

  We can think of predicates as properties. If we say \textit{Socrates is a philosopher} and \textit{Plato is a philosopher}, in propositional logic both these statements, represented as $P$ and $Q$, as utterances that are either true or false, and they are completely independent from one another. However, we may want to view them as an application of a predicate \textit{ $\ast$ is a philosopher} on the entities \textit{Socrates} and \textit{Plato}. This motives the formalism of the domain of discourse and the predicate. 

  \begin{definition}[Domain of Discourse]
    Given an individual $x$, its \textbf{domain of discourse} is the set over which certain variables of interest in some formal treatment may range. 
  \end{definition}

  \begin{definition}[Predicate]
    A \textbf{predicate} $P$ is a symbol that represents a property or a relation of a certain individual $x$ in a domain of discourse. Using predicates, $P(x)$ can be viewed as a proposition about the individual $x$. 
  \end{definition} 

  Note that a predicate itself is not a proposition, since saying \textit{$\ast$ is a philosopher} doesn't have any truth or false meaning to it, akin to a sentence fragment. But it is a placeholder $P(\cdot)$ upon which if an individual $x$ is put in, it makes sense to ask whether $P(x)$ is true. 

  \begin{definition}[Formula]
    A \textbf{formula} is a string of propositions, connectives, predicates, and variables $\phi$ that turns into a proposition once all free variables have been instantiated. 
  \end{definition}

  With predicates alone, all we have really done is add notational convenience. However, if we want to state a proposition not just about $x$, but its domain of discourse, then we can use quantifiers. 

  \begin{definition}[Quantifier]
    A \textbf{quantifier} is an operator that specifies how many individuals in the domain of source satisfy a proposition. The two most used quantifiers are 
    \begin{enumerate}
      \item \textit{Universal Quantification}. $\forall$, which means \textit{for every}. 
      \item \textit{Existential Quantification}. $\exists$, which means \textit{there exists}. 
    \end{enumerate}
  \end{definition} 

  These quantifiers are additional symbols in our language $\mathcal{L}$. If we add the equality symbol, we get first-order logic with equality. 

  \begin{axiom}[Equality]
    \textbf{Equality} is a primitive logical symbol which is always interpreted as the real equality relation between members of the domain of discourse. These equality axioms are: 
    \begin{enumerate}
      \item \textit{Reflexivity}. For each variable $x$, $x = x$. 
      \item \textit{Substitution for Functions}. For all variables $x$ and $y$, and any function symbol $f$, 
        \begin{equation}
          x = y \implies f(x) = f(y)
        \end{equation}
      \item \textit{Substitution for Formulas}. For any variables $x$ and $y$, and any formula $\phi(z)$ with free variable $z$, then 
        \begin{equation}
          x = y \implies (\phi(x) \implies \phi(y))
        \end{equation}
    \end{enumerate}
    Symmetry and transitivity follow from the axioms above. 
  \end{axiom} 

  Ordinary first-order interpretations have a single domain of discourse over which all quantifiers range. \textbf{Many-sorted first-order logic}, or \textbf{typed first-order logic} allows variables to have different \textbf{sorts} or \textbf{types}, i.e. coming from different domains. 

\section{Second-Order Logic} 

  First order logic can quantify over individuals, but not over properties. That is, while we can state something like 

  \begin{center}
    \textit{There exists x such that x is a cube.}
  \end{center} 

  we cannot quantify over a predicate. That is, the statement 
  
  \begin{center}
    \textit{There exists a property $P$ such that a cube satisfies $P$.}
  \end{center}

  This statement does not make sense in first-order logic, but makes sense in second-order logic. 
  
\section{Naive Set Theory}

    Unlike axiomatic set theories, which are defined using formal logic, naive set theory was defined informally at the end of the 19th century by Cantor, in natural language (like English). It describes the aspects of mathematical sets using words (e.g. \textit{satisfying, such as, ...}) and suffices for the everyday use of set theory in modern mathematics. However, as we will see, this leads to paradoxes. 

    \begin{definition}[Set]
      A \textbf{set} is a well-defined collection of distinct objects, called \textbf{elements}. 
    \end{definition}

    This definition tells us \textit{what} a set is, but does not define \textit{how} sets can be formed, and what operations on sets will again produce a set. The term \textit{well-defined} cannot by itself guarantee the consistency and unambiguity of what exactly constitutes and what does not constitute a set, and therefore this is not a formal definition. Attempting to achieve this will be done in axiomatic set theory, like ZFC. 

    \begin{definition}[Membership]
      If $x$ is a member of $A$, we write $x \in A$. For any $x$, it must be the case that either $x \in A$ (exclusive or) $x \not\in A$. 
    \end{definition}
    
    \begin{definition}[Equality]
      Two sets $A$ and $B$ are defined to be equal, denoted as $A = B$, when they have precisely the same elements. That is, if $x \in A \iff x \in B$. This means that a set is completely determined by its elements, and the description is immaterial. 
    \end{definition}

    \begin{definition}[Empty Set]
      There exists an empty set, denoted $\emptyset$ or $\{\}$, which is a set with no members at all. Because a set is described by its elements, there can only be one empty set. 
    \end{definition}

    Now we show how to construct sets. 

    \begin{definition}[Set-Builder Notation]
      We can construct a set in two ways. 
      \begin{enumerate}
        \item We list its elements between curly braces. 
        \begin{enumerate}
          \item The set $\{1, 2\}$ denotes the set containing $1$ and $2$. By equality $\{1, 2\} = \{2, 1\}$. 
          \item Repetition/multiplicity is irrelevant, and so $\{1, 2, 2\} = \{1, 1, 1, 2\} = \{1, 2\}$ 
        \end{enumerate} 

        \item We denote 
        \begin{equation}
          S = \{ x | P(x) \}
        \end{equation} 
        where $P$ is a property. If $x$ satisfies this property, then $x \in S$. 
      \end{enumerate}
      Naive set theory claims that this construction \textit{always} produces a set. Therefore, a well-defined property is enough to always produce a set of elements satisfying $P$. 
    \end{definition} 

    \begin{example}[Empty Set]
      Let $S = \{x \mid x \neq x \}$. For any $x$, $P(x)$ is false and so $S$ contains no elements. Therefore $S = \emptyset$. 
    \end{example}

    \begin{example}[Singleton Set]
      The set $\{x \mid x = a \} = \{a\}$. 
    \end{example}

    \begin{example}[Russell Set]
      Let $R = \{x \mid x \not\in x\}$, i.e. the set of all sets that do not contain themselves as elements. 
    \end{example}

    \begin{theorem}[Russell's Paradox] 
      The Russell set exists and does not exist. 
    \end{theorem}
    \begin{proof}
      We will determine if $R$ is an element of itself. 
      \begin{enumerate}
        \item If $R \in R$, then by it does contain itself, so it does not satisfy the property and $R \not\in R$. 
        \item If $R \not\in R$, then it satisfies the property, so $R \in R$. 
      \end{enumerate}
      Therefore, it is both the case that $x \in R$ and $x \not\in R$, which contradicts the membership definition. Therefore, $R$ is both a set from set-builder construction and not a set due to the membership definition. 
    \end{proof}

    \begin{theorem}[Existence of Universe]
      Let $U$ be the set of everything, known as the \textbf{universal set}. The universal set does exist and does not exist. 
    \end{theorem}
    \begin{proof}
      We can define $U^\prime = \{x \mid \{\} = \{\} \}$, which defines a set. Then the property $P$ that $\{\} = \{\}$ is always true, and $U^\prime$ would contain everything, and by the definition of equality $U = U^\prime$. Now since the Russell set $R$ is both a set and not a set from Russell's paradox, we have $R \in U$ and $R \not\in U$, which means that $U$ cannot exist. Therefore $U$ does not exist. 
    \end{proof}

    So the sufficiency a well-defined property to be able to construct a set is \textit{too powerful} in that we can construct \textit{any} set we want. This leads us to construct the Russell set, which opens up a lot of paradoxes. Therefore, we would like to restrict the notion of well-defined in a way, which leads to axiomatic set theories. 

    \begin{definition}[Subsets]
      Given two sets $A$ and $B$, $A$ is a \textbf{subset} of $B$ if every element of $A$ is also an element of $B$. A subset of $B$ that is not equal to $B$ is called a \textbf{proper subset}. 
    \end{definition}

    \begin{theorem}[Equality]
      It follows from the definition of equality that 
      \begin{equation}
        A \subset B \text{ and } B \subset A \iff A = B
      \end{equation}
    \end{theorem}

    \begin{definition}[Power Set]
      The set of all subsets of a set $A$ is called the \textbf{power set} of $A$, denoted by $2^A$. 
    \end{definition}

    We could define other things like the union, etc., but I won't bother with it when I will define them for ZFC later.  

\section{Zermelo-Fraenkel-Choice (ZFC) Set Theory} 

    So with these paradoxes in mind, we would like to construct an axiomatic formulation of sets. My take is to think that sets ``exist'' out there somewhere in the universe, and our job is to find them. Cantor with his naive set theory believed that for every meaningful property of things there is a set whose members are exactly all the things with that property. Russell shows this this cannot be the case. Nevertheless, \textit{some} sets exist, and we have intuitive experience thinking about finite sets. Therefore, the axioms of set theory are a limited list of \textit{assumptions} that we hope are true about that actually existing universe of sets. As long as they are true, then whatever we conclude from them by valid reasoning steps must also be true.\footnote{This idea is called naive Platonism.} Hence we have the following definition, which first requires the familiar property of acting like a collection of something, and then obeys the axioms we set. 
    
    \begin{definition}[Set]
      A \textbf{set} $X$ is anything 
      \begin{enumerate}
        \item that has the innate property of containing elements, and 
        \item obeys the axioms of ZFC. 
      \end{enumerate}
    \end{definition}  

    Let's first talk about the language, where they are defined formally using the axioms in the next subsection. From first-order logic, note that we have the following symbols in our alphabet $\mathcal{L}_{\mathrm{ZFC}}$. 
    \begin{enumerate}
      \item The logical connectives $\neg$, $\lor$, $\land$. 
      \item The quantifier symbols $\exists, \forall$ 
      \item Brackets $()$. 
    \end{enumerate}
    To represent sets, we also need symbols, and the membership property requires us to define a symbol for that too. 
    \begin{enumerate}
      \item A countably infinite amount of variables used for representing sets. 
      \item The set membership symbol $\in$. In fact, when we say $x \in A$, this is a proposition formed from the predicate $P(x)$. 
    \end{enumerate} 
    This is what we have to work with so far. We will construct the rest of the symbols ($=, \subset, \supset, \cup, \cap$) from the axioms. So far we don't even know if there exists any set that obeys the following axioms! Therefore, we will assert the existence of at least one set, namely the empty set. 

  \subsection{Axioms}

    Now we state the axioms, which is the foundation of ZF set theory. 

    \begin{axiom}[Empty Set]
      The empty set containing no elements exists. 
    \end{axiom}

    \begin{definition}[Empty Set]
      The empty set is denoted $\emptyset$. 
    \end{definition}

    This asserts the existence of at least 1 set, which we will build on to create more sets. 

    \begin{axiom}[Axiom of Extensionality]
      Two sets are equal (are the same set) if they have the same elements. 
      \begin{equation}
        \forall A \forall B \big[ \forall x (x \in A \iff x \in B) \iff A = B\big]
      \end{equation}
    \end{axiom} 

    \begin{definition}[Equality]
      This axiom allows us to define the equality operator $=$, which we now add to our alphabet. 
    \end{definition}

    \begin{theorem}[Sets Don't Contain Repeated Elements]
      Furthermore, this axiom also implies that sets are unique up to distinct elements. That is, 
      \begin{equation}
        \{1, 1, 2\} = \{1, 2\} = \{1, 1, 2, 2\}
      \end{equation}
    \end{theorem}

    \begin{axiom}[Axiom of Regularity]
      Every non-empty set $A$ contains a member $x$ such that $A$ and $x$ are disjoint sets. 
      \begin{equation}
        \forall A \big[ A \neq \emptyset \implies \exists x (x \in A \land A \cap x = \emptyset) \big]
      \end{equation}
      This, along with the axioms of pairing and union, implies that no set is an element of itself and that every set has an ordinal rank. 
    \end{axiom}

    The axiom assists us in regulating which sets are viable and which are not, preventing Russell's paradox. 

    \begin{axiom}[Axiom Schema of Restricted Comprehension, or Specification]
      Subsets, like in naive set theory, are constructed using set builder notation. In general, the \textbf{subset} of a set $A$ obeying a formula $\phi(x)$ with one free variable $x$ may be written as 
      \begin{equation}
        \{x \in A \mid \phi(x) \}
      \end{equation}
      The axiom schema of specification states that this subset always exists.\footnote{Note that this axiom does not allow the construction of entities of the more general form $\{x \mid \phi(x)\}$. This restriction is obviously needed to avoid Russell's paradox, hence the name \textit{restricted} comprehension. } 
    \end{axiom}  

    \begin{definition}[Subset, Superset]
      The axiom of specification allows us to denote subsets. Notationally, if $A$ is a subset of $B$, then we write $A \subset B$. Similarly, we say $A$ is a \textbf{superset} of $B$, written $A \supset B$, if $B \subset A$. 
    \end{definition} 
    
    \begin{definition}[Intersection]
      This also allows us to define intersection as 
      \begin{equation}
        A \cap B \coloneqq \{x \in A \mid x \in B \}
      \end{equation} 
      and we can define the intersection of an arbitrary collection of sets $\mathcal{F}$ as the following. Let $A \in \mathcal{F}$.  
      \begin{equation}
        \bigcap \mathcal{F} \coloneqq \{x \in A \mid \forall B (B \in \mathcal{F} \implies x \in B) \}
      \end{equation}
    \end{definition}

    Unfortunately, the union cannot be expressed in this specification schema, and we need a separate axiom for this. 

    \begin{axiom}[Axiom of Pairing]
      If $A, B$ are sets, then there exists a set which contains $A$ and $B$ as elements.\footnote{For example, if $A = \{1, 2\}$ and $B = \{2, 3\}$,then $\{\{1, 2\}, \{2, 3\}\}$ exists.}
      \begin{equation}
        \forall A \forall B \exists C((A \in C) \land (B \in C))
      \end{equation}
      This allows us to construct sets from old ones. 
    \end{axiom}

    \begin{theorem}[Nested Sets]
      By the axiom of pairing, if we have a set $X$, then $\{X\}$ is also a set, since we can set $A = B = X$ which asserts the existence of $\{X, X\} = \{X\}$. 
    \end{theorem}

    \begin{axiom}[Axiom of Union]
      For any set of sets $\mathcal{F}$, there is a set $A$ containing every element that is a member of $\mathcal{F}$.
      \begin{equation}
        \forall \mathcal{F} \exists A \forall X \forall x \big[ (x \in X \land X \in \mathcal{F}) \implies x \in A \big]
      \end{equation}
    \end{axiom}

    This formula doesn't directly assert the existence of $\cup \mathcal{F}$ (?). 

    \begin{definition}[Union]
      The set $\cup \mathcal{F}$ can be constructed from $A$ in the above using the axiom schema of restricted comprehension. 
      \begin{equation}
        \cup \mathcal{F} = \{ x \in A \mid \exists X (x \in X \land X \in \mathcal{F} ) \}
      \end{equation}
    \end{definition}

    \begin{axiom}[Axiom of Infinity]
      The axiom of infinity guarantees the existence of at least one infinite set. That is, given a set $w$, let $S(w) = w \cup \{w\}$ be a set.\footnote{Since $w$ is a set, by the axiom of pairing $\{w\}$ is a set, and by the axiom of union $w \cup \{w\}$ is a set.} Then, there exists a set $X$ such that 
      \begin{enumerate}
        \item $\emptyset \in X$, and 
        \item if $w \in X$, then $S(w) \in X$. 
      \end{enumerate} 
      In logic terms, 
      \begin{equation}
        \exists X \big[ \emptyset \in X \land \forall y (y \in X \implies S(y) \in X) \big]
      \end{equation}
      Since we have axiomatically claimed the two premises to be true, by propositional logic, namely \textit{modus ponens}, this implies the existence of at least one set $X$ with infinitely many members. 
    \end{axiom}

    \begin{definition}[Von Neumann Ordinals] 
       The \textbf{Von Neumann ordinals} is the minimal set $X$ satisfying the axiom of infinity. It is the set containing 
      \begin{align*}
        0 & = \{\} = \emptyset \\
        1 & = \{0\} = \{\emptyset\} \\
        2 & = \{0,1\} = \{\emptyset,\{\emptyset\}\} \\
        3 & = \{0,1,2\} = \{\emptyset,\{\emptyset\},\{\emptyset,\{\emptyset\}\}\} \\
        4 & = \{0,1,2,3\} = \{\emptyset,\{\emptyset\},\{\emptyset,\{\emptyset\}\},\{\emptyset,\{\emptyset\},\{\emptyset,\{\emptyset\}\}\}\} \\
        \ldots & = \ldots 
      \end{align*} 
      This provides the foundation to construct the most basic mathematical sets: the natural numbers denoted $\mathbb{N}$.  
    \end{definition}

    \begin{axiom}[Axiom of Power Set]
      The axiom of power set states that for any set $A$, there is a set $B$ that contains every subset\footnote{Note that subset is defined by the axiom of restricted comprehension.} of $A$. 
      \begin{equation}
        \forall A \exists B \forall S (S \subset A \implies S \in B)
      \end{equation}
      The axiom of schema of specification is then used to define the power set as the subset of such $B$ containing the subset of $A$ exactly. 
      \begin{equation}
        2^X = \{Y \in B \mid Y \subset X \}
      \end{equation}
    \end{axiom} 

    \begin{definition}[Cartesian Product]
      The power set axiom allows for the definition of the \textbf{Cartesian product} of two sets $X$ and $Y$. Note that if $x \in X, y \in Y$, then by the axiom of union $x, y \in X \cup Y$ and by the axiom of power set $\{x\}, \{x, y\} \in \mathcal{P}(X \cup Y)$. Therefore, using the axiom of power set again we can define
      \begin{equation}
        (x, y) \coloneqq \{\{x\}, \{x, y\}\} \mathcal{P}(\mathcal{P}(X \cup Y))
      \end{equation} 
      and the Cartesian product is defined 
      \begin{equation}
        X \times Y \coloneqq = \{ (x, y) \in \mathcal{P}(\mathcal{P}(X \cup Y))  \mid x \in X \land y \in Y \}
      \end{equation}
      which is axiomatically a valid set by the axiom schema of specification. From this we can define the Cartesian product of any finite collection of sets recursively. 
    \end{definition}

    The definition of Cartesian products allows us to formally define \textbf{correspondences}. The most notable correspondences are \textit{functions} and \textit{relations}. 

    \begin{definition}[Function]
      Given two sets $X, Y$, a function is a 
    \end{definition}

    \begin{definition}[Relation]
      
    \end{definition}

    \begin{axiom}[Axiom Schema of Replacement]
      This axiom asserts that the image of a set under any definable function will fall inside a set. 
    \end{axiom} 

    Again, how do we even know for sure that these axioms aren't contradictory? The answer is that we don't, and that is why we take them as axioms rather than provable theorems. Fortunately, from the formulation in the early 20th century up until now, no contradictions have been found, and if there is one, then it would be very bad news for us.  

  \subsection{Axiom of Choice}

    The axioms up to this point are pretty much undisputed and completes ZF set theory. The next one, though controversial, is required in the proofs of some notable theorems. If we include this axiom of choice, then we have ZFC set theory. The axiom of choice has many equivalent definitions. Informally, note that we have defined the Cartesian product for a finite family of sets. Consequently, functions and relations are also defined for a finite collection of elements from each set. Now we try to extend this to an arbitrary (countably or uncountably infinite, though we haven't defined these terms yet) collection of sets. 

    Colloquially, the axiom of choice says that a Cartesian product of a collection\footnote{Note that this does not have to be finite} of non-empty sets is non-empty. That is, it is possible to construct a new set by choosing one element from each set, even if the collection is infinite. 

    \begin{axiom}[Axiom of Choice]
      For every indexed family $X = \{S_i\}_{i \in I}$ of nonempty sets, there exists an indexed set $\{x_i\}_{i \in I}$ such that $x_i \in S_i$ for every $i \in I$. 
    \end{axiom}

    \begin{definition}[Choice Function]
      This mapping $f: X \rightarrow \cup_{i \in I} S_i$ that maps $S_i \mapsto x_i \in S_i$ is called a \textbf{choice function}. Despite the name, it is not a function according to our definition if $I$ is not finite, and so we must axiomatically construct this. 
    \end{definition}

    The existence of a choice function when $X$ is finite is easily proved from the ZF axioms, and AC only matters for certain infinite sets. It is understandable how this is controversial, since we don't really work with functions over infinite Cartesian products. It is characterized as nonconstructive because it asserts the existence of a choice function but says nothing about how to construct one, unlike the axiom of infinity.  

    This choice function was used in the proof of the following, which turns out to be equivalent. 

    \begin{axiom}[Axiom of Well-Ordering]
      For any set $X$, there exists a binary relation $R$ which \textit{well-orders} $X$, i.e. is a total order and has the property that every nonempty subset of $X$ has a least element under the order $R$. 
      \begin{equation}
        \forall X \exists R (R \text{ well-orders } X)
      \end{equation}
    \end{axiom}

    We can see generally that we would like to use a choice function to select a representative element of each set in $X$. Then we can use these to construct an order. Finally, we state the last form of the axiom of choice. 

    \begin{axiom}[Zorn's Lemma]
      Let $X$ be a partially ordered set that satisfies the two properties. 
      \begin{enumerate}
        \item $P$ is nonempty. 
        \item Every \textit{chain} (a subset $A \subset P$ where $A$ is totally ordered) has an upper bound in $P$. 
      \end{enumerate}
      Then $P$ has at least one maximal element. 
    \end{axiom}

    Zorn's lemma is required to show that every vector space has a basis. 

\section{Natural Numbers and Induction}

  \begin{definition}[Inductive Set, Natural Numbers]
    A set $X \subset \mathbb{R}$ is inductive if for each number $x \in X$, it also contains $x + 1$. The set of \textit{natural numbers}j, denoted $\mathbb{N}$, is the smallest inductive set containing $1$. 
  \end{definition}

  We can use this inductive property of natural numbers to prove properties of them. Note that this can only be used to prove for finite (yet unbounded) numbers! 

  \begin{lemma}[Induction Principle]
    Given $P(n)$, a property depending on positive integer $n$, 
    \begin{enumerate}
      \item if $P(n_0)$ is true for some positive integer $n_0$, and
      \item if for every $k \geq n_0$, $P(k)$ true implies $P(k+1)$ true, 
    \end{enumerate}
    then $P(n)$ is true for all $n \geq n_0$. 
  \end{lemma}

  \begin{lemma}[Strong Induction Principle]
    Given $P(n)$, a property depending on a positive integer $n$, 
    \begin{enumerate}
      \item if $P(n_0), P(n_0 + 1), \ldots, P(n_0 + m)$ are true for some positive integer $n_0$, and nonnegative integer $m$, and 
      \item if for every $k > n_0 + m, P(j)$ is true for all $n_0 \leq j \leq k$ implies $P(k)$ is true, 
    \end{enumerate}
    then $P(n)$ is true for all $n \geq n_0$. 
  \end{lemma}

  The idea behind the strong induction principle leads to the proof using infinite descent. Infinite descent combines strong induction with the fact that every subset of the positive integers has a smallest element, i.e. there is no strictly decreasing infinite sequence of positive integers. 

  \begin{lemma}[Infinite Descent]
    Given $P(n)$, a property depending on positive integer, assume that $P(n)$ is false for a set of integers $\mathcal{S}$. Let the smallest element of $\mathcal{S}$ be $n_0$. If $P(n_0)$ false implies $P(k)$ false, where $k < n_0$, then by contradiction $P(n)$ is true for all $n$. 
  \end{lemma}

\section{Cardinality}

  \begin{definition}[Equipotence]
    Two sets $A$ and $B$ are \textbf{equipotent}, written $A \approx B$, if there exists a bijective map $f: A \rightarrow B$. This implies that their cardinalities are the same: $|A| = |B|$. It has the following properties: 
    \begin{enumerate}
      \item Reflexive: $A \approx A$
      \item Symmetric: $A \approx B$ implies $B \approx A$
      \item Transitive: $A \approx B$ and $B \approx C$ implies $A \approx C$
    \end{enumerate}
  \end{definition}

  \begin{definition}
    For any positive integer $n$, let $J_n$ be the set whose elements are the integers $1, 2, \ldots, n$. For any set $A$, we define 
    \begin{enumerate}
      \item $A$ is \textbf{finite} if $A \approx J_n$ for some $n$. The empty set is also considered to be finite. 
      \item $A$ is \textbf{infinite} if it is not finite. 
      \item $A$ is countable if $A \approx \mathbb{N}$. 
      \item $A$ is uncountable if $A$ is neither finite nor countable. 
      \item $A$ is at most countable if $A$ is finite or countable. 
    \end{enumerate}
  \end{definition}

  At this point, we may already be familiar with the fact that $\mathbb{Q}$ is countable and $\mathbb{R}$ is uncountable. Let us formalize the statement that a countable infinity is the smallest type of infinity. We can show this by taking a countable set and showing that every infinite subset must be countable. If it was uncountable, then this would mean that a countable set contains an uncountable set. 

  \begin{theorem}
    \label{countable smallest}
    Every infinite subset of a countable set $A$ is countable. 
  \end{theorem}

  \begin{theorem}
    An at most countable union of countable sets is countable. 
  \end{theorem}

  \begin{theorem}
    A finite Cartesian product of countable sets is countable. 
  \end{theorem}

  \begin{corollary}
    $\mathbb{Q}$ is countable. 
  \end{corollary}

  Now, how do we prove that a set is uncountable? We can't really use the contrapositive of Theorem $\ref{countable smallest}$, since to prove that an arbitrary set $A$ is uncountable, then we must find an infinite subset that is not countable. But now we must prove that this subset itself is not countable, too! Therefore, we can use this theorem. 

  \begin{theorem}
    Given an arbitrary set $A$, if every countable subset $B$ is a proper subset of $A$, then $A$ is uncountable. 
  \end{theorem}
  \begin{proof}
    Assume that $A$ is countable. Then $A$ itself is a countable subset of $A$, but by the assumption, $A$ should be a proper subset of $A$, which is absurd. Therefore, $A$ is uncountable. 
  \end{proof}

  \begin{theorem}
    Let $A$ be the set of all sequences whose elements are the digits $0$ and $1$. Then, $A$ is uncountable. 
  \end{theorem}

\end{document}

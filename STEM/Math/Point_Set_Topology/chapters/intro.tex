Given a set, what are some fundamental structures that you can put on a set? We can first talk about relations, such as ordering, or functions, such as a norm or a distance. These are constructed as subsets of a Cartesian product of finite $X$'s. Another structure on set $X$ is to define a set of subsets $\T \subset 2^X$ that allow us to interpret how certain elements of a set are ``nearby'' each other without the notion of a metric.\footnote{In ZFC set theory, a topology may be more fundamental in the sense that it is a subset of the power set, while the other structures are subsets of a Cartesian product, which itself is a construction from the power set.} This set of subsets is called a \textit{topology}, with its elements being \textit{open sets}. So how do you define such a thing? Well intuitively, given two elements $x, y \in X$, if there exists two disjoint open sets $U_1, U_2$ such that $x \in U_1$ and $y \in U_2$, then we can \textit{distinguish} these points in such a way. If this is true for all points in $X$, then this gives us a nice \textit{Hausdorff} property to work with. If there exists no open sets that can do this, then $x$ and $y$, although distinct in $X$, may be \textit{indistinguishable} in the topological sense. 

If this notion of nearness can be rigorously defined, we may be able to characterize the elements and subsets of $X$. One nice notion is the concept of \textit{limit points} which asks whether $x$ is ``infinitesimally close'' to a certain set. This allows us to define limits without the notion of a metric, and with this foundation we build the notion of continuity.    

A trivial way to construct such a topology is to take the power set $2^X$ itself. However, this may be ``too big'' in a sense that no interesting properties can be deduced. But this doesn't mean we can take any subset of $2^X$. We compromise by defining topologies to be a subset of $2^X$ with certain properties, which we will mention in the next section. 

The construction of the topology allows us to study properties of these spaces. Moreover, if we have a function that maps from one topological space to another, how do we know what kinds of properties will be preserved and what will be lost? It turns out that these topological properties are invariant under certain mappings called \textit{homeomorphisms}. Therefore, topology can also be seen as a method to study spaces and properties that are preserved under homeomorphisms. 

\begin{example}[idk where to put this for now]
  There is an infinite family of 2-dimensional manifolds, call them $M$ and $N$, and each set in each family is not homeomorphic to another.  
  \begin{enumerate}
    \item $M_0 = S^2$ (sphere). $M_1 = T^2$ (torus). $M_2$ is a donut with two holes. $M_3$ has three holes, and so on. 
    \item $N_1$ is the Mobius strip. $N_2$ is the Klein bottle. 
  \end{enumerate}
\end{example}


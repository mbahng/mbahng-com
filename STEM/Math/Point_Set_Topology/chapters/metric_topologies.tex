\section{Metric Topologies}

  In $\mathbb{R}$, note that every open ball is really just an interval. In fact, every open ball $(x - r, x + r)$ can be expressed with just two elements $a, b \in \mathbb{R}$, as $(a, b)$. Notice that this method of expressing an open set does not even require any metric! Extending this to $\mathbb{R}^n$ would indicate that the topologies of $\mathbb{R}^n$ defined by the endpoint of the open intervals would not necessarily induce any metric either. Notice that these induced topologies is \textbf{not} the open ball topology, which must have an associated metric to it. Rather, this induced, non-metric topology is the box topology! While the box topology and the open ball topology are really the same topology, they are generated by inherently different bases. 

  \begin{definition}[Bounded Set]
    Let $(X, d)$ be a metric space with subset $A$. $A$ is \textbf{bounded} if there exists some number $M$ such that
    \begin{equation}
      d (x, y) \leq M \text{ for all } x,y \in A
    \end{equation}
    If $A$ is bounded, the \textbf{diameter} of $A$ is defined to be the number
    \begin{equation}
      \text{diam}\, A \equiv \sup{\{d(x, y) \mid x, y \in A\}}
    \end{equation}
  \end{definition}

  Note that boundedness on a set is not a topological property since it depends on the particular metric $d$ that is used for $X$. For example, we can construct the following metric that makes every subset in $X$ bounded. 

  \begin{definition}[Standard Bounded Metric]
    Let $(X, d)$ be a metric space. We define a second metric $\Tilde{d}$ on $X$ such that
    \begin{equation}
      \Tilde{d} (x, y) \equiv \min{\{d(x, y), 1\}}
    \end{equation}
    $\Tilde{d}$ is called the \textbf{standard bounded metric corresponding to $d$}. 
  \end{definition}

  If we construct open balls with this metric, it is easy to see that they consist of all open balls with radius less than or equal to 1. That is, the topology $\T$ consists of all open balls
  \begin{equation}
    \T \equiv \{B_r (x) \mid x \in X, r \leq 1\}
  \end{equation}
  It is also clear that the topology induced by $\Tilde{d}$ is the same as the topology induced by $d$! The significance of this construction of the standard bounded metric is that we can now work with a basis consisting of bounded elements, which is much nicer than a basis of open balls that can have arbitrarily large radii.  

  We now introduce a metrization theorem on $\mathbb{R}^n$. 

  \begin{theorem}
    The topologies on $\mathbb{R}^n$ induced by the Euclidean metric $d$ and the square metric $\rho$ are the same as the product topology on $\mathbb{R}^n$. 
  \end{theorem}
  \begin{proof}
    Given $x, y \in \mathbb{R}^n$, simple algebra shows that 
    \begin{align*}
      & \rho(x, y) \leq d(x, y) \leq \sqrt{n} \rho(x, y) \\
      & \;\;\;\; \implies \forall x, \epsilon, \; B_d (x, \epsilon) \subset B_\rho (x, \epsilon) \text{ and } B_\rho (x, \frac{\epsilon}{\sqrt{n}}) \subset B_d (x, \epsilon)
    \end{align*}
    But
    \begin{equation}
      \{ B_\rho (x, \epsilon) \mid x \in \mathbb{R}^n, \epsilon \in \mathbb{R}\} = B_\rho (x, \frac{\epsilon}{\sqrt{n}}) \mid x \in \mathbb{R}^n, \epsilon \in \mathbb{R}\}
    \end{equation}
    which means that the metric topology induced by $d$ is the same as the metric topology induced by $\rho \implies$ the two topologies are the same. We know that the topology induced by $\rho$ is the same as the product topology since 
    \begin{equation}
      \prod_{i=1}^n (x_i - r, x_i + r) = \bigcup_{k=1}^n \mathbb{R}^{k-1} \times (x_k - r, x_k + r) \times \mathbb{R}^{n-k}
    \end{equation}
  \end{proof}

  With this theorem, we have proved that given a topological space $\mathbb{R}^n$ with the product topology, there exists a metric (the Euclidean and square metric) that induces this product topology. We can attempt to extrapolate these formulas to $\mathbb{R}^\omega$ by defining
  \begin{align*}
    & d(x, y) \equiv \bigg(\sum_{i=1}^\infty (x_i - y_i)^2 \bigg)^{\frac{1}{2}} \\
    & \rho(x, y) \equiv \sup{\{|x_i - y_i|\}}
  \end{align*}
  However, the metrics do not in general map to elements of $\mathbb{R}$, since the sequence $(x_i - y_i)_{i \in \mathbb{N}}$ could diverge. Therefore, we can redefine the metric $\rho$ to the following bounded one. 
  \begin{equation}
    \Tilde{\rho} (x, y) \equiv \sup{\{\Tilde{d}(x_i, y_i)\}}
  \end{equation}
  where $\Tilde{d}$ is the standard bounded metric on $\mathbb{R}$. Clearly,
  \begin{equation}
    0 \leq \Tilde{\rho}(x, y) \leq 1
  \end{equation}
  $\Tilde{\rho}$ is indeed a metric on $\mathbb{R}^\omega$, but unfortunately, it does not induce the product topology. We extend this definition to arbitrary $\mathbb{R}^J$. 
  
  \begin{definition}[Uniform Metric]
    Given an indexed set $J$ with points $x, y \in \mathbb{R}^J$, we define
    \begin{equation}
      \Tilde{\rho} \equiv \sup{\{\Tilde{d}(x_\alpha, y_\alpha)\;|\; \alpha \in J\}}
    \end{equation}
    with $\Tilde{d}$ the standard bounded metric on $\mathbb{R}$. $\Tilde{\rho}$ is called the \textbf{uniform metric on $\mathbb{R}^J$}, which induces the \textbf{uniform topology}. 
  \end{definition}

  The uniform topology on $\mathbb{R}^J$ is finer than the product topology, and they are different if $J$ is infinite. Clearly, $0 \leq \Tilde{p} (x, y) \leq 1$, meaning that given the open ball
  \begin{equation}
    B_r (x) \equiv \{y \in \mathbb{R}^J \;|\; \Tilde{p}(y, x) < r\}
  \end{equation}
  if $r \geq 1$, then $B_r (x) = \mathbb{R}^J$ and if $r<1$, then $B_r (x)$ consists of the $n$-dimensional box with "radius" $r$, where $n = \dim{\mathbb{R}^J}$. 


  The next theorem gives us a metric that induces the product topology on infinite dimensional $\mathbb{R}^\omega$ by slightly modifying the uniform metric on $\mathbb{R}$. However, with the box topology $\mathbb{R}^\omega$ is not metrizable. 

  \begin{theorem}
    Let $\Tilde{d} (a, b) \equiv \min{\{|a-b|, 1\}}$ be the standard bounded metric on $\mathbb{R}$. If $x, y \in \mathbb{R}^\omega$, we define
    \begin{equation}
      D(x, y) \equiv \sup{\Big\{ \frac{\Tilde{d}(x_i, y_i)}{i}\Big\}}
    \end{equation}
    Then, $D$ is a metric that induces the product topology on $\mathbb{R}^\omega$. 
  \end{theorem}
  
  It is easy to see that $0 \leq D(x, y) \leq 1$. So, given the open ball
  \begin{equation}
    B_r (x) \equiv \{y \in \mathbb{R}^\omega \; | \; D(x, y) < r\}
  \end{equation}
  $B_r (x) = \mathbb{R}^\omega$ if $r > 1$. When $r \leq 1$, 
  \begin{equation}
    B_r (x) \equiv (y-r, y+r) \times (y-2r, y+2r) \times ... = \prod_{k=1}^\infty (y - k r , y + k r)
  \end{equation}

  \begin{figure}[H]
    \centering 
    \begin{tikzpicture}
      \node[below left] at (0,0) {$y$};
      \draw[fill] (0,0) circle (0.05);
      \draw[<->] (0, -2)--(0,2);
      \draw[<->] (-3,0)--(3,0);
      \draw[fill] (-2,0) circle (0.05); 
      \draw[fill] (2,0) circle (0.05); 
      \draw[fill] (0,1) circle (0.05); 
      \draw[fill] (0,-1) circle (0.05); 
      \node[above left] at (-2,0) {$-2$};
      \node[above right] at (2,0) {$2$};
      \node[above right] at (0,1) {$1$};
      \node[below right] at (0,-1) {$-1$};
      \draw[fill] (0, 0.7) circle (0.05);
      \draw[fill] (1.4, 0) circle (0.05);
      \node[above left] at (0,0.7) {$r$};
      \node[below right] at (1.4,0) {$2r$};
      \draw[dashed] (-1.4, -0.7) rectangle (1.4,0.7);
      \node[right] at (0,2) {$\mathbb{R}_1$};
      \node[above] at (3,0) {$\mathbb{R}_2$};
    \end{tikzpicture}
    \caption{Visually, we take a cross section of this box and look at the slice within $\mathbb{R}_1 \times \mathbb{R}_2$, where the subscripts represent the first and second terms of $x$.}
    \label{fig:cross_sec}
  \end{figure}

  We can extend the applications of the Bolzano Weierstrass Lemma from analysis to metric spaces in general with the following lemma. 

  \begin{lemma}[Sequence Lemma]
    If $X$ be a topological space with $A \subset X$. If there exists a sequence of points of $A$ that converges to $x$, then $x \in \bar{A}$. The converse is true if $X$ is metrizable. 
  \end{lemma}
  \begin{proof}
    $(\rightarrow)$ Our hypothesis says that $x$ is a limit point of $A$, which by definition means that $x \in \bar{A}$. \\
    $(\leftarrow)$ Assuming $X$ is metrizable and $x \in \bar{A}$, let $d$ be a metric for the topology of $X$. Then, for every $n \in \mathbb{N}$, let us define a sequence of open neighborhoods of $x$ to be
    \begin{equation}
      \big(B_{\frac{1}{n}} (x) \big)
    \end{equation}
    Since $x \in \bar{A}$, there exists a point 
    \begin{equation}
      x_n \in A \cap B_{\frac{1}{n}} (x) \text{ for all } n \in \mathbb{N}
    \end{equation}
    This sequence $(x_n)$ that we have proved must exist converges to $x$. 
  \end{proof}

  \begin{theorem}
    Let $f: X \longrightarrow Y$ and let $X$ be metrizable. $f$ is continuous if and only if for every convergent sequence $(x_n) \rightarrow x$ of $X$, the following sequence of $Y$ converges to $f(x)$. That is, 
    \begin{equation}
      \big( f(x_n) \big) \longrightarrow f(x)
    \end{equation}
  \end{theorem}

  We introduce additional methods of constructing continuous functions. 

  \begin{definition}[Uniform Convergence]
    Let $f_n: X \longrightarrow Y$ be a sequence of functions from the set $X$ to the metric space $(Y, d)$. The sequence $(f_n)$ is said to \textbf{converge uniformly} to the function $f: X \longrightarrow Y$ if, given $\epsilon > 0$, there exists a $N \in \mathbb{N}$ such that
    \begin{equation}
      d\big( f_n(x), f(x)\big) < \epsilon
    \end{equation}
    for all $n \geq N$ and for all $x \in X$. 
  \end{definition}

  \begin{theorem}[Uniform Limit Theorem]
    Let $f_n: X \longrightarrow Y$ be a sequence of continuous functions from topological space $X$ to a metric space $Y$. If $f_n$ converges uniformly to $f$, then $f$ is continuous. 
  \end{theorem}
  \begin{proof}
    $(\rightarrow)$ Trivial. \\
    $(\leftarrow)$ Let $V$ be open in $Y$, and let $x_0$ be a point in $f^{-1} (V)$. It suffices to prove that for every $x_0 \in f^{-1} (V)$, there exists a neighborhood $U$ of $x_0$ such that $U \subset F^{-1} (V)$ or equivalently, $F(U) \subset V$. 

    Let $y_0 = f(x_0)$. Since $Y$ is a metric space with metric $d$, we know that there exists an $\epsilon$-ball $B_\epsilon (y_0)$ such that
    \begin{equation}
      B_\epsilon (y_0) \subset V
    \end{equation}
    Then, using uniform convergence, we can choose $N \in \mathbb{N}$ such that for all $n \geq N$ and all $x \in X$, 
    \begin{equation}
      d \big( f_n (x), f(x) \big) < \frac{\epsilon}{4}
    \end{equation}
    which also applies at the point $x = x_0$. 
    \begin{equation}
      d \big( f_n (x_0), f(x_0) \big) < \frac{\epsilon}{4}
    \end{equation}
    Using continuity of $f_n$, choose a neighborhood $U$ of $x_0$ such that $f_n$ carries $U$ into the open $\epsilon/2$-ball centered at $f_n (x_0)$ (note that $f_n (x_0) \neq y_0$), meaning that if $x \in U$
    \begin{equation}
      d \big( f_n (x), f_n (x_0) \big) < \frac{\epsilon}{2}
    \end{equation}
    Adding the three inequalities and using the triangle inequality, we get the fact that if $x \in U$, then 
    \begin{equation}
      d \big( f(x), f(x_0) \big) < \epsilon
    \end{equation}
    meaning that the $f(U) \subset B_\epsilon (x_0) \subset V$. 

    \begin{figure}[H]
      \centering 
      \begin{tikzpicture}[scale=0.6]
        \draw[dashed, teal] (0,0) circle [radius=8];
        \draw[fill] (0,0) circle [radius=0.05];
        \draw[fill] (-1.5,0.5) circle [radius=0.05];
        \draw[dashed, purple] (0,0) circle [radius=2];
        \draw[dashed, purple] (-1.5,0.5) circle [radius=2];
        \node[right] at (0.3,0) {$f(x_0)$};
        \node[above right] at (-1.5,0.5) {$f_N (x_0)$};
        \draw[dashed, blue] (-1.5,0.5) circle [radius=4];
        \draw[fill] (-4,3) circle [radius=0.05];
        \draw[dashed, purple] (-4,3) circle [radius=2];
        \node[right] at (-4,3) {$f_N (x)$};
        \draw[fill] (-5.5, 3.7) circle [radius=0.05];
        \node[above left] at (-5,4.5) {$f(x)$};
        \draw[-, thick] (-5.5, 3.7)--(-4,3)--(-1.5,0.5)--(0,0);
        \draw[->, thick, purple] (0,0)--(0,-2);
        \draw[->, thick, blue] (-1.5,0.5)--(-1.5,-3.5);
        \node[right, purple] at (0,-1) {$\epsilon / 4$};
        \node[left,blue] at (-1.5,-1.5) {$\epsilon / 2$};
        \draw[->, thick, teal] (0,0)--(6.128, 5.1416);
        \node[below right, teal] at (3.064, 2.57) {$\epsilon$};
      \end{tikzpicture}
      \caption{Visually, the three inequalities represent the following open balls in $V \subset Y$.} 
      \label{fig:uniform_limit}
    \end{figure}
  \end{proof}


  \begin{theorem}
    In a metric space $(X, d)$, a set is \textbf{closed} if the limit of every convergent subsequence in $X$ lies in $X$. That is, $X$ contains all of its limit points. 
  \end{theorem}


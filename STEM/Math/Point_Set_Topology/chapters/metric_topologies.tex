\section{Metric Topologies}

  In $\mathbb{R}$, note that every open ball is really just an interval. In fact, every open ball $(x - r, x + r)$ can be expressed with just two elements $a, b \in \mathbb{R}$, as $(a, b)$. Notice that this method of expressing an open set does not even require any metric! Extending this to $\mathbb{R}^n$ would indicate that the topologies of $\mathbb{R}^n$ defined by the endpoint of the open intervals would not necessarily induce any metric either. Notice that these induced topologies is \textbf{not} the open ball topology, which must have an associated metric to it. Rather, this induced, non-metric topology is the box topology! While the box topology and the open ball topology are really the same topology, they are generated by inherently different bases. 

  \begin{definition}[Bounded Set]
    Let $(X, d)$ be a metric space with subset $A$. $A$ is \textbf{bounded} if there exists some number $M$ such that
    \begin{equation}
      d (x, y) \leq M \text{ for all } x,y \in A
    \end{equation}
    If $A$ is bounded, the \textbf{diameter} of $A$ is defined to be the number
    \begin{equation}
      \text{diam}\, A \equiv \sup{\{d(x, y) \mid x, y \in A\}}
    \end{equation}
  \end{definition}

  Note that boundedness on a set is not a topological property since it depends on the particular metric $d$ that is used for $X$. For example, we can construct the following metric that makes every subset in $X$ bounded. 

  \begin{definition}[Standard Bounded Metric]
    Let $(X, d)$ be a metric space. We define a second metric $\Tilde{d}$ on $X$ such that
    \begin{equation}
      \Tilde{d} (x, y) \equiv \min{\{d(x, y), 1\}}
    \end{equation}
    $\Tilde{d}$ is called the \textbf{standard bounded metric corresponding to $d$}. 
  \end{definition}

  If we construct open balls with this metric, it is easy to see that they consist of all open balls with radius less than or equal to 1. That is, the topology $\T$ consists of all open balls
  \begin{equation}
    \T \equiv \{B_r (x) \mid x \in X, r \leq 1\}
  \end{equation}
  It is also clear that the topology induced by $\Tilde{d}$ is the same as the topology induced by $d$! The significance of this construction of the standard bounded metric is that we can now work with a basis consisting of bounded elements, which is much nicer than a basis of open balls that can have arbitrarily large radii.  

  We now introduce a metrization theorem on $\mathbb{R}^n$. 

  \begin{theorem}
    The topologies on $\mathbb{R}^n$ induced by the Euclidean metric $d$ and the square metric $\rho$ are the same as the product topology on $\mathbb{R}^n$. 
  \end{theorem}
  \begin{proof}
    Given $x, y \in \mathbb{R}^n$, simple algebra shows that 
    \begin{align*}
      & \rho(x, y) \leq d(x, y) \leq \sqrt{n} \rho(x, y) \\
      & \;\;\;\; \implies \forall x, \epsilon, \; B_d (x, \epsilon) \subset B_\rho (x, \epsilon) \text{ and } B_\rho (x, \frac{\epsilon}{\sqrt{n}}) \subset B_d (x, \epsilon)
    \end{align*}
    But
    \begin{equation}
      \{ B_\rho (x, \epsilon) \mid x \in \mathbb{R}^n, \epsilon \in \mathbb{R}\} = B_\rho (x, \frac{\epsilon}{\sqrt{n}}) \mid x \in \mathbb{R}^n, \epsilon \in \mathbb{R}\}
    \end{equation}
    which means that the metric topology induced by $d$ is the same as the metric topology induced by $\rho \implies$ the two topologies are the same. We know that the topology induced by $\rho$ is the same as the product topology since 
    \begin{equation}
      \prod_{i=1}^n (x_i - r, x_i + r) = \bigcup_{k=1}^n \mathbb{R}^{k-1} \times (x_k - r, x_k + r) \times \mathbb{R}^{n-k}
    \end{equation}
  \end{proof}

  With this theorem, we have proved that given a topological space $\mathbb{R}^n$ with the product topology, there exists a metric (the Euclidean and square metric) that induces this product topology. We can attempt to extrapolate these formulas to $\mathbb{R}^\omega$ by defining
  \begin{align*}
    & d(x, y) \equiv \bigg(\sum_{i=1}^\infty (x_i - y_i)^2 \bigg)^{\frac{1}{2}} \\
    & \rho(x, y) \equiv \sup{\{|x_i - y_i|\}}
  \end{align*}
  However, the metrics do not in general map to elements of $\mathbb{R}$, since the sequence $(x_i - y_i)_{i \in \mathbb{N}}$ could diverge. Therefore, we can redefine the metric $\rho$ to the following bounded one. 
  \begin{equation}
    \Tilde{\rho} (x, y) \equiv \sup{\{\Tilde{d}(x_i, y_i)\}}
  \end{equation}
  where $\Tilde{d}$ is the standard bounded metric on $\mathbb{R}$. Clearly,
  \begin{equation}
    0 \leq \Tilde{\rho}(x, y) \leq 1
  \end{equation}
  $\Tilde{\rho}$ is indeed a metric on $\mathbb{R}^\omega$, but unfortunately, it does not induce the product topology. We extend this definition to arbitrary $\mathbb{R}^J$. 
  
  \begin{definition}[Uniform Metric]
    Given an indexed set $J$ with points $x, y \in \mathbb{R}^J$, we define
    \begin{equation}
      \Tilde{\rho} \equiv \sup{\{\Tilde{d}(x_\alpha, y_\alpha)\;|\; \alpha \in J\}}
    \end{equation}
    with $\Tilde{d}$ the standard bounded metric on $\mathbb{R}$. $\Tilde{\rho}$ is called the \textbf{uniform metric on $\mathbb{R}^J$}, which induces the \textbf{uniform topology}. 
  \end{definition}

  The uniform topology on $\mathbb{R}^J$ is finer than the product topology, and they are different if $J$ is infinite. Clearly, $0 \leq \Tilde{p} (x, y) \leq 1$, meaning that given the open ball
  \begin{equation}
    B_r (x) \equiv \{y \in \mathbb{R}^J \;|\; \Tilde{p}(y, x) < r\}
  \end{equation}
  if $r \geq 1$, then $B_r (x) = \mathbb{R}^J$ and if $r<1$, then $B_r (x)$ consists of the $n$-dimensional box with "radius" $r$, where $n = \dim{\mathbb{R}^J}$. 


  The next theorem gives us a metric that induces the product topology on infinite dimensional $\mathbb{R}^\omega$ by slightly modifying the uniform metric on $\mathbb{R}$. However, with the box topology $\mathbb{R}^\omega$ is not metrizable. 

  \begin{theorem}
    Let $\Tilde{d} (a, b) \equiv \min{\{|a-b|, 1\}}$ be the standard bounded metric on $\mathbb{R}$. If $x, y \in \mathbb{R}^\omega$, we define
    \begin{equation}
      D(x, y) \equiv \sup{\Big\{ \frac{\Tilde{d}(x_i, y_i)}{i}\Big\}}
    \end{equation}
    Then, $D$ is a metric that induces the product topology on $\mathbb{R}^\omega$. 
  \end{theorem}
  
  It is easy to see that $0 \leq D(x, y) \leq 1$. So, given the open ball
  \begin{equation}
    B_r (x) \equiv \{y \in \mathbb{R}^\omega \; | \; D(x, y) < r\}
  \end{equation}
  $B_r (x) = \mathbb{R}^\omega$ if $r > 1$. When $r \leq 1$, 
  \begin{equation}
    B_r (x) \equiv (y-r, y+r) \times (y-2r, y+2r) \times ... = \prod_{k=1}^\infty (y - k r , y + k r)
  \end{equation}

  \begin{figure}[H]
    \centering 
    \begin{tikzpicture}
      \node[below left] at (0,0) {$y$};
      \draw[fill] (0,0) circle (0.05);
      \draw[<->] (0, -2)--(0,2);
      \draw[<->] (-3,0)--(3,0);
      \draw[fill] (-2,0) circle (0.05); 
      \draw[fill] (2,0) circle (0.05); 
      \draw[fill] (0,1) circle (0.05); 
      \draw[fill] (0,-1) circle (0.05); 
      \node[above left] at (-2,0) {$-2$};
      \node[above right] at (2,0) {$2$};
      \node[above right] at (0,1) {$1$};
      \node[below right] at (0,-1) {$-1$};
      \draw[fill] (0, 0.7) circle (0.05);
      \draw[fill] (1.4, 0) circle (0.05);
      \node[above left] at (0,0.7) {$r$};
      \node[below right] at (1.4,0) {$2r$};
      \draw[dashed] (-1.4, -0.7) rectangle (1.4,0.7);
      \node[right] at (0,2) {$\mathbb{R}_1$};
      \node[above] at (3,0) {$\mathbb{R}_2$};
    \end{tikzpicture}
    \caption{Visually, we take a cross section of this box and look at the slice within $\mathbb{R}_1 \times \mathbb{R}_2$, where the subscripts represent the first and second terms of $x$.}
    \label{fig:cross_sec}
  \end{figure}

  We can extend the applications of the Bolzano Weierstrass Lemma from analysis to metric spaces in general with the following lemma. 

  \begin{lemma}[Sequence Lemma]
    If $X$ be a topological space with $A \subset X$. If there exists a sequence of points of $A$ that converges to $x$, then $x \in \bar{A}$. The converse is true if $X$ is metrizable. 
  \end{lemma}
  \begin{proof}
    $(\rightarrow)$ Our hypothesis says that $x$ is a limit point of $A$, which by definition means that $x \in \bar{A}$. \\
    $(\leftarrow)$ Assuming $X$ is metrizable and $x \in \bar{A}$, let $d$ be a metric for the topology of $X$. Then, for every $n \in \mathbb{N}$, let us define a sequence of open neighborhoods of $x$ to be
    \begin{equation}
      \big(B_{\frac{1}{n}} (x) \big)
    \end{equation}
    Since $x \in \bar{A}$, there exists a point 
    \begin{equation}
      x_n \in A \cap B_{\frac{1}{n}} (x) \text{ for all } n \in \mathbb{N}
    \end{equation}
    This sequence $(x_n)$ that we have proved must exist converges to $x$. 
  \end{proof}

  \begin{theorem}
    Let $f: X \longrightarrow Y$ and let $X$ be metrizable. $f$ is continuous if and only if for every convergent sequence $(x_n) \rightarrow x$ of $X$, the following sequence of $Y$ converges to $f(x)$. That is, 
    \begin{equation}
      \big( f(x_n) \big) \longrightarrow f(x)
    \end{equation}
  \end{theorem}

  We introduce additional methods of constructing continuous functions. 

  \begin{definition}[Uniform Convergence]
    Let $f_n: X \longrightarrow Y$ be a sequence of functions from the set $X$ to the metric space $(Y, d)$. The sequence $(f_n)$ is said to \textbf{converge uniformly} to the function $f: X \longrightarrow Y$ if, given $\epsilon > 0$, there exists a $N \in \mathbb{N}$ such that
    \begin{equation}
      d\big( f_n(x), f(x)\big) < \epsilon
    \end{equation}
    for all $n \geq N$ and for all $x \in X$. 
  \end{definition}

  \begin{theorem}[Uniform Limit Theorem]
    Let $f_n: X \longrightarrow Y$ be a sequence of continuous functions from topological space $X$ to a metric space $Y$. If $f_n$ converges uniformly to $f$, then $f$ is continuous. 
  \end{theorem}
  \begin{proof}
    $(\rightarrow)$ Trivial. \\
    $(\leftarrow)$ Let $V$ be open in $Y$, and let $x_0$ be a point in $f^{-1} (V)$. It suffices to prove that for every $x_0 \in f^{-1} (V)$, there exists a neighborhood $U$ of $x_0$ such that $U \subset F^{-1} (V)$ or equivalently, $F(U) \subset V$. 

    Let $y_0 = f(x_0)$. Since $Y$ is a metric space with metric $d$, we know that there exists an $\epsilon$-ball $B_\epsilon (y_0)$ such that
    \begin{equation}
      B_\epsilon (y_0) \subset V
    \end{equation}
    Then, using uniform convergence, we can choose $N \in \mathbb{N}$ such that for all $n \geq N$ and all $x \in X$, 
    \begin{equation}
      d \big( f_n (x), f(x) \big) < \frac{\epsilon}{4}
    \end{equation}
    which also applies at the point $x = x_0$. 
    \begin{equation}
      d \big( f_n (x_0), f(x_0) \big) < \frac{\epsilon}{4}
    \end{equation}
    Using continuity of $f_n$, choose a neighborhood $U$ of $x_0$ such that $f_n$ carries $U$ into the open $\epsilon/2$-ball centered at $f_n (x_0)$ (note that $f_n (x_0) \neq y_0$), meaning that if $x \in U$
    \begin{equation}
      d \big( f_n (x), f_n (x_0) \big) < \frac{\epsilon}{2}
    \end{equation}
    Adding the three inequalities and using the triangle inequality, we get the fact that if $x \in U$, then 
    \begin{equation}
      d \big( f(x), f(x_0) \big) < \epsilon
    \end{equation}
    meaning that the $f(U) \subset B_\epsilon (x_0) \subset V$. 

    \begin{figure}[H]
      \centering 
      \begin{tikzpicture}[scale=0.6]
        \draw[dashed, teal] (0,0) circle [radius=8];
        \draw[fill] (0,0) circle [radius=0.05];
        \draw[fill] (-1.5,0.5) circle [radius=0.05];
        \draw[dashed, purple] (0,0) circle [radius=2];
        \draw[dashed, purple] (-1.5,0.5) circle [radius=2];
        \node[right] at (0.3,0) {$f(x_0)$};
        \node[above right] at (-1.5,0.5) {$f_N (x_0)$};
        \draw[dashed, blue] (-1.5,0.5) circle [radius=4];
        \draw[fill] (-4,3) circle [radius=0.05];
        \draw[dashed, purple] (-4,3) circle [radius=2];
        \node[right] at (-4,3) {$f_N (x)$};
        \draw[fill] (-5.5, 3.7) circle [radius=0.05];
        \node[above left] at (-5,4.5) {$f(x)$};
        \draw[-, thick] (-5.5, 3.7)--(-4,3)--(-1.5,0.5)--(0,0);
        \draw[->, thick, purple] (0,0)--(0,-2);
        \draw[->, thick, blue] (-1.5,0.5)--(-1.5,-3.5);
        \node[right, purple] at (0,-1) {$\epsilon / 4$};
        \node[left,blue] at (-1.5,-1.5) {$\epsilon / 2$};
        \draw[->, thick, teal] (0,0)--(6.128, 5.1416);
        \node[below right, teal] at (3.064, 2.57) {$\epsilon$};
      \end{tikzpicture}
      \caption{Visually, the three inequalities represent the following open balls in $V \subset Y$.} 
      \label{fig:uniform_limit}
    \end{figure}
  \end{proof}


  \begin{theorem}
    In a metric space $(X, d)$, a set is \textbf{closed} if the limit of every convergent subsequence in $X$ lies in $X$. That is, $X$ contains all of its limit points. 
  \end{theorem}

\subsection{Exercises} 

\begin{exercise}[Munkres 20.1]
  \begin{enumerate} 
    \item[(a)] In $\mathbb{R}^n$, define
    \begin{align*}
      d'(\mathbf{x}, \mathbf{y}) = |x_1 - y_1| + \cdots + |x_n - y_n|.
    \end{align*}
    Show that $d'$ is a metric that induces the usual topology of $\mathbb{R}^n$. Sketch the basis elements under $d'$ when $n = 2$.
    
    \item[(b)] More generally, given $p \geq 1$, define
    \begin{align*}
      d'(\mathbf{x}, \mathbf{y}) = \left[\sum_{i=1}^{n} |x_i - y_i|^p\right]^{1/p}
    \end{align*}
    for $\mathbf{x}, \mathbf{y} \in \mathbb{R}^n$. Assume that $d'$ is a metric. Show that it induces the usual topology on $\mathbb{R}^n$.
  \end{enumerate}
\end{exercise}

\begin{exercise}[Munkres 20.2]
  Show that $\mathbb{R} \times \mathbb{R}$ in the dictionary order topology is metrizable.
\end{exercise}

\begin{exercise}[Munkres 20.3]
  \begin{enumerate} 
    \item[(a)] Let $X$ be a metric space with metric $d$. Show that $d : X \times X \to \mathbb{R}$ is continuous.
    
    \item[(b)] Let $X'$ denote a space having the same underlying set as $X$. Show that if $d : X' \times X' \to \mathbb{R}$ is continuous, then the topology of $X'$ is finer than the topology of $X$.
  \end{enumerate}
  
  One can summarize the result of this exercise as follows: If $X$ has a metric $d$, then the topology induced by $d$ is the coarsest topology relative to which the function $d$ is continuous.
\end{exercise}

\begin{exercise}[Munkres 20.4]
  Consider the product, uniform, and box topologies on $\mathbb{R}^\omega$.
  \begin{enumerate} 
    \item[(a)] In which topologies are the following functions from $\mathbb{R}$ to $\mathbb{R}^\omega$ continuous?
    \begin{align*}
      f(t) &= (t, 2t, 3t, \ldots), \\
      g(t) &= (t, t, t, \ldots), \\
      h(t) &= (t, \frac{1}{2}t, \frac{1}{3}t, \ldots).
    \end{align*}
    
    \item[(b)] In which topologies do the following sequences converge?
    \begin{align*}
      \mathbf{w}_1 &= (1, 1, 1, 1, \ldots), & \mathbf{x}_1 &= (1, 1, 1, 1, \ldots), \\
      \mathbf{w}_2 &= (0, 2, 2, 2, \ldots), & \mathbf{x}_2 &= (0, \frac{1}{2}, \frac{1}{2}, \frac{1}{2}, \ldots), \\
      \mathbf{w}_3 &= (0, 0, 3, 3, \ldots), & \mathbf{x}_3 &= (0, 0, \frac{1}{3}, \frac{1}{3} \ldots), \\
      \ldots & & \ldots \\
      \mathbf{y}_1 &= (1, 0, 0, 0, \ldots), & \mathbf{z}_1 &= (1, 1, 0, 0, \ldots), \\
      \mathbf{y}_2 &= (\frac{1}{2}, \frac{1}{2}, 0, 0, \ldots), & \mathbf{z}_2 &= (\frac{1}{2}, \frac{1}{2}, 0, 0, \ldots), \\
      \mathbf{y}_3 &= (\frac{1}{3}, \frac{1}{3}, \frac{1}{3}, 0, \ldots), & \mathbf{z}_3 &= (\frac{1}{3}, \frac{1}{3}, 0, 0, \ldots), \\
      \ldots & & \ldots
    \end{align*}
  \end{enumerate}
\end{exercise}

\begin{exercise}[Munkres 20.5]
  Let $\mathbb{R}^\infty$ be the subset of $\mathbb{R}^\omega$ consisting of all sequences that are eventually zero. What is the closure of $\mathbb{R}^\infty$ in $\mathbb{R}^\omega$ in the uniform topology? Justify your answer.
\end{exercise}

\begin{exercise}[Munkres 20.6]
  Let $\bar{\rho}$ be the uniform metric on $\mathbb{R}^\omega$. Given $\mathbf{x} = (x_1, x_2, \ldots) \in \mathbb{R}^\omega$ and given $0 < \epsilon < 1$, let
  \begin{align*}
    U(\mathbf{x}, \epsilon) = (x_1 - \epsilon, x_1 + \epsilon) \times \cdots \times (x_n - \epsilon, x_n + \epsilon) \times \cdots.
  \end{align*}
  \begin{enumerate} 
    \item[(a)] Show that $U(\mathbf{x}, \epsilon)$ is not equal to the $\epsilon$-ball $B_{\bar{\rho}}(\mathbf{x}, \epsilon)$.
    \item[(b)] Show that $U(\mathbf{x}, \epsilon)$ is not even open in the uniform topology.
    \item[(c)] Show that
    \begin{align*}
      B_{\bar{\rho}}(\mathbf{x}, \epsilon) = \bigcup_{\delta < \epsilon} U(\mathbf{x}, \delta).
    \end{align*}
  \end{enumerate}
\end{exercise}

\begin{exercise}[Munkres 20.7]
  Consider the map $h : \mathbb{R}^\omega \to \mathbb{R}^\omega$ defined in Exercise 8 of \S19; give $\mathbb{R}^\omega$ the uniform topology. Under what conditions on the numbers $a_i$ and $b_i$ is $h$ continuous? a homeomorphism?
\end{exercise}

\begin{exercise}[Munkres 20.8]
  Let $X$ be the subset of $\mathbb{R}^\omega$ consisting of all sequences $\mathbf{x}$ such that $\sum x_i^2$ converges. Then the formula
  \begin{align*}
    d(\mathbf{x}, \mathbf{y}) = \left[\sum_{i=1}^{\infty}(x_i - y_i)^2\right]^{1/2}
  \end{align*}
  defines a metric on $X$. (See Exercise 10.) On $X$ we have the three topologies it inherits from the box, uniform, and product topologies on $\mathbb{R}^\omega$. We have also the topology given by the metric $d$, which we call the $\ell^2$-topology. (Read "little ell two.")
  \begin{enumerate} 
    \item[(a)] Show that on $X$, we have the inclusions
    \begin{align*}
      \text{box topology} \supset \ell^2\text{-topology} \supset \text{uniform topology}.
    \end{align*}
    
    \item[(b)] The set $\mathbb{R}^\infty$ of all sequences that are eventually zero is contained in $X$. Show that the four topologies that $\mathbb{R}^\infty$ inherits as a subspace of $X$ are all distinct.
    
    \item[(c)] The set
    \begin{align*}
      H = \prod_{n\in\mathbb{Z}_+} [0, 1/n]
    \end{align*}
    is contained in $X$; it is called the Hilbert cube. Compare the four topologies that $H$ inherits as a subspace of $X$.
  \end{enumerate}
\end{exercise}

\begin{exercise}[Munkres 20.9]
  Show that the euclidean metric $d$ on $\mathbb{R}^n$ is a metric, as follows: If $\mathbf{x}, \mathbf{y} \in \mathbb{R}^n$ and $c \in \mathbb{R}$, define
  \begin{align*}
    \mathbf{x} + \mathbf{y} &= (x_1 + y_1, \ldots, x_n + y_n), \\
    c\mathbf{x} &= (cx_1, \ldots, cx_n), \\
    \mathbf{x} \cdot \mathbf{y} &= x_1y_1 + \cdots + x_ny_n.
  \end{align*}
  \begin{enumerate} 
    \item[(a)] Show that $\mathbf{x} \cdot (\mathbf{y} + \mathbf{z}) = (\mathbf{x} \cdot \mathbf{y}) + (\mathbf{x} \cdot \mathbf{z})$.
    \item[(b)] Show that $|\mathbf{x}\cdot\mathbf{y}| \leq \|\mathbf{x}\|\|\mathbf{y}\|$. [Hint: If $\mathbf{x}, \mathbf{y} \neq 0$, let $a = 1/\|\mathbf{x}\|$ and $b = 1/\|\mathbf{y}\|$, and use the fact that $\|a\mathbf{x} \pm b\mathbf{y}\| \geq 0$.]
    \item[(c)] Show that $\|\mathbf{x} + \mathbf{y}\| \leq \|\mathbf{x}\| + \|\mathbf{y}\|$. [Hint: Compute $(\mathbf{x} + \mathbf{y}) \cdot (\mathbf{x} + \mathbf{y})$ and apply (b).]
    \item[(d)] Verify that $d$ is a metric.
  \end{enumerate}
\end{exercise}

\begin{exercise}[Munkres 20.10]
  Let $X$ denote the subset of $\mathbb{R}^\omega$ consisting of all sequences $(x_1, x_2, \ldots)$ such that $\sum x_i^2$ converges. (You may assume the standard facts about infinite series. In case they are not familiar to you, we shall give them in Exercise 11 of the next section.)
  \begin{enumerate} 
    \item[(a)] Show that if $\mathbf{x}, \mathbf{y} \in X$, then $\sum |x_i y_i|$ converges. [Hint: Use (b) of Exercise 9 to show that the partial sums are bounded.]
    \item[(b)] Let $c \in \mathbb{R}$. Show that if $\mathbf{x}, \mathbf{y} \in X$, then so are $\mathbf{x} + \mathbf{y}$ and $c\mathbf{x}$.
    \item[(c)] Show that
    \begin{align*}
      d(\mathbf{x}, \mathbf{y}) = \left[\sum_{i=1}^{\infty}(x_i - y_i)^2\right]^{1/2}
    \end{align*}
    is a well-defined metric on $X$.
  \end{enumerate}
\end{exercise}

\begin{exercise}[Munkres 20.11]
  Show that if $d$ is a metric for $X$, then
  \begin{align*}
    d'(x, y) = d(x, y)/(1 + d(x, y))
  \end{align*}
  is a bounded metric that gives the topology of $X$. [Hint: If $f(x) = x/(1 + x)$ for $x > 0$, use the mean-value theorem to show that $f(a + b) - f(b) \leq f(a)$.]
\end{exercise}

\begin{exercise}[Munkres 21.1]
  Let $A \subset X$. If $d$ is a metric for the topology of $X$, show that $d|A \times A$ is a metric for the subspace topology on $A$.
\end{exercise}

\begin{exercise}[Munkres 21.2]
  Let $X$ and $Y$ be metric spaces with metrics $d_X$ and $d_Y$, respectively. Let $f : X \to Y$ have the property that for every pair of points $x_1, x_2$ of $X$,
  \begin{align*}
    d_Y(f(x_1), f(x_2)) = d_X(x_1, x_2).
  \end{align*}
  Show that $f$ is an imbedding. It is called an \textit{isometric imbedding} of $X$ in $Y$.
\end{exercise}

\begin{exercise}[Munkres 21.3]
  Let $X_n$ be a metric space with metric $d_n$, for $n \in \mathbb{Z}_+$.
  \begin{enumerate}
    \item Show that
      \begin{equation}
        \rho(x,y) = \max\{d_1(x_1, y_1), \ldots, d_n(x_n, y_n)\}
      \end{equation}
      is a metric for the product space $X_1 \times \cdots \times X_n$.
    
    \item Let $\tilde{d_i} = \min\{d_i, 1\}$. Show that
      \begin{equation}
        D(x,y) = \sup\{\tilde{d_i}(x_i, y_i)/i\}
      \end{equation}
      is a metric for the product space $\prod X_i$.
  \end{enumerate}
\end{exercise}
\begin{solution}
  For the first part, we prove the properties of the metric. 
  \begin{enumerate}
    \item Nonnegativity. Note that $\rho$ is the maximum of a finite set of metrics, which must be nonnegative, and so $\rho(x, y) \geq 0$. Second,  
    \begin{align}
      \rho(x, y) = 0 & \iff \max\{d_i (x_i, y_i)\} = 0 \\
                     & \iff d_i (x_i, y_i) = 0 \text{ for all } i = 1, \ldots, n  \\
                     & \iff x_i = y_i \text{ for all } i \\
                     & \iff x = y
    \end{align}

    \item Symmetricity. 
    \begin{equation}
      \rho(x, y) = \max\{d_i (x_i, y_i)\} = \max\{d_i (y_i, x_i)\} = \rho(y, x)
    \end{equation}

    \item Triangle inequality. 
    \begin{align}
      \rho(x, y) + \rho(y, z) & = \max_i \{d_i (x_i, y_i)\} + \max_j \{d_j (y_j, z_j)\} \\
                              & \geq \max_i \{ d_i (x_i, y_i) + d_i (y_i, z_i) \} \\
                              & \geq \max_i \{ d_i (x_i, z_i) \} \\
                              & = \rho(x, z)
    \end{align}
  \end{enumerate} 
  
  For the second part, we do the same. 
  \begin{enumerate}
    \item Nonnegativity. Since $\tilde{d}_i \geq 0$, $\tilde{d}_i / i \geq 0/i = 0$ and so the supremum must be at least $0$ (if it's negative then it will not bound $d_1$.) Second, 
      \begin{align}
        D(x, y) = 0 & \iff \sup\{\tilde{d}_i (x_i, y_i) / i \} = 0 \\
                    & \iff \tilde{d}_i (x_i, y_i) / i = 0 \text{ for all } i \\
                    & \iff \tilde{d}_i (x_i, y_i) = 0 \text{ for all } i \\
                    & \iff \min\{d_i (x_i, y_i), 1 \} = 0 \text{ for all } i \\ 
                    & \iff d_i (x_i, y_i) = 0 \text{ for all } i \\
                    & \iff x_i = y_i \text{ for all } i \\
                    & \iff x = y
      \end{align}

    \item Symmetricity. 
      \begin{equation}
        D(x, y) = \sup\{\tilde{d}_i (x_i, y_i) / i \} = \sup\{\tilde{d}_i (y_i, x_i) / i \} = D(y, x)
      \end{equation}

    \item Triangle inequality. 
      \begin{align}
        D(x, y) + D(y, z) & = \sup_i \{\tilde{d}_i (x_i, y_i) / i \} + \sup_j \{\tilde{d}_j (y_j, z_j) / j \} \\
                          & \geq \sup_i \bigg\{ \frac{\tilde{d}_i (x_i, y_i) + \tilde{d}_i (y_i, z_i)}{i} \bigg\} \\
                          & = \sup_i \bigg\{ \frac{ \min\{d_i (x_i, y_i), 1\} + \min\{ d_i (y_i, z_i), 1\}}{i} \bigg\} \\
                          & \geq \sup_i \bigg\{ \frac{ \min\{d_i (x_i, y_i) + d_i (y_i, z_i), 1\}}{i} \bigg\} \\ 
                          & \geq \sup_i \bigg\{ \frac{ \min\{d_i (x_i, z_i), 1\}}{i} \bigg\} \\ 
                          & = D(x, z)
      \end{align}
  \end{enumerate}
\end{solution}

\begin{exercise}[Munkres 21.4]
  Show that $\mathbb{R}_\ell$ and the ordered square satisfy the first countability axiom. (This result does not, of course, imply that they are metrizable.)
\end{exercise}

\begin{exercise}[Munkres 21.5]
  Theorem. Let $x_n \to x$ and $y_n \to y$ in the space $\mathbb{R}$. Then
  \begin{align*}
    x_n + y_n &\to x + y, \\
    x_n - y_n &\to x - y, \\
    x_ny_n &\to xy,
  \end{align*}
  and provided that each $y_n \neq 0$ and $y \neq 0$,
  \begin{align*}
    x_n/y_n \to x/y.
  \end{align*}
  [Hint: Apply Lemma 21.4; recall from the exercises of \S19 that if $x_n \to x$ and $y_n \to y$, then $x_n \times y_n \to x \times y$.]
\end{exercise}

\begin{exercise}[Munkres 21.6]
  Define $f_n : [0, 1] \to \mathbb{R}$ by the equation $f_n(x) = x^n$. Show that the sequence $(f_n(x))$ converges for each $x \in [0, 1]$, but that the sequence $(f_n)$ does not converge uniformly.
\end{exercise}

\begin{exercise}[Munkres 21.7]
  Let $X$ be a set, and let $f_n : X \to \mathbb{R}$ be a sequence of functions. Let $\bar{\rho}$ be the uniform metric on the space $\mathbb{R}^X$. Show that the sequence $(f_n)$ converges uniformly to the function $f : X \to \mathbb{R}$ if and only if the sequence $(f_n)$ converges to $f$ as elements of the metric space $(\mathbb{R}^X, \bar{\rho})$.
\end{exercise}

\begin{exercise}[Munkres 21.8]
  Let $X$ be a topological space and let $Y$ be a metric space. Let $f_n : X \to Y$ be a sequence of continuous functions. Let $x_n$ be a sequence of points of $X$ converging to $x$. Show that if the sequence $(f_n)$ converges uniformly to $f$, then $(f_n(x_n))$ converges to $f(x)$.
\end{exercise}

\begin{exercise}[Munkres 21.9]
  Let $f_n : \mathbb{R} \to \mathbb{R}$ be the function
  \begin{align*}
    f_n(x) = \frac{1}{n^3[x - (1/n)]^2 + 1}.
  \end{align*}
  See Figure 21.1. Let $f : \mathbb{R} \to \mathbb{R}$ be the zero function.
  \begin{enumerate} 
    \item[(a)] Show that $f_n(x) \to f(x)$ for each $x \in \mathbb{R}$.
    \item[(b)] Show that $f_n$ does not converge uniformly to $f$. (This shows that the converse of Theorem 21.6 does not hold; the limit function $f$ may be continuous even though the convergence is not uniform.)
  \end{enumerate}
\end{exercise}

\begin{exercise}[Munkres 21.10]
  Using the closed set formulation of continuity (Theorem 18.1), show that the following are closed subsets of $\mathbb{R}^2$:
  \begin{align*}
    A &= \{x \times y \mid xy = 1\}, \\
    S^1 &= \{x \times y \mid x^2 + y^2 = 1\}, \\
    B^2 &= \{x \times y \mid x^2 + y^2 \leq 1\}.
  \end{align*}
\end{exercise}

\begin{exercise}[Munkres 21.11]
  Prove the following standard facts about infinite series:
  \begin{enumerate} 
    \item[(a)] Show that if $(s_n)$ is a bounded sequence of real numbers and $s_n \leq s_{n+1}$ for each $n$, then $(s_n)$ converges.
    \item[(b)] Let $(a_n)$ be a sequence of real numbers; define
    \begin{align*}
      s_n = \sum_{i=1}^{n} a_i.
    \end{align*}
    If $s_n \to s$, we say that the \textit{infinite series}
    \begin{align*}
      \sum_{i=1}^{\infty} a_i
    \end{align*}
    converges to $s$ also. Show that if $\sum a_i$ converges to $s$ and $\sum b_i$ converges to $t$, then $\sum(ca_i + b_i)$ converges to $cs + t$.
    \item[(c)] Prove the \textit{comparison test} for infinite series: If $|a_i| \leq b_i$ for each $i$, and if the series $\sum b_i$ converges, then the series $\sum a_i$ converges. [Hint: Show that the series $\sum|a_i|$ and $\sum c_i$ converge, where $c_i = |a_i| + a_i$.]
    \item[(d)] Given a sequence of functions $f_n : X \to \mathbb{R}$, let
    \begin{align*}
      s_n(x) = \sum_{i=1}^{n} f_i(x).
    \end{align*}
    Prove the \textit{Weierstrass M-test} for uniform convergence: If $|f_i(x)| \leq M_i$ for all $x \in X$ and all $i$, and if the series $\sum M_i$ converges, then the sequence $(s_n)$ converges uniformly to a function $s$. [Hint: Let $r_n = \sum_{i=n+1}^{\infty} M_i$. Show that if $k > n$, then $|s_k(x) - s_n(x)| \leq r_n$; conclude that $|s(x) - s_n(x)| \leq r_n$.]
  \end{enumerate}
\end{exercise}

\begin{exercise}[Munkres 21.12]
  Prove continuity of the algebraic operations on $\mathbb{R}$, as follows: Use the metric $d(a, b) = |a - b|$ on $\mathbb{R}$ and the metric on $\mathbb{R}^2$ given by the equation
  \begin{align*}
    \rho((x, y), (x_0, y_0)) = \max\{|x - x_0|, |y - y_0|\}.
  \end{align*}
  \begin{enumerate} 
    \item[(a)] Show that addition is continuous. [Hint: Given $\epsilon$, let $\delta = \epsilon/2$ and note that 
    \begin{align*}
      d(x + y, x_0 + y_0) \leq |x - x_0| + |y - y_0|.
    \end{align*}
    ]
    \item[(b)] Show that multiplication is continuous. [Hint: Given $(x_0, y_0)$ and $0 < \epsilon < 1$, let 
    \begin{align*}
      3\delta = \epsilon/(|x_0| + |y_0| + 1)
    \end{align*}
    and note that
    \begin{align*}
      d(xy, x_0y_0) \leq |x_0||y - y_0| + |y_0||x - x_0| + |x - x_0||y - y_0|.
    \end{align*}
    ]
    \item[(c)] Show that the operation of taking reciprocals is a continuous map from $\mathbb{R} - \{0\}$ to $\mathbb{R}$. [Hint: Show the inverse image of the interval $(a, b)$ is open. Consider five cases, according as $a$ and $b$ are positive, negative, or zero.]
    \item[(d)] Show that the subtraction and quotient operations are continuous.
  \end{enumerate}
\end{exercise}



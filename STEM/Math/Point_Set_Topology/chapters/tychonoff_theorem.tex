\section{The Tychonoff Theorem}

  \begin{theorem}
  An arbitrary product of compact spaces is compact under the product topology. 
  \end{theorem}

  \begin{definition}
  A space $X$ is \textbf{completely regular} if one-point sets are closed in $X$ and if for each point $x_0$ and each closed set $A$ not containing $x_0$, there is a continuous function $f: X \longrightarrow [0,1]$ such that $f(x_0) = 1$ and $f(A) = \{0\}$. 
  \end{definition}

  \begin{theorem}
  A subspace of a completely regular space is completely regular. A product of completely regular spaces is completely regular. 
  \end{theorem}

  \begin{theorem}
  If $X$ is completely regular, then $X$ can be imbedded in $[0,1]^J$ for some $J$. 
  \end{theorem}

  \begin{corollary}
  Let $X$ be a space. The following are equivalent: 
  \begin{enumerate}
      \item $X$ is completely regular. 
      \item $X$ is homeomorphic to a subspace of a compact Hausdorff space. 
      \item $X$ is homeomorphic to a subspace of a normal space. 
  \end{enumerate}
  \end{corollary}

  \begin{definition}
  A \textbf{compactification} of a space $X$ is a compact Hausdorff space $Y$ containing $X$ such that $X$ is dense in $Y$ (that is $\bar{X} = Y$). Two compactifications $Y_1$ and $Y_2$ of $X$ are said to be \textbf{equivalent} if there is a homeomorphism $h: Y_1 \longrightarrow Y_2$ such that $h(x) = x$ for every $x \in X$. 
  \end{definition}

  \begin{theorem}
  Let $X$ be completely regular, and let $\beta(X)$ be its Stone-Cech compatification. Then every bounded continuous real-valued function on $X$ can be uniquely extended to a continuous real-valued function on $\beta(X)$. 
  \end{theorem}

  \begin{lemma}
  Let $A \subset X$, and let $f: A \longrightarrow Z$ be a continuous map of $A$ into the Hausdorff space $Z$. There is at most one extension of $f$ to a continuous function $g: \bar{A} \longrightarrow Z$. 
  \end{lemma}

  \begin{theorem}
  Let $X$ be completely regular. Let $Y_1, Y_2$ be two compactifications of $X$ having the extension property. Then there is a homeomorphism $\phi$ of $Y_1$ onto $Y_2$ such that $\phi(x) = x$ for each $x \in X$. 
  \end{theorem}


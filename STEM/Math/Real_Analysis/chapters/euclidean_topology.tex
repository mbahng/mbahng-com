\section{Euclidean Topology} 

  With the construction of the real line and the real space, the extra properties of completeness, norm, and order (for the real line) allows us to restate these topological properties in terms of these ``higher-order'' properties. It also proves much more results than for general topological spaces. Therefore, the next few sections will focus on reiterating the topological properties of $\mathbb{R}$ and $\mathbb{R}^n$ (this can be done slightly more generally for metric spaces, but we talk about this in point-set topology). In this section, we will restate the notion of open sets, limit points, compactness, connectedness, and separability. Then we can continue in the next section sequences and their limits, and after that we describe continuity. Once this is done, we can focus constructing the derivative and integral, which are unique to Banach spaces.  

\subsection{Open Sets} 

  It is well-known that the set of open-balls of a metric space $(X, d)$ is indeed a topology, which we prove in point-set topology. Once we prove this, we have access to a whole suite of theorems on topological spaces that we can just apply to $\mathbb{R}^n$. We will restate many of these topological theorems for completeness but will not prove them. However, if any of these theorems use any other structure, such as order/metrics/norms/completeness, we will have to prove them. 

  \begin{definition}[Topology]
    Let $X$ be a set and $\mathscr{T}$ be a family of subsets of $X$. Then $\mathscr{T}$ is a \textbf{topology} on $X$\footnote{I will use script letters to denote topologies and capital letters to denote sets.} if it satisfies the following properties. 
    \begin{enumerate}
      \item \textit{Contains Empty and Whole Set}: 
      \begin{equation}
        \emptyset, X \in \mathscr{T}
      \end{equation}

      \item \textit{Closure Under Union}. If $\{U_\alpha\}_{\alpha \in A}$ is a class of sets in $\mathscr{T}$, then 
      \begin{equation}
        \bigcup_{\alpha \in A} U_\alpha \in \mathscr{T}
      \end{equation}

      \item \textit{Closure Under Finite Intersection}: If $U_1, \ldots, U_n$ is a finite class of sets in $\mathscr{T}$, then 
      \begin{equation}
       \bigcap_{i=1}^{n} U_i \in \mathscr{T}
      \end{equation}
    \end{enumerate}
    A \textbf{topological space} is denoted $(X, \mathscr{T})$. 
  \end{definition}

  \begin{theorem}[Euclidean Topology]
    Let $\tau_{\mathbb{R}}$ (which we denote as $\mathscr{T}$) be the set of subsets $S$ of $(\mathbb{R}^n, || \cdot ||)$ satisfying the property that if $x \in S$, then there exists an open $\epsilon$-ball $B(x, \epsilon)$ s.t. $B \subset S$. $\mathscr{T}$ is a topology of $\mathbb{R}^n$. 
  \end{theorem} 
  \begin{proof} 
    We prove the following three properties. 
    \begin{enumerate}
      \item $\emptyset, \mathbb{R}^n$ are open. 
      \item For any collection $\{G_\alpha\}_\alpha$ of open sets, $\cup_\alpha G_\alpha$ is open.  
      \item For any finite collection $G_1, \ldots, G_n$ of open sets, $\cap_{i=1}^n G_i$ is open. 
    \end{enumerate}
    Listed. 
    \begin{enumerate}
      \item Let $x \in \cup_\alpha G_\alpha$. Then, $x \in G_k$ for some $k$ and since $G_k$ is open, there exists a $B_\epsilon (x) \subset G_k \subset \cup_{\alpha} G_\alpha$, proving that $\cup_\alpha G_\alpha$ is open. 
      \item Let $x \in \cap_{i=1}^n G_i$. Then, $x \in G_i$ for every $i$, and so for each $G_i$, there exists an $\epsilon_i > 0$ s.t. $B_{\epsilon_i} (x) \subset G_i$. Since the set $\{e_i\}$ is finite, we can take 
      \[\epsilon = \min_i \{\epsilon_i\}\]
      and see that $B_\epsilon (x) \subset G_i$ for all $i$, which implies that $B_\epsilon (x) \subset \cap_{i=1}^n G_i$. Since we have proved the existence of $\epsilon$, $\cap_{i=1}^n G_i$ is open. 
    \end{enumerate}
  \end{proof}

  \begin{definition}[Open Set]
    An \textbf{open set} is an element of $\mathscr{T}$. 
    \begin{enumerate}
      \item An \textbf{open neighborhood}, or sometimes just the \textbf{neighborhood}, of $x \in \mathbb{R}^n$ is an open set $U_x$ containing $x$. 
      \item A \textbf{punctured neighborhood} is $U_x^{\circ} = U_x \setminus \{x\}$. 
    \end{enumerate}
  \end{definition}

  \begin{theorem}[Equivalence to Open Ball Topology]
    $\mathscr{T}$ is equal to the topology $\mathscr{T}^\prime$ generated by the basis $\mathscr{B}$ of open balls 
    \begin{equation} 
      B(x, r) \coloneqq \{ y \in \mathbb{R}^n \mid ||x - y|| < r\}
    \end{equation}
  \end{theorem} 
  \begin{proof}
    Let $\mathscr{T}$ be the Euclidean topology and $\mathscr{T}^\prime$ be the open ball topology. 
    \begin{enumerate}
      \item We show $\mathscr{T} \subset \mathscr{T}^\prime$. 
      \item We show $\mathscr{T}^\prime \subset \mathscr{T}$. 
    \end{enumerate}
  \end{proof} 

  By defining the topology, we have automatically defined a bunch of topological objects and properties. For clarification, we will restate them. 
  
  \begin{corollary}
    An open ball is an open set. 
  \end{corollary}
  \begin{proof}
    Given $x \in B_r (p)$, we can imagine that $x$ will always have some space between it and the boundary. We want to show that there exists some $\epsilon >0$ s.t. $B_\epsilon (x) \subset B_r (p)$. That is, given any point $y \in B_\epsilon (x)$, we can show that $y \in B_r (p)$. Since $||x - p|| < r$, there exists some space $0 < r - ||x - p||$. There always exists a real number $0 < \epsilon < r - ||x - p||$, so given $y \in B_\epsilon (x)$, we can bound
    \begin{equation}
      ||y - p|| = ||y - x + x - p|| \leq ||y - x|| + ||x - p|| \leq \epsilon + ||x - p|| < r
    \end{equation}
  \end{proof}

  \begin{example}
    Here are some examples of sets which are open and not open. 
    \begin{enumerate}
      \item $U=\{(x,y)\in \mathbb{R}^2 : x^2+y^2 \neq 1\}$ is open since for every point $x \in U$, we just need to find a radius $\epsilon >0$ that is smaller than its distance to the unit circle. 
      \item $(a, b) \times (c, d) \subset \mathbb{R}^2$ is open since given a point $x$, we can take the minimum of its distance between the two sides of the rectangle and construct an open ball. 
      \item $S=\{(x,y)\in \mathbb{R}^2:xy\neq 0\}$ is open since given a point $x \in S$, we can take the minimum of the distance between it and the $x$ and $y$ axes. 
      \item The set of all complex $z$ such that $|z| \leq 1$ is not open since we cannot construct open balls at the boundary points that are fully contained in the set. 
      \item The set $S = \{1/n\}_{n \in \mathbb{N}}$ is not open since given any point $x = 1/n$, we can construct an open ball with radius $\epsilon < 1/(n+1)$, which contains irrationals that are not in $S$. 
    \end{enumerate}
  \end{example}

  \begin{definition}[Interior Point]
    A point $p \in S$ is an \textbf{interior point} if there exists an neighborhood $N$ of $p$ such that $N \subset S$. 
  \end{definition}

  An interior point means that we can always contain the point in $S$ with some ``breathing room." By definition an open set is a set where all of its points are interior points. A set is then said to be open if every point has this breathing room. This can be useful when defining differentiation at a point within an open set, since we can always find a neighborhood to take limits on. 

  Now that we have defined the Euclidean topology, we will prove that the features of topological objects can be reduced to features in $\mathbb{R}^n$. 

  \begin{theorem}[Convexity]
    An open ball is convex in a normed vector space. 
  \end{theorem}
  \begin{proof}
    The normed part is important here, as the properties of the metric is not sufficient. Given $B_r (p)$, $x, y \in B_r (p)$ implies that $||x - p|| < r$ and $||y - p ||<r$. Therefore, 
    \begin{align}
      ||t x + (1 - t)y - p|| & = ||t x - tp + (1 - t) y - (1 - t) p|| \\
      & \leq t ||x - p|| + (1 - t) ||y - p|| \\
      & = t r + (1 - t) r = r 
    \end{align}
  \end{proof}

  What happens if we weaken it to a metric? 

\subsection{Limit Points and Closure} 

  \begin{definition}[Limit Point]
    A point $p \in \mathbb{R}^n$ is a \textbf{limit point} of $S \subset \mathbb{R}^n$ if every punctured neighborhood of $p$ has a nontrivial intersection with $X$.\footnote{The definition just means that if we take a point and draw smaller and smaller circles around it, the circle itself should still overlap with $S$, no matter how small it gets. } The set of all limit points of $S$ is denoted $S^\prime$. 
  \end{definition}

  \begin{theorem}
    Let $A_1, \ldots, A_n$ be a finite collection of sets. Then 
    \[\bigcup_{i=1}^n A_i^\prime = \bigg( \bigcup_{i=1}^n A_i \bigg)^\prime\]
  \end{theorem}
  \begin{proof}
    Let the LHS be $W$ and the RHS be $V$. If $x \in W$, $x \in A_i^\prime$ for some $i$, and so for all $\epsilon > 0$, there exists a $B_\epsilon^\circ (x)$ s.t. 
    \[B_\epsilon^\circ (x) \cap A_i \neq \emptyset \implies B_\epsilon^\circ (x) \cap \bigg( \bigcup_{i=1}^n A_i \bigg) \neq \emptyset\]
    which means that $x \in V$. Now assume that $x \in V$. Then for all $\epsilon > 0$, there exists a $B_\epsilon^\circ (x)$ s.t. 
    \[B_\epsilon^\circ (x) \cap \bigg( \bigcup_{i=1}^n A_i \bigg) \neq \emptyset\]
    which implies that $B_\epsilon^\circ (x) \cap A_i \neq \emptyset$ for some $i$, which means that $x \in A_i^\prime \subset W$. 
  \end{proof}

  A closed set can be defined in many equivalent ways for arbitrary topological spaces. The more general proof is done in topology, but we still prove it in the context of analysis. 

  \begin{definition}[Closed Set]
    A \textbf{closed set} $S \in \mathbb{R}^n$ is a set that contains all of its limit points. 
  \end{definition}

  \begin{theorem}[Alternative Definition of Closed Set]
    A set $S$ is closed iff $S^c$ is open. 
  \end{theorem}
  \begin{proof}
    We prove both ways: 
    \begin{enumerate}
      \item ($\rightarrow$) Given that $S$ is closed, then let $x \in S^c$. $x$ is not a limit point of $S$ since if it were, then it would be in $S$, and so there exists a punctured open neighborhood $B_\epsilon^\circ (x)$ of $x$ s.t. $S \cap B_\epsilon^\circ (x) = \emptyset$. Since $x \not\in S$, we also have $S \cap B_\epsilon (x) = \emptyset$, which implies that $B_\epsilon (x) \subset S^c$. Since for every $x \in S^c$, there exists a $B_\epsilon (x) \subset S^c$, $S^c$ is open. 

      \item ($\leftarrow$) For simplicity, it suffices to prove if $S$ open, then $S^c$ is closed. Given that $S$ is open, we have for every $x \in S$, there exists $B_\epsilon (x) \subset S$, which implies that $B_\epsilon (x) \cap S^c = \emptyset$. Since there exists an $B_\epsilon (x)$ that does not contain points in $S^c$, $x$ cannot be a limit point of $S^c$, and so there exists no limit points of $S^c$ in $S$. Therefore, all limit points of $S^c$ are in $S^c$, proving that $S^c$ is closed.  
    \end{enumerate}
  \end{proof}

  \begin{theorem}
  We have the following topological properties: 
  \begin{enumerate}
      \item For any collection $\{F_\alpha\}_\alpha$ of closed sets, $\cap_\alpha F_\alpha$ is closed. 
      \item For any finite collection $F_1, \ldots, F_n$ of open sets, $\cup_{i=1}^n F_i$ is closed. 
  \end{enumerate}
  \end{theorem}
  \begin{proof}
  Listed. 
  \begin{enumerate}
      \item Let $x$ be a limit point of $\cap_\alpha F_\alpha$, and we want to show that $x \in \cap_\alpha F_\alpha$. By definition of limit points, for every $\epsilon > 0$, we have 
      \[B_\epsilon (x) \cap \bigg( \bigcap_\alpha F_\alpha \bigg) \]
      which means that $B_\epsilon (x) \cap F_\alpha \neq \emptyset$ for all $\alpha$. This means that $x$ is a limit point for every $F_\alpha$, and since they are all closed, $x \in F_\alpha$ for all $\alpha$, which implies that $x \in \cap_\alpha F_\alpha$. 
  \end{enumerate}
  \end{proof}

  We can intuitively see a few properties about this. First, a finite set $S$ of points does not have any limit points, since if we draw small enough circles around a $p \in S$, then at some point the circle will not contain any more points (remember that we're talking about deleted neighborhoods). Following this, we can deduce that a limit point must always have an infinite number of points close to it, as in no matter how small the circle gets, there are always an infinite number of points contained within that circle. This also means that if $p$ is a limit point, then we can construct a sequence of points in $S$ that converges to $p$, since every open ball with smaller and smaller radii will still have points in $S$.

  \begin{theorem}
    If $p$ is a limit point of $S$, then every neighborhood of $p$ contains infinitely many points of $S$. The converse is also true trivially. 
  \end{theorem}
  \begin{proof}
    Assume $p$ is a limit point and that there exists a finite number of points within a deleted neighborhood $B_r^\circ (p)$. Then, we can enumerate them $p_1, p_2, \ldots, p_n$ by their distances to $p$, with 
    \begin{equation}
      d(p_1, p) \leq d(p_2, p) \leq \ldots \leq d(p_n, p)
    \end{equation}
    Since $p_1 \neq p$, we have $d(p_1, p) > 0$ and so, we can choose an $0 < \epsilon < d(p_1, p)$ s.t. $B_\epsilon^\circ (p)$ does not contain any of the $p_i$'s. This neighborhood does not contain any elements of $S$ and so $p$ is not a limit point. 
  \end{proof}

  \begin{corollary}
    A finite set has no limit points. 
  \end{corollary}
  \begin{proof}
    If $S$ is a finite set, then every neighborhood of every point $p$ in $\mathbb{R}^n$ will have at most finite points, which, by the previous theorem, is not a limit point. 
  \end{proof}

  We show a very useful result that will make things much more convenient when proving the following theorems and exercises. This is quite intuitive, since it shows that the limit points of a finite union of sets is the same as the finite union of the limit points of each set. This is clearly not true for infinite unions: 
  \begin{enumerate}
    \item Look at the countable set $\mathbb{Q} \subset \mathbb{R}$. Each $\{q\}^\prime = \emptyset$, but $\mathbb{Q}^\prime = \mathbb{R}$. 
    \item Look at the uncountable set $\mathbb{R}$. Each $\{x \in \mathbb{R}\}^\prime = \emptyset$, but $\mathbb{R}^\prime = \mathbb{R}$. 
  \end{enumerate}

  Now, we give two more definitions for convenience of deriving open and closed sets from any arbitrary set. 

  \begin{definition}[Closure]
    Given a set $S$, let the set of all limit points of $S$ be denoted $S^\prime$. The \textbf{closure} of $S$ is the set $\overline{S} = S \cup S^\prime$. It is the smallest closed set that contains $S$. 
  \end{definition}

  \begin{definition}[Interior]
    Given a set $S$, the \textbf{interior} of $S$ is denoted $S^\circ$, the set of all interior points of $S$. It is the largest open set that is within $S$. 
  \end{definition}

  \begin{theorem}
    Let $E$ be a nonempty set of real numbers which is bounded above. Let $y = \sup{E}$. Then $y \in \overline{E}$. Hence $y \in E$ if $E$ is closed. 
  \end{theorem}
  \begin{proof}
    Assume that $y$ is not a limit point of $E$. Then, there exists some $\epsilon > 0$ s.t. $(y - \epsilon, y + \epsilon)$ does not intersect with $E$. This means that $y - \epsilon$ is an upper bound of $E$, and so $y$ is not the supremum. 
  \end{proof}

  \begin{theorem}
    If $X$ is a metric space and $E \subset X$, then 
    \begin{enumerate}
      \item $\overline{E}$ is closed. 
      \item $E = \overline{E}$ if and only if $E$ is closed. 
      \item $\overline{E} \subset F$ for every closed set $F \subset X$ such that $E \subset F$. That is, if $E \subset F$ closed, then ``increasing" the size of $E$ to its closure will not make it greater than $F$. 
    \end{enumerate}
  \end{theorem}
  \begin{proof}
    Listed. 
    \begin{enumerate}
      \item Let $x$ be a limit point of $\overline{E}$. Then, for every $\epsilon > 0$, we have $B_\epsilon (x) \cap \overline{E} \neq \emptyset$, which means that either $B_\epsilon (x) \cap E \neq \emptyset$ (in which case $x \in E^\prime \implies x \in \overline{E}$ and we are done) or $B_\epsilon (x) \cap E^\prime \neq \emptyset$. We wish to prove that in the latter case, $x$ being a limit point of $E^\prime$ still implies that $x$ is a limit point of $E$. Since $B_\epsilon (x) \cap E^\prime \neq \emptyset$, there must exist a $y \in B_\epsilon (x) \cap E^\prime$. Since $y \in E^\prime$, we can construct an open ball $B_\delta (y)$ containing elements of $E$, and since $B_\epsilon (x)$ is open, we can contain $B_\delta (y)$ entirely within $B_\epsilon (x)$. Therefore, 
      \[B_\delta (y) \cap E \neq \emptyset \implies B_\epsilon (x) \cap E \neq \emptyset\]
      therefore, $x \in E^\prime \implies x \in \overline{E}$. 

      \item If $E$ is closed, then $E^\prime \subset E \implies \overline{E} = E \cup E^\prime = E$. If $E = \overline{E} = E \cup E^\prime$, then $E^\prime \subset E \implies E$ is closed. 

      \item Since $E \subset F$, it suffices to prove that $E^\prime \subset F$. Consider a limit point $x$ of $E$. Then every punctured open neighborhood of $x$ satisfies $B_\epsilon^\circ (x) \cap E \neq \emptyset$. But since $E \subset F$, we have 
      \[B_\epsilon^\circ (x) \cap F \neq \emptyset\]
      and so $x$ is also a limit point of $F$. But since $F$ is closed, $x \in F$. Therefore, $\overline{E} = E \cup E^\prime \subset F$. 
    \end{enumerate}
  \end{proof}

  The first two statements (1) and (2) imply the following. 

  \begin{corollary}
    The closure of the closure of $E$ is equal to the closure of $E$. 
  \end{corollary}
  \begin{proof}
    We know that $\overline{\overline{E}} \supset \overline{E}$, so we must prove that $\overline{\overline{E}} \subset \overline{E}$, which is equivalent to proving that $\overline{E}^\prime \subset \overline{E}$. Let $x \in \overline{E}^\prime$, i.e. is a limit point of $\overline{E}$. Then, for every $\epsilon > 0$, we have $B_\epsilon (x) \cap \overline{E} \neq \emptyset$. Pick a point $y$ from this intersection, and since $B_\epsilon (x)$ is open, we can construct an open ball $B_\delta (y)$ fully contained in $B_\epsilon (x)$. Since $y \in \overline{E}$, $y$ is a limit point of $E$, which implies 
    \begin{equation}
      B_\delta (y) \cap E \neq \emptyset \implies B_\epsilon (x) \cap E \neq \emptyset
    \end{equation}
    and therefore $x$ is a limit point of $E$, $x \in \overline{E}$. 
  \end{proof}

\subsection{Compactness}

  \begin{definition}[Open Cover]
    An \textbf{open cover} of a set $E$ in a metric space $X$ is a collection $\{G_\alpha\}$ of open subsets of $X$ such that $E \subset \cup_\alpha G_\alpha$. 
  \end{definition}

  \begin{definition}[Compact Set]
    A subset $S$ of a metric space $X$ is said to be \textbf{compact} if every open cover of $S$ contains a finite subcover. 
  \end{definition}

  While openness behaves differently depending on its embedding space, compactness stays constant. Therefore, we don't have to worry about talking about which space a compact set is embedded in. 

  \begin{theorem}
    Suppose $K \subseteq Y \subseteq X$. Then $K$ is compact relative to $X$ if and only if $K$ is compact relative to $Y$. 
  \end{theorem}
  \begin{proof}

  \end{proof}

  \begin{theorem}
    A finite union of compact sets is compact. 
  \end{theorem}
  \begin{proof}
    It suffices to prove for two sets $A, B$ by induction. Take an arbitrary cover $\mathscr{L}$ of $A \cup B$. Then $\mathscr{L}$ is a cover of $A$, so it has a finite subcover $\mathscr{F} \subset \mathscr{L}$. It is also a cover of $B$, so it has a finite subcover $\mathscr{G} \subset \mathscr{L}$. Therefore, $\mathscr{F} \cup \mathscr{G} \subset \mathscr{L}$ is a cover of $A \cup B$, and since it is the union of finite covers, it is finite. 
  \end{proof}

  \begin{theorem}
    Compact subsets of metric spaces are closed. 
  \end{theorem}
  \begin{proof}
    We would like to show that if $A$ is compact in $X$, then $A^c$ is open. What we would like to do is if we have some $x \in A^c$, then we must prove that there exists some open set $B_\epsilon (x)$ that is disjoint with $A$. For every point $a \in A$, we can construct an open balls $V_a = B_{d(x, a)/2} (a)$ and $U_a = B_{d(x, a)/2} (x)$. We know that if $y \in B_{d(x, a)/2}(a)$, then assuming $y \in B_{d(x, a)/2} (x)$ will give
    \begin{equation}
      d(x, a) \leq d(x, y) + d(y, a) < \frac{d(x, a)}{2} + \frac{d(x, a)}{2} = d(x, a)
    \end{equation}
    which is absurd. 
    Since $\{V_a\}_{a \in A}$ forms an open covering of $A$, then by compactness we can take a finite subcover $V_{a_1}, \ldots, V_{a_n}$, along with the respective neighborhoods of $x$ $U_{a_1}, \ldots, U_{a_n}$. Since we have established 
    \begin{equation}
      V_{a_i} \cap U_{a_i} = \emptyset \implies \bigcap_{i=1}^n V_{a_i} \cap \bigg( \bigcup_{i=1}^n U_{a_i} \bigg) = \emptyset
    \end{equation}
    and since $\cap_{i=1}^n V_{a_i}$ is open (as it is the intersection of open sets) and disjoint from an open cover of $A$ and hence from $A$, we have proved that $A^c$ is open, and so $A$ is closed. 
  \end{proof}

  The general notion of compactness\footnote{According to Terry Tao, a compact set is "small," in the sense that it is easy to deal with. While this may sound counterintuitive at first, since $[0,1]$ is considered compact while $(0,1)$, a subset of $[0,1]$, is considered noncompact. More generally, a set that is compact may be large in area and complicated, but the fact that it is compact means we can interact with it in a finite way using open sets, the building blocks of topology. That finite collection of open sets makes it possible to account for all the points in a set in a finite way. This is easily noticed, since functions defined over compact sets have more controlled behavior than those defined over noncompact sets. Similarly, classifying noncompact spaces are more difficult and less satisfying. } for topological spaces is not needed for analysis. Rather, we make use of the following theorem which allows us to focus on the compactness of subsets in Euclidean spaces $\mathbb{R}^n$. 

  \begin{theorem}[Heine-Borel]
    Let $E \subset \mathbb{R}^k$. The following are equivalent. 
    \begin{enumerate}
      \item $E$ is closed and bounded 
      \item $E$ is compact. 
      \item Every infinite subset of $E$ has a limit point in $E$. 
    \end{enumerate}
  \end{theorem}

  \begin{example}
    An open set in $\mathbb{R}^2$ is not compact. Take the open rectangle $ R = (0,1)^2 \subset \mathbb{R}^2$. There exists an infinite cover of $R$
    \[R = \bigcup_{n=0}^\infty \big(0,1\big) \times \bigg( 0, \frac{ 2^{n+1} - 1}{2^{n+1}} \bigg) \]
    that does not have a finite subcover. 
  \end{example}

  \begin{theorem}
    Closed subsets of compact sets are compact. 
  \end{theorem}
  \begin{proof}

  \end{proof}

  \begin{theorem}
  If $F$ is closed and $K$ is compact, then $F \cap K$ is compact. 
  \end{theorem}

  Clearly, the limit point of an open set is its boundary points. Note that a sequence of points can also have a limit point. 

  \begin{theorem}[Bolzano-Weierstrass Theorem]
    Every bounded infinite sequence in $\mathbb{R}^n$ has an accumulation point. That is, there exists a point $p \in \mathbb{R}^n$ such that every open neighborhood $U_p$ contains an infinite subset of the sequence. 
  \end{theorem}
  \begin{proof}
    The fact that the infinite sequence is bounded means that there exists some closed subset $I \in \mathbb{R}^n$ that contains all point of the sequence. By definition $I$ is compact, so by the Heine-Borel theorem, every cover of $I$ has a finite subcover. 

    Now, assume that there exists an infinite sequence in $I$ that is not convergent, i.e. has no limit point. Then, each point $x_i \in I$ would have a neighborhood $U(x_i)$ containing at most a finite number of points in the sequence. We can define $I$ such that the union of the neighborhoods is a cover of $I$. That is, 
    \[I \subset \bigcup_{i=1}^\infty U(x_i)\]
    However, since every $U(x_i)$ contains at most a finite number of points, we must have an infinite open neighborhoods to cover $I \implies$ we cannot have a finite subcover. This contradicts the fact that $I$ is compact. 
  \end{proof}

  In fact, compactness actually implies completeness. 

  \begin{theorem}
    Compact metric spaces are complete. 
  \end{theorem}

\subsection{Connectedness}

  \begin{definition}[Separate, Connected Sets]
    Two subsets $A$ and $B$ of a metric space $X$ are said to be \textbf{separated} if both $A \cap \overline{B}$ and $\overline{A} \cap B$ are empty, i.e. if no point of $A$ lies in the closure of $B$ and no point of $B$ lies in the closure of $A$. 
  \end{definition}

  \begin{example}
    It is clear that separate sets imply disjointness. However, this is not true for the other way around. 
    \begin{enumerate}
      \item $(0, 1)$ and $[1, 2)$ are disjoint but not separate. 
      \item The rationals and irrationals are disjoint, but not separate. 
    \end{enumerate}
  \end{example}

  \begin{theorem}
  A subset $E$ of the real line $\mathbb{R}$ is connected if and only if it has the following property: if $x \in E, y \in E$ and $x < z < y$, then $z \in E$. 
  \end{theorem}
  \begin{proof}

  \end{proof}

\subsection{Separability}

\subsection{Perfect Sets}

  \begin{definition}[Perfect Sets]
    A set $P$ is perfect if it is closed and all of its points are limit points of $P$. In other words, the limit points of $P$ and $P$ itself coincide. 
    \begin{equation}
      P^\prime = P
    \end{equation}
  \end{definition}

  \begin{theorem}
    Let $P$ be a nonempty perfect set in $\mathbb{R}^k$. Then $P$ is uncountable. 
  \end{theorem}

\subsection{Exercises}

  \begin{exercise}[Math 531 Spring 2025, PS3.1]
    Determine for each of the following sets, whether or not it is countable. Justify your answers
    \begin{enumerate}
      \item The set of all functions $f : \{0,1\} \to \mathbb{N}$.
      \item The set $B_n$ of all functions $f : \{1,...,n\} \to \mathbb{N}$
      \item The set $C = \cup_{n\in\mathbb{N}}B_n$
      \item The set of all functions $f : \mathbb{N} \to \{0,1\}$.
      \item The set of all functions $f : \mathbb{N} \to \{0,1\}$ that are ``eventually zero''
        (We say that $f$ is eventually zero if there exists some $N \geq 1$ so that
        $f(n) = 0$ for all $n \geq N$.)
      \item $G$ the set of all functions $f : \mathbb{N} \to \mathbb{N}$ that are eventually constant.
    \end{enumerate}
  \end{exercise} 
  \begin{solution}
    Listed. 
    \begin{enumerate}
      \item Countable since bijective to $\mathbb{N} \times \mathbb{N}$. We define the bijection as 
      \begin{equation}
        (a_0, a_1) \in \mathbb{N} \times \mathbb{N} \mapsto f(i) = a_i 
      \end{equation}

      \item Countable since bijective to $\mathbb{N}^n$. We define the bijection as 
      \begin{equation}
        (a_1, \ldots, a_n) \in \mathbb{N}^n \mapsto f(i) = a_i
      \end{equation}

      \item Countable since we proved that $B_n$ is countable, and a countable union of countable sets are countable from Rudin Theorem 2.12. 

      \item Uncountable since we can create a bijection from the set of all sequences $(a_i)$ of $0$ or $1$, which from Rudin Theorem 2.14 is uncountable. 
      \begin{equation}
        (a_i)_{i \in \mathbb{N}} \mapsto f(i) = a_i
      \end{equation}

      \item Countable. Call this set $B$, and call the set of functions $f$ that have their final $1$ at index $k$ to be $A_k$. Then, 
      \begin{equation}
        B = \cup_{k=1}^\infty A_k 
      \end{equation}
      where $A_0 = A_1 = 1$, and $|A_k| = 2^{k-1}$ for $k \geq 2$. Since $B$ is the countable union of at most countable sets, $B$ must be countable. 

      \item Countable. Call this set $B$. Let $A_k$ be the set of functions that are eventually constant to value $k$. Let $A_{ki}$ be the set of functions that are always $k$ starting from index $i$ (where $i$ is the smallest element). Since everything is determined to be $k$ at $i$ and beyond, $A_{ki}$ can be divided up into the first $i-1$ elements of any natural number, followed by a sequence of $k$'s. Therefore $|A_{ki}| \approx \mathbb{N}^{i-1}$, where $\approx$ means equipotent, and so $A_{ki}$ is countable. Therefore since countable unions of countable sets are countable, 
      \begin{equation}
        A_k = \bigcup_{i=1}^\infty A_{ki} \text{ is countable } \implies B = \sum_{k=1}^\infty A_k \text{ is countable}
      \end{equation}
    \end{enumerate}
  \end{solution}

  \begin{exercise}[Math 531 Spring 2025, PS3.2]
    Tell if the following subsets $A \subset \mathbb{R}$ (with the usual metric $d(x,y) = |x-y|$)
    are open or closed. Also, find $(i)$ the limit points of $A$, $(ii)$ the interior of
    $A$, $(iii)$ $\bar{A}$.
    \begin{enumerate}
      \item $A = \mathbb{Q}$
      \item $A = (0,1]$
      \item $A = \{1, \frac{1}{2}, \frac{1}{4}, ...\}$
      \item $A = \{0, 1, \frac{1}{2}, \frac{1}{4}, ...\}$
      \item $A = \mathbb{Z}$
    \end{enumerate}
  \end{exercise} 
  \begin{solution}
    Listed. We denote $A^\prime$ as the limit points of $A$ and the interior as $A^o$. 
    \begin{enumerate}
      \item Not open nor closed. $A^\prime = \mathbb{R}$, $A^o = \emptyset$. $\bar{A} = \mathbb{R}$. 
      \item Not open nor closed. $A^\prime = [0, 1]$, $A^o = (0, 1)$. $\bar{A} = [0, 1]$. 
      \item Not open nor closed. $A^\prime = \{0\}$, $A^o = \emptyset$. $\bar{A} = \{ 0 \} \cup A$. 
      \item Closed. $A^\prime = \{0\}$, $A^o = \emptyset$. $\bar{A} = A$. 
      \item Closed. $A^\prime = \emptyset$. $A^o = \emptyset$. $\bar{A} = A$. 
    \end{enumerate}
  \end{solution}
  
  \begin{exercise}[Math 531 Spring 2025, PS3.3]
    Prove the following statements subsets $A,B$ of a general metric space $(X,d)$.
    \begin{itemize}
      \item $\overline{A \cup B} = \bar{A} \cup \bar{B}$.
      \item Show by example that $\overline{A \cap B} \neq \bar{A} \cap \bar{B}$.
    \end{itemize} 
  \end{exercise} 
  \begin{solution}
    For the first part, we show bidirectionally. 
    \begin{enumerate}
      \item $\overline{A \cup B} \subset \overline{A} \cup \overline{B}$. Let $x \in \overline{A \cup B}$. If $x \in A \cup B$, then it must be the case that either $x \in A \subset (A \cup A^\prime) = \overline{A}$ or $x \in B \subset (B \cup B^\prime) = \overline{B}$, which means $x \in \overline{A} \cup \overline{B}$. Now assume not. Then $x \in (A \cup B)^\prime$. Therefore, for any $r > 0$, we know that $B(x, r) \cap (A \cup B) \neq \emptyset$. Now let us take a sequence $(r_n = \frac{1}{n})_{n \in \mathbb{N}}$, and for each $r_n$ we have some element $x_n \in (A \cup B)$. Given that we have a countably infinite sequence of $x_n$, each which may be in $A$ or $B$, by the pigeonhole principle either $A$ or $B$ must be hit infinitely many times. If $x_n \in A$ infinitely many times, then $x \in \overline{A}$, and analogous for $B$. 
      \item $\overline{A} \cup \overline{B} \subset \overline{A \cup B}$. WLOG let $x \in \overline{A}$. If $x \in A$, then $x \in (A \cup B) \subset \overline{A \cup B}$. If $x \not\in A$, then $x \in A^\prime$. Therefore for every $r > 0$, $B(x, r) \cap A \neq \emptyset$. But this means 
      \begin{equation}
        \emptyset \neq (B(x, r) \cap A) \cup (B(x, r) \cap B) = B(x, r) \cap (A \cup B) \implies x \in (A \cup B)^\prime \subset \overline{A \cup B}
      \end{equation}
    \end{enumerate}
    For a counterexample, consider the sequences 
    \begin{equation}
      A = (x_n) = \frac{1}{n} \qquad B = (y_n) = -\frac{1}{n}
    \end{equation}
    for $n \in \mathbb{N}$. $\overline{A} = A \cup \{0\}, \overline{B} = B \cup \{0\}$, and so $\overline{A} \cap \overline{B} = \{0\}$. However, $A \cap B = \emptyset \implies \overline{A \cap B} = \emptyset$. 
  \end{solution}

  \begin{exercise}[Math 531 Spring 2025, PS3.4]
    Consider the set of rationals in canonical form (such that numerator and denominator are relatively prime) with potential distance:
    \begin{equation}
      d_1(\frac{p_1}{q_1}, \frac{p_2}{q_2}) = |q_1 - q_2|.
    \end{equation}
    Is this a metric? Prove that the following defines a metric
    \begin{equation}
      d_2(\frac{p_1}{q_1}, \frac{p_2}{q_2}) = |p_1 - p_2| + |q_1 - q_2|.
    \end{equation}
  \end{exercise} 
  \begin{solution}
    This is not a metric since 
    \begin{equation}
      d_1 \bigg( \frac{2}{1}, \frac{3}{1} \bigg) = 1 - 1 = 0
    \end{equation} 
    when $2/1 \neq 3/1$. For $d_2$, we show that it satisfies the three properties. 
    \begin{enumerate}
      \item \textit{Nonnegativity}. Since it is the sum of 2 absolute values which are norms and therefore nonnegative, it must be nonnegative by ordered field properties. We see that 
      \begin{align}
        \frac{p_1}{q_1} = \frac{p_2}{q_2} & \iff p_1 = p_2 \text{ and } q_1 = q_2 \\
                                          & \iff |p_1 - p_2| = |q_1 - q_2| = 0 \\
                                          & \iff |p_1 - p_2| + |q_1 - q_2| = 0 
      \end{align}

      \item For symmetricity, note that 
        \begin{equation}
          d_2 \bigg( \frac{p_1}{q_1} , \frac{p_2}{q_2} \bigg) = |p_1 - p_2| + |q_1 - q_2| = |p_2 - p_1| + |q_2 - q_1| = d_2 \bigg( \frac{p_1}{q_1} , \frac{p_2}{q_2} \bigg)
        \end{equation}

        \item For triangle inequality, we see that for any $p_1/q_1, p_2/q_2, p_3/q_3$, 
        \begin{align}
          d_2 \bigg( \frac{p_1}{q_1}, \frac{p_3}{q_3} \bigg) & = |p_1 - p_3| + |q_1 - q_3| \\
                                                             & = |(p_1 - p_2) + (p_2 - p_3)| + |(q_1 - q_2) + (q_2 - q_3)| \\
                                                             & \leq |p_1 - p_2| + |p_2 - p_3| + |q_1 - q_2| + |q_2 - q_3| && \tag{subadditivity of norm} \\
                                                             & = d_2 \bigg( \frac{p_1}{q_1}, \frac{p_2}{q_2} \bigg) + d_2 \bigg( \frac{p_2}{q_2}, \frac{p_3}{q_3} \bigg) 
        \end{align}
    \end{enumerate}
  \end{solution}

  \begin{exercise}[Math 531 Spring 2025, PS3.5]
    Let $M = \{x_1,..., x_3\}$ be a set with three points. Describe the set of all metrics on $M$. What if $M$ has four points?
  \end{exercise} 
  \begin{solution}
    If $M$ has 3 points call them $x_1, x_2, x_3$, then the metric is completely defined by the three values 
    \begin{align}
      d(x_1, x_2) & = d(x_2, x_1) \\
      d(x_2, x_3) & = d(x_3, x_2) \\
      d(x_3, x_1) & = d(x_3, x_1) 
    \end{align}
    where $d(x, x) = 0$. We must make sure that the triangle inequality satisfies for these 3 numbers. Therefore we can think of this as the set of all triangles in $\mathbb{R}^2$ (that are equivalent under translation and rotation, but not permutation of points). 

    Similarly for 4 points, we can visualize the metrics as the set of all tetarhedra in $\mathbb{R}^3$ (since each face is a triangle, and therefore for any three points the triangle inequality is guaranteed to be satisfied), equivalent under translation and rotation, but not permutation of the 4 points. 
  \end{solution}

  \begin{exercise}[Math 531 Spring 2025, PS3.6]
    Let $P$ be a polynomial of degree $n \geq 1$. Prove that if $P(0) = 0$, then $P(x) = xQ(x)$, for some polynomial $Q$ of degree $n-1$. Deduce that if $P(a) = 0$, then we can write $P(x) = (x-a)Q(x)$ for some $Q$ of degree $n-1$.
  \end{exercise}
  \begin{solution}
    A $n$th degree polynomial will have the form 
    \begin{equation}
      p(x) = \sum_{i=0}^n c_i x^i
    \end{equation}
    Since $p(0) = c_0 = 0 \implies c_0 = 0$. This means that 
    \begin{equation}
      p(x) = \sum_{i=1}^n c_i x^i = x \sum_{i=0}^{n-1} c_{i+1} x^i \text{ where } Q(x) = \sum_{i=0}^{n-1} c_{i+1} x^i
    \end{equation} 
    If $p(a) = 0$, we can construct $f(x) = p(x + a)$, where $f$ is a polynomial since the expansion does not increase its degree. Since $f(x) = p(a) = 0$, by above $f$ can be factorized $f(x) = x g(x)$ for some $(n-1)$th degree polynomial $g$, and by substitution this means that $p(x) = f(x - a) = (x - a) g(x - a)$. 
  \end{solution}

  \begin{exercise}[Math 531 Spring 2025, PS3.7]
    Consider all polynomials $P : \mathbb{R} \to \mathbb{R}$ of degree less than or equal to $n$. Call this set $\mathcal{P}_n$. Let's define potential distances on $\mathcal{P}_n$.
    \begin{equation}
      d_1(p,q) = |p(0) - q(0)|.
    \end{equation}
    Show this defines a distance on $\mathcal{P}_0$ but not on $\mathcal{P}_n$ for $n \geq 1$. Now consider
    \begin{equation}
      d_N(p,q) = \sum_{j=0}^N |p(j) - q(j)|
    \end{equation}
    Show that this defines a distance on $\mathcal{P}_n$, for every $n \leq N$. What does the solution say about polynomials of degree $N$?
  \end{exercise}
  \begin{solution}
    If $n = 0$, $\mathcal{P}_n$ is a set of constant functions $P$, where each constant function $P$ is determined completely by its value at any point, e.g. $0$. We check the properties. 
    \begin{enumerate}
      \item $d_1 (p, q) \geq 0$ since we take the norm at the end. We can see that 
      \begin{align}
        d_1 (p, q) = 0 & \iff |p(0) - q(0)| \\
                       & \iff p(0) = q(0) \\
                       & \iff p = q
      \end{align} 

      \item It is clearly symmetric. 
      \begin{equation}
        d_1 (p, q) = |p(0) - q(0)| = |q(0) - p(0)| = d_1(q, p)
      \end{equation}

      \item It satisfies the triangle inequality by subadditivity of the norm. 
      \begin{align}
        d_1 (p, r) & = |p(0) - r(0)| \\
                   & = |(p(0) - q(0)) + (q(0) - r(0))| \\
                   & \leq |p(0) - q(0)| + |q(0) - r(0)| \\
                   & = d_1 (p, q) + d_1 (q, r)
      \end{align}
    \end{enumerate}
    It doesn't satisfy for $P_n$ because consider $p(x) = x$ and $q(x) = x^2$. They are not the same function but $d_1 (p, q) = |p(0) - q(0)| = 0$. For $d_N$ defined on $\mathcal{P}_n$ for $n \leq N$, we verify the properties. 
    \begin{enumerate}
      \item This is the sum of norms, so it must be nonnegative. Now we see that if $p = q$, then $p(x) = q(x) \implies |p(x) - q(x)| = 0 \implies d_N (p, q) = 0$. For the other way around, suppose $d_N(p, q) = 0$. Then from problem 3.8, we are solving the linear equation $0 = V b - V c$, where $b, c$ are the vectors representing the coefficients of $p, q$, and $V$ is the Vandermonde matrix with $a_i = i$. By linearity, this is equivalent to solving $0 = V(b - c)$, and since we showed that $V$ is invertible (since $a_i$'s are distinct), $V$ has a trivial kernel and therefore $b - c = 0 \iff b = c \implies p = q$. 

      \item Symmetricity is trivial. 
      \begin{equation}
        d_N(p,q) = \sum_{j=0}^N |p(j) - q(j)| = \sum_{j=0}^N |q(j) - p(j)| = d_N (q, p)
      \end{equation} 

      \item For triangle inequality, 
      \begin{align}
        d_N (p, r) & = \sum_{j=0}^N |p(j) - r(j)| \\
                   & = \sum_{j=0}^N |(p(j) - q(j)) + (q(j) - r(j))| \\  
                   & \leq \sum_{j=0}^N |p(j) - q(j)| + |q(j) - r(j)| \\
                   & = \sum_{j=0}^N |p(j) - q(j)| + \sum_{j=0}^N |q(j) - r(j)| \\
                   & = d_N (p, q) + d_N (q, r)
      \end{align}
    \end{enumerate}
    This shows that we need to ``sample'' more points from higher-degree polynomials to get the metric as they are higher-dimensional. 
  \end{solution}

  \begin{exercise}[Math 531 Spring 2025, PS3.8]
    Given distinct numbers $a_0,...,a_N$ and numbers $b_0,...,b_N$, prove that there exists a polynomial $P$ of degree $N$ with the property that
    \begin{equation}
      P(a_i) = b_i,
    \end{equation}
    for $0 \leq i \leq N$. The most direct way to solve this problem, in my view, is to write the system equations you are trying to solve as a linear system for the coefficients of $P$. This will give you some matrix $M$ that depends on the numbers $a_0,...,a_N$. The key is to show that the determinant of this matrix is non-zero. It turns out that the determinant of this matrix is equal to
    \begin{equation}
      \prod_{0\leq i<j\leq N}(a_i - a_j),
    \end{equation}
    up to a potential $-$ sign depending on how you defined $M$. Prove this and deduce the result.
  \end{exercise}
  \begin{solution}
    We can write the system of equations using the Vandermonde matrix $V \in \mathbb{R}^{(N+1) \times (N+1)}$ and $c$ is the vector of coefficients of $P$. 
    \begin{equation}
      b = V c \iff \begin{bmatrix}
        b_0 \\ b_1 \\ \vdots \\ b_N
      \end{bmatrix} = \begin{bmatrix}
        1 & a_0 & a_0^2 & \ldots & a_0^N \\ 
        1 & a_1 & a_1^2 & \ldots & a_1^N \\ 
        \vdots & \vdots & \ddots & \vdots \\
        1 & a_N & a_N^2 & \ldots & a_N^N 
      \end{bmatrix} \begin{bmatrix}
        c_0 \\ c_1 \\ \vdots \\ c_N
      \end{bmatrix}
    \end{equation} 
    To calculate the determinant of $V$, we prove using induction. Clearly for $N = 1$ we have 
    \begin{equation}
      \det \begin{pmatrix}
        1 & a_0 \\ 1 & a_1 
      \end{pmatrix} = a_1 - a_0
    \end{equation}
    Now assume that this formula holds for some $N-1 \in \mathbb{N}$. Then for $N$, we can take $V$ and subtract $a_0$ times the $i$th column from the $(i+1)$st column. This gives us
    \begin{equation}
      V = 
      \begin{bmatrix}
        1 & 0 & 0 & \ldots & 0 \\
        1 & a_1-a_0 & a_1^2-a_0a_1 & \ldots & a_1^N-a_0a_1^{N-1} \\
        1 & a_2-a_0 & a_2^2-a_0a_2 & \ldots & a_2^N-a_0a_2^{N-1} \\
        \vdots & \vdots & \vdots & \ddots & \vdots \\
        1 & a_N-a_0 & a_N^2-a_0a_N & \ldots & a_N^N-a_0a_N^{N-1}
      \end{bmatrix}
    \end{equation}
    When calculating the determinant, we can perform the cofactor expansion by the first row, and then for each $i$th row factor out $(a_i - a_0)$ to get 
    \begin{equation}
      \det V = \prod_{j=1}^{N} (a_j - a_0) \det
      \begin{bmatrix}
        1 & a_1 & \ldots & a_1^{N-1} \\
        1 & a_2 & \ldots & a_2^{N-1} \\
        \vdots & \vdots & \ddots & \vdots \\
        1 & a_N & \ldots & a_N^{N-1}
      \end{bmatrix}
    \end{equation}
    which is the $(N-1) \times (N-1)$ Vandermonde matrix. Therefore, we can apply our inductive hypothesis to get 
    \begin{equation}
      \det V = \prod_{j=1}^N (a_j - a_0) \prod_{1 \leq i < j \leq N} (a_j - a_i) = \prod_{0 \leq i \leq j \leq N} (a_j - a_i)
    \end{equation}
    Note that this has a $0$ determinant iff $a_i = a_j$ for some $i \neq j$. Therefore sicne $a_i$'s are distinct, it must be nonzero. Therefore, this matrix is nonsingular, i.e. invertible, and we can solve the matrix equation to get 
    \begin{equation}
      c = V^{-1} b
    \end{equation}
    which from linear algebra is guaranteed to exist and is unique. 
  \end{solution}
  
  \begin{exercise}[Rudin 2.1]
    Prove that the empty set is a subset of every set. 
  \end{exercise}
  \begin{solution}
    It must suffice that if $x \in \emptyset$, then $x \in A$ for any arbitrary set $A$. This is vacuously true, since the initial condition is never met. 
  \end{solution}

  \begin{exercise}
    Show that the empty function $f: \emptyset \rightarrow X$, where $X$ is an arbitrary set, is always injective. If $X = \emptyset$, then $f$ is bijective. 
  \end{exercise}
  \begin{solution}
    Given distinct $x, y \in \emptyset$, $f(x) \neq f(y)$ is vacuously true, but if $X \neq \emptyset$, then there exists a $w \in X$ with no preimage. If $X = \emptyset$, then the statement for all $w \in X$, there exists an $x \in \emptyset$ s.t. $f(x) = w$ is vacuously true. 
  \end{solution}

  \begin{exercise}[Rudin 2.2]
    A complex number $z$ is said to be algebraic if there are integers $a_0, a_1, \ldots, a_n$, not all zero, such that
    \begin{equation}
      a_0z^n + a_1z^{n-1} + \ldots + a_{n-1}z + a_n = 0.
    \end{equation}
    Prove that the set of all algebraic complex numbers is countable. Hint: For every positive integer $N$ there are only finitely many equations with 
    \begin{equation}
      n + |a_0| + |a_1| + \ldots + |a_n| = N
    \end{equation}
  \end{exercise}
  \begin{solution}
    Consider all polynomials s.t. $n + \sum_{i=0}^n |a_i| = N$. There is only a finite number of them, and each polynomial has at most $n$ distinct complex roots. So this set is finite, an unioning over all $N \in \mathbb{N}$ gives an at most countable set of roots. 
  \end{solution}

  \begin{exercise}[Rudin 2.3]
    Prove there exists real numbers which are not algebraic. 
  \end{exercise}
  \begin{solution}
    From the previous exercise, if there were no no real numbers which are not algebraic, then every real number is algebraic. This contradicts the fact that the set of all complex numbers is countable. 
  \end{solution}

  \begin{exercise}[Rudin 2.4]
    Is the set of all irrational real numbers countable? 
  \end{exercise}
  \begin{solution}
    No. Assume that it is countable. We have $\mathbb{Q}$ countable. Then, by assumption, we must have $\mathbb{R} = \mathbb{Q} \cup \mathbb{Q}^c$ be the union of countable sets, which must be countable, contradicting the fact that it is uncountable. 
  \end{solution}

  \begin{exercise}[Rudin 2.5]
    Construct a bounded set of real numbers which exactly 3 limit points. 
  \end{exercise}
  \begin{solution}
    We can construct the union of 3 sequences that converge onto the limit points $0, 1, 2$. 
    \begin{equation}
      \big\{ \frac{1}{n} \big\}_{n \in \mathbb{N}} \cup \big\{ \frac{1}{n} + 1\}_{n \in \mathbb{N}} \cup \big\{ \frac{1}{n} + 2 \big\}_{n \in \mathbb{N}}
    \end{equation}
  \end{solution}

  \begin{exercise}
    Prove that the union of the limit points of sets is equal to the limit points of the union of the sets. 
    \begin{equation}
      \bigcup_{k=1}^m A_k'=\left(\bigcup_{k=1}^m A_k\right)^{\!\prime}
    \end{equation}
  \end{exercise}

  \begin{exercise}[Rudin 2.6]
    Let $E^\prime$ be the set of all limit points of a set $E$. Prove that $E^\prime$ is closed. Prove that $E$ and $\overline{E}$ have the same limit points. (Recall that $\overline{E} = E \cup E^\prime$). Do $E$ and $E^\prime$ always have the same limit points? 
  \end{exercise}
  \begin{solution}
    Listed. 
    \begin{enumerate}
        \item Let $x$ be a limit point of $E^\prime$. Then, for every $\epsilon > 0$, $U = B_\epsilon (x) \cap E^\prime \neq \emptyset$. Take a $y \in U$. Since $y \in B_\epsilon (x)$, which is open, we can construct an open ball $B_\delta (y) \subset B_\epsilon (x)$. Since $y \in E^\prime$, $B_\delta (y)$ must contain elements of $E$, which means that $B_\epsilon (x)$ must also contain elements of $E$, and so $x$ is a limit point of $E \implies x \in E^\prime$ and $E^\prime$ is closed. 

        \item To prove that $E^\prime \subset \overline{E}^\prime$, we know that if $x \in E^\prime$, then for every $\epsilon > 0$, there exists a $B_\epsilon ^\circ (x)$ that has a nontrivial intersection with $E$ which means that it has a nontrivial intersection with $E \cup E^\prime$. To prove that $\overline{E}^\prime \subset E^\prime$, we know that if $y \in \overline{E}^\prime$, then for every $\delta > 0$ there exists a $B_\delta (x)$ that has a nontrivial intersection with $\overline{E}$. If $B_\delta (x)$ intersects $E$ then we are done. If $B_\delta (x)$ intersects $E^\prime$, then we can find a $y \in E^\prime \cap B_\delta (x)$. Since $B_\delta (x)$ is open, we can construct $B_\varepsilon (y) \subset B_\delta (x)$ and since $y \in E^\prime$, we know that $B_\varepsilon (y)$ contains an element of $E$, which means that $B_\delta (x)$ contains an element of $E$. Therefore, $E^\prime = \overline{E}^\prime$. 

        \item No. Consider the set $E = \{1/n\}_{n \in \mathbb{N}}$. $E^\prime = \{0\}$, but $E^{\prime\prime} = \emptyset$. 
    \end{enumerate}
  \end{solution}

  \begin{exercise}[Rudin 2.7]
    Let $A_1, A_2, \ldots$ be subsets of a metric space. 
    \begin{enumerate}
      \item If $B_n = \cup_{i=1}^n A_i$, prove that $\bar{B}_n = \cup_{i=1}^n \bar{A}_i$ for $n = 1, 2, 3, \ldots$ 
      \item If $B = \cup_{i=1}^\infty A_i$, prove that $\bar{B} \supset \cup_{i=1}^\infty \bar{A}_i$. 
    \end{enumerate}
  \end{exercise}
  \begin{solution}
    Listed. 
    \begin{enumerate}
        \item We will prove that $\overline{B_n} \subseteq \cup_{i=1}^n \overline{A_i}$ and $\cup_{i=1}^n \overline{A_i} \subseteq \overline{B_n}$. If $x \in B_n$, then $x \in \cup_{i=1}^n A_i$. Therefore, assume that $x \in B_n^\prime$. Then for every $\epsilon > 0$, there exists a $B_\epsilon^\circ (x)$ s.t. 
        \[B_\epsilon^\circ (x) \cap B_n \neq \emptyset \implies B_\epsilon^\circ (x) \cap \bigg( \bigcup_{i=1}^n A_n \bigg) \neq \emptyset \]
        This means that there exists some $i = i(\epsilon)$, a function of $\epsilon$, s.t. $B_\epsilon^\circ (x) \cap A_i \neq \emptyset$. However, this $i$ may change if we unfix $\epsilon$. We have so far proved that just for one $\epsilon > 0$ there exists an $i$. Now if we take a sequence of $\epsilon = 1, \frac{1}{2}, \frac{1}{3}, \ldots$, we have a sequence of $i(\epsilon)$'s living in $\{1, \ldots, n\}$. By the pigeonhole principle, there must be at least one $i$ that is hit infinitely many times, and so we can choose this $i$, that works for all $\epsilon > 0 \implies x \in A_i^\prime \subseteq \cup_{i=1}^n \overline{A_i}$. If $x \in \cup_{i=1}^n \overline{A_i}$, then there exists an $\overline{A_i}$ s.t. $x \in \overline{A_i}$. If $x \in A_i$, then we are done. If $x \in A_i^\prime$, then for every $\epsilon > 0$, there exists a $B_\epsilon^\circ (x)$ s.t. 
        \[B_\epsilon^\circ (x) \cap A_i \neq \emptyset \implies B_\epsilon^\circ (x) \cap \bigg( \bigcup_{i=1}^n A_i \bigg) \neq \emptyset\]
        and so $x \in B_n^\prime \subset \overline{B_n}$. 

        \item $x \in \cup_{i=1}^\infty \overline{A_i} \implies x \in \overline{A_i}$ for some $i$. If $x \in A_i$, then $x \in B$ and we are done. If $x \in A_i^\prime$, then for every $\epsilon > 0$ there exists $B_\epsilon (x) $ s.t. 
        \[B_\epsilon^\circ (x) \cap A_i \neq \emptyset \implies B_\epsilon^\circ (x) \cap \bigg( \bigcup_{i=1}^\infty A_i \bigg) \neq \emptyset\] 
        and so $B_\epsilon^\circ (x) \cap B \neq \emptyset \implies x \in B^\prime \subset \overline{B}$. 
    \end{enumerate}
  \end{solution}

  \begin{exercise}[Rudin 2.8]
    Is every point of every open set $E \subset \mathbb{R}^2$ a limit point of $E$? Answer the same question for closed sets in $\mathbb{R}^2$. 
  \end{exercise}
  \begin{solution}
    Yes for open. Given any $x \in U$ open, there always exists an $\epsilon > 0$ s.t. 
    \begin{equation}
      B_\epsilon^\circ (x) \subset B_\epsilon (x) \subset U
    \end{equation}
    and so $B_\epsilon^\circ (x)$ has a nontrivial intersection with $U$. If $U$ is closed, then no. Note that for closed $U$, we have that every limit point is in $U$, but not every point in $U$ is a limit point. Consider the isolated point $U = \{x\}$. $x$ is not a limit point of $U$. 
  \end{solution}

  \begin{exercise}[Rudin 2.9]
    Let $E^\circ$ denote the set of all interior points of $E$ in $X$. Prove the following:
    \begin{enumerate}
      \item[(a)] $E^\circ$ is always open.
      \item[(b)] $E$ is open if and only if $E^\circ = E$.
      \item[(c)] If $G \subseteq E$ and $G$ is open, then $G \subset E^\circ$.
      \item[(d)] Prove that the complement of $E^\circ$ is the closure of the complement of $E$. 
      \item[(e)] Do $E$ and $\bar{E}$ always have the same interiors? 
      \item[(f)] Do $E$ and $E^\circ$ always have the same closures? 
    \end{enumerate}
  \end{exercise}
  \begin{solution}
    Listed. 
    \begin{enumerate}
        \item We assume that $E^\circ$ is not open (this does not mean that $E^\circ$ is necessarily closed!). That is, there exists an $x \in E^\circ $ s.t. we can't construct an open ball $B_\epsilon (x) \subseteq E^\circ$. Since $x \in E^\circ \subset E$, by definition of an interior point we can construct a $B_\epsilon (x) \subset E$. But from our assumption $B_\epsilon (x) \not\subset E^\circ$. We choose a $y \in B_\epsilon (x) \setminus E^\circ$. Since $B_\epsilon (x)$ is open, there exists a $\delta > 0$ s.t. 
        \[B_\delta (y) \subset B_\epsilon (x) \subset E\]
        But the fact that we can construct an open ball around $y$ means that $y \in E^\circ$, leading to a contradiction. 

        \item If $E$ is open, then by definition $E \subset E^\circ$. Now $E^\circ \subset E$ holds for all sets since $E^\circ$ must be composed of points from $E$. If $E = E^\circ$, then for every $x \in E$, $x \in E^\circ$, so by definition there exists an $\epsilon > 0$ s.t. $B_\epsilon (x) \subset E$, which means that $E$ is open. 

        \item Let $x \in G$ open. Then there exists an $\epsilon > 0$ s.t. $B_\epsilon (x) \subset G$, and so $B_\epsilon (x) \subset E$. Since we can always construct an open ball around $x$ contained within $E$, $x \in E^\circ$ and $G \subset E^\circ$. 

        \item ($(E^\circ)^c \subset \overline{E^c}$) If $x \in (E^\circ)^c$, then there exists no $\epsilon > 0$ s.t. $B_\epsilon (x) \subset E$. Then, for any $\epsilon > 0$, $B_\epsilon (x) \not\subset E \implies B_\epsilon (x) \cap E^c \neq \emptyset \implies x \in E^c \subset \overline{E^c}$. ($\overline{E^c} \subset (E^\circ)^c)$) If $x \in \overline{E^c}$, then $x \in E^c$ or $x \in E^{c \prime}$. If $x \in E^c$, note $E^\circ \subset E \implies (E^\circ)^c \supset E^c \implies x \in (E^\circ)^c$. If $x \in E^{c \prime}$, then for all $\epsilon > 0$ $B_\epsilon (x) \cap E^c \neq \emptyset \implies B_\epsilon (x) \not\subset E \implies x \in E^\circ$. 

        \item No. Consider the rationals $\mathbb{Q} \subset \mathbb{R}$. $\mathbb{Q}^\circ = \emptyset$ but $\overline{\mathbb{Q}}^\circ = \mathbb{R}^\circ = \mathbb{R}$. It is true and straightforward to prove that $E^\circ \subset \overline{E}^\circ$.  Let $x \in E^\circ$. Then there exists an $\epsilon > 0$ s.t. $B_\epsilon(x) \subset E \implies B_\epsilon (x) \subset \overline{E} \implies x \in \overline{E}^\circ$. 

        \item No. Consider $\mathbb{Q} \subset \mathbb{R}$. Then $\overline{\mathbb{Q}} = \mathbb{R}$ and $\overline{\mathbb{Q}^\circ} = \overline{\emptyset} = \emptyset$.  
    \end{enumerate}
  \end{solution}

  \begin{exercise}[Rudin 2.10]
    Let $X$ be an infinite set. For $p \in X$ and $q \in X$, define 
    \begin{equation}
      d(p, q) = \begin{cases} 1 & \text{ if } p \neq q \\ 0 & \text{ if } p = q \end{cases}
    \end{equation}
    Prove that this is a metric. Which subsets of the resulting metric space are open? Which are closed? Which are compact? 
  \end{exercise}
  \begin{solution}
    This is a metric since clearly it satisfies symmetry and the fact that $d(p, p) = 0$. The triangle inequality 
    \begin{equation}
      d(p, r) \leq d(p, q) + d(q, r)
    \end{equation}
    is trivially satisfied if $p = r$, and if $p \neq r$, then either $p \neq q$ or $q \neq r$, and so the RHS $\geq 1$. An open $\epsilon$-ball around $x \in X$ is either $X$, when $\epsilon > 1$, or $\{x\}$ when $\epsilon \leq 1$. Therefore 
  \end{solution}

  \begin{exercise}[Rudin 2.11]
    For $x \in \mathbb{R}$ and $y \in \mathbb{R}$, define 
    \begin{align*}
        d_1 (x, y) & = (x - y)^2 \\ 
        d_2 (x, y) & = \sqrt{|x - y|}\\ 
        d_3 (x, y) & = |x^2 - y^2| \\ 
        d_4 (x, y) & = |x - 2y| \\ 
        d_5 (x, y) & = \frac{|x - y|}{1 + |x - y|}
    \end{align*}
    Determine, for each of these, whether it is a metric or not. 
  \end{exercise}
  \begin{solution}
    Listed. Positive semidefiniteness and symmetry are easy to check. 
    \begin{enumerate}
        \item The triangle inequality gives 
        \begin{align*}
            d_1 (x, z) \leq d_1 (x, y) + d_1 (y, z) & \iff (x - z)^2 \leq (x - y)^2 + (y - z)^2 \\
            & \iff 0 \leq (x - y) (y - z)
        \end{align*}
        which is not satisfied if $x < y < z$, so this is not a valid metric. 
        
        \item The triangle inequality gives $\sqrt{|x - z|} \leq \sqrt{|x - y|} + \sqrt{|y - z|}$, and since both sides are positive this inequality is equivalent to squaring both sides to get 
        \[|x - z| \leq |x - y| + |y - z| + 2 \sqrt{|x - y| |y - z|}\]
        which is true since $|x - z| \leq |x - y| + |y - z|$ of the Euclidean distance satisfies the triangle inequality and $0 \leq \sqrt{|x - y| |y - z|}$. 
        
        \item This does not satisfy triangle inequality, as taking $0, 1, 2$ gives 
        \[d_3 (0, 2) = 4 > 1 + 1 = d_3 (0, 1) + d_3 (1, 2)\]
        
        \item This does not satisfy symmetry. 
        
        \item For simplicity, let us set $A = |x - y|, B = |y - z|, C = |x - z|$. Then, we get 
        \[\frac{C}{1 + C} \leq \frac{A}{1 + A} + \frac{B}{1 + B} \iff C \leq A + B + 2 AB + ABC\]
        where $C \leq A + B$ is true by triangle inequality of Euclidean distance, $0 \leq AB$, and $0 \leq ABC$. 
    \end{enumerate}
    Intuitively, we want a metric that doesn't ``blow up" the distance between $x$ and $y$. More precisely, we want a valid metric $d(x, y)$ to be $O(|x - y|)$. Having something like a quadratic growth rate $(x - y)^2$ will blow the distance $d(x, z)$ up too much overpowering the individual $d(x, y) + d(y, z)$. 
  \end{solution}

  \begin{exercise}[Rudin 2.12]
    Let $K \subset \mathbb{R}$ consist of $0$ and the numbers $1/n$ for $n = 1, 2, 3, \ldots$. Prove that $K$ is compact directly from the definition (without using the Heine-Borel theorem). 
  \end{exercise}
  \begin{solution}
    Every open cover of $K$ must have an open set $G$ s.t. $0 \in G$. Since $G$ is open, there exists an open neighborhood $B_\epsilon (0) \subset G$ around $0$. By the Archimidean principle, there exists an $N \in \mathbb{N}$ s.t. 
    \begin{equation}
      \epsilon N > 1 \implies \epsilon > \frac{1}{N}
    \end{equation}
    and so, $B_\epsilon (0)$ contains all points $\{1/n\}$ for $n > N$. For the rest of the points $1, 1/2, \ldots, 1/N$, we can simply construct a finite cover over each of them, hence getting a finite cover. 
  \end{solution}

  \begin{exercise}[Rudin 2.13]
    Construct a compact set of real numbers whose limit points form a countable set. 
  \end{exercise}
  \begin{solution}
    Consider the set 
    \begin{equation}
      E = \bigg\{ \bigg( \frac{1}{10}\bigg)^n + \bigg( \frac{1}{10} \bigg)^{n+k} \; : \; n \in \{0\} \cup \mathbb{N}, k \in \mathbb{N} \bigg\} \cup \{0\}
    \end{equation}
    This is clearly bounded by $0$ and $1.1$. Let us represent the elements of this set by $(n, k)$. We can show that 
    \begin{equation}
      (n_1, k_1) > (n_2, k_2)
    \end{equation}
    if $n_1 < n_2$ or $n_1 = n_2$ and $k_1 < k_2$. Therefore, to prove closedness, we must prove that every limit point is a point in $E$. We can do this by proving that a point not in $E$ cannot be a limit point. Choose any $x \not\in E$. Then, due to the ordering, we can see that there exists a $(n, k)$ s.t. 
    \begin{equation}
      A = \bigg( \frac{1}{10}\bigg)^n + \bigg( \frac{1}{10} \bigg)^{n+k} < k < \bigg( \frac{1}{10}\bigg)^n + \bigg( \frac{1}{10} \bigg)^{n+k+1} = B
    \end{equation}
    and so we can take $\epsilon = \min\{k - A, B - k\}$ and show that $B_\epsilon (x)$ does not contain $A$ nor $B$, and so has an empty intersection with $E$. Therefore, it cannot be a limit point of $E$ and is closed. Since $E$ is bounded and closed in $\mathbb{R}$, it is compact. Its limit points contain $1, 0.1, 0.01, \ldots, 0$ (simply by fixing $n$ and letting $k \rightarrow \infty$, and so $E^\prime$ is infinite. We have just shown that since $E$ is closed, $E^\prime \subset E$. But $E$ is countable, so $E^\prime$ is countable. 
  \end{solution}

  \begin{exercise}[Rudin 2.14]
    Given an example of an open cover of the segment $(0, 1)$ which has no finite subcover. 
  \end{exercise}
  \begin{solution}
    Consider 
    \begin{equation}
      (0, 1/2) \cup \bigg( \bigcup_{i=1}^\infty \Big[ 1 - \frac{1}{2^i}, 1 - \frac{1}{2^{i+1}} \Big) \bigg)
    \end{equation}
  \end{solution}

  \begin{exercise}[Rudin 2.15]

  \end{exercise}

  \begin{exercise}[Rudin 2.16]
    Regard $\mathbb{Q}$, the set of all rational numbers, as a metric space, with $d(p, q) = |p - q|$. Let $E$ be the set of all $p \in \mathbb{Q}$ s.t. $2 < p^2 < 3$. Show that $E$ is closed and bounded in $\mathbb{Q}$ but is not compact. Is $E$ open in $\mathbb{Q}$? 
  \end{exercise}
  \begin{solution}
    $E$ is clearly bounded by $0$ and $2$ since $0^2 < 2 < p^2 < 3 < 2^2$. It is closed and we can show this by showing that $E^c$ is open. Let $x \in E^c$. Then, $x^2< 2$ or $x^2 > 3$. 
    \begin{enumerate}
      \item $x^2 < 2 \iff -\sqrt{2} < x < \sqrt{2}$. Now let $\epsilon = \min\{ \sqrt{2} - x, x + \sqrt{2}\} > 0$. Then by the Archimidean property there exists a $n \in \mathbb{N}$ s.t. $0 < \frac{1}{n} < \epsilon$. Therefore, the image of $B_{1/n} (x) \subset \mathbb{Q}$ will map onto $(0, 2)$. 

      \item $x^2 > 3 \iff x > \sqrt{3}$ or $x < - \sqrt{3}$. If $x > \sqrt{3}$, then by AP there exists a $n \in \mathbb{N}$ s.t. $x - \frac{1}{n} > \sqrt{3} \implies (x - \frac{1}{n})^2 > 3$. If $x < - \sqrt{3}$, then by AP there exist $n \in \mathbb{N}$ s.t. $x + \frac{1}{n} < -\sqrt{3} \implies (x + \frac{1}{n})^2 > 3$. Either way, the image of $B_{1/n} (x)$ will map within $E^c$. 
    \end{enumerate}
    It is not compact because $E$ is not closed in $\mathbb{R}$. The limit points of $E$ in $\mathbb{R}$ is $[\sqrt{2}, \sqrt{3}] \cup [-\sqrt{3}, -\sqrt{2}]$, which contains irrationals and is clearly not a subset of $E$. Since it is not closed in $\mathbb{R}$, it is not compact in $\mathbb{R}$, and it is not compact in $\mathbb{Q} \subset \mathbb{R}$. It is open because 
    \begin{equation}
      E = \big( (\sqrt{2}, \sqrt{3}) \cup (-\sqrt{3}, \sqrt{2})\big) \cup \mathbb{Q} \subset \mathbb{R}
    \end{equation}
    which is the union of open $(\sqrt{2}, \sqrt{3}) \cup (-\sqrt{3}, \sqrt{2})\big)$ and subset $\mathbb{Q} \subset \mathbb{R}$, and so it is open. 
  \end{solution}

  \begin{exercise}[Rudin 2.17]
    Let $E$ be the set of all $x \in [0, 1]$ whose decimal expansion consists of only the digits $4$ and $7$. Is $E$ countable? Is $E$ dense in $[0, 1]$? Is $E$ compact? Is $E$ perfect? 
  \end{exercise}

  \begin{exercise}[Rudin 2.18]
    Is there a nonempty perfect set in $\mathbb{R}$ which contains no rational number? 
  \end{exercise}

  \begin{exercise}[Rudin 2.19]
    Listed. 
    \begin{enumerate}
        \item If $A$ and $B$ are disjoint closed sets in some metric space $X$, prove that they are separated. 
        \item Prove the same for disjoint open sets. 
        \item Fix $p \in X$, $\delta > 0$, define $A$ to be the set of all $q \in X$ for which $d(p, q) < \delta$. Define $B$ similarly, with $>$ in place of $<$. Prove that $A$ and $B$ are separated. 
        \item Prove that every connected metric space with at least two points is uncountable. 
    \end{enumerate}
  \end{exercise}
  \begin{solution}
    Listed. 
    \begin{enumerate}
        \item This is trivial with the fact that the closure of the closure of $A$ is the closure of $A$. 
        \item Let $A, B$ be open. We wish to show that if $x \in A^\prime$, then $x \not\in B$. Assume $x \in B$. Then there exists $\epsilon > 0$ s.t. $B_\epsilon (x) \subset B$. But $B \cap A = \emptyset \implies B_\epsilon (x) \cap A = \emptyset$ and so $x \not\in A^\prime$, which is a contradiction. 

        \item Clearly, $A \cap B = \emptyset$. Not let $x \in A \implies$ there exists $\epsilon > 0$ s.t. $B_\epsilon (x) \subset A \implies B_\epsilon (x) \cap B = \emptyset \implies x \in B^\prime$. We can prove similarly to show that $x \in B \implies x \not\in A^\prime$. 

        \item Assume $X$ is countable (solutionis very similar for finite). Then, we can enumerate a $X = \{x_i\}_{i=1}^\infty$. We wish to show that $X$ can be decomposed into the union of an open ball and the interior of its complement as shown in (3). We fix $p \in X$. Then, we take the set $D = \{d(p, x)\}_{x \neq p} \subset \mathbb{R}$. Since $D$ is a countable subset of $\mathbb{R}$, there must exist some $\alpha > 0$ s.t. $\alpha \not\in D$. This $\alpha$ partitions the distances into two sets, and we can define 
        \[X = \{q \in X \mid d(p, q) < \alpha\} \cup \{q \in X \mid d(p, q) > \alpha\}\]
        and by (3), these two sets are separated, which means that $X$ is not connected, leading to a contradiction. 
    \end{enumerate}
  \end{solution}

  \begin{exercise}[Rudin 2.20]
    Are closures and interiors of connected sets always connected? Look at subsets of $\mathbb{R}^2$. 
  \end{exercise}
  \begin{solution}
    The interiors are not always connected. Consider the two closed balls $\overline{B_1 ((1, 0))}$ and $\overline{B_1 ((-1, 0))}$ as subsets of $\mathbb{R}^2$. They are connected but their interiors, which are the two open balls, are not connected. 

    As for closures, they are always connected. Let $W$ be connected. Then for any partition $A \cup B = W$, $\overline{A} \cap B \neq \emptyset$ WLOG. Consider $\overline{W} = W \cup W^\prime$ and take any partition $\overline{W} = C \cup D$. Then, label $A = C \cap W, A^\ast = C \cap W^\prime, B = D \cap W, B^\ast = D \cap W^\prime$. This implies that $C = A \cup A^\ast, D = B \cup B^\ast$, and $A \cup B = W$ (which is connected). Then, we can show that 
    \begin{align*}
        \overline{C} \cap D & = (\overline{A \cup A^\ast} \cap D) = (\overline{A} \cup \overline{A^\ast}) \cap D = (\overline{A} \cap D) \cup (\overline{A^\ast} \cap D) \\
        & =(\overline{A} \cap B) \cup (\overline{A} \cap B^\ast) \cup (\overline{A^\ast} \cap D)
    \end{align*}
    which cannot be empty since by connectedness of $W$, $\overline{A} \cap B \neq \emptyset$. Therefore, $\overline{W}$ is connected. 
  \end{solution}

  \begin{exercise}[Rudin 2.21]
    Let $A$ and $B$ be separated subsets of some $\mathbb{R}^k$. Suppose $a \in A, b \in B$ and define 
    \begin{equation}
      p(t) = (1 - t) a + t b
    \end{equation}
    for $t \in \mathbb{R}$. Put $A_0 = p^{-1} (A), B_0 = p^{-1} (B)$. 
    \begin{enumerate}
        \item Prove that $A_0$ and $B_0$ are separated subsets of $\mathbb{R}$. 
        \item Prove that there exists a $t_0 \in (0, 1)$ s.t. $p(t_0) \not\in A \cup B$. 
        \item Prove that every convex subset of $\mathbb{R}^k$ is connected. 
    \end{enumerate}
  \end{exercise}

  \begin{exercise}[Rudin 2.22]
    A metric space is called \textit{separable} if it contains a countable dense subset. Show that $\mathbb{R}^k$ is separable. 
  \end{exercise}
  \begin{solution}
    Consider the set $\mathbb{Q}^k \subset \mathbb{R}^k$. It is a finite Cartesian product (and hence, a countable union) of countable $\mathbb{Q}$, and so it is countable. $\mathbb{Q}^k$ is dense in $\mathbb{R}^k$ since given any $x \in \mathbb{R}^k$, we claim $x$ is a limit point of $\mathbb{Q}^k$. Given any $\epsilon > 0$, we can construct $B_\epsilon^\circ (x)$. For each coordinate $x_i$, by density of rationals in $\mathbb{R}$ we can choose a $q_i \in \mathbb{Q}$ s.t. $0 < d(x_i, q_i) < \epsilon / k$. Then, using triangle inequality, we can take the distances between each coordinate changed from $x_i$ to $q_i$. Let $q^k$ be the vector $x$ with the components $x_1, \ldots, x_k$ changed to $q_1, \ldots, q_k$, respectively. 
    \begin{equation}
      d(x, q) = d(x, q^1) + d(q^1, q^2) + \ldots + d(q^{k-1}, q_k) < \frac{\epsilon}{k} + \ldots + \frac{\epsilon}{k} = \epsilon
    \end{equation}
    and so $q \in B_\epsilon^\circ (x)$. Hence the intersection of $\mathbb{Q}^k$ and $B_\epsilon^\circ (x)$ for any $\epsilon > 0$ is nontrivial, so $x$ is a limit point of $\mathbb{Q}^k$. 
  \end{solution}

  \begin{exercise}[Rudin 2.23]
    A collection $\{V_\alpha\}$ of open subsets of $X$ is said to be a \textit{base} for $X$ if the following is true: For every $x \in X$ and every open set $G \subset X$ such that $x \in G$, we have $x \in V_\alpha \subset G$ for some $\alpha$. In other words, every open set in $X$ is the union of a subcollection of $\{V_\alpha\}$. Prove that every separable metric space has a countable base. 
  \end{exercise}
  \begin{solution}
    Since $X$ is separable it contains a countable dense subset, call it $S$. Then for every $x \in S$, we can look at the set of all open balls with center $x$ and rational radii, call it $\mathcal{B}$. Then $\mathcal{B}$ is countable. Now consider an open set $U$. By definition, for every $x \in U$, there exists an $\epsilon > 0$ s.t. $B_\epsilon (x) \subset U$. By AP, we can find a $n \in \mathbb{N}$ s.t. $0 < \frac{1}{n} < \epsilon$, and therefore we can find an open ball $B \in \mathcal{B}$ s.t. $B (x) \subset U$. We claim that 
    \begin{equation}
      W \coloneqq \bigcup_{x \in U} B (x) = U
    \end{equation}
    If $x \in U$, then by construction it is contained in $B(x) \subset \cup_{x \in U } B(x)$, and so $U \subset W$. If $x \in W$, then it is in $B(x)$, which is fully contained in $U$ and so $W \subset U$. Therefore every open set can be constructed by a countable union of open balls in countable $\mathcal{B}$. 
  \end{solution} 

  \begin{exercise}[Rudin 2.24]
    Let $X$ be a metric space in which every infinite subset has a limit point. Prove that $X$ is separable. 
  \end{exercise}
  \begin{solution}
    We fix $\delta > 0$. Choose $x_1 \in X$. Then choose $x_2 \in X$ s.t. $d(x_1, x_2) \geq \delta$, and keep doing this until we choose $x_{j + 1} \in X$ s.t. $d(x_{j+1}, x_i) \geq \delta$ for all $i \in 1, \ldots, j$. 
    \begin{enumerate}
        \item We claim that this must stop after a finite number of steps. Assume it doesn't. Then by assumption $V = \{x_i\}_{i=1}^\infty$ should have a limit point in $X$, denote it $x$. Choose $\frac{\delta}{2} > 0$. Then, $B_{\delta / 2}^\circ (x) \cap V \neq \emptyset$. This intersection can only have one point since if it had two $x^\prime, x^{\prime\prime}$, then since both are in $B_{\delta /2} (x)$, then 
        \[d(x^\prime, x^{\prime\prime}) \leq d(x^\prime, x) + d(x, x^{\prime\prime}) \leq \frac{\delta}{2} + \frac{\delta}{2} = \delta\]
        and since they are both in $V$, then $d(x^\prime, x^{\prime\prime}) \geq \delta$, which is a contradiction. Since there is a finite number of points in $B_{\delta /2} (x)$ of $V$, $x$ cannot be a limit point. So this must terminate at some finite $J < \infty$. 
        \item Denote $W = \{x_i\}_{i=1}^J$. Then, $\mathscr{B}_\delta = \{B_\delta (x) \mid x \in W\}$ must cover $X$, since if it didn't, there would exist a $y \in X$ s.t. $d(y, x) \geq \delta$ for all $x \in W$, and we can add another element in $W$. 
        \item Consider $\delta = 1, 1/2, 1/3, \ldots$ and construct the same cover 
        \[\mathscr{B}_k = \{B_{1/k} (x_{ki}) \mid i = 1, \ldots, J_k\}\]
        which is finite. Therefore, $\mathscr{B} = \cup_{k=1}^\infty \mathscr{B}_k$ must be countable. 
        \item We claim that countable $\{x_{ki}\}_{k, i}$ is dense. Consider any $x \in X$. For every $\epsilon > 0$, we can find an arbitrarily large $n \in \mathbb{N}$ s.t. $0 < \frac{1}{n} < \epsilon$. Since $\mathscr{B}_n$ is an open cover, there must exist some $x_{n i}$ s.t. $x \in B_{1/n} (x_{ni})$, which by symmetry implies that $x_{ni} \in B_{1/n} (x) \subset B_\epsilon (x)$. Therefore, there always exists an $x_{ni}$ in every $B_\epsilon (x)$, and so $B_\epsilon (x) \cap \{x_{ki}\} \neq \emptyset \implies x$ is a limit point of $\{x_{ki}\}$ and so it is dense. 
    \end{enumerate}
  \end{solution}

  \begin{exercise}[Rudin 2.25]
    Prove that every compact metric space $K$ has a countable base, and that $K$ is therefore separable. 
  \end{exercise}
  \begin{solution}
    For every $n \in \mathbb{N}$, let us consider an open covering $\mathscr{F}_n \coloneqq \{B_{1/n} (x_n) \mid x_n \in K\}$. Since $K$ is compact, it has a finite subcovering 
    \begin{equation}
      \mathscr{G}_n \coloneqq \{B_{1/n} (x_{ni}) \mid i = 1, \ldots, k(n)\}
    \end{equation}
    Now consider the union $\mathscr{G} = \cup_{i=1}^n \mathscr{G}_n$, which is countable. We claim that $\mathscr{G}$ is a base. Consider any open set $U$. Then for every $x \in U$, we want to show that $x$ is contained in a $B_{1/n} (x_{ni}) \subset U$. Since $U$ is open , there exists a $\epsilon > 0$ s.t. $B_\epsilon (x) \subset U$. Now by AP, there exists a $n \in \mathbb{N}$ s.t. $0 < \frac{1}{n} < \frac{\epsilon}{2}$. Therefore $B_{1/n} (x) \subset B_{\epsilon} (x)$. Since $\mathscr{G}$ is an open covering, there must exist some $B_{1/n} (x_{ni})$ that contains $x$. Now we wish to show that $B_{1/n} (x_{ni})$ is fully contained in $U$. Let $y \in B_{1/n} (x_{ni})$. Then, by triangle inequality, 
    \begin{equation}
      d(y, x) = d(y, x_{ni}) + d(x_{ni}, x) < \frac{1}{n} + \frac{1}{n} < \epsilon
    \end{equation}
    and therefore $x \in B_{1/n} (x_{ni}) \subset B_{\epsilon} (x)$. Therefore, for every $x \in U$, we can construct an open ball of $\mathscr{G}$ containing $x$ and contained in $U$, proving that this is a base. 

    We claim that the set of all $\mathscr{P} = \{x_{ni}\}_{n, i}$ forms a countable dense subset. This is clearly countable since $\mathscr{G}$ is countable. We must prove that the closure of $\mathscr{P} = K$. Let $x \in K$. Given any $\epsilon > 0$, we wish to show that $B_{\epsilon} (x) \cap \mathscr{P} \neq \emptyset$. Since $B_\epsilon (x)$ is open, it can be covered by a subcollection of $\mathscr{G}$, and so their centers must be in $B_\epsilon (x)$, proving that $B_{\epsilon} (x) \cap \mathscr{P} \neq \emptyset$. Therefore, $x$ is a limit point of $\mathscr{P}$. 
  \end{solution}

  \begin{exercise}[Rudin 2.26]

  \end{exercise}

  \begin{exercise}[Rudin 2.27]

  \end{exercise}

  \begin{exercise}[Rudin 2.28]

  \end{exercise}

  \begin{exercise}[Rudin 2.29]
    Prove that every open set in $\mathbb{R}$ is the union of an at most countable collection of disjoint segments. 
  \end{exercise}
  \begin{solution}
    Let $U \subset \mathbb{R}$ be open. Then for all $x \in U$ there exists $\epsilon > 0$ s.t. $(x - \epsilon, x + \epsilon) \subset U$. Now since $\mathbb{R}$ is separable (by exercise Rudin 2.22), it has a countable dense subset $\mathbb{Q}$. Consider all segments of rational radius and rational centers 
    \begin{equation}
      \mathscr{B} = \{ (q - p, q + p) \subset \mathbb{R} \mid q, p \in \mathbb{Q}\}
    \end{equation}
    This is clearly countable. We claim that every open $U$ can be expressed as the union of a subset of $\mathscr{B}$. Now by AP, there exists $n \in \mathbb{N}$ s.t. $0 < \frac{1}{n} < \frac{\epsilon}{2}$, so for all $x \in U$, there exists $n \in \mathbb{N}$ s.t. $(x - \frac{1}{n}, x + \frac{1}{n}) \subset U$. Now since $\mathbb{Q}$ is dense in $\mathbb{R}$, $x \in \mathbb{Q}^\prime \implies (x - \frac{1}{n}, x + \frac{1}{n}) \cap \mathbb{Q} \neq \emptyset$. Say $r$ is in this intersection. Then, by symmetry of metric, $x \in (r - \frac{1}{n}, r + \frac{1}{n})$. Therefore, for all $x \in U$, we have found an open ball in $\mathscr{B}$ that contains $x$. Now, we must show that this actually is fully contained in $U$. This is easy, since if $y \in B_{1/n} (r)$, then 
    \begin{equation}
      d(y, x) \leq d(y, r) + d(r, x) \leq \frac{1}{n} + \frac{1}{n} < \epsilon
    \end{equation}
    and so $B_{1/n} (r)$ is complete contained in the $\epsilon$-ball around $x$, which is a subset of $U$. So for all $x \in U$, we found an open set $U_x \in \mathscr{B}$ covering $x$ and fully contained in $U$, which means that $\cup_{x \in U} U_x = U$. Now for some intervals $B_1, B_2 \in \mathscr{B}$, if $B_1 \cap B_2 \neq \emptyset$, take their union, which is another segment, and keep doing this until $B_i \cap B_j \neq \emptyset$ for all $i, j$. The cardinality of this new pruned set will be less than or equal to $\mathscr{B}$, which is countable, and so this must be at most countable. 
  \end{solution}


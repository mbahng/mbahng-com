\section{Exercises}

\subsection{The Real Numbers}

  \begin{exercise}[Math 531 Spring 2025, PS2.1]
    Prove that the set of all matrices of the form:
    \begin{equation}
      \begin{bmatrix}
        a & -b \\
        b & a
      \end{bmatrix},
    \end{equation}
    with $a,b \in \mathbb{R}$ forms a field with the usual sum and product operations of
    matrices. What does this field resemble? Give extensions to $3 \times 3$ and
    $4 \times 4$ matrices.
  \end{exercise}
  \begin{solution}
    
  \end{solution}

  \begin{exercise}[Math 531 Spring 2025, PS2.2]
    Why can't the field of complex numbers (with its usual operations) be
    made into an ordered field?
  \end{exercise}
  \begin{solution}
    Solution is shown as theorem. 
  \end{solution}

  \begin{exercise}[Math 531 Spring 2025, PS2.3]
    Prove there are no finite ordered fields.
  \end{exercise}
  \begin{solution}
    Solution is shown as theorem. 
  \end{solution}

  \begin{exercise}[Math 531 Spring 2025, PS2.4]
    Prove that if $x$ and $n$ are natural numbers, then
    \begin{equation}
      x^n - 1 = (x-1)(1 + x + x^2 + ... + x^{n-1}).
    \end{equation}
  \end{exercise}
  \begin{solution}
    We use the commutative addition and multiplication, plus distributive property in $\mathbb{Z}$. 
    \begin{align}
      (x - 1) \bigg( \sum_{i=0}^{n-1} x^i \bigg) & = x \sum_{i=0}^{n-1} x^i - \sum_{i=0}^{n-1} x_i \\
                                                 & = \sum_{i=1}^n x^i - \sum_{i=0}^{n-1} x^i \\
                                                 & = x^n + \sum_{i=1}^{n-1} x^i - \sum_{i=1}^{n-1} x^i - 1 \\
                                                 & = x^n - 1
    \end{align}
  \end{solution}

  \begin{exercise}[Math 531 Spring 2025, PS2.7]
    Prove that there is no $q \in \mathbb{Q}$ for which
    \begin{equation}
      q^2 + q = 4.
    \end{equation}
  \end{exercise}
  \begin{solution}
    Assume that such a $q \in \mathbb{Q}$ in canonical form exists. Then by the field properties, since $\frac{1}{4} \in \mathbb{Q}$, 
    \begin{equation}
      q^2 + q + \frac{1}{4} = 4 + \frac{1}{4} = \frac{17}{4} \in \mathbb{Q}
    \end{equation}
    But by distributive properties, $(q + \frac{1}{2})^2 \in \mathbb{Q}$. We claim that there exists no rational $x = a/b$ ($a, b$ coprime) s.t. $x^2 = 17/4$. If there were, then clearly $a \neq 0$ and 
    \begin{align}
      \frac{a^2}{b^2} = \frac{17}{4} & \implies 4a^2 = 17 b^2 \\
                                     & \implies 2 | b, \text{ and so } b = 2b^\prime \text{ for some } b^\prime \in \mathbb{N} \\
                                     & \implies a^2 = 17 (b^\prime)^2 \\ 
                                     & \implies 17 | a \text{ and so } a = 17 a^\prime \text{ for some } a^\prime \in \mathbb{Z} \\
                                     & \implies 17 (a^\prime)^2 = (b^\prime)^2
    \end{align}
    which implies that $17|b^\prime$, but this contradicts the assumption that $a, b$ are coprime. Therefore $q + \frac{1}{2} \not\in \mathbb{Q} \implies q \not\in \mathbb{Q}$. 
  \end{solution}

  \begin{exercise}[Math 531 Spring 2025, PS2.8]
    Let $X$ be an ordered set with the least upper bound property. Prove that $X$ has the greatest lower bound property.
  \end{exercise}
  \begin{solution}
    Shown in theorem above. 
  \end{solution}

  \begin{exercise}[Math 531 Spring 2025, PS2.9]
    Prove that if $x, y \in \mathbb{Q}$ we have that
    \begin{equation}
      ||x| - |y|| \leq |x - y|.
    \end{equation}
  \end{exercise}
  \begin{solution}
    By subadditivity of the norm we have 
    \begin{align}
      |x| \leq |x - y| + |y| & \implies |x| - |y| \leq |x - y| \\
      |y| \leq |y - x| + |x| & \implies |y| - |x| \leq |y - x| 
    \end{align}
    But $|y - x| = |-1 (x - y)| = |-1| \cdot |x - y| = |x - y|$, and so 
    \begin{equation}
      \max\{ |x| - |y|, |y| - |x| \} \leq |x - y|
    \end{equation}
    and the LHS is the definition of the norm $||x| - |y||$ in $\mathbb{Q}$. 
  \end{solution}

  \begin{exercise}[Rudin 1.1]
    If $r$ is rational ($r \neq 0$) and $x$ is irrational, prove that $r + x$ and $rx$ are irrational. 
  \end{exercise}
  \begin{solution}
    If we assume that $r x = t$ and $r + x = s$ are rational, then this violates the field axioms of $\mathbb{Q}$ since then $x = t r^{-1}$ and $x = s + (-r)$ are rational. 
  \end{solution}

  \begin{exercise}[Rudin 1.2]
    Prove that there is no rational number whose square is $12$. 
  \end{exercise}
  \begin{solution}
    Assume that there exists a number $p/q$ such that $p$ and $q$ are both not even. Then, 
    \begin{equation}
      \bigg( \frac{p}{q} \bigg)^2 = 12 \implies p^2 = 12q^2 = 3 (2 q)^2
    \end{equation}
    So $p$ much be even $p = 2 p^\prime$. Therefore, $p^{\prime 2} = 3 q^2$, and $q$ must be odd. This means that $p^\prime$ must be odd. We can rewrite the equation 
    \begin{equation}
      p^{\prime 2} - q^2 = 2 q^2 \implies (p^\prime + q) (p^\prime - q) = 2q^2
    \end{equation}
    where the left hand side is divisible by $4$ but the right hand side is divisible by at most $2$, leading to a contradiction. 
  \end{solution}

  \begin{exercise}[Rudin 1.3]
    Prove that the axioms of multiplication imply the following. 
    \begin{enumerate}
      \item If $x \neq 0$ and $xy = xz$, then $y = z$. 
      \item If $x \neq 0$ and $xy = x$, then $y = 1$. 
      \item If $x \neq 0$ and $xy = 1$, then $y = x^{-1}$. 
      \item If $x \neq 0$, then $(x^{-1})^{-1} = x$. 
    \end{enumerate}
  \end{exercise}
  \begin{solution}
    Listed. 
    \begin{enumerate}
      \item $xy = xz \implies \frac{1}{x} \cdot x y = \frac{1}{x} x z \implies y = z$ 
      \item $x y = x \implies \frac{1}{x} x y = \frac{1}{x} x \implies y = 1$ 
      \item $x y = 1 \implies \frac{1}{x} x y = \frac{1}{x} 1 \implies y = \frac{1}{x}$ 
      \item $(x^{-1})^{-1} \cdot x^{-1} = 1 \implies (x^{-1})^{-1} \cdot x^{-1} \cdot x = 1 \cdot x \implies (x^{-1})^{-1} = x$
    \end{enumerate}
  \end{solution}

  \begin{exercise}[Rudin 1.4]
    Let $E$ be a nonempty subset of an ordered set; suppose $\alpha$ is a lower bound of $E$ and $\beta$ is an upper bound of $E$. Prove that $\alpha \leq \beta$. 
  \end{exercise}
  \begin{solution}
    Since $E$ is nonempty, we choose any $x \in E$. By definition, $\alpha \leq x$ and $x \leq \beta$, and by transitive property of orderings, we have $\alpha \leq \beta$. 
  \end{solution}

  \begin{exercise}[Rudin 1.5]
    Let $A$ be a nonempty set of real numbers which is bounded below. Let $-A$ be the set of all numbers $-x$, where $x \in A$. Prove that 
    \begin{equation}
      \inf A = -\sup(-A)
    \end{equation}
  \end{exercise}
  \begin{solution}
    We would like to prove that $\inf A \leq -\sup(-A)$ and $\inf A \geq -\sup(-A)$. For the first part, we start off with the definition of the infimum. 
    \begin{align*}
      \inf A \leq x \; \forall x \in A & \implies - \inf A \geq -x \; \forall x \in A \\
      & \implies - \inf A \geq x \forall x \in -A \\
      & \implies -\inf A \geq \sup(-A) \\
      & \implies \inf A \leq - \sup(-A)
    \end{align*}
    For the second part, we start with the definition of the supremum. 
    \begin{align*}
      \sup(-A) \geq x \forall x \in -A & \implies \sup(-A) \geq -x \forall x \in A \\
      & \implies -\sup(-A) \geq x \forall x \in A \\
      & \implies -\sup(-A) \leq \inf A
    \end{align*}
  \end{solution}

  \begin{exercise}[Rudin 1.6]
    Fix $b > 1$.
    \begin{enumerate}
      \item If $m$, $n$, $p$, $q$ are integers, $n > 0$, $q > 0$, and $r = m/n = p/q$, prove that
      \begin{equation}
        (b^m)^{1/n} = (b^p)^{1/q}.
      \end{equation}
      Hence it makes sense to define $b^r = (b^m)^{1/n}$.
      
      \item Prove that $b^{r+s} = b^rb^s$ if $r$ and $s$ are rational.
      
      \item If $x$ is real, define $B(x)$ to be the set of all numbers $b^t$, where $t$ is rational and $t \leq x$. Prove that 
      \begin{equation}
        b^r = \sup B(r)
      \end{equation}
      when $r$ is rational. Hence it makes sense to define
      \begin{equation}
        b^x = \sup B(x)
      \end{equation}
      for every real $x$.
      
      \item Prove that $b^{x+y} = b^xb^y$ for all real $x$ and $y$.
    \end{enumerate}
  \end{exercise}
  \begin{solution}
    Proved in theorem above. 
  \end{solution}

  \begin{exercise}[Rudin 1.7]
    Fix $b > 1$, $y > 0$, and prove that there is a unique real $x$ such that $b^x = y$, by
    completing the following outline. (This $x$ is called the logarithm of $y$ to the base $b$.)
    \begin{enumerate}
      \item For any positive integer $n$, $b^n - 1 \geq n(b-1)$.
      \item Hence $b - 1 \geq n(b^{1/n} - 1)$.
      \item If $t > 1$ and $n > (b-1)/(t-1)$, then $b^{1/n} < t$.
      \item If $w$ is such that $b^w > y$, then $b^{-(1/n)} < y$ for sufficiently large $n$; to see this,
        apply part (c) with $t = y \cdot b^{-w}$.
      \item If $b^w > y$, then $b^{w-(1/n)} > y$ for sufficiently large $n$.
      \item Let $A$ be the set of all $w$ such that $b^w < y$, and show that $x = \sup A$ satisfies
        $b^x = y$.
      \item Prove that this $x$ is unique.
    \end{enumerate}
  \end{exercise}
  \begin{solution}
    Proved in theorem above. 
  \end{solution}

  \begin{exercise}[Rudin 1.8]
    Prove that no order can be defined in the complex field that turns it into an ordered field. 
  \end{exercise}
  \begin{solution}
    Note that if $x \geq 0$, then $-x \leq 0$ for all $x$ of any ordered field. Since if $x \geq 0$ and $-x > 0$, then $x -x > 0$, which is absurd. Therefore, one of either $i$ or $-i$ should be greater than $0$. But $i^2 = (-i)^2 = -1$, so this means that $-1 > 0$, which implies that $0 < 1$. But either $1$ or $-1$ must $\geq 0$. 
  \end{solution}

  \begin{exercise}[Rudin 1.9]
    Equip $\mathbb{C}$ with the dictionary order. That is, given $z = a + bi$ and $w = c + di$, $z < w$ if $a < c$, or if $a = c$ and $b < d$. Does this ordered set have a least upper bound property? 
  \end{exercise}
  \begin{solution}
    No it does not. Consider the set $S = \{ a + b i \in \mathbb{C} \mid a \leq 3\}$. $S$ is bounded by $4$, but it doesn't have a least upper bound. Given any $3 + bi$, this is not an upper bound since we can construct $3 + (b + \epsilon) i \in S$. Given any $a + bi$ where $a > 3$, we can always find a lower bound of form $a + (b - \epsilon) i$ that also bounds $S$. 
  \end{solution}

  \begin{exercise}[Rudin 1.10]
    Suppose $z = a + bi$, $w = u + iv$, and 
    \begin{equation}
      a = \bigg( \frac{|w| + u}{2} \bigg)^{1/2} \text{ and } b = \bigg( \frac{|w| - u}{2} \bigg)^{1/2}
    \end{equation}
    Prove that $z^2 = w$ if $v \geq 0$ and that $(\bar{z})^2 = w$ if $v \leq 0$. Conclude that every complex number (with one exception!) has two complex square roots. 
  \end{exercise}
  \begin{solution}
    We can calculate 
    \begin{equation}
      z^2 = (a^2 - b^2) + 2 a b i = u + \sqrt{v^2} i = \begin{cases} u + v i & \text{ if } v \geq 0 \\ u - vi & \text{ if } v \leq 0 \end{cases} 
    \end{equation}
    Since if we assume $v \geq 0$, then we have $z^2 = w$. We also get 
    \begin{equation}
      \bar{z}^2 = (a^2 - b^2) - 2 a b i = u - \sqrt{v^2} i = \begin{cases} u - v i & \text{ if } v \geq 0 \\ u + vi & \text{ if } v \leq 0 \end{cases}
    \end{equation}
    and assuming $v \leq 0$, we have $\bar{z}^2 = w$. Therefore, every complex number $w$ has both $\pm z$ as its square root if $v \geq 0$, $\pm \bar{z}$ if $v \leq 0$, and just one root if $z = 0$. 
  \end{solution}

  \begin{exercise}[Rudin 1.11]
    If $z$ is a complex number, prove that there exists an $r \geq 0$ and a complex number $w$ with $|w| = 1$ s.t. $z = rw$. Are $w$ and $r$ always uniquely determined by $z$? 
  \end{exercise}
  \begin{solution}
    If $z = 0$, then $r = 0$ and there is no unique $w$. If $z = a + bi \neq 0$, then define 
    \begin{equation}
      r = |z| = (a^2 + b^2)^{1/2}, \;\; w = \frac{1}{r} z
    \end{equation}
    which proves existence. As for uniqueness, assume that there are two forms 
    \begin{equation}
      z = r w = r^\prime w^\prime
    \end{equation}
    Then, $w = \frac{r^\prime}{r} w^\prime \implies |w| = \big| \frac{r^\prime}{r} \big| |w^\prime| = 1$, which implies that $r^\prime / r = 1$ and so $r = r^\prime$. This means that $w = w^\prime$. 
  \end{solution}

  \begin{exercise}[Rudin 1.12]
    If $z_1, \ldots, z_n$ are complex, prove that
    \begin{equation}
      |z_1 + z_2 + \ldots + z_n| \leq |z_1| + \ldots + |z_n|
    \end{equation}
  \end{exercise}
  \begin{solution}
    By induction, it suffices to prove $|z_1 + z_2| \leq |z_1| + |z_2|$. We have 
    \begin{align*}
      |z_1 + z_2|^2 & = (z_1 + z_2) (\overline{z_1 + z_2}) \\
      & = (z_1 + z_2) (\bar{z_1} + \bar{z_2}) \\ 
      & = z_1 \bar{z_1} + z_1 \bar{z_2} + z_2 \bar{z_1} + z_2 \bar{z_2} \\
      & = |z_1|^2 + |z_2|^2 + z_1 \bar{z_2} + z_2 \bar{z_1} \\
      & = |z_1|^2 + |z_2|^2 + 2 (a c + bd) \\
      & \leq |z_1|^2 + |z_2|^2 + 2 \sqrt{a^2 + b^2} \sqrt{c^2 + d^2} \;\; (\text{Schwartz}) \\
      & = |z_1|^2 + |z_2|^2 + 2 |z_1| |z_2| \\
      & = (|z_1| + |z_2|)^2
    \end{align*}
    since both sides are positive, we can take their square root to get the desired result. 
  \end{solution}

  \begin{exercise}[Rudin 1.13]
    If $x, y$ are complex, prove that 
    \begin{equation}
      \big| |x| - |y| \big| \leq |x - y|
    \end{equation}
  \end{exercise}
  \begin{solution}
    Since both sides are nonnegative, we can square both sides. Note that due to Cauchy Schwartz inequality, $2|x| |y| \geq x \bar{y} + y \bar{x}$ since expanding them gives 
    \begin{equation}
      2 \sqrt{a^2 + b^2} \sqrt{c^2 + d^2} \geq 2 (ac + bd)
    \end{equation}
    Therefore, the following inequality is true: 
    \begin{equation}
      |x|^2 + |y|^2 - 2 |x| |y| \leq x \bar{x} + y \bar{y} - x \bar{y} - y \bar{x}
    \end{equation}
    which reduces to form $(|x| - |y|)^2 \leq |x - y|^2$. 
  \end{solution}

  \begin{exercise}[Rudin 1.14]
    If $z$ is a complex number s.t. $|z| = 1$, that is such that $z \bar{z} = 1$, compute 
    \begin{equation}
      |1 + z|^2 + |1 - z|^2
    \end{equation}
  \end{exercise}
  \begin{solution}
    Compute. 
    \begin{equation}
      (1 + z) (1 + \bar{z}) + (1 - z) (1 - \bar{z}) = 1 + z + \bar{z} + z \bar{z} + 1 - z - \bar{z} + z \bar{z} = 4
    \end{equation}
  \end{solution}

  \begin{exercise}[Rudin 1.15]
    Under what conditions does equality hold in the Schwarz inequality? 
  \end{exercise}
  \begin{solution}
    If they are antiparallel, since 
    \begin{equation}
      \langle x, y \rangle = ||x || ||y || \cos{\theta}
    \end{equation}
  \end{solution}

  \begin{exercise}[Rudin 1.16]
    Suppose $k \geq 3$, $\mathbf{x}, \mathbf{y} \in \mathbb{R}^k$, $|\mathbf{x} - \mathbf{y}| = d > 0$, and $r > 0$. Prove: 
    \begin{enumerate}
      \item[a)] If $2r > d$, there are infinitely many $\mathbf{z} \in \mathbb{R}^k$ s.t. 
      \[|\mathbf{z} - \mathbf{x}| = |\mathbf{z} - \mathbf{y}| = r\]
      \item[b)] If $2r = d$, there is exactly one such $\mathbf{z}$. 
      \item[c)] If $2r < d$, there is no such $\mathbf{z}$. 
    \end{enumerate}
  \end{exercise}
  \begin{solution}
    
  \end{solution}

  \begin{exercise}[Rudin 1.17]
    Prove that
    \begin{equation}
      |\mathbf{x} + \mathbf{y}|^2 + |\mathbf{x} - \mathbf{y}|^2 = 2|\mathbf{x}|^2 + 2|\mathbf{y}|^2
    \end{equation}
  \end{exercise}
  \begin{solution}
    This is trivial if we simply expand 
    \begin{align}
      |\mathbf{x} + \mathbf{y}|^2 + |\mathbf{x} - \mathbf{y}|^2 & = \langle \mathbf{x} + \mathbf{y}, \mathbf{x} + \mathbf{y} \rangle + \langle \mathbf{x} - \mathbf{y}, \mathbf{x} - \mathbf{y} \rangle \\
      & = \langle \mathbf{x}, \mathbf{x} \rangle + 2 \langle \mathbf{x}, \mathbf{y} \rangle + \langle \mathbf{y}, \mathbf{y} \rangle + \langle \mathbf{x}, \mathbf{x} \rangle - 2 \langle \mathbf{x}, \mathbf{y} \rangle + \langle \mathbf{y}, \mathbf{y} \rangle \\
      & = 2 \langle \mathbf{x}, \mathbf{x} \rangle + 2 \langle \mathbf{y}, \mathbf{y} \rangle \\
      & = 2|\mathbf{x}|^2 + 2|\mathbf{y}|^2
    \end{align}
  \end{solution}

  \begin{exercise}[Rudin 1.18]
    If $k \geq 2$ and $\mathbf{x} \in \mathbb{R}^k$, prove that there exists $\mathbf{y} \in \mathbb{R}^k$ s.t. $\mathbf{y} \neq \mathbf{0}$, but $\mathbf{x} \cdot \mathbf{y} = 0$. Is this also true if $k = 1$? 
  \end{exercise}
  \begin{solution}
    Let $x \in \mathbb{R}^k$ and $\ell \in \mathbb{R}^{k \ast}$, the dual space. By Riesz representation theorem, we can define the canonical isomorphism $\ell \mapsto y$ between these two spaces as 
    \begin{equation}
      \ell (x) = (x, y)
    \end{equation}
    Since $y \neq 0$ by assumption, $\ell \neq 0$, and so its rank is at least $1$. Since $\ell$ maps to $\mathbb{R}$, the rank has to be $1$. By rank nullity theorem, we have 
    \begin{equation}
      \dim N(\ell) = k - \mathrm{rank}(\ell) = k - 1
    \end{equation}
    and so there exists nontrivial annihilators $\ell$ of $x$, which can be mapped to a nontrivial $y \in \mathbb{R}^k$. 
  \end{solution}

  \begin{exercise}[Rudin 1.19]
    Suppose $\mathbf{a}, \mathbf{b} \in \mathbb{R}^k$. Find $\mathbf{c} \in \mathbb{R}^k$ and $r > 0$ s.t. 
    \begin{equation}
      |\mathbf{x} - \mathbf{a}| = 2 | \mathbf{x} - \mathbf{b}|
    \end{equation}
    if and only if $|\mathbf{x} - \mathbf{c}| = r$. 
  \end{exercise}
  \begin{solution}
    If we draw out the circle, it must contain two points on the line drawn by connecting $A$ and $B$. Since it must be symmetric, its center and radius can then be easily calculated to be 
    \begin{equation}
      r = \frac{2}{3} |b - a|, \;\;\; c = \frac{1}{3} (4b - a)
    \end{equation}
  \end{solution}

  \begin{exercise}[Zorich 2.2.1]
    Using the principle of induction, show that 
    \begin{enumerate}
      \item the sum $x_1 + \ldots + x_n$ of real numbers is defined independently of the insertion of parentheses to specify the order of addition. 
      \item the same is true of the product $x_1 \ldots x_n$ 
      \item $|x_1 + \ldots + x_n| \leq |x_1| + \ldots + |x_n|$ 
      \item $|x_1 \ldots x_n| \leq |x_1| \ldots |x_n|$
      \item For any $n, m \in \mathbb{N}$ such that $m < n$, $(n - m) \in \mathbb{N}$. 
      \item $(1 + x)^n \geq 1 + n x$ for $x > - 1$ and $n \in \mathbb{N}$, equality holding for when $n=1$ or $x=0$. 
      \item $(a + b)^n = a^n + _n C_1 a^{n-1} b^1 + \ldots + b^n$ (aka binomial theorem). 
    \end{enumerate}
  \end{exercise}
  \begin{solution}
    Listed. 
    \begin{enumerate}
      \item Let $n$ denote the number of elements in the sum. We prove by strong law of induction. The base case for when $n=1, 2, 3$ is trivially true. 
      \begin{align*}
        x_1 & = x_1 & (\text{identity}) \\
        x_1 + x_2 & = x_1 + x_2 & (\text{identity}) \\
        (x_1 + x_2) + x_3 & = x_1 + (x_2 + x_3) & (\text{associativity}) 
      \end{align*}
      Then, the sum of $n=k$ parameters is defined by $k-2$ pairs of parentheses defining the order of the sum. These parentheses define a sequence of $k-1$ 2-fold additions. Now, assume that the claim is true for 
      \begin{equation}
        S_n \equiv x_1 + \ldots x_n \text{ for } n = 1, 2, \ldots, k
      \end{equation}
      Then, for a specific sum $S_{k+1}$ of $k+1$ elements with $k-1$ parentheses, we can reduce the sum to its final 2-fold addition 
      \begin{equation}
        S_{k+1} \equiv \underbrace{(x_1 + \ldots + x_i)}_{\varphi_1}  + \underbrace{(x_{i+1} + \ldots + x_{k+1})}_{\varphi_2}
      \end{equation}
      Since $i, k-i+1 < k$, by the strong law $\varphi_1$ and $\varphi_2$ are independent of the order of sum. 
      \item Exactly identical to (a). 
      \item By the triangle inequality $|x_1 + x+2| \leq |x_1| + |x_2|$. Now, assume for $n=k$ is true. Then, let $S_k = x_1 + \ldots + x_k$, so 
      \begin{equation}
        |x_1 + \ldots + x_k + x_{k+1}| = |S_k + x_{k+1}| \leq |S_k| + |x_{k+1}| \leq \sum_{i=1}^{k+1} |x_{i}|
      \end{equation}
      \item Same as (c). 
      \item Let us fix $m$ to be any element of $\mathbb{N}$. Then, the base case is for $n = m + 1$ (which is in $\mathbb{N}$ since it is inductive), so 
      \begin{equation}
        n - m = (m + 1) - m = 1 \in \mathbb{N}
      \end{equation}
      Now, given that for some integer $n \geq m+1$, $n - m \in \mathbb{N}$ is true, we have 
      \begin{align*}
        (n + 1) - m & = n + (1 - m ) & (\text{associativity}) \\
        & = n + (-m + 1) & (\text{commutativity}) \\
        & = (n - m) + 1 & (\text{associativity})
      \end{align*}
      where $(n - m) + 1 \in \mathbb{N}$ by inductive property of $\mathbb{N}$. 
      \item We prove by induction. For $n=1$, it is trivial that $(1 + x)^1 \geq 1 + 1 \cdot x$. Now assume that the claim is true for some $k \in \mathbb{N}$. Then, 
      \begin{align*}
        (1 + x)^{k+1} = (1 + x)^k (1 + x) & \geq (1 + k x) (1 + x) \\
        = 1 + (k+1) x + k x^2 \\ 
        \geq 1 + (k+1) x
      \end{align*}
      where equality holds if $x = 0 \implies 1^{k+1} = 1^k \cdot 1 = 1$ or $n=1 \implies $ trivial case. 
      \item The base case for $n=1$ is trivial since $(a + b)^1 = \binom{1}{0} a + \binom{1}{1} b$. We introduce Newton's identity. 
      \begin{align*}
        \binom{k}{j-1} + \binom{k}{j} & = \frac{k!}{(j-1)! (k-j+1)!} + \frac{k!}{j! (k-j)!} \\
        & = k! \bigg( \frac{j}{j! (k-j+1)!} + \frac{k-j+1}{j! (k-j+1)!} \bigg) \\
        & = k! \cdot \frac{k+1}{j! (k-j+1)!} \\
        & = \frac{(k+1)!}{j!(k-j+1)!} = \binom{k+1}{j}
      \end{align*}
      Now assuming that the binomial formula holds for some $n=k$, we have 
      \begin{align}
        (a + b)^{k+1} & = (a + b)^k (a + b) \\
        & = \bigg( \sum_{j=0}^k \binom{k}{j} a^j b^{k-j} \bigg) (a + b) \\
        & = \sum_{j=0}^k \binom{k}{j} a^{j+1} b^{k-j} + \sum_{j=0}^k \binom{k}{j} a^j b^{k-j + 1} \\
        & = \binom{k}{0} a^0 b^{k+1} + \binom{k}{k} a^{j+1} b^0 + \sum_{j=0}^{k-1} \binom{k}{j} a^{j+1} b^{k-j} + \sum_{j=1}^k \binom{k}{j} a^j b^{k-j+1} \\
        & = \binom{k+1}{0} a^0 b^{k+1} + \binom{k+1}{k+1} a^{j+1} b^0 + \sum_{j=1}^k \bigg[ \binom{k}{j-1} + \binom{k}{j} \bigg] \, a^j b^{k-j + 1} \\
        & = \sum_{j=0}^{k+1} \binom{k+1}{j} a^j b^{k-j+1} 
      \end{align}
    \end{enumerate}
  \end{solution}

  \begin{exercise}[Zorich 2.2.3]
    Show that an inductive set is not bounded above. 
  \end{exercise}
  \begin{solution}
    Assume that a $X$ is a nonempty inductive set that is bounded above. By definition, there exists a number $B \in \mathbb{R}$ such that $\max{X} < B$. Then, this means that there exists no numbers in $[B, B + 1)$. Since $X$ is inductive, this means that there cannot exist any elements of $X$ in the interval $[B-1, B)$, and similarly for the interval $[B-2, B)$, and so on, meaning that if $x \in X$, then $x \not\in [B-k, B-k + 1)$ for all $k\in \mathbb{Z}$. By the Archimidean principle, this implies that $X = \emptyset$, contradicting our assumption. 
  \end{solution}

  \begin{exercise}[Zorich 2.2.4]
    Prove the following. 
    \begin{enumerate}
      \item An inductive set is infinite (that is, equipollent with one of its subsets different from itself). 
      \item The set $E_n = \{x \in \mathbb{N}\,|\, x \leq n\}$ is finite. 
    \end{enumerate}
  \end{exercise}
  \begin{solution}
    Listed. 
    \begin{enumerate}
      \item Assume that an inductive set $X$ is finite $\implies$ $X$ is bounded above (we can choose upper bound $B = \max{X} + 1$). But from 2.2.3, an inductive set cannot be bounded above, contradicting our assumption. 
      \item It is trivial that $E_1 = \{1\}$ is finite since $\text{card}{E_1} = 1$. Now, if for some $k$, $E_k$ is finite with cardinality $e_k$, then $\text{card}{E_{k+1}} = e_k + 1$, which implies finiteness.  
    \end{enumerate}
  \end{solution}

  \begin{exercise}[Zorich 2.2.5]
    Listed. 
    \begin{enumerate}
      \item Let $m, n \in \mathbb{N}$ and $m >n$. Their greatest common divisor $\gcd(m, n) = d \in \mathbb{N}$ can be found in a finite number of steps using the following algorithm of Euclid involving successive divisions with remainder. 
      \begin{align*}
        m & = q_1 n + r_1 \\
        n & = q_2 r_1 + r_2 \\
        r_1 & = q_3 r_2 + r_3 \\
        \ldots & = \ldots \\
        r_{k-2} & = q_{k} r_{k-1} + r_{k} \\
        r_{k-1} & = q_{k+1} r_k + 0
      \end{align*}
      Then $d=r_k$ 
      \item If $d = \gcd(m, n)$, one can choose numbers $p, q \in \mathbb{Z}$ such that $pm + qn = d$. 
    \end{enumerate}
  \end{exercise}
  \begin{solution}
    Listed. 
    \begin{enumerate}
      \item 
      \item Letting $n = r_0$, notice that the equations above satisfy for $i=0, 1, \ldots$
      \begin{equation}
        r_i = q_{i+2} r_{i+1} + r_{i+2} \implies r_{i} - q_{i+2} r_{i+1} = r_{i+2} \tag{1} \label{Euclid}
      \end{equation}
      Note that the second-to-last equation allows us to write $r_k$ as a linear combination of $r_{k-2}$ and $r_{k-1}$: $r_k = r_{k-2} - q_k r_{k-1}$. Now by applying \eqref{Euclid}, we can reduce the above to a linear combination of $r_{k-3}$ and $r_{k-2}$. 
      \begin{align*}
        r_k & = r_{k-2} - q_k r_{k-1} \\
        & = r_{k-2} - q_k (r_{k-3} - q_{k-1} r_{k-3}) \\
        & = (1 + q_{k-1} q_k) r_{k-2} - q_k r_{k-3} 
      \end{align*}
      and repeatedly doing this allows us to reduce $r_k$ to a linear combination $q_0 r_0 + q_1 r_1$. By the ring properties of $\mathbb{Z}$, the new linear coefficients are also in $\mathbb{Z}$. Reducing one last time using the first equation in the Euclidean algorithm gives 
      \begin{align*}
        r_k & = q_0 r_0 + q_1 r_1 \\
        & = q_0 n + q_1 (m - q_1 n) \\
        & = q_1 m + (q_0 - q_1) n \\
        & = p m + q n 
      \end{align*}
    \end{enumerate}
  \end{solution}

  \begin{exercise}[Zorich 2.2.9]
    Show that if the natural number $n$ is not of the form $k^m$, where $k, m \in \mathbb{N}$, then the equation $x^m = n$ has no rational roots. 
  \end{exercise}
  \begin{solution}
    Assume that there is a rational solution $x = p/q$, with $p, q \in \mathbb{N}$ of the equation. Then, 
    \begin{equation}
      \bigg( \frac{p}{q}\bigg)^m = \frac{p^m}{q^m} = n \implies p^m = q^m n
    \end{equation}
    By the fundamental theorem of arithmetic, the exponents of the prime factors of $p^m$ must all be multiples of $m$, and so it must be so for the right hand side $\implies x$ must be of form $x = k^m$ for some $k$. This is a contradiction. 
  \end{solution}

  \begin{exercise}[Zorich 2.2.12]
    Knowing that $\frac{m}{n} \equiv m \cdot n^{-1}$ by definition, where $m \in \mathbb{Z}$ and $n \in \mathbb{N}$, derive the ``rules'' for addition, multiplication, and division of fractions, and also the condition for two fractions to be equal. 
  \end{exercise}
  \begin{solution}
    We can construct a $\mathbb{Q}$ as a quotient space $\mathbb{Z} \times \mathbb{N} / \sim$, where $\sim$ is an equivalent relation where 
    \begin{equation}
      (q_1, p_1) \sim (q_2 , p_2) \text{ iff } q_1 p_2 = p_1 q_2
    \end{equation}
    which is the familiar equivalence relation from ``simplifying'' a fraction. We define addition and multiplication as the following 
    \begin{align*}
      (a, b) + (c, d) & = (ad + bc, bd) \\
      (a, b) \cdot (c, d) & = (ac, bd) 
    \end{align*}
    which turns out to be algebraically closed in $\mathbb{Q}$. The additive identity is the equivalence class $0 = \{(0, c) \,|\, c \in \mathbb{N}\}$, and the multiplicative identity is the equivalence class $1 = \{(c, c) \,|\, c \in \mathbb{N}\}$. It is easy to check that $+$ is commutative, the additive inverse is $-(a, b) = (-a, b)$, and the multiplicative inverse is $(a, b)^{-1} = (b, a)$. We can subtract and divide these elements of $\mathbb{Q}$, called ``fractions,'' as such: 
    \begin{align*}
      (a, b) - (c, d) & = (a, b) + (-(c, d)) = (a, b) + (-c, d) = (ad - bc, bd) \\
      (a, b) \div (c, d) & = (a, b) \cdot (c, d)^{-1} = (a, b) \cdot (d, c) = (ad, bc) 
    \end{align*}
  \end{solution}

  \begin{exercise}[Zorich 2.2.13]
    Verify that the rational numbers $\mathbb{Q}$ satisfy all the axioms for real numbers except for the axiom of completeness. 
  \end{exercise}
  \begin{solution}
    From continuing the steps of 2.2.14, we can prove $\mathbb{Q}$ is an algebraic field (associativity, commutativity of addition and multiplication, along with distributive property).We can actually define the order relation $\leq_\mathbb{Q}$ in two ways: 
    \begin{enumerate}
      \item $(a, b) \leq (c, d)$ iff $ad \leq_{\mathbb{Z}} bc$, where $\leq_{\mathbb{Z}}$ is the order relation on $\mathbb{Z}$ (which can be defined much more simply). 
      \item Recognizing that $\mathbb{Q} \subset \mathbb{R}$, we define the canonical injection map $i: \mathbb{Q} \longrightarrow \mathbb{R}$ and by abuse of language, endow the relation $\leq_{\mathbb{Q}}$ as the restriction of $\leq_{\mathbb{R}}$ onto $\mathbb{Q}$. That is, for $(a, b), (c, d) \in \mathbb{Q}$, 
      \begin{equation}
        (a, b) \leq_{\mathbb{Q}} (c, d) \text{ iff } i(a, b) \leq_{\mathbb{R}} i(c, d)
      \end{equation}
    \end{enumerate}
    The ordering for the 1st step can be checked for consistency. 
    \begin{enumerate}
      \item $(a, b) \leq (a, b)$ since $ab \leq ab$ (true in $\mathbb{Z}$) 
      \item $(a, b) \leq (c, d), (c, d) \leq (a, b)$ means that $ad \leq bc$ and $bc \leq ad \implies ad = bc$ (true in $\mathbb{Z}$) 
      \item $(a, b) \leq (c, d) \leq (e, f)$ implies $ad \leq bc, cf \leq de$. Multiplying positive (important that $f >0$!) to the first inequality gives $adf \leq bcf$, and multiplying positive $b$ to the second gives $bcf \leq bde$, and by interpreting $\leq$ as the ordering defined on $\mathbb{Z}$, we use transitive property of $\leq_\mathbb{Z}$ to get $adf \leq bde \implies af \leq be \iff (a, b) \leq (e, f)$. 
      \item For any $(a, b), (c, d) \in \mathbb{Q}$, $(a, b) \leq (c, d)$ or $(a, b) \geq (c, d)$, which is equivalent to $ad \leq bc$ or $ad \geq bc$, which is true in $\mathbb{Z}$. 
    \end{enumerate}
    It is easy to prove $(a, b) \leq (c, d) \implies (a, b) + (p, q) \leq (c, d) + (p, q)$, and $0_{\mathbb{Q}} \leq (a, b), (c, d) \implies 0_\mathbb{Q} \leq (a, b) \cdot (c, d)$. However, $\mathbb{Q}$ is \textbf{not} complete. We prove this by showing that the subset $X = \{x \in \mathbb{Q} \,|\,x^2 \leq 2 \} \subset \mathbb{Q}$ does not satisfy the least upper bound property. Assume that there is a least upper bound $c \in \mathbb{Q}$. $c \neq \sqrt{2}$ (you should know how to prove irrationality of $\sqrt{2}$!), we have either $c > \sqrt{2}$ or $c < \sqrt{2}$. 
    \begin{enumerate}
      \item Let $c < \sqrt{2} \iff c - \sqrt{2} > 0$. By the Archimidean principle, there exists a $k \in \mathbb{N}$ such that $0 < \frac{1}{k} < c - \sqrt{2}$. Then, $\frac{1}{k} \in \mathbb{Q}$ and $\mathbb{Q}$ is a field, so $c - \frac{1}{k} \in \mathbb{Q}$. 
      \begin{equation}
        c - \frac{1}{k} < c - c + \sqrt{2} = \sqrt{2}
      \end{equation}
      So $c$ is not least and so it must be the case that $c < \sqrt{2}$. 
      \item Let $c < \sqrt{2} \iff \sqrt{2} - c > 0$. By the Archimidean principle, there exists a $k \in \mathbb{N}$ such that $0 < \frac{1}{k} < \sqrt{2} - c$. Then, $c + \frac{1}{k} \in \mathbb{Q}$ and 
      \begin{equation}
        c + \frac{1}{k} < c + (\sqrt{2} - c) = c
      \end{equation}
      So $c$ is not an upper bound. 
    \end{enumerate}
    Note that given a well-defined $c = \sup{X}$ and in the case where $c < \sqrt{2}$, we have $2 - c^2 > 0$, so we can choose a well-defined $\delta$ satisfying (by Archimidean principle) 
    \begin{equation}
      0 < \delta < \min \bigg\{1, \frac{2 - c^2}{2c + 1} \bigg\}
    \end{equation}
    which gives us 
    \begin{align*}
      (c + \delta)^2 & = c^2 + \delta(2c + \delta) & \\
      & < c^2 + \delta (2 c + 1) & (\delta < 1) \\
      & < c^2 + (2 - c^2) = 2 & 
    \end{align*}
    meaning that $c$ is not an upper bound. Similarly for when $c > \sqrt{2}$. 
  \end{solution}

  \begin{exercise}[Zorich 2.2.15]
    Prove the equivalence of these two statements. 
    \begin{enumerate}
      \item If $X$ and $Y$ are nonempty sets of $\mathbb{R}$ having the property that $x \leq y$ for every $x \in X, y \in Y$, then there exists $c \in \mathbb{R}$ such that $x \leq c \leq y$ for all $x \in X$ and $y \in Y$. 
      \item Every set $X \subset \mathbb{R}$ that is bounded above has a least upper bound. 
    \end{enumerate}
  \end{exercise}
  \begin{solution}
    Let $S_1$ be the first statement and $S_2$ the second. 
    \begin{enumerate}
      \item ($S_2 \implies S_1$). Let $X \subset \mathbb{R}$ be a set that is bounded above, and $Y$ is a set such that $x \leq y$ for all $x \in X, y \in Y$. Then, by LUB principle, there exists $c = \sup{X} \in \mathbb{R}$. Now, we claim that $c \leq y$ for all $y \in Y$. Assume it doesn't: then there exists $y^\prime \in Y$ such that $y^\prime < c$. But since we assumed $x \leq y$ for all $x \in X, y \in Y$, we have $x \leq y^\prime$ for all $x \in X$, which means that $y^\prime$ is an upper bound of $X$. But $y^\prime < c$, contradicting the given fact that $c$ was the least upper bound. 

      \item ($S_1 \implies S_2$). Given a nonempty set $X \subset \mathbb{R}$, we wish to show the existence of $\sup{X}$. We are guaranteed the existence of nonempty set $Y \subset \mathbb{R}$ such that $x \leq y$ for all $x \in X, y \in Y$, which implies that $X$ must be bounded above. Then, by $S_1$, there must exist a $c \in \mathbb{R}$ such that 
      \begin{equation}
        x \leq c \leq y \text{ for all } x \in X, y \in Y
      \end{equation}
      We claim that $c = \sup{X}$. It is an upper bound of $X$ since $x \leq c$ for all $x \in X$. It is least since the set of all upper bounds of $X$ is $Y$, and $c \leq y$ for all $y \in Y$. 
    \end{enumerate}
  \end{solution}

  \begin{exercise}[Olmsted 1.15]
    Prove \textbf{Dedekind's Theorem}: Let the real numbers be divided into two nonempty sets $A$ and $B$ such that (i) if $x \in A$ and if $y \in B$, then $x < y$ and (ii) if $x \in \mathbb{R}$ then either $x \in A$ or $x \in B$, then there exists a number $c$ (which may belong to either $A$ or $B$) such that any number less than $c$ belongs to $A$ and any number greater than $c$ belongs to $B$. 
  \end{exercise}
  \begin{solution}
    This is really the same statement as Zorich 2.2.15.a, the original statement of completeness, but with the extra condition that the sets $A = X, B = Y$ must be disjoint.
  \end{solution}

  \begin{exercise}[Olmsted 1.7]
    If $x$ is an irrational number, under what conditions on the rational numbers $a, b, c, d$ is $(ax + b)/(cx + d)$ rational? 
  \end{exercise}
  \begin{solution}
    Note that a trivial solution is $a = b = c = d = 1$ which gives $1$. Since 
    \begin{equation}
      \frac{ax + b}{cx + d} = \frac{acx + ad - ad + bc}{cx + d} = a + \frac{bc - ad}{cx + d} 
    \end{equation}
    for the above to be rational it is necessary that $1/(cx + d)$ is rational. But this cannot be the case, which leaves us with the condition that $bc = ad$. 
  \end{solution}

  \begin{exercise}[Olmsted 1.8]
    Prove that the system of integers satisfies the axiom of completeness. 
  \end{exercise}
  \begin{solution}
    Let $S \subset \mathbb{Z}$ be bounded from above. It must have a maximum element (justify?), call it $c$. Then we claim that $c \in \mathbb{Z}$ is the least upper bound. Being the maximum, it is an upper bound, and $c$ is least since the next smallest element is $c-1$, which is less than $c \in S$, and therefore cannot be an upper bound. 
  \end{solution}

  \begin{exercise}[Zorich 2.2.16/Olmsted 1.16]
    Prove the following. 
    \begin{enumerate}
      \item If $A \subset B \subset \mathbb{R}$, then $\sup{A} \leq \sup{B}$ and $\inf{A} \geq \inf{B}$. 
      \item Let $\mathbb{R} \supset X \neq \emptyset$ and $R \supset Y \neq \emptyset$. If $x \leq y$ for all $x \in X, y \in Y$, then $X$ is bounded above , $Y$ is bounded below, and $\sup{X} \leq \inf{Y}$. 
      \item If the sets $X, Y$ in (b), are such that $X \cup Y = \mathbb{R}$, then $\sup{X} = \inf{Y}$. 
      \item If $X$ and $Y$ are the sets defined in (c), then either $X$ has a maximal element or $Y$ as a minimal element. 
      \item Show that Dedekind's theorem is equivalent to the axiom of completeness. 
    \end{enumerate}
  \end{exercise}
  \begin{solution}
    Listed. 
    \begin{enumerate}
      \item Let 
      \begin{align*}
        A^\prime & = \{ x \in \mathbb{R} \,|\, x \geq a \; \forall a \in A\} \\
        B^\prime & = \{ x \in \mathbb{R} \,|\, x \geq b \; \forall b \in B\}
      \end{align*}
      where we can easily verify that $B^\prime \subset A^\prime$. By definition, we get $\sup{B} = \min{B^\prime}$ and $\sup{A} = \min{A^\prime}$. But since $B^\prime \subset A^\prime$, for any $b^\prime \in B^\prime$, there exists an $a^\prime \in A^\prime$ such that $a^\prime \leq b^\prime$, which implies that $\sup{B} = \min{B^\prime} \leq \min{A^\prime} = \sup{A}$. 

      \item $X$ is bounded above by any element of $Y$. $Y$ is bounded below by any element of $X$. By the completion axiom, there exists a $c \in \mathbb{R}$ such that
      \begin{equation}
        x \leq c \leq y \text{ for all } x \in X, y \in Y
      \end{equation}
      Since $c$ is an upper bound of $X$, $\sup{X} \leq c$ by definition, and since $c$ is a lower bound of $Y$, $\inf{Y} \geq c$ by definition. Therefore, $\sup{X} \leq c \leq \inf{Y}$. 

      \item From completeness there exists a $c \in \mathbb{R}$ such that $x \leq c \leq y$ for all $x \in X, y \in Y$. $Y$ is, by definition, the set of \textit{all} upper bounds of $X$ (i.e. \textit{every} upper bound of $X$ is in $Y$, unlike $Y$ defined in 2.2.16.b). Since $c \leq y$ for all $y \in Y$, $c$ is minimal and so $c = \sup{X}$. $X$ is the set of all lower bounds of $Y$ by definition, so $c \geq x$ for all $x \in X \implies c = \inf{Y}$. So, $\inf{Y} = c = \sup{X}$. 

      \item We know that there exists $c = \inf{Y} = \sup{X}$. Since $X \cup Y = \mathbb{R}$, $c$ must be in at least $X$ or $Y$. If $c \in X$, then $c = \sup{X} = \max{X}$, and if $c \in Y$, then $c = \inf{Y} = \min{Y}$. 
      \item This is the same statement as Zorich 2.2.15.a (an iff equivalence, not just one way implying). 
    \end{enumerate}
  \end{solution}

  \begin{exercise}[Olmsted 1.13]
    Let $S$ be a nonempty set of numbers bounded above, and let $x$ be the least upper bound of $S$. Prove that $x$ has the two properties corresponding to an arbitrary positive number $\epsilon$: 
    \begin{enumerate}
      \item every element $s \in S$ satisfies the inequality $s < x + \epsilon$
      \item at least one element $s \in S$ satisfies the inequality $s > x - \epsilon$
    \end{enumerate}
  \end{exercise}
  \begin{solution}
    Listed. 
    \begin{enumerate}
      \item $x$ is an upper bound $\implies s \leq x$ for all $s \in S$, which implies that $s \leq x < x + \epsilon$. 
      \item By definition, $x - \epsilon$ cannot be an upper bound, so $x - \epsilon \geq s$ for all $s \in S$ is not true. Therefore, there must exist one $s \in S$ such that $s > x - \epsilon$. 
    \end{enumerate}
  \end{solution}

  \begin{exercise}[Zorich 2.2.18]
    Let $-A$ be the set of numbers of the form $-a$, where $a \in A \subset \mathbb{R}$. Show that $\sup(-A) = -\inf(A)$. 
  \end{exercise}
  \begin{solution}
    If $A$ is unbounded below, then $-\inf{A} = \infty$ and $-A$ is unbounded above, implying that $\sup{A} = \infty$. Now assume that $A$ is bounded below, then by completeness, it must have a greatest lower bound. Let us define the set $B = \{b \in \mathbb{R} \,|\, b \leq a \; \forall a \in A\}$. From 2.2.16.b, we have $b \leq \inf{A} \leq a$ for all $a \in A, b \in B$. Multiplying by $-1$ gives $-b \geq -\inf{A} \geq -a$ for all $a \in A, b \in B$, which is equivalent to saying 
    \begin{equation}
      a \leq -\inf{A} \leq b \text{ for all } a \in -A, b \in -B
    \end{equation}
    by definition of $-A, -B$. $-\inf{A}$ is clearly an upper bound of $-A$, and since  
    \begin{align*}
      B & = \{b \in \mathbb{R}\,|\, b \leq a \; \forall a \in A\} \\
      & = \{b \in \mathbb{R}\,|\, -b \geq -a \; \forall a \in A\} \\
      & = \{b \in \mathbb{R}\,|\, -b \geq a \; \forall a \in -A\} 
    \end{align*}
    implies that $-B = \{b \in \mathbb{R}\,|\, b \geq a \; \forall a \in -A\}$ is the set of all upper bounds of $A$. So, $-\inf{A}$ is the least upper bound of $-A$, i.e. $-\inf{A} = \sup(-A)$. 
  \end{solution}

  \begin{exercise}[Zorich 2.2.21]
    Show that the set $\mathbb{Q}(\sqrt{n})$ of numbers of the form $a + b \sqrt{n}$ where $a, b \in \mathbb{Q}$, $n$ is a fixed natural number that is not the square of any integer, is an ordered set satisfying the principle of Archimedes but not the axiom of completeness. 
  \end{exercise}
  \begin{solution}
    The order on $\mathbb{Q}(\sqrt{n})$ can be embedded from the ordering on the reals by defining the canonical injection map $i:\mathbb{Q}(\sqrt{n}) \longrightarrow \mathbb{R}$ and defining for any $x, y \in \mathbb{Q}(\sqrt{n})$, 
    \begin{equation}
      x \leq_{\mathbb{Q}(\sqrt{n})} y \iff i(x) \leq_{\mathbb{R}} i(y)
    \end{equation}
    Now, let $h > 0$ be any fixed real number, and $x = (a,b) = a + b\sqrt{n}$. By the Archimidean principle, we can find a $k \in \mathbb{Z}$ such that
    \begin{equation}
      (k - 1) h \leq x \leq k h \text{ for some } x \in \mathbb{Q}(\sqrt{n}) \subset \mathbb{R}
    \end{equation}
    We now show that $\mathbb{Q}(\sqrt{n})$ is not complete since it doesn't satisfy the LUB property. Since there are infinite prime numbers in $\mathbb{N}$, choose a prime number $p$ that is not a factor of $n$. Then, we are guaranteed that $pn$ is not a perfect square, and can define the set 
    \begin{equation}
      X = \{x \in \mathbb{Q}(\sqrt{n})\,|\, x < \sqrt{pn} \} \subset \mathbb{Q}(\sqrt{n})
    \end{equation}
    and assume that $c = c_1 + c_2 \sqrt{n} = \sup{X}$ exists ($c_1, c_2 \in \mathbb{Q}$). Clearly, $c \neq \sqrt{pn} \not\in \mathbb{Q}(\sqrt{n})$. 
    \begin{enumerate}
      \item Assume $c < \sqrt{pn} \iff 0 < \sqrt{pn} - c \in \mathbb{R}$. By the Archimidean principle, there exists a $k \in \mathbb{N}$ such that $0 < \frac{1}{k} < \sqrt{pn} - c $. Then, we can verify that $c + \frac{1}{k} = (c_1 + \frac{1}{k}) + c_2 \sqrt{n} \in \mathbb{Q}(\sqrt{n})$ and 
      \begin{equation}
        c + \frac{1}{k} < c + \sqrt{pn} - c = \sqrt{pn} \implies c + \frac{1}{k} \in X
      \end{equation}
      implies that $c$ is not an upper bound. So we must turn to case 2. 
      \item Assume $c > \sqrt{pn} \iff c - \sqrt{pn} > 0$. By AP, there exists a $k \in \mathbb{N}$ such that $0 < \frac{1}{k} < c - \sqrt{pn}$. Then, we can verify that $c - \frac{1}{k} \in \mathbb{Q}(\sqrt{n})$ and 
      \begin{equation}
        c - \frac{1}{k} > c - c + \sqrt{pn} = \sqrt{pn}
      \end{equation}
      implies that $c - \frac{1}{k}$ is an upper bound of $X$, so $c$ is not least. 
    \end{enumerate}
    Therefore, by contradiction, $c$ does not exist. 
  \end{solution}

  \begin{exercise}[Zorich 2.2.22]
    Let $n \in \mathbb{N}$ and $n > 1$. In the set $E_n = \{0, 1, \ldots, n-1\}$, we define the sum and product of two elements as the remainders when the usual sum and product in $\mathbb{R}$ are divided by $n$. With these operations on it, the set $E_n$ is denoted $\mathbb{Z}_n$. 
    \begin{enumerate}
      \item Show that if $n$ is not a prime number, then there are nonzero numbers $m, k \in \mathbb{Z}_n$ such that $m \cdot k = 0$, i.e. there exist nonzero zero divisors. 
      \item Show that if $p$ is prime, then there are no zero divisors in $\mathbb{Z}_p$ and $\mathbb{Z}_p$ is a field. 
      \item Show that, no matter what the prime $p$, $\mathbb{Z}_p$ cannot be ordered in a way consistent with the arithmetic operations on it. 
    \end{enumerate}
  \end{exercise}
  \begin{solution}
    Listed. 
    \begin{enumerate}
      \item $n$ is composite implies that there exist $1 < m, k < n$ such that $n = m k$. These factors $m, k$ are precisely the zero divisors of $\mathbb{Z}_n$ since $m k = n \equiv 0 \pmod{n}$. 
      \item With $p$ prime, assume that there are nontrivial zero divisors $1 < m, k < p$ in $\mathbb{Z}_p$. Then, $m k \equiv 0 \pmod{n} \implies m k = l p$ for some $l \in \mathbb{N}$. But this implies that $m$ or $k$ must divide $p$, which is impossible since $1 < m, k < p$. Then prove field axioms. 
      \item For any field, we must have $0 \leq 1$, because if not, then 
      \begin{equation}
        0 > 1 \implies 0 < 1^{-1} \cdot 1 = 1^{-1} \implies 0 \cdot 0 < 1^{-1} \cdot 1^{-1} = 1
      \end{equation}
      So, $0 \leq 1$ implies that $0 \leq 1 \leq 2 \leq \ldots \leq p-1$. But 
      \begin{equation}
        0 + 1 \leq (p - 1) + 1 = 0
      \end{equation}
      is false, so any ordering is impossible. 
    \end{enumerate}
  \end{solution}

  \begin{exercise}[Zorich 2.2.23]
    Show that if $\mathbb{R}$ and $\mathbb{R}^\prime$ are two models of the set of real numbers and $f: \mathbb{R} \longrightarrow \mathbb{R}^\prime$ (with $f \not\equiv 0^\prime$) is a mapping such that $f(x + y) = f(x) + f(y)$ and $f(x \cdot y) = f(x) \cdot f(y)$ for any $x, y \in \mathbb{R}$. Prove that $f$ is an order-preserving isomorphism. 
  \end{exercise}
  \begin{solution}
    Let $0, 0^\prime$ be the additive identity of $\mathbb{R}, \mathbb{R}^\prime$, respectively, and $1, 1^\prime$ the multiplicative identity. We claim that $f(0) = 0^\prime$ since
    \begin{align*}
      f(0) & = f(0 + 0) & \text{(definition of additive identity)} \\
      & = f(0) + f(0) & (\text{homomorphism over } + )
    \end{align*}
    which implies that $f(0) + f(0) = f(0) = 0^\prime + f(0) $. Since $f(0)$ lives in field $\mathbb{R}^\prime$, its additive identity $-f(0)$ is well defined, and we get $f(0) = f(0) + f(0) + (-f(0)) = 0^\prime + f(0) + (-f(0)) = 0^\prime$. We also claim that $f(1) = 1^\prime$ since 
    \begin{align*}
      f(1) & = f(1 \cdot 1) & \text{(definition of multiplicative identity)} \\
      & = f(1) \cdot f(1) & (\text{homomorphism over } \cdot ) 
    \end{align*}
    which implies that $f(1) \cdot f(1) = 1^\prime \cdot f(1)$. Since $f(1)$ lives in field $\mathbb{R}^\prime$, its multiplicative identity $f(1)^{-1}$ is well defined, and we get $f(1) = f(1) \cdot f(1) \cdot f(1)^{-1} = 1^\prime \cdot f(1) \cdot f(1)^{-1} = 1^\prime$. Now that we have proved mapping of identities, this implies the mapping of inverses. 
    \begin{align*}
      0^\prime & = f(0) = f(x - x) = f(x) + f(-x) \implies f(-x) = -f(x) \\
      1^\prime & = f(1) = f(x \cdot x^{-1}) = f(x) \cdot f(x^{-1}) \implies f(x^{-1}) = f(x)^{-1}
    \end{align*}
    With these conditions, we have proved that $f$ is a homomorphism of fields. Now we prove that $f$ is a bijection, but first, we claim that $f(x) = 0^\prime \implies x = 0$. Assume that there exists a nonzero $x \in \mathbb{R}$ such that $f(x) = 0^\prime$. Then, $x^{-1}$ is well defined, and 
    \begin{align*}
      f(x) \cdot f(x^{-1}) & = f(x) \cdot f(x)^{-1} = 0^\prime \\
      f(x) \cdot f(x^{-1}) & = f(x \cdot x^{-1}) = f(1) = 1^\prime 
    \end{align*}
    which implies that $0^\prime = 1^\prime$. So, $f(1) = 1^\prime = 0^\prime$, and so for all $k \in \mathbb{R}$, $f(k) =f(k \cdot 1) = f(k) \cdot f(1) = f(k) \cdot 0^\prime = 0^\prime \implies f \equiv 0^\prime$, leading to a contradiction of the assumption that $f^\prime \not\equiv 0^\prime$. 
    \begin{enumerate}
      \item ($f$ injective). Assume $f$ is not injective, i.e. there exists distinct $x_1, x_2 \in \mathbb{R}$ s.t. $f(x_1) = f(x_2)$. Then, using that fact $f(x) = 0 \implies x = 0$, 
      \begin{equation}
        0 = f(x_1) - f(x_2) = f(x_1 - x_2) \implies x_1 - x_2 = 0 \implies x_1 = x_2
      \end{equation}
      \item ($f$ surjective). Let $y$ be any nonzero element in $\mathbb{R}^\prime$ (clearly if $y=0^\prime$ then its preimage is $0$) and $y^{-1}$ its multiplicative inverse. Assume there exists no $x \in \mathbb{R}$ satisfying $f(x) = y$, meaning that there exist no $x$ satisfying
      \begin{equation}
        f(x) \cdot  = y \cdot y^{-1} = 1^\prime
      \end{equation}
      But since $f$ maps inverses to inverses, we can choose $x = (y^{-1})^{-1}$, which leads to 
      \begin{equation}
        f(x) \cdot y^{-1} = (
      \end{equation}
    \end{enumerate}
    Finally, we prove that $f$ is order preserving. Assume that $x \leq y \iff 0 \leq y - x$ , we wish to prove that 
    \begin{equation}
      f(x) \leq f(y) \iff 0 \leq f(y) - f(x) = f(y - x)
    \end{equation}
    Therefore, since this preservation of ordering is really the statement $0 \leq y - x \implies 0 \leq f(y - x)$, it suffices to prove that $0 \leq x \implies 0 \leq f(x)$. Now, assume that we have a $x$ such that $f(x) < 0^\prime$. Adding it with the equation $f(1) = 1^\prime$ gives us 
    \begin{equation}
      f(x + 1) < 1^\prime
    \end{equation}
    It is easy to prove that $0 \leq x \iff 0 \leq x^{-1}$. Now assume that $0 > f(x)$. \textbf{INCOMPLETE}
  \end{solution}

  \begin{exercise}[Density of Rationals in $\mathbb{R}$]
    Prove that for any two distinct $a < b \in \mathbb{R}$, there exists an infinite number of rational numbers between $a$ and $b$. 
  \end{exercise}
  \begin{solution}
    Since $a< b$, then $b - a > 0$ and by the Archimidean principle, there exists a $k \in \mathbb{N}$ such that 
    \begin{equation}
      0 < \frac{1}{k} < b - a \implies 1 < kb - ka
    \end{equation}
    which implies that the length of $[ka, kb)$ greater than $1$. By the inductive property of $\mathbb{Z}$, there must be an integer $p \in [ka, kb)$. If there were not, then this would imply that $[ka+1, kb+1)$ and $[ka-1, kb-1)$ had no integers and repeating would mean that there were no integers in $\mathbb{R}$. Therefore, 
    \begin{equation}
      ka \leq p < kb \implies a \leq \frac{p}{k} < b
    \end{equation}
    for all $a, b \in \mathbb{R}$, with $p/k \in \mathbb{Q}$. If $a$ is irrational we can replace the $\leq$ to $<$, leaving $a \leq \frac{p}{k} < b$, and if $a$ is rational, we can construct another rational $a + \frac{1}{k} \in (a, b)$. 
  \end{solution}

  \begin{exercise}[Nested Interval Lemma]
    With the fact that $\mathbb{R}$ is complete, prove the following. 
    \begin{enumerate}
      \item For a sequence of closed nested intervals $I_1 \supset I_2 \supset \ldots$ of $\mathbb{R}$, there exists a point $c \in \mathbb{R}$ belonging to all these intervals. 
      \item Furthermore, if the hypothesis also satisfies the fact that for any $\epsilon > 0$, there exists a $k \in \mathbb{N}$ such that $|I_k| < \epsilon$ (i.e. the length of the intervals decreases to $0$), then the point $c$ common to all sets is unique. 
    \end{enumerate}
  \end{exercise}
  \begin{solution}
    Listed. 
    \begin{enumerate}
      \item Let $I_n = [a_n, b_n]$, with $a_n < b_n$ finite for all $n \in \mathbb{N}$. For all $n\in \mathbb{N}$, we have $I_n = [a_n, b_n]$ and can take the two subsets $X_n = (-\infty, a_n)$ and $Y_n = (b_n, \infty)$, where $x \leq y$ for every $x \in X_n, y \in Y_n$. We also have the fact that $\mathbb{R} = X_n \cup I_n \cup Y_n$. Since $\mathbb{R}$ is complete, there exists a $c$ such that $x \leq c \leq y$ for all $x \in X, y \in Y$. But $x \leq c \iff c \not\in X_n$ and $c \leq y \iff c \not\in Y_n$, so for all $n \in \mathbb{N}$, $c$ must be in $I_n$. 
      
      \item Since we have proved (a), it now suffices to prove uniqueness of $c$. Let there be two distinct points $c_1, c_2 \in \mathbb{R}$ belonging to these intervals. Without loss of generality, assume $c_1 - c_2 > 0$, and choose 
      \begin{equation}
        \epsilon = \frac{c_1 - c_2}{3}
      \end{equation}
      Then, there should exist a $k \in \mathbb{N}$ such that $|I_k| < \epsilon$. Since $I_k$ must contain $c_1$, it must be a subset of $[c_1 - \epsilon, c_1 + \epsilon]$ (should be able to see why) and similarly for $c_2$. 
      \begin{align*}
        I_k & \subset \bigg[ c_1 - \frac{c_1 - c_2}{3}, c_1 + \frac{c_1 - c_2}{3} \bigg] = \bigg[ \frac{2c_1 + c_2}{3}, \frac{4c_1 - c_2}{3} \bigg] = L\\
        I_k & \subset \bigg[ c_2 - \frac{c_1 - c_2}{3}, c_2 + \frac{c_1 - c_2}{3} \bigg] = \bigg[ \frac{-c_1 + 4 c_2}{3}, \frac{c_1 + 2c_2}{3} \bigg] = M
      \end{align*}
      But since $c_1 > c_2 \implies \frac{c_1 + 2c_2}{3} < \frac{2c_1 + c_2}{3}$, $L$ and $M$ are disjoint $\implies I_k$, as a subset of both, leaves us with $I_k = \emptyset$, contradicting that it is a closed interval. 
    \end{enumerate}
  \end{solution}

  \begin{exercise}{Compactness of Closed Interval in $\mathbb{R}$}
    Prove that any system of open intervals covering (i.e. an open cover of) a closed interval contains a finite subsystem that covers the closed interval. Another way to state this is by saying that every closed interval of $\mathbb{R}$ is compact. 
  \end{exercise}
  \begin{solution}
    A closed interval with a finite open covering is trivially compact since any subcovering is also finite. We only need to deal with when a closed interval $I = [a, b]$ has an infinite open covering $\{U_\alpha \}_{\alpha \in A}$, which means that the set of indices $A$ is infinite. Assume that there exists no finite covering of $I$. Then, we divide $I$ into two halves 
    \begin{equation}
      I_1 = \Big[ a, \frac{a + b}{2}\Big], \;\; I_2 = \Big[ \frac{a + b}{2}, b \Big]
    \end{equation}
    and define a subcovering for each of them. That is, we can define $A_1 \subset A$ and $A_2 \subset A$ such that $\{U_\alpha\}_{\alpha \in A_1} \subset \{U_\alpha \}_{\alpha \in A}$ is a covering of $I_1$ and $\{U_\alpha\}_{\alpha \in A_2} \subset \{U_\alpha \}_{\alpha \in A}$ is a covering of $I_2$. At least one of $A_1$ or $A_2$ must be infinite, since if they were both finite, then we can define a finite covering $\{U_\alpha\}_{\alpha \in A_1 \cup A_2}$ of $I$. Choose the interval with the infinite covering and repeat this procedure, which will result in a nested interval that decreases in length by a half. 
    \begin{equation}
      I \supset I_1 \supset I_2 \supset \ldots
    \end{equation}
    By the nested interval lemma, there exists a unique point $c$ common to all these intervals. But since $c \in [a, b]$, the open cover $\{U\}$ should contain an open interval $(c - \delta_1, c + \delta_2)$ containing $c$. We wish to prove that this interval is a superset of some $I_k$ in the sequence above, contradicting the fact that $I_k$ has an infinite cover. Since the length of each $I_i$ decreases arbitrarily (i.e. we can choose any $\epsilon > 0$ and find a $I_k$ with length less than $\epsilon$), we choose $\epsilon = \frac{1}{2} \min\{\delta_1, \delta_2 \}$, and we should be able to find some $I_k$ that is a subinterval of $[c - \epsilon, c + \epsilon]$, which itself is a subinterval of $(c - \delta_1, c + \delta_2)$. 
    \begin{equation}
      I_k \subset \Big[ c - \frac{1}{2} \min\{\delta_1, \delta_2 \}, c + \frac{1}{2} \min\{\delta_1, \delta_2 \} \Big] \subset (c - \delta_1, c + \delta_2)
    \end{equation}
    Therefore, $(c - \delta_1, c + \delta_2)$ is a finite cover of $I_k$, contradicting the fact that all $I_k$'s have infinite covers. 
  \end{solution}

  \begin{exercise}[Bolzano-Weierstrass Theorem]
    Prove that every bounded infinite set of real numbers has at least one limit point. (A limit point $p$ of set $X$ is a point such that every open neighborhood of $p$ contains an infinite number of elements of $X$). 
  \end{exercise}
  \begin{solution}
    Let the set of points be denoted $X$, and let $a$ be the lower bound and $b$ be the upper bound. Then, $X \subset [a, b] = I$. Now divide $[a, b]$ into halves $[a, \frac{a + b}{2}] \cup [\frac{a + b}{2}, b]$. At least one of the halves must have an infinite number of points; choose the interval with infinite points as $I_1$ and doing this repeatedly gives the nested sequence 
    \begin{equation}
      I \supset I_1 \supset I_2 \supset \ldots
    \end{equation}
    By the nested interval lemma, there exists at least one point $c \in \mathbb{R}$ that is in all these intervals. Furthermore, since $|I_i| = \frac{1}{2^i} (b - a)$ decreases to $0$, we can choose a $\epsilon > 0$ and find an interval $I_k$ with $|I_k| < \epsilon$. We claim that $c$ is a limit point of $X$. Given an $\epsilon$, we wish to prove that there are an infinite number of points within the $\epsilon$-neighborhood $(c - \epsilon, c + \epsilon)$ of $c$. Since we can find some $I_k$ with $|I_k| < \epsilon$, we can see that 
    \begin{equation}
      I_k \subset (c - \epsilon, c + \epsilon)
    \end{equation}
    and therefore the $\epsilon$-neighborhood of $c$ contains $I_k$, which contains an infinite number of points in $X$. 

    We can construct another proof that is dependent on the compactness lemma. This construction will be useful for problem 2.3.4. Let $X$ be a given subset of $\mathbb{R}$, and it follows from the definition of boundedness that $X$ is contained in some closed interval $I \subset \mathbb{R}$. We show that at least one point of $I$ is a limit point of $X$. Assume that it is not. Then each point $x \in I$ would have a neighborhood $U(x)$ containing at most a finite number of points from $X$. The totality of such neighborhoods $\{U(x)\}$ constructed for the points $x \in I$ forms an open covering of $X$. Since $I$ is closed, it is compact and therefore we can find a finite subcovering $\{U_i(x)\}_{i}$ of open intervals that cover $I$ and therefore cover $X$. This open cover $\{U_i(x)\}_{i}$ of $X$ is a finite union of sets that each contain at most a finite number of points from $X$, so the covering of $X$ contains a finite number of points from $X$, a contradiction that $X$ contains infinite points. 
  \end{solution}

  \begin{exercise}[Zorich 2.3.1]
    Show that 
    \begin{enumerate}
      \item if $I$ is any system of nested closed intervals, then 
      \[\sup\{ a \in \mathbb{R}\,|\, [a, b] \in I\} = \alpha \leq \beta = \inf\{ b \in \mathbb{R}\,|\,[a, b] \in I\}\]
      and 
      \[[\alpha, \beta] = \bigcap_{[a, b] \in I} [a, b]\]
      \item if $I$ is a system of nested open intervals $(a, b)$, the intersection
      \[\bigcap_{(a, b) \in I} (a, b)\] 
      may happen to be empty. 
    \end{enumerate}
  \end{exercise}
  \begin{solution}
    Listed. 
    \begin{enumerate}
      \item (May be tempted to say that $a_1 \leq a_2 \leq \ldots$, but this assumes that the indexing set $I$ is countable). We claim that for any two intervals $[a_n, b_n]$ and $[a_m, b_m]$ in $I$, 
      \[a_n \leq b_m\]
      Assume that $a_n > b_m$. Then $b_n \geq a_n > b_m \geq a_m$ implies that $[a_n, b_n]$ and $[a_m, b_m]$ are disjoint, contradicting the fact that they are nested. Now given that $X$ is the set of $a_n$'s and $Y$ is the set of $b_n$'s, we have $x \leq y$ for all $x \in X, y \in Y$. So by 2.2.16.b, we have $\sup{X} \geq \inf{Y}$. 
      \\
      To prove the second statement, we show that trying to ``expand'' the interval $[\alpha, \beta]$ will lead to a contradiction. Since $\alpha$ is the LUB, given any $\epsilon > 0$, there exists a $(a_l, b_l) \in X$ such that $\alpha - \epsilon < a_l < \alpha$, which implies that $ [\alpha, \beta] \subset [a_l, \beta] \subset [\alpha - \epsilon, \beta]$. Assuming that this extended interval is the intersection, we should be able to choose any point in $[\alpha - \epsilon, \beta]$ and find that it is in every element of $I$. We choose a point in $[\alpha - \epsilon, a_l)$, which is not in the interval $(a_l, b_l)$. We do the same for $\beta \mapsto \beta + \epsilon$. We also check that ``shrinking'' the interval $[\alpha, \beta] \mapsto [\alpha + \epsilon, \beta]$ is no good, since we can find an element in $[\alpha, \alpha + \epsilon)$ that is in every interval in $I$. 
      
      \item Take the system of nested open intervals 
      \[(0, 1) \supset (0, \frac{1}{2}) \supset (0, \frac{1}{3}) \ldots (0, \frac{1}{n}) \supset \ldots\]
      Take their infinite intersection, denote it $S$, and assume that some $\epsilon \in (0, 1)$ is in $S$. Since $\epsilon$ is a real number, by the Archimidean principle there exists a $k \in \mathbb{N}$ such that $\frac{1}{k} < \epsilon$. Therefore, $\epsilon \not\in (0, \frac{1}{k}) \implies \epsilon \not\in S$. 
    \end{enumerate}
  \end{solution}

  \begin{exercise}[Zorich 2.3.2]
    Show that 
    \begin{enumerate}
      \item from a system of closed intervals covering a closed interval it is not always possible to choose a finite subsystem covering the interval. 
      \item from a system of open intervals covering a open interval it is not always possible to choose a finite subsystem covering the interval. 
      \item from a system of closed intervals covering a open interval it is not always possible to choose a finite subsystem covering the interval. 
    \end{enumerate}
  \end{exercise}
  \begin{solution}
    We show with the interval $(0, 1)$ or $[0, 1]$. Using linear transformations it is easy to generalize this to any other interval $(a, b)$ or $[a, b]$. 
    \begin{enumerate}
      \item Consider the infinite covering
      \[[0, 1] = \big[0, \frac{1}{2}\big] \cup \big[\frac{1}{2}, \frac{3}{4}\big] \cup \big[\frac{3}{4}, \frac{7}{8}\big] \cup \ldots \]
      \item Consider the infinite covering 
      \[(0, 1) = \big(0, \frac{1}{2}\big) \cup \big(\frac{1}{2}, \frac{3}{4}\big) \cup \big(\frac{3}{4}, \frac{7}{8}\big) \cup \ldots \]
      \item Consider the infinite covering 
      \[(0, 1) = \big[0, \frac{1}{2}\big] \cup \big[\frac{1}{2}, \frac{3}{4}\big] \cup \big[\frac{3}{4}, \frac{7}{8}\big] \cup \ldots \]
    \end{enumerate}
  \end{solution}

  \begin{exercise}[Zorich 2.3.3]
    Show that if we only take the set $\mathbb{Q}$ of rational numbers instead of the complete set $\mathbb{R}$ of real numbers, with the definitions of closed, open, and neighborhood of a point $r \in \mathbb{Q}$ to mean respectively the corresponding subsets of $\mathbb{Q}$, then none of the three lemmas is true. 
  \end{exercise}
  \begin{solution}
    We prove only for the nested interval lemma. We choose the series of nested intervals 
    \[\bigg( \sqrt{2} - \frac{1}{n}, \sqrt{2} + \frac{1}{n} \bigg)\]
    with $n \in \mathbb{N}$. Assume that there is a $r \in \mathbb{Q}$ such that 
    \[r \in \bigg( \sqrt{2} - \frac{1}{n}, \sqrt{2} + \frac{1}{n} \bigg) \text{ for all } n \in \mathbb{N}\]
    which is equivalent to saying that $\big| r - \sqrt{2}\big| < \frac{1}{n}$ for all $n \in \mathbb{N}$. Clearly, $r \neq \sqrt{2}$, and by the Archimidean principle, there exists a $k \in \mathbb{N}$ such that 
    \[0 < \frac{1}{k} < |r - \sqrt{2}|\]
    which contradicts the above. 
  \end{solution}

  \begin{exercise}[Zorich 2.3.4]
    Show that the three lemmas above are equivalent to the axiom of completeness. 
  \end{exercise}
  \begin{solution}
    Note that from the proofs, completeness implies nested interval lemma, which implies compactness of closed intervals, which implies the Bolzano-Weierstrass theorem. So, it is sufficient to prove that Bolzano-Weierstrass theorem implies completeness to determine equivalence. There are not a lot of direct proofs, so we prove that Weierstrass implies nested interval, which implies completeness. 
    \begin{enumerate}
      \item (Weierstrass $\implies$ Nested) Assume that we have $\mathbb{R}$ with the Bolzano-Weierstrass theorem. Take the series of nested closed intervals 
      \[I = [a, b] \supset I_1 = [a_1, b_1] \supset I_2 = [a_2, b_2] \supset \ldots\]
      We see that $a \leq a_i \leq b$, so the infinite sequence of monotonically nondecreasing values $a_i$ is bounded. Therefore, it must have a limit point, which we will denote as $c$. We claim that $a_i \leq c$ for all $a_i$. Since if it were not, then $c < a_i$ for some $i$, and choosing $\epsilon = 0.5 (a_i - c)$, the $\epsilon$-neighborhood of $c$ will not contain $a_j$ for $j \geq i$ since 
      \[c < a_i \implies 0.5 c < 0.5 a_i \implies c + \epsilon = 0.5c + 0.5 a_i < a_i< a_{i+1} < \ldots\]. 
      With similar reasoning, we can conclude that $b_i \geq c$ for all $b_i$. This implies that $a_i \leq c \leq b_i$ for all $i$ which is equivalent to saying that $c \in [a_i, b_i] = I_i$ for all $i \in \mathbb{N}$. 
      
      \item (Nested $\implies$ LUB Principle) Let $X \subset \mathbb{R}$ be a set that is bounded above, with $b_1$ any upper bound. Since $X$ is nonempty, there exists $a_1 \in X$ that is not an upper bound (otherwise, $X$ would be a singleton set and it trivially has a least upper bound). Consider the well-defined interval $[a_0, b_0]$. Take the mean $m_0 = 0.5 (a_0 + b_0)$, and if $m_0$ is an upper bound, set it to $b_1$ (with $a_1 = a_0$) and $a_1$ if else (with $b_1 = b_0$). Then, we have a sequence of nested intervals 
      \[[a_0, b_0] \supset [a_1, b_1] \supset [a_2, b_2] \supset \ldots \]
      of decreasing lengths $|I_k| = \frac{1}{2^{k}} (b - a)$. All of them must contain a unique common point $c \in \mathbb{R}$ by the nested intervals lemma, which implies that 
      \[a_0 \leq a_1 \leq a_2 \leq \ldots \leq c \leq \ldots \leq b_2 \leq b_1 \leq b_0\]
      I claim two things: 
      \begin{enumerate}
          \item $c$ is an upper bound for $X$. Suppose it were not, then there exists some $x \in X$ such that $c < x$, and let the distance between them be $\epsilon = x - c > 0$. By AP, we can choose $k \in \mathbb{N}$ such that $\frac{1}{k} < \epsilon$. All the $b_n$ are upper bounds of $X$, so we have $x \leq b_n$. Subtracting $c$ on both sides gives 
          \[0 < x - c = \epsilon \leq b_n - c \leq |I_n| = \frac{1}{2^n} (b_0 - a_0)\]
          where the last inequality follows from $c \in I_n = [a_n, b_n]$, so the maximum distance it can be from the endpoint $b_n$ is $|I_n|$. The inequality above holds for all $n \in \mathbb{N}$, so increasing $n$ arbitrarily should decrease $\frac{1}{2^n} (b_0 - a_0)$ past $\epsilon$. To formalize this, we use the inequality
          \[\frac{1}{2^n} < \frac{1}{n} \text{ for all } n \in \mathbb{N}\]
          and so we have 
          \[\epsilon \leq b_n - c < \frac{1}{n} (b_0 - a_0)\]
          We choose the natural number $n = \lceil \frac{2(b_0 - a_0)}{\epsilon} \rceil$, which does not satisfy the inequality above since 
          \[\epsilon < \frac{1}{n} (b_0 - a_0) = \frac{1}{\lceil 2(b_0 - a_0)/\epsilon \rceil} (b_0 - a_0) \leq \frac{\epsilon}{2(b_0 - a_0)} (b_0 - a_0) = \frac{\epsilon}{2}\]
          This leads to a contradiction. 
          \item We now prove that $c$ is least. Assume that $c$ is not least $\implies$ there exists an upper bound $B$ such that $B < c$ and $x \leq B$ for all $x \in X$. \textbf{INCOMPLETE}
      \end{enumerate}
    \end{enumerate}
  \end{solution}

  \begin{exercise}[Zorich 2.4.1]
  Show that the set of real numbers has the same cardinality as the points of the interval $(-1, 1)$. 
  \end{exercise}
  \begin{solution}
    We define the bijective map $\rho: (-1, 1) \longrightarrow \mathbb{R}$ 
    \[p(x) = \begin{cases} 
    0 & \text{ if } x = 0 \\
    \frac{1}{x} & \text{ if } x \neq 0 
    \end{cases}\]
  \end{solution}

  \begin{exercise}[Zorich 2.4.2]
    Give an explicit one-to-one correspondence between 
    \begin{enumerate}
      \item the points of two open intervals 
      \item the points of two closed intervals 
      \item the point of a closed interval and an open interval 
      \item the points of the closed interval $[0, 1]$ and $\mathbb{R}$
    \end{enumerate}
  \end{exercise}
  \begin{solution}
    Listed. 
    \begin{enumerate}
      \item $\rho: (a, b) \longrightarrow (c, d)$ defined 
      \[\rho(x) = \frac{d - c}{b - a} (x - a) + c \]
      \item the extension of $\rho$ defined on (a) to $[a, b]$
      \item From (a) and (b), it suffices to prove a bijection from $(0, 1)$ to $[0, 1]$. We extract a countably infinite sequence from $(0, 1)$, say 
      \[x_1 = \frac{1}{3}, \; x_2 = \frac{1}{4}, \ldots, x_i = \frac{1}{i+2}\]
      Then, we define bijection $\rho: (0, 1) \longrightarrow [0, 1]$ as 
      \[\rho (x) = \begin{cases}
      x & \text{ if } x \not\in \{x_i\} \\
      0 & \text{ if } x = x_1 = \frac{1}{2} \\
      1 & \text{ if } x = x_2 = \frac{1}{3} \\
      x_{i-2} & \text{ if } x = x_i \text{ for } i > 2
      \end{cases}\]
      Colloquially, we extract a copy of $\mathbb{N}$ from $(0, 1)$ and use the bijection $\mathbb{N} \simeq \mathbb{N} \cup \{0, -1\}$ to ``shift'' the terms. 
      \item We compose the bijections $\rho_1 : [0, 1] \longrightarrow (0, 1)$ and $\rho_2: (0, 1) \longrightarrow \mathbb{R}$. 
    \end{enumerate}
  \end{solution}

  \begin{exercise}[Zorich 2.4.3]
    Show that 
    \begin{enumerate}
      \item every infinite set contains a countable subset
      \item the set of even integers has the same cardinality as the set of all natural numbers
      \item the union of an infinite set and an at most countable set has the same cardinality as the original infinite set. 
      \item the set of irrational numbers has the cardinality of the continuum 
      \item the set of transcendental numbers has the cardinality of the continuum
    \end{enumerate}
  \end{exercise}
  \begin{solution}
    Listed. 
    \begin{enumerate}
      \item Let $A$ be an infinite set. By axiom of choice, choose $a_0 \in A$. Then, $A \setminus \{a_0\} \neq \emptyset$ since $A$ is infinite. By induction, assume you have chosen $a_0, a_1, \ldots, a_k \in A$. Then, since $A$ is infinite, $A \setminus \{a_0, a_1, \ldots, a_k\} \neq \emptyset$, so we can choose $a_{k+1} \in A \setminus \{a_0, \ldots, a_k\}$. Thus, we have constructed a countable subset $\{a_k\}_{k \in \mathbb{N}}$ of $A$. 
      \item Given the quotient ring $2\mathbb{Z}$, define the bijection $\rho: 2\mathbb{Z} \longrightarrow \mathbb{N}$ as 
      \[p(x) = \begin{cases} 
      x + 2 & \text{ if } x \geq 0 \\
      -x - 1 & \text{ if } x < 0 
      \end{cases}\]
      \item From (a), we can extract a countable set from original set $A$, call it $X$. Since the product of countable sets is countable ($\mathbb{N} \cup \mathbb{N}$ is countable), we can define a bijection $\Tilde{\rho}: X \longrightarrow X \cup B$. Therefore, we can define a bijection $\rho: A \longrightarrow A \cup B$ as 
      \[\rho(x) = \begin{cases} 
      x & \text{ if } x \in A \setminus X \\
      \Tilde{\rho}(x) & \text{ if } x \in X
      \end{cases}\]
      \item $\mathbb{Q}$ is countable and $\mathbb{R}$ is uncountable. So, $\mathbb{R} \setminus \mathbb{Q}$ must be uncountable since if it were countable, then the union of the rationals and irrationals, which is $\mathbb{R}$, would be countable. 
      \item It suffices to prove that the set of algebraic numbers (numbers that are possible roots of a polynomial with integer coefficients with leading coefficient nonzero) is countable, since we can apply (d) right after. The set of all $k$th degree polynomials with integer coefficients is isomorphic to $\mathbb{Z}^k$ through the map 
      \[a_k x^k + a_{k-1} x^{k-1} + \ldots + a_2 x^2 + a_1 x^1 + a_0 \mapsto (a^{k-1}, a^{k-2}, \ldots, a_1, a_0)\]
      and the union of these countable sets (minus the $0$ map) 
      \[P = \bigg(\bigcup_{k = 1}^\infty \mathbb{Z}^k\bigg) \setminus \{0\} = \big(\mathbb{Z} \setminus \{0\}\big) \cup \mathbb{Z}^2 \cup \ldots \]
      is countable. For any element in $\mathbb{Z}^k$, there are at most $k$ real roots, and so we can define the set of roots of an element $z \in \mathbb{Z}^k \subset P$ as a $j$-tuple of algebraic numbers, which can have at most $j=k$ roots. 
      \[r(z) = \underbrace{(r_{1z}, r_{2z}, \ldots, r_{jz})}_{j \leq k}\]
      Therefore, the union of all these $j$-tuples for all $z \in P$ 
      \[\bigcup_{z \in P} r(z) = \bigcup_{k = 1}^\infty \bigcup_{z \in \mathbb{Z}^k} r(z)\]
      is a countable union of a countable union of finite sets, making it countable. 
    \end{enumerate}
  \end{solution}

  \begin{exercise}[Zorich 2.4.4]
    Show that
    \begin{enumerate}
      \item the set of increasing sequences of natural numbers has the same cardinality as the set of fractions of the form $0.\alpha_1 \alpha_2 \ldots$ 
      \item the set of all subsets of countable set has cardinality of the continuum
    \end{enumerate}
  \end{exercise}
  \begin{solution}
    Listed. 
    \begin{enumerate}
      \item Given a sequence of increasing naturals $S = (n_1, n_2, \ldots)$, we can define a binary expansion $0.\alpha_1 \alpha_2 \ldots$ where $\alpha_i = 1$ if and only if $i \in \mathbb{N}$ is in $S$ and $\alpha_i = 0$ if not. This is clearly a bijection. 
      \item The set of all segments of increasing natural is equipotent with $2^\mathbb{N}$, since the elements of each sequence define a subset of $\mathbb{N}$. Cantor's diagonalization argument proves that the set of infinite binary expansions is uncountable, and by (a), this proves that $2^\mathbb{N}$ is uncountable. 
    \end{enumerate}
    This is very interesting since $\mathbb{N} \simeq \mathbb{R}$, but $2^{\mathbb{N}} \simeq \mathbb{R}$, and the set of all infinite $q$-ary expansions is equipotent to $\mathbb{R}$ too. 
  \end{solution}

  \begin{exercise}[Zorich 2.4.5]
    Show that 
    \begin{enumerate}
      \item the set $\mathcal{P}(X)$ of subsets of a set $X$ has the same cardinality as the set of all functions $f: X \longrightarrow \{0, 1\}$. 
      \item for a finite set $X$ of $n$ elements, $\card{\mathcal{P}(X)} = 2^n$ 
      \item one can write $\card{\mathcal{P}(X)} = 2^{\text{card}{X}}$, which implies $\text{card}\, \mathcal{P}(\mathbb{N}) = 2^{\card \mathbb{N}} = \card \mathbb{R}$ 
      \item for any set $X$, $\card{X} < 2^{\card{X}}$
    \end{enumerate}
  \end{exercise}
  \begin{solution}
    Listed. 
    \begin{enumerate}
      \item An element $Y \in \mathcal{P}(X)$ is a subset of $X$ by definition. Letting 
      \[f_Y (x) = \begin{cases} 0 & \text{ if } x \not\in Y \\
      1 & \text{ if } x \in Y \end{cases}\]
      we can construct the bijective map $Y \mapsto f_Y$. 
      \item We can prove this using the identity (which can be proved using induction) 
        \[\sum_{k=0}^n \binom{n}{k} = 2^n \]
      \item Let $F(X; \, \{0, 1\})$ be the set of all binary valued functions from $X$ to $\{0, 1\}$. From (a), $\card{\mathcal{P}(X)} \simeq F(X; \, \{0, 1\})$. Each binary-valued function $f$ is determined by the assignment $f(x)$ for each $x \in X$. Since $f(x)$ has two possible values, the assignment of $f(x)$ for all $x \in X$ has $\{0, 1\}^{\card{X}}$ possible choices. This gives another bijection $F(X; \, \{0, 1\}) \simeq \{0, 1\}^{\card{X}}$, so 
      \[\mathcal{P}(X) \simeq \{0, 1\}^{\card{X}} \implies \card{\mathcal{P}(x)} = \card(\{0, 1\}^{\card{X}}) = 2^{\card{X}} \]
      \item If $X$ is finite, then letting $n = \card{X}$, we can simply prove $n < 2^n$ by induction (which we will not do here). If $X$ is countable, then $\mathcal{P}(X)$ is uncountable (from 2.4.4) and so using (c), 
      \[\card{X} = \card{\mathbb{N}} < \card{\mathbb{R}} = \card{\mathcal{P}(X)} = 2^{\card{X}}\]
      For uncountable sets (and for the two cases mentioned above), we can use Cantor's theorem, which states that $\card{X} < \card{\mathcal{P}(X)}$, and so using (c), we have $\card{X} < \card{\mathcal{P}(X)} = 2^{\card{X}}$. 
    \end{enumerate}
  \end{solution}

  \begin{exercise}[Zorich 2.4.6]
    Let $X_1, \ldots, X_m$ be a finite system of finite sets. Show that 
    \begin{align*}
      \card \bigg( \bigcup_{i=1}^m X_i \bigg) & = \sum_{i_1} \card{X_{i_1}} - \sum_{i_1 < i_2} \card(X_{i_1} \cap X_{i_2}) + \ldots \\
      & \sum_{i_1 < i_2 < i_3} \card(X_{i_1} \cap X_{i_2} \cap X_{i_3}) - \ldots + (-1)^{m-1} \card (X_1 \cap \ldots \cap X_m) \\
      & = \sum_{k=1}^{m} \sum_{1 \leq i_1 \ldots i_k \leq m} (-1)^{k-1} \, \card\bigg( \bigcap_{j=1}^k X_{i_j} \bigg)
    \end{align*}
  \end{exercise}
  \begin{solution}
    Ignoring Russell's paradox (defining the universe set of all sets), we can use the commutative, associative, and distributive properties of $\cup, \cap$ on the algebra of sets. We prove using induction on $m$. For $m=1$, we trivially have $\card{X_1} = \card{X_1}$, and for $m = 2$, we claim 
    \[\card( X_1 \cup X_2) = \card(X_1) + \card(X_2) - \card(X_1 \cap X_2)\]
    $X_1$ and $X_2 \setminus X_1$ are clearly exclusive sets by definition, with $X_1 \cup X_2 = X_1 \cup (X_2 \setminus X_1)$, so 
    \[\card(X_1 \cup X_2) = \card\big( X_1 \cup (X_2 \setminus X_1) \big) = \card(X_1) + \card(X_2 \setminus X_1) \tag{2} \label{CardOne}\]
    By definition, the set $X_2 \setminus X_1$ and $X_1 \cap X_2$ are disjoint and satisfies $X_2 = (X_2 \setminus X_1) \cup (X_1 \cap X_2)$ (also by definition), so 
    \[\card(X_2) = \card(X_2 \setminus X_1) + \card(X_1 \cap X_2) \tag{3} \label{CardTwo}\]
    and substituting \eqref{CardTwo} into \eqref{CardOne} gives the claim for $m=2$. Assuming that the claim is satisfied for some $m$, we have 
    \begin{align*}
      \card \bigg( \bigcup_{i=1}^{m+1} X_i \bigg) & = \card \bigg( \bigg[ \bigcup_{i=1}^m X_i \bigg] \cup X_{m+1} \bigg) & \\
      & = \card \bigg( \bigcup_{i=1}^k X_i \bigg) + \card(X_{m+1}) - \card\bigg( \bigg[ \bigcup_{i=1}^m X_i\bigg] \cap X_{m+1} \bigg) & (\text{claim for } m=2) \\
      & = \card \bigg( \bigcup_{i=1}^k X_i \bigg) + \card(X_{m+1}) - \card\bigg(\bigcup_{i=1}^m (X_i \cap X_{m+1}) \bigg) & (\text{distributive prop.})  \\
      & = \sum_{k=1}^m \sum_{1 \leq i_1 \ldots i_k \leq m} (-1)^{k-1} \card\bigg( \bigcap_{j=1}^k X_{i_j} \bigg) + \card(X_{m+1}) \\ 
      & \;\;\;\;\;\; - \sum_{k=1}^m \sum_{1 \leq i_1 \ldots i_k \leq m} (-1)^{k-1} \card\bigg( \bigcap_{j=1}^k (X_{i_j} \cap X_{m+1}) \bigg) 
    \end{align*}
    With a bit of thought, we can see that the $k$th term of the second summation contributes to adding another term to the $k+1$th summation term of the first. Therefore, we must try to shift the summation over by $1$ index. Let us simplify this by taking the summations and extracting the first and last term, respectively. We have 
    \begin{align*}
    \sum_{k=1}^m \sum_{1 \leq i_1 \ldots i_k \leq m} (-1)^{k-1} \card\bigg( \bigcap_{j=1}^k X_{i_j} \bigg) &= \sum_{1 \leq i_1 \ldots i_k \leq m} \card(X_{i_1}) \\
    &+ \sum_{k=2}^m \sum_{1 \leq i_1 \ldots i_k \leq m} (-1)^{k-1} \card\bigg( \bigcap_{j=1}^k X_{i_j} \bigg)
    \end{align*}
    and 
    \begin{align*}
      \sum_{k=1}^m \sum_{1 \leq i_1 \ldots i_k \leq m} & (-1)^{k-1} \card\bigg( \bigcap_{j=1}^k (X_{i_j} \cap X_{m+1}) \bigg) \\
      & = \sum_{k=1}^m \sum_{1 \leq i_1 \ldots i_k \leq m} (-1)^{k-1} \card\bigg( \bigg[ \bigcap_{j=1}^k X_{i_j} \bigg] \cap X_{m+1} \bigg) \\
      & =  \sum_{k=1}^{m-1} \sum_{1 \leq i_1 \ldots i_k \leq m} (-1)^{k-1} \card\bigg( \bigcap_{j=1}^k (X_{i_j} \cap X_{m+1}) \bigg) + (-1)^{m-1} \card\bigg( \bigcap_{j=1}^{m+1} X_j \bigg) \\
      & = \sum_{k=2}^{m} \sum_{1 \leq i_1 \ldots i_k \leq m} (-1)^{k-2} \card\bigg( \bigcap_{j=1}^{k-1} (X_{i_j} \cap X_{m+1}) \bigg) + (-1)^{m-1} \card\bigg( \bigcap_{j=1}^{m+1} X_j \bigg)
    \end{align*}
    So subtracting the summations gives 
    \begin{align*}
      \sum_{k=1}^m & \sum_{1 \leq i_1 \ldots i_k \leq m} (-1)^{k-1} \card\bigg( \bigcap_{j=1}^k X_{i_j} \bigg) - \sum_{k=1}^m \sum_{1 \leq i_1 \ldots i_k \leq m} (-1)^{k-1} \card\bigg( \bigcap_{j=1}^k (X_{i_j} \cap X_{m+1}) \bigg) + |X_{m+1}| \\
      & = \sum_{1 \leq i_1 \ldots i_k \leq m} \card(x_i) + \sum_{k=2}^m \sum_{1 \leq i_1 \ldots i_k \leq m} (-1)^{k-1} \card\bigg( \bigcap_{j=1}^k X_{i_j} \bigg) + \card(X_{m+1})\\
      & \;\;\;\;\; + \sum_{k=2}^{m} \sum_{1 \leq i_1 \ldots i_k \leq m} (-1)^{k-1} \card\bigg( \bigcap_{j=1}^{k-1} (X_{i_j} \cap X_{m+1}) \bigg) + (-1)^{m} \card\bigg( \bigcap_{j=1}^{m+1} X_j \bigg) \\
      & = \sum_{1 \leq i_1 \ldots i_k \leq m+1} \card(X_i) + \sum_{k=2}^m \sum_{1 \leq i_1 \ldots i_k \leq m} (-1)^{k-1} \Bigg[ \card\bigg( \bigcap_{j=1}^k X_{i_j} \bigg) \\
      & +  \card\bigg( \bigg[ \bigcap_{j=1}^{k-1} X_{i_j} \bigg] \cap X_{m+1} \bigg) \Bigg] + (-1)^{m} \card\bigg( \bigcap_{j=1}^{m+1} X_j \bigg) 
    \end{align*}
    and since the set of sequences of $k$ terms bounded by $m+1$ (of form $1 \leq i_1 \ldots i_k \leq m+1$) is the set of sequences of $k$ terms bounded by $m$ (of form $1 \leq i_1 \ldots i_k \leq m$) unioned with the set of sequences of $k$ terms with max element $m+1$ (of form $1 \leq i_1 \ldots i_k = m+1$), we have 
    \[\sum_{1 \leq i_1 \ldots i_k \leq m} (-1)^{k-1} \Bigg[ \card\bigg( \bigcap_{j=1}^k X_{i_j} \bigg) +  \card\bigg( \bigg[ \bigcap_{j=1}^{k-1} X_{i_j} \bigg] \cap X_{m+1} \bigg) \Bigg] = \sum_{1 \leq i_1 \ldots i_k \leq m+1} \card\bigg( \bigcap_{j=1}^k X_{i_j} \bigg)\]
    and therefore, substituting the above and observing that the independent terms are the first and last terms of the summation gives 
    \begin{align*}
      \card \bigg( \bigcup_{i=1}^{m+1} X_i \bigg) & = \sum_{1 \leq i_1 \ldots i_k \leq m+1} \card(X_i) + \sum_{k=2}^m \sum_{1 \leq i_1 \ldots i_k \leq m+1} (-1)^{k-1} \card\bigg( \bigcap_{j=1}^k X_{i_j} \bigg) \\ 
      & \;\;\;\;\;\;\;\;\;\; + \ldots + (-1)^{m} \card\bigg( \bigcap_{j=1}^{m+1} X_j \bigg) \\
      & = \sum_{k=1}^{m+1} \sum_{1 \leq i_1 \ldots i_k \leq m+1} (-1)^{k-1} \card\bigg( \bigcap_{j=1}^k X_{i_j} \bigg) 
    \end{align*}
  \end{solution}

  \begin{exercise}[Zorich 2.4.7]
    On the closed interval $[0, 1] \subset \mathbb{R}$, describe the sets of numbers $x \in [0, 1]$ whose ternary representation $x = 0.\alpha_1 \alpha_2 \ldots $, $\alpha_i \in \{0, 1, 2\}$ has the property. 
    \begin{enumerate}
      \item $\alpha_1 \neq 1$
      \item $\alpha_1 \neq 1$ and $\alpha_2 \neq 1$ 
      \item For all $i \in \mathbb{N}$, $\alpha_i \neq 1$ (the Cantor set)
    \end{enumerate}
  \end{exercise}
  \begin{solution}
    Listed. 
    \begin{enumerate}
      \item $[0, \frac{1}{3}) \cup [\frac{2}{3}, 1)$ 
      \item $[0, \frac{1}{9}) \cup [\frac{2}{9}, \frac{3}{9}) \cup [\frac{6}{9}, \frac{7}{9}) \cup [\frac{8}{9}, 1)$
      \item Made by recursively removing the middle third of every partitioned intervals. 
    \end{enumerate}
  \end{solution}

  \begin{exercise}[Zorich 2.4.8]
    Show that 
    \begin{enumerate}
      \item the set of numbers $x \in [0, 1]$ whose ternary representation does not contain $1$ has the same cardinality as the set of all numbers whose binary representation has the form $0.\beta_1 \beta_2 \ldots$ 
      \item the Cantor set has the same cardinality as the closed interval $[0, 1]$ 
    \end{enumerate}
  \end{exercise}
  \begin{solution}
    Listed. 
    \begin{enumerate}
      \item We can define a bijection $0.\alpha_1 \alpha_2 \ldots \mapsto 0.\beta_1 \beta_2 \ldots$ as $\alpha_i = 0 \iff \beta_i = 0$ and $\alpha_i = 1 \iff \beta_i = 2$. 
      \item The map above defines a bijection between the Cantor set and the set of all infinite binary expansions in $[0, 1]$, which is uncountable by Cantor's diagonalization theorem. 
    \end{enumerate}
  \end{solution}

\subsection{Euclidean Topology} 

  \begin{exercise}[Math 531 Spring 2025, PS5.1]
    We know what it means for a metric space $(X,d)$ to be compact. We say that it is sequentially compact if every sequence in $(X,d)$ has a convergent subsequence. Prove that a metric space is compact if and only if it is sequentially compact.
  \end{exercise}
  \begin{solution}
    We prove bidirectionally. 
    \begin{enumerate}
      \item $(\rightarrow)$. Assume that $X$ is compact and let $(x_n)$ be a sequence in $X$. For any $\epsilon > 0$, let 
      \begin{equation}
        \mathscr{C}_{\epsilon} \coloneqq \{B_{\epsilon} (x) \mid x \in X\}
      \end{equation}
      be an open cover of open balls. Then there exists a finite subcover $\mathcal{F}_\epsilon \subset \mathscr{C}_\epsilon$. There is a countable sequence $(x_n)$, with each point in at least one open set in $\mathscr{C}$. By the pigeonhole principle, at least one open set must be hit infinitely many times, call this $B (x^\ast, \epsilon)$. Now consider for $\epsilon = \frac{1}{n}$ for $n \in \mathbb{N}$. 
      \item $(\leftarrow)$. 
    \end{enumerate}
  \end{solution}

  \begin{exercise}[Math 531 Spring 2025, PS5.2]
    Give an example of a sequence of real numbers $x_n$ for which
    \begin{equation}
      |x_{n+1} - x_n| \to 0
    \end{equation}
    as $n \to \infty$, but $x_n$ is not convergent.
  \end{exercise}
  \begin{solution}
    Consider the sequence 
    \begin{equation}
      x_n = \sum_{i=1}^n \frac{1}{i}
    \end{equation} 
    It is the case that $x_{n+1} - x_n = 1/(n+1)$ which tends to $0$, but this is a harmonic series which is not convergent. 
  \end{solution}

  \begin{exercise}[Math 531 Spring 2025, PS5.3]
    Let $\{x_n\}_{n=0}^{\infty}$ be a sequence of real numbers. Assume that
    \begin{equation}
      |x_{n+1} - x_n| \leq c|x_n - x_{n-1}|
    \end{equation}
    for all $n \geq 1$, for some fixed $c < 1$. Prove that $x_n$ is convergent. Hint: you may want to use the formula that you proved in a previous homework for $1 + c + c^2 + \cdots + c^N$.
  \end{exercise}
  \begin{solution}
    
  \end{solution}

  \begin{exercise}[Math 531 Spring 2025, PS5.4]
    Let $x_n$ be a sequence of rational numbers defined recursively by:
    \begin{equation}
      x_0 = 1
    \end{equation}
    \begin{equation}
      x_{n+1} = \frac{1}{x_n + 2}
    \end{equation}
    when $n \geq 0$. The first few terms of this sequence are $1, \frac{1}{3}, \frac{3}{7}, \ldots$ Prove that the sequence is convergent and find its limit.
  \end{exercise}
  \begin{solution}
    
  \end{solution}

  \begin{exercise}[Math 531 Spring 2025, PS5.5]
    As we know, every bounded sequence of real numbers has a convergence subsequence.
    \begin{enumerate}
      \item Let's say we have two sequences $a_n$ and $b_n$ that are bounded. Find a single sequence of indices $\{n_k\}$ so that \emph{both} $a_{n_k}$ and $b_{n_k}$ are convergent. This is called a common convergent subsequence for $a_n$ and $b_n$.
      
      \item Show that for any finite number of bounded sequences of real numbers, we can find a common convergent subsequence.
      
      \item Now suppose have a sequence of bounded sequences. Find a common convergent subsequence. What does this remind you of?
    \end{enumerate}
  \end{exercise}
  \begin{solution}
    
  \end{solution}

  \begin{exercise}[Math 531 Spring 2025, PS5.6]
    Given a sequence of real numbers $x_n$, we can define the sequence of its means by:
    \begin{equation}
      x_1, \frac{x_1 + x_2}{2}, \frac{x_1 + x_2 + x_3}{3}, \ldots
    \end{equation}
    Call the sequence of means $y_n$. Prove that if $x_n \to x$, then $y_n \to x$. Discuss the examples $x_n = (-1)^n$, $x_n = n$, $x_n = (-1)^n n$, and $x_n = (-1)^n\sqrt{n}$. The fourth example shows that it is possible to average out chaotic behavior (so long as it isn't focused in one direction). The third example shows that this is impossible if the system becomes too chaotic.
  \end{exercise}
  \begin{solution}
    
  \end{solution}

  \begin{exercise}[Math 531 Spring 2025, PS4.1]
    Let $(X,d)$ be a metric space. Assume that $K$ is compact and $F$ is closed 
    in $(X,d)$. Assume $K \cap F = \emptyset$. Prove that
    \begin{equation}
      \inf_{x \in F, y \in K} d(x,y) > 0.
    \end{equation}
    Show by an example that this number could be zero if $K$ is only assumed 
    to be closed (rather than compact).
  \end{exercise}
  \begin{solution}
    $K \cap F = \emptyset \implies K \subset F^c$ with $F^c$ open. This means that for every $x \in K \subset F^c$, there exists a $r_x > 0$ s.t. $B(x, r_x) \subset F^c \iff B(x, r_x) \cap F = \emptyset$. 

    Now we take the covering $\{B(x, \frac{r_x}{2}) \mid x \in K\}$, and since $K$ is compact there must be a finite subcovering, which we denote $C = \{B(x_i, \frac{r_i}{2}) \mid i = 1, \ldots n \}$. Denote $r^\ast = \min\{r_i\}$, which is positive since we take the minimum of a finite number of positive elements. 

    Now for any $x \in K$ and $y \in F$, $x$ must be in some $B(x_i, \frac{r_i}{2}) \iff d(x, x_i) < \frac{r_i}{2} \iff -d(x, x_i) > - \frac{r_i}{2}$. With the same $i$, since $B(x_i, r_i)$ is disjoint from $F$, we have $d(x_i, y) \geq r_i$. Therefore, by the triangle inequality, 
    \begin{equation}
      d(x, y) \geq d(x_i, y) - d(x_i, x) = r_i - \frac{r_i}{2} = \frac{r_i}{2} \geq \frac{r^\ast}{2}
    \end{equation}
    and thus we have found a nontrivial lower bound. 
  \end{solution}

  \begin{exercise}[Math 531 Spring 2025, PS4.2]
    Consider $\mathbb{R}$ with the usual metric. Find an open cover of $\mathbb{Q}$ that does not 
    cover $\mathbb{R}$.
  \end{exercise}
  \begin{solution}
    $\mathbb{Q} = (-\infty, \sqrt{2}) \cup (\sqrt{2}, +\infty)$. 
  \end{solution}

  \begin{exercise}[Math 531 Spring 2025, PS4.3]
    $\mathbb{R}$ is not compact with the usual metric since it is not bounded. Let us,
    however, define the following metric on $\mathbb{R}$:
    \begin{equation}
      d_{\ast}(x,y) = \frac{|x-y|}{(1+|x|)(1+|y|)}.
    \end{equation}
    Verify that $(\mathbb{R},d_{\ast})$ is a metric space. Prove that all subsets of $(\mathbb{R},d_{\ast})$ are
    bounded. Show that $\mathbb{R}$ still isn't compact with this metric. What is the
    problem?
  \end{exercise}
  \begin{solution}
    We first verify metric. 
    \begin{enumerate}
      \item Since $|x - y| \geq 0, 1 + |x| \geq 1, 1 + |y| \geq 1$, $d_\ast (x, y) \geq 0$. We also see that 
      \begin{equation}
        d_\ast (x, y) = 0 \iff |x - y| = 0 \iff x = y
      \end{equation}

      \item It is symmetric since 
      \begin{equation}
        d_\ast (x, y) = \frac{|x - y|}{(1 + |x|)(1 + |y|)} = \frac{|y - x|}{(1 + |y|)(1 + |x|)} = d_\ast (y, x)
      \end{equation}

        \item It satisfies triangle inequality since 
        \begin{align}
          d_\ast (x, y) + d_\ast (y, z) & = \frac{|x - y|}{(1 + |x|)(1 + |y|)} + \frac{|y - z|}{(1 + |y|)(1 + |z|)} \\
                                        & = \frac{|x - y|(1 + |z|) + |y - z|(1 + |x|)}{(1 + |x|)(1 + |y|)(1 + |z|)} \\
                                        & = \frac{|x - y| + |x - y| \cdot |z| + |y - z| + |y - z| \cdot |x|}{(1 + |x|)(1 + |y|)(1 + |z|)} \\ 
                                        & \geq \frac{|x - z| + (|x| - |y|) \cdot |z| + (|y| - |z|) \cdot |x|}{(1 + |x|)(1 + |y|)(1 + |z|)} \\
                                        & = \frac{|x - z| + |x| |y| - |y| |z|}{(1 + |x|)(1 + |y|)(1 + |z|)} \\ 
                                        & = \frac{(1 + |y|)|x - z|}{(1 + |x|)(1 + |y|)(1 + |z|)} \\  
                                        & = \frac{|x - z|}{(1 + |x|)(1 + |z|)} \\ 
                                        & = d_\ast (x, z)
        \end{align}
        where the inequality comes from $|x - y| + |y - z| \geq |x - z|$, $|x - y| \geq ||x| - |y|| \geq |x| - |y|$, and $|y - z| \geq ||y| - |z|| \geq |y| - |z|$. 
    \end{enumerate}
    This is bounded since for any $x, y \in \mathbb{R}$, we can show that the numerator is bounded by the denominator. Since both are positive from (1), it suffices to prove $|x - y|^2 \leq (1 + |x|)^2 (1 + |y|)^2$. 
    \begin{align}
      (1 + |x|)^2 (1 + |y|)^2 & = (1 + 2|x| + |x|^2) (1 + 2|y| + |y|^2) \\
                              & = 1 + 2|x| + 2|y| + 4|x||y| + |x|^2 + |y|^2 + 2|x||y|^2 + 2|y||x|^2 + |x|^2 |y|^2 \\ 
                              & \geq |x|^2 + |y|^2 + 2 |x| |y| \\
                              & \geq |x|^2 + |y|^2 - 2 xy \\
                              & = |x - y|^2
    \end{align}
    where the first inequality holds since all the terms in the expansion are nonnegative and the second holds since $|x||y| \geq xy$. Therefore, $d_\ast (x, y) \leq 1$. $\mathbb{R}$ is still not compact since we can construct the set of open balls $B_r (0)$ around $0$ w.r.t. $d_\ast$. Consider the cover 
    \begin{equation}
      \mathscr{C} = \{B_{1 - \frac{1}{n}} (0)\}_{n \in \mathbb{N}, n \geq 2}
    \end{equation}
    Now assume that there is a finite subcover. Then there must be a maximum index $N \in \mathbb{N}$ in this cover. I claim that this does not cover $\mathbb{R}$. Consider the element $y = N - 1 \in \mathbb{R}$. The distance is 
    \begin{equation}
      d_\ast (x, y) = \frac{|0 - (N - 1)|}{(1 + |0|)(1 + |N-1|)} = \frac{N-1}{N} = 1 - \frac{1}{N}
    \end{equation} 
    and so $y \not\in \mathscr{C}$. Hence $\mathscr{C}$ is not a cover of $\mathbb{R}$. 
  \end{solution}

  \begin{exercise}[Math 531 Spring 2025, PS4.4]
    Consider the set $X = \mathbb{R} \cup \{Gandalf\}$. Define a metric $d_{\ast}$ on $X$ by:
    \begin{equation}
      d_{\ast}(x,y) = \frac{|x-y|}{(1+|x|)(1+|y|)}
    \end{equation}
    for $x,y \in \mathbb{R}$, while
    \begin{equation}
      d_{\ast}(Gandalf,x) = \frac{1}{1+|x|},
    \end{equation}
    for all $x \in \mathbb{R}$. Verify that $d_{\ast}$ is a metric on $X$ (you don't have to do much
    for this, since you already did part of it in the previous problem). Prove
    that $(X,d_{\ast})$ is a compact metric space.
  \end{exercise}
  \begin{solution}
    
  \end{solution}

  \begin{exercise}[Math 531 Spring 2025, PS4.5]
    Let $X$ be any set and endow it with the metric $d(x,y) = 1$ if $x \neq y$ and
    $d(x,x) = 0$. Check that this is a metric. Find all compact sets in $(X,d)$.
  \end{exercise}
  \begin{solution}
    It trivially satisfies nonnegativity since it's either $0$ or $1$, and $d(x, x) = 0$. It is symmetric as well. As for triangle inequality this is trivial. All compact sets are finite sets. 
  \end{solution}

  \begin{exercise}[Math 531 Spring 2025, PS3.1]
    Determine for each of the following sets, whether or not it is countable. Justify your answers
    \begin{enumerate}
      \item The set of all functions $f : \{0,1\} \to \mathbb{N}$.
      \item The set $B_n$ of all functions $f : \{1,...,n\} \to \mathbb{N}$
      \item The set $C = \cup_{n\in\mathbb{N}}B_n$
      \item The set of all functions $f : \mathbb{N} \to \{0,1\}$.
      \item The set of all functions $f : \mathbb{N} \to \{0,1\}$ that are ``eventually zero''
        (We say that $f$ is eventually zero if there exists some $N \geq 1$ so that
        $f(n) = 0$ for all $n \geq N$.)
      \item $G$ the set of all functions $f : \mathbb{N} \to \mathbb{N}$ that are eventually constant.
    \end{enumerate}
  \end{exercise} 
  \begin{solution}
    Listed. 
    \begin{enumerate}
      \item Countable since bijective to $\mathbb{N} \times \mathbb{N}$. We define the bijection as 
      \begin{equation}
        (a_0, a_1) \in \mathbb{N} \times \mathbb{N} \mapsto f(i) = a_i 
      \end{equation}

      \item Countable since bijective to $\mathbb{N}^n$. We define the bijection as 
      \begin{equation}
        (a_1, \ldots, a_n) \in \mathbb{N}^n \mapsto f(i) = a_i
      \end{equation}

      \item Countable since we proved that $B_n$ is countable, and a countable union of countable sets are countable from Rudin Theorem 2.12. 

      \item Uncountable since we can create a bijection from the set of all sequences $(a_i)$ of $0$ or $1$, which from Rudin Theorem 2.14 is uncountable. 
      \begin{equation}
        (a_i)_{i \in \mathbb{N}} \mapsto f(i) = a_i
      \end{equation}

      \item Countable. Call this set $B$, and call the set of functions $f$ that have their final $1$ at index $k$ to be $A_k$. Then, 
      \begin{equation}
        B = \cup_{k=1}^\infty A_k 
      \end{equation}
      where $A_0 = A_1 = 1$, and $|A_k| = 2^{k-1}$ for $k \geq 2$. Since $B$ is the countable union of at most countable sets, $B$ must be countable. 

      \item Countable. Call this set $B$. Let $A_k$ be the set of functions that are eventually constant to value $k$. Let $A_{ki}$ be the set of functions that are always $k$ starting from index $i$ (where $i$ is the smallest element). Since everything is determined to be $k$ at $i$ and beyond, $A_{ki}$ can be divided up into the first $i-1$ elements of any natural number, followed by a sequence of $k$'s. Therefore $|A_{ki}| \approx \mathbb{N}^{i-1}$, where $\approx$ means equipotent, and so $A_{ki}$ is countable. Therefore since countable unions of countable sets are countable, 
      \begin{equation}
        A_k = \bigcup_{i=1}^\infty A_{ki} \text{ is countable } \implies B = \sum_{k=1}^\infty A_k \text{ is countable}
      \end{equation}
    \end{enumerate}
  \end{solution}

  \begin{exercise}[Math 531 Spring 2025, PS3.2]
    Tell if the following subsets $A \subset \mathbb{R}$ (with the usual metric $d(x,y) = |x-y|$)
    are open or closed. Also, find $(i)$ the limit points of $A$, $(ii)$ the interior of
    $A$, $(iii)$ $\bar{A}$.
    \begin{enumerate}
      \item $A = \mathbb{Q}$
      \item $A = (0,1]$
      \item $A = \{1, \frac{1}{2}, \frac{1}{4}, ...\}$
      \item $A = \{0, 1, \frac{1}{2}, \frac{1}{4}, ...\}$
      \item $A = \mathbb{Z}$
    \end{enumerate}
  \end{exercise} 
  \begin{solution}
    Listed. We denote $A^\prime$ as the limit points of $A$ and the interior as $A^o$. 
    \begin{enumerate}
      \item Not open nor closed. $A^\prime = \mathbb{R}$, $A^o = \emptyset$. $\bar{A} = \mathbb{R}$. 
      \item Not open nor closed. $A^\prime = [0, 1]$, $A^o = (0, 1)$. $\bar{A} = [0, 1]$. 
      \item Not open nor closed. $A^\prime = \{0\}$, $A^o = \emptyset$. $\bar{A} = \{ 0 \} \cup A$. 
      \item Closed. $A^\prime = \{0\}$, $A^o = \emptyset$. $\bar{A} = A$. 
      \item Closed. $A^\prime = \emptyset$. $A^o = \emptyset$. $\bar{A} = A$. 
    \end{enumerate}
  \end{solution}
  
  \begin{exercise}[Math 531 Spring 2025, PS3.3]
    Prove the following statements subsets $A,B$ of a general metric space $(X,d)$.
    \begin{itemize}
      \item $\overline{A \cup B} = \bar{A} \cup \bar{B}$.
      \item Show by example that $\overline{A \cap B} \neq \bar{A} \cap \bar{B}$.
    \end{itemize} 
  \end{exercise} 
  \begin{solution}
    For the first part, we show bidirectionally. 
    \begin{enumerate}
      \item $\overline{A \cup B} \subset \overline{A} \cup \overline{B}$. Let $x \in \overline{A \cup B}$. If $x \in A \cup B$, then it must be the case that either $x \in A \subset (A \cup A^\prime) = \overline{A}$ or $x \in B \subset (B \cup B^\prime) = \overline{B}$, which means $x \in \overline{A} \cup \overline{B}$. Now assume not. Then $x \in (A \cup B)^\prime$. Therefore, for any $r > 0$, we know that $B(x, r) \cap (A \cup B) \neq \emptyset$. Now let us take a sequence $(r_n = \frac{1}{n})_{n \in \mathbb{N}}$, and for each $r_n$ we have some element $x_n \in (A \cup B)$. Given that we have a countably infinite sequence of $x_n$, each which may be in $A$ or $B$, by the pigeonhole principle either $A$ or $B$ must be hit infinitely many times. If $x_n \in A$ infinitely many times, then $x \in \overline{A}$, and analogous for $B$. 
      \item $\overline{A} \cup \overline{B} \subset \overline{A \cup B}$. WLOG let $x \in \overline{A}$. If $x \in A$, then $x \in (A \cup B) \subset \overline{A \cup B}$. If $x \not\in A$, then $x \in A^\prime$. Therefore for every $r > 0$, $B(x, r) \cap A \neq \emptyset$. But this means 
      \begin{equation}
        \emptyset \neq (B(x, r) \cap A) \cup (B(x, r) \cap B) = B(x, r) \cap (A \cup B) \implies x \in (A \cup B)^\prime \subset \overline{A \cup B}
      \end{equation}
    \end{enumerate}
    For a counterexample, consider the sequences 
    \begin{equation}
      A = (x_n) = \frac{1}{n} \qquad B = (y_n) = -\frac{1}{n}
    \end{equation}
    for $n \in \mathbb{N}$. $\overline{A} = A \cup \{0\}, \overline{B} = B \cup \{0\}$, and so $\overline{A} \cap \overline{B} = \{0\}$. However, $A \cap B = \emptyset \implies \overline{A \cap B} = \emptyset$. 
  \end{solution}

  \begin{exercise}[Math 531 Spring 2025, PS3.4]
    Consider the set of rationals in canonical form (such that numerator and denominator are relatively prime) with potential distance:
    \begin{equation}
      d_1(\frac{p_1}{q_1}, \frac{p_2}{q_2}) = |q_1 - q_2|.
    \end{equation}
    Is this a metric? Prove that the following defines a metric
    \begin{equation}
      d_2(\frac{p_1}{q_1}, \frac{p_2}{q_2}) = |p_1 - p_2| + |q_1 - q_2|.
    \end{equation}
  \end{exercise} 
  \begin{solution}
    This is not a metric since 
    \begin{equation}
      d_1 \bigg( \frac{2}{1}, \frac{3}{1} \bigg) = 1 - 1 = 0
    \end{equation} 
    when $2/1 \neq 3/1$. For $d_2$, we show that it satisfies the three properties. 
    \begin{enumerate}
      \item \textit{Nonnegativity}. Since it is the sum of 2 absolute values which are norms and therefore nonnegative, it must be nonnegative by ordered field properties. We see that 
      \begin{align}
        \frac{p_1}{q_1} = \frac{p_2}{q_2} & \iff p_1 = p_2 \text{ and } q_1 = q_2 \\
                                          & \iff |p_1 - p_2| = |q_1 - q_2| = 0 \\
                                          & \iff |p_1 - p_2| + |q_1 - q_2| = 0 
      \end{align}

      \item For symmetricity, note that 
        \begin{equation}
          d_2 \bigg( \frac{p_1}{q_1} , \frac{p_2}{q_2} \bigg) = |p_1 - p_2| + |q_1 - q_2| = |p_2 - p_1| + |q_2 - q_1| = d_2 \bigg( \frac{p_1}{q_1} , \frac{p_2}{q_2} \bigg)
        \end{equation}

        \item For triangle inequality, we see that for any $p_1/q_1, p_2/q_2, p_3/q_3$, 
        \begin{align}
          d_2 \bigg( \frac{p_1}{q_1}, \frac{p_3}{q_3} \bigg) & = |p_1 - p_3| + |q_1 - q_3| \\
                                                             & = |(p_1 - p_2) + (p_2 - p_3)| + |(q_1 - q_2) + (q_2 - q_3)| \\
                                                             & \leq |p_1 - p_2| + |p_2 - p_3| + |q_1 - q_2| + |q_2 - q_3| && \tag{subadditivity of norm} \\
                                                             & = d_2 \bigg( \frac{p_1}{q_1}, \frac{p_2}{q_2} \bigg) + d_2 \bigg( \frac{p_2}{q_2}, \frac{p_3}{q_3} \bigg) 
        \end{align}
    \end{enumerate}
  \end{solution}

  \begin{exercise}[Math 531 Spring 2025, PS3.5]
    Let $M = \{x_1,..., x_3\}$ be a set with three points. Describe the set of all metrics on $M$. What if $M$ has four points?
  \end{exercise} 
  \begin{solution}
    If $M$ has 3 points call them $x_1, x_2, x_3$, then the metric is completely defined by the three values 
    \begin{align}
      d(x_1, x_2) & = d(x_2, x_1) \\
      d(x_2, x_3) & = d(x_3, x_2) \\
      d(x_3, x_1) & = d(x_3, x_1) 
    \end{align}
    where $d(x, x) = 0$. We must make sure that the triangle inequality satisfies for these 3 numbers. Therefore we can think of this as the set of all triangles in $\mathbb{R}^2$ (that are equivalent under translation and rotation, but not permutation of points). 

    Similarly for 4 points, we can visualize the metrics as the set of all tetarhedra in $\mathbb{R}^3$ (since each face is a triangle, and therefore for any three points the triangle inequality is guaranteed to be satisfied), equivalent under translation and rotation, but not permutation of the 4 points. 
  \end{solution}

  \begin{exercise}[Math 531 Spring 2025, PS3.6]
    Let $P$ be a polynomial of degree $n \geq 1$. Prove that if $P(0) = 0$, then $P(x) = xQ(x)$, for some polynomial $Q$ of degree $n-1$. Deduce that if $P(a) = 0$, then we can write $P(x) = (x-a)Q(x)$ for some $Q$ of degree $n-1$.
  \end{exercise}
  \begin{solution}
    A $n$th degree polynomial will have the form 
    \begin{equation}
      p(x) = \sum_{i=0}^n c_i x^i
    \end{equation}
    Since $p(0) = c_0 = 0 \implies c_0 = 0$. This means that 
    \begin{equation}
      p(x) = \sum_{i=1}^n c_i x^i = x \sum_{i=0}^{n-1} c_{i+1} x^i \text{ where } Q(x) = \sum_{i=0}^{n-1} c_{i+1} x^i
    \end{equation} 
    If $p(a) = 0$, we can construct $f(x) = p(x + a)$, where $f$ is a polynomial since the expansion does not increase its degree. Since $f(x) = p(a) = 0$, by above $f$ can be factorized $f(x) = x g(x)$ for some $(n-1)$th degree polynomial $g$, and by substitution this means that $p(x) = f(x - a) = (x - a) g(x - a)$. 
  \end{solution}

  \begin{exercise}[Math 531 Spring 2025, PS3.7]
    Consider all polynomials $P : \mathbb{R} \to \mathbb{R}$ of degree less than or equal to $n$. Call this set $\mathcal{P}_n$. Let's define potential distances on $\mathcal{P}_n$.
    \begin{equation}
      d_1(p,q) = |p(0) - q(0)|.
    \end{equation}
    Show this defines a distance on $\mathcal{P}_0$ but not on $\mathcal{P}_n$ for $n \geq 1$. Now consider
    \begin{equation}
      d_N(p,q) = \sum_{j=0}^N |p(j) - q(j)|
    \end{equation}
    Show that this defines a distance on $\mathcal{P}_n$, for every $n \leq N$. What does the solution say about polynomials of degree $N$?
  \end{exercise}
  \begin{solution}
    If $n = 0$, $\mathcal{P}_n$ is a set of constant functions $P$, where each constant function $P$ is determined completely by its value at any point, e.g. $0$. We check the properties. 
    \begin{enumerate}
      \item $d_1 (p, q) \geq 0$ since we take the norm at the end. We can see that 
      \begin{align}
        d_1 (p, q) = 0 & \iff |p(0) - q(0)| \\
                       & \iff p(0) = q(0) \\
                       & \iff p = q
      \end{align} 

      \item It is clearly symmetric. 
      \begin{equation}
        d_1 (p, q) = |p(0) - q(0)| = |q(0) - p(0)| = d_1(q, p)
      \end{equation}

      \item It satisfies the triangle inequality by subadditivity of the norm. 
      \begin{align}
        d_1 (p, r) & = |p(0) - r(0)| \\
                   & = |(p(0) - q(0)) + (q(0) - r(0))| \\
                   & \leq |p(0) - q(0)| + |q(0) - r(0)| \\
                   & = d_1 (p, q) + d_1 (q, r)
      \end{align}
    \end{enumerate}
    It doesn't satisfy for $P_n$ because consider $p(x) = x$ and $q(x) = x^2$. They are not the same function but $d_1 (p, q) = |p(0) - q(0)| = 0$. For $d_N$ defined on $\mathcal{P}_n$ for $n \leq N$, we verify the properties. 
    \begin{enumerate}
      \item This is the sum of norms, so it must be nonnegative. Now we see that if $p = q$, then $p(x) = q(x) \implies |p(x) - q(x)| = 0 \implies d_N (p, q) = 0$. For the other way around, suppose $d_N(p, q) = 0$. Then from problem 3.8, we are solving the linear equation $0 = V b - V c$, where $b, c$ are the vectors representing the coefficients of $p, q$, and $V$ is the Vandermonde matrix with $a_i = i$. By linearity, this is equivalent to solving $0 = V(b - c)$, and since we showed that $V$ is invertible (since $a_i$'s are distinct), $V$ has a trivial kernel and therefore $b - c = 0 \iff b = c \implies p = q$. 

      \item Symmetricity is trivial. 
      \begin{equation}
        d_N(p,q) = \sum_{j=0}^N |p(j) - q(j)| = \sum_{j=0}^N |q(j) - p(j)| = d_N (q, p)
      \end{equation} 

      \item For triangle inequality, 
      \begin{align}
        d_N (p, r) & = \sum_{j=0}^N |p(j) - r(j)| \\
                   & = \sum_{j=0}^N |(p(j) - q(j)) + (q(j) - r(j))| \\  
                   & \leq \sum_{j=0}^N |p(j) - q(j)| + |q(j) - r(j)| \\
                   & = \sum_{j=0}^N |p(j) - q(j)| + \sum_{j=0}^N |q(j) - r(j)| \\
                   & = d_N (p, q) + d_N (q, r)
      \end{align}
    \end{enumerate}
    This shows that we need to ``sample'' more points from higher-degree polynomials to get the metric as they are higher-dimensional. 
  \end{solution}

  \begin{exercise}[Math 531 Spring 2025, PS3.8]
    Given distinct numbers $a_0,...,a_N$ and numbers $b_0,...,b_N$, prove that there exists a polynomial $P$ of degree $N$ with the property that
    \begin{equation}
      P(a_i) = b_i,
    \end{equation}
    for $0 \leq i \leq N$. The most direct way to solve this problem, in my view, is to write the system equations you are trying to solve as a linear system for the coefficients of $P$. This will give you some matrix $M$ that depends on the numbers $a_0,...,a_N$. The key is to show that the determinant of this matrix is non-zero. It turns out that the determinant of this matrix is equal to
    \begin{equation}
      \prod_{0\leq i<j\leq N}(a_i - a_j),
    \end{equation}
    up to a potential $-$ sign depending on how you defined $M$. Prove this and deduce the result.
  \end{exercise}
  \begin{solution}
    We can write the system of equations using the Vandermonde matrix $V \in \mathbb{R}^{(N+1) \times (N+1)}$ and $c$ is the vector of coefficients of $P$. 
    \begin{equation}
      b = V c \iff \begin{bmatrix}
        b_0 \\ b_1 \\ \vdots \\ b_N
      \end{bmatrix} = \begin{bmatrix}
        1 & a_0 & a_0^2 & \ldots & a_0^N \\ 
        1 & a_1 & a_1^2 & \ldots & a_1^N \\ 
        \vdots & \vdots & \ddots & \vdots \\
        1 & a_N & a_N^2 & \ldots & a_N^N 
      \end{bmatrix} \begin{bmatrix}
        c_0 \\ c_1 \\ \vdots \\ c_N
      \end{bmatrix}
    \end{equation} 
    To calculate the determinant of $V$, we prove using induction. Clearly for $N = 1$ we have 
    \begin{equation}
      \det \begin{pmatrix}
        1 & a_0 \\ 1 & a_1 
      \end{pmatrix} = a_1 - a_0
    \end{equation}
    Now assume that this formula holds for some $N-1 \in \mathbb{N}$. Then for $N$, we can take $V$ and subtract $a_0$ times the $i$th column from the $(i+1)$st column. This gives us
    \begin{equation}
      V = 
      \begin{bmatrix}
        1 & 0 & 0 & \ldots & 0 \\
        1 & a_1-a_0 & a_1^2-a_0a_1 & \ldots & a_1^N-a_0a_1^{N-1} \\
        1 & a_2-a_0 & a_2^2-a_0a_2 & \ldots & a_2^N-a_0a_2^{N-1} \\
        \vdots & \vdots & \vdots & \ddots & \vdots \\
        1 & a_N-a_0 & a_N^2-a_0a_N & \ldots & a_N^N-a_0a_N^{N-1}
      \end{bmatrix}
    \end{equation}
    When calculating the determinant, we can perform the cofactor expansion by the first row, and then for each $i$th row factor out $(a_i - a_0)$ to get 
    \begin{equation}
      \det V = \prod_{j=1}^{N} (a_j - a_0) \det
      \begin{bmatrix}
        1 & a_1 & \ldots & a_1^{N-1} \\
        1 & a_2 & \ldots & a_2^{N-1} \\
        \vdots & \vdots & \ddots & \vdots \\
        1 & a_N & \ldots & a_N^{N-1}
      \end{bmatrix}
    \end{equation}
    which is the $(N-1) \times (N-1)$ Vandermonde matrix. Therefore, we can apply our inductive hypothesis to get 
    \begin{equation}
      \det V = \prod_{j=1}^N (a_j - a_0) \prod_{1 \leq i < j \leq N} (a_j - a_i) = \prod_{0 \leq i \leq j \leq N} (a_j - a_i)
    \end{equation}
    Note that this has a $0$ determinant iff $a_i = a_j$ for some $i \neq j$. Therefore sicne $a_i$'s are distinct, it must be nonzero. Therefore, this matrix is nonsingular, i.e. invertible, and we can solve the matrix equation to get 
    \begin{equation}
      c = V^{-1} b
    \end{equation}
    which from linear algebra is guaranteed to exist and is unique. 
  \end{solution}
  
  \begin{exercise}[Rudin 2.1]
    Prove that the empty set is a subset of every set. 
  \end{exercise}
  \begin{solution}
    It must suffice that if $x \in \emptyset$, then $x \in A$ for any arbitrary set $A$. This is vacuously true, since the initial condition is never met. 
  \end{solution}

  \begin{exercise}
    Show that the empty function $f: \emptyset \rightarrow X$, where $X$ is an arbitrary set, is always injective. If $X = \emptyset$, then $f$ is bijective. 
  \end{exercise}
  \begin{solution}
    Given distinct $x, y \in \emptyset$, $f(x) \neq f(y)$ is vacuously true, but if $X \neq \emptyset$, then there exists a $w \in X$ with no preimage. If $X = \emptyset$, then the statement for all $w \in X$, there exists an $x \in \emptyset$ s.t. $f(x) = w$ is vacuously true. 
  \end{solution}

  \begin{exercise}[Rudin 2.2]
    A complex number $z$ is said to be algebraic if there are integers $a_0, a_1, \ldots, a_n$, not all zero, such that
    \begin{equation}
      a_0z^n + a_1z^{n-1} + \ldots + a_{n-1}z + a_n = 0.
    \end{equation}
    Prove that the set of all algebraic complex numbers is countable. Hint: For every positive integer $N$ there are only finitely many equations with 
    \begin{equation}
      n + |a_0| + |a_1| + \ldots + |a_n| = N
    \end{equation}
  \end{exercise}
  \begin{solution}
    Consider all polynomials s.t. $n + \sum_{i=0}^n |a_i| = N$. There is only a finite number of them, and each polynomial has at most $n$ distinct complex roots. So this set is finite, an unioning over all $N \in \mathbb{N}$ gives an at most countable set of roots. 
  \end{solution}

  \begin{exercise}[Rudin 2.3]
    Prove there exists real numbers which are not algebraic. 
  \end{exercise}
  \begin{solution}
    From the previous exercise, if there were no no real numbers which are not algebraic, then every real number is algebraic. This contradicts the fact that the set of all complex numbers is countable. 
  \end{solution}

  \begin{exercise}[Rudin 2.4]
    Is the set of all irrational real numbers countable? 
  \end{exercise}
  \begin{solution}
    No. Assume that it is countable. We have $\mathbb{Q}$ countable. Then, by assumption, we must have $\mathbb{R} = \mathbb{Q} \cup \mathbb{Q}^c$ be the union of countable sets, which must be countable, contradicting the fact that it is uncountable. 
  \end{solution}

  \begin{exercise}[Rudin 2.5]
    Construct a bounded set of real numbers which exactly 3 limit points. 
  \end{exercise}
  \begin{solution}
    We can construct the union of 3 sequences that converge onto the limit points $0, 1, 2$. 
    \begin{equation}
      \big\{ \frac{1}{n} \big\}_{n \in \mathbb{N}} \cup \big\{ \frac{1}{n} + 1\}_{n \in \mathbb{N}} \cup \big\{ \frac{1}{n} + 2 \big\}_{n \in \mathbb{N}}
    \end{equation}
  \end{solution}

  \begin{exercise}
    Prove that the union of the limit points of sets is equal to the limit points of the union of the sets. 
    \begin{equation}
      \bigcup_{k=1}^m A_k'=\left(\bigcup_{k=1}^m A_k\right)^{\!\prime}
    \end{equation}
  \end{exercise}
  \begin{solution}
    
  \end{solution}

  \begin{exercise}[Rudin 2.6]
    Let $E^\prime$ be the set of all limit points of a set $E$. Prove that $E^\prime$ is closed. Prove that $E$ and $\overline{E}$ have the same limit points. (Recall that $\overline{E} = E \cup E^\prime$). Do $E$ and $E^\prime$ always have the same limit points? 
  \end{exercise}
  \begin{solution}
    Listed. 
    \begin{enumerate}
        \item Let $x$ be a limit point of $E^\prime$. Then, for every $\epsilon > 0$, $U = B_\epsilon (x) \cap E^\prime \neq \emptyset$. Take a $y \in U$. Since $y \in B_\epsilon (x)$, which is open, we can construct an open ball $B_\delta (y) \subset B_\epsilon (x)$. Since $y \in E^\prime$, $B_\delta (y)$ must contain elements of $E$, which means that $B_\epsilon (x)$ must also contain elements of $E$, and so $x$ is a limit point of $E \implies x \in E^\prime$ and $E^\prime$ is closed. 

        \item To prove that $E^\prime \subset \overline{E}^\prime$, we know that if $x \in E^\prime$, then for every $\epsilon > 0$, there exists a $B_\epsilon ^\circ (x)$ that has a nontrivial intersection with $E$ which means that it has a nontrivial intersection with $E \cup E^\prime$. To prove that $\overline{E}^\prime \subset E^\prime$, we know that if $y \in \overline{E}^\prime$, then for every $\delta > 0$ there exists a $B_\delta (x)$ that has a nontrivial intersection with $\overline{E}$. If $B_\delta (x)$ intersects $E$ then we are done. If $B_\delta (x)$ intersects $E^\prime$, then we can find a $y \in E^\prime \cap B_\delta (x)$. Since $B_\delta (x)$ is open, we can construct $B_\varepsilon (y) \subset B_\delta (x)$ and since $y \in E^\prime$, we know that $B_\varepsilon (y)$ contains an element of $E$, which means that $B_\delta (x)$ contains an element of $E$. Therefore, $E^\prime = \overline{E}^\prime$. 

        \item No. Consider the set $E = \{1/n\}_{n \in \mathbb{N}}$. $E^\prime = \{0\}$, but $E^{\prime\prime} = \emptyset$. 
    \end{enumerate}
  \end{solution}

  \begin{exercise}[Rudin 2.7]
    Let $A_1, A_2, \ldots$ be subsets of a metric space. 
    \begin{enumerate}
      \item If $B_n = \cup_{i=1}^n A_i$, prove that $\bar{B}_n = \cup_{i=1}^n \bar{A}_i$ for $n = 1, 2, 3, \ldots$ 
      \item If $B = \cup_{i=1}^\infty A_i$, prove that $\bar{B} \supset \cup_{i=1}^\infty \bar{A}_i$. 
    \end{enumerate}
  \end{exercise}
  \begin{solution}
    Listed. 
    \begin{enumerate}
        \item We will prove that $\overline{B_n} \subseteq \cup_{i=1}^n \overline{A_i}$ and $\cup_{i=1}^n \overline{A_i} \subseteq \overline{B_n}$. If $x \in B_n$, then $x \in \cup_{i=1}^n A_i$. Therefore, assume that $x \in B_n^\prime$. Then for every $\epsilon > 0$, there exists a $B_\epsilon^\circ (x)$ s.t. 
        \[B_\epsilon^\circ (x) \cap B_n \neq \emptyset \implies B_\epsilon^\circ (x) \cap \bigg( \bigcup_{i=1}^n A_n \bigg) \neq \emptyset \]
        This means that there exists some $i = i(\epsilon)$, a function of $\epsilon$, s.t. $B_\epsilon^\circ (x) \cap A_i \neq \emptyset$. However, this $i$ may change if we unfix $\epsilon$. We have so far proved that just for one $\epsilon > 0$ there exists an $i$. Now if we take a sequence of $\epsilon = 1, \frac{1}{2}, \frac{1}{3}, \ldots$, we have a sequence of $i(\epsilon)$'s living in $\{1, \ldots, n\}$. By the pigeonhole principle, there must be at least one $i$ that is hit infinitely many times, and so we can choose this $i$, that works for all $\epsilon > 0 \implies x \in A_i^\prime \subseteq \cup_{i=1}^n \overline{A_i}$. If $x \in \cup_{i=1}^n \overline{A_i}$, then there exists an $\overline{A_i}$ s.t. $x \in \overline{A_i}$. If $x \in A_i$, then we are done. If $x \in A_i^\prime$, then for every $\epsilon > 0$, there exists a $B_\epsilon^\circ (x)$ s.t. 
        \[B_\epsilon^\circ (x) \cap A_i \neq \emptyset \implies B_\epsilon^\circ (x) \cap \bigg( \bigcup_{i=1}^n A_i \bigg) \neq \emptyset\]
        and so $x \in B_n^\prime \subset \overline{B_n}$. 

        \item $x \in \cup_{i=1}^\infty \overline{A_i} \implies x \in \overline{A_i}$ for some $i$. If $x \in A_i$, then $x \in B$ and we are done. If $x \in A_i^\prime$, then for every $\epsilon > 0$ there exists $B_\epsilon (x) $ s.t. 
        \[B_\epsilon^\circ (x) \cap A_i \neq \emptyset \implies B_\epsilon^\circ (x) \cap \bigg( \bigcup_{i=1}^\infty A_i \bigg) \neq \emptyset\] 
        and so $B_\epsilon^\circ (x) \cap B \neq \emptyset \implies x \in B^\prime \subset \overline{B}$. 
    \end{enumerate}
  \end{solution}

  \begin{exercise}[Rudin 2.8]
    Is every point of every open set $E \subset \mathbb{R}^2$ a limit point of $E$? Answer the same question for closed sets in $\mathbb{R}^2$. 
  \end{exercise}
  \begin{solution}
    Yes for open. Given any $x \in U$ open, there always exists an $\epsilon > 0$ s.t. 
    \begin{equation}
      B_\epsilon^\circ (x) \subset B_\epsilon (x) \subset U
    \end{equation}
    and so $B_\epsilon^\circ (x)$ has a nontrivial intersection with $U$. If $U$ is closed, then no. Note that for closed $U$, we have that every limit point is in $U$, but not every point in $U$ is a limit point. Consider the isolated point $U = \{x\}$. $x$ is not a limit point of $U$. 
  \end{solution}

  \begin{exercise}[Rudin 2.9]
    Let $E^\circ$ denote the set of all interior points of $E$ in $X$. Prove the following:
    \begin{enumerate}
      \item[(a)] $E^\circ$ is always open.
      \item[(b)] $E$ is open if and only if $E^\circ = E$.
      \item[(c)] If $G \subseteq E$ and $G$ is open, then $G \subset E^\circ$.
      \item[(d)] Prove that the complement of $E^\circ$ is the closure of the complement of $E$. 
      \item[(e)] Do $E$ and $\bar{E}$ always have the same interiors? 
      \item[(f)] Do $E$ and $E^\circ$ always have the same closures? 
    \end{enumerate}
  \end{exercise}
  \begin{solution}
    Listed. 
    \begin{enumerate}
        \item We assume that $E^\circ$ is not open (this does not mean that $E^\circ$ is necessarily closed!). That is, there exists an $x \in E^\circ $ s.t. we can't construct an open ball $B_\epsilon (x) \subseteq E^\circ$. Since $x \in E^\circ \subset E$, by definition of an interior point we can construct a $B_\epsilon (x) \subset E$. But from our assumption $B_\epsilon (x) \not\subset E^\circ$. We choose a $y \in B_\epsilon (x) \setminus E^\circ$. Since $B_\epsilon (x)$ is open, there exists a $\delta > 0$ s.t. 
        \[B_\delta (y) \subset B_\epsilon (x) \subset E\]
        But the fact that we can construct an open ball around $y$ means that $y \in E^\circ$, leading to a contradiction. 

        \item If $E$ is open, then by definition $E \subset E^\circ$. Now $E^\circ \subset E$ holds for all sets since $E^\circ$ must be composed of points from $E$. If $E = E^\circ$, then for every $x \in E$, $x \in E^\circ$, so by definition there exists an $\epsilon > 0$ s.t. $B_\epsilon (x) \subset E$, which means that $E$ is open. 

        \item Let $x \in G$ open. Then there exists an $\epsilon > 0$ s.t. $B_\epsilon (x) \subset G$, and so $B_\epsilon (x) \subset E$. Since we can always construct an open ball around $x$ contained within $E$, $x \in E^\circ$ and $G \subset E^\circ$. 

        \item ($(E^\circ)^c \subset \overline{E^c}$) If $x \in (E^\circ)^c$, then there exists no $\epsilon > 0$ s.t. $B_\epsilon (x) \subset E$. Then, for any $\epsilon > 0$, $B_\epsilon (x) \not\subset E \implies B_\epsilon (x) \cap E^c \neq \emptyset \implies x \in E^c \subset \overline{E^c}$. ($\overline{E^c} \subset (E^\circ)^c)$) If $x \in \overline{E^c}$, then $x \in E^c$ or $x \in E^{c \prime}$. If $x \in E^c$, note $E^\circ \subset E \implies (E^\circ)^c \supset E^c \implies x \in (E^\circ)^c$. If $x \in E^{c \prime}$, then for all $\epsilon > 0$ $B_\epsilon (x) \cap E^c \neq \emptyset \implies B_\epsilon (x) \not\subset E \implies x \in E^\circ$. 

        \item No. Consider the rationals $\mathbb{Q} \subset \mathbb{R}$. $\mathbb{Q}^\circ = \emptyset$ but $\overline{\mathbb{Q}}^\circ = \mathbb{R}^\circ = \mathbb{R}$. It is true and straightforward to prove that $E^\circ \subset \overline{E}^\circ$.  Let $x \in E^\circ$. Then there exists an $\epsilon > 0$ s.t. $B_\epsilon(x) \subset E \implies B_\epsilon (x) \subset \overline{E} \implies x \in \overline{E}^\circ$. 

        \item No. Consider $\mathbb{Q} \subset \mathbb{R}$. Then $\overline{\mathbb{Q}} = \mathbb{R}$ and $\overline{\mathbb{Q}^\circ} = \overline{\emptyset} = \emptyset$.  
    \end{enumerate}
  \end{solution}

  \begin{exercise}[Rudin 2.10]
    Let $X$ be an infinite set. For $p \in X$ and $q \in X$, define 
    \begin{equation}
      d(p, q) = \begin{cases} 1 & \text{ if } p \neq q \\ 0 & \text{ if } p = q \end{cases}
    \end{equation}
    Prove that this is a metric. Which subsets of the resulting metric space are open? Which are closed? Which are compact? 
  \end{exercise}
  \begin{solution}
    This is a metric since clearly it satisfies symmetry and the fact that $d(p, p) = 0$. The triangle inequality 
    \begin{equation}
      d(p, r) \leq d(p, q) + d(q, r)
    \end{equation}
    is trivially satisfied if $p = r$, and if $p \neq r$, then either $p \neq q$ or $q \neq r$, and so the RHS $\geq 1$. An open $\epsilon$-ball around $x \in X$ is either $X$, when $\epsilon > 1$, or $\{x\}$ when $\epsilon \leq 1$. Therefore 
  \end{solution}

  \begin{exercise}[Rudin 2.11]
    For $x \in \mathbb{R}$ and $y \in \mathbb{R}$, define 
    \begin{align*}
        d_1 (x, y) & = (x - y)^2 \\ 
        d_2 (x, y) & = \sqrt{|x - y|}\\ 
        d_3 (x, y) & = |x^2 - y^2| \\ 
        d_4 (x, y) & = |x - 2y| \\ 
        d_5 (x, y) & = \frac{|x - y|}{1 + |x - y|}
    \end{align*}
    Determine, for each of these, whether it is a metric or not. 
  \end{exercise}
  \begin{solution}
    Listed. Positive semidefiniteness and symmetry are easy to check. 
    \begin{enumerate}
        \item The triangle inequality gives 
        \begin{align*}
            d_1 (x, z) \leq d_1 (x, y) + d_1 (y, z) & \iff (x - z)^2 \leq (x - y)^2 + (y - z)^2 \\
            & \iff 0 \leq (x - y) (y - z)
        \end{align*}
        which is not satisfied if $x < y < z$, so this is not a valid metric. 
        
        \item The triangle inequality gives $\sqrt{|x - z|} \leq \sqrt{|x - y|} + \sqrt{|y - z|}$, and since both sides are positive this inequality is equivalent to squaring both sides to get 
        \[|x - z| \leq |x - y| + |y - z| + 2 \sqrt{|x - y| |y - z|}\]
        which is true since $|x - z| \leq |x - y| + |y - z|$ of the Euclidean distance satisfies the triangle inequality and $0 \leq \sqrt{|x - y| |y - z|}$. 
        
        \item This does not satisfy triangle inequality, as taking $0, 1, 2$ gives 
        \[d_3 (0, 2) = 4 > 1 + 1 = d_3 (0, 1) + d_3 (1, 2)\]
        
        \item This does not satisfy symmetry. 
        
        \item For simplicity, let us set $A = |x - y|, B = |y - z|, C = |x - z|$. Then, we get 
        \[\frac{C}{1 + C} \leq \frac{A}{1 + A} + \frac{B}{1 + B} \iff C \leq A + B + 2 AB + ABC\]
        where $C \leq A + B$ is true by triangle inequality of Euclidean distance, $0 \leq AB$, and $0 \leq ABC$. 
    \end{enumerate}
    Intuitively, we want a metric that doesn't ``blow up" the distance between $x$ and $y$. More precisely, we want a valid metric $d(x, y)$ to be $O(|x - y|)$. Having something like a quadratic growth rate $(x - y)^2$ will blow the distance $d(x, z)$ up too much overpowering the individual $d(x, y) + d(y, z)$. 
  \end{solution}

  \begin{exercise}[Rudin 2.12]
    Let $K \subset \mathbb{R}$ consist of $0$ and the numbers $1/n$ for $n = 1, 2, 3, \ldots$. Prove that $K$ is compact directly from the definition (without using the Heine-Borel theorem). 
  \end{exercise}
  \begin{solution}
    Every open cover of $K$ must have an open set $G$ s.t. $0 \in G$. Since $G$ is open, there exists an open neighborhood $B_\epsilon (0) \subset G$ around $0$. By the Archimidean principle, there exists an $N \in \mathbb{N}$ s.t. 
    \begin{equation}
      \epsilon N > 1 \implies \epsilon > \frac{1}{N}
    \end{equation}
    and so, $B_\epsilon (0)$ contains all points $\{1/n\}$ for $n > N$. For the rest of the points $1, 1/2, \ldots, 1/N$, we can simply construct a finite cover over each of them, hence getting a finite cover. 
  \end{solution}

  \begin{exercise}[Rudin 2.13]
    Construct a compact set of real numbers whose limit points form a countable set. 
  \end{exercise}
  \begin{solution}
    Consider the set 
    \begin{equation}
      E = \bigg\{ \bigg( \frac{1}{10}\bigg)^n + \bigg( \frac{1}{10} \bigg)^{n+k} \; : \; n \in \{0\} \cup \mathbb{N}, k \in \mathbb{N} \bigg\} \cup \{0\}
    \end{equation}
    This is clearly bounded by $0$ and $1.1$. Let us represent the elements of this set by $(n, k)$. We can show that 
    \begin{equation}
      (n_1, k_1) > (n_2, k_2)
    \end{equation}
    if $n_1 < n_2$ or $n_1 = n_2$ and $k_1 < k_2$. Therefore, to prove closedness, we must prove that every limit point is a point in $E$. We can do this by proving that a point not in $E$ cannot be a limit point. Choose any $x \not\in E$. Then, due to the ordering, we can see that there exists a $(n, k)$ s.t. 
    \begin{equation}
      A = \bigg( \frac{1}{10}\bigg)^n + \bigg( \frac{1}{10} \bigg)^{n+k} < k < \bigg( \frac{1}{10}\bigg)^n + \bigg( \frac{1}{10} \bigg)^{n+k+1} = B
    \end{equation}
    and so we can take $\epsilon = \min\{k - A, B - k\}$ and show that $B_\epsilon (x)$ does not contain $A$ nor $B$, and so has an empty intersection with $E$. Therefore, it cannot be a limit point of $E$ and is closed. Since $E$ is bounded and closed in $\mathbb{R}$, it is compact. Its limit points contain $1, 0.1, 0.01, \ldots, 0$ (simply by fixing $n$ and letting $k \rightarrow \infty$, and so $E^\prime$ is infinite. We have just shown that since $E$ is closed, $E^\prime \subset E$. But $E$ is countable, so $E^\prime$ is countable. 
  \end{solution}

  \begin{exercise}[Rudin 2.14]
    Given an example of an open cover of the segment $(0, 1)$ which has no finite subcover. 
  \end{exercise}
  \begin{solution}
    Consider 
    \begin{equation}
      (0, 1/2) \cup \bigg( \bigcup_{i=1}^\infty \Big[ 1 - \frac{1}{2^i}, 1 - \frac{1}{2^{i+1}} \Big) \bigg)
    \end{equation}
  \end{solution}

  \begin{exercise}[Rudin 2.15]

  \end{exercise}
  \begin{solution}
    
  \end{solution}

  \begin{exercise}[Rudin 2.16]
    Regard $\mathbb{Q}$, the set of all rational numbers, as a metric space, with $d(p, q) = |p - q|$. Let $E$ be the set of all $p \in \mathbb{Q}$ s.t. $2 < p^2 < 3$. Show that $E$ is closed and bounded in $\mathbb{Q}$ but is not compact. Is $E$ open in $\mathbb{Q}$? 
  \end{exercise}
  \begin{solution}
    $E$ is clearly bounded by $0$ and $2$ since $0^2 < 2 < p^2 < 3 < 2^2$. It is closed and we can show this by showing that $E^c$ is open. Let $x \in E^c$. Then, $x^2< 2$ or $x^2 > 3$. 
    \begin{enumerate}
      \item $x^2 < 2 \iff -\sqrt{2} < x < \sqrt{2}$. Now let $\epsilon = \min\{ \sqrt{2} - x, x + \sqrt{2}\} > 0$. Then by the Archimidean property there exists a $n \in \mathbb{N}$ s.t. $0 < \frac{1}{n} < \epsilon$. Therefore, the image of $B_{1/n} (x) \subset \mathbb{Q}$ will map onto $(0, 2)$. 

      \item $x^2 > 3 \iff x > \sqrt{3}$ or $x < - \sqrt{3}$. If $x > \sqrt{3}$, then by AP there exists a $n \in \mathbb{N}$ s.t. $x - \frac{1}{n} > \sqrt{3} \implies (x - \frac{1}{n})^2 > 3$. If $x < - \sqrt{3}$, then by AP there exist $n \in \mathbb{N}$ s.t. $x + \frac{1}{n} < -\sqrt{3} \implies (x + \frac{1}{n})^2 > 3$. Either way, the image of $B_{1/n} (x)$ will map within $E^c$. 
    \end{enumerate}
    It is not compact because $E$ is not closed in $\mathbb{R}$. The limit points of $E$ in $\mathbb{R}$ is $[\sqrt{2}, \sqrt{3}] \cup [-\sqrt{3}, -\sqrt{2}]$, which contains irrationals and is clearly not a subset of $E$. Since it is not closed in $\mathbb{R}$, it is not compact in $\mathbb{R}$, and it is not compact in $\mathbb{Q} \subset \mathbb{R}$. It is open because 
    \begin{equation}
      E = \big( (\sqrt{2}, \sqrt{3}) \cup (-\sqrt{3}, \sqrt{2})\big) \cup \mathbb{Q} \subset \mathbb{R}
    \end{equation}
    which is the union of open $(\sqrt{2}, \sqrt{3}) \cup (-\sqrt{3}, \sqrt{2})\big)$ and subset $\mathbb{Q} \subset \mathbb{R}$, and so it is open. 
  \end{solution}

  \begin{exercise}[Rudin 2.17]
    Let $E$ be the set of all $x \in [0, 1]$ whose decimal expansion consists of only the digits $4$ and $7$. Is $E$ countable? Is $E$ dense in $[0, 1]$? Is $E$ compact? Is $E$ perfect? 
  \end{exercise}
  \begin{solution}
    
  \end{solution}

  \begin{exercise}[Rudin 2.18]
    Is there a nonempty perfect set in $\mathbb{R}$ which contains no rational number? 
  \end{exercise}
  \begin{solution}
    
  \end{solution}

  \begin{exercise}[Rudin 2.19]
    Listed. 
    \begin{enumerate}
        \item If $A$ and $B$ are disjoint closed sets in some metric space $X$, prove that they are separated. 
        \item Prove the same for disjoint open sets. 
        \item Fix $p \in X$, $\delta > 0$, define $A$ to be the set of all $q \in X$ for which $d(p, q) < \delta$. Define $B$ similarly, with $>$ in place of $<$. Prove that $A$ and $B$ are separated. 
        \item Prove that every connected metric space with at least two points is uncountable. 
    \end{enumerate}
  \end{exercise}
  \begin{solution}
    Listed. 
    \begin{enumerate}
        \item This is trivial with the fact that the closure of the closure of $A$ is the closure of $A$. 
        \item Let $A, B$ be open. We wish to show that if $x \in A^\prime$, then $x \not\in B$. Assume $x \in B$. Then there exists $\epsilon > 0$ s.t. $B_\epsilon (x) \subset B$. But $B \cap A = \emptyset \implies B_\epsilon (x) \cap A = \emptyset$ and so $x \not\in A^\prime$, which is a contradiction. 

        \item Clearly, $A \cap B = \emptyset$. Not let $x \in A \implies$ there exists $\epsilon > 0$ s.t. $B_\epsilon (x) \subset A \implies B_\epsilon (x) \cap B = \emptyset \implies x \in B^\prime$. We can prove similarly to show that $x \in B \implies x \not\in A^\prime$. 

        \item Assume $X$ is countable (solutionis very similar for finite). Then, we can enumerate a $X = \{x_i\}_{i=1}^\infty$. We wish to show that $X$ can be decomposed into the union of an open ball and the interior of its complement as shown in (3). We fix $p \in X$. Then, we take the set $D = \{d(p, x)\}_{x \neq p} \subset \mathbb{R}$. Since $D$ is a countable subset of $\mathbb{R}$, there must exist some $\alpha > 0$ s.t. $\alpha \not\in D$. This $\alpha$ partitions the distances into two sets, and we can define 
        \[X = \{q \in X \mid d(p, q) < \alpha\} \cup \{q \in X \mid d(p, q) > \alpha\}\]
        and by (3), these two sets are separated, which means that $X$ is not connected, leading to a contradiction. 
    \end{enumerate}
  \end{solution}

  \begin{exercise}[Rudin 2.20]
    Are closures and interiors of connected sets always connected? Look at subsets of $\mathbb{R}^2$. 
  \end{exercise}
  \begin{solution}
    The interiors are not always connected. Consider the two closed balls $\overline{B_1 ((1, 0))}$ and $\overline{B_1 ((-1, 0))}$ as subsets of $\mathbb{R}^2$. They are connected but their interiors, which are the two open balls, are not connected. 

    As for closures, they are always connected. Let $W$ be connected. Then for any partition $A \cup B = W$, $\overline{A} \cap B \neq \emptyset$ WLOG. Consider $\overline{W} = W \cup W^\prime$ and take any partition $\overline{W} = C \cup D$. Then, label $A = C \cap W, A^\ast = C \cap W^\prime, B = D \cap W, B^\ast = D \cap W^\prime$. This implies that $C = A \cup A^\ast, D = B \cup B^\ast$, and $A \cup B = W$ (which is connected). Then, we can show that 
    \begin{align*}
        \overline{C} \cap D & = (\overline{A \cup A^\ast} \cap D) = (\overline{A} \cup \overline{A^\ast}) \cap D = (\overline{A} \cap D) \cup (\overline{A^\ast} \cap D) \\
        & =(\overline{A} \cap B) \cup (\overline{A} \cap B^\ast) \cup (\overline{A^\ast} \cap D)
    \end{align*}
    which cannot be empty since by connectedness of $W$, $\overline{A} \cap B \neq \emptyset$. Therefore, $\overline{W}$ is connected. 
  \end{solution}

  \begin{exercise}[Rudin 2.21]
    Let $A$ and $B$ be separated subsets of some $\mathbb{R}^k$. Suppose $a \in A, b \in B$ and define 
    \begin{equation}
      p(t) = (1 - t) a + t b
    \end{equation}
    for $t \in \mathbb{R}$. Put $A_0 = p^{-1} (A), B_0 = p^{-1} (B)$. 
    \begin{enumerate}
        \item Prove that $A_0$ and $B_0$ are separated subsets of $\mathbb{R}$. 
        \item Prove that there exists a $t_0 \in (0, 1)$ s.t. $p(t_0) \not\in A \cup B$. 
        \item Prove that every convex subset of $\mathbb{R}^k$ is connected. 
    \end{enumerate}
  \end{exercise}
  \begin{solution}
    
  \end{solution}

  \begin{exercise}[Rudin 2.22]
    A metric space is called \textit{separable} if it contains a countable dense subset. Show that $\mathbb{R}^k$ is separable. 
  \end{exercise}
  \begin{solution}
    Consider the set $\mathbb{Q}^k \subset \mathbb{R}^k$. It is a finite Cartesian product (and hence, a countable union) of countable $\mathbb{Q}$, and so it is countable. $\mathbb{Q}^k$ is dense in $\mathbb{R}^k$ since given any $x \in \mathbb{R}^k$, we claim $x$ is a limit point of $\mathbb{Q}^k$. Given any $\epsilon > 0$, we can construct $B_\epsilon^\circ (x)$. For each coordinate $x_i$, by density of rationals in $\mathbb{R}$ we can choose a $q_i \in \mathbb{Q}$ s.t. $0 < d(x_i, q_i) < \epsilon / k$. Then, using triangle inequality, we can take the distances between each coordinate changed from $x_i$ to $q_i$. Let $q^k$ be the vector $x$ with the components $x_1, \ldots, x_k$ changed to $q_1, \ldots, q_k$, respectively. 
    \begin{equation}
      d(x, q) = d(x, q^1) + d(q^1, q^2) + \ldots + d(q^{k-1}, q_k) < \frac{\epsilon}{k} + \ldots + \frac{\epsilon}{k} = \epsilon
    \end{equation}
    and so $q \in B_\epsilon^\circ (x)$. Hence the intersection of $\mathbb{Q}^k$ and $B_\epsilon^\circ (x)$ for any $\epsilon > 0$ is nontrivial, so $x$ is a limit point of $\mathbb{Q}^k$. 
  \end{solution}

  \begin{exercise}[Rudin 2.23]
    A collection $\{V_\alpha\}$ of open subsets of $X$ is said to be a \textit{base} for $X$ if the following is true: For every $x \in X$ and every open set $G \subset X$ such that $x \in G$, we have $x \in V_\alpha \subset G$ for some $\alpha$. In other words, every open set in $X$ is the union of a subcollection of $\{V_\alpha\}$. Prove that every separable metric space has a countable base. 
  \end{exercise}
  \begin{solution}
    Since $X$ is separable it contains a countable dense subset, call it $S$. Then for every $x \in S$, we can look at the set of all open balls with center $x$ and rational radii, call it $\mathcal{B}$. Then $\mathcal{B}$ is countable. Now consider an open set $U$. By definition, for every $x \in U$, there exists an $\epsilon > 0$ s.t. $B_\epsilon (x) \subset U$. By AP, we can find a $n \in \mathbb{N}$ s.t. $0 < \frac{1}{n} < \epsilon$, and therefore we can find an open ball $B \in \mathcal{B}$ s.t. $B (x) \subset U$. We claim that 
    \begin{equation}
      W \coloneqq \bigcup_{x \in U} B (x) = U
    \end{equation}
    If $x \in U$, then by construction it is contained in $B(x) \subset \cup_{x \in U } B(x)$, and so $U \subset W$. If $x \in W$, then it is in $B(x)$, which is fully contained in $U$ and so $W \subset U$. Therefore every open set can be constructed by a countable union of open balls in countable $\mathcal{B}$. 
  \end{solution} 

  \begin{exercise}[Rudin 2.24]
    Let $X$ be a metric space in which every infinite subset has a limit point. Prove that $X$ is separable. 
  \end{exercise}
  \begin{solution}
    We fix $\delta > 0$. Choose $x_1 \in X$. Then choose $x_2 \in X$ s.t. $d(x_1, x_2) \geq \delta$, and keep doing this until we choose $x_{j + 1} \in X$ s.t. $d(x_{j+1}, x_i) \geq \delta$ for all $i \in 1, \ldots, j$. 
    \begin{enumerate}
        \item We claim that this must stop after a finite number of steps. Assume it doesn't. Then by assumption $V = \{x_i\}_{i=1}^\infty$ should have a limit point in $X$, denote it $x$. Choose $\frac{\delta}{2} > 0$. Then, $B_{\delta / 2}^\circ (x) \cap V \neq \emptyset$. This intersection can only have one point since if it had two $x^\prime, x^{\prime\prime}$, then since both are in $B_{\delta /2} (x)$, then 
        \[d(x^\prime, x^{\prime\prime}) \leq d(x^\prime, x) + d(x, x^{\prime\prime}) \leq \frac{\delta}{2} + \frac{\delta}{2} = \delta\]
        and since they are both in $V$, then $d(x^\prime, x^{\prime\prime}) \geq \delta$, which is a contradiction. Since there is a finite number of points in $B_{\delta /2} (x)$ of $V$, $x$ cannot be a limit point. So this must terminate at some finite $J < \infty$. 
        \item Denote $W = \{x_i\}_{i=1}^J$. Then, $\mathscr{B}_\delta = \{B_\delta (x) \mid x \in W\}$ must cover $X$, since if it didn't, there would exist a $y \in X$ s.t. $d(y, x) \geq \delta$ for all $x \in W$, and we can add another element in $W$. 
        \item Consider $\delta = 1, 1/2, 1/3, \ldots$ and construct the same cover 
        \[\mathscr{B}_k = \{B_{1/k} (x_{ki}) \mid i = 1, \ldots, J_k\}\]
        which is finite. Therefore, $\mathscr{B} = \cup_{k=1}^\infty \mathscr{B}_k$ must be countable. 
        \item We claim that countable $\{x_{ki}\}_{k, i}$ is dense. Consider any $x \in X$. For every $\epsilon > 0$, we can find an arbitrarily large $n \in \mathbb{N}$ s.t. $0 < \frac{1}{n} < \epsilon$. Since $\mathscr{B}_n$ is an open cover, there must exist some $x_{n i}$ s.t. $x \in B_{1/n} (x_{ni})$, which by symmetry implies that $x_{ni} \in B_{1/n} (x) \subset B_\epsilon (x)$. Therefore, there always exists an $x_{ni}$ in every $B_\epsilon (x)$, and so $B_\epsilon (x) \cap \{x_{ki}\} \neq \emptyset \implies x$ is a limit point of $\{x_{ki}\}$ and so it is dense. 
    \end{enumerate}
  \end{solution}

  \begin{exercise}[Rudin 2.25]
    Prove that every compact metric space $K$ has a countable base, and that $K$ is therefore separable. 
  \end{exercise}
  \begin{solution}
    For every $n \in \mathbb{N}$, let us consider an open covering $\mathscr{F}_n \coloneqq \{B_{1/n} (x_n) \mid x_n \in K\}$. Since $K$ is compact, it has a finite subcovering 
    \begin{equation}
      \mathscr{G}_n \coloneqq \{B_{1/n} (x_{ni}) \mid i = 1, \ldots, k(n)\}
    \end{equation}
    Now consider the union $\mathscr{G} = \cup_{i=1}^n \mathscr{G}_n$, which is countable. We claim that $\mathscr{G}$ is a base. Consider any open set $U$. Then for every $x \in U$, we want to show that $x$ is contained in a $B_{1/n} (x_{ni}) \subset U$. Since $U$ is open , there exists a $\epsilon > 0$ s.t. $B_\epsilon (x) \subset U$. Now by AP, there exists a $n \in \mathbb{N}$ s.t. $0 < \frac{1}{n} < \frac{\epsilon}{2}$. Therefore $B_{1/n} (x) \subset B_{\epsilon} (x)$. Since $\mathscr{G}$ is an open covering, there must exist some $B_{1/n} (x_{ni})$ that contains $x$. Now we wish to show that $B_{1/n} (x_{ni})$ is fully contained in $U$. Let $y \in B_{1/n} (x_{ni})$. Then, by triangle inequality, 
    \begin{equation}
      d(y, x) = d(y, x_{ni}) + d(x_{ni}, x) < \frac{1}{n} + \frac{1}{n} < \epsilon
    \end{equation}
    and therefore $x \in B_{1/n} (x_{ni}) \subset B_{\epsilon} (x)$. Therefore, for every $x \in U$, we can construct an open ball of $\mathscr{G}$ containing $x$ and contained in $U$, proving that this is a base. 

    We claim that the set of all $\mathscr{P} = \{x_{ni}\}_{n, i}$ forms a countable dense subset. This is clearly countable since $\mathscr{G}$ is countable. We must prove that the closure of $\mathscr{P} = K$. Let $x \in K$. Given any $\epsilon > 0$, we wish to show that $B_{\epsilon} (x) \cap \mathscr{P} \neq \emptyset$. Since $B_\epsilon (x)$ is open, it can be covered by a subcollection of $\mathscr{G}$, and so their centers must be in $B_\epsilon (x)$, proving that $B_{\epsilon} (x) \cap \mathscr{P} \neq \emptyset$. Therefore, $x$ is a limit point of $\mathscr{P}$. 
  \end{solution}

  \begin{exercise}[Rudin 2.26]

  \end{exercise}
  \begin{solution}
    
  \end{solution}

  \begin{exercise}[Rudin 2.27]

  \end{exercise}
  \begin{solution}
    
  \end{solution}

  \begin{exercise}[Rudin 2.28]

  \end{exercise}
  \begin{solution}
    
  \end{solution}

  \begin{exercise}[Rudin 2.29]
    Prove that every open set in $\mathbb{R}$ is the union of an at most countable collection of disjoint segments. 
  \end{exercise}
  \begin{solution}
    Let $U \subset \mathbb{R}$ be open. Then for all $x \in U$ there exists $\epsilon > 0$ s.t. $(x - \epsilon, x + \epsilon) \subset U$. Now since $\mathbb{R}$ is separable (by exercise Rudin 2.22), it has a countable dense subset $\mathbb{Q}$. Consider all segments of rational radius and rational centers 
    \begin{equation}
      \mathscr{B} = \{ (q - p, q + p) \subset \mathbb{R} \mid q, p \in \mathbb{Q}\}
    \end{equation}
    This is clearly countable. We claim that every open $U$ can be expressed as the union of a subset of $\mathscr{B}$. Now by AP, there exists $n \in \mathbb{N}$ s.t. $0 < \frac{1}{n} < \frac{\epsilon}{2}$, so for all $x \in U$, there exists $n \in \mathbb{N}$ s.t. $(x - \frac{1}{n}, x + \frac{1}{n}) \subset U$. Now since $\mathbb{Q}$ is dense in $\mathbb{R}$, $x \in \mathbb{Q}^\prime \implies (x - \frac{1}{n}, x + \frac{1}{n}) \cap \mathbb{Q} \neq \emptyset$. Say $r$ is in this intersection. Then, by symmetry of metric, $x \in (r - \frac{1}{n}, r + \frac{1}{n})$. Therefore, for all $x \in U$, we have found an open ball in $\mathscr{B}$ that contains $x$. Now, we must show that this actually is fully contained in $U$. This is easy, since if $y \in B_{1/n} (r)$, then 
    \begin{equation}
      d(y, x) \leq d(y, r) + d(r, x) \leq \frac{1}{n} + \frac{1}{n} < \epsilon
    \end{equation}
    and so $B_{1/n} (r)$ is complete contained in the $\epsilon$-ball around $x$, which is a subset of $U$. So for all $x \in U$, we found an open set $U_x \in \mathscr{B}$ covering $x$ and fully contained in $U$, which means that $\cup_{x \in U} U_x = U$. Now for some intervals $B_1, B_2 \in \mathscr{B}$, if $B_1 \cap B_2 \neq \emptyset$, take their union, which is another segment, and keep doing this until $B_i \cap B_j \neq \emptyset$ for all $i, j$. The cardinality of this new pruned set will be less than or equal to $\mathscr{B}$, which is countable, and so this must be at most countable. 
  \end{solution}

\subsection{Sequences in Euclidean Space}

  \begin{exercise}[Math 531 Spring 2025, PS4.6]
    Consider the set of all bounded sequences of real numbers. That is, we
    consider sequences $\{x_n\}$ for which
    \begin{equation}
      \sup_{n\in\mathbb{N}} |x_n|
    \end{equation}
    exists. For example, the sequence $\{1,2,3,\ldots\}$ does not belong to the set,
    but the sequence $\{1,-\frac{1}{2},\frac{1}{3},-\frac{1}{4},\ldots\}$ does. Call this set $X$. Endow it with
    a metric:
    \begin{equation}
      d(\{x_n\},\{y_n\}) = \sup_{n\in\mathbb{N}} |x_n - y_n|.
    \end{equation}
    Explain why this is a metric. Make sure to explain why the supremum on
    the right hand side exists.
  \end{exercise}
  \begin{solution}
    
  \end{solution}

  \begin{exercise}[Math 531 Spring 2025, PS4.7]
    Consider the metric space $(X,d)$ from the previous problem. Is $\overline{B_1(\{0\})}$
    a compact set? Here, $\{0\}$ is just the sequence of zeros: $\{0,0,0,0,\ldots\}$.
  \end{exercise}
  \begin{solution}
    
  \end{solution}
  
  \begin{exercise}[Rudin 3.1]
    Prove that convergence of $\{x_n\}$ implies convergence of $\{|x_n|\}$. Is the converse true? 
  \end{exercise}
  \begin{solution}
    If $\{x_n\}$ converges to $x$, then for all $\epsilon > 0$, there exists a $N \in \mathbb{N}$ s.t. $|x_n - x| < \epsilon$ if $n > N$. We use the inequality $\big| |x_n| - |x| \big| \leq |x_n - x|$ to show that then for every $\epsilon > 0$ there exists a $N \in \mathbb{N}$ s.t. 
    \[ \big| |x_n| - |x| \big| \leq |x_n - x| \leq \epsilon \]
    and so $\{|x_n|\}$ converges to $|x|$. 
  \end{solution}

  \begin{exercise}[Rudin 3.2]
    Calculate 
    \[\lim_{n \rightarrow \infty} \sqrt{n^2 + n} - n\]
  \end{exercise}
  \begin{solution}
    We can compute  
    \begin{align*}
        \lim_{n \rightarrow \infty} \sqrt{n^2 + n} - n & = \lim_{n \rightarrow \infty} (\sqrt{n^2 + n} - n) \cdot \frac{\sqrt{n^2 + n} + n}{\sqrt{n^2 + n} + n} = \lim_{n \rightarrow \infty} \frac{n}{\sqrt{n^2 + n} + n}
    \end{align*}
    where 
    \[A_n = \frac{n}{\sqrt{n^2 + 2n + 1} + n} \leq \frac{n}{2n + 1} \leq \frac{n}{\sqrt{n^2 + n} + n} \leq \frac{n}{\sqrt{n^2} + n} = \frac{n}{2n} = \frac{1}{2} = C_n\]
    $C_n$ is ultimately constant. It suffices to prove that $A_n$ limits to $\frac{1}{2}$ by showing that 
    \[\frac{n}{2n + 1} = \frac{n/n}{(2n+1)/n} = \frac{1}{2 + \frac{1}{n}}\]
    where $\{\frac{1}{n}\}$ is infinitesimal. 
  \end{solution}

  \begin{exercise}[Rudin 3.3]
    If $s_1 = \sqrt{2}$ and 
    \[s_{n+1} = \sqrt{2 + \sqrt{s_n}}\]
    for $n = 1, 2, \ldots$, prove that $\{s_n\}$ converges and that $s_n < 2$ for $n = 1, 2, \ldots$. 
  \end{exercise}
  \begin{solution}
    We can show that $s_n < 2$ by induction. $s_1 = \sqrt{2} < 2$, so the base case is proved. Now, given that $s_n < 2$, $\sqrt{s_n} < 2 \implies 2 + \sqrt{s_n} < 2 + \sqrt{2} < 4 \implies s_{n+1} = \sqrt{2 + \sqrt{s_n}} < 2$ and we are done. 
  \end{solution}

  \begin{exercise}[Rudin 3.4]
    Find the upper and lower limits of the sequence $\{s_n\}$ defined by
    \begin{align*}
      s_1 = 0; \quad s_{2m} = \frac{s_{2m-1}}{2}; \quad s_{2m+1} = \frac{1}{2} + s_{2m}.
    \end{align*}
  \end{exercise}
  \begin{solution}
    
  \end{solution}

  \begin{exercise}[Rudin 3.5]
    For any two real sequences $\{a_n\}$, $\{b_n\}$, prove that
    \begin{align*}
      \limsup_{n \to \infty} (a_n + b_n) \leq \limsup_{n \to \infty} a_n + \limsup_{n \to \infty} b_n,
    \end{align*}
    provided the sum on the right is not of the form $\infty - \infty$.
  \end{exercise}
  \begin{solution}
    
  \end{solution}

  \begin{exercise}[Rudin 3.6]
    Investigate the behavior (convergence or divergence) of $\Sigma a_n$ if
    \begin{enumerate} 
      \item[(a)] $a_n = \sqrt{n+1} - \sqrt{n}$;
      \item[(b)] $a_n = (\sqrt{n+1} - \sqrt{n})/n$;
      \item[(c)] $a_n = (\sqrt[n]{n} - 1)^n$;
      \item[(d)] $a_n = \frac{1}{1+z^n}$, for complex values of $z$.
    \end{enumerate}
  \end{exercise}
  \begin{solution}
    
  \end{solution}

  \begin{exercise}[Rudin 3.7]
    Prove that the convergence of $\Sigma a_n$ implies the convergence of
    \begin{align*}
      \sum \frac{\sqrt{a_n}}{n},
    \end{align*}
    if $a_n \geq 0$.
  \end{exercise}
  \begin{solution}
    
  \end{solution}

  \begin{exercise}[Rudin 3.8]
    If $\Sigma a_n$ converges, and if $\{b_n\}$ is monotonic and bounded, prove that $\Sigma a_n b_n$ converges.
  \end{exercise}
  \begin{solution}
    
  \end{solution}

  \begin{exercise}[Rudin 3.9]
    Find the radius of convergence of each of the following power series:
    \begin{enumerate} 
      \item[(a)] $\sum n^3 z^n$,
      \item[(b)] $\sum \frac{2^n}{n!} z^n$,
      \item[(c)] $\sum \frac{2^n}{n^2} z^n$,
      \item[(d)] $\sum \frac{n^3}{3^n} z^n$.
    \end{enumerate}
  \end{exercise}
  \begin{solution}
    
  \end{solution}

  \begin{exercise}[Rudin 3.10]
    Suppose that the coefficients of the power series $\sum a_n z^n$ are integers, infinitely many of which are distinct from zero. Prove that the radius of convergence is at most 1.
  \end{exercise}
  \begin{solution}
    
  \end{solution}

  \begin{exercise}[Rudin 3.11]
    Suppose $a_n > 0$, $s_n = a_1 + \cdots + a_n$, and $\Sigma a_n$ diverges.
    \begin{enumerate} 
      \item[(a)] Prove that $\sum \frac{a_n}{1+a_n}$ diverges.
      \item[(b)] Prove that
      \begin{align*}
        \frac{a_{N+1}}{s_{N+1}} + \cdots + \frac{a_{N+k}}{s_{N+k}} \geq 1 - \frac{s_N}{s_{N+k}}
      \end{align*}
      and deduce that $\sum \frac{a_n}{s_n}$ diverges.
      \item[(c)] Prove that
      \begin{align*}
        \frac{a_n}{s_n^2} \leq \frac{1}{s_{n-1}} - \frac{1}{s_n}
      \end{align*}
      and deduce that $\sum \frac{a_n}{s_n^2}$ converges.
      \item[(d)] What can be said about
      \begin{align*}
        \sum \frac{a_n}{1+na_n} \text{ and } \sum \frac{a_n}{1+n^2a_n}?
      \end{align*}
    \end{enumerate}
  \end{exercise}
  \begin{solution}
    
  \end{solution}

  \begin{exercise}[Rudin 3.12]
    Suppose $a_n > 0$ and $\Sigma a_n$ converges. Put
    \begin{align*}
      r_n = \sum_{m=n}^{\infty} a_m.
    \end{align*}
    \begin{enumerate} 
      \item[(a)] Prove that
      \begin{align*}
        \frac{a_m}{r_m} + \cdots + \frac{a_n}{r_n} > 1 - \frac{r_n}{r_m}
      \end{align*}
      if $m < n$, and deduce that $\sum \frac{a_n}{r_n}$ diverges.
      \item[(b)] Prove that
      \begin{align*}
        \frac{a_n}{\sqrt{r_n}} < 2(\sqrt{r_n} - \sqrt{r_{n+1}})
      \end{align*}
      and deduce that $\sum \frac{a_n}{\sqrt{r_n}}$ converges.
    \end{enumerate}
  \end{exercise}
  \begin{solution}
    
  \end{solution}

  \begin{exercise}[Rudin 3.13]
    Prove that the Cauchy product of two absolutely convergent series converges absolutely.
  \end{exercise}
  \begin{solution}
    
  \end{solution}

  \begin{exercise}[Rudin 3.14]
    If $\{s_n\}$ is a complex sequence, define its arithmetic means $\sigma_n$ by
    \begin{align*}
      \sigma_n = \frac{s_0 + s_1 + \cdots + s_n}{n+1} \quad (n = 0, 1, 2, \ldots).
    \end{align*}
    \begin{enumerate} 
      \item[(a)] If $\lim s_n = s$, prove that $\lim \sigma_n = s$.
      \item[(b)] Construct a sequence $\{s_n\}$ which does not converge, although $\lim \sigma_n = 0$.
      \item[(c)] Can it happen that $s_n > 0$ for all $n$ and that $\limsup s_n = \infty$, although $\lim \sigma_n = 0$?
      \item[(d)] Put $a_n = s_n - s_{n-1}$, for $n \geq 1$. Show that
      \begin{align*}
        s_n - \sigma_n = \frac{1}{n+1} \sum_{k=1}^{n} ka_k.
      \end{align*}
      Assume that $\lim (na_n) = 0$ and that $\{\sigma_n\}$ converges. Prove that $\{s_n\}$ converges. [This gives a converse of (a), but under the additional assumption that $na_n \to 0$.]
      \item[(e)] Derive the last conclusion from a weaker hypothesis: Assume $M < \infty$, $|na_n| \leq M$ for all $n$, and $\lim \sigma_n = \sigma$. Prove that $\lim s_n = \sigma$, by completing the following outline:
      
      If $m < n$, then
      \begin{align*}
        s_n - \sigma_n = \frac{m+1}{n-m}(\sigma_n - \sigma_m) + \frac{1}{n-m} \sum_{i=m+1}^{n} (s_n - s_i).
      \end{align*}
      For these $i$,
      \begin{align*}
        |s_n - s_i| \leq \frac{(n-i)M}{i+1} \leq \frac{(n-m-1)M}{m+2}.
      \end{align*}
      Fix $\varepsilon > 0$ and associate with each $n$ the integer $m$ that satisfies
      \begin{align*}
        m \leq \frac{n-\varepsilon}{1+\varepsilon} < m+1.
      \end{align*}
      Then $(m+1)/(n-m) \leq 1/\varepsilon$ and $|s_n - s_i| < M\varepsilon$. Hence
      \begin{align*}
        \limsup_{n \to \infty} |s_n - \sigma| \leq M\varepsilon.
      \end{align*}
      Since $\varepsilon$ was arbitrary, $\lim s_n = \sigma$.
    \end{enumerate}
  \end{exercise}
  \begin{solution}
    
  \end{solution}

  \begin{exercise}[Rudin 3.15]
    Definition 3.21 can be extended to the case in which the $a_n$ lie in some fixed $\mathbb{R}^k$. Absolute convergence is defined as convergence of $\Sigma |\mathbf{a}_n|$. Show that Theorems 3.22, 3.23, 3.25(a), 3.33, 3.34, 3.42, 3.45, 3.47, and 3.55 are true in this more general setting. (Only slight modifications are required in any of the proofs.)
  \end{exercise}
  \begin{solution}
    
  \end{solution}

  \begin{exercise}[Rudin 3.16]
    Fix a positive number $\alpha$. Choose $x_1 > \sqrt{\alpha}$, and define $x_2, x_3, x_4, \ldots$, by the recursion formula
    \begin{align*}
      x_{n+1} = \frac{1}{2}\left(x_n + \frac{\alpha}{x_n}\right).
    \end{align*}
    \begin{enumerate} 
      \item[(a)] Prove that $\{x_n\}$ decreases monotonically and that $\lim x_n = \sqrt{\alpha}$.
      \item[(b)] Put $\varepsilon_n = x_n - \sqrt{\alpha}$, and show that
      \begin{align*}
        \varepsilon_{n+1} = \frac{\varepsilon_n^2}{2x_n} < \frac{\varepsilon_n^2}{2\sqrt{\alpha}}
      \end{align*}
      so that, setting $\beta = 2\sqrt{\alpha}$,
      \begin{align*}
        \varepsilon_{n+1} < \beta \left(\frac{\varepsilon_1}{\beta}\right)^{2^n} \quad (n = 1, 2, 3, \ldots).
      \end{align*}
      \item[(c)] This is a good algorithm for computing square roots, since the recursion formula is simple and the convergence is extremely rapid. For example, if $\alpha = 3$ and $x_1 = 2$, show that $\varepsilon_1/\beta < \frac{1}{10}$ and that therefore
      \begin{align*}
        \varepsilon_5 < 4 \cdot 10^{-16}, \quad \varepsilon_6 < 4 \cdot 10^{-32}.
      \end{align*}
    \end{enumerate}
  \end{exercise}
  \begin{solution}
    
  \end{solution}

  \begin{exercise}[Rudin 3.17]
    Fix $\alpha > 1$. Take $x_1 > \sqrt{\alpha}$, and define
    \begin{align*}
      x_{n+1} = \frac{\alpha + x_n}{1 + x_n} = x_n + \frac{\alpha - x_n^2}{1 + x_n}.
    \end{align*}
    \begin{enumerate} 
      \item[(a)] Prove that $x_1 > x_3 > x_5 > \cdots$.
      \item[(b)] Prove that $x_2 < x_4 < x_6 < \cdots$.
      \item[(c)] Prove that $\lim x_n = \sqrt{\alpha}$.
      \item[(d)] Compare the rapidity of convergence of this process with the one described in Exercise 16.
    \end{enumerate}
  \end{exercise}
  \begin{solution}
    
  \end{solution}

  \begin{exercise}[Rudin 3.18]
    Replace the recursion formula of Exercise 16 by
    \begin{align*}
      x_{n+1} = \frac{p-1}{p}x_n + \frac{\alpha}{p}x_n^{-p+1}
    \end{align*}
    where $p$ is a fixed positive integer, and describe the behavior of the resulting sequences $\{x_n\}$.
  \end{exercise}
  \begin{solution}
    
  \end{solution}

  \begin{exercise}[Rudin 3.19]
    Associate to each sequence $a = \{\alpha_n\}$, in which $\alpha_n$ is 0 or 2, the real number
    \begin{align*}
      x(a) = \sum_{n=1}^{\infty} \frac{\alpha_n}{3^n}.
    \end{align*}
    Prove that the set of all $x(a)$ is precisely the Cantor set described in Sec. 2.44.
  \end{exercise}
  \begin{solution}
    
  \end{solution}

  \begin{exercise}[Rudin 3.20]
    Suppose $\{p_n\}$ is a Cauchy sequence in a metric space $X$, and some subsequence $\{p_{n_i}\}$ converges to a point $p \in X$. Prove that the full sequence $\{p_n\}$ converges to $p$.
  \end{exercise}
  \begin{solution}
    
  \end{solution}

  \begin{exercise}[Rudin 3.21]
    Prove the following analogue of Theorem 3.10(b): If $\{E_n\}$ is a sequence of closed nonempty and bounded sets in a complete metric space $X$, if $E_n \supset E_{n+1}$, and if
    \begin{align*}
      \lim_{n \to \infty} \text{diam } E_n = 0,
    \end{align*}
    then $\bigcap_{1}^{\infty} E_n$ consists of exactly one point.
  \end{exercise}
  \begin{solution}
    
  \end{solution}

  \begin{exercise}[Rudin 3.22]
    Suppose $X$ is a nonempty complete metric space, and $\{G_n\}$ is a sequence of dense open subsets of $X$. Prove Baire's theorem, namely, that $\bigcap_{1}^{\infty}G_n$ is not empty. (In fact, it is dense in $X$.) Hint: Find a shrinking sequence of neighborhoods $E_n$ such that $E_n \subset G_n$, and apply Exercise 21.
  \end{exercise}
  \begin{solution}
    
  \end{solution}

  \begin{exercise}[Rudin 3.23]
    Suppose $\{p_n\}$ and $\{q_n\}$ are Cauchy sequences in a metric space $X$. Show that the sequence $\{d(p_n, q_n)\}$ converges. Hint: For any $m, n$,
    \begin{align*}
      d(p_n, q_n) \leq d(p_n, p_m) + d(p_m, q_m) + d(q_m, q_n);
    \end{align*}
    it follows that
    \begin{align*}
      |d(p_n, q_n) - d(p_m, q_m)|
    \end{align*}
    is small if $m$ and $n$ are large.
  \end{exercise}
  \begin{solution}
    
  \end{solution}

  \begin{exercise}[Rudin 3.24]
    Let $X$ be a metric space.
    \begin{enumerate} 
      \item[(a)] Call two Cauchy sequences $\{p_n\}$, $\{q_n\}$ in $X$ equivalent if
      \begin{align*}
        \lim_{n \to \infty} d(p_n, q_n) = 0.
      \end{align*}
      Prove that this is an equivalence relation.
      \item[(b)] Let $X^*$ be the set of all equivalence classes so obtained. If $P \in X^*, Q \in X^*$, $\{p_n\} \in P, \{q_n\} \in Q$, define
      \begin{align*}
        \Delta(P, Q) = \lim_{n \to \infty} d(p_n, q_n);
      \end{align*}
      by Exercise 23, this limit exists. Show that the number $\Delta(P, Q)$ is unchanged if $\{p_n\}$ and $\{q_n\}$ are replaced by equivalent sequences, and hence that $\Delta$ is a distance function in $X^*$.
      \item[(c)] Prove that the resulting metric space $X^*$ is complete.
      \item[(d)] For each $p \in X$, there is a Cauchy sequence all of whose terms are $p$; let $P_p$ be the element of $X^*$ which contains this sequence. Prove that
      \begin{align*}
        \Delta(P_p, P_q) = d(p, q)
      \end{align*}
      for all $p, q \in X$. In other words, the mapping $\varphi$ defined by $\varphi(p) = P_p$ is an isometry (i.e., a distance-preserving mapping) of $X$ into $X^*$.
      \item[(e)] Prove that $\varphi(X)$ is dense in $X^*$, and that $\varphi(X) = X^*$ if $X$ is complete. By (d), we may identify $X$ and $\varphi(X)$ and thus regard $X$ as embedded in the complete metric space $X^*$. We call $X^*$ the completion of $X$.
    \end{enumerate}
  \end{exercise}
  \begin{solution}
    
  \end{solution}

  \begin{exercise}[Rudin 3.25]
    Let $X$ be the metric space whose points are the rational numbers, with the metric $d(x, y) = |x - y|$. What is the completion of this space? (Compare Exercise 24.)
  \end{exercise}
  \begin{solution}
    
  \end{solution}

\subsection{Limits and Continuous Functions}

\subsection{Differentiation of Single-Variable Functions}

\subsection{Riemann Integration} 

\subsection{Sequences of Functions} 

\subsection{Multivariate Functions} 

\subsection{TBD} 

  \begin{exercise}[Math 531 Spring 2025, PS6.1]
    Give a direct proof that
    \begin{equation}
      \sum_{n=1}^{\infty} \frac{1}{k^2+k}
    \end{equation}
    converges and find the exact value of the series.
  \end{exercise}
  \begin{solution}
    This is a telescoping series. 
    \begin{equation}
      \sum_{n=1}^\infty \frac{1}{k^2 + k} = \sum_{n=1}^\infty \frac{1}{k} - \frac{1}{k-1} = 1 - \frac{1}{2} + \frac{1}{2} - \frac{1}{3} + \frac{1}{3} - \frac{1}{4} + \ldots 
    \end{equation}
    and so the $n$th partial sum is $1 - \frac{1}{n} \to 1$ as $n \to \infty$. 
  \end{solution}

  \begin{exercise}[Math 531 Spring 2025, PS6.2]
    Suppose that $a_n$ is a sequence of non-negative real numbers and suppose that $\sum_{n=1}^{\infty} a_n$ converges. Prove that there exists a sequence $b_n$ with $\lim_{n \to \infty} b_n = +\infty$ so that $\sum_{n=1}^{\infty} a_n b_n$ is still convergent.
  \end{exercise}
  \begin{solution}
    Let us take $r_n = \sum_{m=n}^\infty a_n$. Since $a_n$ is nonnegative, $r_n = 0 \iff a_m = 0$ for all $m \geq n$. In this case set 
    \begin{equation}
      (b_m) \coloneqq \begin{cases}
        1 & \text{ if } m < n \\ 
        m & \text{ if } m \geq n
      \end{cases}
    \end{equation}
    and 
    \begin{equation}
      \sum_{n=1}^\infty a_n = \sum_{m=1}^n a_m = \sum_{m=1}^n a_m b_m = \sum_{m=1}^\infty a_m b_m < +\infty 
    \end{equation}
    So we may assume $r_n > 0$. Now define $(b_n) = \big( \frac{1}{\sqrt{r_n}} \big)$. By the Cauchy criterion, $r_n \to 0$ as $n \to +\infty$, so $b_n \to +\infty$. But we claim that $\frac{a_n}{\sqrt{r_n}} < 2(\sqrt{r_n} - \sqrt{r_{n+1}})$ since 
    \begin{equation}
      \frac{a_n}{\sqrt{r_n}} (\sqrt{r_n} + \sqrt{r_{n+1}}) = a_n + a_n \frac{\sqrt{r_{n+1}}}{\sqrt{r_n}} < 2 a_n = 2(r_n - r_{n+1})  
    \end{equation}
    Since the series $\sum (\sqrt{r_n} - \sqrt{r_{n+1}})$ converges to $\sqrt{r_1}$, it follows by the comparison test that $\sum \frac{a_n}{\sqrt{r_n}}$ also converges. 
  \end{solution}

  \begin{exercise}[Math 531 Spring 2025, PS6.3]
    Prove that $\sum_{n=1}^{\infty} \frac{\sin(n)}{n}$ converges\footnote{Here, we are going to assume that we know a little bit about trigonometric functions!}. Hint: Use summation by parts and the fact that $\sin(x) = \frac{1}{2i}(\exp(ix) - \exp(-ix))$ where $i = \sqrt{-1}$.
  \end{exercise}
  \begin{solution}
    We can see that 
    \begin{equation}
      \bigg| \mathrm{Im} \sum_{n=1}^N e^{in} \bigg| = \bigg| \sum_{n=1}^N \sin(n) \bigg| \leq \bigg| e^i \frac{1 - e^{in}}{1 - e^i} \bigg| \leq \frac{2}{|1 - e^i|} < +\infty
    \end{equation} 
    and so the partial sums form a bounded sequence, implying that $\sum \frac{\sin(n)}{n}$ is convergent. 
  \end{solution}

  \begin{exercise}[Math 531 Spring 2025, PS6.4]
    Fix $c \in (0, \infty)$.
    \begin{itemize}
      \item Assume $x_n$ is a sequence of numbers with $x_n \to 0$. Prove that
        \begin{equation}
          c^{x_n} \to 1.
        \end{equation}
      
      \item Deduce that, if $x_n \to x$, then
        \begin{equation}
          c^{x_n} \to c^x,
        \end{equation}
        for any $x \in \mathbb{R}$.
    \end{itemize}
  \end{exercise}
  \begin{solution}
    $x_n \to 0$ as $n \to \infty$ means that $\forall \delta > 0$, $\exists N \in \mathbb{N}$ s.t. $|x_n| < \delta$ for all $n \geq N$. Now take $\epsilon > 0$. Then we wish to show that $\exists M \in \mathbb{N}$ s.t. $|1 - c^{x_m}| < \epsilon$ for all $n \geq M$. We choose $\delta = \min\{-\log_c (1 - \epsilon), \log_c (1 + \epsilon)\}$. Then $\exists M$ s.t. 
    \begin{equation}
      x_n \in \big( -\log_c (1 - \epsilon), \log_c (1 + \epsilon) \big) \iff |x_n| < \delta
    \end{equation}
    for $n \geq M$ by convergence $x_n \to 0$. Since $f(x) = c^x$ is monotonically increasing, this is equivalent to 
    \begin{equation}
      f(x_n) = c^{x_n}
    \end{equation}
  \end{solution}

  \begin{exercise}[Math 531 Spring 2025, PS6.5]
    For every $z \in \mathbb{C}$, verify that the series
    \begin{equation}
      E(z) = \sum_{n=0}^{\infty} \frac{z^n}{n!}
    \end{equation}
    converges. Prove that for every $z, w \in \mathbb{C}$ we have that $E(z)E(w) = E(z + w)$. Deduce that, for $q \in \mathbb{Q}$, we have that $E(q) = e^q$. Deduce that $E(x) = e^x$, for all $x \in \mathbb{R}$.
  \end{exercise}
  \begin{solution}

  \end{solution}

  \begin{exercise}[Math 531 Spring 2025, PS6.6]
    Let $X$ be a complete metric space. Suppose that $f : X \to X$ is such that $d(f(x), f(y)) \leq \frac{1}{2}d(x, y)$ for all $x, y \in X$.
    \begin{enumerate}
      \item[(a)] Prove that $f$ is continuous
      
      \item[(b)] Pick any $x_0 \in X$. Define a sequence of points $\{x_n\}$ in $X$ by:
        \begin{equation}
          x_{n+1} = f(x_n).
        \end{equation}
        Prove that $\{x_n\}$ converges. Hint: use a previous homework assignment.
      
      \item[(c)] Denote the limit of $\{x_n\}$ by $x$. Prove that $f(x) = x$.
      
      \item[(d)] Prove that if $f(y) = y$ and $f(x) = x$ then $x = y$.
    \end{enumerate}
  \end{exercise}
  \begin{solution}

  \end{solution}

  \begin{exercise}[Math 531 Spring 2025, PS6.7]
    Does there exist a continuous function $f : [0, 1) \to \mathbb{R}$ that is onto? If so, construct one from scratch. If not, prove such a function cannot exist.
  \end{exercise}
  \begin{solution}

  \end{solution}

  \begin{exercise}[Math 531 Spring 2025, PS7.1]
    If a function $f : [0, 1] \to \mathbb{R}$ is continuous, it is uniformly continuous. This means that, given $\epsilon > 0$, there exists $\delta(\epsilon) > 0$ so that
    \begin{equation}
      d(x, y) < \delta(\epsilon) \Rightarrow d(f(x), f(y)) < \epsilon.
    \end{equation}
    \begin{itemize}
      \item Prove that if $f(x) = 1$ for all $x$, we can take $\delta(\epsilon) = +\infty$.
      \item Prove that if $f(x) = Mx$, we can take $\delta(\epsilon) = \frac{\epsilon}{M}$.
      \item Prove that if $f(x) = \sqrt{x}$, we can take $\delta(\epsilon) = \frac{\epsilon^2}{4}$.
      \item Prove that if $f(x) = \frac{x}{x+1}$, we can take $\delta(\epsilon) = \epsilon$.
      \item Prove that if $f(x) = x^N$, for some $N \in \mathbb{N}$, then we can take $\delta(\epsilon) = \frac{\epsilon}{N}$.
    \end{itemize}
    You cannot use differentiation for Problem 1. Do everything from scratch.
  \end{exercise}
  \begin{solution}

  \end{solution}

  \begin{exercise}[Math 531 Spring 2025, PS7.2]
    Let $X$ be a general metric space. A function $f : X \to Y$ is called compact if the image of every closed ball $B_r(x)$ is compact in $Y$. Prove that any continuous $f : \mathbb{R}^n \to \mathbb{R}^n$ is compact. Give an example of a discontinuous function $f : \mathbb{R} \to \mathbb{R}$ that is compact.
  \end{exercise}
  \begin{solution}

  \end{solution}

  \begin{exercise}[Math 531 Spring 2025, PS7.3]
    Prove that if $f : \mathbb{R} \to \mathbb{R}$ sends connected sets to connected sets and compact sets to compact sets, then it is continuous. Hint: Assume that $f : \mathbb{R} \to \mathbb{R}$ sends every connected set to a connected set. Assume also that $f$ is discontinuous at some $x \in \mathbb{R}$. Find a compact set $K$ so that $f(K)$ is not compact.
  \end{exercise}
  \begin{solution}

  \end{solution}

  \begin{exercise}[Math 531 Spring 2025, PS7.4]
    \begin{itemize}
      \item Let $X$ be a general metric space and assume $K \subset X$ is compact. Let $f : K \to K$ and assume that
      \begin{equation}
        d(f(x), f(y)) < d(x, y)
      \end{equation}
      for all $x \neq y \in K$. Show that there exists $x \in K$ with $f(x) = x$.
      
      \item Find a function $f : [0, \infty) \to [0, \infty)$ with the property that $|f(x) - f(y)| < |x - y|$, for all non-equal $x, y \in [0, \infty)$, but for which there is no point $x \in [0, \infty)$ for which $f(x) = x$.
    \end{itemize}
    
    Hint for the first part: define the function $g : K \to \mathbb{R}$ by $g(x) = d(f(x), x)$. First prove that $g$ is continuous. Then deduce that $g$ must attain its minimum at some $x^*$. Then show $g(f(x^*)) < g(x^*)$ unless $f(x^*) = x^*$. Conclude that $f(x^*) = x^*$.
  \end{exercise}
  \begin{solution}

  \end{solution}

  \begin{exercise}[Math 531 Spring 2025, PS7.5]
    Coming back to Problem 1 above, assume that $f$ is differentiable on $[0, 1]$ and that $|f'(x)| \leq M$ for every $x \in [0, 1]$. Prove that we can take $\delta(\epsilon) = \frac{\epsilon}{M}$.
  \end{exercise}
  \begin{solution}

  \end{solution}

  \begin{exercise}[Math 531 Spring 2025, PS7.6]
    A function $f : X \to X$ is said to be Hölder continuous of degree $\alpha$ for some $\alpha \in (0, \infty)$ if there exists $M > 0$ so that
    \begin{equation}
      d(f(x), f(y)) \leq C \cdot d(x, y)^{\alpha}.
    \end{equation}
    
    Prove that if $f$ is Hölder continuous of degree $\alpha > 0$, then $f$ is continuous. Give an example of a function on $\mathbb{R}$ that is Hölder continuous of degree $\frac{1}{2}$ but not of degree $1$. Prove that continuously differentiable functions $[0,1]$ are Hölder continuous of degree $1$. Prove that the only functions on $\mathbb{R}$ that are Hölder continuous of degree larger than $1$ are constants.
  \end{exercise}
  \begin{solution}

  \end{solution}

  \begin{exercise}[Math 531 Spring 2025, PS7.7]
    Suppose that $f : [0, 2] \to [0, 2]$ is twice differentiable. Suppose $f(0) = 0$, $f(1) = 1$, and $f(2) = 2$. Prove that there exists $c \in (0, 2)$ so that $f''(c) = 0$.
  \end{exercise}
  \begin{solution}

  \end{solution}

  \begin{exercise}[Math 531 Spring 2025, PS7.8]
    Suppose that $f$ is differentiable on $[0, 1]$ satisfies:
    \begin{equation}
      f'(x) = f(x).
    \end{equation}
    
    \begin{itemize}
      \item Prove that $f$ is automatically infinitely differentiable.
      
      \item Let $M = \max\{f(x) : x \in [0, 1]\}$. Why does $M$ exist? Similarly, let $M(\epsilon) = \max\{|f(x)| : x \in [0, \epsilon]\}$.
      
      \item Prove that for all $x \in [0, \epsilon]$, we have that
      \begin{equation}
        |f(x)| \leq M(\epsilon)x + |f(0)|.
      \end{equation}
      
      \item Deduce that if $f(0) = 0$, we have that
      \begin{equation}
        M\left(\frac{1}{2}\right) = 0.
      \end{equation}
      Similarly, deduce that $M(c) = 0$ for all $c \in [0, 1]$.
      
      \item What you have just proved is that if $f'(x) = f(x)$ while $f(0) = 0$, it follows that $f(x) = 0$ for all $x$.
      
      \item Assume that $E : \mathbb{R} \to [0, \infty)$ is differentiable and
      \begin{equation}
        |E'(t)| \leq CE(t).
      \end{equation}
      Prove that if $E(0) = 0$, then $E(t) = 0$ for all $t$.
    \end{itemize}
  \end{exercise}
  \begin{solution}

  \end{solution}

  \begin{exercise}[Math 531 Spring 2025, PS8.1]
    Prove that if $f : [0, 1] \to \mathbb{R}$ is differentiable and $f' > 0$ on $[0, 1]$, then $f$ is strictly increasing on $[0, 1]$.
  \end{exercise}
  \begin{solution}

  \end{solution}

  \begin{exercise}[Math 531 Spring 2025, PS8.2]
    Explain why if $x(t)$ represents the position of a particle at time $t$, $x'(t)$ is called the velocity of the particle, $x''(t)$ is called its acceleration, and $x'''(t)$ is called its jerk.
  \end{exercise}
  \begin{solution}

  \end{solution}

  \begin{exercise}[Math 531 Spring 2025, PS8.3]
    A function $f : \mathbb{R} \to \mathbb{R}$ is said to be $T$ periodic if $f(t + T) = f(t)$, for all $t \in \mathbb{R}$. Now, given a function $\tilde{f}$ defined on $[0, T)$, we can always extend $\tilde{f}$ to be $T$ periodic in the following way. First, we can write
    \begin{equation}
      \mathbb{R} = \bigcup_{n\in\mathbb{Z}} [nT,(n + 1)T).
    \end{equation}
    Then define for $t \in [nT,(n + 1)T)$ :
    \begin{equation}
      f(t) = \tilde{f}(t - nT).
    \end{equation}
    
    \begin{itemize}
      \item Show that $f$ defined this way is $T$ periodic.
      \item Suppose $\tilde{f}$ is continuous on $[0, T)$. Does this mean that its extension $f$ will also be continuous on $\mathbb{R}$? What condition do you have to add?
      \item Suppose $\tilde{f}$ is continuously differentiable on $[0, T)$. What conditions do you have to put on $\tilde{f}$ to ensure that $f$ is continuously differentiable? How about $k$-times continuously differentiable?
    \end{itemize}
  \end{exercise}
  \begin{solution}

  \end{solution}

  \begin{exercise}[Math 531 Spring 2025, PS8.4]
    Assume $f : \mathbb{R} \to \mathbb{R}$ is differentiable and $|f'(x)| \leq \frac{1}{1+x^2}$. Prove that
    \begin{equation}
      \lim_{x\to\infty}f(x)
    \end{equation}
    exists. You are not allowed to use integration. Hint: How do you prove convergence without knowing what the limit is?
  \end{exercise}
  \begin{solution}

  \end{solution}

  \begin{exercise}[Math 531 Spring 2025, PS8.5]
    In a previous homework assignment, we defined the function
    \begin{equation}
      E(x) = \sum_{n=0}^{\infty} \frac{x^n}{n!}.
    \end{equation}
    
    We proved it is continuous, satisfies $E(z+w) = E(z)E(w)$ for all $z, w \in \mathbb{C}$, and then deduced that for all $x \in \mathbb{R}$, we have that $E(x) = e^x$.
    
    \begin{itemize}
      \item Prove that $E'(0) = 1$. Hint: write $\frac{H(x)-H(0)}{x} = \frac{1}{x} \sum_{n=1}^{\infty} \frac{x^n}{n!} = \sum_{n=1}^{\infty} \frac{x^{n-1}}{n!}$. Then prove that
      \begin{equation}
        \lim_{x\to 0} \sum_{n=1}^{\infty} \frac{x^{n-1}}{n!} = 1.
      \end{equation}
      
      \item By studying the difference quotient, prove that $E'(x) = E(x)$ for all $x \in \mathbb{R}$. This is much easier than the preceding point.
      
      \item Prove that $\lim_{x\to\infty} E(x) = \infty$ and $\lim_{x\to-\infty} E(x) = 0$.
      
      \item Prove that $E : (-\infty, \infty) \to (0, \infty)$ is 1-1 and onto.
      
      \item Let $L = E^{-1}: (0, \infty) \to (-\infty, \infty)$. Prove that
      \begin{equation}
        L'(t) = \frac{1}{t},
      \end{equation}
      for all $t \in (0, \infty)$.
    \end{itemize}
  \end{exercise}
  \begin{solution}

  \end{solution}

  \begin{exercise}[Math 531 Spring 2025, PS8.6]
    For each $k \in \mathbb{N}$, consider $\sum_{j=1}^N j^k$. It turns out that this can be expressed as a polynomial of degree $k + 1$ in $N$. For example,
    \begin{equation}
      \sum_{j=1}^N j^0 = N,
    \end{equation}
    is a polynomial of degree 1 in $N$. Similarly,
    \begin{equation}
      \sum_{j=1}^N j = \frac{N(N + 1)}{2} = \frac{N^2}{2} + \frac{N}{2},
    \end{equation}
    is a polynomial of degree 2 in $N$. If we write:
    \begin{equation}
      \sum_{j=1}^N j^k = a_0 + a_1N + a_2N^2 + ... + a_{k+1}N^{k+1},
    \end{equation}
    what is the value of $a_{k+1}$? Hint: divide by $N^{k+1}$ and take the limit as $N \to \infty$.
  \end{exercise}
  \begin{solution}

  \end{solution}

  \begin{exercise}[Math 531 Spring 2025, PS9.1]
    Fix $E \subset \mathbb{R}$ and take a sequence of functions $f_n : E \to \mathbb{R}$. Assume that every subsequence of $f_n$ has a subsequence converging uniformly to $f : E \to \mathbb{R}$. Prove that $f_n \to f$ uniformly.
  \end{exercise}
  \begin{solution}

  \end{solution}

  \begin{exercise}[Math 531 Spring 2025, PS9.2]
    In the following, each bullet point is a separate question. Give examples of sequences of functions $f_n : E \to \mathbb{R}$ for which:
    \begin{itemize}
      \item $E = [0, 1]$ and $|f_n(x)| \leq 1$ for all $n \in \mathbb{N}$ and $x \in E$, but $f_n$ has no uniformly convergent subsequence.
      \item $E = [0, 1]$ and all the $f_n$ are differentiable and $|f'_n(x)| \leq 1$ for all $x$ and $n$, but $f_n$ has no uniformly convergence subsequence.
      \item $E = \mathbb{R}$, $|f_n(x)|+|f'_n(x)| \leq 1$ for all $x$ and $n$, but $f_n$ has no uniformly convergent subsequence.
    \end{itemize}
  \end{exercise}
  \begin{solution}

  \end{solution}

  \begin{exercise}[Math 531 Spring 2025, PS9.3]
    Assume we have a twice differentiable function $f : [0, \infty) \to \mathbb{R}$. Assume that $f''(x) > 0$ for all $x \in [0, \infty)$, while $f'(0) \geq 0$. Prove that $\lim_{x\to\infty} f(x) = +\infty$.
  \end{exercise}
  \begin{solution}

  \end{solution}

  \begin{exercise}[Math 531 Spring 2025, PS9.4]
    We proved that the function $E : \mathbb{C} \to \mathbb{C}$ defined by:
    \begin{equation}
      E(z) = \sum_{k=0}^{\infty} \frac{z^k}{k!}
    \end{equation}
    satisfies $E(z + w) = E(z)E(w)$. Let us now investigate the real and imaginary parts of $E(it)$, where $t \in \mathbb{R}$. Let us call the real part $C(t)$ and the imaginary part $S(t)$.
    \begin{itemize}
      \item Prove that $E(\overline{z}) = \overline{E(z)}$, for every $z \in \mathbb{C}$.
      \item Deduce that $|E(it)| = 1$ for all $t \in \mathbb{R}$ and thus:
      \begin{equation}
        C(t)^2 + S(t)^2 = 1,
      \end{equation}
      for all $t \in \mathbb{R}$.
      \item Prove that $C(-t) = C(t)$ for all $t$ and that $S(-t) = -S(t)$ for all $t$.
      \item Prove that $C(0) = 1$ and $S(0) = 0$, while $C'(t) = -S(t)$ and $S'(t) = C(t)$, for all $t \in \mathbb{R}$.
      \item Deduce that $C''(t) = -C(t)$ and prove that there must be some $t > 0$ for which $C(t) = 0$. (Hint: Use Problem 3)
      \item Prove that there is a smallest $t_* > 0$ for which $C(t_*) = 0$.
      \item Define $\pi = 2t_*$ so that $C(\frac{\pi}{2}) = 0$. Since $S$ is increasing on $[0, t_*]$, deduce that $S(\frac{\pi}{2}) = 1$.
      \item Now use the formula $E(z + w) = E(z)E(w)$ to deduce that
      \begin{equation}
        C(t + 2\pi) = C(t), \quad S(t + 2\pi) = S(t),
      \end{equation}
      for all $t \in \mathbb{R}$.
      \item It is now reasonable to unveil that $C$ and $S$ are none other but our old friends: $\cos$ and $\sin$.
    \end{itemize}
  \end{exercise}
  \begin{solution}

  \end{solution}

  \begin{exercise}[Math 531 Spring 2025, PS9.5]
    Let us take $X$ to be the set of continuous functions on $[0, \frac{1}{2}]$. As shown in class, $X$ can be made into a complete metric space with the distance function:
    \begin{equation}
      d(f, g) = \sup_{x\in[0, \frac{1}{2}]} |f(x) - g(x)|
    \end{equation}
    Let us also define $J : X \to X$ by:
    \begin{equation}
      J(f)(x) = 1 + \int_0^x f(t)dt.
    \end{equation}
    \begin{itemize}
      \item Prove that indeed $J : X \to X$.
      \item Prove that for every $f, g \in X$, we have that
      \begin{equation}
        d(J(f), J(g)) \leq \frac{1}{2}d(f, g).
      \end{equation}
      \item Deduce that there is a unique $f_* \in X$ for which $J(f_*) = f_*$, using Homework 6, Problem 6.
      \item Define the sequence $f_n$ by $f_n = J(f_{n-1})$ for $n \geq 1$, while $f_0 \equiv 1$. Find a nice formula for $f_n$. What then is $f_*$? (Look at Homework 6, Problem 6 again).
    \end{itemize}
  \end{exercise}
  \begin{solution}

  \end{solution}

  \begin{exercise}[Math 531 Spring 2025, PS9.6]
    \begin{itemize}
      \item Prove that
      \begin{equation}
        \lim_{n\to\infty} \int_0^1 \sin(nt)dt = 0.
      \end{equation}
      \item Let $f : [0, 1] \to \mathbb{R}$ be continuously differentiable. Prove that
      \begin{equation}
        \lim_{n\to\infty} \int_0^1 f(t) \cos(nt)dt = 0.
      \end{equation}
    \end{itemize}
  \end{exercise}
  \begin{solution}

  \end{solution}

  \begin{exercise}[Math 531 Spring 2025, PS9.7]
    Prove that the curve $\gamma : [0, 1] \to \mathbb{R}^2$ defined by:
    \begin{equation}
      \gamma(t) = (t, t\sin(\frac{1}{t}))
    \end{equation}
    is not rectifiable. Hint: show that if we restrict the curve to $[\epsilon, 1]$, then the resulting curve is rectifiable and its length can be computed readily using the formula:
    \begin{equation}
      \int_\epsilon^1 |\gamma'(t)|dt \geq \int_\epsilon^1 \frac{1}{t}|\cos(\frac{1}{t})|dt - 1.
    \end{equation}
    Next, prove that
    \begin{equation}
      \int_\epsilon^1 \frac{1}{t}|\cos\frac{1}{t}|dt \geq \frac{1}{1000} \log \frac{1}{\epsilon},
    \end{equation}
    for all $\epsilon > 0$ small. Then conclude that $\gamma$ isn't rectifiable.
  \end{exercise}
  \begin{solution}

  \end{solution}


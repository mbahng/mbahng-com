\section{Number Systems} 

\subsection{The Rationals}

  \subsubsection{Field Properties}  

    \begin{definition}[Field]
      A \textbf{field} is an algebraic structure $(\mathbb{F}, +, \cdot)$ where 
      \begin{enumerate}
        \item $\mathbb{F}$ is an abelian group under $+$, with $0$ being the \textit{additive identity}. 
        \item $\mathbb{F} \setminus \{0\}$ is an abelian group under $\cdot$, with $1$ being the \textit{multiplicative identity}. 
        \item It connects the two operations through the \textit{distributive property}.
        \begin{equation}
          x \cdot (y + z) = x \cdot y + x \cdot z
        \end{equation}
      \end{enumerate}
    \end{definition} 

    \begin{lemma}[Left = Right Distributivity]
      Left and right distributivity are equivalent. 
      \begin{equation}
        x \cdot (y + z) = (y + z) \cdot x
      \end{equation}
    \end{lemma} 
    \begin{proof}
      \begin{align}
        x \cdot (y + z) & = x \cdot y + x \cdot z && \tag{Distributive} \\
                        & = y \cdot x + z \cdot x && \tag{Commutative} \\
                        & = (y + z) \cdot x && \tag{Distributive} 
      \end{align}
    \end{proof} 

    \begin{lemma}[Properties of Addition]
      The properties of addition hold in a field. 
      \begin{enumerate}
        \item If $x + y = x + z$, then $y = z$. 
        \item If $x + y = x$, then $y = 0$. 
        \item If $x + y = 0$, then $y = -x$. 
        \item $(-(-x)) = x$. 
      \end{enumerate}
    \end{lemma}
    \begin{proof}
      For the first, we have 
      \begin{align}
        x + y = x + z & \implies -x + (x + y) = -x + (x + z) && \tag{addition is a function} \\
                      & \implies (-x + x) + y = (-x + x) + z && \tag{$+$ is associative} \\
                      & \implies 0 + y = 0 + z && \tag{definition of additive inverse} \\
                      & \implies y = z && \tag{definition of identity}
      \end{align} 
      For the second, we can set $z = 0$ and apply the first property. For the third, we have 
      \begin{align}
        x + y = 0 & \implies -x + (x + y) = -x + 0 && \tag{addition is a function} \\
                  & \implies (-x + x) + y = -x + 0 && \tag{$+$ is associative} \\
                  & \implies 0 + y = -x + 0 && \tag{definition of additive inverse} \\
                  & \implies y = -x && \tag{definition of identity}
      \end{align}
      For the fourth, we simply follow that if $y$ is an inverse of $z$, then $z$ is an inverse of $y$. Therefore, $-x$ being an inverse of $x$ implies that $x$ is an inverse of $-x$. $-(-x)$ must also be an inverse of $-x$. Since inverses are unique\footnote{This is proved in algebra.}, $x = -(-x)$. 
    \end{proof}

    \begin{lemma}[Properties of Multiplication]
      The properties of multiplication hold in a field. 
      \begin{enumerate}
        \item If $x \neq 0$ and $xy = xz$, then $y = z$. 
        \item If $x \neq 0$ and $xy = x$, then $y = 1$. 
        \item If $x \neq 0$ and $xy = 1$, then $y = x^{-1}$. 
        \item If $x \neq 0$, then $(x^{-1})^{-1} = x$. 
      \end{enumerate}
    \end{lemma}
    \begin{proof}
      The proof is almost identical to the first. Since $x \neq 0$, we can always assume that $x^{-1}$ exists. For the first, we have
      \begin{align}
        x y = x z & \implies x^{-1} (x y) = x^{-1} (x z) && \tag{multiplication is a function} \\
                  & \implies (x^{-1} x) y = (x^{-1} x) z && \tag{$\times$ is associative} \\
                  & \implies 1 y = 1 z && \tag{definition of multiplicative inverse} \\  
                  & \implies y = z && \tag{definition of identity}
      \end{align}
      For the second, we can set $z = 1$ and apply the first property. For the third, we have 
      \begin{align}
        xy = 1 & \implies x^{-1} (x y) = x^{-1} 1 && \tag{multiplication is a function} \\
               & \implies (x^{-1} x) y = x^{-1} 1 && \tag{$\times$ is associative} \\
               & \implies 1 y = x^{-1} 1 && \tag{definition of multiplicative inverse} \\
               & \implies y = x^{-1} && \tag{definition of identity}
      \end{align}
      For the fourth, we simply see that $x^{-1}$ is a multiplicative inverse of both $x$ and $(x^{-1})^{-1}$ in the group $(\mathbb{F} \setminus \{0\}, \times)$, and since inverses are unique, they must be equal. 
    \end{proof}

    \begin{lemma}[Properties of Distribution]
      For any $x, y, z \in \mathbb{F}$, the field axioms satisfy 
      \begin{enumerate}
        \item $0 \cdot x = 0$.
        \item If $x \neq 0$ and $y \neq 0$, then $x y \neq 0$.
        \item $-1 \cdot x = -x$. 
        \item $(-x) y = - (xy) = x (-y)$. 
        \item $(-x) (-y) = xy$. 
      \end{enumerate}
    \end{lemma} 
    \begin{proof}
      For the first, note that 
      \begin{align}
        0 x & = (0 + 0) \cdot x = 0 x + 0x 
      \end{align}
      and subtracting $0x$ from both sides gives $0 = 0x$. For the second, we can claim that $xy \neq 0$ equivalently claiming that it will have an identity. Since $x, y \neq 0$, their inverses exists, and we claim that $(xy)^{-1} = y^{-1} x^{-1}$ is an inverse. We can see that by associativity, 
      \begin{equation}
        (y^{-1} x^{-1}) (xy) = y^{-1} (x^{-1} x) y = y^{-1} y = 1
      \end{equation} 
      For the third, we see that 
      \begin{equation}
        0 = 0 \cdot x = (1 + (-1)) \cdot x = 1 \cdot x + (-1) \cdot x = x + (-1) \cdot x 
      \end{equation}
      which implies that $-1 \cdot x$ is the additive inverse. The fourth follows immediately from the third by the associative property. For the fifth we can see that 
      \begin{align}
        (-x) (-y) & = (-1) x (-1) y && \tag{property 3} \\
                  & = (-1) (-1) x y && \tag{$\times$ is commutative} \\
                  & = -1 \cdot (-xy) && \tag{property 3} \\
                  & = -(-xy) && \tag{property 3} \\
                  & = xy && \tag{addition property 4}
      \end{align}
    \end{proof}

    Now that we've reviewed some fields, let's construct $\mathbb{Q}$ from $\mathbb{Z}$ and verify it's a field. 

    \begin{definition}[Rationals]
      Given the ordered ring of integers $(\mathbb{Z}, +_{\mathbb{Z}}, \times_{\mathbb{Z}}, \leq_{\mathbb{Z}})$ the \textbf{rational numbers} $(\mathbb{Q}, +_{\mathbb{Q}}, \times_{\mathbb{Q}})$ are defined as such. 
      \begin{enumerate}
        \item $\mathbb{Q}$ is the quotient space on $\mathbb{Z} \times \mathbb{Z} \setminus \{0\}$ with the equivalence relation $\sim$ 
        \begin{equation}
          (a, b) \sim (c, d) \iff a \times_{\mathbb{Z}} d = b \times_{\mathbb{Z}} c
        \end{equation} 
        We denote this class as $(a, b)$, where $b > 0$, since if $b < 0$, we know that $(-a, -b)$ are also in this order. 

        \item The additive and multiplicative identities are 
        \begin{equation}
          0_{\mathbb{Q}} \coloneqq (0_{\mathbb{Z}}, a), \;\;\; 1_{\mathbb{Q}} \coloneqq (a, a)
        \end{equation}

        \item Addition on $\mathbb{Q}$ is defined 
        \begin{equation}
          (a, b) +_{\mathbb{Q}} (c, d) \coloneqq \big( (a \times_{\mathbb{Z}} d) +_{\mathbb{Z}} (b \times_{\mathbb{Z}} c), b \times_{\mathbb{Z}} d \big) 
        \end{equation}

        \item The additive inverse is defined 
        \begin{equation}
          -(a, b) \coloneqq (-a, b)
        \end{equation}

        \item Multiplication on $\mathbb{Q}$ is defined 
        \begin{equation}
          (a, b) \times_{\mathbb{Q}} (c, d) \coloneqq \big( a \times_{\mathbb{Z}} c, b \times_{\mathbb{Z}} d \big)
        \end{equation} 

        \item The multiplicative inverse is defined 
        \begin{equation}
          (a, b)^{-1} \coloneqq (b, a)
        \end{equation}
      \end{enumerate}
    \end{definition}

    \begin{theorem}[Rationals are a Field]
      $\mathbb{Q}$ is a field. 
    \end{theorem} 
    \begin{proof}
      We do a few things. 
      \begin{enumerate}
        \item Verify the additive identity. 
        \begin{equation}
          (a, b) + (0, c) = (ac + 0b, bc) = (ac, bc) \sim (a, b)
        \end{equation}
        \item Verify the multiplicative identity. 
        \begin{equation}
          (a, b) \times (c, c) = (ac, bc) \sim (a, b)
        \end{equation}
        \item Additive inverse is actually an inverse. 
        \begin{equation}
          (a, b) + (-a, b) = (ab + (-ba), bb) = (0, bb) \sim (0, 1)
        \end{equation}
        \item Multiplicative inverse is actually an inverse. 
        \begin{equation}
          (a, b) \times (b, a) = (ab, ba) = (ab, ab) \sim (1, 1)
        \end{equation}
        \item Addition is commutative. 
        \begin{equation}
          (a, b) + (c, d) = (ad + bc, bd) = (cb + ad, bd) = (c, d) + (a, b)
        \end{equation}
        \item Addition is associative. 
        \begin{align}
          (a, b) + ((c, d) + (e, f)) & = (a, b) + (cf + de, df) \\
                                     & = (adf + bcf + bde, bdf) \\
                                     & = (ad + bc, bd) + (e, f) \\
                                     & = ((a, b) + (c, d)) + (e, f)
        \end{align}
        \item Multiplication is commutative. 
        \begin{equation}
          (a, b) \times (c, d) = (ac, bd) = (ca, db) = (c, d) \times (a, b)
        \end{equation}
        \item Multiplication is associative. 
        \begin{align}
          (a, b) \times ((c, d) \times (e, f)) & = (a, b) \times (ce, df) \\ 
                                               & = (ace, bdf) \\
                                               & = (ac, bd) \times (e, f) \\
                                               & = ((a, b) \times (c, d)) \times (e, f)
        \end{align}
        \item Multiplication distributes over addition. 
          \begin{align}
            (a, b) \times ((c, d) + (e, f)) & = (a, b) \times (c, d) + (a, b) \times (e, f) \\
                                            & = (ac, bd) + (ae, bf) \\
                                            & = (abcf + abde, b^2 df) \\
                                            & = (acf + ade, bdf)  
                                            & = (a, b) \times (cf + de, df)
          \end{align}
      \end{enumerate}
    \end{proof} 

    We have successfully defined the rationals, but now these are almost completely separate elements. We know that all integers are rational numbers, and so to show that the rationals are an extension of $\mathbb{Z}$ we want to identify a \textit{canonical injection} $\iota: \mathbb{Z} \rightarrow \mathbb{Q}$. This can't just be any canonical injection; it must preserve the algebraic structure between the two sets and must therefore be a \textit{ring homomorphism}. 

    \begin{theorem}[Canonical Injection of $\mathbb{Z}$ to $\mathbb{Q}$ is a Ring Homomorphism]
      Let us define the canonical injection $\iota: \mathbb{Z} \rightarrow \mathbb{Q}$ to be $\iota(a) = (a, 1)$. This is a ring homomorphism. 
    \end{theorem}
    \begin{proof} 
      We show a few things. 
      \begin{enumerate}
        \item Preservation of addition. 
          \begin{align}
            \iota(a) +_{\mathbb{Q}} \iota(b) & = (a, 1) +_{\mathbb{Q}} (b, 1) \\
                                             & = (1a +_{\mathbb{Z}} 1b, 1^2) \\
                                             & = (a +_{\mathbb{Z}} b, 1) \\
                                             & = \iota(a +_{\mathbb{Z}} b) 
          \end{align}
        \item Preservation of multiplication. 
          \begin{align}
            \iota(a) \times_{\mathbb{Q}} \iota(b) & = (a, 1) \times_{\mathbb{Q}} (b, 1) \\
                                                  & = (a \times_{\mathbb{Z}} b, 1^2) \\
                                                  & = (a \times_{\mathbb{Z}} b, 1) \\
                                                  & = \iota(a \times_{\mathbb{Z}} b, 1)
          \end{align}
        \item Preservation of multiplicative identity. 
          \begin{equation}
            \iota(1_{\mathbb{Z}}) = (1, 1) = 1_{\mathbb{Q}}
          \end{equation}
      \end{enumerate}
    \end{proof} 

  \subsubsection{Ordered Field Properties} 

    Great, so we have established that $\mathbb{Q}$ is a field. The next property we want to formalize is order. 

    \begin{definition}[Partial, Total/Linear Order]
      A \textbf{partial order} on a set $X$ is a relation $\leq$ satisfying. 
      \begin{enumerate}
        \item Reflexive: $x \leq x$ 
        \item Antisymmetric: $x \leq y, y \leq x \implies x = y$
        \item Transitivity: $x \leq y, y \leq z \implies x \leq z$
      \end{enumerate}
      Note that when we say $x \leq y$, this means "$x$ is related to $y$" (but does not necessarily mean that $y$ is related to $x$), or "$x$ is less than or equal to $y$." A set $X$ with a partial order is called a partially ordered set. 

      Additionally, given elements $x, y$ of partially order set $X$, if either $x \leq y$ or $y \leq x$, then $x$ and $y$ are \textbf{comparable}. Otherwise, they are \textbf{incomparable}. A partial order in which every pair of elements is comparable is called a \textbf{total order}, or \textbf{linear order}. Note that from this $\leq$ relation, we can similarly define 
      \begin{enumerate}
        \item $\leq$: less than or equal to 
        \item $\geq$: greater than or equal to 
        \item $<$: strictly less than ($x < y$ iff $x\leq y, x \neq y$)
        \item $>$: strictly greater than ($x > y$ iff $x \geq y, x \neq y$)
      \end{enumerate}
    \end{definition} 

    \begin{example}[Partially Ordered Sets]
      We list some examples of partially ordered sets. 
      \begin{enumerate}
        \item The real numbers ordered by the standard "less-than-or-equal" relation $\leq$ (totally ordered set as well). 
        \item The set of subsets of a given set $X$ ordered by inclusion. That is, the power set $2^X$ with the partial order $\subseteq$ is partially ordered. 
        \item The set of natural numbers equipped with the relation of divisibility. 
        \item The set of subspaces of a vector space ordered by inclusion. 
        \item For a partially ordered set $P$, the sequence space containing all sequences of elements from $P$, where sequence $a$ precedes sequence $b$ if every item in $a$ precedes the corresponding item in $b$. 
      \end{enumerate}
    \end{example} 

    We now want to define the natural ordering of the rationals. There are countless ways to do it, but I just take the difference and claim that it is greater than $0$. 

    \begin{theorem}[Order on Rationals]
      The order $\leq_{\mathbb{Q}}$ defined on the rationals as 
      \begin{equation}
        (a, b) \leq_{\mathbb{Q}} (c, d) \iff ad \leq_{\mathbb{Z}} bc
      \end{equation}
      is a total order. Remember that we have defined $b, d > 0$. 
    \end{theorem}
    \begin{proof}
      We prove the three properties. 
      \begin{enumerate}
        \item Reflexive. 
        \begin{equation}
          (a, b) \leq_{\mathbb{Q}} (a, b) \iff ab \leq_{\mathbb{Z}} ab
        \end{equation} 

        \item Antisymmetric. 
        \begin{align}
          (a, b) \leq_{\mathbb{Q}} (c, d) & \implies ad \leq_{\mathbb{Z}} bc
          (c, d) \leq_{\mathbb{Q}} (a, b) & \implies bc \leq_{\mathbb{Z}} ad
        \end{align} 
        This implies that both $ad = bc$, which by definition means that they are in the same equivalence class. 

        \item Transitivity. Assume that $(a, b) \leq (c, d)$ and $(c, d) \leq (e, f)$. Then, we notice that $b, d, f > 0$ and therefore by the ordered ring property\footnote{If $a \leq b$ and $0 \leq c$, then $ac \leq bc$.} of $\mathbb{Z}$, we have 
        \begin{align}
          (a, b) \leq_{\mathbb{Q}} (c, d) & \implies ad \leq_{\mathbb{Z}} bc \implies adf \leq_{\mathbb{Z}} bcf \\ 
          (c, d) \leq_{\mathbb{Q}} (e, f) & \implies cf \leq_{\mathbb{Z}} de \implies bcf \leq_{\mathbb{Z}} bde
        \end{align}
        Therefore from transitivity of the ordering on $\mathbb{Z}$ we have $adf \leq bde$. By the ordered ring property\footnote{If $a \leq b$, then $a + c \leq b + c$.}  we have $0 \leq bde - adf = d(be - af)$. But notice that $d > 0$ from our definition of rationals, and therefore it must be the case that $0 \leq be - af \implies af \leq_{\mathbb{Z}} be$, which by definition means $(a, b) \leq_{\mathbb{Q}} (e, f)$. 
      \end{enumerate}
    \end{proof} 

    As soon as we define an order the concept of extrema and bounds are well defined. Let's define them too. 

    \begin{definition}[Extrema, Bounds]
      Given a set $X$, 
      \begin{enumerate}
        \item $x \in X$ is a \textbf{maximum} $X$ if $y \leq x$ for all $y \in X$. 
        \item $x \in X$ is a \textbf{minimum} $X$ if $x \leq y$ for all $y \in X$. 
      \end{enumerate}
      Given a totally ordered set $X$ and some subset $S \subset X$. 
      \begin{enumerate}
        \item $x \in X$ is an \textbf{upper bound} of $S$ if $x \geq y$ for all $y \in S$. 
        \item $x \in X$ is a \textbf{lower bound} of $S$ if $x \leq y$ for all $y \in S$. 
        \item $x \in X$ is a \textbf{supremum}, or \textbf{least upper bound}, of $S$ if $x$ is the minimum of the set of all upper bounds of $S$. 
        \item $x \in X$ is a \textbf{infimum}, or \textbf{greatest lower bound}, of $S$ if $x$ is the maximum of the set of all lower bounds of $S$. 
      \end{enumerate}
      Note that we have defined max/min separately from the concept of bounds. You can define the maximum of a set with just knowing the set, but the bounds require \textit{both} some subset $S$ with respect to an enclosing set $X$.\footnote{For example, it makes sense to define the maximum of a set $S = [0, 1]$ by itself, but not an upper bound for it. If $X = \mathbb{Q}$, then the supremum is $1$, but if $X$ was the set of all irrationals, then this has no supremum.} Intuitively, the main difference between the supremum/infimum and maximum/minimum is that the supremum/infimum accounts for limit points of the subset $S$. 
    \end{definition}

    Note that given a set, we can really put whatever order we want on it. However, consider the field with the following order. 
    \begin{equation}
      \mathbb{F} = \{0, 1\}, \; 0 < 1
    \end{equation} 
    This does not behave well with respect to its operations because for example if we have $0 < 1$, then adding the same element to both sides should preserve the ordering. But this is not the case since $0 + 1 = 1 > 1 + 1 = 0$. While it may be easy to define an order, we would like it to be an ordered field. 

    \begin{definition}[Ordered Field]
      An \textbf{ordered field} is a field that has an order satisfying 
      \begin{enumerate}
        \item $y < z \implies x + y < x + z$ for all $x \in \mathbb{F}$. 
        \item $x > 0, y > 0 \implies xy > 0$. 
      \end{enumerate}
    \end{definition}

    \begin{theorem}[Properties]
      In an totally ordered field, 
      \begin{enumerate}
        \item $x > 0 \implies -x < 0$. 
        \item $x \neq 0 \implies x^2 > 0$. 
        \item If $x > 0$, then $y < z \implies xy < xz$. 
      \end{enumerate}
    \end{theorem} 
    \begin{proof}
      The first property is a single-liner 
      \begin{equation}
        0 < x \implies 0 + -x < x + -x \implies -x < 0 
      \end{equation}
      For the second property, it must be the case that $x > 0$ or $x < 0$. If $x > 0$, then by definition $x^2 > 0$. If $x < 0$, then 
      \begin{equation}
        x^2 = 1 \cdot x^2 = (-1)^2 \cdot x^2 = (-1 \cdot x)^2 = (-x)^2
      \end{equation}
      and since $-x > 0$ from the first property, we have $x^2 = (-x)^2 > 0$. For the third, we use the distributive property. 
      \begin{align}
        y < z & \implies 0 < z - y \\ 
              & \implies 0 = x 0 < x(z - y) = xz - xy \\
              & \implies xy < xz
      \end{align}
    \end{proof}

    As we have hinted, the rationals is an ordered field. 

    \begin{theorem}[Rationals are an Ordered Field]
      $\mathbb{Q}$ is an ordered field. 
    \end{theorem} 
    \begin{proof}
      We show that our defined order satisfies the definition. 
      \begin{enumerate}
        \item Assume that $y = (a, b) \leq (c, d) = z$. Let $x = (e, f)$. Then $x + y = (af + be, bf)$, $x + z = (cf + de, df)$. Therefore 
        \begin{align}
          (af + be) df & = adf^2 + bedf \\ 
                       & \leq bcf^2 + bedf \\
                       & = (cf + de) bf
        \end{align} 
        But $(af + be) df = (cf + de) bf$ is equivalent to saying $(af + be, bf) \leq_{\mathbb{Q}} (cf + de, df)$, i.e. $x + y \leq x + z$!  

        \item Let $x = (a, b), y = (c, d)$. Since $0 < x, 0 < y$, by construction this means that $0 < a, 0 < c$ (since $b, d > 0$ in the canonical rational form). By the ordered ring property of the integers, $0 < ac$. So 
        \begin{equation}
          0 < ac \iff 0 \cdot bd < ac \cdot 1 \iff (0, 1) < (ac, bd)  \iff 0_{\mathbb{Q}} < (a, c) \times_{\mathbb{Q}} (b, d) = x y
        \end{equation}
      \end{enumerate}
    \end{proof} 

    Not only is it an ordered field, but it also is consistent with the ordering on $\mathbb{Z}$! It's nice how all these properties seem to fit together. 

    \begin{theorem}[Preservation of Order]
      The canonical injection $\iota$ is an \textit{order homomorphism}. That is, for $a, b \in \mathbb{Z}$, 
      \begin{equation}
        a \leq_{\mathbb{Z}} b \iff \iota(a) \leq_{\mathbb{Q}} \iota(b)
      \end{equation}
    \end{theorem}
    \begin{proof} 
      \begin{align}
        a \leq_{\mathbb{Z}} b & \iff a \cdot 1 \leq_{\mathbb{Z}} b \cdot 1 \\
                              & \iff (a, 1) \leq_{\mathbb{Q}} (b, 1) \\
                              & \iff \iota(a) \leq_{\mathbb{Q}} \iota(b)
      \end{align}
    \end{proof}

    Note that an order can be used to generate an order topology, which we will define below. 

    \begin{definition}[Order Topology on $\mathbb{Q}$]
      The order topology on $\mathbb{Q}$ is the topology generated by the set $\mathscr{B}$ of all open intervals 
      \begin{equation}
        (a, b) \coloneqq \{ x \in \mathbb{Q} \mid a < x < b\}
      \end{equation}
    \end{definition}

    \begin{theorem}[Finite Fields]
      There are no finite ordered fields. 
    \end{theorem} 
    \begin{proof}
      Assume $\mathbb{F}$ is such an ordered field. It must be the case that $0, 1 \in \mathbb{F}$, with $0 < 1$. Therefore, we also have $0 + 1 < 1 + 1 \implies 1 < 1 + 1$. Repeating this we get 
      \begin{equation}
        0 < 1 < 1 + 1 < 1 + 1 + 1 < \ldots
      \end{equation}
      where these elements must be distinct (since only one of $>, <, =$ must be true for a totally ordered set). Since this can be done for a countably infinite number of times, $\mathbb{F}$ cannot be finite. 
    \end{proof}

  \subsubsection{Norm} 

    Note that we can also define a norm on the rationals with just the order and algebraic properties. 

    \begin{theorem}[Norm on $\mathbb{Q}$] 
      The following is indeed a norm on $\mathbb{Q}$. 
      \begin{equation}
        |x| \coloneqq \begin{cases} x & \text{ if } x \geq 0 \\ -x & \text{ if } x < 0 \end{cases}
      \end{equation} 
    \end{theorem} 

    It is well known that the metric induced by any norm is indeed a metric. Therefore we state the metric as a definition. 

    \begin{definition}[Metric on $\mathbb{Q}$]
      The Euclidean metric on $\mathbb{Q}$ is defined 
      \begin{equation}
        d(x, y) \coloneqq |x - y| = \begin{cases} x - y & \text{ if } x \geq y \\ y - x & \text{ if } x < y \end{cases}
      \end{equation}
    \end{definition}

    Thus we get to what we want: the induced topology of open balls. 
    Again, since we know from point-set topology that metric topologies are indeed topologies, we will state this as a definition rather than a theorem.  

    \begin{definition}[Open-Ball Topology on $\mathbb{Q}$]
      The Euclidean topology on $\mathbb{Q}$ is the topology generated by the set $\mathscr{B}$ of all open balls
      \begin{equation}
        B(x, r) \coloneqq \{ y \in \mathbb{Q} \mid |x - y| < r \}
      \end{equation} 
    \end{definition}

    Note that this is the same topology as the order topology. This should however be proved. 

    \begin{theorem}[Metric and Order Topologies on $\mathbb{Q}$]
      The metric and order topologies on $\mathbb{Q}$ are the same topologies. 
    \end{theorem}
    \begin{proof}
      
    \end{proof}

\subsection{The Reals}

    By constructing the topology of $\mathbb{Q}$ earlier, we can talk about convergence. The first question to ask (if you were the first person inventing the reals) is ``how do I know that there exists some other numbers at all?'' The first clue is trying to find the side length of a square with area $2$. As we see, this number is not rational. 

    \begin{theorem}[$\sqrt{2}$ is Not Rational]
      There exists no $x \in \mathbb{Q}$ s.t. $x^2 = 2$. 
    \end{theorem}
    \begin{proof}
      
    \end{proof} 

    We can ``imagine'' that a square with area $2$ certainly exists, but the fact that its side length is undefined is certainly unsettling. I don't know about you, but I would try to ``invent'' $\sqrt{2}$. We can maybe do this in multiple ways. 
    \begin{enumerate}
      \item I write out the decimal expansion one by one, which gives our first exposure to sequences. 
      \begin{equation}
        1, 1.4, 1.41, 1.414, \ldots
      \end{equation} 
      It is clear that on $\mathbb{Q}$, this sequence does not converge. Our intuition tells that that if the terms get closer and closer to each other, they must be getting closer and closer to \textit{something}, though that something is not in $\mathbb{Q}$. This motivates the definition for \textit{Cauchy completeness}. 

      \item I would write out maybe some nested intervals so that $\sqrt{2}$ \textit{must}  lie within each interval. 
      \begin{equation}
        [1, 2] \supset [1.4, 1.5] \supset [1.41, 1.42] \supset \ldots 
      \end{equation}
      This motivates the definition of \textit{nested-interval completeness}. 

      \item I would define the set of all rationals such that $x^2 < 2$, and try to define $\sqrt{2}$ as the max or supremum of this set. We will quickly find that neither the max nor the supremum exists in $\mathbb{Q}$, and this motivates the definition for \textit{Dedekind completeness}. 
    \end{enumerate}

    All three of these methods points at the same intuition that there should not be any "gaps" or "missing points" in the set that we will construct to be $\mathbb{R}$. This contrasts with the rational numbers, whose corresponding number line has a "gap" at each irrational value. 

  \subsubsection{Dedekind Completeness} 

    \begin{definition}[Dedekind Cut] 
      A \textbf{Dedekind cut} is a partition of the rationals $\mathbb{Q} = A \sqcup A^\prime$ satisfying the three properties.\footnote{This can really be defined for any totally ordered set. } 
      \begin{enumerate}
        \item $A \neq \emptyset$ and $A \neq \mathbb{Q}$.\footnote{By relaxing this property, we can actually complete $\mathbb{Q}$ to the extended real number line. }
        \item $x < y$ for all $x \in A, y \in A^\prime$. 
        \item The maximum element of $A$ does not exist in $\mathbb{Q}$. 
      \end{enumerate}
      The minimum of $A^\prime$ may exist in $\mathbb{Q}$, and if it does, the cut is said to be \textbf{generated} by $\min A^\prime$. 
    \end{definition}

    Note that in $\mathbb{Q}$, there will be two types of cuts: 
    \begin{enumerate}
      \item ones that are generated by rational numbers, such as 
      \begin{equation}
        A = \{x \in \mathbb{Q} \mid x < 2/3 \}, A^\prime = \{ x \in \mathbb{Q} \mid x \geq 2/3 \} 
      \end{equation}
      \item and the ones that are not 
      \begin{equation}
        A = \{x \in \mathbb{Q} \mid x^2 < 2 \}, A^\prime = \{x \in \mathbb{Q} \mid x^2 \geq 2 \}
      \end{equation}
    \end{enumerate}
    We can intuitively see that the set of all Dedekind cuts $(A, A^\prime)$ will ``extend'' the rationals into a bigger set. We can then define some operations and an order to construct this into an ordered field, and finally it will have the property that we call ``completeness.''

    \begin{definition}[Dedekind Completeness]
      A totally ordered algebraic field $\mathbb{F}$ is \textbf{complete} if every Dedekind cut of $\mathbb{F}$ is generated by an element of $\mathbb{F}$. 
    \end{definition} 

    \begin{theorem}
      $\mathbb{Q}$ is not Dedekind-complete. 
    \end{theorem}
    \begin{proof}
      The counter-example is given above for the cut 
      \begin{equation}
        A = \{x \in \mathbb{Q} \mid x^2 < 2 \}, A^\prime = \{x \in \mathbb{Q} \mid x^2 \geq 2 \}
      \end{equation}
    \end{proof} 

    Now we have the tools to define the reals, giving us the beefy theorem. 

    \begin{theorem}[Reals as the Dedekind-Completion of Rationals]
      Let $\mathbb{R}$ be the set of all Dedekind cuts $(A, A^\prime)$ of $\mathbb{Q}$ of $\mathbb{Q}$. For convenience we can uniquely represent $(A, A^\prime)$ with just $A$ since $A^\prime = \mathbb{Q} \setminus A$. By doing this we can intuitively think of a real number as being represented by the set of all smaller rational numbers. Let $A, B$ be two Dedekind cuts. Then, we define the following order and operations. 
      \begin{enumerate}
        \item \textit{Order}. $A \leq_{\mathbb{R}} B \iff A \subset B$. 
        \item \textit{Addition}. $A +_{\mathbb{R}} B \coloneqq \{ a +_{\mathbb{Q}} b \mid a \in A, b \in B \}$. 
        \item \textit{Additive Identity}. $0_{\mathbb{R}} \coloneqq \{x \in \mathbb{Q} \mid x < 0 \}$. 
        \item \textit{Additive Inverse}. $-B \coloneqq \{ a - b \mid a < 0 , b \in (\mathbb{Q} \setminus B) \}$.
        \item \textit{Multiplication}. If $A, B \geq 0$, then we define $A \times_{\mathbb{R}} B \coloneqq \{ a \times_{\mathbb{Q}} b \mid a \in A, b \in B, a, b \geq 0\} \cup 0_{\mathbb{R}}$. If $A$ or $B$ is negative, then we use the identity $A \times B = -(A \times_{\mathbb{R}} -B) = -(-A \times_{\mathbb{R}} B) = (-A \times_{\mathbb{R}} -B)$ to convert $A, B$ to both positives and apply the previous definition. 
        \item \textit{Multiplicative Identity}. $1_{\mathbb{R}} = \{x \in \mathbb{Q} \mid x < 1 \}$. 
        \item \textit{Multiplicative Inverse}. If $B > 0$, $B^{-1} \coloneqq \{ a \times_{\mathbb{Q}} b^{-1} \mid a \in 1_{\mathbb{R}}, b \in (\mathbb{Q} \setminus B) \}$. If $B$ is negative, then we compute $B^{-1} = -((-B)^{-1})$ by first converting to a positive number and then applying the definition above. 
      \end{enumerate}
      We claim that $(\mathbb{R}, +_{\mathbb{R}}, \times_{\mathbb{R}}, \leq_{\mathbb{R}})$ is a totally ordered field, and the canonical injection $\iota: \mathbb{Q} \rightarrow \mathbb{R}$ defined 
      \begin{equation}
        \iota(q) = \{x \in \mathbb{Q} \mid x < q \}
      \end{equation}
      is an ordered field isomorphism. Finally, by construction $\mathbb{R}$ is Dedekind-complete. 
    \end{theorem} 

    \begin{definition}[Least Upper Bound Property]
      A totally ordered algebraic field $\mathbb{F}$ (must it be a field?) is complete if every nonempty set of $F$ having an upper bound must have a least upper bound (supremum) in $F$. 
    \end{definition} 

    \begin{theorem}[LUB is Equivalent to GLB]
      A set $(X, \leq)$ has the least upper bound property iff it has the greatest lower bound property.\footnote{Every set bounded below has a greatest lower bound.} 
    \end{theorem}
    \begin{proof}
      We will prove one direction since the other is the same logic. Let $S \subset X$ be a nonempty set that is bounded below by some $l \in X$. Let $L \subset X$ be the set of all lower bounds of $S$. Since $l$ exists, it is nonempty. Furthermore, $L$ is bounded above by any element of $S$. Due to LUB property $L$ has a least upper bound, call it $z = \sup{L}$. We claim that $z = \inf{S}$. 
      \begin{enumerate}
        \item $z$ is a lower bound of $S$. Assume that it is not. Then there exists $s \in S$ s.t. $s < z$. But by construction $s$ is an upper bound for $L$ and so $z$ s not the \textit{least} upper bound, a contradiction. 
        \item $z$ is a \textit{greatest} lower bound. Assume that $z$ is not. Then there exists a $z^\prime \in X$ s.t. $z < z^\prime \leq s$ for all $s \in S$. But since $z^\prime, z$ are lower bounds, this means $z, z^\prime \in L$ by definition and $z < z^\prime$ contradicts the fact that $z$ is an upper bound of $L$. 
      \end{enumerate}
      We are done. 
    \end{proof}

    \begin{theorem}[Dedekind Completeness Equals Least-Upper-Bound Property]
      Dedekind completeness is equivalent to the least upper bound property. 
    \end{theorem}
    \begin{proof}
      
    \end{proof}

    \begin{definition}[Archimidean Principle]
      An ordered ring $(X, +, \cdot, \leq)$ that embeds the naturals $\mathbb{N}$\footnote{as in, there exists an ordered ring homomorphism $\iota: \mathbb{N} \rightarrow X$} is said to obey the \textbf{Archimedean principle} if given any $x, y \in X$ s.t. $x, y > 0$, there exists an $n \in \mathbb{N}$ s.t. $\iota(n) \cdot x > y$. Usually, we don't care about the canonical injection and write $nx > y$. 
    \end{definition}

    By the canonical injections $\mathbb{N} \rightarrow \mathbb{Z} \rightarrow \mathbb{Q} \rightarrow \mathbb{R}$, we can talk about whether this set has the Archimedean property. In fact Dedekind completeness does imply it. 

    \begin{theorem}
      $\mathbb{R}$ satisfies the Archimedean principle. 
    \end{theorem}
    \begin{proof}
      Assume that this property doesn't hold. Then for any fixed $x$, $nx < y$ for all $n \in \mathbb{N}$. Consider the set 
      \begin{equation}
        A = \bigcup_{n \in \mathbb{N}} (-\infty, nx), \;\;\; B = \mathbb{R} \setminus A
      \end{equation}
      $A$ by definition is nonempty, and $B$ is nonempty since it contains $y$. Then, we can show that $a \in A, b \in B \implies a < b$ using proof by contradiction. Assume that there exists $a^\prime \in A, b^\prime \in B$ s.t. $a^\prime > b^\prime$. Since $a^\prime \in A$, there exists a $n^\prime \in \mathbb{N}$ s.t. $a^\prime \in (-\infty, n^\prime x) \iff a^\prime < n^\prime x$. But by transitivity of order, this means $b^\prime < n^\prime x \iff b^\prime \in (-\infty, n^\prime x) \implies b^\prime \in A$. 

      Going back to the main proof, we see that $A$ is upper bounded by $y$, and so by the least upper bound property it has a supremum $z = \sup{A}$. 
      \begin{enumerate}
        \item If $z \in A$, then by the induction principle\footnote{Note that $\mathbb{N}$ is defined recursively as $1 \in \mathbb{N}$ and if $n \in \mathbb{N}$, then $n+1 \in \mathbb{N}$. } $z + x \in A$, contradicting that $z$ is an upper bound. 
        \item If $z \not\in A$, then by the induction principle\footnote{The contrapositive of the recursive definition of $\mathbb{N}$ is: if $n \not\in \mathbb{N}$, then $n-1 \not\in \mathbb{N}$.} $z-x \not\in A \implies z-x \in B$. Since every element of $B$ upper bounds $A$ and since $x > 0$, this means that $z-x < z$ is a smaller upper bound of $A$, contradicting that $z$ is a least upper bound. 
      \end{enumerate}
      Therefore, it must be the case that $nx > y$ for some $n \in \mathbb{N}$. 
    \end{proof}

  \subsubsection{Cauchy Completeness} 

    \begin{definition}[Cauchy Sequence]
      A sequence $a_n$ in a metric space $(X, d)$ is a \textbf{Cauchy sequence} if for every $\epsilon > 0$, there exists an $N$ s.t. 
      \begin{equation}
        d(a_i, a_j) < \epsilon
      \end{equation}
      for every $i, j > N$. We call this \textbf{Cauchy convergence}. 
    \end{definition}

    Note that it is not sufficient to say that a sequence is Cauchy by claiming that each term becomes arbitrarily close to the preceding term. That is, 
    \begin{equation}
      \lim_{n \rightarrow \infty} |x_{n+1} - x_{n}| = 0
    \end{equation}
    For example, look at the sequence 
    \begin{equation}
      a_n = \sqrt{n} \implies a_{n+1} - a_{n} = \frac{1}{\sqrt{n+1} + \sqrt{n}} < \frac{1}{2\sqrt{n}}
    \end{equation}
    However, it is clear that $a_n$ gets arbitrarily large, meaning that a finite interval can contain at most a finite number of terms in $\{a_n\}$. 

    It is trivial that convergence implies Cauchy convergence, but the other direction is not true. Therefore, we would like to work in a space where these two are equivalent, and this is called completeness. 

    \begin{definition}[Cauchy Completeness]
      A metric space $(X, d)$ is complete if every Cauchy sequence in that space converges to an element in $X$. 
    \end{definition} 

    \begin{theorem}
      $\mathbb{Q}$ is not Cauchy-complete. 
    \end{theorem}
    \begin{proof}
      Let $a_n$ be the largest number $x$ up to the $n$th decimal expansion such that $x^2$ does not exceed $2$. The first few terms are 
      \begin{equation}
        1.4, 1.41, 1.414, \ldots
      \end{equation}
    \end{proof}

    Therefore, we can construct the reals as equivalence classes over Cauchy sequences. Rather than using the order, we take advantage of the metric. 

    \begin{theorem}[Reals as the Cauchy-Completion of the Rationals]
      Let $\mathbb{R}$ be the quotient space of all Cauchy sequences $(x_n)$ of rational numbers with the equivalence relation $(x_n) = (y_n)$ iff their difference tends to $0$.\footnote{This equivalence class reflects that the same real number can be approximated in many different sequences. In fact, this shows \textit{by definition} that $1, 1, \ldots$ and $0.9, 0.99, 0.999, \ldots$ are the same number!} That is, for every rational $\epsilon > 0$, there exists an integer $N$ s.t. for all naturals $n > N$, $|x_n - y_n| < \epsilon$. 
      \begin{enumerate}
        \item \textit{Order}. $(x_n) \leq_{\mathbb{R}} (y_n)$ iff $x = y$ or there exists $N \in \mathbb{N}$ s.t. $x_n \leq_{\mathbb{Q}} y_n$ for all $n > N$. 
        \item \textit{Addition}. $(x_n) + (y_n) \coloneqq (x_n + y_n)$. 
        \item \textit{Additive Identity}. $0_{\mathbb{R}} \coloneqq (0_{\mathbb{Q}})$. 
        \item \textit{Additive Inverse}. $-(x_n) \coloneqq (-x_n)$. 
        \item \textit{Multiplication}. $(x_n) \times_{\mathbb{R}} (y_n) = (x_n \times_{\mathbb{Q}} y_n)$. 
        \item \textit{Multiplicative Identity}. $1_{\mathbb{R}} \coloneqq (1)$. 
        \item \textit{Multiplicative Inverse}. $(x_n)^{-1} \coloneqq (x_n^{-1})$. 
      \end{enumerate}
      We claim that $(\mathbb{R}, +_{\mathbb{R}}, \times_{\mathbb{R}}, \leq_{\mathbb{R}})$ is a totally ordered field, and the canonical injection $\iota: \mathbb{Q} \rightarrow \mathbb{R}$ defined 
      \begin{equation}
        \iota(q) = (q)
      \end{equation} 
      is an ordered field isomorphism. Finally, by construction $\mathbb{R}$ is Cauchy-complete. 
    \end{theorem}

  \subsubsection{Nested Intervals Completeness} 

    The final way we prove is using nested-intervals completeness.  

    \begin{definition}[Nested Interval Completeness, Cantor's Intersection Theorem]
      Let $F$ be a totally ordered algebraic field. Let $I_n= [a_n, b_n]$ ($a_n < b_n$) be a sequence of closed intervals, and suppose that these intervals are nested in the sense that
      \[I_1 \supset I_2 \supset I_3 \supset \ldots\]
      where 
      \[\lim_{n \rightarrow + \infty} b_n - a_n = 0\]
      $F$ is complete if the intersection of all of these intervals $I_n$ contains exactly one point. That is, 
      \[\bigcup_{n=1}^\infty I_n \in F\]
    \end{definition}

    Note that defining nested intervals requires only an ordered field. One may look at this and try to ask if this is a specific instance of the following conjecture: The intersection of a nested sequence of nonempty closed sets in a topological space has exactly 1 point. This claim may not even make sense, actually. If we define nested in terms of proper subsets, then for a finite topological space a sequence cannot exist since we will run out of open sets and so this claim is vacuously true and false. If we allow $S_n = S_{n+1}$ then we can just select $X \supset X \supset \ldots$, which is obviously not true. However, a slightly weaker claim is that every proper nested non-empty closed sets has a non-empty intersection is a consequence of compactness. 

    \begin{theorem}
      $\mathbb{Q}$ is not nested-interval complete. 
    \end{theorem}
    \begin{proof}
      Consider the intervals $[a_i, b_i]$ where $a_i$ is the largest number $x$ up to the $n$th decimal expansion such that $x^2$ does not exceed $2$, and $b_i$ is the smallest number $x$ up to the $n$th decimal expansion such that $x^2$ is not smaller than $2$. The first few terms are 
      \begin{equation}
        [1.4, 1.5], [1.41, 1.42], [1.414, 1.415], \ldots
      \end{equation}
    \end{proof}

    Therefore, we can complete $\mathbb{Q}$. It turns out that this is equivalent to the construction using Dedekind cuts, and by definition this new set is nested interval complete. However, like Cauchy completeness, this actually does not imply the Archimedean property. 

  \subsubsection{Properties of the Real Line} 

    Now that we have completed it, we can define the real numbers. 

    \begin{definition}[The Real Numbers]
      The \textbf{set of real numbers}, denoted $\mathbb{R}$, is a totally ordered complete Archimedean field. 
    \end{definition} 

    It seems that the real numbers is \textit{any} set that satisfies the definition above. Does this mean that there are multiple real number lines? 

    \begin{example}[Multiple Reals?]
      For example, let us construct three distinct sets satisfying these axioms: 
      \begin{enumerate}
        \item A line $\mathbb{L}$ with $+$ associated with the translation of $\mathbb{L}$ along itself and $\cdot$ associated with the "stretching/compressing" of the line around the additive origin $0$. 
        \item An uncountable list of numbers with possibly infinite decimal points, known as the decimal number system. 
        \begin{equation}
          \ldots, -2.583\ldots, \ldots , 0, \ldots, 1.2343\ldots, \ldots, \sqrt{2}, \ldots
        \end{equation}
        \item A circle with a point removed, with addition and multiplication defined similarly as the line. 
      \end{enumerate}
    \end{example}

    We will show that there is only one set, up to isomorphism, that satisfies all these properties. 

    \begin{theorem}[Uniqueness]
      $\mathbb{R}$ is unique up to field isomorphism. That is, if two individuals construct two ordered complete Archimedean fields $\mathbb{R}_A$ and $\mathbb{R}_B$, then 
      \begin{equation}
        \mathbb{R}_A \simeq \mathbb{R}_B
      \end{equation}
    \end{theorem}  
    \begin{proof}
      The proof is actually much longer than I expected, so I draw a general outline.\footnote{Followed from \href{https://math.ucr.edu/~res/math205A/uniqreals.pdf}{here}.} We want to show how to construct an isomorphism $f: \mathbb{R}_A \rightarrow \mathbb{R}_B$. 
      \begin{enumerate}
        \item Realize that there are unique embeddings of $\mathbb{N}$ in $\mathbb{R}_A$ and $\mathbb{R}_B$ that preserve the inductive principle, the order, closure of addition, and closure of multiplication, the additive identity, and the multiplicative identity. Call these ordered doubly-monoid (since it's a monoid w.r.t. $+$ and $\times$) homomorphisms $\iota_A, \iota_B$. 
        \item Construct an isomorphism $f_1: \iota_A(\mathbb{N}) \rightarrow \iota_B(\mathbb{N})$ that preserves the inductive principle, order, addition, and multiplication. This is easy to do by just constructing $f_1 = \iota_B \circ \iota_A^{-1}$. 
        \item Extend $f_1$ to the ordered ring isomorphism $f_2$ by explicitly defining what it means to map additive inverses, i.e. negative numbers. 
        \item Extend $f_2$ to the ordered field isomorphism $f_3$ by explicitly defining what it means to map multiplicative inverses, i.e. reciprocals. 
        \item Extend $f_3$ to the ordered field isomorphism on the entire domain $\mathbb{R}_A$ and codomain $\mathbb{R}_B$. There is no additional operations that we need to support, but we should explicitly show that this is both injective and surjective, which completes our proof. 
      \end{enumerate}
    \end{proof}

    \begin{corollary}[Dedekind and Cauchy Completness are Equivalent for Reals]
      Let $\mathbb{R}_D, \mathbb{R}_C$ be the Dedekind and Cauchy completion of $\mathbb{Q}$. Then $\mathbb{R}_D \simeq \mathbb{R}_C$.\footnote{Note that this is only true for totally ordered Archimidean fields! The two completeness properties are not equal in general!}
    \end{corollary}

    The second new property is that the reals are uncountable. 

    \begin{theorem}[Cantor's Diagonalization]
      The real numbers are uncountable.
    \end{theorem}
    \begin{proof}
      We proceed by contradiction. Suppose the real numbers are countable. Then there exists a bijection $f: \mathbb{N} \to \mathbb{R}$. This means we can list all real numbers in $[0,1]$ as an infinite sequence.\footnote{This must be explicitly proven, but we can take the set of all Cauchy sequences of rationals in their decimal expansion and construct the reals this way.}
      
      \begin{align*}
        f(1) &= 0.a_{11}a_{12}a_{13}\dots \\
        f(2) &= 0.a_{21}a_{22}a_{23}\dots \\
        f(3) &= 0.a_{31}a_{32}a_{33}\dots \\
        &\vdots
      \end{align*}
      
      where each $a_{ij}$ is a digit between 0 and 9.
      
      Now construct a new real number $r = 0.r_1r_2r_3\dots$ where:
      \begin{equation}
        r_n = \begin{cases}
          1 & \text{if } a_{nn} \neq 1 \\
          2 & \text{if } a_{nn} = 1
        \end{cases}
      \end{equation}
      This number $r$ is different from $f(n)$ for every $n \in \mathbb{N}$, since $r$ differs from $f(n)$ in the $n$th decimal place. Therefore $r \in [0,1]$ but $r \notin \text{range}(f)$, contradicting that $f$ is surjective. Thus our assumption that the real numbers are countable must be false.
    \end{proof}

    Provide examples of ordered, Cauchy-complete fields that are not Archimedean.  

  \subsubsection{Roots, Exponentials, and Logarithms} 

    Now we will focus on some other operations that become well-defined in the reals. We know that $x^{n}$ for $n \in \mathbb{N}$ denotes repeated multiplication and $x^{-1}$ denotes the multiplicative inverse. We need to build up on this notation. As a general outline, we will show that $x^{-n}$ is well defined, then $x^q, q \in \mathbb{Q}$ is well-defined, and finally $x^r, r \in \mathbb{R}$ is well-defined. For the naturals, we have defined $x^n$ as the repeated multiplication of $n$. It is trivial that the canonical injection $\iota_0: \mathbb{N} \rightarrow \mathbb{R}$ commutes with the exponential map of naturals. We prove that $\iota_1: \mathbb{Z} \rightarrow \mathbb{R}$ also commutes. 

    \begin{lemma}[Integer Exponents]
      We have 
      \begin{enumerate}
        \item For $x_1, \ldots, x_n \in \mathbb{R}$, $(x_1 \ldots x_n)^{-1} = x_n^{-1} \ldots x_1^{-1}$. 
        \item For $x \in \mathbb{R}$, $x > 0$, $(x^n)^{-1} = (x^{-1})^n$. This value is denoted $x^{-n}$. 
        \item For $x \in \mathbb{R}$ and $w, z \in \mathbb{Z}$, $x^{w + z} = x^w x^z$. 
        \item For $w, z \in \mathbb{Z}$, $x^{wz} = (x^z)^w = (x^w)^z$. 
      \end{enumerate}
    \end{lemma}
    \begin{proof}
      Listed. 
      \begin{enumerate}
        \item The proof is trivial, but for $n = 2$ and $x_1 = x, x_2 = y$, we see that by associativity, $(x^{-1} x^{-1}) (x y) = y^{-1} (x^{-1} x) y = y^{-1} y = 1$ and we know inverses are unique. 
        \item Set $x_i = x$ using (1). 
        \item If $w, z > 0$ this is trivial by the associative property. If either or both are negative, say $w < 0 < z$, then we set $w^\prime = -w > 0$ and using (2) we know that 
        \begin{equation}
          x^{w} x^{z} = (x^{-1})^{w^\prime} x^z = x^{-w^\prime + z} = x^{w + z}
        \end{equation}
        by associativity in the second last equality. 
      \end{enumerate}
    \end{proof}

    Therefore, we have successfully defined $x^z$ for all $z \in \mathbb{Z}$, and if $z$ is negative, we're allowed to ``swap'' the $-1$ and $|z|$ in the exponents. Now we want to extend this into rational exponents, first by proving the existence and uniqueness of $n$th roots for any real. The proof is a little involved, but the general idea is that we want to use the LUB property to define the $n$th root as the supremum of a set.  

    \begin{theorem}[Existence of Nth Roots]
      For any real $x > 0$ and every $n \in \mathbb{N}$ there is one and only one positive real $y \in \mathbb{R}$ s.t. $y^n = x$. This is denoted $x^{1/n}$. 
    \end{theorem}
    \begin{proof}
      Let $E$ be the set consisting of all reals $t \in \mathbb{R}$ s.t. $t^n < x$. We show that 
      \begin{enumerate}
        \item it is nonempty. Consider $t = x/(1+x)$. Then $0 \leq t < 1 \implies t^n \leq t < x$. Thus $t \in E$ and $E$ is nonempty. 
        \item it is bounded. Consider any number $s = 1 + x$. Then $s^n \geq s > x$, so $s \not\in E$, and $s = 1 + x$ is an upper bound of $E$. 
      \end{enumerate}
      Therefore, $E$ is a nonempty set that is upper bounded, so it has a least upper bound, called $y = \sup{E}$. We claim that $y^n = x$, proving by contradiction. For both cases, we use the fact that the identity $b^n - a^n = (b - a) (b^{n-1} + b^{n-2} a + \ldots + a^{n-1})$ gives the inequality 
      \begin{equation}
        b^n - a^n < (b-a) n b^{n-1} \text{ for } 0 < a < b
      \end{equation}
      \begin{enumerate}
        \item Assume $y^n < x$. Then we choose a fixed $0 < h < 1$ s.t. 
        \begin{equation}
          h < \frac{x - y^n}{n(y + 1)^{n-1}}
        \end{equation}
        Then by putting $a = y, b = y + h$, we have 
        \begin{equation}
          (y + h)^n - y^n < hn (y + h)^{n-1} < hn (y + 1)^{n-1} < x - y^n 
        \end{equation}
        and thus $y^n < (y + h)^n < x$. This means that $y + h \in E$, and so $y$ is not an upper bound. 

        \item Assume $y^n > x$. Then we set a fixed number 
        \begin{equation}
          k = \frac{y^n - x}{n y^{n-1}} 
        \end{equation}
        Then $0 < k < y$. If we take any $t \in \mathbb{R}$ s.t. $t \geq y - k$, this implies that $t^n \geq (y -k)^n \implies -t^n \geq -(y - k)^n$, and so 
        \begin{equation}
          y^n - t^n \leq y^n - (y - k)^n < k ny^{n-1} = y^n - x
        \end{equation}
        Thus $t^n > x$ and $t \not\in E$. So it must be the case that $t < y - k$, and so $y - k$ is an upper bound of $E$, contradicting that $y$ is least. 
      \end{enumerate}
    \end{proof}

    At this point, rooting has been introduced as sort of an independent map from exponentiation. We show that they have the nice property of commuting. 

    \begin{lemma}[Rooting and Exponentiation Commute]
      For $p \in \mathbb{Z}, q \in \mathbb{N}$ and $x \in \mathbb{R}$ with $x > 0$, we have 
      \begin{equation}
        (x^{p})^{1/q} = (x^{1/q})^p
      \end{equation}
    \end{lemma}
    \begin{proof}
      If $p > 0$, then let $r = (x^p)^{1/q}$. By definition $r^q = x^p$. Let $s = x^{1/q}$ By definition $s^q = x$. Therefore $r^q = (s^q)^p = s^{qp}$ from the lemma on integer exponents. But since roots are well-defined and unique 
      \begin{equation}
        r = (r^q)^{1/q} = (s^{qp})^{1/q} = s^p \implies (x^p)^{1/q} = (x^{1/q})^p
      \end{equation}
      If $p = 0$, this is trivially $0$, and if $p < 0$ the by the same logic with $p = -p^\prime$ for $p^\prime > 0$ and $y = x^{-1} > 0$. we know 
      \begin{align}
        (x^p)^{1/q} = \big( (y^{-1})^{-p^\prime} \big)^{1/q} = (y^{-(-p^\prime)})^{1/q} & = (y^{p^\prime})^{1/q} \\ 
                           & = (y^{1/q})^{p^\prime} = ((x^{-1})^{1/q})^{p^\prime} = (x^{1/q})^{-p^\prime} = (x^{1/q})^p
      \end{align}
    \end{proof}

    \begin{theorem}[Rational Exponential Function]
      Given $m, p \in \mathbb{Z}$ and $n, q \in \mathbb{N}$, prove that 
      \begin{equation}
        (b^m)^{1/n} = (b^p){1/q}
      \end{equation}
      Hence it makes sens to define $b^r = (b^m)^{1/n}$, since every element of the equivalence class $r$ of each rational number maps to the same value. 
    \end{theorem} 
    \begin{proof}
      Since $m/n = p/q \implies mq = np$, 
      \begin{align}
        b^{mq} = b^{np} & \implies (b^m)^q = (b^p)^n \\
                        & \implies b^m = ((b^m)^q)^{1/q} = ((b^p)^n)^{1/q} \\
                        & \implies b^m = ((b^p)^{1/q})^n \\
                        & \implies (b^m)^{1/n} = (b^p)^{1/q}
      \end{align}
      Therefore we can define for any $r \in \mathbb{Q}$ 
      \begin{equation}
        x^r = x^{m/n} = (x^{m})^{1/n} = (x^{1/n})^m
      \end{equation}
      where the final equality holds from the commutativity of rooting and exponentiation. 
    \end{proof}

    It turns out that this is a homomorphism. 

    \begin{corollary}[Rational Exponential Function is a Homomorphism]
      The rational exponential function is a homomorphism. That is, given $r, s \in \mathbb{Q}$ and $x \in \mathbb{R}$, 
      \begin{equation}
        x^{r + s} = x^r \cdot x^s
      \end{equation}
    \end{corollary}
    \begin{proof}
      Let $r = m/n, s = p/q$. Then 
      \begin{align}
        x^{r+s} = x^{m/n + p/q} & = x^{\frac{mq + np}{nq}} && \tag{addition on $\mathbb{Q}$}\\
                                & = (x^{mq + np})^{1/nq} && \tag{exp and roots commute}\\
                                & = (x^{mq} + x^{np})^{1/nq} && \tag{int exp lemma}\\
                                & = (x^{mq})^{1/nq} (x^{np})^{1/nq} && \tag{int exp lemma}\\
                                & = x^{mq/nq} x^{np/nq} && \tag{exp and roots commute} \\
                                & = x^{m/n} x^{p/q} && \tag{relation from $\mathbb{Q}$}
      \end{align}
    \end{proof}

    With rational exponents defined, we can use the least upper bound property to define a consistent extension of a real exponent. 
    
    \begin{lemma} 
      If $r \in \mathbb{Q}$ with $r \geq 0$, then for $x \in \mathbb{R}$, $x > 1$, $1 \leq b^r$. 
    \end{lemma}
    \begin{proof}
      Let $r = m/n$. Then $x^r = x^{m/n} = (x^m)^{1/n}$. Since $1 < x$, and $m \geq 0$, we have 
      \begin{equation}
        1 \leq x \leq x^2 \leq \ldots \leq x^m \implies 1 \leq b^m
      \end{equation}
      Now set $y = x^{m/n}$ and assume that $y < 1$. Then 
      \begin{equation}
        x^m = y^n < y^{n-1} < \ldots < y < 1
      \end{equation}
      and so $x^m < 1$, which is a contradiction. So it must be the case that $y > 1$. 
    \end{proof}

    \begin{lemma}[Monotonicity of Rational Exponents]
      If $x, y \in \mathbb{R}$, then for any rational $r \in \mathbb{Q}$ with $r < x + y$, there exists a $p, q \in \mathbb{Q}$ s.t. $p < x, q < y$ and $p + q = r$. The converse is true as well. 
    \end{lemma}
    \begin{proof}
      $r < x + y \implies r - y < x$. By density of $\mathbb{Q}$ in $\mathbb{R}$, we can choose $r - y < p < x$. Then $-r + y > -p > x \implies r - r + y > r - p > r - x \implies y > r - p > r - x$, and we set $q = r - p$. We are done. The converse is trivial since given $p, q \in \mathbb{Q}$ with $p < x, q < y$, then by the ordered field properties $p + q < x + y$. 
    \end{proof}

    \begin{corollary}[Real Exponential Function]
      Given $x\in \mathbb{R}$, we define 
      \begin{equation}
        B(x) \coloneqq \{ x^q \in \mathbb{R} \mid q \in \mathbb{Q}, \; q \leq x \}
      \end{equation}
      We claim that given $r \in \mathbb{R}$, 
      \begin{equation}
        x^r \coloneqq \sup B(r)
      \end{equation}
      is well-defined and is a homomorphism extension of the rational exponential function. That is, 
      \begin{equation}
        \sup{B(x + y)} = \sup{B(x)} \cdot \sup{B(y)}
      \end{equation}
    \end{corollary}
    \begin{proof}
      To show that $x^r \coloneqq \sup B(r)$ where $B(r) = \{x^t \in \mathbb{R} \mid t \in \mathbb{Q}, t \leq r \}$, 
      \begin{enumerate}
        \item We show it's an upper bound. Assume it wasn't. Then $x^r < x^t$ for some $t \in \mathbb{Q}$ satisfying $t \leq r$. But $t \leq r \implies 0 \leq r - t$, and by the previous lemma, $1 \leq x^{r - t}$. So $1 \leq x^{r-t} = x^{r} x^{-t} = x^r (x^t)^{-1} \implies x^t \leq x^r$, which is a contradiction. 
        \item We show that it is least. Assume that it is not. Then $\exists r^\prime \in \mathbb{Q}$ s.t. $x^t \leq x^{r^\prime}$ and $r^\prime < r$. Now let $s \in \mathbb{Q}$ be an element between $r^\prime$ and $r$, which is guaranteed to exist due to density of rationals in reals. But $s < r$, so by definition $x^s \in B(r)$, but 
        \begin{align}
          0 < s - r^\prime & \implies 1 < b^{s - r^\prime} \\
                           & \implies b^{r^\prime} (b^{r^\prime})^{-1} < b^s (b^{-r^\prime}) \\
                           & \implies 1 < b^s (b^{r^\prime})^{-1} \\
                           & \implies b^{r^\prime} < b^s
        \end{align}
        and so $b^{r^\prime}$ is not an upper bound for $B(r)$. By contradiction, $b^r$ is least. 
      \end{enumerate}
      Since this is defined, the analogous definition for real numbers is consistent with that of hte rationals, and it is upper bounded by the Archimedean principle, so such a supremum must exist. Note that $t$ is rational. For the second part, from the previous lemma and the homomorphism properties of the rational exponent, 
      \begin{align}
        B(x + y) = B^\prime (x + y) & \coloneqq \{b^{p+q} \in \mathbb{R} \mid p, q \in \mathbb{Q}, p \leq x, q \leq y\} \\
                                    & = \{b^p b^q \in \mathbb{R} \mid p, q \in \mathbb{Q}, p \leq x, q \leq y\} \\
      \end{align}
      Therefore we can treat $B$ and $B^\prime$ as the same set. 
      \begin{enumerate}
        \item Prove upper bound $\sup{B(x + y)} \leq \sup{B(x)} \sup{B(y)}$. Given $\alpha \in B^\prime (x + y)$, there exists $p_\alpha, q_{\alpha} \in \mathbb{Q}$ (with $p_\alpha < x$, $q_\alpha < y$) s.t. $b^{p_{\alpha}} b^{q_{\alpha}} = \alpha$. But 
        \begin{equation}
          b^{p_{\alpha}} b^{q_{\alpha}} \leq \sup_{p_{\alpha}} \{ b^{p_{\alpha}}\} \cdot \sup_{q_{\alpha}} \{b^{q_{\alpha}}\} = \sup{B(x)} \sup{B(y)}
        \end{equation}

      \item To prove least, assume there exists $K \in \mathbb{R}$ s.t. $\sup{B^\prime(x + y)} \leq K < \sup{B(x)} \sup{B(y)}$. Then, since the image of $b^x$ is always positive, we assume $0 < K$. We bound its factors as so: $K < \sup{B(x)} \sup{B(y)} \implies K/\sup{B(x)} < \sup{B(y)}$. By density of the rationals, there exists a $\beta \in \mathbb{Q}$, s.t. 
      \begin{equation}
        \frac{K}{\sup{B(x)}} < \beta < \sup{B(y)}
      \end{equation}
      This means $K/\beta < \sup{B(x)}$ and $\beta < \sup{B(y)}$. But this means that there exists $\phi, \gamma \in B(x), B(y)$ s.t. $K/\beta < \phi, \beta < \gamma \implies K = (K/\beta) \cdot \beta < \phi \gamma \implies \phi \gamma \in B^\prime(x + y)$ by definition. So $K$ is not an upper bound. 
      \end{enumerate}
    \end{proof}

    Furthermore, this is an isomorphism, and the inverse is defined. Let's define this analytically. 

    \begin{theorem}[Logarithm]
      For $b > 1$ and $y > 0$, there is a unique real number $x$ s.t. $b^x = y$. We claim 
      \begin{equation}
        x = \sup\{ w \in \mathbb{R} b^w < y \}
      \end{equation}
      $x$ is called the \textbf{logarithm of $y$ to the base $b$}. 
    \end{theorem}
    \begin{proof}
      We use the inequality $b^n - 1 \leq n (b-1)$ for all $n \in \mathbb{N}$.\footnote{We prove by induction. For $n=1$ $b^1 - 1 \leq 1 (b-1)$. Assume that this holds for some $n$. Then $b^{n+1} - 1 = b^{n+1} - b + b - 1 = b (b^n - 1) + (b-1) \geq bn (b-1) + (b-1) = (bn + 1) (b-1) \geq (n+1) (b-1)$, where the last step follows since $b \geq 1 \implies bn \geq n \implies bn + 1 \geq n + 1$. } By substituting $b = b^{1/n}$ (valid since $b > 1 \iff b^{1/n} > 1$) so $b-1 \geq n(b^{1/n} - 1)$. Now set some $t > 1$, and by Archimidean principle, we can choose some $n \in \mathbb{N}$ s.t. $n > \frac{b-1}{t-1}$. Then $n (t-1) > b-1$, and with the inequality derived we get 
      \begin{equation}
        n (t - 1) > b - 1 \geq n (b^{1/n} - 1) \implies t > b^{1/n}
      \end{equation} 
      This allows us to prove 2 things. 
      \begin{enumerate}
        \item If $w$ satisfies $b^w < y$, then $b^{w + (1/n)} < y$ for sufficiently large $n$. Setting $t = y b^{-w}$ (which is greater than $1$ since $b^w < y$) gives $y \cdot b^{-w} > b^{1/n} \implies b^w b^{1/n} < y \implies b^{w + (1/n)} < y$. 
        \item If $w$ satisfies $b^w > y$, then $b^{w - (1/n)} > y$ for sufficiently large $n$. Setting $t = b^w y^{-1}$ (which is greater than $1$ since $b^w > y$) gives $b^w y^{-1} > b^{1/n} \implies b^{w - (1/n)} > y$. 
      \end{enumerate}
      Now we can prove existence. Let $A$ the set of all $w$ s.t. $b^w < y$. We claim that $x = \sup{A}$. 
      \begin{enumerate}
        \item Assume that $b^x < y$. We know that there exists $n \in \mathbb{N}$ s.t. $b^{x + (1/n)} < y \implies x + (1/n) \in A$, contradicting that $x$ is an upper bound. 
        \item Assume that $b^x > $. We know that there exists $n \in \mathbb{N}$ s.t. $b^{x - (1/n)} > y \implies x - (1/n)$ is also an upper bound for $A$, contradicting that $x$ is least. Therefore $b^x = y$. 
      \end{enumerate}
      We now prove uniqueness. Assume that there are two such $x$'s , call them $x, x^\prime$. By total ordering and $x \neq x^\prime$, WLOG let $x > x^\prime \implies x - x^\prime > 0 \implies b^{x - x^\prime} > 1$. By density of rationals, since we can choose $r \in \mathbb{R}$ s.t. $0 < r < x - x^\prime$, we have $1 < b^r < b^{x - x^\prime}$ and so $B(r) \subset B(x - x^\prime)$. Since $1 < b^{x - x^\prime} \implies 1 \cdot b^{x^\prime} < b^{x - x^\prime} \cdot b^{x^\prime} = b^x$, we have $b^{x^\prime} < b^x$ and they cannot both by $y$. So $x = x^\prime$. 
    \end{proof}

\subsection{The Extended Reals and Hyperreals}

  Great! We have officially constructed the reals, and we can finally feel satisfied about defining metrics, norms, and inner products as mappings to the codomain $\mathbb{R}$. Now let's make the concept of infinite numbers a bit more rigorous. In short, what we do is just add the numbers $\pm \infty$ to $\mathbb{R}$, which we call the extended reals, and try and extend the properties from $\mathbb{R}$ to the extended reals. We will see that not all properties can be transferred. 

  \begin{theorem}[Extended Real Number Line]
    The \textbf{extended real number line} is the set $\overline{\mathbb{R}} \coloneqq \mathbb{R} \cup \{-\infty, +\infty\}$. We define the following operations. 
    \begin{enumerate}
      \item \textit{Order}. $-\infty \leq x$ and $x \leq +\infty$ for all $x \in \overline{\mathbb{R}}$. 
      \item \textit{Addition}. $+\infty - \infty = 0$. $x + \infty = +\infty$ and $x - \infty = -\infty$ for all $x \in \mathbb{R}$. 
      \item \textit{Multiplication}. 
      \begin{equation}
        x \times \infty = \begin{cases} +\infty & \text{ if } x > 0 \\ 0 & \text{ if } x = 0 \\ -\infty & \text{ if } x < 0 \end{cases}
      \end{equation}
    \end{enumerate}
    It turns out that this is still Dedekind-complete, which is nice. Unfortunately we lose a lot of structure. 
    \begin{enumerate}
      \item this is not even a field since the multiplicative inverse of $\pm \infty$ is not defined. 
      \item the Archimedean principle does not hold 
      \item we cannot define a metric nor a norm. 
      \item we can define the order topology, however. 
    \end{enumerate}
  \end{theorem} 

  The loss of the field property is quite bad, and we might want to recover this. Therefore, we can add more elements that serve to be the multiplicative inverse of infinity. We call these inverses \textit{infinitesimals} and the new set the \textit{hyperreal numbers}. 

  \begin{theorem}[Hyperreals]
    The \textbf{hyperreals} 
  \end{theorem}

  In fact, when Newton first invented calculus, the hyperreals were what he worked with, and you can surprisingly build a good chunk of calculus with this. Even though this is a dead topic at this point, a lot of modern notation is based off of this number system, so it's good to see how it works. For example, when we write the integral 
  \begin{equation}
    \int f(x) \,dx
  \end{equation} 
  we are saying that we are taking the uncountable sum of the terms $f(x) \,dx$, the multiplication of the real number $f(x)$ and the infinitesimal number $dx$ living in the hyperreals. Unfortunately, we cannot fully construct a rigorous theory of calculus with only infinitesimals. However, in practice (especially physics) people tend to manipulate and do algebra with infinitesimals, so having a good foundation on what you can and cannot do with them is practical. While the focus won't be on \textit{smooth infinitesimal analysis (SIA)}, I will include some alternate constructions later purely with infinitesimals. 

\subsection{Complex Numbers} 

  The next field that will be particularly important is the complex numbers. It is straightforward to construct $\mathbb{C}$, but let's motivate this for a minute. 

  \begin{example}[Polynomial Roots]
    The roots of the polynomial 
    \begin{equation}
      f(x) = x^2 + 1
    \end{equation}
    does not exist in $\mathbb{R}$. 
  \end{example} 

  Therefore, we would like to construct a new space that contains all possible roots for all possible polynomials with real coefficients. We call this $\mathbb{C}$. Clearly, by constructing polynomials of the form $x^2 - r^2$ for some $r \in \mathbb{R}$, we know that $\mathbb{R} \subset \mathbb{C}$. Therefore, we want to create a further extension of $\mathbb{R}$, along with some canonical injection $\iota: \mathbb{R} \rightarrow \mathbb{C}$ that is also a field homomorphism. It turns out that once we construct this field, there is no possible way that we can make it an ordered field. However, the norm extends naturally into $\mathbb{C}$ such that $\iota$ is isometric. Finally, we can define a new operator called \textit{conjugation} that gives us additional structure. 

  This is not the only way to construct the complex plane however. Rather than defining all these from scratch, we could just define the addition operations with an isometric vector space isomorphism from $\mathbb{R}^2$ to $\mathbb{C}$ actually, and then define multiplication. Another way is to start again with $\mathbb{Q} \times \mathbb{Q}$, define a norm on it, complete it, and finally define the addition and multiplication operations that satisfy the field property.   

  \begin{theorem}[Construction of the Complex Numbers]
    Let $\mathbb{C}$ be defined as the space $\mathbb{R} \times \mathbb{R}$ with the following operations. 
    \begin{enumerate}
      \item \textit{Addition}. $x = (a, b), y = (c, d) \implies x +_{\mathbb{C}} y = (a + c, b + d)$. 
      \item \textit{Additive Identity}. $0_{\mathbb{C}} = (0, 0)$. 
      \item \textit{Additive Inverse}. $x = (a, b) \implies -x = (-a, -b)$. 
      \item \textit{Multiplication}. $x = (a, b), y = (c, d) \implies x \times_{\mathbb{C}} y = (ac - bd, ad + bc)$. 
      \item \textit{Multiplicative Identity}. $1_{\mathbb{C}} = (1, 0)$. 
      \item \textit{Multiplicative Inverse}. 
      \begin{equation}
        x = (a, b) \implies x^{-1} = \bigg( \frac{a}{a^2 + b^2}, \frac{-b}{a^2 + b^2} \bigg)
      \end{equation}
    \end{enumerate}
    Our first claim is that $(\mathbb{C}, +_{\mathbb{C}}, \times_{\mathbb{C}})$ is a field. Furthermore, we define the additional structures
    \begin{enumerate}
      \item \textit{Conjugate}. $x = (a, b) \implies \overline{x} = (a, -b)$. 
      \item \textit{Norm}. $|x|_{\mathbb{C}} = x \times_{\mathbb{C}} \overline{x} = a^2 + b^2$. 
      \item \textit{Metric}. This is the norm-induced metric. $d_{\mathbb{C}}(x, y) = |x - y|_{\mathbb{C}}$. 
      \item \textit{Topology}. This is the metric-induced topology generated by the open balls $B(x, r) \coloneqq \{y \in \mathbb{C} | d(x, y) < r\}$, where $x \in \mathbb{C}, r \in \mathbb{R}$. 
    \end{enumerate} 
    Our second claim is that the canonical injection $\iota: \mathbb{R} \rightarrow \mathbb{C}$ defined 
    \begin{equation}
      \iota(r) = (r, 0)
    \end{equation}
    is an isometric field isomorphism. Our third claim is that $\mathbb{C}$ is Cauchy-complete with respect to this metric. 
  \end{theorem} 

  Note that we do not talk about order $\mathbb{C}$, and so the concepts of Dedekind completeness, least upper bound properties, or Archimedean principle is meaningless in the complex plane. 

  \begin{definition}[Imaginary Number] 
    Let us denote $i = (0, 1)$ which we call the \textbf{imaginary number}, which has the property that $i^2 = 1$. With this notation, we can see through abuse of notation that 
    \begin{equation}
      (a, b) = (a, 0) + (0, b) = (a, 0) + (b, 0) (0, 1) = a + bi
    \end{equation} 
    Therefore, we generally write complex numbers as $z = a + bi$, and we define the real and imaginary components as $\re(z)$ and $\im(z)$, respectively. 
  \end{definition}

  Note that the identity $x^2 + 1 \equiv (x + i) (x - i)$ implies that the equation $x^2 = -1$ has exactly two solutions in $\mathbb{C}$, $i$ and $-i$. Therefore, if a subfield of $\mathbb{C}$ contains one of these solutions, it must contain the other (since $i$ and $-i$ are additive and multiplicative inverses). 

  Furthermore, since $i$ is defined to be $\sqrt{-1}$, we could replace $i$ with $-i$ and our calculations would still be consistent throughout the rest of mathematics. In fact, $i$ and $-i$ behave \textbf{exactly} identically and cannot be distinguished in an abstract sense. Visually, the complex plane "flipped" across the real number axis produces the same complex plane. 

  \begin{theorem}[Uniqueness of $\mathbb{C}$]
    $\mathbb{C}$ is unique up to an isomorphism that maps all real numbers to themselves. Every complex number can be uniquely written as $a + bi$, where $a, b \in \mathbb{R}$ and $i$ is a fixed element such that $i^2 = -1$. 
  \end{theorem}
  \begin{proof}
    Consider the subset of $\mathbb{C}$
    \begin{equation}
      K \equiv \{ a + bi \; | \; a, b \in \mathbb{R}\}
    \end{equation}
    By evaluating its operations, we can check for closure, identity, and invertibility of nonzero elements to conclude that $K$ is a subfield of $\mathbb{C} \implies$ by prop. (iii), $K = \mathbb{C} \implies$ every element in $\mathbb{C}$ can be written in form $a + bi$. To prove uniqueness, we assume that $p \in \mathbb{C}$ can be written in distinct forms $p = a + bi = a^{\prime} + b^\prime i$. Then
    \begin{align*}
       a + bi = a^{\prime} + b^\prime i & \implies (a - a^\prime)^2 = (b^\prime i - b i)^2 = - (b^\prime - b)^2 \\
       & \implies a - a^\prime = b^\prime - b = 0
    \end{align*}
    To prove uniqueness of $\mathbb{C}$ up to ismorphism, we assume that $\mathbb{C}^\prime$ exists with $i^\prime$ such that $i^{\prime 2}$ containing elements $a + b i'$. Let $f: \mathbb{C} \longrightarrow \mathbb{C}^\prime$ defined 
    \begin{equation}
      f( a + bi) = a + bi^\prime
    \end{equation}
    Then, 
    \begin{align*}
      f\big((a + b i) + (c + d i) \big) & = f\big( (a + c) + (b + d)i \big) \\
      & = (a + c) + (b + d) i^\prime \\
      & = (a + b i^\prime) + (c + d i^\prime) \\
      & = f(a + b i) + f( c + d i) \\
      f\big( \kappa (a + b i)\big) & = f\big( \kappa a + \kappa b i\big) \\
      & = \kappa a + \kappa b i^\prime \\
      & = \kappa (a + b i^\prime) \\
      & = \kappa f(a + b i)
    \end{align*}
    So, $f$ is an isomorphism, and $\mathbb{C} \simeq \mathbb{C}^\prime$. From analysis, we can construct and prove the existence of $\mathbb{R}$. We then define the map
    \begin{equation}
      \rho: \mathbb{R}^2 \longrightarrow \mathbb{C}, \; \rho(a, b) \equiv a + bi
    \end{equation}
    with $\rho(1, 0)$ as the multiplicative identity and $\rho(0,1) \equiv i$. Therefore, every element of $\mathbb{C}$ can be uniquely represented as an element of $\mathbb{R}^2$. 
  \end{proof}

  Unfortunately, we lose the ordering. 

  \begin{theorem}[Order on Complex Plane]
    There exists no order on $\mathbb{C}$ that makes it a totally ordered field.
  \end{theorem}
  \begin{proof}
    We attempt to construct an order on $i$ and $0$ in $\mathbb{C}$. 
    \begin{enumerate}
      \item If $i = 0$, then $i^4 = 0 \cdot i^3 \implies 1 = 0$, which contradicts that $0 < 1$. 
      \item If $i \neq 0$, then $i^2 > 0$ from the field axioms, and so $-1 > 0$. But this also means that $1 = i^4 > 0$. This contradicts the ordered field property that $x > 0 \iff -x < 0$. 
    \end{enumerate}
    Therefore $\mathbb{C}$ cannot be turned into an ordered field. 
  \end{proof}

  \begin{theorem}[Conjugation is an Isomorphism]
    Conjugation is an isometric field automorphism of $\mathbb{C}$. 
    \begin{equation}
      c = a + b i \mapsto \bar{c} = a - b i
    \end{equation}
    This is identically defined by replacing $i$ with $-i$. Clearly, $\bar{\bar{c}} = c$. 
  \end{theorem}
  \begin{proof}
    
  \end{proof}

  \begin{proposition}[Properties of Conjugation]
    For any $c \in \mathbb{C}$, $c + \bar{c}$ and $c \bar{c}$ are real. 
  \end{proposition}
  \begin{proof}
    Using the fact that the complex conjugate is an isomorphism, 
    \begin{align*}
      & \bar{c + \bar{c}} = \bar{c} + \bar{\bar{c}} = \bar{c} + c = c + \bar{c} \\
      & \bar{ c \bar{c}} = \bar{c} \bar{\bar{c}} = \bar{c} c = c \bar{c}
    \end{align*}
  \end{proof}
  Note that we proved this abstractly using only the properties given above, and did not decompose $c$ to its \textbf{algebraic form} $a + b i$. 

  If $c = a + b i, \; a, b \in \mathbb{R}$, then 
  \begin{equation}
    c + \bar{c} = 2a, \; c \bar{c} = a^2 + b^2
  \end{equation}

  In case the reader is unaware, it is common to interpret complex numbers $c = a + b i$ as points or vectors $(a, b)$ on the complex plane. 

  \begin{definition}[Polar Form of Complex Numbers]
    The \textbf{polar representation}, or \textbf{trigonometric representation}, of a complex number $c = a + b i$ is defined using the equations 
    \begin{equation}
      a = r \cos{\varphi}, \; b = r\sin{\varphi} \implies c = r (\cos{\varphi} + i \sin{\varphi})
    \end{equation}
    where $r = |c|$ and $\varphi$ is the \textbf{argument} of $c$, which is 
    the angle formed by the corresponding vector with the polar axis defined within the interval $[0, 2\pi)$. 
    \begin{equation}
      \text{arg}(c) \equiv \tan^{-1}{\frac{b}{a}}
    \end{equation}
    This mapping can be defined 
    \begin{equation}
      \rho: \mathbb{R} \times \frac{\mathbb{R}}{2 \pi} \longrightarrow \mathbb{C}, \; \rho(r, \varphi) = r (\cos{\varphi} + i \sin{\varphi})
    \end{equation}
  \end{definition}

  \begin{theorem}
    $\rho$ is "similar" to a homomorphism in the following way. By defining the domain and codomain as groups, 
    \begin{equation}
      \rho: \big( \mathbb{R}, \times \big) \times \Big( \frac{\mathbb{R}}{2 \pi} \Big) \longrightarrow \big( \mathbb{C}, \times \big)
    \end{equation}
    we can see that
    \begin{equation}
      \rho (r_1, \varphi_1) \times \rho(r_2, \varphi_2) = \rho(r_1 \times r_2, \varphi_1 + \varphi_2) 
    \end{equation}
    or equivalently, 
    \begin{equation}
      r_1 (\cos{\varphi_1} + i \sin{\varphi_1}) \cdot r_2 (\cos{\varphi_2} + i \sin{\varphi_2}) = r_1 r_2 (\cos{(\varphi_1 + \varphi_2)} + i \sin{(\varphi_1 + \varphi_2)})
    \end{equation}
  \end{theorem}

  \begin{corollary}
    The formula for the ratio of complex numbers is defined
    \begin{equation}
      \frac{r_1 (\cos{\varphi_1} + i \sin{\varphi_1})}{r_2 (\cos{\varphi_2} + i \sin{\varphi_2})} = \frac{r_1}{r_2} (\cos{(\varphi_1 - \varphi_2)} + i \sin{(\varphi_1 - \varphi_2)})
    \end{equation}
  \end{corollary}

  \begin{corollary}
    The positive integer power of a complex number can be written using \textbf{De Moivre's formula}. 
    \begin{equation}
      \big(r(\cos{\varphi} + i \sin{\varphi})\big)^n = r^n (\cos{n \varphi} + i \sin{n \varphi})
    \end{equation}
  \end{corollary}

  We can use this formula to extract a root of $n$th degree from a complex number $c = r(\cos{\varphi} + i \sin{\varphi})$, which means to solve the equation $z^n = c$. Let $z = s (\cos{\psi} + i \sin{\psi})$. Then by De Moivre's formula, 
  \begin{align*}
    z^n & = s^n (\cos{n \psi} + i \sin{n \psi}) = r(\cos{\varphi} + i \sin{\varphi}) \\
    & \implies s = \sqrt[n]{r}, \; \psi = \frac{\varphi + 2\pi k}{n} \\
    & \implies z = \sqrt[n]{r} \bigg( \cos{\frac{\varphi + 2\pi k}{n}} + i \sin{\frac{\varphi + 2\pi k}{n}}\bigg) \text{ for } k = 0, 1, ..., n-1
  \end{align*}
  Geometrically, the $n$ solutions lie at the vertices of a regular $n$-gon centered at the origin. When $c = 1$, the solutions are the $n$th roots of unity.

\subsection{Dual Numbers}

  Another similar number system. 

\subsection{Euclidean Space} 

  Congratulations! We have rigorously constructed both the reals and complex numbers, and this becomes the cornerstone to construct other fundamental sets. Now we consider spaces of the form $\mathbb{R}^n$ or $\mathbb{C}^n$, which we call \textit{Euclidean spaces}, and construct them. This is actually quite easy since we understand linear algebra. 

  \begin{definition}[Convex Sets]
    A set $S$ is convex if for every point $x, y \in S$, the point 
    \begin{equation}
      z = t x + (1 - t) y \in S
    \end{equation}
    where $0 \leq t \leq 1$. 
  \end{definition}


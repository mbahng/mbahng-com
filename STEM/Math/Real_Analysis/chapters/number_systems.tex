\section{Number Systems} 

\subsection{The Rationals}

  \subsubsection{Field Properties}  

    \begin{definition}[Field]
      A \textbf{field} is an algebraic structure $(\mathbb{F}, +, \cdot)$ where 
      \begin{enumerate}
        \item $\mathbb{F}$ is an abelian group under $+$, with $0$ being the \textit{additive identity}. 
        \item $\mathbb{F} \setminus \{0\}$ is an abelian group under $\cdot$, with $1$ being the \textit{multiplicative identity}. 
        \item It connects the two operations through the \textit{distributive property}.
        \begin{equation}
          x \cdot (y + z) = x \cdot y + x \cdot z
        \end{equation}
      \end{enumerate}
    \end{definition} 

    \begin{lemma}[Left = Right Distributivity]
      Left and right distributivity are equivalent. 
      \begin{equation}
        x \cdot (y + z) = (y + z) \cdot x
      \end{equation}
    \end{lemma} 
    \begin{proof}
      \begin{align}
        x \cdot (y + z) & = x \cdot y + x \cdot z && \tag{Distributive} \\
                        & = y \cdot x + z \cdot x && \tag{Commutative} \\
                        & = (y + z) \cdot x && \tag{Distributive} 
      \end{align}
    \end{proof} 

    \begin{lemma}[Properties of Addition]
      The properties of addition hold in a field. 
      \begin{enumerate}
        \item If $x + y = x + z$, then $y = z$. 
        \item If $x + y = x$, then $y = 0$. 
        \item If $x + y = 0$, then $y = -x$. 
        \item $(-(-x)) = x$. 
      \end{enumerate}
    \end{lemma}
    \begin{proof}
      For the first, we have 
      \begin{align}
        x + y = x + z & \implies -x + (x + y) = -x + (x + z) && \tag{addition is a function} \\
                      & \implies (-x + x) + y = (-x + x) + z && \tag{$+$ is associative} \\
                      & \implies 0 + y = 0 + z && \tag{definition of additive inverse} \\
                      & \implies y = z && \tag{definition of identity}
      \end{align} 
      For the second, we can set $z = 0$ and apply the first property. For the third, we have 
      \begin{align}
        x + y = 0 & \implies -x + (x + y) = -x + 0 && \tag{addition is a function} \\
                  & \implies (-x + x) + y = -x + 0 && \tag{$+$ is associative} \\
                  & \implies 0 + y = -x + 0 && \tag{definition of additive inverse} \\
                  & \implies y = -x && \tag{definition of identity}
      \end{align}
      For the fourth, we simply follow that if $y$ is an inverse of $z$, then $z$ is an inverse of $y$. Therefore, $-x$ being an inverse of $x$ implies that $x$ is an inverse of $-x$. $-(-x)$ must also be an inverse of $-x$. Since inverses are unique\footnote{This is proved in algebra.}, $x = -(-x)$. 
    \end{proof}

    \begin{lemma}[Properties of Multiplication]
      The properties of multiplication hold in a field. 
      \begin{enumerate}
        \item If $x \neq 0$ and $xy = xz$, then $y = z$. 
        \item If $x \neq 0$ and $xy = x$, then $y = 1$. 
        \item If $x \neq 0$ and $xy = 1$, then $y = x^{-1}$. 
        \item If $x \neq 0$, then $(x^{-1})^{-1} = x$. 
      \end{enumerate}
    \end{lemma}
    \begin{proof}
      The proof is almost identical to the first. Since $x \neq 0$, we can always assume that $x^{-1}$ exists. For the first, we have
      \begin{align}
        x y = x z & \implies x^{-1} (x y) = x^{-1} (x z) && \tag{multiplication is a function} \\
                  & \implies (x^{-1} x) y = (x^{-1} x) z && \tag{$\times$ is associative} \\
                  & \implies 1 y = 1 z && \tag{definition of multiplicative inverse} \\  
                  & \implies y = z && \tag{definition of identity}
      \end{align}
      For the second, we can set $z = 1$ and apply the first property. For the third, we have 
      \begin{align}
        xy = 1 & \implies x^{-1} (x y) = x^{-1} 1 && \tag{multiplication is a function} \\
               & \implies (x^{-1} x) y = x^{-1} 1 && \tag{$\times$ is associative} \\
               & \implies 1 y = x^{-1} 1 && \tag{definition of multiplicative inverse} \\
               & \implies y = x^{-1} && \tag{definition of identity}
      \end{align}
      For the fourth, we simply see that $x^{-1}$ is a multiplicative inverse of both $x$ and $(x^{-1})^{-1}$ in the group $(\mathbb{F} \setminus \{0\}, \times)$, and since inverses are unique, they must be equal. 
    \end{proof}

    \begin{lemma}[Properties of Distribution]
      For any $x, y, z \in \mathbb{F}$, the field axioms satisfy 
      \begin{enumerate}
        \item $0 \cdot x = 0$.
        \item If $x \neq 0$ and $y \neq 0$, then $x y \neq 0$.
        \item $-1 \cdot x = -x$. 
        \item $(-x) y = - (xy) = x (-y)$. 
        \item $(-x) (-y) = xy$. 
      \end{enumerate}
    \end{lemma} 
    \begin{proof}
      For the first, note that 
      \begin{align}
        0 x & = (0 + 0) \cdot x = 0 x + 0x 
      \end{align}
      and subtracting $0x$ from both sides gives $0 = 0x$. For the second, we can claim that $xy \neq 0$ equivalently claiming that it will have an identity. Since $x, y \neq 0$, their inverses exists, and we claim that $(xy)^{-1} = y^{-1} x^{-1}$ is an inverse. We can see that by associativity, 
      \begin{equation}
        (y^{-1} x^{-1}) (xy) = y^{-1} (x^{-1} x) y = y^{-1} y = 1
      \end{equation} 
      For the third, we see that 
      \begin{equation}
        0 = 0 \cdot x = (1 + (-1)) \cdot x = 1 \cdot x + (-1) \cdot x = x + (-1) \cdot x 
      \end{equation}
      which implies that $-1 \cdot x$ is the additive inverse. The fourth follows immediately from the third by the associative property. For the fifth we can see that 
      \begin{align}
        (-x) (-y) & = (-1) x (-1) y && \tag{property 3} \\
                  & = (-1) (-1) x y && \tag{$\times$ is commutative} \\
                  & = -1 \cdot (-xy) && \tag{property 3} \\
                  & = -(-xy) && \tag{property 3} \\
                  & = xy && \tag{addition property 4}
      \end{align}
    \end{proof}

    Now that we've reviewed some fields, let's construct $\mathbb{Q}$ from $\mathbb{Z}$ and verify it's a field. 

    \begin{definition}[Rationals]
      Given the ordered ring of integers $(\mathbb{Z}, +_{\mathbb{Z}}, \times_{\mathbb{Z}}, \leq_{\mathbb{Z}})$ the \textbf{rational numbers} $(\mathbb{Q}, +_{\mathbb{Q}}, \times_{\mathbb{Q}})$ are defined as such. 
      \begin{enumerate}
        \item $\mathbb{Q}$ is the quotient space on $\mathbb{Z} \times \mathbb{Z} \setminus \{0\}$ with the equivalence relation $\sim$ 
        \begin{equation}
          (a, b) \sim (c, d) \iff a \times_{\mathbb{Z}} d = b \times_{\mathbb{Z}} c
        \end{equation} 
        We denote this class as $(a, b)$, where $b > 0$, since if $b < 0$, we know that $(-a, -b)$ are also in this order. 

        \item The additive and multiplicative identities are 
        \begin{equation}
          0_{\mathbb{Q}} \coloneqq (0_{\mathbb{Z}}, a), \;\;\; 1_{\mathbb{Q}} \coloneqq (a, a)
        \end{equation}

        \item Addition on $\mathbb{Q}$ is defined 
        \begin{equation}
          (a, b) +_{\mathbb{Q}} (c, d) \coloneqq \big( (a \times_{\mathbb{Z}} d) +_{\mathbb{Z}} (b \times_{\mathbb{Z}} c), b \times_{\mathbb{Z}} d \big) 
        \end{equation}

        \item The additive inverse is defined 
        \begin{equation}
          -(a, b) \coloneqq (-a, b)
        \end{equation}

        \item Multiplication on $\mathbb{Q}$ is defined 
        \begin{equation}
          (a, b) \times_{\mathbb{Q}} (c, d) \coloneqq \big( a \times_{\mathbb{Z}} c, b \times_{\mathbb{Z}} d \big)
        \end{equation} 

        \item The multiplicative inverse is defined 
        \begin{equation}
          (a, b)^{-1} \coloneqq (b, a)
        \end{equation}
      \end{enumerate}
    \end{definition}

    \begin{theorem}[Rationals are a Field]
      $\mathbb{Q}$ is a field. 
    \end{theorem} 
    \begin{proof}
      We do a few things. 
      \begin{enumerate}
        \item Verify the additive identity. 
        \begin{equation}
          (a, b) + (0, c) = (ac + 0b, bc) = (ac, bc) \sim (a, b)
        \end{equation}
        \item Verify the multiplicative identity. 
        \begin{equation}
          (a, b) \times (c, c) = (ac, bc) \sim (a, b)
        \end{equation}
        \item Additive inverse is actually an inverse. 
        \begin{equation}
          (a, b) + (-a, b) = (ab + (-ba), bb) = (0, bb) \sim (0, 1)
        \end{equation}
        \item Multiplicative inverse is actually an inverse. 
        \begin{equation}
          (a, b) \times (b, a) = (ab, ba) = (ab, ab) \sim (1, 1)
        \end{equation}
        \item Addition is commutative. 
        \begin{equation}
          (a, b) + (c, d) = (ad + bc, bd) = (cb + ad, bd) = (c, d) + (a, b)
        \end{equation}
        \item Addition is associative. 
        \begin{align}
          (a, b) + ((c, d) + (e, f)) & = (a, b) + (cf + de, df) \\
                                     & = (adf + bcf + bde, bdf) \\
                                     & = (ad + bc, bd) + (e, f) \\
                                     & = ((a, b) + (c, d)) + (e, f)
        \end{align}
        \item Multiplication is commutative. 
        \begin{equation}
          (a, b) \times (c, d) = (ac, bd) = (ca, db) = (c, d) \times (a, b)
        \end{equation}
        \item Multiplication is associative. 
        \begin{align}
          (a, b) \times ((c, d) \times (e, f)) & = (a, b) \times (ce, df) \\ 
                                               & = (ace, bdf) \\
                                               & = (ac, bd) \times (e, f) \\
                                               & = ((a, b) \times (c, d)) \times (e, f)
        \end{align}
        \item Multiplication distributes over addition. 
          \begin{align}
            (a, b) \times ((c, d) + (e, f)) & = (a, b) \times (c, d) + (a, b) \times (e, f) \\
                                            & = (ac, bd) + (ae, bf) \\
                                            & = (abcf + abde, b^2 df) \\
                                            & = (acf + ade, bdf)  
                                            & = (a, b) \times (cf + de, df)
          \end{align}
      \end{enumerate}
    \end{proof} 

    We have successfully defined the rationals, but now these are almost completely separate elements. We know that all integers are rational numbers, and so to show that the rationals are an extension of $\mathbb{Z}$ we want to identify a \textit{canonical injection} $\iota: \mathbb{Z} \rightarrow \mathbb{Q}$. This can't just be any canonical injection; it must preserve the algebraic structure between the two sets and must therefore be a \textit{ring homomorphism}. 

    \begin{theorem}[Canonical Injection of $\mathbb{Z}$ to $\mathbb{Q}$ is a Ring Homomorphism]
      Let us define the canonical injection $\iota: \mathbb{Z} \rightarrow \mathbb{Q}$ to be $\iota(a) = (a, 1)$. This is a ring homomorphism. 
    \end{theorem}
    \begin{proof} 
      We show a few things. 
      \begin{enumerate}
        \item Preservation of addition. 
          \begin{align}
            \iota(a) +_{\mathbb{Q}} \iota(b) & = (a, 1) +_{\mathbb{Q}} (b, 1) \\
                                             & = (1a +_{\mathbb{Z}} 1b, 1^2) \\
                                             & = (a +_{\mathbb{Z}} b, 1) \\
                                             & = \iota(a +_{\mathbb{Z}} b) 
          \end{align}
        \item Preservation of multiplication. 
          \begin{align}
            \iota(a) \times_{\mathbb{Q}} \iota(b) & = (a, 1) \times_{\mathbb{Q}} (b, 1) \\
                                                  & = (a \times_{\mathbb{Z}} b, 1^2) \\
                                                  & = (a \times_{\mathbb{Z}} b, 1) \\
                                                  & = \iota(a \times_{\mathbb{Z}} b, 1)
          \end{align}
        \item Preservation of multiplicative identity. 
          \begin{equation}
            \iota(1_{\mathbb{Z}}) = (1, 1) = 1_{\mathbb{Q}}
          \end{equation}
      \end{enumerate}
    \end{proof} 

  \subsubsection{Ordered Field Properties} 

    Great, so we have established that $\mathbb{Q}$ is a field. The next property we want to formalize is order. 

    \begin{definition}[Partial, Total/Linear Order]
      A \textbf{partial order} on a set $X$ is a relation $\leq$ satisfying. 
      \begin{enumerate}
        \item Reflexive: $x \leq x$ 
        \item Antisymmetric: $x \leq y, y \leq x \implies x = y$
        \item Transitivity: $x \leq y, y \leq z \implies x \leq z$
      \end{enumerate}
      Note that when we say $x \leq y$, this means "$x$ is related to $y$" (but does not necessarily mean that $y$ is related to $x$), or "$x$ is less than or equal to $y$." A set $X$ with a partial order is called a partially ordered set. 

      Additionally, given elements $x, y$ of partially order set $X$, if either $x \leq y$ or $y \leq x$, then $x$ and $y$ are \textbf{comparable}. Otherwise, they are \textbf{incomparable}. A partial order in which every pair of elements is comparable is called a \textbf{total order}, or \textbf{linear order}. Note that from this $\leq$ relation, we can similarly define 
      \begin{enumerate}
        \item $\leq$: less than or equal to 
        \item $\geq$: greater than or equal to 
        \item $<$: strictly less than ($x < y$ iff $x\leq y, x \neq y$)
        \item $>$: strictly greater than ($x > y$ iff $x \geq y, x \neq y$)
      \end{enumerate}
    \end{definition} 

    \begin{example}[Partially Ordered Sets]
      We list some examples of partially ordered sets. 
      \begin{enumerate}
        \item The real numbers ordered by the standard "less-than-or-equal" relation $\leq$ (totally ordered set as well). 
        \item The set of subsets of a given set $X$ ordered by inclusion. That is, the power set $2^X$ with the partial order $\subseteq$ is partially ordered. 
        \item The set of natural numbers equipped with the relation of divisibility. 
        \item The set of subspaces of a vector space ordered by inclusion. 
        \item For a partially ordered set $P$, the sequence space containing all sequences of elements from $P$, where sequence $a$ precedes sequence $b$ if every item in $a$ precedes the corresponding item in $b$. 
      \end{enumerate}
    \end{example} 

    We now want to define the natural ordering of the rationals. There are countless ways to do it, but I just take the difference and claim that it is greater than $0$. 

    \begin{theorem}[Order on Rationals]
      The order $\leq_{\mathbb{Q}}$ defined on the rationals as 
      \begin{equation}
        (a, b) \leq_{\mathbb{Q}} (c, d) \iff ad \leq_{\mathbb{Z}} bc
      \end{equation}
      is a total order. Remember that we have defined $b, d > 0$. 
    \end{theorem}
    \begin{proof}
      We prove the three properties. 
      \begin{enumerate}
        \item Reflexive. 
        \begin{equation}
          (a, b) \leq_{\mathbb{Q}} (a, b) \iff ab \leq_{\mathbb{Z}} ab
        \end{equation} 

        \item Antisymmetric. 
        \begin{align}
          (a, b) \leq_{\mathbb{Q}} (c, d) & \implies ad \leq_{\mathbb{Z}} bc
          (c, d) \leq_{\mathbb{Q}} (a, b) & \implies bc \leq_{\mathbb{Z}} ad
        \end{align} 
        This implies that both $ad = bc$, which by definition means that they are in the same equivalence class. 

        \item Transitivity. Assume that $(a, b) \leq (c, d)$ and $(c, d) \leq (e, f)$. Then, we notice that $b, d, f > 0$ and therefore by the ordered ring property\footnote{If $a \leq b$ and $0 \leq c$, then $ac \leq bc$.} of $\mathbb{Z}$, we have 
        \begin{align}
          (a, b) \leq_{\mathbb{Q}} (c, d) & \implies ad \leq_{\mathbb{Z}} bc \implies adf \leq_{\mathbb{Z}} bcf \\ 
          (c, d) \leq_{\mathbb{Q}} (e, f) & \implies cf \leq_{\mathbb{Z}} de \implies bcf \leq_{\mathbb{Z}} bde
        \end{align}
        Therefore from transitivity of the ordering on $\mathbb{Z}$ we have $adf \leq bde$. By the ordered ring property\footnote{If $a \leq b$, then $a + c \leq b + c$.}  we have $0 \leq bde - adf = d(be - af)$. But notice that $d > 0$ from our definition of rationals, and therefore it must be the case that $0 \leq be - af \implies af \leq_{\mathbb{Z}} be$, which by definition means $(a, b) \leq_{\mathbb{Q}} (e, f)$. 
      \end{enumerate}
    \end{proof} 

    As soon as we define an order the concept of extrema and bounds are well defined. Let's define them too. 

    \begin{definition}[Extrema, Bounds]
      Given a set $X$, 
      \begin{enumerate}
        \item $x \in X$ is a \textbf{maximum} $X$ if $y \leq x$ for all $y \in X$. 
        \item $x \in X$ is a \textbf{minimum} $X$ if $x \leq y$ for all $y \in X$. 
      \end{enumerate}
      Given a totally ordered set $X$ and some subset $S \subset X$. 
      \begin{enumerate}
        \item $x \in X$ is an \textbf{upper bound} of $S$ if $x \geq y$ for all $y \in S$. 
        \item $x \in X$ is a \textbf{lower bound} of $S$ if $x \leq y$ for all $y \in S$. 
        \item $x \in X$ is a \textbf{supremum}, or \textbf{least upper bound}, of $S$ if $x$ is the minimum of the set of all upper bounds of $S$. 
        \item $x \in X$ is a \textbf{infimum}, or \textbf{greatest lower bound}, of $S$ if $x$ is the maximum of the set of all lower bounds of $S$. 
      \end{enumerate}
      Note that we have defined max/min separately from the concept of bounds. You can define the maximum of a set with just knowing the set, but the bounds require \textit{both} some subset $S$ with respect to an enclosing set $X$.\footnote{For example, it makes sense to define the maximum of a set $S = [0, 1]$ by itself, but not an upper bound for it. If $X = \mathbb{Q}$, then the supremum is $1$, but if $X$ was the set of all irrationals, then this has no supremum.} Intuitively, the main difference between the supremum/infimum and maximum/minimum is that the supremum/infimum accounts for limit points of the subset $S$. 
    \end{definition}

    Note that given a set, we can really put whatever order we want on it. However, consider the field with the following order. 
    \begin{equation}
      \mathbb{F} = \{0, 1\}, \; 0 < 1
    \end{equation} 
    This does not behave well with respect to its operations because for example if we have $0 < 1$, then adding the same element to both sides should preserve the ordering. But this is not the case since $0 + 1 = 1 > 1 + 1 = 0$. While it may be easy to define an order, we would like it to be an ordered field. 

    \begin{definition}[Ordered Field]
      An \textbf{ordered field} is a field that has an order satisfying 
      \begin{enumerate}
        \item $y < z \implies x + y < x + z$ for all $x \in \mathbb{F}$. 
        \item $x > 0, y > 0 \implies xy > 0$. 
      \end{enumerate}
    \end{definition}

    \begin{theorem}[Properties]
      In an totally ordered field, 
      \begin{enumerate}
        \item $x > 0 \implies -x < 0$. 
        \item $x \neq 0 \implies x^2 > 0$. 
        \item If $x > 0$, then $y < z \implies xy < xz$. 
      \end{enumerate}
    \end{theorem} 
    \begin{proof}
      The first property is a single-liner 
      \begin{equation}
        0 < x \implies 0 + -x < x + -x \implies -x < 0 
      \end{equation}
      For the second property, it must be the case that $x > 0$ or $x < 0$. If $x > 0$, then by definition $x^2 > 0$. If $x < 0$, then 
      \begin{equation}
        x^2 = 1 \cdot x^2 = (-1)^2 \cdot x^2 = (-1 \cdot x)^2 = (-x)^2
      \end{equation}
      and since $-x > 0$ from the first property, we have $x^2 = (-x)^2 > 0$. For the third, we use the distributive property. 
      \begin{align}
        y < z & \implies 0 < z - y \\ 
              & \implies 0 = x 0 < x(z - y) = xz - xy \\
              & \implies xy < xz
      \end{align}
    \end{proof}

    As we have hinted, the rationals is an ordered field. 

    \begin{theorem}[Rationals are an Ordered Field]
      $\mathbb{Q}$ is an ordered field. 
    \end{theorem} 
    \begin{proof}
      We show that our defined order satisfies the definition. 
      \begin{enumerate}
        \item Assume that $y = (a, b) \leq (c, d) = z$. Let $x = (e, f)$. Then $x + y = (af + be, bf)$, $x + z = (cf + de, df)$. Therefore 
        \begin{align}
          (af + be) df & = adf^2 + bedf \\ 
                       & \leq bcf^2 + bedf \\
                       & = (cf + de) bf
        \end{align} 
        But $(af + be) df = (cf + de) bf$ is equivalent to saying $(af + be, bf) \leq_{\mathbb{Q}} (cf + de, df)$, i.e. $x + y \leq x + z$!  

        \item Let $x = (a, b), y = (c, d)$. Since $0 < x, 0 < y$, by construction this means that $0 < a, 0 < c$ (since $b, d > 0$ in the canonical rational form). By the ordered ring property of the integers, $0 < ac$. So 
        \begin{equation}
          0 < ac \iff 0 \cdot bd < ac \cdot 1 \iff (0, 1) < (ac, bd)  \iff 0_{\mathbb{Q}} < (a, c) \times_{\mathbb{Q}} (b, d) = x y
        \end{equation}
      \end{enumerate}
    \end{proof} 

    Not only is it an ordered field, but it also is consistent with the ordering on $\mathbb{Z}$! It's nice how all these properties seem to fit together. 

    \begin{theorem}[Preservation of Order]
      The canonical injection $\iota$ is an \textit{order homomorphism}. That is, for $a, b \in \mathbb{Z}$, 
      \begin{equation}
        a \leq_{\mathbb{Z}} b \iff \iota(a) \leq_{\mathbb{Q}} \iota(b)
      \end{equation}
    \end{theorem}
    \begin{proof} 
      \begin{align}
        a \leq_{\mathbb{Z}} b & \iff a \cdot 1 \leq_{\mathbb{Z}} b \cdot 1 \\
                              & \iff (a, 1) \leq_{\mathbb{Q}} (b, 1) \\
                              & \iff \iota(a) \leq_{\mathbb{Q}} \iota(b)
      \end{align}
    \end{proof}

    Note that an order can be used to generate an order topology, which we will define below. 

    \begin{definition}[Order Topology on $\mathbb{Q}$]
      The order topology on $\mathbb{Q}$ is the topology generated by the set $\mathscr{B}$ of all open intervals 
      \begin{equation}
        (a, b) \coloneqq \{ x \in \mathbb{Q} \mid a < x < b\}
      \end{equation}
    \end{definition}

    \begin{theorem}[Finite Fields]
      There are no finite ordered fields. 
    \end{theorem} 
    \begin{proof}
      Assume $\mathbb{F}$ is such an ordered field. It must be the case that $0, 1 \in \mathbb{F}$, with $0 < 1$. Therefore, we also have $0 + 1 < 1 + 1 \implies 1 < 1 + 1$. Repeating this we get 
      \begin{equation}
        0 < 1 < 1 + 1 < 1 + 1 + 1 < \ldots
      \end{equation}
      where these elements must be distinct (since only one of $>, <, =$ must be true for a totally ordered set). Since this can be done for a countably infinite number of times, $\mathbb{F}$ cannot be finite. 
    \end{proof}

  \subsubsection{Norm} 

    Note that we can also define a norm on the rationals with just the order and algebraic properties. 

    \begin{theorem}[Norm on $\mathbb{Q}$] 
      The following is indeed a norm on $\mathbb{Q}$. 
      \begin{equation}
        |x| \coloneqq \begin{cases} x & \text{ if } x \geq 0 \\ -x & \text{ if } x < 0 \end{cases}
      \end{equation} 
    \end{theorem} 

    It is well known that the metric induced by any norm is indeed a metric. Therefore we state the metric as a definition. 

    \begin{definition}[Metric on $\mathbb{Q}$]
      The Euclidean metric on $\mathbb{Q}$ is defined 
      \begin{equation}
        d(x, y) \coloneqq |x - y| = \begin{cases} x - y & \text{ if } x \geq y \\ y - x & \text{ if } x < y \end{cases}
      \end{equation}
    \end{definition}

    Thus we get to what we want: the induced topology of open balls. 
    Again, since we know from point-set topology that metric topologies are indeed topologies, we will state this as a definition rather than a theorem.  

    \begin{definition}[Open-Ball Topology on $\mathbb{Q}$]
      The Euclidean topology on $\mathbb{Q}$ is the topology generated by the set $\mathscr{B}$ of all open balls
      \begin{equation}
        B(x, r) \coloneqq \{ y \in \mathbb{Q} \mid |x - y| < r \}
      \end{equation} 
    \end{definition}

    Note that this is the same topology as the order topology. This should however be proved. 

    \begin{theorem}[Metric and Order Topologies on $\mathbb{Q}$]
      The metric and order topologies on $\mathbb{Q}$ are the same topologies. 
    \end{theorem}
    \begin{proof}
      
    \end{proof}

\subsection{The Reals}

    By constructing the topology of $\mathbb{Q}$ earlier, we can talk about convergence. The first question to ask (if you were the first person inventing the reals) is ``how do I know that there exists some other numbers at all?'' The first clue is trying to find the side length of a square with area $2$. As we see, this number is not rational. 

    \begin{theorem}[$\sqrt{2}$ is Not Rational]
      There exists no $x \in \mathbb{Q}$ s.t. $x^2 = 2$. 
    \end{theorem}
    \begin{proof}
      
    \end{proof} 

    We can ``imagine'' that a square with area $2$ certainly exists, but the fact that its side length is undefined is certainly unsettling. I don't know about you, but I would try to ``invent'' $\sqrt{2}$. We can maybe do this in multiple ways. 
    \begin{enumerate}
      \item I write out the decimal expansion one by one, which gives our first exposure to sequences. 
      \begin{equation}
        1, 1.4, 1.41, 1.414, \ldots
      \end{equation} 
      It is clear that on $\mathbb{Q}$, this sequence does not converge. Our intuition tells that that if the terms get closer and closer to each other, they must be getting closer and closer to \textit{something}, though that something is not in $\mathbb{Q}$. This motivates the definition for \textit{Cauchy completeness}. 

      \item I would write out maybe some nested intervals so that $\sqrt{2}$ \textit{must}  lie within each interval. 
      \begin{equation}
        [1, 2] \supset [1.4, 1.5] \supset [1.41, 1.42] \supset \ldots 
      \end{equation}
      This motivates the definition of \textit{nested-interval completeness}. 

      \item I would define the set of all rationals such that $x^2 < 2$, and try to define $\sqrt{2}$ as the max or supremum of this set. We will quickly find that neither the max nor the supremum exists in $\mathbb{Q}$, and this motivates the definition for \textit{Dedekind completeness}. 
    \end{enumerate}

    All three of these methods points at the same intuition that there should not be any "gaps" or "missing points" in the set that we will construct to be $\mathbb{R}$. This contrasts with the rational numbers, whose corresponding number line has a "gap" at each irrational value. 

  \subsubsection{Dedekind Completeness} 

    \begin{definition}[Dedekind Cut] 
      A \textbf{Dedekind cut} is a partition of the rationals $\mathbb{Q} = A \sqcup A^\prime$ satisfying the three properties.\footnote{This can really be defined for any totally ordered set. } 
      \begin{enumerate}
        \item $A \neq \emptyset$ and $A \neq \mathbb{Q}$.\footnote{By relaxing this property, we can actually complete $\mathbb{Q}$ to the extended real number line. }
        \item $x < y$ for all $x \in A, y \in A^\prime$. 
        \item The maximum element of $A$ does not exist in $\mathbb{Q}$. 
      \end{enumerate}
      The minimum of $A^\prime$ may exist in $\mathbb{Q}$, and if it does, the cut is said to be \textbf{generated} by $\min A^\prime$. 
    \end{definition}

    Note that in $\mathbb{Q}$, there will be two types of cuts: 
    \begin{enumerate}
      \item ones that are generated by rational numbers, such as 
      \begin{equation}
        A = \{x \in \mathbb{Q} \mid x < 2/3 \}, A^\prime = \{ x \in \mathbb{Q} \mid x \geq 2/3 \} 
      \end{equation}
      \item and the ones that are not 
      \begin{equation}
        A = \{x \in \mathbb{Q} \mid x^2 < 2 \}, A^\prime = \{x \in \mathbb{Q} \mid x^2 \geq 2 \}
      \end{equation}
    \end{enumerate}
    We can intuitively see that the set of all Dedekind cuts $(A, A^\prime)$ will ``extend'' the rationals into a bigger set. We can then define some operations and an order to construct this into an ordered field, and finally it will have the property that we call ``completeness.''

    \begin{definition}[Dedekind Completeness]
      A totally ordered algebraic field $\mathbb{F}$ is \textbf{complete} if every Dedekind cut of $\mathbb{F}$ is generated by an element of $\mathbb{F}$. 
    \end{definition} 

    \begin{theorem}
      $\mathbb{Q}$ is not Dedekind-complete. 
    \end{theorem}
    \begin{proof}
      The counter-example is given above for the cut 
      \begin{equation}
        A = \{x \in \mathbb{Q} \mid x^2 < 2 \}, A^\prime = \{x \in \mathbb{Q} \mid x^2 \geq 2 \}
      \end{equation}
    \end{proof} 

    Now we have the tools to define the reals, giving us the beefy theorem. 

    \begin{theorem}[Reals as the Dedekind-Completion of Rationals]
      Let $\mathbb{R}$ be the set of all Dedekind cuts $(A, A^\prime)$ of $\mathbb{Q}$ of $\mathbb{Q}$. For convenience we can uniquely represent $(A, A^\prime)$ with just $A$ since $A^\prime = \mathbb{Q} \setminus A$. By doing this we can intuitively think of a real number as being represented by the set of all smaller rational numbers. Let $A, B$ be two Dedekind cuts. Then, we define the following order and operations. 
      \begin{enumerate}
        \item \textit{Order}. $A \leq_{\mathbb{R}} B \iff A \subset B$. 
        \item \textit{Addition}. $A +_{\mathbb{R}} B \coloneqq \{ a +_{\mathbb{Q}} b \mid a \in A, b \in B \}$. 
        \item \textit{Additive Identity}. $0_{\mathbb{R}} \coloneqq \{x \in \mathbb{Q} \mid x < 0 \}$. 
        \item \textit{Additive Inverse}. $-B \coloneqq \{ a - b \mid a < 0 , b \in (\mathbb{Q} \setminus B) \}$.
        \item \textit{Multiplication}. If $A, B \geq 0$, then we define $A \times_{\mathbb{R}} B \coloneqq \{ a \times_{\mathbb{Q}} b \mid a \in A, b \in B, a, b \geq 0\} \cup 0_{\mathbb{R}}$. If $A$ or $B$ is negative, then we use the identity $A \times B = -(A \times_{\mathbb{R}} -B) = -(-A \times_{\mathbb{R}} B) = (-A \times_{\mathbb{R}} -B)$ to convert $A, B$ to both positives and apply the previous definition. 
        \item \textit{Multiplicative Identity}. $1_{\mathbb{R}} = \{x \in \mathbb{Q} \mid x < 1 \}$. 
        \item \textit{Multiplicative Inverse}. If $B > 0$, $B^{-1} \coloneqq \{ a \times_{\mathbb{Q}} b^{-1} \mid a \in 1_{\mathbb{R}}, b \in (\mathbb{Q} \setminus B) \}$. If $B$ is negative, then we compute $B^{-1} = -((-B)^{-1})$ by first converting to a positive number and then applying the definition above. 
      \end{enumerate}
      We claim that $(\mathbb{R}, +_{\mathbb{R}}, \times_{\mathbb{R}}, \leq_{\mathbb{R}})$ is a totally ordered field, and the canonical injection $\iota: \mathbb{Q} \rightarrow \mathbb{R}$ defined 
      \begin{equation}
        \iota(q) = \{x \in \mathbb{Q} \mid x < q \}
      \end{equation}
      is an ordered field isomorphism. Finally, by construction $\mathbb{R}$ is Dedekind-complete. 
    \end{theorem} 

    \begin{definition}[Least Upper Bound Property]
      A totally ordered algebraic field $\mathbb{F}$ (must it be a field?) is complete if every nonempty set of $F$ having an upper bound must have a least upper bound (supremum) in $F$. 
    \end{definition}

    \begin{theorem}[Dedekind Completeness Equals Least-Upper-Bound Property]
      Dedekind completeness is equivalent to the least upper bound property. 
    \end{theorem}
    \begin{proof}
      
    \end{proof}

    \begin{definition}[Archimidean Principle]
      An ordered ring $(X, +, \cdot, \leq)$ that embeds the naturals $\mathbb{N}$\footnote{as in, there exists an ordered ring homomorphism $\iota: \mathbb{N} \rightarrow X$} is said to obey the \textbf{Archimedean principle} if given any $x, y \in X$ s.t. $x, y > 0$, there exists an $n \in \mathbb{N}$ s.t. $\iota(n) \cdot x > y$. Usually, we don't care about the canonical injection and write $nx > y$. 
    \end{definition}

    By the canonical injections $\mathbb{N} \rightarrow \mathbb{Z} \rightarrow \mathbb{Q} \rightarrow \mathbb{R}$, we can talk about whether this set has the Archimedean property. In fact Dedekind completeness does imply it. 

    \begin{theorem}
      $\mathbb{R}$ satisfies the Archimedean principle. 
    \end{theorem}
    \begin{proof}
      Assume that this property doesn't hold. Then for any fixed $x$, $nx < y$ for all $n \in \mathbb{N}$. Consider the set 
      \begin{equation}
        A = \bigcup_{n \in \mathbb{N}} (-\infty, nx), \;\;\; B = \mathbb{R} \setminus A
      \end{equation}
      $A$ by definition is nonempty, and $B$ is nonempty since it contains $y$. Then, we can show that $a \in A, b \in B \implies a < b$ using proof by contradiction. Assume that there exists $a^\prime \in A, b^\prime \in B$ s.t. $a^\prime > b^\prime$. Since $a^\prime \in A$, there exists a $n^\prime \in \mathbb{N}$ s.t. $a^\prime \in (-\infty, n^\prime x) \iff a^\prime < n^\prime x$. But by transitivity of order, this means $b^\prime < n^\prime x \iff b^\prime \in (-\infty, n^\prime x) \implies b^\prime \in A$. 

      Going back to the main proof, we see that $A$ is upper bounded by $y$, and so by the least upper bound property it has a supremum $z = \sup{A}$. 
      \begin{enumerate}
        \item If $z \in A$, then by the induction principle\footnote{Note that $\mathbb{N}$ is defined recursively as $1 \in \mathbb{N}$ and if $n \in \mathbb{N}$, then $n+1 \in \mathbb{N}$. } $z + x \in A$, contradicting that $z$ is an upper bound. 
        \item If $z \not\in A$, then by the induction principle\footnote{The contrapositive of the recursive definition of $\mathbb{N}$ is: if $n \not\in \mathbb{N}$, then $n-1 \not\in \mathbb{N}$.} $z-x \not\in A \implies z-x \in B$. Since every element of $B$ upper bounds $A$ and since $x > 0$, this means that $z-x < z$ is a smaller upper bound of $A$, contradicting that $z$ is a least upper bound. 
      \end{enumerate}
      Therefore, it must be the case that $nx > y$ for some $n \in \mathbb{N}$. 
    \end{proof}

  \subsubsection{Cauchy Completeness} 

    \begin{definition}[Cauchy Sequence]
      A sequence $a_n$ in a metric space $(X, d)$ is a \textbf{Cauchy sequence} if for every $\epsilon > 0$, there exists an $N$ s.t. 
      \begin{equation}
        d(a_i, a_j) < \epsilon
      \end{equation}
      for every $i, j > N$. We call this \textbf{Cauchy convergence}. 
    \end{definition}

    Note that it is not sufficient to say that a sequence is Cauchy by claiming that each term becomes arbitrarily close to the preceding term. That is, 
    \begin{equation}
      \lim_{n \rightarrow \infty} |x_{n+1} - x_{n}| = 0
    \end{equation}
    For example, look at the sequence 
    \begin{equation}
      a_n = \sqrt{n} \implies a_{n+1} - a_{n} = \frac{1}{\sqrt{n+1} + \sqrt{n}} < \frac{1}{2\sqrt{n}}
    \end{equation}
    However, it is clear that $a_n$ gets arbitrarily large, meaning that a finite interval can contain at most a finite number of terms in $\{a_n\}$. 

    It is trivial that convergence implies Cauchy convergence, but the other direction is not true. Therefore, we would like to work in a space where these two are equivalent, and this is called completeness. 

    \begin{definition}[Cauchy Completeness]
      A metric space $(X, d)$ is complete if every Cauchy sequence in that space converges to an element in $X$. 
    \end{definition} 

    \begin{theorem}
      $\mathbb{Q}$ is not Cauchy-complete. 
    \end{theorem}
    \begin{proof}
      Let $a_n$ be the largest number $x$ up to the $n$th decimal expansion such that $x^2$ does not exceed $2$. The first few terms are 
      \begin{equation}
        1.4, 1.41, 1.414, \ldots
      \end{equation}
    \end{proof}

    Therefore, we can construct the reals as equivalence classes over Cauchy sequences. Rather than using the order, we take advantage of the metric. 

    \begin{theorem}[Reals as the Cauchy-Completion of the Rationals]
      Let $\mathbb{R}$ be the quotient space of all Cauchy sequences $(x_n)$ of rational numbers with the equivalence relation $(x_n) = (y_n)$ iff their difference tends to $0$.\footnote{This equivalence class reflects that the same real number can be approximated in many different sequences. In fact, this shows \textit{by definition} that $1, 1, \ldots$ and $0.9, 0.99, 0.999, \ldots$ are the same number!} That is, for every rational $\epsilon > 0$, there exists an integer $N$ s.t. for all naturals $n > N$, $|x_n - y_n| < \epsilon$. 
      \begin{enumerate}
        \item \textit{Order}. $(x_n) \leq_{\mathbb{R}} (y_n)$ iff $x = y$ or there exists $N \in \mathbb{N}$ s.t. $x_n \leq_{\mathbb{Q}} y_n$ for all $n > N$. 
        \item \textit{Addition}. $(x_n) + (y_n) \coloneqq (x_n + y_n)$. 
        \item \textit{Additive Identity}. $0_{\mathbb{R}} \coloneqq (0_{\mathbb{Q}})$. 
        \item \textit{Additive Inverse}. $-(x_n) \coloneqq (-x_n)$. 
        \item \textit{Multiplication}. $(x_n) \times_{\mathbb{R}} (y_n) = (x_n \times_{\mathbb{Q}} y_n)$. 
        \item \textit{Multiplicative Identity}. $1_{\mathbb{R}} \coloneqq (1)$. 
        \item \textit{Multiplicative Inverse}. $(x_n)^{-1} \coloneqq (x_n^{-1})$. 
      \end{enumerate}
      We claim that $(\mathbb{R}, +_{\mathbb{R}}, \times_{\mathbb{R}}, \leq_{\mathbb{R}})$ is a totally ordered field, and the canonical injection $\iota: \mathbb{Q} \rightarrow \mathbb{R}$ defined 
      \begin{equation}
        \iota(q) = (q)
      \end{equation} 
      is an ordered field isomorphism. Finally, by construction $\mathbb{R}$ is Cauchy-complete. 
    \end{theorem}

  \subsubsection{Nested Intervals Completeness} 

    The final way we prove is using nested-intervals completeness.  

    \begin{definition}[Nested Interval Completeness, Cantor's Intersection Theorem]
      Let $F$ be a totally ordered algebraic field. Let $I_n= [a_n, b_n]$ ($a_n < b_n$) be a sequence of closed intervals, and suppose that these intervals are nested in the sense that
      \[I_1 \supset I_2 \supset I_3 \supset \ldots\]
      where 
      \[\lim_{n \rightarrow + \infty} b_n - a_n = 0\]
      $F$ is complete if the intersection of all of these intervals $I_n$ contains exactly one point. That is, 
      \[\bigcup_{n=1}^\infty I_n \in F\]
    \end{definition}

    Note that defining nested intervals requires only an ordered field. One may look at this and try to ask if this is a specific instance of the following conjecture: The intersection of a nested sequence of nonempty closed sets in a topological space has exactly 1 point. This claim may not even make sense, actually. If we define nested in terms of proper subsets, then for a finite topological space a sequence cannot exist since we will run out of open sets and so this claim is vacuously true and false. If we allow $S_n = S_{n+1}$ then we can just select $X \supset X \supset \ldots$, which is obviously not true. However, a slightly weaker claim is that every proper nested non-empty closed sets has a non-empty intersection is a consequence of compactness. 

    \begin{theorem}
      $\mathbb{Q}$ is not nested-interval complete. 
    \end{theorem}
    \begin{proof}
      Consider the intervals $[a_i, b_i]$ where $a_i$ is the largest number $x$ up to the $n$th decimal expansion such that $x^2$ does not exceed $2$, and $b_i$ is the smallest number $x$ up to the $n$th decimal expansion such that $x^2$ is not smaller than $2$. The first few terms are 
      \begin{equation}
        [1.4, 1.5], [1.41, 1.42], [1.414, 1.415], \ldots
      \end{equation}
    \end{proof}

    Therefore, we can complete $\mathbb{Q}$. It turns out that this is equivalent to the construction using Dedekind cuts, and by definition this new set is nested interval complete. However, like Cauchy completeness, this actually does not imply the Archimedean property. 

  \subsubsection{Properties of the Real Line} 

    Now that we have completed it, we can define the real numbers. 

    \begin{definition}[The Real Numbers]
      The \textbf{set of real numbers}, denoted $\mathbb{R}$, is a totally ordered complete Archimedean field. 
    \end{definition} 

    It seems that the real numbers is \textit{any} set that satisfies the definition above. Does this mean that there are multiple real number lines? 

    \begin{example}[Multiple Reals?]
      For example, let us construct three distinct sets satisfying these axioms: 
      \begin{enumerate}
        \item A line $\mathbb{L}$ with $+$ associated with the translation of $\mathbb{L}$ along itself and $\cdot$ associated with the "stretching/compressing" of the line around the additive origin $0$. 
        \item An uncountable list of numbers with possibly infinite decimal points, known as the decimal number system. 
        \begin{equation}
          \ldots, -2.583\ldots, \ldots , 0, \ldots, 1.2343\ldots, \ldots, \sqrt{2}, \ldots
        \end{equation}
        \item A circle with a point removed, with addition and multiplication defined similarly as the line. 
      \end{enumerate}
    \end{example}

    We will show that there is only one set, up to isomorphism, that satisfies all these properties. 

    \begin{theorem}[Uniqueness]
      $\mathbb{R}$ is unique up to field isomorphism. That is, if two individuals construct two ordered complete Archimedean fields $\mathbb{R}_A$ and $\mathbb{R}_B$, then 
      \begin{equation}
        \mathbb{R}_A \simeq \mathbb{R}_B
      \end{equation}
    \end{theorem}  
    \begin{proof}
      The proof is actually much longer than I expected, so I draw a general outline.\footnote{Followed from \href{https://math.ucr.edu/~res/math205A/uniqreals.pdf}{here}.} We want to show how to construct an isomorphism $f: \mathbb{R}_A \rightarrow \mathbb{R}_B$. 
      \begin{enumerate}
        \item Realize that there are unique embeddings of $\mathbb{N}$ in $\mathbb{R}_A$ and $\mathbb{R}_B$ that preserve the inductive principle, the order, closure of addition, and closure of multiplication, the additive identity, and the multiplicative identity. Call these ordered doubly-monoid (since it's a monoid w.r.t. $+$ and $\times$) homomorphisms $\iota_A, \iota_B$. 
        \item Construct an isomorphism $f_1: \iota_A(\mathbb{N}) \rightarrow \iota_B(\mathbb{N})$ that preserves the inductive principle, order, addition, and multiplication. This is easy to do by just constructing $f_1 = \iota_B \circ \iota_A^{-1}$. 
        \item Extend $f_1$ to the ordered ring isomorphism $f_2$ by explicitly defining what it means to map additive inverses, i.e. negative numbers. 
        \item Extend $f_2$ to the ordered field isomorphism $f_3$ by explicitly defining what it means to map multiplicative inverses, i.e. reciprocals. 
        \item Extend $f_3$ to the ordered field isomorphism on the entire domain $\mathbb{R}_A$ and codomain $\mathbb{R}_B$. There is no additional operations that we need to support, but we should explicitly show that this is both injective and surjective, which completes our proof. 
      \end{enumerate}
    \end{proof}

    \begin{corollary}
      Let $\mathbb{R}_D, \mathbb{R}_C$ be the Dedekind and Cauchy completion of $\mathbb{Q}$. Then $\mathbb{R}_D \simeq \mathbb{R}_C$. 
    \end{corollary}

    The second new property is that the reals are uncountable. 

    \begin{theorem}[Cantor's Diagonalization]
      The real numbers are uncountable. 
    \end{theorem} 
    \begin{proof}
      We 
    \end{proof}

    Provide examples of ordered, Cauchy-complete fields that are not Archimedean.  

  \subsubsection{Roots and Exponential Maps} 

    The real numbers also guarantee the existence of $n$th roots. 

    \begin{theorem}[Existence of Nth Roots]
      Is this true for any complete field? 
    \end{theorem}

    For the integers, we have defined $x^y$ as the repeated multiplication of $y$. Since we have proven the unique existence of $n$th roots, we can extend this to the rationals. 

    \begin{theorem}[Rational Exponential Function]
      
    \end{theorem} 

    It turns out that this is a homomorphism. 

    \begin{corollary}[Rational Exponential Function is a Homomorphism]
      The rational exponential function is a homomorphism. That is, given $p, q \in \mathbb{Q}$ and $x \in \mathbb{R}$, 
      \begin{equation}
        x^{p + q} = x^p \cdot x^q
      \end{equation}
    \end{corollary}

    With rational exponents defined, we can use the least upper bound property to define a consistent extension of a real exponent. 

    \begin{corollary}[Real Exponential Function]
      Given $x\in \mathbb{R}$, we define 
      \begin{equation}
        B(x) \coloneqq \{ x^q \in \mathbb{R} \mid q \in \mathbb{Q}, \; q \leq x \}
      \end{equation}
      We claim that given $r \in \mathbb{R}$, 
      \begin{equation}
        x^r \coloneqq \sup B(r)
      \end{equation}
      is well-defined and is a homomorphism extension of the rational exponential function. 
    \end{corollary}

    Furthermore, this is an isomorphism, and the inverse is defined. Let's define this analytically.\footnote{}

    \begin{theorem}[Logarithm]
      
    \end{theorem}

\subsection{The Extended Reals and Hyperreals}

  Great! We have officially constructed the reals, and we can finally feel satisfied about defining metrics, norms, and inner products as mappings to the codomain $\mathbb{R}$. Now let's make the concept of infinite numbers a bit more rigorous. In short, what we do is just add the numbers $\pm \infty$ to $\mathbb{R}$, which we call the extended reals, and try and extend the properties from $\mathbb{R}$ to the extended reals. We will see that not all properties can be transferred. 

  \begin{theorem}[Extended Real Number Line]
    The \textbf{extended real number line} is the set $\overline{\mathbb{R}} \coloneqq \mathbb{R} \cup \{-\infty, +\infty\}$. We define the following operations. 
    \begin{enumerate}
      \item \textit{Order}. $-\infty \leq x$ and $x \leq +\infty$ for all $x \in \overline{\mathbb{R}}$. 
      \item \textit{Addition}. $+\infty - \infty = 0$. $x + \infty = +\infty$ and $x - \infty = -\infty$ for all $x \in \mathbb{R}$. 
      \item \textit{Multiplication}. 
      \begin{equation}
        x \times \infty = \begin{cases} +\infty & \text{ if } x > 0 \\ 0 & \text{ if } x = 0 \\ -\infty & \text{ if } x < 0 \end{cases}
      \end{equation}
    \end{enumerate}
    It turns out that this is still Dedekind-complete, which is nice. Unfortunately we lose a lot of structure. 
    \begin{enumerate}
      \item this is not even a field since the multiplicative inverse of $\pm \infty$ is not defined. 
      \item the Archimedean principle does not hold 
      \item we cannot define a metric nor a norm. 
      \item we can define the order topology, however. 
    \end{enumerate}
  \end{theorem} 

  The loss of the field property is quite bad, and we might want to recover this. Therefore, we can add more elements that serve to be the multiplicative inverse of infinity. We call these inverses \textit{infinitesimals} and the new set the \textit{hyperreal numbers}. 

  \begin{theorem}[Hyperreals]
    The \textbf{hyperreals} 
  \end{theorem}

  In fact, when Newton first invented calculus, the hyperreals were what he worked with, and you can surprisingly build a good chunk of calculus with this. Even though this is a dead topic at this point, a lot of modern notation is based off of this number system, so it's good to see how it works. For example, when we write the integral 
  \begin{equation}
    \int f(x) \,dx
  \end{equation} 
  we are saying that we are taking the uncountable sum of the terms $f(x) \,dx$, the multiplication of the real number $f(x)$ and the infinitesimal number $dx$ living in the hyperreals. Unfortunately, we cannot fully construct a rigorous theory of calculus with only infinitesimals. However, in practice (especially physics) people tend to manipulate and do algebra with infinitesimals, so having a good foundation on what you can and cannot do with them is practical. While the focus won't be on \textit{smooth infinitesimal analysis (SIA)}, I will include some alternate constructions later purely with infinitesimals. 

\subsection{Complex Numbers} 

  The next field that will be particularly important is the complex numbers. It is straightforward to construct $\mathbb{C}$, but let's motivate this for a minute. 

  \begin{example}[Polynomial Roots]
    The roots of the polynomial 
    \begin{equation}
      f(x) = x^2 + 1
    \end{equation}
    does not exist in $\mathbb{R}$. 
  \end{example} 

  Therefore, we would like to construct a new space that contains all possible roots for all possible polynomials with real coefficients. We call this $\mathbb{C}$. Clearly, by constructing polynomials of the form $x^2 - r^2$ for some $r \in \mathbb{R}$, we know that $\mathbb{R} \subset \mathbb{C}$. Therefore, we want to create a further extension of $\mathbb{R}$, along with some canonical injection $\iota: \mathbb{R} \rightarrow \mathbb{C}$ that is also a field homomorphism. It turns out that once we construct this field, there is no possible way that we can make it an ordered field. However, the norm extends naturally into $\mathbb{C}$ such that $\iota$ is isometric. Finally, we can define a new operator called \textit{conjugation} that gives us additional structure. 

  This is not the only way to construct the complex plane however. Rather than defining all these from scratch, we could just define the addition operations with an isometric vector space isomorphism from $\mathbb{R}^2$ to $\mathbb{C}$ actually, and then define multiplication. Another way is to start again with $\mathbb{Q} \times \mathbb{Q}$, define a norm on it, complete it, and finally define the addition and multiplication operations that satisfy the field property.   

  \begin{theorem}[Construction of the Complex Numbers]
    Let $\mathbb{C}$ be defined as the space $\mathbb{R} \times \mathbb{R}$ with the following operations. 
    \begin{enumerate}
      \item \textit{Addition}. $x = (a, b), y = (c, d) \implies x +_{\mathbb{C}} y = (a + c, b + d)$. 
      \item \textit{Additive Identity}. $0_{\mathbb{C}} = (0, 0)$. 
      \item \textit{Additive Inverse}. $x = (a, b) \implies -x = (-a, -b)$. 
      \item \textit{Multiplication}. $x = (a, b), y = (c, d) \implies x \times_{\mathbb{C}} y = (ac - bd, ad + bc)$. 
      \item \textit{Multiplicative Identity}. $1_{\mathbb{C}} = (1, 0)$. 
      \item \textit{Multiplicative Inverse}. 
      \begin{equation}
        x = (a, b) \implies x^{-1} = \bigg( \frac{a}{a^2 + b^2}, \frac{-b}{a^2 + b^2} \bigg)
      \end{equation}
    \end{enumerate}
    Our first claim is that $(\mathbb{C}, +_{\mathbb{C}}, \times_{\mathbb{C}})$ is a field. Furthermore, we define the additional structures
    \begin{enumerate}
      \item \textit{Conjugate}. $x = (a, b) \implies \overline{x} = (a, -b)$. 
      \item \textit{Norm}. $|x|_{\mathbb{C}} = x \times_{\mathbb{C}} \overline{x} = a^2 + b^2$. 
      \item \textit{Metric}. This is the norm-induced metric. $d_{\mathbb{C}}(x, y) = |x - y|_{\mathbb{C}}$. 
      \item \textit{Topology}. This is the metric-induced topology generated by the open balls $B(x, r) \coloneqq \{y \in \mathbb{C} | d(x, y) < r\}$, where $x \in \mathbb{C}, r \in \mathbb{R}$. 
    \end{enumerate} 
    Our second claim is that the canonical injection $\iota: \mathbb{R} \rightarrow \mathbb{C}$ defined 
    \begin{equation}
      \iota(r) = (r, 0)
    \end{equation}
    is an isometric field isomorphism. Our third claim is that $\mathbb{C}$ is Cauchy-complete with respect to this metric. 
  \end{theorem} 

  Note that we do not talk about order $\mathbb{C}$, and so the concepts of Dedekind completeness, least upper bound properties, or Archimedean principle is meaningless in the complex plane. 

  \begin{definition}[Imaginary Number] 
    Let us denote $i = (0, 1)$ which we call the \textbf{imaginary number}, which has the property that $i^2 = 1$. With this notation, we can see through abuse of notation that 
    \begin{equation}
      (a, b) = (a, 0) + (0, b) = (a, 0) + (b, 0) (0, 1) = a + bi
    \end{equation} 
    Therefore, we generally write complex numbers as $z = a + bi$, and we define the real and imaginary components as $\re(z)$ and $\im(z)$, respectively. 
  \end{definition}

  Unfortunately, we lose the ordering. 

  \begin{theorem}[Order on Complex Plane]
    There exists no order on $\mathbb{C}$ that makes it a totally ordered field.
  \end{theorem}
  \begin{proof}
    We attempt to construct an order on $i$ and $0$ in $\mathbb{C}$. 
    \begin{enumerate}
      \item If $i = 0$, then $i^4 = 0 \cdot i^3 \implies 1 = 0$, which contradicts that $0 < 1$. 
      \item If $i \neq 0$, then $i^2 > 0$ from the field axioms, and so $-1 > 0$. But this also means that $1 = i^4 > 0$. This contradicts the ordered field property that $x > 0 \iff -x < 0$. 
    \end{enumerate}
    Therefore $\mathbb{C}$ cannot be turned into an ordered field. 
  \end{proof}

  \begin{theorem}[Properties of Conjugation]
    Conjugation is an isometric field automorphism of $\mathbb{C}$. 
  \end{theorem}
  \begin{proof}
    
  \end{proof}

  \subsubsection{Properties of the Complex Numbers} 

\subsection{Dual Numbers}

  Another similar number system. 

\subsection{Euclidean Space} 

  Congratulations! We have rigorously constructed both the reals and complex numbers, and this becomes the cornerstone to construct other fundamental sets. Now we consider spaces of the form $\mathbb{R}^n$ or $\mathbb{C}^n$, which we call \textit{Euclidean spaces}, and construct them. This is actually quite easy since we understand linear algebra. 

  \begin{definition}[Convex Sets]
    A set $S$ is convex if for every point $x, y \in S$, the point 
    \begin{equation}
      z = t x + (1 - t) y \in S
    \end{equation}
    where $0 \leq t \leq 1$. 
  \end{definition}

\subsection{Exercises} 

  \begin{exercise}[Math 531 Spring 2025, PS2.1]
    Prove that the set of all matrices of the form:
    \begin{equation}
      \begin{bmatrix}
        a & -b \\
        b & a
      \end{bmatrix},
    \end{equation}
    with $a,b \in \mathbb{R}$ forms a field with the usual sum and product operations of
    matrices. What does this field resemble? Give extensions to $3 \times 3$ and
    $4 \times 4$ matrices.
  \end{exercise}
  \begin{solution}
    
  \end{solution}

  \begin{exercise}[Math 531 Spring 2025, PS2.2]
    Why can't the field of complex numbers (with its usual operations) be
    made into an ordered field?
  \end{exercise}
  \begin{solution}
    Solution is shown as theorem. 
  \end{solution}

  \begin{exercise}[Math 531 Spring 2025, PS2.3]
    Prove there are no finite ordered fields.
  \end{exercise}
  \begin{solution}
    Solution is shown as theorem. 
  \end{solution}

  \begin{exercise}[Math 531 Spring 2025, PS2.4]
    Prove that if $x$ and $n$ are natural numbers, then
    \begin{equation}
      x^n - 1 = (x-1)(1 + x + x^2 + ... + x^{n-1}).
    \end{equation}
  \end{exercise}
  \begin{solution}
    We use the commutative addition and multiplication, plus distributive property in $\mathbb{Z}$. 
    \begin{align}
      (x - 1) \bigg( \sum_{i=0}^{n-1} x^i \bigg) & = x \sum_{i=0}^{n-1} x^i - \sum_{i=0}^{n-1} x_i \\
                                                 & = \sum_{i=1}^n x^i - \sum_{i=0}^{n-1} x^i \\
                                                 & = x^n + \sum_{i=1}^{n-1} x^i - \sum_{i=1}^{n-1} x^i - 1 \\
                                                 & = x^n - 1
    \end{align}
  \end{solution}

  \begin{exercise}[Math 531 Spring 2025, PS2.6]
    Assume that $S$ is a set with exactly $n$ elements. Assume that $T : S \to S$.
    Prove that there exists some $x \in S$ so that
    \begin{equation}
      T^j(x) = x,
    \end{equation}
    for some $j \in \{1,2,...,n\}$. Here $T^j$ means the composition of $T$ with itself
    $j$-times.
  \end{exercise}
  \begin{solution}
    
  \end{solution}

  \begin{exercise}[Math 531 Spring 2025, PS2.7]
    Prove that there is no $q \in \mathbb{Q}$ for which
    \begin{equation}
      q^2 + q = 4.
    \end{equation}
  \end{exercise}
  \begin{solution}
    
  \end{solution}

  \begin{exercise}[Math 531 Spring 2025, PS2.8]
    Let $X$ be an ordered set with the least upper bound property. Prove that
    $X$ has the greatest lower bound property.
  \end{exercise}
  \begin{solution}
    
  \end{solution}

  \begin{exercise}[Math 531 Spring 2025, PS2.9]
    Prove that if $x, y \in \mathbb{Q}$ we have that
    \begin{equation}
      ||x| - |y|| \leq |x - y|.
    \end{equation}
  \end{exercise}
  \begin{solution}
    
  \end{solution}

  \begin{exercise}[Rudin 1.1]
    If $r$ is rational ($r \neq 0$) and $x$ is irrational, prove that $r + x$ and $rx$ are irrational. 
  \end{exercise}
  \begin{solution}
    If we assume that $r x = t$ and $r + x = s$ are rational, then this violates the field axioms of $\mathbb{Q}$ since then $x = t r^{-1}$ and $x = s + (-r)$ are rational. 
  \end{solution}

  \begin{exercise}[Rudin 1.2]
    Prove that there is no rational number whose square is $12$. 
  \end{exercise}
  \begin{solution}
    Assume that there exists a number $p/q$ such that $p$ and $q$ are both not even. Then, 
    \begin{equation}
      \bigg( \frac{p}{q} \bigg)^2 = 12 \implies p^2 = 12q^2 = 3 (2 q)^2
    \end{equation}
    So $p$ much be even $p = 2 p^\prime$. Therefore, $p^{\prime 2} = 3 q^2$, and $q$ must be odd. This means that $p^\prime$ must be odd. We can rewrite the equation 
    \begin{equation}
      p^{\prime 2} - q^2 = 2 q^2 \implies (p^\prime + q) (p^\prime - q) = 2q^2
    \end{equation}
    where the left hand side is divisible by $4$ but the right hand side is divisible by at most $2$, leading to a contradiction. 
  \end{solution}

  \begin{exercise}[Rudin 1.3]
    Prove that the axioms of multiplication imply the following. 
    \begin{enumerate}
      \item If $x \neq 0$ and $xy = xz$, then $y = z$. 
      \item If $x \neq 0$ and $xy = x$, then $y = 1$. 
      \item If $x \neq 0$ and $xy = 1$, then $y = x^{-1}$. 
      \item If $x \neq 0$, then $(x^{-1})^{-1} = x$. 
    \end{enumerate}
  \end{exercise}
  \begin{solution}
    Listed. 
    \begin{enumerate}
      \item $xy = xz \implies \frac{1}{x} \cdot x y = \frac{1}{x} x z \implies y = z$ 
      \item $x y = x \implies \frac{1}{x} x y = \frac{1}{x} x \implies y = 1$ 
      \item $x y = 1 \implies \frac{1}{x} x y = \frac{1}{x} 1 \implies y = \frac{1}{x}$ 
      \item $(x^{-1})^{-1} \cdot x^{-1} = 1 \implies (x^{-1})^{-1} \cdot x^{-1} \cdot x = 1 \cdot x \implies (x^{-1})^{-1} = x$
    \end{enumerate}
  \end{solution}

  \begin{exercise}[Rudin 1.4]
    Let $E$ be a nonempty subset of an ordered set; suppose $\alpha$ is a lower bound of $E$ and $\beta$ is an upper bound of $E$. Prove that $\alpha \leq \beta$. 
  \end{exercise}
  \begin{solution}
    Since $E$ is nonempty, we choose any $x \in E$. By definition, $\alpha \leq x$ and $x \leq \beta$, and by transitive property of orderings, we have $\alpha \leq \beta$. 
  \end{solution}

  \begin{exercise}[Rudin 1.5]
    Let $A$ be a nonempty set of real numbers which is bounded below. Let $-A$ be the set of all numbers $-x$, where $x \in A$. Prove that 
    \begin{equation}
      \inf A = -\sup(-A)
    \end{equation}
  \end{exercise}
  \begin{solution}
    We would like to prove that $\inf A \leq -\sup(-A)$ and $\inf A \geq -\sup(-A)$. For the first part, we start off with the definition of the infimum. 
    \begin{align*}
      \inf A \leq x \; \forall x \in A & \implies - \inf A \geq -x \; \forall x \in A \\
      & \implies - \inf A \geq x \forall x \in -A \\
      & \implies -\inf A \geq \sup(-A) \\
      & \implies \inf A \leq - \sup(-A)
    \end{align*}
    For the second part, we start with the definition of the supremum. 
    \begin{align*}
      \sup(-A) \geq x \forall x \in -A & \implies \sup(-A) \geq -x \forall x \in A \\
      & \implies -\sup(-A) \geq x \forall x \in A \\
      & \implies -\sup(-A) \leq \inf A
    \end{align*}
  \end{solution}

  \begin{exercise}[Rudin 1.8]
    Prove that no order can be defined in the complex field that turns it into an ordered field. 
  \end{exercise}
  \begin{solution}
    Note that if $x \geq 0$, then $-x \leq 0$ for all $x$ of any ordered field. Since if $x \geq 0$ and $-x > 0$, then $x -x > 0$, which is absurd. Therefore, one of either $i$ or $-i$ should be greater than $0$. But $i^2 = (-i)^2 = -1$, so this means that $-1 > 0$, which implies that $0 < 1$. But either $1$ or $-1$ must $\geq 0$. 
  \end{solution}

  \begin{exercise}[Rudin 1.9]
    Equip $\mathbb{C}$ with the dictionary order. That is, given $z = a + bi$ and $w = c + di$, $z < w$ if $a < c$, or if $a = c$ and $b < d$. Does this ordered set have a least upper bound property? 
  \end{exercise}
  \begin{solution}
    No it does not. Consider the set $S = \{ a + b i \in \mathbb{C} \mid a \leq 3\}$. $S$ is bounded by $4$, but it doesn't have a least upper bound. Given any $3 + bi$, this is not an upper bound since we can construct $3 + (b + \epsilon) i \in S$. Given any $a + bi$ where $a > 3$, we can always find a lower bound of form $a + (b - \epsilon) i$ that also bounds $S$. 
  \end{solution}

  \begin{exercise}[Rudin 1.10]
    Suppose $z = a + bi$, $w = u + iv$, and 
    \begin{equation}
      a = \bigg( \frac{|w| + u}{2} \bigg)^{1/2} \text{ and } b = \bigg( \frac{|w| - u}{2} \bigg)^{1/2}
    \end{equation}
    Prove that $z^2 = w$ if $v \geq 0$ and that $(\bar{z})^2 = w$ if $v \leq 0$. Conclude that every complex number (with one exception!) has two complex square roots. 
  \end{exercise}
  \begin{solution}
    We can calculate 
    \begin{equation}
      z^2 = (a^2 - b^2) + 2 a b i = u + \sqrt{v^2} i = \begin{cases} u + v i & \text{ if } v \geq 0 \\ u - vi & \text{ if } v \leq 0 \end{cases} 
    \end{equation}
    Since if we assume $v \geq 0$, then we have $z^2 = w$. We also get 
    \begin{equation}
      \bar{z}^2 = (a^2 - b^2) - 2 a b i = u - \sqrt{v^2} i = \begin{cases} u - v i & \text{ if } v \geq 0 \\ u + vi & \text{ if } v \leq 0 \end{cases}
    \end{equation}
    and assuming $v \leq 0$, we have $\bar{z}^2 = w$. Therefore, every complex number $w$ has both $\pm z$ as its square root if $v \geq 0$, $\pm \bar{z}$ if $v \leq 0$, and just one root if $z = 0$. 
  \end{solution}

  \begin{exercise}[Rudin 1.11]
    If $z$ is a complex number, prove that there exists an $r \geq 0$ and a complex number $w$ with $|w| = 1$ s.t. $z = rw$. Are $w$ and $r$ always uniquely determined by $z$? 
  \end{exercise}
  \begin{solution}
    If $z = 0$, then $r = 0$ and there is no unique $w$. If $z = a + bi \neq 0$, then define 
    \begin{equation}
      r = |z| = (a^2 + b^2)^{1/2}, \;\; w = \frac{1}{r} z
    \end{equation}
    which proves existence. As for uniqueness, assume that there are two forms 
    \begin{equation}
      z = r w = r^\prime w^\prime
    \end{equation}
    Then, $w = \frac{r^\prime}{r} w^\prime \implies |w| = \big| \frac{r^\prime}{r} \big| |w^\prime| = 1$, which implies that $r^\prime / r = 1$ and so $r = r^\prime$. This means that $w = w^\prime$. 
  \end{solution}

  \begin{exercise}[Rudin 1.12]
    If $z_1, \ldots, z_n$ are complex, prove that
    \begin{equation}
      |z_1 + z_2 + \ldots + z_n| \leq |z_1| + \ldots + |z_n|
    \end{equation}
  \end{exercise}
  \begin{solution}
    By induction, it suffices to prove $|z_1 + z_2| \leq |z_1| + |z_2|$. We have 
    \begin{align*}
      |z_1 + z_2|^2 & = (z_1 + z_2) (\overline{z_1 + z_2}) \\
      & = (z_1 + z_2) (\bar{z_1} + \bar{z_2}) \\ 
      & = z_1 \bar{z_1} + z_1 \bar{z_2} + z_2 \bar{z_1} + z_2 \bar{z_2} \\
      & = |z_1|^2 + |z_2|^2 + z_1 \bar{z_2} + z_2 \bar{z_1} \\
      & = |z_1|^2 + |z_2|^2 + 2 (a c + bd) \\
      & \leq |z_1|^2 + |z_2|^2 + 2 \sqrt{a^2 + b^2} \sqrt{c^2 + d^2} \;\; (\text{Schwartz}) \\
      & = |z_1|^2 + |z_2|^2 + 2 |z_1| |z_2| \\
      & = (|z_1| + |z_2|)^2
    \end{align*}
    since both sides are positive, we can take their square root to get the desired result. 
  \end{solution}

  \begin{exercise}[Rudin 1.13]
    If $x, y$ are complex, prove that 
    \begin{equation}
      \big| |x| - |y| \big| \leq |x - y|
    \end{equation}
  \end{exercise}
  \begin{solution}
    Since both sides are nonnegative, we can square both sides. Note that due to Cauchy Schwartz inequality, $2|x| |y| \geq x \bar{y} + y \bar{x}$ since expanding them gives 
    \begin{equation}
      2 \sqrt{a^2 + b^2} \sqrt{c^2 + d^2} \geq 2 (ac + bd)
    \end{equation}
    Therefore, the following inequality is true: 
    \begin{equation}
      |x|^2 + |y|^2 - 2 |x| |y| \leq x \bar{x} + y \bar{y} - x \bar{y} - y \bar{x}
    \end{equation}
    which reduces to form $(|x| - |y|)^2 \leq |x - y|^2$. 
  \end{solution}

  \begin{exercise}[Rudin 1.14]
    If $z$ is a complex number s.t. $|z| = 1$, that is such that $z \bar{z} = 1$, compute 
    \begin{equation}
      |1 + z|^2 + |1 - z|^2
    \end{equation}
  \end{exercise}
  \begin{solution}
    Compute. 
    \begin{equation}
      (1 + z) (1 + \bar{z}) + (1 - z) (1 - \bar{z}) = 1 + z + \bar{z} + z \bar{z} + 1 - z - \bar{z} + z \bar{z} = 4
    \end{equation}
  \end{solution}

  \begin{exercise}[Rudin 1.15]
    Under what conditions does equality hold in the Schwarz inequality? 
  \end{exercise}
  \begin{solution}
    If they are antiparallel, since 
    \begin{equation}
      \langle x, y \rangle = ||x || ||y || \cos{\theta}
    \end{equation}
  \end{solution}

  \begin{exercise}[Rudin 1.16]
    Suppose $k \geq 3$, $\mathbf{x}, \mathbf{y} \in \mathbb{R}^k$, $|\mathbf{x} - \mathbf{y}| = d > 0$, and $r > 0$. Prove: 
    \begin{enumerate}
      \item[a)] If $2r > d$, there are infinitely many $\mathbf{z} \in \mathbb{R}^k$ s.t. 
      \[|\mathbf{z} - \mathbf{x}| = |\mathbf{z} - \mathbf{y}| = r\]
      \item[b)] If $2r = d$, there is exactly one such $\mathbf{z}$. 
      \item[c)] If $2r < d$, there is no such $\mathbf{z}$. 
    \end{enumerate}
  \end{exercise}

  \begin{exercise}[Rudin 1.17]
    Prove that
    \begin{equation}
      |\mathbf{x} + \mathbf{y}|^2 + |\mathbf{x} - \mathbf{y}|^2 = 2|\mathbf{x}|^2 + 2|\mathbf{y}|^2
    \end{equation}
  \end{exercise}
  \begin{solution}
    This is trivial if we simply expand 
    \begin{align}
      |\mathbf{x} + \mathbf{y}|^2 + |\mathbf{x} - \mathbf{y}|^2 & = \langle \mathbf{x} + \mathbf{y}, \mathbf{x} + \mathbf{y} \rangle + \langle \mathbf{x} - \mathbf{y}, \mathbf{x} - \mathbf{y} \rangle \\
      & = \langle \mathbf{x}, \mathbf{x} \rangle + 2 \langle \mathbf{x}, \mathbf{y} \rangle + \langle \mathbf{y}, \mathbf{y} \rangle + \langle \mathbf{x}, \mathbf{x} \rangle - 2 \langle \mathbf{x}, \mathbf{y} \rangle + \langle \mathbf{y}, \mathbf{y} \rangle \\
      & = 2 \langle \mathbf{x}, \mathbf{x} \rangle + 2 \langle \mathbf{y}, \mathbf{y} \rangle \\
      & = 2|\mathbf{x}|^2 + 2|\mathbf{y}|^2
    \end{align}
  \end{solution}

  \begin{exercise}[Rudin 1.18]
    If $k \geq 2$ and $\mathbf{x} \in \mathbb{R}^k$, prove that there exists $\mathbf{y} \in \mathbb{R}^k$ s.t. $\mathbf{y} \neq \mathbf{0}$, but $\mathbf{x} \cdot \mathbf{y} = 0$. Is this also true if $k = 1$? 
  \end{exercise}
  \begin{solution}
    Let $x \in \mathbb{R}^k$ and $\ell \in \mathbb{R}^{k \ast}$, the dual space. By Riesz representation theorem, we can define the canonical isomorphism $\ell \mapsto y$ between these two spaces as 
    \begin{equation}
      \ell (x) = (x, y)
    \end{equation}
    Since $y \neq 0$ by assumption, $\ell \neq 0$, and so its rank is at least $1$. Since $\ell$ maps to $\mathbb{R}$, the rank has to be $1$. By rank nullity theorem, we have 
    \begin{equation}
      \dim N(\ell) = k - \mathrm{rank}(\ell) = k - 1
    \end{equation}
    and so there exists nontrivial annihilators $\ell$ of $x$, which can be mapped to a nontrivial $y \in \mathbb{R}^k$. 
  \end{solution}

  \begin{exercise}[Rudin 1.19]
    Suppose $\mathbf{a}, \mathbf{b} \in \mathbb{R}^k$. Find $\mathbf{c} \in \mathbb{R}^k$ and $r > 0$ s.t. 
    \begin{equation}
      |\mathbf{x} - \mathbf{a}| = 2 | \mathbf{x} - \mathbf{b}|
    \end{equation}
    if and only if $|\mathbf{x} - \mathbf{c}| = r$. 
  \end{exercise}
  \begin{solution}
    If we draw out the circle, it must contain two points on the line drawn by connecting $A$ and $B$. Since it must be symmetric, its center and radius can then be easily calculated to be 
    \begin{equation}
      r = \frac{2}{3} |b - a|, \;\;\; c = \frac{1}{3} (4b - a)
    \end{equation}
  \end{solution}

  \begin{exercise}[Zorich 2.2.1]
    Using the principle of induction, show that 
    \begin{enumerate}
      \item the sum $x_1 + \ldots + x_n$ of real numbers is defined independently of the insertion of parentheses to specify the order of addition. 
      \item the same is true of the product $x_1 \ldots x_n$ 
      \item $|x_1 + \ldots + x_n| \leq |x_1| + \ldots + |x_n|$ 
      \item $|x_1 \ldots x_n| \leq |x_1| \ldots |x_n|$
      \item For any $n, m \in \mathbb{N}$ such that $m < n$, $(n - m) \in \mathbb{N}$. 
      \item $(1 + x)^n \geq 1 + n x$ for $x > - 1$ and $n \in \mathbb{N}$, equality holding for when $n=1$ or $x=0$. 
      \item $(a + b)^n = a^n + _n C_1 a^{n-1} b^1 + \ldots + b^n$ (aka binomial theorem). 
    \end{enumerate}
  \end{exercise}
  \begin{solution}
    Listed. 
    \begin{enumerate}
      \item Let $n$ denote the number of elements in the sum. We prove by strong law of induction. The base case for when $n=1, 2, 3$ is trivially true. 
      \begin{align*}
        x_1 & = x_1 & (\text{identity}) \\
        x_1 + x_2 & = x_1 + x_2 & (\text{identity}) \\
        (x_1 + x_2) + x_3 & = x_1 + (x_2 + x_3) & (\text{associativity}) 
      \end{align*}
      Then, the sum of $n=k$ parameters is defined by $k-2$ pairs of parentheses defining the order of the sum. These parentheses define a sequence of $k-1$ 2-fold additions. Now, assume that the claim is true for 
      \begin{equation}
        S_n \equiv x_1 + \ldots x_n \text{ for } n = 1, 2, \ldots, k
      \end{equation}
      Then, for a specific sum $S_{k+1}$ of $k+1$ elements with $k-1$ parentheses, we can reduce the sum to its final 2-fold addition 
      \begin{equation}
        S_{k+1} \equiv \underbrace{(x_1 + \ldots + x_i)}_{\varphi_1}  + \underbrace{(x_{i+1} + \ldots + x_{k+1})}_{\varphi_2}
      \end{equation}
      Since $i, k-i+1 < k$, by the strong law $\varphi_1$ and $\varphi_2$ are independent of the order of sum. 
      \item Exactly identical to (a). 
      \item By the triangle inequality $|x_1 + x+2| \leq |x_1| + |x_2|$. Now, assume for $n=k$ is true. Then, let $S_k = x_1 + \ldots + x_k$, so 
      \begin{equation}
        |x_1 + \ldots + x_k + x_{k+1}| = |S_k + x_{k+1}| \leq |S_k| + |x_{k+1}| \leq \sum_{i=1}^{k+1} |x_{i}|
      \end{equation}
      \item Same as (c). 
      \item Let us fix $m$ to be any element of $\mathbb{N}$. Then, the base case is for $n = m + 1$ (which is in $\mathbb{N}$ since it is inductive), so 
      \begin{equation}
        n - m = (m + 1) - m = 1 \in \mathbb{N}
      \end{equation}
      Now, given that for some integer $n \geq m+1$, $n - m \in \mathbb{N}$ is true, we have 
      \begin{align*}
        (n + 1) - m & = n + (1 - m ) & (\text{associativity}) \\
        & = n + (-m + 1) & (\text{commutativity}) \\
        & = (n - m) + 1 & (\text{associativity})
      \end{align*}
      where $(n - m) + 1 \in \mathbb{N}$ by inductive property of $\mathbb{N}$. 
      \item We prove by induction. For $n=1$, it is trivial that $(1 + x)^1 \geq 1 + 1 \cdot x$. Now assume that the claim is true for some $k \in \mathbb{N}$. Then, 
      \begin{align*}
        (1 + x)^{k+1} = (1 + x)^k (1 + x) & \geq (1 + k x) (1 + x) \\
        = 1 + (k+1) x + k x^2 \\ 
        \geq 1 + (k+1) x
      \end{align*}
      where equality holds if $x = 0 \implies 1^{k+1} = 1^k \cdot 1 = 1$ or $n=1 \implies $ trivial case. 
      \item The base case for $n=1$ is trivial since $(a + b)^1 = \binom{1}{0} a + \binom{1}{1} b$. We introduce Newton's identity. 
      \begin{align*}
        \binom{k}{j-1} + \binom{k}{j} & = \frac{k!}{(j-1)! (k-j+1)!} + \frac{k!}{j! (k-j)!} \\
        & = k! \bigg( \frac{j}{j! (k-j+1)!} + \frac{k-j+1}{j! (k-j+1)!} \bigg) \\
        & = k! \cdot \frac{k+1}{j! (k-j+1)!} \\
        & = \frac{(k+1)!}{j!(k-j+1)!} = \binom{k+1}{j}
      \end{align*}
      Now assuming that the binomial formula holds for some $n=k$, we have 
      \begin{align}
        (a + b)^{k+1} & = (a + b)^k (a + b) \\
        & = \bigg( \sum_{j=0}^k \binom{k}{j} a^j b^{k-j} \bigg) (a + b) \\
        & = \sum_{j=0}^k \binom{k}{j} a^{j+1} b^{k-j} + \sum_{j=0}^k \binom{k}{j} a^j b^{k-j + 1} \\
        & = \binom{k}{0} a^0 b^{k+1} + \binom{k}{k} a^{j+1} b^0 + \sum_{j=0}^{k-1} \binom{k}{j} a^{j+1} b^{k-j} + \sum_{j=1}^k \binom{k}{j} a^j b^{k-j+1} \\
        & = \binom{k+1}{0} a^0 b^{k+1} + \binom{k+1}{k+1} a^{j+1} b^0 + \sum_{j=1}^k \bigg[ \binom{k}{j-1} + \binom{k}{j} \bigg] \, a^j b^{k-j + 1} \\
        & = \sum_{j=0}^{k+1} \binom{k+1}{j} a^j b^{k-j+1} 
      \end{align}
    \end{enumerate}
  \end{solution}

  \begin{exercise}[Zorich 2.2.3]
    Show that an inductive set is not bounded above. 
  \end{exercise}
  \begin{solution}
    Assume that a $X$ is a nonempty inductive set that is bounded above. By definition, there exists a number $B \in \mathbb{R}$ such that $\max{X} < B$. Then, this means that there exists no numbers in $[B, B + 1)$. Since $X$ is inductive, this means that there cannot exist any elements of $X$ in the interval $[B-1, B)$, and similarly for the interval $[B-2, B)$, and so on, meaning that if $x \in X$, then $x \not\in [B-k, B-k + 1)$ for all $k\in \mathbb{Z}$. By the Archimidean principle, this implies that $X = \emptyset$, contradicting our assumption. 
  \end{solution}

  \begin{exercise}[Zorich 2.2.4]
    Prove the following. 
    \begin{enumerate}
      \item An inductive set is infinite (that is, equipollent with one of its subsets different from itself). 
      \item The set $E_n = \{x \in \mathbb{N}\,|\, x \leq n\}$ is finite. 
    \end{enumerate}
  \end{exercise}
  \begin{solution}
    Listed. 
    \begin{enumerate}
      \item Assume that an inductive set $X$ is finite $\implies$ $X$ is bounded above (we can choose upper bound $B = \max{X} + 1$). But from 2.2.3, an inductive set cannot be bounded above, contradicting our assumption. 
      \item It is trivial that $E_1 = \{1\}$ is finite since $\text{card}{E_1} = 1$. Now, if for some $k$, $E_k$ is finite with cardinality $e_k$, then $\text{card}{E_{k+1}} = e_k + 1$, which implies finiteness.  
    \end{enumerate}
  \end{solution}

  \begin{exercise}[Zorich 2.2.5]
    Listed. 
    \begin{enumerate}
      \item Let $m, n \in \mathbb{N}$ and $m >n$. Their greatest common divisor $\gcd(m, n) = d \in \mathbb{N}$ can be found in a finite number of steps using the following algorithm of Euclid involving successive divisions with remainder. 
      \begin{align*}
        m & = q_1 n + r_1 \\
        n & = q_2 r_1 + r_2 \\
        r_1 & = q_3 r_2 + r_3 \\
        \ldots & = \ldots \\
        r_{k-2} & = q_{k} r_{k-1} + r_{k} \\
        r_{k-1} & = q_{k+1} r_k + 0
      \end{align*}
      Then $d=r_k$ 
      \item If $d = \gcd(m, n)$, one can choose numbers $p, q \in \mathbb{Z}$ such that $pm + qn = d$. 
    \end{enumerate}
  \end{exercise}
  \begin{solution}
    Listed. 
    \begin{enumerate}
      \item 
      \item Letting $n = r_0$, notice that the equations above satisfy for $i=0, 1, \ldots$
      \begin{equation}
        r_i = q_{i+2} r_{i+1} + r_{i+2} \implies r_{i} - q_{i+2} r_{i+1} = r_{i+2} \tag{1} \label{Euclid}
      \end{equation}
      Note that the second-to-last equation allows us to write $r_k$ as a linear combination of $r_{k-2}$ and $r_{k-1}$: $r_k = r_{k-2} - q_k r_{k-1}$. Now by applying \eqref{Euclid}, we can reduce the above to a linear combination of $r_{k-3}$ and $r_{k-2}$. 
      \begin{align*}
        r_k & = r_{k-2} - q_k r_{k-1} \\
        & = r_{k-2} - q_k (r_{k-3} - q_{k-1} r_{k-3}) \\
        & = (1 + q_{k-1} q_k) r_{k-2} - q_k r_{k-3} 
      \end{align*}
      and repeatedly doing this allows us to reduce $r_k$ to a linear combination $q_0 r_0 + q_1 r_1$. By the ring properties of $\mathbb{Z}$, the new linear coefficients are also in $\mathbb{Z}$. Reducing one last time using the first equation in the Euclidean algorithm gives 
      \begin{align*}
        r_k & = q_0 r_0 + q_1 r_1 \\
        & = q_0 n + q_1 (m - q_1 n) \\
        & = q_1 m + (q_0 - q_1) n \\
        & = p m + q n 
      \end{align*}
    \end{enumerate}
  \end{solution}

  \begin{exercise}[Zorich 2.2.9]
    Show that if the natural number $n$ is not of the form $k^m$, where $k, m \in \mathbb{N}$, then the equation $x^m = n$ has no rational roots. 
  \end{exercise}
  \begin{solution}
    Assume that there is a rational solution $x = p/q$, with $p, q \in \mathbb{N}$ of the equation. Then, 
    \begin{equation}
      \bigg( \frac{p}{q}\bigg)^m = \frac{p^m}{q^m} = n \implies p^m = q^m n
    \end{equation}
    By the fundamental theorem of arithmetic, the exponents of the prime factors of $p^m$ must all be multiples of $m$, and so it must be so for the right hand side $\implies x$ must be of form $x = k^m$ for some $k$. This is a contradiction. 
  \end{solution}

  \begin{exercise}[Zorich 2.2.12]
    Knowing that $\frac{m}{n} \equiv m \cdot n^{-1}$ by definition, where $m \in \mathbb{Z}$ and $n \in \mathbb{N}$, derive the ``rules'' for addition, multiplication, and division of fractions, and also the condition for two fractions to be equal. 
  \end{exercise}
  \begin{solution}
    We can construct a $\mathbb{Q}$ as a quotient space $\mathbb{Z} \times \mathbb{N} / \sim$, where $\sim$ is an equivalent relation where 
    \begin{equation}
      (q_1, p_1) \sim (q_2 , p_2) \text{ iff } q_1 p_2 = p_1 q_2
    \end{equation}
    which is the familiar equivalence relation from ``simplifying'' a fraction. We define addition and multiplication as the following 
    \begin{align*}
      (a, b) + (c, d) & = (ad + bc, bd) \\
      (a, b) \cdot (c, d) & = (ac, bd) 
    \end{align*}
    which turns out to be algebraically closed in $\mathbb{Q}$. The additive identity is the equivalence class $0 = \{(0, c) \,|\, c \in \mathbb{N}\}$, and the multiplicative identity is the equivalence class $1 = \{(c, c) \,|\, c \in \mathbb{N}\}$. It is easy to check that $+$ is commutative, the additive inverse is $-(a, b) = (-a, b)$, and the multiplicative inverse is $(a, b)^{-1} = (b, a)$. We can subtract and divide these elements of $\mathbb{Q}$, called ``fractions,'' as such: 
    \begin{align*}
      (a, b) - (c, d) & = (a, b) + (-(c, d)) = (a, b) + (-c, d) = (ad - bc, bd) \\
      (a, b) \div (c, d) & = (a, b) \cdot (c, d)^{-1} = (a, b) \cdot (d, c) = (ad, bc) 
    \end{align*}
  \end{solution}

  \begin{exercise}[Zorich 2.2.13]
    Verify that the rational numbers $\mathbb{Q}$ satisfy all the axioms for real numbers except for the axiom of completeness. 
  \end{exercise}
  \begin{solution}
    From continuing the steps of 2.2.14, we can prove $\mathbb{Q}$ is an algebraic field (associativity, commutativity of addition and multiplication, along with distributive property).We can actually define the order relation $\leq_\mathbb{Q}$ in two ways: 
    \begin{enumerate}
      \item $(a, b) \leq (c, d)$ iff $ad \leq_{\mathbb{Z}} bc$, where $\leq_{\mathbb{Z}}$ is the order relation on $\mathbb{Z}$ (which can be defined much more simply). 
      \item Recognizing that $\mathbb{Q} \subset \mathbb{R}$, we define the canonical injection map $i: \mathbb{Q} \longrightarrow \mathbb{R}$ and by abuse of language, endow the relation $\leq_{\mathbb{Q}}$ as the restriction of $\leq_{\mathbb{R}}$ onto $\mathbb{Q}$. That is, for $(a, b), (c, d) \in \mathbb{Q}$, 
      \begin{equation}
        (a, b) \leq_{\mathbb{Q}} (c, d) \text{ iff } i(a, b) \leq_{\mathbb{R}} i(c, d)
      \end{equation}
    \end{enumerate}
    The ordering for the 1st step can be checked for consistency. 
    \begin{enumerate}
      \item $(a, b) \leq (a, b)$ since $ab \leq ab$ (true in $\mathbb{Z}$) 
      \item $(a, b) \leq (c, d), (c, d) \leq (a, b)$ means that $ad \leq bc$ and $bc \leq ad \implies ad = bc$ (true in $\mathbb{Z}$) 
      \item $(a, b) \leq (c, d) \leq (e, f)$ implies $ad \leq bc, cf \leq de$. Multiplying positive (important that $f >0$!) to the first inequality gives $adf \leq bcf$, and multiplying positive $b$ to the second gives $bcf \leq bde$, and by interpreting $\leq$ as the ordering defined on $\mathbb{Z}$, we use transitive property of $\leq_\mathbb{Z}$ to get $adf \leq bde \implies af \leq be \iff (a, b) \leq (e, f)$. 
      \item For any $(a, b), (c, d) \in \mathbb{Q}$, $(a, b) \leq (c, d)$ or $(a, b) \geq (c, d)$, which is equivalent to $ad \leq bc$ or $ad \geq bc$, which is true in $\mathbb{Z}$. 
    \end{enumerate}
    It is easy to prove $(a, b) \leq (c, d) \implies (a, b) + (p, q) \leq (c, d) + (p, q)$, and $0_{\mathbb{Q}} \leq (a, b), (c, d) \implies 0_\mathbb{Q} \leq (a, b) \cdot (c, d)$. However, $\mathbb{Q}$ is \textbf{not} complete. We prove this by showing that the subset $X = \{x \in \mathbb{Q} \,|\,x^2 \leq 2 \} \subset \mathbb{Q}$ does not satisfy the least upper bound property. Assume that there is a least upper bound $c \in \mathbb{Q}$. $c \neq \sqrt{2}$ (you should know how to prove irrationality of $\sqrt{2}$!), we have either $c > \sqrt{2}$ or $c < \sqrt{2}$. 
    \begin{enumerate}
      \item Let $c < \sqrt{2} \iff c - \sqrt{2} > 0$. By the Archimidean principle, there exists a $k \in \mathbb{N}$ such that $0 < \frac{1}{k} < c - \sqrt{2}$. Then, $\frac{1}{k} \in \mathbb{Q}$ and $\mathbb{Q}$ is a field, so $c - \frac{1}{k} \in \mathbb{Q}$. 
      \begin{equation}
        c - \frac{1}{k} < c - c + \sqrt{2} = \sqrt{2}
      \end{equation}
      So $c$ is not least and so it must be the case that $c < \sqrt{2}$. 
      \item Let $c < \sqrt{2} \iff \sqrt{2} - c > 0$. By the Archimidean principle, there exists a $k \in \mathbb{N}$ such that $0 < \frac{1}{k} < \sqrt{2} - c$. Then, $c + \frac{1}{k} \in \mathbb{Q}$ and 
      \begin{equation}
        c + \frac{1}{k} < c + (\sqrt{2} - c) = c
      \end{equation}
      So $c$ is not an upper bound. 
    \end{enumerate}
    Note that given a well-defined $c = \sup{X}$ and in the case where $c < \sqrt{2}$, we have $2 - c^2 > 0$, so we can choose a well-defined $\delta$ satisfying (by Archimidean principle) 
    \begin{equation}
      0 < \delta < \min \bigg\{1, \frac{2 - c^2}{2c + 1} \bigg\}
    \end{equation}
    which gives us 
    \begin{align*}
      (c + \delta)^2 & = c^2 + \delta(2c + \delta) & \\
      & < c^2 + \delta (2 c + 1) & (\delta < 1) \\
      & < c^2 + (2 - c^2) = 2 & 
    \end{align*}
    meaning that $c$ is not an upper bound. Similarly for when $c > \sqrt{2}$. 
  \end{solution}

  \begin{exercise}[Zorich 2.2.15]
    Prove the equivalence of these two statements. 
    \begin{enumerate}
      \item If $X$ and $Y$ are nonempty sets of $\mathbb{R}$ having the property that $x \leq y$ for every $x \in X, y \in Y$, then there exists $c \in \mathbb{R}$ such that $x \leq c \leq y$ for all $x \in X$ and $y \in Y$. 
      \item Every set $X \subset \mathbb{R}$ that is bounded above has a least upper bound. 
    \end{enumerate}
  \end{exercise}
  \begin{solution}
    Let $S_1$ be the first statement and $S_2$ the second. 
    \begin{enumerate}
      \item ($S_2 \implies S_1$). Let $X \subset \mathbb{R}$ be a set that is bounded above, and $Y$ is a set such that $x \leq y$ for all $x \in X, y \in Y$. Then, by LUB principle, there exists $c = \sup{X} \in \mathbb{R}$. Now, we claim that $c \leq y$ for all $y \in Y$. Assume it doesn't: then there exists $y^\prime \in Y$ such that $y^\prime < c$. But since we assumed $x \leq y$ for all $x \in X, y \in Y$, we have $x \leq y^\prime$ for all $x \in X$, which means that $y^\prime$ is an upper bound of $X$. But $y^\prime < c$, contradicting the given fact that $c$ was the least upper bound. 

      \item ($S_1 \implies S_2$). Given a nonempty set $X \subset \mathbb{R}$, we wish to show the existence of $\sup{X}$. We are guaranteed the existence of nonempty set $Y \subset \mathbb{R}$ such that $x \leq y$ for all $x \in X, y \in Y$, which implies that $X$ must be bounded above. Then, by $S_1$, there must exist a $c \in \mathbb{R}$ such that 
      \begin{equation}
        x \leq c \leq y \text{ for all } x \in X, y \in Y
      \end{equation}
      We claim that $c = \sup{X}$. It is an upper bound of $X$ since $x \leq c$ for all $x \in X$. It is least since the set of all upper bounds of $X$ is $Y$, and $c \leq y$ for all $y \in Y$. 
    \end{enumerate}
  \end{solution}

  \begin{exercise}[Olmsted 1.15]
    Prove \textbf{Dedekind's Theorem}: Let the real numbers be divided into two nonempty sets $A$ and $B$ such that (i) if $x \in A$ and if $y \in B$, then $x < y$ and (ii) if $x \in \mathbb{R}$ then either $x \in A$ or $x \in B$, then there exists a number $c$ (which may belong to either $A$ or $B$) such that any number less than $c$ belongs to $A$ and any number greater than $c$ belongs to $B$. 
  \end{exercise}
  \begin{solution}
    This is really the same statement as Zorich 2.2.15.a, the original statement of completeness, but with the extra condition that the sets $A = X, B = Y$ must be disjoint.
  \end{solution}

  \begin{exercise}[Olmsted 1.7]
    If $x$ is an irrational number, under what conditions on the rational numbers $a, b, c, d$ is $(ax + b)/(cx + d)$ rational? 
  \end{exercise}
  \begin{solution}
    Note that a trivial solution is $a = b = c = d = 1$ which gives $1$. Since 
    \begin{equation}
      \frac{ax + b}{cx + d} = \frac{acx + ad - ad + bc}{cx + d} = a + \frac{bc - ad}{cx + d} 
    \end{equation}
    for the above to be rational it is necessary that $1/(cx + d)$ is rational. But this cannot be the case, which leaves us with the condition that $bc = ad$. 
  \end{solution}

  \begin{exercise}[Olmsted 1.8]
    Prove that the system of integers satisfies the axiom of completeness. 
  \end{exercise}
  \begin{solution}
    Let $S \subset \mathbb{Z}$ be bounded from above. It must have a maximum element (justify?), call it $c$. Then we claim that $c \in \mathbb{Z}$ is the least upper bound. Being the maximum, it is an upper bound, and $c$ is least since the next smallest element is $c-1$, which is less than $c \in S$, and therefore cannot be an upper bound. 
  \end{solution}

  \begin{exercise}[Zorich 2.2.16/Olmsted 1.16]
    Prove the following. 
    \begin{enumerate}
      \item If $A \subset B \subset \mathbb{R}$, then $\sup{A} \leq \sup{B}$ and $\inf{A} \geq \inf{B}$. 
      \item Let $\mathbb{R} \supset X \neq \emptyset$ and $R \supset Y \neq \emptyset$. If $x \leq y$ for all $x \in X, y \in Y$, then $X$ is bounded above , $Y$ is bounded below, and $\sup{X} \leq \inf{Y}$. 
      \item If the sets $X, Y$ in (b), are such that $X \cup Y = \mathbb{R}$, then $\sup{X} = \inf{Y}$. 
      \item If $X$ and $Y$ are the sets defined in (c), then either $X$ has a maximal element or $Y$ as a minimal element. 
      \item Show that Dedekind's theorem is equivalent to the axiom of completeness. 
    \end{enumerate}
  \end{exercise}
  \begin{solution}
    Listed. 
    \begin{enumerate}
      \item Let 
      \begin{align*}
        A^\prime & = \{ x \in \mathbb{R} \,|\, x \geq a \; \forall a \in A\} \\
        B^\prime & = \{ x \in \mathbb{R} \,|\, x \geq b \; \forall b \in B\}
      \end{align*}
      where we can easily verify that $B^\prime \subset A^\prime$. By definition, we get $\sup{B} = \min{B^\prime}$ and $\sup{A} = \min{A^\prime}$. But since $B^\prime \subset A^\prime$, for any $b^\prime \in B^\prime$, there exists an $a^\prime \in A^\prime$ such that $a^\prime \leq b^\prime$, which implies that $\sup{B} = \min{B^\prime} \leq \min{A^\prime} = \sup{A}$. 

      \item $X$ is bounded above by any element of $Y$. $Y$ is bounded below by any element of $X$. By the completion axiom, there exists a $c \in \mathbb{R}$ such that
      \begin{equation}
        x \leq c \leq y \text{ for all } x \in X, y \in Y
      \end{equation}
      Since $c$ is an upper bound of $X$, $\sup{X} \leq c$ by definition, and since $c$ is a lower bound of $Y$, $\inf{Y} \geq c$ by definition. Therefore, $\sup{X} \leq c \leq \inf{Y}$. 

      \item From completeness there exists a $c \in \mathbb{R}$ such that $x \leq c \leq y$ for all $x \in X, y \in Y$. $Y$ is, by definition, the set of \textit{all} upper bounds of $X$ (i.e. \textit{every} upper bound of $X$ is in $Y$, unlike $Y$ defined in 2.2.16.b). Since $c \leq y$ for all $y \in Y$, $c$ is minimal and so $c = \sup{X}$. $X$ is the set of all lower bounds of $Y$ by definition, so $c \geq x$ for all $x \in X \implies c = \inf{Y}$. So, $\inf{Y} = c = \sup{X}$. 

      \item We know that there exists $c = \inf{Y} = \sup{X}$. Since $X \cup Y = \mathbb{R}$, $c$ must be in at least $X$ or $Y$. If $c \in X$, then $c = \sup{X} = \max{X}$, and if $c \in Y$, then $c = \inf{Y} = \min{Y}$. 
      \item This is the same statement as Zorich 2.2.15.a (an iff equivalence, not just one way implying). 
    \end{enumerate}
  \end{solution}

  \begin{exercise}[Olmsted 1.13]
    Let $S$ be a nonempty set of numbers bounded above, and let $x$ be the least upper bound of $S$. Prove that $x$ has the two properties corresponding to an arbitrary positive number $\epsilon$: 
    \begin{enumerate}
      \item every element $s \in S$ satisfies the inequality $s < x + \epsilon$
      \item at least one element $s \in S$ satisfies the inequality $s > x - \epsilon$
    \end{enumerate}
  \end{exercise}
  \begin{solution}
    Listed. 
    \begin{enumerate}
      \item $x$ is an upper bound $\implies s \leq x$ for all $s \in S$, which implies that $s \leq x < x + \epsilon$. 
      \item By definition, $x - \epsilon$ cannot be an upper bound, so $x - \epsilon \geq s$ for all $s \in S$ is not true. Therefore, there must exist one $s \in S$ such that $s > x - \epsilon$. 
    \end{enumerate}
  \end{solution}

  \begin{exercise}[Zorich 2.2.18]
    Let $-A$ be the set of numbers of the form $-a$, where $a \in A \subset \mathbb{R}$. Show that $\sup(-A) = -\inf(A)$. 
  \end{exercise}
  \begin{solution}
    If $A$ is unbounded below, then $-\inf{A} = \infty$ and $-A$ is unbounded above, implying that $\sup{A} = \infty$. Now assume that $A$ is bounded below, then by completeness, it must have a greatest lower bound. Let us define the set $B = \{b \in \mathbb{R} \,|\, b \leq a \; \forall a \in A\}$. From 2.2.16.b, we have $b \leq \inf{A} \leq a$ for all $a \in A, b \in B$. Multiplying by $-1$ gives $-b \geq -\inf{A} \geq -a$ for all $a \in A, b \in B$, which is equivalent to saying 
    \begin{equation}
      a \leq -\inf{A} \leq b \text{ for all } a \in -A, b \in -B
    \end{equation}
    by definition of $-A, -B$. $-\inf{A}$ is clearly an upper bound of $-A$, and since  
    \begin{align*}
      B & = \{b \in \mathbb{R}\,|\, b \leq a \; \forall a \in A\} \\
      & = \{b \in \mathbb{R}\,|\, -b \geq -a \; \forall a \in A\} \\
      & = \{b \in \mathbb{R}\,|\, -b \geq a \; \forall a \in -A\} 
    \end{align*}
    implies that $-B = \{b \in \mathbb{R}\,|\, b \geq a \; \forall a \in -A\}$ is the set of all upper bounds of $A$. So, $-\inf{A}$ is the least upper bound of $-A$, i.e. $-\inf{A} = \sup(-A)$. 
  \end{solution}

  \begin{exercise}[Zorich 2.2.21]
    Show that the set $\mathbb{Q}(\sqrt{n})$ of numbers of the form $a + b \sqrt{n}$ where $a, b \in \mathbb{Q}$, $n$ is a fixed natural number that is not the square of any integer, is an ordered set satisfying the principle of Archimedes but not the axiom of completeness. 
  \end{exercise}
  \begin{solution}
    The order on $\mathbb{Q}(\sqrt{n})$ can be embedded from the ordering on the reals by defining the canonical injection map $i:\mathbb{Q}(\sqrt{n}) \longrightarrow \mathbb{R}$ and defining for any $x, y \in \mathbb{Q}(\sqrt{n})$, 
    \begin{equation}
      x \leq_{\mathbb{Q}(\sqrt{n})} y \iff i(x) \leq_{\mathbb{R}} i(y)
    \end{equation}
    Now, let $h > 0$ be any fixed real number, and $x = (a,b) = a + b\sqrt{n}$. By the Archimidean principle, we can find a $k \in \mathbb{Z}$ such that
    \begin{equation}
      (k - 1) h \leq x \leq k h \text{ for some } x \in \mathbb{Q}(\sqrt{n}) \subset \mathbb{R}
    \end{equation}
    We now show that $\mathbb{Q}(\sqrt{n})$ is not complete since it doesn't satisfy the LUB property. Since there are infinite prime numbers in $\mathbb{N}$, choose a prime number $p$ that is not a factor of $n$. Then, we are guaranteed that $pn$ is not a perfect square, and can define the set 
    \begin{equation}
      X = \{x \in \mathbb{Q}(\sqrt{n})\,|\, x < \sqrt{pn} \} \subset \mathbb{Q}(\sqrt{n})
    \end{equation}
    and assume that $c = c_1 + c_2 \sqrt{n} = \sup{X}$ exists ($c_1, c_2 \in \mathbb{Q}$). Clearly, $c \neq \sqrt{pn} \not\in \mathbb{Q}(\sqrt{n})$. 
    \begin{enumerate}
      \item Assume $c < \sqrt{pn} \iff 0 < \sqrt{pn} - c \in \mathbb{R}$. By the Archimidean principle, there exists a $k \in \mathbb{N}$ such that $0 < \frac{1}{k} < \sqrt{pn} - c $. Then, we can verify that $c + \frac{1}{k} = (c_1 + \frac{1}{k}) + c_2 \sqrt{n} \in \mathbb{Q}(\sqrt{n})$ and 
      \begin{equation}
        c + \frac{1}{k} < c + \sqrt{pn} - c = \sqrt{pn} \implies c + \frac{1}{k} \in X
      \end{equation}
      implies that $c$ is not an upper bound. So we must turn to case 2. 
      \item Assume $c > \sqrt{pn} \iff c - \sqrt{pn} > 0$. By AP, there exists a $k \in \mathbb{N}$ such that $0 < \frac{1}{k} < c - \sqrt{pn}$. Then, we can verify that $c - \frac{1}{k} \in \mathbb{Q}(\sqrt{n})$ and 
      \begin{equation}
        c - \frac{1}{k} > c - c + \sqrt{pn} = \sqrt{pn}
      \end{equation}
      implies that $c - \frac{1}{k}$ is an upper bound of $X$, so $c$ is not least. 
    \end{enumerate}
    Therefore, by contradiction, $c$ does not exist. 
  \end{solution}

  \begin{exercise}[Zorich 2.2.22]
    Let $n \in \mathbb{N}$ and $n > 1$. In the set $E_n = \{0, 1, \ldots, n-1\}$, we define the sum and product of two elements as the remainders when the usual sum and product in $\mathbb{R}$ are divided by $n$. With these operations on it, the set $E_n$ is denoted $\mathbb{Z}_n$. 
    \begin{enumerate}
      \item Show that if $n$ is not a prime number, then there are nonzero numbers $m, k \in \mathbb{Z}_n$ such that $m \cdot k = 0$, i.e. there exist nonzero zero divisors. 
      \item Show that if $p$ is prime, then there are no zero divisors in $\mathbb{Z}_p$ and $\mathbb{Z}_p$ is a field. 
      \item Show that, no matter what the prime $p$, $\mathbb{Z}_p$ cannot be ordered in a way consistent with the arithmetic operations on it. 
    \end{enumerate}
  \end{exercise}
  \begin{solution}
    Listed. 
    \begin{enumerate}
      \item $n$ is composite implies that there exist $1 < m, k < n$ such that $n = m k$. These factors $m, k$ are precisely the zero divisors of $\mathbb{Z}_n$ since $m k = n \equiv 0 \pmod{n}$. 
      \item With $p$ prime, assume that there are nontrivial zero divisors $1 < m, k < p$ in $\mathbb{Z}_p$. Then, $m k \equiv 0 \pmod{n} \implies m k = l p$ for some $l \in \mathbb{N}$. But this implies that $m$ or $k$ must divide $p$, which is impossible since $1 < m, k < p$. Then prove field axioms. 
      \item For any field, we must have $0 \leq 1$, because if not, then 
      \begin{equation}
        0 > 1 \implies 0 < 1^{-1} \cdot 1 = 1^{-1} \implies 0 \cdot 0 < 1^{-1} \cdot 1^{-1} = 1
      \end{equation}
      So, $0 \leq 1$ implies that $0 \leq 1 \leq 2 \leq \ldots \leq p-1$. But 
      \begin{equation}
        0 + 1 \leq (p - 1) + 1 = 0
      \end{equation}
      is false, so any ordering is impossible. 
    \end{enumerate}
  \end{solution}

  \begin{exercise}[Zorich 2.2.23]
    Show that if $\mathbb{R}$ and $\mathbb{R}^\prime$ are two models of the set of real numbers and $f: \mathbb{R} \longrightarrow \mathbb{R}^\prime$ (with $f \not\equiv 0^\prime$) is a mapping such that $f(x + y) = f(x) + f(y)$ and $f(x \cdot y) = f(x) \cdot f(y)$ for any $x, y \in \mathbb{R}$. Prove that $f$ is an order-preserving isomorphism. 
  \end{exercise}
  \begin{solution}
    Let $0, 0^\prime$ be the additive identity of $\mathbb{R}, \mathbb{R}^\prime$, respectively, and $1, 1^\prime$ the multiplicative identity. We claim that $f(0) = 0^\prime$ since
    \begin{align*}
      f(0) & = f(0 + 0) & \text{(definition of additive identity)} \\
      & = f(0) + f(0) & (\text{homomorphism over } + )
    \end{align*}
    which implies that $f(0) + f(0) = f(0) = 0^\prime + f(0) $. Since $f(0)$ lives in field $\mathbb{R}^\prime$, its additive identity $-f(0)$ is well defined, and we get $f(0) = f(0) + f(0) + (-f(0)) = 0^\prime + f(0) + (-f(0)) = 0^\prime$. We also claim that $f(1) = 1^\prime$ since 
    \begin{align*}
      f(1) & = f(1 \cdot 1) & \text{(definition of multiplicative identity)} \\
      & = f(1) \cdot f(1) & (\text{homomorphism over } \cdot ) 
    \end{align*}
    which implies that $f(1) \cdot f(1) = 1^\prime \cdot f(1)$. Since $f(1)$ lives in field $\mathbb{R}^\prime$, its multiplicative identity $f(1)^{-1}$ is well defined, and we get $f(1) = f(1) \cdot f(1) \cdot f(1)^{-1} = 1^\prime \cdot f(1) \cdot f(1)^{-1} = 1^\prime$. Now that we have proved mapping of identities, this implies the mapping of inverses. 
    \begin{align*}
      0^\prime & = f(0) = f(x - x) = f(x) + f(-x) \implies f(-x) = -f(x) \\
      1^\prime & = f(1) = f(x \cdot x^{-1}) = f(x) \cdot f(x^{-1}) \implies f(x^{-1}) = f(x)^{-1}
    \end{align*}
    With these conditions, we have proved that $f$ is a homomorphism of fields. Now we prove that $f$ is a bijection, but first, we claim that $f(x) = 0^\prime \implies x = 0$. Assume that there exists a nonzero $x \in \mathbb{R}$ such that $f(x) = 0^\prime$. Then, $x^{-1}$ is well defined, and 
    \begin{align*}
      f(x) \cdot f(x^{-1}) & = f(x) \cdot f(x)^{-1} = 0^\prime \\
      f(x) \cdot f(x^{-1}) & = f(x \cdot x^{-1}) = f(1) = 1^\prime 
    \end{align*}
    which implies that $0^\prime = 1^\prime$. So, $f(1) = 1^\prime = 0^\prime$, and so for all $k \in \mathbb{R}$, $f(k) =f(k \cdot 1) = f(k) \cdot f(1) = f(k) \cdot 0^\prime = 0^\prime \implies f \equiv 0^\prime$, leading to a contradiction of the assumption that $f^\prime \not\equiv 0^\prime$. 
    \begin{enumerate}
      \item ($f$ injective). Assume $f$ is not injective, i.e. there exists distinct $x_1, x_2 \in \mathbb{R}$ s.t. $f(x_1) = f(x_2)$. Then, using that fact $f(x) = 0 \implies x = 0$, 
      \begin{equation}
        0 = f(x_1) - f(x_2) = f(x_1 - x_2) \implies x_1 - x_2 = 0 \implies x_1 = x_2
      \end{equation}
      \item ($f$ surjective). Let $y$ be any nonzero element in $\mathbb{R}^\prime$ (clearly if $y=0^\prime$ then its preimage is $0$) and $y^{-1}$ its multiplicative inverse. Assume there exists no $x \in \mathbb{R}$ satisfying $f(x) = y$, meaning that there exist no $x$ satisfying
      \begin{equation}
        f(x) \cdot  = y \cdot y^{-1} = 1^\prime
      \end{equation}
      But since $f$ maps inverses to inverses, we can choose $x = (y^{-1})^{-1}$, which leads to 
      \begin{equation}
        f(x) \cdot y^{-1} = (
      \end{equation}
    \end{enumerate}
    Finally, we prove that $f$ is order preserving. Assume that $x \leq y \iff 0 \leq y - x$ , we wish to prove that 
    \begin{equation}
      f(x) \leq f(y) \iff 0 \leq f(y) - f(x) = f(y - x)
    \end{equation}
    Therefore, since this preservation of ordering is really the statement $0 \leq y - x \implies 0 \leq f(y - x)$, it suffices to prove that $0 \leq x \implies 0 \leq f(x)$. Now, assume that we have a $x$ such that $f(x) < 0^\prime$. Adding it with the equation $f(1) = 1^\prime$ gives us 
    \begin{equation}
      f(x + 1) < 1^\prime
    \end{equation}
    It is easy to prove that $0 \leq x \iff 0 \leq x^{-1}$. Now assume that $0 > f(x)$. \textbf{INCOMPLETE}
  \end{solution}

  \begin{exercise}[Density of Rationals in $\mathbb{R}$]
    Prove that for any two distinct $a < b \in \mathbb{R}$, there exists an infinite number of rational numbers between $a$ and $b$. 
  \end{exercise}
  \begin{solution}
    Since $a< b$, then $b - a > 0$ and by the Archimidean principle, there exists a $k \in \mathbb{N}$ such that 
    \begin{equation}
      0 < \frac{1}{k} < b - a \implies 1 < kb - ka
    \end{equation}
    which implies that the length of $[ka, kb)$ greater than $1$. By the inductive property of $\mathbb{Z}$, there must be an integer $p \in [ka, kb)$. If there were not, then this would imply that $[ka+1, kb+1)$ and $[ka-1, kb-1)$ had no integers and repeating would mean that there were no integers in $\mathbb{R}$. Therefore, 
    \begin{equation}
      ka \leq p < kb \implies a \leq \frac{p}{k} < b
    \end{equation}
    for all $a, b \in \mathbb{R}$, with $p/k \in \mathbb{Q}$. If $a$ is irrational we can replace the $\leq$ to $<$, leaving $a \leq \frac{p}{k} < b$, and if $a$ is rational, we can construct another rational $a + \frac{1}{k} \in (a, b)$. 
  \end{solution}

  \begin{exercise}[Nested Interval Lemma]
    With the fact that $\mathbb{R}$ is complete, prove the following. 
    \begin{enumerate}
      \item For a sequence of closed nested intervals $I_1 \supset I_2 \supset \ldots$ of $\mathbb{R}$, there exists a point $c \in \mathbb{R}$ belonging to all these intervals. 
      \item Furthermore, if the hypothesis also satisfies the fact that for any $\epsilon > 0$, there exists a $k \in \mathbb{N}$ such that $|I_k| < \epsilon$ (i.e. the length of the intervals decreases to $0$), then the point $c$ common to all sets is unique. 
    \end{enumerate}
  \end{exercise}
  \begin{solution}
    Listed. 
    \begin{enumerate}
      \item Let $I_n = [a_n, b_n]$, with $a_n < b_n$ finite for all $n \in \mathbb{N}$. For all $n\in \mathbb{N}$, we have $I_n = [a_n, b_n]$ and can take the two subsets $X_n = (-\infty, a_n)$ and $Y_n = (b_n, \infty)$, where $x \leq y$ for every $x \in X_n, y \in Y_n$. We also have the fact that $\mathbb{R} = X_n \cup I_n \cup Y_n$. Since $\mathbb{R}$ is complete, there exists a $c$ such that $x \leq c \leq y$ for all $x \in X, y \in Y$. But $x \leq c \iff c \not\in X_n$ and $c \leq y \iff c \not\in Y_n$, so for all $n \in \mathbb{N}$, $c$ must be in $I_n$. 
      
      \item Since we have proved (a), it now suffices to prove uniqueness of $c$. Let there be two distinct points $c_1, c_2 \in \mathbb{R}$ belonging to these intervals. Without loss of generality, assume $c_1 - c_2 > 0$, and choose 
      \begin{equation}
        \epsilon = \frac{c_1 - c_2}{3}
      \end{equation}
      Then, there should exist a $k \in \mathbb{N}$ such that $|I_k| < \epsilon$. Since $I_k$ must contain $c_1$, it must be a subset of $[c_1 - \epsilon, c_1 + \epsilon]$ (should be able to see why) and similarly for $c_2$. 
      \begin{align*}
        I_k & \subset \bigg[ c_1 - \frac{c_1 - c_2}{3}, c_1 + \frac{c_1 - c_2}{3} \bigg] = \bigg[ \frac{2c_1 + c_2}{3}, \frac{4c_1 - c_2}{3} \bigg] = L\\
        I_k & \subset \bigg[ c_2 - \frac{c_1 - c_2}{3}, c_2 + \frac{c_1 - c_2}{3} \bigg] = \bigg[ \frac{-c_1 + 4 c_2}{3}, \frac{c_1 + 2c_2}{3} \bigg] = M
      \end{align*}
      But since $c_1 > c_2 \implies \frac{c_1 + 2c_2}{3} < \frac{2c_1 + c_2}{3}$, $L$ and $M$ are disjoint $\implies I_k$, as a subset of both, leaves us with $I_k = \emptyset$, contradicting that it is a closed interval. 
    \end{enumerate}
  \end{solution}

  \begin{exercise}{Compactness of Closed Interval in $\mathbb{R}$}
    Prove that any system of open intervals covering (i.e. an open cover of) a closed interval contains a finite subsystem that covers the closed interval. Another way to state this is by saying that every closed interval of $\mathbb{R}$ is compact. 
  \end{exercise}
  \begin{solution}
    A closed interval with a finite open covering is trivially compact since any subcovering is also finite. We only need to deal with when a closed interval $I = [a, b]$ has an infinite open covering $\{U_\alpha \}_{\alpha \in A}$, which means that the set of indices $A$ is infinite. Assume that there exists no finite covering of $I$. Then, we divide $I$ into two halves 
    \begin{equation}
      I_1 = \Big[ a, \frac{a + b}{2}\Big], \;\; I_2 = \Big[ \frac{a + b}{2}, b \Big]
    \end{equation}
    and define a subcovering for each of them. That is, we can define $A_1 \subset A$ and $A_2 \subset A$ such that $\{U_\alpha\}_{\alpha \in A_1} \subset \{U_\alpha \}_{\alpha \in A}$ is a covering of $I_1$ and $\{U_\alpha\}_{\alpha \in A_2} \subset \{U_\alpha \}_{\alpha \in A}$ is a covering of $I_2$. At least one of $A_1$ or $A_2$ must be infinite, since if they were both finite, then we can define a finite covering $\{U_\alpha\}_{\alpha \in A_1 \cup A_2}$ of $I$. Choose the interval with the infinite covering and repeat this procedure, which will result in a nested interval that decreases in length by a half. 
    \begin{equation}
      I \supset I_1 \supset I_2 \supset \ldots
    \end{equation}
    By the nested interval lemma, there exists a unique point $c$ common to all these intervals. But since $c \in [a, b]$, the open cover $\{U\}$ should contain an open interval $(c - \delta_1, c + \delta_2)$ containing $c$. We wish to prove that this interval is a superset of some $I_k$ in the sequence above, contradicting the fact that $I_k$ has an infinite cover. Since the length of each $I_i$ decreases arbitrarily (i.e. we can choose any $\epsilon > 0$ and find a $I_k$ with length less than $\epsilon$), we choose $\epsilon = \frac{1}{2} \min\{\delta_1, \delta_2 \}$, and we should be able to find some $I_k$ that is a subinterval of $[c - \epsilon, c + \epsilon]$, which itself is a subinterval of $(c - \delta_1, c + \delta_2)$. 
    \begin{equation}
      I_k \subset \Big[ c - \frac{1}{2} \min\{\delta_1, \delta_2 \}, c + \frac{1}{2} \min\{\delta_1, \delta_2 \} \Big] \subset (c - \delta_1, c + \delta_2)
    \end{equation}
    Therefore, $(c - \delta_1, c + \delta_2)$ is a finite cover of $I_k$, contradicting the fact that all $I_k$'s have infinite covers. 
  \end{solution}

  \begin{exercise}{Bolzano-Weierstrass Theorem}
    Prove that every bounded infinite set of real numbers has at least one limit point. (A limit point $p$ of set $X$ is a point such that every open neighborhood of $p$ contains an infinite number of elements of $X$). 
  \end{exercise}
  \begin{solution}
    Let the set of points be denoted $X$, and let $a$ be the lower bound and $b$ be the upper bound. Then, $X \subset [a, b] = I$. Now divide $[a, b]$ into halves $[a, \frac{a + b}{2}] \cup [\frac{a + b}{2}, b]$. At least one of the halves must have an infinite number of points; choose the interval with infinite points as $I_1$ and doing this repeatedly gives the nested sequence 
    \begin{equation}
      I \supset I_1 \supset I_2 \supset \ldots
    \end{equation}
    By the nested interval lemma, there exists at least one point $c \in \mathbb{R}$ that is in all these intervals. Furthermore, since $|I_i| = \frac{1}{2^i} (b - a)$ decreases to $0$, we can choose a $\epsilon > 0$ and find an interval $I_k$ with $|I_k| < \epsilon$. We claim that $c$ is a limit point of $X$. Given an $\epsilon$, we wish to prove that there are an infinite number of points within the $\epsilon$-neighborhood $(c - \epsilon, c + \epsilon)$ of $c$. Since we can find some $I_k$ with $|I_k| < \epsilon$, we can see that 
    \begin{equation}
      I_k \subset (c - \epsilon, c + \epsilon)
    \end{equation}
    and therefore the $\epsilon$-neighborhood of $c$ contains $I_k$, which contains an infinite number of points in $X$. 
  \end{solution}
  \begin{solution}
    We can construct another proof that is dependent on the compactness lemma. This construction will be useful for problem 2.3.4. Let $X$ be a given subset of $\mathbb{R}$, and it follows from the definition of boundedness that $X$ is contained in some closed interval $I \subset \mathbb{R}$. We show that at least one point of $I$ is a limit point of $X$. Assume that it is not. Then each point $x \in I$ would have a neighborhood $U(x)$ containing at most a finite number of points from $X$. The totality of such neighborhoods $\{U(x)\}$ constructed for the points $x \in I$ forms an open covering of $X$. Since $I$ is closed, it is compact and therefore we can find a finite subcovering $\{U_i(x)\}_{i}$ of open intervals that cover $I$ and therefore cover $X$. This open cover $\{U_i(x)\}_{i}$ of $X$ is a finite union of sets that each contain at most a finite number of points from $X$, so the covering of $X$ contains a finite number of points from $X$, a contradiction that $X$ contains infinite points. 
  \end{solution}

  \begin{exercise}[Zorich 2.3.1]
    Show that 
    \begin{enumerate}
      \item if $I$ is any system of nested closed intervals, then 
      \[\sup\{ a \in \mathbb{R}\,|\, [a, b] \in I\} = \alpha \leq \beta = \inf\{ b \in \mathbb{R}\,|\,[a, b] \in I\}\]
      and 
      \[[\alpha, \beta] = \bigcap_{[a, b] \in I} [a, b]\]
      \item if $I$ is a system of nested open intervals $(a, b)$, the intersection
      \[\bigcap_{(a, b) \in I} (a, b)\] 
      may happen to be empty. 
    \end{enumerate}
  \end{exercise}
  \begin{solution}
    Listed. 
    \begin{enumerate}
      \item (May be tempted to say that $a_1 \leq a_2 \leq \ldots$, but this assumes that the indexing set $I$ is countable). We claim that for any two intervals $[a_n, b_n]$ and $[a_m, b_m]$ in $I$, 
      \[a_n \leq b_m\]
      Assume that $a_n > b_m$. Then $b_n \geq a_n > b_m \geq a_m$ implies that $[a_n, b_n]$ and $[a_m, b_m]$ are disjoint, contradicting the fact that they are nested. Now given that $X$ is the set of $a_n$'s and $Y$ is the set of $b_n$'s, we have $x \leq y$ for all $x \in X, y \in Y$. So by 2.2.16.b, we have $\sup{X} \geq \inf{Y}$. 
      \\
      To prove the second statement, we show that trying to ``expand'' the interval $[\alpha, \beta]$ will lead to a contradiction. Since $\alpha$ is the LUB, given any $\epsilon > 0$, there exists a $(a_l, b_l) \in X$ such that $\alpha - \epsilon < a_l < \alpha$, which implies that $ [\alpha, \beta] \subset [a_l, \beta] \subset [\alpha - \epsilon, \beta]$. Assuming that this extended interval is the intersection, we should be able to choose any point in $[\alpha - \epsilon, \beta]$ and find that it is in every element of $I$. We choose a point in $[\alpha - \epsilon, a_l)$, which is not in the interval $(a_l, b_l)$. We do the same for $\beta \mapsto \beta + \epsilon$. We also check that ``shrinking'' the interval $[\alpha, \beta] \mapsto [\alpha + \epsilon, \beta]$ is no good, since we can find an element in $[\alpha, \alpha + \epsilon)$ that is in every interval in $I$. 
      
      \item Take the system of nested open intervals 
      \[(0, 1) \supset (0, \frac{1}{2}) \supset (0, \frac{1}{3}) \ldots (0, \frac{1}{n}) \supset \ldots\]
      Take their infinite intersection, denote it $S$, and assume that some $\epsilon \in (0, 1)$ is in $S$. Since $\epsilon$ is a real number, by the Archimidean principle there exists a $k \in \mathbb{N}$ such that $\frac{1}{k} < \epsilon$. Therefore, $\epsilon \not\in (0, \frac{1}{k}) \implies \epsilon \not\in S$. 
    \end{enumerate}
  \end{solution}

  \begin{exercise}[Zorich 2.3.2]
    Show that 
    \begin{enumerate}
      \item from a system of closed intervals covering a closed interval it is not always possible to choose a finite subsystem covering the interval. 
      \item from a system of open intervals covering a open interval it is not always possible to choose a finite subsystem covering the interval. 
      \item from a system of closed intervals covering a open interval it is not always possible to choose a finite subsystem covering the interval. 
    \end{enumerate}
  \end{exercise}
  \begin{solution}
    We show with the interval $(0, 1)$ or $[0, 1]$. Using linear transformations it is easy to generalize this to any other interval $(a, b)$ or $[a, b]$. 
    \begin{enumerate}
      \item Consider the infinite covering
      \[[0, 1] = \big[0, \frac{1}{2}\big] \cup \big[\frac{1}{2}, \frac{3}{4}\big] \cup \big[\frac{3}{4}, \frac{7}{8}\big] \cup \ldots \]
      \item Consider the infinite covering 
      \[(0, 1) = \big(0, \frac{1}{2}\big) \cup \big(\frac{1}{2}, \frac{3}{4}\big) \cup \big(\frac{3}{4}, \frac{7}{8}\big) \cup \ldots \]
      \item Consider the infinite covering 
      \[(0, 1) = \big[0, \frac{1}{2}\big] \cup \big[\frac{1}{2}, \frac{3}{4}\big] \cup \big[\frac{3}{4}, \frac{7}{8}\big] \cup \ldots \]
    \end{enumerate}
  \end{solution}

  \begin{exercise}[Zorich 2.3.3]
    Show that if we only take the set $\mathbb{Q}$ of rational numbers instead of the complete set $\mathbb{R}$ of real numbers, with the definitions of closed, open, and neighborhood of a point $r \in \mathbb{Q}$ to mean respectively the corresponding subsets of $\mathbb{Q}$, then none of the three lemmas is true. 
  \end{exercise}
  \begin{solution}
    We prove only for the nested interval lemma. We choose the series of nested intervals 
    \[\bigg( \sqrt{2} - \frac{1}{n}, \sqrt{2} + \frac{1}{n} \bigg)\]
    with $n \in \mathbb{N}$. Assume that there is a $r \in \mathbb{Q}$ such that 
    \[r \in \bigg( \sqrt{2} - \frac{1}{n}, \sqrt{2} + \frac{1}{n} \bigg) \text{ for all } n \in \mathbb{N}\]
    which is equivalent to saying that $\big| r - \sqrt{2}\big| < \frac{1}{n}$ for all $n \in \mathbb{N}$. Clearly, $r \neq \sqrt{2}$, and by the Archimidean principle, there exists a $k \in \mathbb{N}$ such that 
    \[0 < \frac{1}{k} < |r - \sqrt{2}|\]
    which contradicts the above. 
  \end{solution}

  \begin{exercise}[Zorich 2.3.4]
    Show that the three lemmas above are equivalent to the axiom of completeness. 
  \end{exercise}
  \begin{solution}
    Note that from the proofs, completeness implies nested interval lemma, which implies compactness of closed intervals, which implies the Bolzano-Weierstrass theorem. So, it is sufficient to prove that Bolzano-Weierstrass theorem implies completeness to determine equivalence. There are not a lot of direct proofs, so we prove that Weierstrass implies nested interval, which implies completeness. 
    \begin{enumerate}
      \item (Weierstrass $\implies$ Nested) Assume that we have $\mathbb{R}$ with the Bolzano-Weierstrass theorem. Take the series of nested closed intervals 
      \[I = [a, b] \supset I_1 = [a_1, b_1] \supset I_2 = [a_2, b_2] \supset \ldots\]
      We see that $a \leq a_i \leq b$, so the infinite sequence of monotonically nondecreasing values $a_i$ is bounded. Therefore, it must have a limit point, which we will denote as $c$. We claim that $a_i \leq c$ for all $a_i$. Since if it were not, then $c < a_i$ for some $i$, and choosing $\epsilon = 0.5 (a_i - c)$, the $\epsilon$-neighborhood of $c$ will not contain $a_j$ for $j \geq i$ since 
      \[c < a_i \implies 0.5 c < 0.5 a_i \implies c + \epsilon = 0.5c + 0.5 a_i < a_i< a_{i+1} < \ldots\]. 
      With similar reasoning, we can conclude that $b_i \geq c$ for all $b_i$. This implies that $a_i \leq c \leq b_i$ for all $i$ which is equivalent to saying that $c \in [a_i, b_i] = I_i$ for all $i \in \mathbb{N}$. 
      
      \item (Nested $\implies$ LUB Principle) Let $X \subset \mathbb{R}$ be a set that is bounded above, with $b_1$ any upper bound. Since $X$ is nonempty, there exists $a_1 \in X$ that is not an upper bound (otherwise, $X$ would be a singleton set and it trivially has a least upper bound). Consider the well-defined interval $[a_0, b_0]$. Take the mean $m_0 = 0.5 (a_0 + b_0)$, and if $m_0$ is an upper bound, set it to $b_1$ (with $a_1 = a_0$) and $a_1$ if else (with $b_1 = b_0$). Then, we have a sequence of nested intervals 
      \[[a_0, b_0] \supset [a_1, b_1] \supset [a_2, b_2] \supset \ldots \]
      of decreasing lengths $|I_k| = \frac{1}{2^{k}} (b - a)$. All of them must contain a unique common point $c \in \mathbb{R}$ by the nested intervals lemma, which implies that 
      \[a_0 \leq a_1 \leq a_2 \leq \ldots \leq c \leq \ldots \leq b_2 \leq b_1 \leq b_0\]
      I claim two things: 
      \begin{enumerate}
          \item $c$ is an upper bound for $X$. Suppose it were not, then there exists some $x \in X$ such that $c < x$, and let the distance between them be $\epsilon = x - c > 0$. By AP, we can choose $k \in \mathbb{N}$ such that $\frac{1}{k} < \epsilon$. All the $b_n$ are upper bounds of $X$, so we have $x \leq b_n$. Subtracting $c$ on both sides gives 
          \[0 < x - c = \epsilon \leq b_n - c \leq |I_n| = \frac{1}{2^n} (b_0 - a_0)\]
          where the last inequality follows from $c \in I_n = [a_n, b_n]$, so the maximum distance it can be from the endpoint $b_n$ is $|I_n|$. The inequality above holds for all $n \in \mathbb{N}$, so increasing $n$ arbitrarily should decrease $\frac{1}{2^n} (b_0 - a_0)$ past $\epsilon$. To formalize this, we use the inequality
          \[\frac{1}{2^n} < \frac{1}{n} \text{ for all } n \in \mathbb{N}\]
          and so we have 
          \[\epsilon \leq b_n - c < \frac{1}{n} (b_0 - a_0)\]
          We choose the natural number $n = \lceil \frac{2(b_0 - a_0)}{\epsilon} \rceil$, which does not satisfy the inequality above since 
          \[\epsilon < \frac{1}{n} (b_0 - a_0) = \frac{1}{\lceil 2(b_0 - a_0)/\epsilon \rceil} (b_0 - a_0) \leq \frac{\epsilon}{2(b_0 - a_0)} (b_0 - a_0) = \frac{\epsilon}{2}\]
          This leads to a contradiction. 
          \item We now prove that $c$ is least. Assume that $c$ is not least $\implies$ there exists an upper bound $B$ such that $B < c$ and $x \leq B$ for all $x \in X$. \textbf{INCOMPLETE}
      \end{enumerate}
    \end{enumerate}
  \end{solution}

  \begin{exercise}[Zorich 2.4.1]
  Show that the set of real numbers has the same cardinality as the points of the interval $(-1, 1)$. 
  \end{exercise}
  \begin{solution}
    We define the bijective map $\rho: (-1, 1) \longrightarrow \mathbb{R}$ 
    \[p(x) = \begin{cases} 
    0 & \text{ if } x = 0 \\
    \frac{1}{x} & \text{ if } x \neq 0 
    \end{cases}\]
  \end{solution}

  \begin{exercise}[Zorich 2.4.2]
    Give an explicit one-to-one correspondence between 
    \begin{enumerate}
      \item the points of two open intervals 
      \item the points of two closed intervals 
      \item the point of a closed interval and an open interval 
      \item the points of the closed interval $[0, 1]$ and $\mathbb{R}$
    \end{enumerate}
  \end{exercise}
  \begin{solution}
    Listed. 
    \begin{enumerate}
      \item $\rho: (a, b) \longrightarrow (c, d)$ defined 
      \[\rho(x) = \frac{d - c}{b - a} (x - a) + c \]
      \item the extension of $\rho$ defined on (a) to $[a, b]$
      \item From (a) and (b), it suffices to prove a bijection from $(0, 1)$ to $[0, 1]$. We extract a countably infinite sequence from $(0, 1)$, say 
      \[x_1 = \frac{1}{3}, \; x_2 = \frac{1}{4}, \ldots, x_i = \frac{1}{i+2}\]
      Then, we define bijection $\rho: (0, 1) \longrightarrow [0, 1]$ as 
      \[\rho (x) = \begin{cases}
      x & \text{ if } x \not\in \{x_i\} \\
      0 & \text{ if } x = x_1 = \frac{1}{2} \\
      1 & \text{ if } x = x_2 = \frac{1}{3} \\
      x_{i-2} & \text{ if } x = x_i \text{ for } i > 2
      \end{cases}\]
      Colloquially, we extract a copy of $\mathbb{N}$ from $(0, 1)$ and use the bijection $\mathbb{N} \simeq \mathbb{N} \cup \{0, -1\}$ to ``shift'' the terms. 
      \item We compose the bijections $\rho_1 : [0, 1] \longrightarrow (0, 1)$ and $\rho_2: (0, 1) \longrightarrow \mathbb{R}$. 
    \end{enumerate}
  \end{solution}

  \begin{exercise}[Zorich 2.4.3]
    Show that 
    \begin{enumerate}
      \item every infinite set contains a countable subset
      \item the set of even integers has the same cardinality as the set of all natural numbers
      \item the union of an infinite set and an at most countable set has the same cardinality as the original infinite set. 
      \item the set of irrational numbers has the cardinality of the continuum 
      \item the set of transcendental numbers has the cardinality of the continuum
    \end{enumerate}
  \end{exercise}
  \begin{solution}
    Listed. 
    \begin{enumerate}
      \item Let $A$ be an infinite set. By axiom of choice, choose $a_0 \in A$. Then, $A \setminus \{a_0\} \neq \emptyset$ since $A$ is infinite. By induction, assume you have chosen $a_0, a_1, \ldots, a_k \in A$. Then, since $A$ is infinite, $A \setminus \{a_0, a_1, \ldots, a_k\} \neq \emptyset$, so we can choose $a_{k+1} \in A \setminus \{a_0, \ldots, a_k\}$. Thus, we have constructed a countable subset $\{a_k\}_{k \in \mathbb{N}}$ of $A$. 
      \item Given the quotient ring $2\mathbb{Z}$, define the bijection $\rho: 2\mathbb{Z} \longrightarrow \mathbb{N}$ as 
      \[p(x) = \begin{cases} 
      x + 2 & \text{ if } x \geq 0 \\
      -x - 1 & \text{ if } x < 0 
      \end{cases}\]
      \item From (a), we can extract a countable set from original set $A$, call it $X$. Since the product of countable sets is countable ($\mathbb{N} \cup \mathbb{N}$ is countable), we can define a bijection $\Tilde{\rho}: X \longrightarrow X \cup B$. Therefore, we can define a bijection $\rho: A \longrightarrow A \cup B$ as 
      \[\rho(x) = \begin{cases} 
      x & \text{ if } x \in A \setminus X \\
      \Tilde{\rho}(x) & \text{ if } x \in X
      \end{cases}\]
      \item $\mathbb{Q}$ is countable and $\mathbb{R}$ is uncountable. So, $\mathbb{R} \setminus \mathbb{Q}$ must be uncountable since if it were countable, then the union of the rationals and irrationals, which is $\mathbb{R}$, would be countable. 
      \item It suffices to prove that the set of algebraic numbers (numbers that are possible roots of a polynomial with integer coefficients with leading coefficient nonzero) is countable, since we can apply (d) right after. The set of all $k$th degree polynomials with integer coefficients is isomorphic to $\mathbb{Z}^k$ through the map 
      \[a_k x^k + a_{k-1} x^{k-1} + \ldots + a_2 x^2 + a_1 x^1 + a_0 \mapsto (a^{k-1}, a^{k-2}, \ldots, a_1, a_0)\]
      and the union of these countable sets (minus the $0$ map) 
      \[P = \bigg(\bigcup_{k = 1}^\infty \mathbb{Z}^k\bigg) \setminus \{0\} = \big(\mathbb{Z} \setminus \{0\}\big) \cup \mathbb{Z}^2 \cup \ldots \]
      is countable. For any element in $\mathbb{Z}^k$, there are at most $k$ real roots, and so we can define the set of roots of an element $z \in \mathbb{Z}^k \subset P$ as a $j$-tuple of algebraic numbers, which can have at most $j=k$ roots. 
      \[r(z) = \underbrace{(r_{1z}, r_{2z}, \ldots, r_{jz})}_{j \leq k}\]
      Therefore, the union of all these $j$-tuples for all $z \in P$ 
      \[\bigcup_{z \in P} r(z) = \bigcup_{k = 1}^\infty \bigcup_{z \in \mathbb{Z}^k} r(z)\]
      is a countable union of a countable union of finite sets, making it countable. 
    \end{enumerate}
  \end{solution}

  \begin{exercise}[Zorich 2.4.4]
    Show that
    \begin{enumerate}
      \item the set of increasing sequences of natural numbers has the same cardinality as the set of fractions of the form $0.\alpha_1 \alpha_2 \ldots$ 
      \item the set of all subsets of countable set has cardinality of the continuum
    \end{enumerate}
  \end{exercise}
  \begin{solution}
    Listed. 
    \begin{enumerate}
      \item Given a sequence of increasing naturals $S = (n_1, n_2, \ldots)$, we can define a binary expansion $0.\alpha_1 \alpha_2 \ldots$ where $\alpha_i = 1$ if and only if $i \in \mathbb{N}$ is in $S$ and $\alpha_i = 0$ if not. This is clearly a bijection. 
      \item The set of all segments of increasing natural is equipotent with $2^\mathbb{N}$, since the elements of each sequence define a subset of $\mathbb{N}$. Cantor's diagonalization argument proves that the set of infinite binary expansions is uncountable, and by (a), this proves that $2^\mathbb{N}$ is uncountable. 
    \end{enumerate}
    This is very interesting since $\mathbb{N} \simeq \mathbb{R}$, but $2^{\mathbb{N}} \simeq \mathbb{R}$, and the set of all infinite $q$-ary expansions is equipotent to $\mathbb{R}$ too. 
  \end{solution}

  \begin{exercise}[Zorich 2.4.5]
    Show that 
    \begin{enumerate}
      \item the set $\mathcal{P}(X)$ of subsets of a set $X$ has the same cardinality as the set of all functions $f: X \longrightarrow \{0, 1\}$. 
      \item for a finite set $X$ of $n$ elements, $\card{\mathcal{P}(X)} = 2^n$ 
      \item one can write $\card{\mathcal{P}(X)} = 2^{\text{card}{X}}$, which implies $\text{card}\, \mathcal{P}(\mathbb{N}) = 2^{\card \mathbb{N}} = \card \mathbb{R}$ 
      \item for any set $X$, $\card{X} < 2^{\card{X}}$
    \end{enumerate}
  \end{exercise}
  \begin{solution}
    Listed. 
    \begin{enumerate}
      \item An element $Y \in \mathcal{P}(X)$ is a subset of $X$ by definition. Letting 
      \[f_Y (x) = \begin{cases} 0 & \text{ if } x \not\in Y \\
      1 & \text{ if } x \in Y \end{cases}\]
      we can construct the bijective map $Y \mapsto f_Y$. 
      \item We can prove this using the identity (which can be proved using induction) 
        \[\sum_{k=0}^n \binom{n}{k} = 2^n \]
      \item Let $F(X; \, \{0, 1\})$ be the set of all binary valued functions from $X$ to $\{0, 1\}$. From (a), $\card{\mathcal{P}(X)} \simeq F(X; \, \{0, 1\})$. Each binary-valued function $f$ is determined by the assignment $f(x)$ for each $x \in X$. Since $f(x)$ has two possible values, the assignment of $f(x)$ for all $x \in X$ has $\{0, 1\}^{\card{X}}$ possible choices. This gives another bijection $F(X; \, \{0, 1\}) \simeq \{0, 1\}^{\card{X}}$, so 
      \[\mathcal{P}(X) \simeq \{0, 1\}^{\card{X}} \implies \card{\mathcal{P}(x)} = \card(\{0, 1\}^{\card{X}}) = 2^{\card{X}} \]
      \item If $X$ is finite, then letting $n = \card{X}$, we can simply prove $n < 2^n$ by induction (which we will not do here). If $X$ is countable, then $\mathcal{P}(X)$ is uncountable (from 2.4.4) and so using (c), 
      \[\card{X} = \card{\mathbb{N}} < \card{\mathbb{R}} = \card{\mathcal{P}(X)} = 2^{\card{X}}\]
      For uncountable sets (and for the two cases mentioned above), we can use Cantor's theorem, which states that $\card{X} < \card{\mathcal{P}(X)}$, and so using (c), we have $\card{X} < \card{\mathcal{P}(X)} = 2^{\card{X}}$. 
    \end{enumerate}
  \end{solution}

  \begin{exercise}[Zorich 2.4.6]
    Let $X_1, \ldots, X_m$ be a finite system of finite sets. Show that 
    \begin{align*}
      \card \bigg( \bigcup_{i=1}^m X_i \bigg) & = \sum_{i_1} \card{X_{i_1}} - \sum_{i_1 < i_2} \card(X_{i_1} \cap X_{i_2}) + \ldots \\
      & \sum_{i_1 < i_2 < i_3} \card(X_{i_1} \cap X_{i_2} \cap X_{i_3}) - \ldots + (-1)^{m-1} \card (X_1 \cap \ldots \cap X_m) \\
      & = \sum_{k=1}^{m} \sum_{1 \leq i_1 \ldots i_k \leq m} (-1)^{k-1} \, \card\bigg( \bigcap_{j=1}^k X_{i_j} \bigg)
    \end{align*}
  \end{exercise}
  \begin{solution}
    Ignoring Russell's paradox (defining the universe set of all sets), we can use the commutative, associative, and distributive properties of $\cup, \cap$ on the algebra of sets. We prove using induction on $m$. For $m=1$, we trivially have $\card{X_1} = \card{X_1}$, and for $m = 2$, we claim 
    \[\card( X_1 \cup X_2) = \card(X_1) + \card(X_2) - \card(X_1 \cap X_2)\]
    $X_1$ and $X_2 \setminus X_1$ are clearly exclusive sets by definition, with $X_1 \cup X_2 = X_1 \cup (X_2 \setminus X_1)$, so 
    \[\card(X_1 \cup X_2) = \card\big( X_1 \cup (X_2 \setminus X_1) \big) = \card(X_1) + \card(X_2 \setminus X_1) \tag{2} \label{CardOne}\]
    By definition, the set $X_2 \setminus X_1$ and $X_1 \cap X_2$ are disjoint and satisfies $X_2 = (X_2 \setminus X_1) \cup (X_1 \cap X_2)$ (also by definition), so 
    \[\card(X_2) = \card(X_2 \setminus X_1) + \card(X_1 \cap X_2) \tag{3} \label{CardTwo}\]
    and substituting \eqref{CardTwo} into \eqref{CardOne} gives the claim for $m=2$. Assuming that the claim is satisfied for some $m$, we have 
    \begin{align*}
      \card \bigg( \bigcup_{i=1}^{m+1} X_i \bigg) & = \card \bigg( \bigg[ \bigcup_{i=1}^m X_i \bigg] \cup X_{m+1} \bigg) & \\
      & = \card \bigg( \bigcup_{i=1}^k X_i \bigg) + \card(X_{m+1}) - \card\bigg( \bigg[ \bigcup_{i=1}^m X_i\bigg] \cap X_{m+1} \bigg) & (\text{claim for } m=2) \\
      & = \card \bigg( \bigcup_{i=1}^k X_i \bigg) + \card(X_{m+1}) - \card\bigg(\bigcup_{i=1}^m (X_i \cap X_{m+1}) \bigg) & (\text{distributive prop.})  \\
      & = \sum_{k=1}^m \sum_{1 \leq i_1 \ldots i_k \leq m} (-1)^{k-1} \card\bigg( \bigcap_{j=1}^k X_{i_j} \bigg) + \card(X_{m+1}) \\ 
      & \;\;\;\;\;\; - \sum_{k=1}^m \sum_{1 \leq i_1 \ldots i_k \leq m} (-1)^{k-1} \card\bigg( \bigcap_{j=1}^k (X_{i_j} \cap X_{m+1}) \bigg) 
    \end{align*}
    With a bit of thought, we can see that the $k$th term of the second summation contributes to adding another term to the $k+1$th summation term of the first. Therefore, we must try to shift the summation over by $1$ index. Let us simplify this by taking the summations and extracting the first and last term, respectively. We have 
    \begin{align*}
    \sum_{k=1}^m \sum_{1 \leq i_1 \ldots i_k \leq m} (-1)^{k-1} \card\bigg( \bigcap_{j=1}^k X_{i_j} \bigg) &= \sum_{1 \leq i_1 \ldots i_k \leq m} \card(X_{i_1}) \\
    &+ \sum_{k=2}^m \sum_{1 \leq i_1 \ldots i_k \leq m} (-1)^{k-1} \card\bigg( \bigcap_{j=1}^k X_{i_j} \bigg)
    \end{align*}
    and 
    \begin{align*}
      \sum_{k=1}^m \sum_{1 \leq i_1 \ldots i_k \leq m} & (-1)^{k-1} \card\bigg( \bigcap_{j=1}^k (X_{i_j} \cap X_{m+1}) \bigg) \\
      & = \sum_{k=1}^m \sum_{1 \leq i_1 \ldots i_k \leq m} (-1)^{k-1} \card\bigg( \bigg[ \bigcap_{j=1}^k X_{i_j} \bigg] \cap X_{m+1} \bigg) \\
      & =  \sum_{k=1}^{m-1} \sum_{1 \leq i_1 \ldots i_k \leq m} (-1)^{k-1} \card\bigg( \bigcap_{j=1}^k (X_{i_j} \cap X_{m+1}) \bigg) + (-1)^{m-1} \card\bigg( \bigcap_{j=1}^{m+1} X_j \bigg) \\
      & = \sum_{k=2}^{m} \sum_{1 \leq i_1 \ldots i_k \leq m} (-1)^{k-2} \card\bigg( \bigcap_{j=1}^{k-1} (X_{i_j} \cap X_{m+1}) \bigg) + (-1)^{m-1} \card\bigg( \bigcap_{j=1}^{m+1} X_j \bigg)
    \end{align*}
    So subtracting the summations gives 
    \begin{align*}
      \sum_{k=1}^m & \sum_{1 \leq i_1 \ldots i_k \leq m} (-1)^{k-1} \card\bigg( \bigcap_{j=1}^k X_{i_j} \bigg) - \sum_{k=1}^m \sum_{1 \leq i_1 \ldots i_k \leq m} (-1)^{k-1} \card\bigg( \bigcap_{j=1}^k (X_{i_j} \cap X_{m+1}) \bigg) + |X_{m+1}| \\
      & = \sum_{1 \leq i_1 \ldots i_k \leq m} \card(x_i) + \sum_{k=2}^m \sum_{1 \leq i_1 \ldots i_k \leq m} (-1)^{k-1} \card\bigg( \bigcap_{j=1}^k X_{i_j} \bigg) + \card(X_{m+1})\\
      & \;\;\;\;\; + \sum_{k=2}^{m} \sum_{1 \leq i_1 \ldots i_k \leq m} (-1)^{k-1} \card\bigg( \bigcap_{j=1}^{k-1} (X_{i_j} \cap X_{m+1}) \bigg) + (-1)^{m} \card\bigg( \bigcap_{j=1}^{m+1} X_j \bigg) \\
      & = \sum_{1 \leq i_1 \ldots i_k \leq m+1} \card(X_i) + \sum_{k=2}^m \sum_{1 \leq i_1 \ldots i_k \leq m} (-1)^{k-1} \Bigg[ \card\bigg( \bigcap_{j=1}^k X_{i_j} \bigg) \\
      & +  \card\bigg( \bigg[ \bigcap_{j=1}^{k-1} X_{i_j} \bigg] \cap X_{m+1} \bigg) \Bigg] + (-1)^{m} \card\bigg( \bigcap_{j=1}^{m+1} X_j \bigg) 
    \end{align*}
    and since the set of sequences of $k$ terms bounded by $m+1$ (of form $1 \leq i_1 \ldots i_k \leq m+1$) is the set of sequences of $k$ terms bounded by $m$ (of form $1 \leq i_1 \ldots i_k \leq m$) unioned with the set of sequences of $k$ terms with max element $m+1$ (of form $1 \leq i_1 \ldots i_k = m+1$), we have 
    \[\sum_{1 \leq i_1 \ldots i_k \leq m} (-1)^{k-1} \Bigg[ \card\bigg( \bigcap_{j=1}^k X_{i_j} \bigg) +  \card\bigg( \bigg[ \bigcap_{j=1}^{k-1} X_{i_j} \bigg] \cap X_{m+1} \bigg) \Bigg] = \sum_{1 \leq i_1 \ldots i_k \leq m+1} \card\bigg( \bigcap_{j=1}^k X_{i_j} \bigg)\]
    and therefore, substituting the above and observing that the independent terms are the first and last terms of the summation gives 
    \begin{align*}
      \card \bigg( \bigcup_{i=1}^{m+1} X_i \bigg) & = \sum_{1 \leq i_1 \ldots i_k \leq m+1} \card(X_i) + \sum_{k=2}^m \sum_{1 \leq i_1 \ldots i_k \leq m+1} (-1)^{k-1} \card\bigg( \bigcap_{j=1}^k X_{i_j} \bigg) \\ 
      & \;\;\;\;\;\;\;\;\;\; + \ldots + (-1)^{m} \card\bigg( \bigcap_{j=1}^{m+1} X_j \bigg) \\
      & = \sum_{k=1}^{m+1} \sum_{1 \leq i_1 \ldots i_k \leq m+1} (-1)^{k-1} \card\bigg( \bigcap_{j=1}^k X_{i_j} \bigg) 
    \end{align*}
  \end{solution}

  \begin{exercise}[Zorich 2.4.7]
    On the closed interval $[0, 1] \subset \mathbb{R}$, describe the sets of numbers $x \in [0, 1]$ whose ternary representation $x = 0.\alpha_1 \alpha_2 \ldots $, $\alpha_i \in \{0, 1, 2\}$ has the property. 
    \begin{enumerate}
      \item $\alpha_1 \neq 1$
      \item $\alpha_1 \neq 1$ and $\alpha_2 \neq 1$ 
      \item For all $i \in \mathbb{N}$, $\alpha_i \neq 1$ (the Cantor set)
    \end{enumerate}
  \end{exercise}
  \begin{solution}
    Listed. 
    \begin{enumerate}
      \item $[0, \frac{1}{3}) \cup [\frac{2}{3}, 1)$ 
      \item $[0, \frac{1}{9}) \cup [\frac{2}{9}, \frac{3}{9}) \cup [\frac{6}{9}, \frac{7}{9}) \cup [\frac{8}{9}, 1)$
      \item Made by recursively removing the middle third of every partitioned intervals. 
    \end{enumerate}
  \end{solution}

  \begin{exercise}[Zorich 2.4.8]
    Show that 
    \begin{enumerate}
      \item the set of numbers $x \in [0, 1]$ whose ternary representation does not contain $1$ has the same cardinality as the set of all numbers whose binary representation has the form $0.\beta_1 \beta_2 \ldots$ 
      \item the Cantor set has the same cardinality as the closed interval $[0, 1]$ 
    \end{enumerate}
  \end{exercise}
  \begin{solution}
    Listed. 
    \begin{enumerate}
      \item We can define a bijection $0.\alpha_1 \alpha_2 \ldots \mapsto 0.\beta_1 \beta_2 \ldots$ as $\alpha_i = 0 \iff \beta_i = 0$ and $\alpha_i = 1 \iff \beta_i = 2$. 
      \item The map above defines a bijection between the Cantor set and the set of all infinite binary expansions in $[0, 1]$, which is uncountable by Cantor's diagonalization theorem. 
    \end{enumerate}
  \end{solution}


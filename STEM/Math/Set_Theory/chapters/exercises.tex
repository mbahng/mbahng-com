\section{Exercises}

  \begin{exercise}[Math 531 Spring 2025, PS2.6]
    Assume that $S$ is a set with exactly $n$ elements. Assume that $T : S \to S$.
    Prove that there exists some $x \in S$ so that
    \begin{equation}
      T^j(x) = x,
    \end{equation}
    for some $j \in \{1,2,...,n\}$. Here $T^j$ means the composition of $T$ with itself
    $j$-times.
  \end{exercise}
  \begin{solution}
    Assume that the statement is false, and there exists no such $x \in S$. Let's choose any $x \in S$ and write out 
    \begin{equation}
      x = T^0 (x) , T^1 (x), T^2 (x), \ldots, T^n (x)
    \end{equation}
    These are $n+1$ elements living in a space $S$ of size $n$, so by pigeonhole principle there exists a repeat. Let us choose any of these repeats and label them $0 \leq i < j \leq n$ s.t. $T^i (x) = T^j (x)$. It cannot be the case that $i = 0$ since we assumed that it was false. Therefore, it must be the case that $1 \leq i < j \leq n \implies j - i \leq n-1$. Consider the sequence 
    \begin{equation}
      y = T^0 (y) = T^i (x), T^1 (y) = T^{i+1} (x), \ldots, T^n (y)
    \end{equation}
    Starting from $y = T^i (x) \in S$. Since $0 < j - i \leq n - 1$, we know that $T^{j-i} (y) = T^j (x)$ lies in this sequence. Since both $y = T^0 (y) = T^i (x)$ and $T^{j-i} (y) = T^j (x)$ are equal and present, we have shown an instance of when this claim is true, and the statement is true. 
  \end{solution}

  \begin{exercise}[Shifrin Appendix A.3.1]
    Consider the following relations on the set $\mathbb{Z}$:
    \begin{enumerate}
      \item[(i)] $(a, b) \in R_1$ \quad if $ab \geq 0$
      \item[(ii)] $(a, b) \in R_2$ \quad if $ab > 0$
      \item[(iii)] $(a, b) \in R_3$ \quad if $ab > 0$ or $a = b = 0$
    \end{enumerate}
    Decide whether each is an equivalence relation. (If not, which requirements fail?)
  \end{exercise}
  \begin{solution}
    Listed. 
    \begin{enumerate}
      \item Not a relation since transitivity fails. $4 \sim 0$ since $4 \cdot 0 = 0 \geq 0$. $0 \sim -2$ since $0 \cdot -2 = 0 \geq 0$. But $4 \not\sim -2$ since $4 \cdot -2 = -8 < 0$. 
      \item Not a relation since $0$ is not related to itself. $0 \not\sim 0$ since $0 \cdot 0 = 0 \not> 0$. 
      \item It is a relation. 
    \end{enumerate}
  \end{solution}

  \begin{exercise}[Shifrin Appendix A.3.2]
    Find the flaw in the so-called proof in the Remark on p. 380.
  \end{exercise}

  \begin{exercise}[Shifrin Appendix A.3.3]
    Define a relation on $\mathbb{R}$ as follows: $x \sim y$ if and only if $x - y$ is an integer. Prove that $\sim$ is an equivalence relation and describe the set of equivalence classes.
  \end{exercise}
  \begin{solution}
    We prove the three properties. 
    \begin{enumerate}
      \item \textit{Identity}. For $x \in \mathbb{R}$, $x - x = 0 \in \mathbb{Z}$ so $x \sim x$. 
      \item \textit{Symmetricity}. For $x, y \in \mathbb{R}$, if $x - y \in \mathbb{Z}$, then by the ring properties $-(x - y) = -1 \cdot (x - y) = -1 \cdot x + -1 \cdot -y) = -x + y = y - x \in \mathbb{Z}$ and so $y \sim x$. 
      \item \textit{Transitivity}. For $x, y, z \in \mathbb{R}$, if $x \sim y$ and $y \sim z$. Then $x - y, y - z \in \mathbb{Z}$. Since $\mathbb{Z}$ is closed under addition and addition is associative, we have 
        \begin{equation}
          (x - y) + (y - z) = x + (-y + y) - z = x + 0 + -z = x - z \in \mathbb{Z}
        \end{equation}
        and so $x \sim z$. 
    \end{enumerate}
    The set of all equivalence classes can be represented by the interval $[0, 1)$, where each $x \in \mathbb{R}$ gets mapped to $x \pmod{1}$. 
  \end{solution}

  \begin{exercise}[Shifrin Appendix A.3.4]
    Define a relation on $\mathbb{N}$ as follows: $x \sim y$ if and only if $x$ and $y$ have the same last digit in their base-ten representation. Prove that $\sim$ is an equivalence relation, and describe the set of equivalence classes.
  \end{exercise}

  \begin{exercise}[Shifrin Appendix A.3.5]
    Define a relation on $\mathbb{R}^2$ as follows: $(x_1, x_2) \sim (y_1, y_2)$ if and only if $x_1^2 + x_2^2 = y_1^2 + y_2^2$. Prove that $\sim$ is an equivalence relation, and describe the set of equivalence classes.
  \end{exercise}

  \begin{exercise}[Shifrin Appendix A.3.6]
    \begin{enumerate}
      \item[(a)] Define a relation on $\mathbb{R}$ as follows: $x$ and $y$ are related if $|x - y| < 1$. Decide whether this is an equivalence relation.
      \item[(b)] Define an equivalence relation on $\mathbb{R}$ whose equivalence classes are intervals of length 1.
    \end{enumerate}
  \end{exercise}

  \begin{exercise}[Shifrin Appendix A.3.7]
    Which of the following functions $f: \mathbb{Q} \to \mathbb{Q}$ are well-defined?
    \begin{enumerate}
      \item[(a)] $f(\frac{a}{b}) = \frac{a+1}{b+1}$
      \item[(b)] $f(\frac{a}{b}) = \frac{a+b}{b}$
      \item[(c)] $f(\frac{a}{b}) = \frac{2a^2}{3b^2}$
      \item[(d)] $f(\frac{a}{b}) = \frac{b}{a}$
      \item[(e)] $f(\frac{a}{b}) = \frac{a^2+ab+b^2}{a^2+b^2}$
    \end{enumerate}
  \end{exercise}

  \begin{exercise}[Shifrin Appendix A.3.8]
    Define an equivalence relation on $X = \mathbb{N} \times \mathbb{N}$ as follows:
    $$(a, b) \sim (c, d) \Leftrightarrow a + d = c + b.$$
    \begin{enumerate}
      \item[(a)] Prove that $\sim$ is indeed an equivalence relation.
      \item[(b)] Identify the set of equivalence classes.
    \end{enumerate}
  \end{exercise}

  \begin{exercise}[Shifrin Appendix A.3.9]
    Define a relation on $\mathbb{N}$ as follows: $x \sim y$ if and only if there are integers $j$ and $k$ so that $x|y^j$ and $y|x^k$.
    \begin{enumerate}
      \item[(a)] Show that $\sim$ is an equivalence relation.
      \item[(b)] Determine the equivalence classes $[1]$, $[2]$, $[9]$, $[10]$, and $[20]$.
      \item[(c)] Describe explicitly the equivalence classes $[x]$ in general.
    \end{enumerate}
  \end{exercise}
  \begin{solution}
    We first prove the three properties. 
    \begin{enumerate}
      \item \textit{Identity}. Since we can set $j = k = 1$ to get $x | x$, trivially $x \sim x$. 
      \item \textit{Symmetricity}. Assume that $x \sim y$. Then there exists some $j, k$ such that $x | y^j$ and $y | x^k$. We can just swap $j, k$ to identify the integers that hold for $y \sim x$. 
      \item \textit{Transitivity}. Assume that $x \sim y, y \sim z$. Then there exists some integers $i, j, k, l$ s.t. $x | y^i, y | x^j, y | z^k, z | y^l$. We would like to find integers $a, b$ such that $x | z^a$ and $z | x^b$. We can set $a = ik$ and $b = jl$. By replacing unimportant integer constants with $\ast$, we write 
      \begin{align}
        z^{ik} & = (z^k)^i = (\ast y)^i = \ast y^i = \ast x \\
        x^{jl} & = (x^j)^l = (\ast y)^l = \ast y^l = \ast z
      \end{align}
      and therefore $x \sim z$. 
    \end{enumerate}
    The equivalence classes are as shown, with brief explanations. 
    \begin{enumerate}
      \item $[1] = \{1\}$. Since we are looking for naturals where $y | 1^k$, i.e. $y | 1$, $y = 1$ is the only natural. 
      \item $[2] = \{2^n\}_{n=1}^\infty$, i.e. all naturals with powers of $2$. $2 | y^j$ tells me that $y$ must at least be even, but $y | 2^k$ tells me that $y$ must be a power of $2$. 
      \item $[9] = \{3^n\}_{n=1}^\infty$. $9 | y^j$ tells me that as long as $y$ is a multiple of $3$, we can set $j \geq 2$. $y | 9^k$ tells me that $y$ must not have any other prime divisors. 
      \item $[10] = \{2^i \cdot 5^j\}_{i, j \in \mathbb{N}}$. $10 | y^j$ tells me that $y$ must have at least $2$ and $5$ as prime divisors. $y | 10^k$ tells me that it cannot have any other prime divisors. 
      \item $[20] = [10]$. $20 | y^j$ tells me that $y$ must have at least $2$ and $5$, and then by setting $j = 2$, $y^2$ is guaranteed to be divisible by $20$. $y | 20^k$ tells me that it cannot have any other divisors. This is the same definition as that of $[10]$.  
    \end{enumerate}
    The equivalence class $[x]$ is described as such. Given a natural $x \in \mathbb{N}$, let $P = \{p_1, \ldots, p_n\} \subset \mathbb{N}$ be the set of prime divisors of $x$, which is guaranteed to be finite and unique by the fundamental theorem of arithmetic. Then, 
    \begin{equation}
      [x] \coloneqq \{ p_1^{i_1} \cdot p_2^{i_2} \cdot \ldots \cdot p_n^{i_n} \mid i_1, \ldots, i_n \geq 1 \}
    \end{equation}
  \end{solution}

  \begin{exercise}[Shifrin Appendix A.3.10]
    Given a function $f: S \to T$, consider the following relation on $S$:
    $$x \sim y \Leftrightarrow f(x) = f(y).$$
    \begin{enumerate}
      \item[(a)] Prove that $\sim$ is an equivalence relation.
      \item[(b)] Prove that if $f$ maps onto $T$, then there is a one-to-one correspondence between the set of equivalence classes and $T$.
    \end{enumerate}
  \end{exercise}
  
  \begin{exercise}[Math 531 Spring 2025 PS1.2]
    Prove that $\mathbb{Q}$ is countable. 
  \end{exercise}
  \begin{solution}
    Proved in theorem above. 
  \end{solution}

  \begin{exercise}[Math 531 Spring 2025, PS2.5]
    Prove that if $X$ is a set and $A,B,C \subset X$, we have that
    \begin{equation}
      A \cap (B \cup C) = (A \cap B) \cup (A \cap C),
    \end{equation}
    \begin{equation}
      A \cup (B \cap C) = (A \cup B) \cap (A \cup C),
    \end{equation}
    \begin{equation}
      X \setminus (A \cup B) = (X \setminus A) \cap (X \setminus B),
    \end{equation}
    \begin{equation}
      X \setminus (A \cap B) = (X \setminus A) \cup (X \setminus B).
    \end{equation}
  \end{exercise}
  \begin{solution}
    Proved in deMorgan's laws. 
  \end{solution}

  \begin{exercise}[Shifrin Abstract Algebra Appendix 2.3]
    Let $f: X \to Y$. Let $A,B \subset X$ and $C,D \subset Y$. Prove or give a counterexample (if possible, provide sufficient hypotheses for each statement to be valid):
    \begin{enumerate}
      \item $f(A) \cup f(B) = f(A \cup B)$
      \item $f(A) \cap f(B) = f(A \cap B)$
      \item $f(A - B) = f(A) - f(B)$
      \item $f^{-1}(C) \cup f^{-1}(D) = f^{-1}(C \cup D)$
      \item $f^{-1}(C) \cap f^{-1}(D) = f^{-1}(C \cap D)$
      \item $f^{-1}(C - D) = f^{-1}(C) - f^{-1}(D)$
      \item $f(f^{-1}(C)) = C$
      \item $f^{-1}(f(A)) = A$
    \end{enumerate} 
  \end{exercise}
  \begin{solution}
    Listed. 
    \begin{enumerate}
      \item 
    \end{enumerate}
  \end{solution}

  \begin{exercise}[Munkres 1.1]
    Check the distributive laws for $\cup$ and $\cap$ and DeMorgan's laws.
  \end{exercise}
  \begin{solution}
    
  \end{solution}

  \begin{exercise}[Munkres 1.2]
    Determine which of the following statements are true for all sets $A$, $B$, $C$, and $D$. If a double implication fails, determine whether one or the other of the possible implications holds. If an equality fails, determine whether the statement becomes true if the "equals" symbol is replaced by one or the other of the inclusion symbols $\subset$ or $\supset$.
    \begin{enumerate}
      \item $A \subset B$ and $A \subset C \iff A \subset (B \cup C)$.
      \item $A \subset B$ or $A \subset C \iff A \subset (B \cup C)$.
      \item $A \subset B$ and $A \subset C \iff A \subset (B \cap C)$.
      \item $A \subset B$ or $A \subset C \iff A \subset (B \cap C)$.
      \item $A - (A - B) = B$.
      \item $A - (B - A) = A - B$.
      \item $A \cap (B - C) = (A \cap B) - (A \cap C)$.
      \item $A \cup (B - C) = (A \cup B) - (A \cup C)$.
      \item $(A \cap B) \cup (A - B) = A$.
      \item $A \subset C$ and $B \subset D \Rightarrow (A \times B) \subset (C \times D)$.
      \item The converse of (j).
      \item The converse of (j), assuming that $A$ and $B$ are nonempty.
      \item $(A \times B) \cup (C \times D) = (A \cup C) \times (B \cup D)$.
      \item $(A \times B) \cap (C \times D) = (A \cap C) \times (B \cap D)$.
      \item $A \times (B - C) = (A \times B) - (A \times C)$.
      \item $(A - B) \times (C - D) = (A \times C - B \times C) - A \times D$.
      \item $(A \times B) - (C \times D) = (A - C) \times (B - D)$.
    \end{enumerate}
  \end{exercise}
  \begin{solution}
    
  \end{solution}

  \begin{exercise}[Munkres 1.3]
    \begin{enumerate}
      \item Write the contrapositive and converse of the following statement: "If $x < 0$, then $x^2 - x > 0$," and determine which (if any) of the three statements are true.
      \item Do the same for the statement "If $x > 0$, then $x^2 - x > 0$."
    \end{enumerate}
  \end{exercise}
  \begin{solution}
    
  \end{solution}

  \begin{exercise}[Munkres 1.4]
    Let $A$ and $B$ be sets of real numbers. Write the negation of each of the following statements:
    \begin{enumerate}
      \item For every $a \in A$, it is true that $a^2 \in B$.
      \item For at least one $a \in A$, it is true that $a^2 \in B$.
      \item For every $a \in A$, it is true that $a^2 \notin B$.
      \item For at least one $a \notin A$, it is true that $a^2 \in B$.
    \end{enumerate}
  \end{exercise}
  \begin{solution}
    
  \end{solution}

  \begin{exercise}[Munkres 1.5]
    Let $\mathcal{A}$ be a nonempty collection of sets. Determine the truth of each of the following statements and of their converses:
    \begin{enumerate}
      \item $x \in \bigcup_{A \in \mathcal{A}} A \Rightarrow x \in A$ for at least one $A \in \mathcal{A}$.
      \item $x \in \bigcup_{A \in \mathcal{A}} A \Rightarrow x \in A$ for every $A \in \mathcal{A}$.
      \item $x \in \bigcap_{A \in \mathcal{A}} A \Rightarrow x \in A$ for at least one $A \in \mathcal{A}$.
      \item $x \in \bigcap_{A \in \mathcal{A}} A \Rightarrow x \in A$ for every $A \in \mathcal{A}$.
    \end{enumerate}
  \end{exercise}
  \begin{solution}
    
  \end{solution}

  \begin{exercise}[Munkres 1.7]
    Given sets $A$, $B$, and $C$, express each of the following sets in terms of $A$, $B$, and $C$, using the symbols $\cup$, $\cap$, and $-$.

    $D = \{x \mid x \in A \text{ and } (x \in B \text{ or } x \in C)\}$,

    $E = \{x \mid (x \in A \text{ and } x \in B) \text{ or } x \in C\}$,

    $F = \{x \mid x \in A \text{ and } (x \in B \Rightarrow x \in C)\}$.
  \end{exercise}
  \begin{solution}
    
  \end{solution}

  \begin{exercise}[Munkres 1.8]
    If a set $A$ has two elements, show that $\mathcal{P}(A)$ has four elements. How many elements does $\mathcal{P}(A)$ have if $A$ has one element? Three elements? No elements? Why is $\mathcal{P}(A)$ called the power set of $A$?
  \end{exercise}
  \begin{solution}
    
  \end{solution}

  \begin{exercise}[Munkres 1.9]
    Formulate and prove DeMorgan's laws for arbitrary unions and intersections.
  \end{exercise}
  \begin{solution}
    
  \end{solution}

  \begin{exercise}[Munkres 1.10]
    Let $\mathbb{R}$ denote the set of real numbers. For each of the following subsets of $\mathbb{R} \times \mathbb{R}$, determine whether it is equal to the cartesian product of two subsets of $\mathbb{R}$.
    \begin{enumerate}
      \item $\{(x, y) \mid x \text{ is an integer}\}$.
      \item $\{(x, y) \mid 0 < y \leq 1\}$.
      \item $\{(x, y) \mid y > x\}$.
      \item $\{(x, y) \mid x \text{ is not an integer and } y \text{ is an integer}\}$.
      \item $\{(x, y) \mid x^2 + y^2 < 1\}$.
    \end{enumerate}
  \end{exercise}
  \begin{solution}
    
  \end{solution}

  \begin{exercise}[Munkres 2.1]
    Let $f: A \to B$. Let $A_0 \subset A$ and $B_0 \subset B$.
    \begin{enumerate}
      \item Show that $A_0 \subset f^{-1}(f(A_0))$ and that equality holds if $f$ is injective.
      \item Show that $f(f^{-1}(B_0)) \subset B_0$ and that equality holds if $f$ is surjective.
    \end{enumerate}
  \end{exercise}
  \begin{solution}
    Solved in $\ref{preserve_back_forth}$. 
  \end{solution}

  \begin{exercise}[Munkres 2.2]
    Let $f: A \to B$ and let $A_i \subset A$ and $B_i \subset B$ for $i = 0$ and $i = 1$. Show that $f^{-1}$ preserves inclusions, unions, intersections, and differences of sets:
    \begin{enumerate}
      \item[a)] $B_0 \subset B_1 \Rightarrow f^{-1}(B_0) \subset f^{-1}(B_1)$.
      \item[b)] $f^{-1}(B_0 \cup B_1) = f^{-1}(B_0) \cup f^{-1}(B_1)$.
      \item[c)] $f^{-1}(B_0 \cap B_1) = f^{-1}(B_0) \cap f^{-1}(B_1)$.
      \item[d)] $f^{-1}(B_0 - B_1) = f^{-1}(B_0) - f^{-1}(B_1)$.
    \end{enumerate}
    Show that $f$ preserves inclusions and unions only:
    \begin{enumerate}[resume]
      \item[e)] $A_0 \subset A_1 \Rightarrow f(A_0) \subset f(A_1)$.
      \item[f)] $f(A_0 \cup A_1) = f(A_0) \cup f(A_1)$.
      \item[g)] $f(A_0 \cap A_1) \subset f(A_0) \cap f(A_1)$; show that equality holds if $f$ is injective.
      \item[h)] $f(A_0 - A_1) \supset f(A_0) - f(A_1)$; show that equality holds if $f$ is injective.
    \end{enumerate}
  \end{exercise}
  \begin{solution}
     (a)-(d) proved in \ref{preserve_preimage}. (e)-(h) proved in \ref{preserve_image}.  
  \end{solution}

  \begin{exercise}[Munkres 2.3]
    Show that (b), (c), (f), and (g) of Exercise 2 hold for arbitrary unions and intersections.
  \end{exercise}
  \begin{solution}
    
  \end{solution}

  \begin{exercise}[Munkres 2.4]
    Let $f: A \to B$ and $g: B \to C$.
    \begin{enumerate}
      \item If $C_0 \subset C$, show that $(g \circ f)^{-1}(C_0) = f^{-1}(g^{-1}(C_0))$.
      \item If $f$ and $g$ are injective, show that $g \circ f$ is injective.
      \item If $g \circ f$ is injective, what can you say about injectivity of $f$ and $g$?
      \item If $f$ and $g$ are surjective, show that $g \circ f$ is surjective.
      \item If $g \circ f$ is surjective, what can you say about surjectivity of $f$ and $g$?
      \item Summarize your answers to (b)--(e) in the form of a theorem.
    \end{enumerate}
  \end{exercise}
  \begin{solution}
    
  \end{solution}

  \begin{exercise}[Munkres 2.5]
    In general, let us denote the identity function for a set $C$ by $i_C$. That is, define $i_C: C \to C$ to be the function given by the rule $i_C(x) = x$ for all $x \in C$. Given $f: A \to B$, we say that a function $g: B \to A$ is a left inverse for $f$ if $g \circ f = i_A$; and we say that $h: B \to A$ is a right inverse for $f$ if $f \circ h = i_B$.
    \begin{enumerate}
      \item Show that if $f$ has a left inverse, $f$ is injective; and if $f$ has a right inverse, $f$ is surjective.
      \item Give an example of a function that has a left inverse but no right inverse.
      \item Give an example of a function that has a right inverse but no left inverse.
      \item Can a function have more than one left inverse? More than one right inverse?
      \item Show that if $f$ has both a left inverse $g$ and a right inverse $h$, then $f$ is bijective and $g = h = f^{-1}$.
    \end{enumerate}
  \end{exercise}
  \begin{solution}
    
  \end{solution}

  \begin{exercise}[Munkres 2.6]
    Let $f: \mathbb{R} \to \mathbb{R}$ be the function $f(x) = x^3 - x$. By restricting the domain and range of $f$ appropriately, obtain from $f$ a bijective function $g$. Draw the graphs of $g$ and $g^{-1}$. (There are several possible choices for $g$.)
  \end{exercise}
  \begin{solution}
    
  \end{solution}

  \begin{exercise}[Munkres 3.1]
    Define two points $(x_0, y_0)$ and $(x_1, y_1)$ of the plane to be equivalent if $y_0 - x_0^2 = y_1 - x_1^2$. Check that this is an equivalence relation and describe the equivalence classes.
  \end{exercise}
  \begin{solution}
    
  \end{solution}

  \begin{exercise}[Munkres 3.2]
    Let $C$ be a relation on a set $A$. If $A_0 \subset A$, define the restriction of $C$ to $A_0$ to be the relation $C \cap (A_0 \times A_0)$. Show that the restriction of an equivalence relation is an equivalence relation.
  \end{exercise}
  \begin{solution}
    
  \end{solution}

  \begin{exercise}[Munkres 3.3]
    Here is a "proof" that every relation $C$ that is both symmetric and transitive is also reflexive: "Since $C$ is symmetric, $aCb$ implies $bCa$. Since $C$ is transitive, $aCb$ and $bCa$ together imply $aCa$, as desired." Find the flaw in this argument.
  \end{exercise}
  \begin{solution}
    
  \end{solution}

  \begin{exercise}[Munkres 3.4]
    Let $f: A \to B$ be a surjective function. Let us define a relation on $A$ by setting $a_0 \sim a_1$ if $f(a_0) = f(a_1)$.
    \begin{enumerate}
      \item Show that this is an equivalence relation.
      \item Let $A^*$ be the set of equivalence classes. Show there is a bijective correspondence of $A^*$ with $B$.
    \end{enumerate}
  \end{exercise}
  \begin{solution}
    
  \end{solution}

  \begin{exercise}[Munkres 3.5]
    Let $S$ and $S'$ be the following subsets of the plane:

    $S = \{(x, y) \mid y = x + 1 \text{ and } 0 < x < 2\}$,

    $S' = \{(x, y) \mid y - x \text{ is an integer}\}$.
    \begin{enumerate}
      \item Show that $S'$ is an equivalence relation on the real line and $S' \supset S$. Describe the equivalence classes of $S'$.
      \item Show that given any collection of equivalence relations on a set $A$, their intersection is an equivalence relation on $A$.
      \item Describe the equivalence relation $T$ on the real line that is the intersection of all equivalence relations on the real line that contain $S$. Describe the equivalence classes of $T$.
    \end{enumerate}
  \end{exercise}
  \begin{solution}
    
  \end{solution}

  \begin{exercise}[Munkres 3.6]
    Define a relation on the plane by setting $(x_0, y_0) < (x_1, y_1)$ if either $y_0 - x_0^2 < y_1 - x_1^2$, or $y_0 - x_0^2 = y_1 - x_1^2$ and $x_0 < x_1$. Show that this is an order relation on the plane, and describe it geometrically.
  \end{exercise}
  \begin{solution}
    
  \end{solution}

  \begin{exercise}[Munkres 3.7]
    Show that the restriction of an order relation is an order relation.
  \end{exercise}
  \begin{solution}
    
  \end{solution}

  \begin{exercise}[Munkres 3.8]
    Check that the relation defined in Example 7 is an order relation.
  \end{exercise}
  \begin{solution}
    
  \end{solution}

  \begin{exercise}[Munkres 3.9]
    Check that the dictionary order is an order relation.
  \end{exercise}
  \begin{solution}
    
  \end{solution}

  \begin{exercise}[Munkres 3.10]
    \begin{enumerate}
      \item Show that the map $f: (-1, 1) \to \mathbb{R}$ of Example 9 is order-preserving.
      \item Show that the equation $g(y) = 2y/[1 + (1 + 4y^2)^{1/2}]$ defines a function $g: \mathbb{R} \to (-1, 1)$ that is both a left and a right inverse of $f$.
    \end{enumerate}
  \end{exercise}
  \begin{solution}
    
  \end{solution}

  \begin{exercise}[Munkres 3.11]
    Show that an element in an ordered set has at most one immediate successor and at most one immediate predecessor. Show that a subset of an ordered set has at most one smallest element and at most one largest element.
  \end{exercise}
  \begin{solution}
    
  \end{solution}

  \begin{exercise}[Munkres 3.12]
    Let $\mathbb{Z}_+$ denote the set of positive integers. Consider the following order relations on $\mathbb{Z}_+ \times \mathbb{Z}_+$:
    \begin{enumerate}
      \item The dictionary order.
      \item $(x_0, y_0) < (x_1, y_1)$ if either $x_0 - y_0 < x_1 - y_1$, or $x_0 - y_0 = x_1 - y_1$ and $y_0 < y_1$.
      \item $(x_0, y_0) < (x_1, y_1)$ if either $x_0 + y_0 < x_1 + y_1$, or $x_0 + y_0 = x_1 + y_1$ and $y_0 < y_1$.
    \end{enumerate}
    In these order relations, which elements have immediate predecessors? Does the set have a smallest element? Show that all three order types are different.
  \end{exercise}
  \begin{solution}
    
  \end{solution}

  \begin{exercise}[Munkres 3.13]
    Prove the following theorem. If an ordered set $A$ has the least upper bound property, then it has the greatest lower bound property.
  \end{exercise}
  \begin{solution}
    
  \end{solution}

  \begin{exercise}[Munkres 3.14]
    If $C$ is a relation on a set $A$, define a new relation $D$ on $A$ by letting $(b, a) \in D$ if $(a, b) \in C$.
    \begin{enumerate}
      \item Show that $C$ is symmetric if and only if $C = D$.
      \item Show that if $C$ is an order relation, $D$ is also an order relation.
      \item Prove the converse of the theorem in Exercise 13.
    \end{enumerate}
  \end{exercise}
  \begin{solution}
    
  \end{solution}

  \begin{exercise}[Munkres 3.15]
    Assume that the real line has the least upper bound property.
    \begin{enumerate}
      \item Show that the sets

      $[0, 1] = \{x \mid 0 \leq x \leq 1\}$,

      $[0, 1) = \{x \mid 0 \leq x < 1\}$

      have the least upper bound property.
      \item Does $[0, 1] \times [0, 1]$ in the dictionary order have the least upper bound property? What about $[0, 1) \times [0, 1]$?
    \end{enumerate}
  \end{exercise}
  \begin{solution}
    
  \end{solution}
  
  \begin{exercise}[Munkres Topology 5.5] 
    Which of the following subset of $\mathbb{R}^\omega$ can be expressed as the Cartesian product of subsets of $\mathbb{R}$?\footnote{Note that the existence of these sets depend on the axiom of choice.}
  \end{exercise}
  \begin{solution}
    Listed. We will denote the sets in question as $A$. 
    \begin{enumerate}
      \item We claim that 
        \begin{equation}
          A = \mathbb{Z} \times \mathbb{Z} \times \ldots
        \end{equation}
      \item Let us denote $\mathbb{R}_{\geq i}$ be the set of reals greater than or equal to $i$. This is clearly a subset of $\mathbb{R}$. Then 
        \begin{equation}
          A = \prod_{i=1}^\infty \mathbb{R}_{\geq i}
        \end{equation}
      \item We claim 
        \begin{equation}
          A = \bigg( \prod_{i=1}^{100} \mathbb{R} \bigg) \times \bigg( \prod_{j=1}^\infty \mathbb{Z} \bigg)
        \end{equation}
      \item This is not possible
    \end{enumerate}
  \end{solution}

  \begin{exercise}[Math 531 Spring 2025, PS1.1]
    Find a formula for the sum of the first $n$ odd numbers and prove that it is correct.
  \end{exercise}
  \begin{solution}
    I claim that $f(n) = n^2$. I prove using induction. For $n = 1$, $f(1) = n^2 = 1^2 = 1$. Now assume $f$ holds for some $k \in \mathbb{N}$. Then, the $k$th off number is $2k-1$. Therefore 
    \begin{equation}
      f(k+1) = f(k) + (2k + 1) = k^2 + 2k + 1 = (k + 1)^2
    \end{equation}
    and the formula holds for $k=1$. By the principle of induction, $f(n) = n^2$ is true for all $n \in \mathbb{N}$. 
  \end{solution}

  \begin{exercise}[Shifrin Abstract Algebra 1.1.4.C]
    We check for $n = 1$ denoting our formula as $f$. Indeed, we have 
    \begin{equation}
      f(1) = \frac{1 \cdot 2 \cdot 3}{6} = 1 = 1^2
    \end{equation} 
    For the induction step, assume that $f(k)$ is true for some $k \in \mathbb{N}$. Then, 
    \begin{align}
      f(k+1) & = f(k) + (k+1)^2 \\
             & = \frac{k (k + 1) (2k + 1)}{6} + \frac{6 (k+1)^2}{6} \\
             & = \frac{(k+1) \{ k (2k+1) + 6(k+1)\}}{6} \\
             & = \frac{(k+1) (2k^2 + 7k + 6)}{6} \\
             & = \frac{(k+1)(k+2)(2(k+1) + 1)}{6} \\
             & = f(k+1)
    \end{align}
    Therefore $f$ holds for all $n \in \mathbb{N}$. 
  \end{exercise} 

  \begin{exercise}[Shifrin Abstract Algebra 1.1.4.G]
    We prove the base cases for $n = 1, 2, 3$. 
    \begin{enumerate}
      \item $n = 1$. $n+2 = 3$ is divisible by $3$. 
      \item $n = 2$. $n+4 = 6$ is divisible by $3$. 
      \item $n = 3$. $n+2 = 3$ is divisible by $3$. 
    \end{enumerate} 
    For our inductive step, assume that for some $n = k \in \mathbb{N}$, one of the elements in $S_k = \{k, k+2, k+4\}$ is divisible by $3$. Let us denote this element $a$. We wish to show that this claim is true for $n = k+3$ on the set $S_{k+3} = \{k+3, k+5, k+7\}$. Since $a \in S_k$, this means that $a+3 \in S_{k+3}$, and $3 | a \implies 3 | (a+3)$. So we can always identify the element $a+3$. Since we proved the base cases for $n=1, 2, 3$, and proved the recursive step, we have essentially proved the claim for all naturals of the form $3k+1, 3k+3, 3k+3$ ($k \in \mathbb{N}_0$), which is precisely the natural numbers. 
  \end{exercise}

  \begin{exercise}[Shifrin Abstract Algebra 1.1.4.J]
    Let $n = 1$. Then $1 + x \geq 1 + x$ trivially. For the induction step, assume that this inequality holds for some $n \in \mathbb{N}$. Then, we have 
    \begin{align}
      1 + (n+1) x & = 1 + nx + x \\
                  & \leq (1+x)^n + x \\ 
                  & \leq (1+x)^n + x(1+x)^n \\
                  & = (1+x)^{n+1}
    \end{align} 
    where the prove the penultimate step by applying the ordered field axioms to the 2 cases: 
    \begin{enumerate}
      \item If $x \geq 0$, then addition preserves order so $1 + x \geq 0 + 1 = 1$. Since $1 + x, 1 > 0$, order is preserved under multiplication by a positive element, so $(1+x)^2 \geq 1+x \geq 1$. Using induction, we can show that for all $n \in \mathbb{N}$, $(1+x)^n \geq 1$, and again by preservation of order under multiplication by a positive element, this implies $x (1 + x)^n \geq x$ for all $n \in \mathbb{N}$. 
      \item If $0 > x > -1$, we have $0 < 1 + x < 1$ and by the same induction proof, we can bound $0 < (1+x)^n < 1$ for all $n$. Finally by reversal of order under multiplication by a negative element, we have $x (1+x)^n > x$. 
    \end{enumerate}
    Therefore, we take the less restrictive of the 2 bounds: $x(1+x)^n \geq x$. 
  \end{exercise}

  \begin{exercise}[Shifrin Abstract Algebra 1.1.7]
    Let us denote 
    \begin{equation}
      x = \frac{1 + \sqrt{5}}{2} , \;\;\; y = \frac{1 - \sqrt{5}}{2}
    \end{equation} 
    Note the identities 
    \begin{align}
      x^2 & = \bigg( \frac{1 + \sqrt{5}}{2} \bigg)^2 = \frac{3 + 2 \sqrt{5}}{2} = 1 + x \\
      y^2 & = \bigg( \frac{1 - \sqrt{5}}{2} \bigg)^2 = \frac{3 - 2 \sqrt{5}}{2} = 1 + y
    \end{align}
    We check the base case for $n = 1$
    \begin{equation}
      a_1 = \frac{1}{\sqrt{5}} (x + y) = \frac{1}{\sqrt{5}} \frac{2 \sqrt{5}}{2} = 1
    \end{equation} 
    and for $n = 2$ 
    \begin{equation}
      a_2 = \frac{1}{\sqrt{5}} (x^2 - y^2) = \frac{1}{\sqrt{5}} ((1+x) - (1+y)) = \frac{1}{\sqrt{5}} (x + y) = a_1 = 1
    \end{equation}
    For the inductive step, assume that this formula holds for some $k-1, k \in \mathbb{N}$. Then, we have 
    \begin{align}
      a_{k+1} & = a_{k} + a_{k-1} \\
              & = \frac{1}{\sqrt{5}} (x^{k-1} - y^{k-1}) + \frac{1}{\sqrt{5}} (x^k - y^k) \\
              & = \frac{1}{\sqrt{5}} \big\{ x^{k-1} (1 + x) - y^{k-1} (1 + y)\big\} \\
              & = \frac{1}{\sqrt{5}} (x^{k+1} - y^{k+1})
    \end{align}
    and we are done. 
  \end{exercise}



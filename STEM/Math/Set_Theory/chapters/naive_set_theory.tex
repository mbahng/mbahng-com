\section{Naive Set Theory}

Unlike axiomatic set theories, which are defined using formal logic, naive set theory was defined informally at the end of the 19th century by Cantor, in natural language (like English). It describes the aspects of mathematical sets using words (e.g. \textit{satisfying, such as, ...}) and suffices for the everyday use of set theory in modern mathematics. However, as we will see, this leads to paradoxes. 

\begin{definition}[Set]
  A \textbf{set} is a well-defined collection of distinct objects, called \textbf{elements}. 
\end{definition}

This definition tells us \textit{what} a set is, but does not define \textit{how} sets can be formed, and what operations on sets will again produce a set. The term \textit{well-defined} cannot by itself guarantee the consistency and unambiguity of what exactly constitutes and what does not constitute a set, and therefore this is not a formal definition. Attempting to achieve this will be done in axiomatic set theory, like ZFC. 

\begin{definition}[Membership]
  If $x$ is a member of $A$, we write $x \in A$. For any $x$, it must be the case that either $x \in A$ (exclusive or) $x \not\in A$. 
\end{definition}

\begin{definition}[Equality]
  Two sets $A$ and $B$ are defined to be equal, denoted as $A = B$, when they have precisely the same elements. That is, if $x \in A \iff x \in B$. This means that a set is completely determined by its elements, and the description is immaterial. 
\end{definition}

\begin{definition}[Empty Set]
  There exists an empty set, denoted $\emptyset$ or $\{\}$, which is a set with no members at all. Because a set is described by its elements, there can only be one empty set. 
\end{definition}

Now we show how to construct sets. 

\begin{definition}[Set-Builder Notation]
  We can construct a set in two ways. 
  \begin{enumerate}
    \item We list its elements between curly braces. 
    \begin{enumerate}
      \item The set $\{1, 2\}$ denotes the set containing $1$ and $2$. By equality $\{1, 2\} = \{2, 1\}$. 
      \item Repetition/multiplicity is irrelevant, and so $\{1, 2, 2\} = \{1, 1, 1, 2\} = \{1, 2\}$ 
    \end{enumerate} 

    \item We denote 
    \begin{equation}
      S = \{ x | P(x) \}
    \end{equation} 
    where $P$ is a property. If $x$ satisfies this property, then $x \in S$. 
  \end{enumerate}
  Naive set theory claims that this construction \textit{always} produces a set. Therefore, a well-defined property is enough to always produce a set of elements satisfying $P$. 
\end{definition} 

\begin{example}[Empty Set]
  Let $S = \{x \mid x \neq x \}$. For any $x$, $P(x)$ is false and so $S$ contains no elements. Therefore $S = \emptyset$. 
\end{example}

\begin{example}[Singleton Set]
  The set $\{x \mid x = a \} = \{a\}$. 
\end{example}

\begin{example}[Russell Set]
  Let $R = \{x \mid x \not\in x\}$, i.e. the set of all sets that do not contain themselves as elements. 
\end{example}

\begin{theorem}[Russell's Paradox] 
  The Russell set exists and does not exist. 
\end{theorem}
\begin{proof}
  We will determine if $R$ is an element of itself. 
  \begin{enumerate}
    \item If $R \in R$, then by it does contain itself, so it does not satisfy the property and $R \not\in R$. 
    \item If $R \not\in R$, then it satisfies the property, so $R \in R$. 
  \end{enumerate}
  Therefore, it is both the case that $x \in R$ and $x \not\in R$, which contradicts the membership definition. Therefore, $R$ is both a set from set-builder construction and not a set due to the membership definition. 
\end{proof}

\begin{theorem}[Existence of Universe]
  Let $U$ be the set of everything, known as the \textbf{universal set}. The universal set does exist and does not exist. 
\end{theorem}
\begin{proof}
  We can define $U^\prime = \{x \mid \{\} = \{\} \}$, which defines a set. Then the property $P$ that $\{\} = \{\}$ is always true, and $U^\prime$ would contain everything, and by the definition of equality $U = U^\prime$. Now since the Russell set $R$ is both a set and not a set from Russell's paradox, we have $R \in U$ and $R \not\in U$, which means that $U$ cannot exist. Therefore $U$ does not exist. 
\end{proof}

So the sufficiency a well-defined property to be able to construct a set is \textit{too powerful} in that we can construct \textit{any} set we want. This leads us to construct the Russell set, which opens up a lot of paradoxes. Therefore, we would like to restrict the notion of well-defined in a way, which leads to axiomatic set theories. 

\begin{definition}[Subsets]
  Given two sets $A$ and $B$, $A$ is a \textbf{subset} of $B$ if every element of $A$ is also an element of $B$. A subset of $B$ that is not equal to $B$ is called a \textbf{proper subset}. 
\end{definition}

\begin{theorem}[Equality]
  It follows from the definition of equality that 
  \begin{equation}
    A \subset B \text{ and } B \subset A \iff A = B
  \end{equation}
\end{theorem}

\begin{definition}[Power Set]
  The set of all subsets of a set $A$ is called the \textbf{power set} of $A$, denoted by $2^A$. 
\end{definition}

We could define other things like the union, etc., but I won't bother with it when I will define them for ZFC later.  


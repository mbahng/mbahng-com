\section{Ordinal Numbers} 

  We have introduced the natural numbers as an inductive set that contains the empty set and recursively adding each element through the sucessor function $S(x) = x \cup \{x\}$. This gives us the natural numbers $\mathbb{N}$, but there is no barrier to stopping, and so we can count beyond the natural numbers by imagining some infinite number $\omega$ and continuing the counting process into the transfinite. 
  \begin{align}
    \omega & = \mathbb{N} = \{0, 1, 2, \ldots \} \\ 
    S(\omega) & = \omega \cup \{\omega\} = \{0, 1, 2 \ldots, \omega\} \\
    S(S(\omega)) & = S(\omega) \cup \{S(\omega)\} = \{0, 1, 2, \ldots, \omega, S(\omega)\}
    \ldots & = \ldots 
  \end{align}

  This is the motivation behind \textit{ordinal numbers}, which are numbers that describe the order of some element in a set, analogous to how cardinal numbers were designed to describe the size of sets. 

\subsection{Axiom of Replacement}

  \begin{axiom}[Axiom Schema of Replacement]
    This axiom asserts that the image of a set under any definable function will fall inside a set. 
  \end{axiom}

\subsection{Transfinite Induction and Recursion} 

  The induction principle and recursion theorem are the main tools for proving theorems about the natural numbers and constructing functions with domain $\mathbb{N}$, respectively. We can generalize them to ordinal numbers. 

\subsection{Ordinal Arithmetic} 





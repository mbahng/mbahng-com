\section{Rules of Set Theory} 

\subsection{Set Operations}

  Let's first talk about rules following the union, intersection, and set minus operators. 

  \begin{theorem}[deMorgan's Laws]
    If $X$ is a set and $A,B,C \subset X$, then 
    \begin{align}
      A \cap (B \cup C) & = (A \cap B) \cup (A \cap C) \\
      A \cup (B \cap C) & = (A \cup B) \cap (A \cup C) \\ 
      X \setminus (A \cup B) & = (X \setminus A) \cap (X \setminus B) \\
      X \setminus (A \cap B) & = (X \setminus A) \cup (X \setminus B)
    \end{align}
  \end{theorem}
  \begin{proof}
    We prove them 
    \begin{enumerate}
      \item $A \cap (B \cup C) = (A \cap B) \cup (A \cap C)$. 
        \begin{enumerate}
          \item $A \cap (B \cup C) \subset (A \cap B) \cup (A \cap C)$. Assume $x \in A \cap (B \cup C)$. Then $x \in A$ and $x \in B \cup C$. If $x \in B$, then $x \in A \cap B$. If $x \in C$, then $x \in A \cap C$. Therefore, since $x \in B \cup C$, it must be the case that $x \in A \cap B$ or $x \in A \cap C$, which by definition implies $x \in (A \cap B) \cup (A \cap C)$. 

          \item $A \cap (B \cup C) \supset (A \cap B) \cup (A \cap C)$. Assume that $x \in (A \cap B) \cup (A \cap C)$. Then WLOG let $x \in A \cap B$. Then $x \in A$ and $x \in B \subset (B \cup C)$, so by definition $x \in A \cap (B \cup C)$. 
        \end{enumerate}

      \item $A \cup (B \cap C) = (A \cup B) \cap (A \cup C)$.
        \begin{enumerate}
          \item $A \cup (B \cap C) \subset (A \cup B) \cap (A \cup C)$. Assume $x \in A \cup(B \cap C)$. Then $x \in A$ or $x \in B \cap C$. If $x \in A$, then since $A \subset (A \cup B)$ and $A \subset (A \cup C)$, we have $x \in (A \cup B)$ and $x \in (A \cup C)$, which by definition means $x \in (A \cup B) \cap (A \cup C)$. If $x \not\in A$, then $x \in B \cap C \implies x \in B \subset (A \cup B)$ and $x \in C \subset (A \cup C)$, and so $x \in (A \cup B) \cap (A \cup C)$. 

          \item $A \cup (B \cap C) \supset (A \cup B) \cap (A \cup C)$. Assume $x \in (A \cup B) \cap (A \cup C)$. Then $x \in A \cup B$. If $x \in A$, then since $A \subset A \cup (B \cap C)$, $x \in A \cup (B \cap C)$. If $x \not\in A$, then $x \in B$. Since $x \in A \cup C$, $x \in C$ also. Therefore by definition $x \in (B \cap C) \subset A \cup (B \cap C) \implies x \in A \cup (B \cap C)$. 
        \end{enumerate}

      \item $X \setminus (A \cup B) = (A \setminus A) \cap (X \setminus B)$.
        \begin{enumerate}
          \item $X \setminus (A \cup B) \subset (A \setminus A) \cap (X \setminus B)$. Assume $x \in X \setminus (A \cup B) \iff x \in X$ and $x \not\in (A \cup B)$. Since $x \not\in (A \cup B$, $x \not\in A$ and $x \not\in B$. However, $x \in X$, so $x \not\in A \implies x \in X \setminus A$. Same goes for $B$, and so $x \in (X \setminus A) \cap (X \setminus B)$. 

          \item $X \setminus (A \cup B) \supset (A \setminus A) \cap (X \setminus B)$. Assume $x \in (X \setminus A) \cap (X \setminus B)$. Then $x \in X \setminus A \iff X \in X$ and $x \not\in A$, and $x \in X \setminus B \iff x \in X$ and $x \not\in B$. Since $x \not\in A$ and $x \not\in B$, $x \not\in A \cup B$. Combined with the fact that $x \in X$, we have $x \in X \setminus (A \cup B)$. 
        \end{enumerate}

      \item $X \setminus (A \cap B) = (A \setminus A) \cup (X \setminus B)$.
        \begin{enumerate}
          \item $X \setminus (A \cap B) \subset (A \setminus A) \cup (X \setminus B)$. Let $x \in X \setminus (A \cap B)$. Then $x \in X$ and $x \not\in A \cap B$. Since $x \not\in A \cap B$, it must be the case that at least $x \not\in A$ or $x \not\in B$. WLOG let $x \not\in A$. Then $x \in X$ and $x \not\in A \implies x \in (X \setminus A) \subset (X \setminus) \cup (X \setminus B) \implies x \in (X \setminus A) \cup (X \setminus B)$. 

          \item $X \setminus (A \cap B) \supset (A \setminus A) \cup (X \setminus B)$. WLOG let $x \in (X \setminus A)$. Then $x \in X$ and $x \not\in A$, and $x \not\in A \implies x \not\in (A \cap B) \subset A$ (contrapositive is trivial). Therefore, $x \in X$ and $x \not\in (A \cap B) \iff  x \in X \setminus (A \cap B)$. 
        \end{enumerate}
    \end{enumerate}
  \end{proof} 

  \begin{corollary}[Symmetric Difference]
    Given sets $X, Y$, 
    \begin{equation}
      (X \setminus Y) \cap (Y \setminus X) = (X \cup Y) - (X \cap Y)
    \end{equation}
    This is called the \textbf{symmetric difference} between two sets. 
  \end{corollary}
  \begin{proof}
    
  \end{proof} 

\subsection{Functions}

  Now let's see how these operations behavior under functions. 

  \begin{theorem}[Preservation Under Mapping Back and Forth]
    Given $f: A \rightarrow B$, with $A_0, A_1 \subset A$ and $B_0, B_1 \subset B$, the following hold 
    \begin{enumerate}
      \item $A_0 \subset f^{-1} (f(A_0))$, with equality holding if $f$ is injective. 
      \item $f(f^{-1}(B_0)) \subset B_0$, with equality holding if $f$ is surjective. 
    \end{enumerate}
  \end{theorem} 
  \begin{proof} 
    Listed. 
    \begin{enumerate}
      \item Assume that $x \in A_0$. Then $f(x) \in f(A_0)$. The preimage is 
      \begin{equation}
        f^{-1} (f(A_0)) \coloneqq \{ y \in A \mid f(y) \in f(A_0) \}
      \end{equation}
      and $x$ certainly satisfies the condition that $f(x) \in f(A_0)$. Therefore $x \in f^{-1} (f(A_0))$ and so $A_0 \subset f^{-1} (f(A_0))$. 

      Now assume that $f$ is injective. It suffices to prove that $f^{-1} (f(A_0)) \subset A_0$ since the other direction is proven for all functions. We prove this by proving the contrapositive, i.e. $x \not\in A_0 \implies x \not\in f^{-1} (f(A_0))$. Suppose $x \not\in A_0 \implies f(x) \not\in f(A_0) \implies f^{-1} (f(x)) \not\subset f^{-1} (f(A_0))$ by definition of the image and preimage. But note that since $f$ is injective, $f^{-1} (f(x)) = x$.\footnote{More specifically, if we treat $x$ as the singleton set, $f(x)$ is also a singleton set by definition of a function. Since $f$ is injective, the preimage of a singleton set must be a singleton set. If it were not, then there exists $x, y$ with $x \neq y$ that maps to the same $z$, which contradicts the definition of injectivity.} and thus $x \not\in f^{-1} (f(A_0))$. 

      \item We prove this using the contrapositive. Assume that $x \not\in B_0$. Then, with abuse of notation, we have by definition of the preimage and the image $f^{-1} (x) \not\subset f^{-1} (B_0) \implies f(f^{-1} (x)) \not\subset f(f^{-1}(B_0))$. But $f (f^{-1} (x)) = \{x\}$, since we are just mapping the preimage of $x$ back across to $f$. Therefore, $x \notin f( f^{-1} (B_0))$. 

      Now assume that $f$ is surjective. It suffices to prove that $B_0 \subset f (f^{-1}(B_0))$. Assume $y \in B_0$. Since $f$ is surjective, we know that $f^{-1} (y)$ is nonempty in $A$. We can state $f^{-1}(y) \subset f^{-1} (B_0)$\footnote{The formal proof of this is given in Munkres 1.2.2.a.} which then implies $f(f^{-1} (y)) \subset f (f^{-1} (B_0))$.\footnote{Again formal proof of this given in Munkres 1.2.2.e.} But $f (f^{-1} (y)) = y$ as mentioned previously, and so $y \in f(f^{-1} (B_0))$. 
    \end{enumerate}
  \end{proof}

  \begin{example}
    To see why equality does not hold in general for the two cases, look at the counterexamples below. 
    \begin{enumerate}
      \item $A_0 \not\supset f^{-1} (f(A_0))$. 
      \item $f(f^{-1}(B_0)) \not\supset B_0$. Consider $X = Y = \{0, 1\}$ and $f: X \rightarrow Y$ defined $f(0) = f(1) = 0$. Consider $C = Y$. We have $f^{-1} (C) = f^{-1} (0) \cup f^{-1} (1) = X \cup \emptyset = X$. Then $f(f^{-1} (C)) = f(X) = \{0\} \neq C$. 
    \end{enumerate}
  \end{example}

  \begin{theorem}[Preservation Under Preimages]
    Given $f: A \rightarrow B$, with $A_\alpha \subset A$ and $B_\alpha \subset B$, $f$ preserves the inclusion, union, intersection, and set difference under the preimage. 
    \begin{enumerate}
      \item \textit{Inclusion}. $B_0 \subset B_1 \implies f^{-1} (B_0) \subset f^{-1} (B_1)$. 
      \item \textit{Union}. $f^{-1} (\cup B_\alpha) = \cup_\alpha f^{-1} (B_\alpha)$. 
      \item \textit{Intersection}. $f^{-1} (\cap B_\alpha) = \cap_\alpha f^{-1} (B_\alpha)$.
      \item \textit{Set Difference}. $f^{-1}(B_0 \setminus B_1) = f^{-1} (B_0) \setminus f^{-1} (B_1)$. 
    \end{enumerate}
  \end{theorem} 
  \begin{proof}
    Listed. 
    \begin{enumerate}
      \item \textit{Inclusion}. If $x \in B_0$, then $f^{-1} (x) \subset A$ maps to $x$ by definition. But since $x \in B_0$, $f^{-1} (x)$ maps to a point in $B_0$, and so $f^{-1} (x) \subset f^{-1} (B_0)$. Since $B_0 \subset B_1$ by assumption, $x \in B_1$, and by the previous logic but with $B_0$ replaced by $B_1$ we have $f^{-1}(x) \subset f^{-1} (B_1)$. We have just proved that $f^{-1} (x) \in f^{-1} (B_0)  \implies f^{-1} (x) \in f^{-1} (B_1)$, and so $f^{-1} (B_0) \subset f^{-1} (B_1)$. 

      \item \textit{Union}. We prove bidirectionally. 
      \begin{enumerate}
        \item $f^{-1} (B_0 \cup B_1) \subset f^{-1} (B_0) \cup f^{-1} (B_1)$. Let $x \in f^{-1} (B_0 \cup B_1)$ which by definition of the preimage means $f(x) \in B_0 \cup B_1$. Therefore $f(x) \in B_0$ or $B_1$. Without loss of generality, let $f(x) \in B_0$. Then we have 
          \begin{equation}
            x \in f^{-1} (f(x)) \subset f^{-1} (B_0)
          \end{equation} 
          where the first inclusion comes from [Munkres 1.2.1.a] when treating $A_0 = \{x\}$, and the second subset comes from [Munkres 1.2.2.a] when treating $B_0 = \{f(x)\}, B_1 = B_1$. Therefore $x \in f^{-1} (B_0) \subset f^{-1} (B_0) \cup f^{-1} (B_1)$. 
        \item $f^{-1} (B_0) \cup f^{-1} (B_1) \subset f^{-1} (B_0 \cup B_1)$. Let $x \in f^{-1}(B_0) \cup f^{-1} (B_1)$. Without loss of generality, let $x \in f^{-1}(B_0)$ which by definition of the preimage implies $f(x) \in B_0 \subset (B_0 \cup B_1) \implies f(x) \in (B_0 \cup B_1)$. Therefore, we have 
          \begin{equation}
            x \in f^{-1} (f(x)) \subset f^{-1} (B_0 \cup B_1)
          \end{equation} 
          where the inclusion claim comes from [Munkres 1.2.1.a] when treating $A_0 = \{x\}$, and the subset claim comes from [Munkres 1.2.2.a] when treating $B_0 = \{f(x)\}, B_1 = B_0 \cup B_1$. Therefore $x \in f^{-1} (B_0 \cup B_1)$. 
      \end{enumerate}
      Therefore, $f^{-1} (B_0) \cup f^{-1} (B_1) = f^{-1} (B_0 \cup B_1)$. 

      \item \textit{Intersection}. We prove bidirectionally. 
      \begin{enumerate}
        \item $f^{-1} (B_0 \cap B_1) \subset f^{-1} (B_0) \cap f^{-1} (B_1)$. Assume $x \in f^{-1} (B_0 \cap B_1)$, which by definition of the preimage means $f(x) \in B_0 \cap B_1$. So 
          \begin{align}
            f(x) \in B_0 & \implies x \in f^{-1} (f(x)) \subset f^{-1} (B_0) \\
            f(x) \in B_1 & \implies x \in f^{-1} (f(x)) \subset f^{-1} (B_1)
          \end{align}
          where the inclusion claim comes from [Munkres 1.2.1.a] when treating $A_0 = \{x\}$, and the subset claim comes from [Munkres 1.2.2.a] when treating $f(x)$ as a singleton set. Therefore $x$ is in both of the preimages and so $x \in f^{-1} (B_0) \cap f^{-1} (B_1)$. 
        \item $f^{-1} (B_0) \cap f^{-1} (B_1) \subset f^{-1} (B_0 \cap B_1)$. Let $x \in f^{-1} (B_0) \cap f^{-1} (B_1)$. Then by definition of intersection and preimage, 
          \begin{align}
            x \in f^{-1} (B_0) & \implies f(x) \in B_0 \\
            x \in f^{-1} (B_1) & \implies f(x) \in B_1 
          \end{align} 
          and so $f(x) \in B_0 \cap B_1$ by definition of intersection. This means by definition of the preimage that $x \in f^{-1}(B_0 \cap B_1)$. 
      \end{enumerate}

      \item \textit{Set Difference}. We prove bidirectionally. 
      \begin{enumerate}
        \item $f^{-1} (B_0 \setminus B_1) \subset f^{-1} (B_0) \setminus f^{-1} (B_1)$. Let $x \in f^{-1}(B_0 \setminus B_1)$ which by definition of preimage means $f(x) \in B|0 \setminus B_1$. This implies two things. First, 
          \begin{equation}
            f(x) \in B_0 \implies x \in f^{-1} (f(x)) \subset f^{-1} (B_0)
          \end{equation}
          where the inclusion comes from [Munkres 1.2.1.a] when treating $A_0 = \{x\}$ as the single set, and the subset claim comes from [Munkres 1.2.2.a] stating that inclusions are preserved under the preimage operator. Secondly, we claim that  
          \begin{align}
            f(x) \not\in B_1 & \implies x \not\in f^{-1} (B_1)
          \end{align}
          since if $x \in f^{-1} (B_1)$, then $f(x) \in B_1$ by definition of the preimage. 

        \item $f^{-1} (B_0) \setminus f^{-1} (B_1) \subset f^{-1} (B_0 \setminus B_1)$. Let $x \in f^{-1} (B_0) \setminus f^{-1} (B_1)$. Then the following holds 
          \begin{align}
            x \in f^{-1} (B_0) & \implies f(x) \in B_0 \\
            x \not\in f^{-1} (B_1) & \implies f(x) \not\in B_1
          \end{align} 
          from the definition of the preimage and the contrapositive of its implication. Therefore $f(x) \in B_0 \setminus B_1$ which by definition of the preimage $x \in f^{-1} (B_0 \setminus B_1)$. 
      \end{enumerate}
    \end{enumerate}
  \end{proof}

  \begin{theorem}[Preservation Under Images]
    Given $f: A \rightarrow B$, with $A_\alpha \subset A$ and $B_\alpha \subset B$, $f$ preserves the inclusion and union under the image, but inclusion properties for the intersection and set difference hold. 
    \begin{enumerate}
      \item \textit{Inclusion}. $A_0 \subset A_1 \implies f(A_0) \subset f(A_1)$. 
      \item \textit{Union}. $f(\cup_\alpha A_\alpha) = \cup_\alpha f(A_\alpha)$. 
      \item \textit{Intersection}. $f(A_0 \cap A_1) \subset f(A_0) \cap f (A_1)$, and equality holds if $f$ is injective. 
      \item \textit{Set Difference}. $f(A_0 \setminus A_1) \supset f(A_0) \setminus f(A_1)$, and equality holds if $f$ is injective. 
    \end{enumerate}
  \end{theorem} 
  \begin{proof}
    Listed. 
    \begin{enumerate}
      \item \textit{Inclusion}. Let $x \in A_0$. Then by definition of the image $f(x) \in f(A_0)$. Since $A_0 \subset A_1$, then $x \in A_1$ and it immediately follows that $f(x) \in f(A_1)$. Therefore $f(A_0) \subset f(A_1)$. 

      \item \textit{Union}. We prove bidirectionally. 
      \begin{enumerate}
        \item $f(A_0 \cup A_1) \subset f(A_0) \cup f(A_1)$. Let $y \in f(A_0 \cup A_1)$. Then by definition there exists some $x \in A_0 \cup A_1$ s.t. $f(x) = y$. WLOG let $x \in A_0$. Then by definition $y = f(x) \in f(A_0) \subset f(A_0) \cup f(A_1)$. 

        \item $f(A_0) \cup f(A_1) \subset f(A_0 \cup A_1)$. Let $y \in f(A_0) \cup f(A_1)$. WLOG $y \in f(A_0)$, and there exists some $x \in A_0$ s.t. $f(x) = y$. Since $x \in A_0$, $x \in A_0 \cup A_1$, and by definition $y = f(x) = f(A_0) \cup f(A_1)$. 
      \end{enumerate}

      \item \textit{Intersection}. Assume that $y \in f(A_0 \cap A_1)$. Then by definition there exists some $x \in A_0 \cap A_1$ s.t. $f(x) = y$. So we have 
      \begin{align}
        x \in A_0 & \implies f(x) \in f(A_0) \\
        x \in A_1 & \implies f(x) \in f(A_1)
      \end{align} 
      and therefore $y = f(x) \in f(A_0) \cap f(A_1)$. 

      To prove equality, it suffices to show that $f(A_0) \cap f(A_1) \subset f(A_0 \cap A_1)$ if $f$ is injective. Assume that $y \in f(A_0) \cap f(A_1)$. Then $y \in f(A_0)$, and so there exists an $x \in A_0$ s.t. $y = f(x) \in f(A_0)$. By the same logic there exists an $x^\prime \in A_1$ s.t. $y = f(x^\prime) \in f(A_1)$. But since $f$ is injective, this implies that $x = x^\prime$. So $x \in A_0 \cap A_1$, and so $y = f(x) \in f(A_0 \cap A_1)$. 

      \item \textit{Set Difference}. Assume that $y \in f(A_0) \setminus f(A_1)$. Since $y \in f(A_0)$, there exists some $x \in A_0$ s.t. $y = f(x)$. Since $y \not\in f(A_1)$, there exists no $x^\prime \in A_1$ s.t. $y = f(x^\prime)$.  Therefore, $x \in A_0 \setminus A_1 \implies y = f(x) \in f(A_0 \setminus A_1)$. 

      To prove equality, it suffices to show that $f(A_0 \setminus A_1) \subset f(A_0) \setminus f(A_1)$ if $f$ is injective. Assume that $y \in f(A_0 \setminus A_1)$. Then there exists some $x \in A_0 \setminus A_1$ s.t. $f(x) = y$. We claim that $x$ is unique since if there were two $x, x^\prime$, then $f(x) = f(x^\prime)$ with $x \neq x^\prime$, which means $f$ is not injective. We see that $x \in A_0 \implies y = f(x) \in f(A_0)$, and $x \not\in A_1 \implies y = f(x) \not\in f(A_1)$. Therefore, $x \in f(A_0) \setminus f(A_1)$. 
    \end{enumerate}
  \end{proof} 

  \begin{example}[Intersection Not Necessarily Preserved]
    Note that intersection is not necessarily preserved. To see why, look at the counterexample. Consider $A = \{0, 1\}, B = \{1, 2\}$, and define 
    \begin{equation}
      f(0) = f(2) = 0, f(1) = 1
    \end{equation} 
    Then $f(A) = f(B) = \{0, 1\} \implies f(A) \cap f(B) = \{0, 1\}$. On the other hand, we have $A \cap B = \{1\} \implies f(A \cap B) = \{1\}$. 
  \end{example}

  \begin{theorem}[Composition]
    Let $f: X \rightarrow Y$ and $g: Y \rightarrow Z$. 
    \begin{enumerate}
      \item $f$ injective and $g$ injective $\implies$ $g \circ f$ injective. 
      \item $f$ surjective and $g$ surjective $\implies$ $g \circ f$ surjective. 
      \item $f$ bijective and $g$ bijective $\implies$ $g \circ f$ bijective. 
    \end{enumerate}
  \end{theorem}

  \begin{theorem}[Injectivity/Surjectivity]
    Let $f: X \rightarrow Y$, $g: Y \rightarrow Z$, and $h = g \circ f$. The following hold: 
    \begin{enumerate}
      \item $h$ injective $\implies$ $f$ injective. 
      \item $h$ surjective $\implies$ $g$ surjective. 
      \item $h$ bijective $\implies$ $f$ injective and $g$ bijective. 
    \end{enumerate}
  \end{theorem} 

  \begin{corollary}[Bijection Equals Existence of Inverse]
    $f: X \rightarrow Y$ has a inverse function $f^{-1}: B \rightarrow A$ iff it is bijective. 
  \end{corollary}

  \begin{corollary}[Decomposition]
    Any function $h: X \rightarrow Y$ can be decomposed to the form $h = g \circ f$, where $f$ is injective and $g$ is surjective. 

  \end{corollary}
  \begin{proof}
    Given $X$, let us define an equivalence class where for any $x, y \in X$, $x \sim y$ iff $f(x) = f(y)$. Call this quotient space $X / \sim$. Then we can define the mappings. 
    \begin{enumerate}
      \item $\iota: X \rightarrow X / \sim$ which maps each element to its equivalence class. $\iota(x) = [x]$
      \item $f^\prime: X / \sim \rightarrow Y$ which maps each class to the element of $Y$ that it maps to. $f^\prime ([x]) = f(x)$. 
    \end{enumerate}
    $\iota$ is surjective since for every $[x] \in X / \sim$, there exists at least one element $x \in X$ that maps to it. $f^\prime$ is injective since have squished all the points $x$ that map to the same $y$ into a single class $[x]$. 

    \begin{figure}[H]
      \centering 
      \begin{tikzcd}
        X \arrow[r, "f"] \arrow[d, "\iota"] & Y \\
        X / \sim \arrow[ru, "f^\prime"'] & 
      \end{tikzcd}
      \caption{Decomposition of $f$ into surjective $\iota$ and injective $f^\prime$. } 
      \label{fig:decomposition}
    \end{figure}
  \end{proof}

  \begin{theorem}[Inverse of Inverses]
    If $f$ is bijective, then $f = (f^{-1})^{-1}$. 
  \end{theorem}

  \begin{theorem}[Finite Set Mappings]
    Suppose $X$ and $Y$ are finite sets, each with $n$ elements, and $f: X \rightarrow Y$. If $f$ is injective or bijective, then $f$ is bijective. 
  \end{theorem} 

  \begin{theorem}[Inverse of Compositions]
    If $f, g$ are both bijective, then 
    \begin{equation}
      (f \circ g)^{-1} = g^{-1} \circ f^{-1}
    \end{equation}
  \end{theorem}

\subsection{Cardinality}

  \begin{definition}[Cardinality]
    The \textbf{cardinality}, or the \textbf{cardinal number}, of a set $X$ is the number of elements in $X$. 
  \end{definition} 

  \begin{definition}[Equipotence]
    Two sets $A$ and $B$ are \textbf{equipotent}, written $A \approx B$, if there exists a bijective map $f: A \rightarrow B$. This implies that their cardinalities are the same: $|A| = |B|$. It has the following properties: 
    \begin{enumerate}
      \item Reflexive: $A \approx A$
      \item Symmetric: $A \approx B$ implies $B \approx A$
      \item Transitive: $A \approx B$ and $B \approx C$ implies $A \approx C$
    \end{enumerate}
  \end{definition}

  \begin{definition}[Classes of Cardinal Numbers]
    For any positive integer $n$, let $J_n$ be the set whose elements are the integers $1, 2, \ldots, n$. For any set $A$, we define 
    \begin{enumerate}
      \item $A$ is \textbf{finite} if $A \approx J_n$ for some $n$. The empty set is also considered to be finite. 
      \item $A$ is \textbf{infinite} if it is not finite. 
      \item $A$ is countable if $A \approx \mathbb{N}$. 
      \item $A$ is uncountable if $A$ is neither finite nor countable. 
      \item $A$ is at most countable if $A$ is finite or countable. 
    \end{enumerate}
  \end{definition}

  At this point, we may already be familiar with the fact that $\mathbb{Q}$ is countable and $\mathbb{R}$ is uncountable. Let us formalize the statement that a countable infinity is the smallest type of infinity. We can show this by taking a countable set and showing that every infinite subset must be countable. If it was uncountable, then this would mean that a countable set contains an uncountable set. 

  \begin{theorem}
    \label{countable smallest}
    Every infinite subset of a countable set $A$ is countable. 
  \end{theorem}

  \begin{theorem}
    An at most countable union of countable sets is countable. 
  \end{theorem}

  \begin{theorem}
    A finite Cartesian product of countable sets is countable. 
  \end{theorem}

  Now, how do we prove that a set is uncountable? We can't really use the contrapositive of Theorem $\ref{countable smallest}$, since to prove that an arbitrary set $A$ is uncountable, then we must find an infinite subset that is not countable. But now we must prove that this subset itself is not countable, too! Therefore, we can use this theorem. 

  \begin{theorem}
    Given an arbitrary set $A$, if every countable subset $B$ is a proper subset of $A$, then $A$ is uncountable. 
  \end{theorem}
  \begin{proof}
    Assume that $A$ is countable. Then $A$ itself is a countable subset of $A$, but by the assumption, $A$ should be a proper subset of $A$, which is absurd. Therefore, $A$ is uncountable. 
  \end{proof}

\subsection{Exercises}

  \begin{exercise}[Math 531 Spring 2025 PS1.2]
    Prove that $\mathbb{Q}$ is countable. 
  \end{exercise}
  \begin{solution}
    Proved in theorem above. 
  \end{solution}

  \begin{exercise}[Math 531 Spring 2025, PS2.5]
    Prove that if $X$ is a set and $A,B,C \subset X$, we have that
    \begin{equation}
      A \cap (B \cup C) = (A \cap B) \cup (A \cap C),
    \end{equation}
    \begin{equation}
      A \cup (B \cap C) = (A \cup B) \cap (A \cup C),
    \end{equation}
    \begin{equation}
      X \setminus (A \cup B) = (X \setminus A) \cap (X \setminus B),
    \end{equation}
    \begin{equation}
      X \setminus (A \cap B) = (X \setminus A) \cup (X \setminus B).
    \end{equation}
  \end{exercise}
  \begin{solution}
    Proved in deMorgan's laws. 
  \end{solution}

  \begin{exercise}[Shifrin Abstract Algebra Appendix 2.3]
    Let $f: X \to Y$. Let $A,B \subset X$ and $C,D \subset Y$. Prove or give a counterexample (if possible, provide sufficient hypotheses for each statement to be valid):
    \begin{enumerate}
      \item $f(A) \cup f(B) = f(A \cup B)$
      \item $f(A) \cap f(B) = f(A \cap B)$
      \item $f(A - B) = f(A) - f(B)$
      \item $f^{-1}(C) \cup f^{-1}(D) = f^{-1}(C \cup D)$
      \item $f^{-1}(C) \cap f^{-1}(D) = f^{-1}(C \cap D)$
      \item $f^{-1}(C - D) = f^{-1}(C) - f^{-1}(D)$
      \item $f(f^{-1}(C)) = C$
      \item $f^{-1}(f(A)) = A$
    \end{enumerate} 
  \end{exercise}
  \begin{solution}
    Listed. 
    \begin{enumerate}
      \item 
    \end{enumerate}
  \end{solution}

  \begin{exercise}[Munkres 1.1]
    Check the distributive laws for $\cup$ and $\cap$ and DeMorgan's laws.
  \end{exercise}
  \begin{solution}
    
  \end{solution}

  \begin{exercise}[Munkres 1.2]
    Determine which of the following statements are true for all sets $A$, $B$, $C$, and $D$. If a double implication fails, determine whether one or the other of the possible implications holds. If an equality fails, determine whether the statement becomes true if the "equals" symbol is replaced by one or the other of the inclusion symbols $\subset$ or $\supset$.
    \begin{enumerate}
      \item $A \subset B$ and $A \subset C \iff A \subset (B \cup C)$.
      \item $A \subset B$ or $A \subset C \iff A \subset (B \cup C)$.
      \item $A \subset B$ and $A \subset C \iff A \subset (B \cap C)$.
      \item $A \subset B$ or $A \subset C \iff A \subset (B \cap C)$.
      \item $A - (A - B) = B$.
      \item $A - (B - A) = A - B$.
      \item $A \cap (B - C) = (A \cap B) - (A \cap C)$.
      \item $A \cup (B - C) = (A \cup B) - (A \cup C)$.
      \item $(A \cap B) \cup (A - B) = A$.
      \item $A \subset C$ and $B \subset D \Rightarrow (A \times B) \subset (C \times D)$.
      \item The converse of (j).
      \item The converse of (j), assuming that $A$ and $B$ are nonempty.
      \item $(A \times B) \cup (C \times D) = (A \cup C) \times (B \cup D)$.
      \item $(A \times B) \cap (C \times D) = (A \cap C) \times (B \cap D)$.
      \item $A \times (B - C) = (A \times B) - (A \times C)$.
      \item $(A - B) \times (C - D) = (A \times C - B \times C) - A \times D$.
      \item $(A \times B) - (C \times D) = (A - C) \times (B - D)$.
    \end{enumerate}
  \end{exercise}
  \begin{solution}
    
  \end{solution}

  \begin{exercise}[Munkres 1.3]
    \begin{enumerate}
      \item Write the contrapositive and converse of the following statement: "If $x < 0$, then $x^2 - x > 0$," and determine which (if any) of the three statements are true.
      \item Do the same for the statement "If $x > 0$, then $x^2 - x > 0$."
    \end{enumerate}
  \end{exercise}
  \begin{solution}
    
  \end{solution}

  \begin{exercise}[Munkres 1.4]
    Let $A$ and $B$ be sets of real numbers. Write the negation of each of the following statements:
    \begin{enumerate}
      \item For every $a \in A$, it is true that $a^2 \in B$.
      \item For at least one $a \in A$, it is true that $a^2 \in B$.
      \item For every $a \in A$, it is true that $a^2 \notin B$.
      \item For at least one $a \notin A$, it is true that $a^2 \in B$.
    \end{enumerate}
  \end{exercise}
  \begin{solution}
    
  \end{solution}

  \begin{exercise}[Munkres 1.5]
    Let $\mathcal{A}$ be a nonempty collection of sets. Determine the truth of each of the following statements and of their converses:
    \begin{enumerate}
      \item $x \in \bigcup_{A \in \mathcal{A}} A \Rightarrow x \in A$ for at least one $A \in \mathcal{A}$.
      \item $x \in \bigcup_{A \in \mathcal{A}} A \Rightarrow x \in A$ for every $A \in \mathcal{A}$.
      \item $x \in \bigcap_{A \in \mathcal{A}} A \Rightarrow x \in A$ for at least one $A \in \mathcal{A}$.
      \item $x \in \bigcap_{A \in \mathcal{A}} A \Rightarrow x \in A$ for every $A \in \mathcal{A}$.
    \end{enumerate}
  \end{exercise}
  \begin{solution}
    
  \end{solution}

  \begin{exercise}[Munkres 1.7]
    Given sets $A$, $B$, and $C$, express each of the following sets in terms of $A$, $B$, and $C$, using the symbols $\cup$, $\cap$, and $-$.

    $D = \{x \mid x \in A \text{ and } (x \in B \text{ or } x \in C)\}$,

    $E = \{x \mid (x \in A \text{ and } x \in B) \text{ or } x \in C\}$,

    $F = \{x \mid x \in A \text{ and } (x \in B \Rightarrow x \in C)\}$.
  \end{exercise}
  \begin{solution}
    
  \end{solution}

  \begin{exercise}[Munkres 1.8]
    If a set $A$ has two elements, show that $\mathcal{P}(A)$ has four elements. How many elements does $\mathcal{P}(A)$ have if $A$ has one element? Three elements? No elements? Why is $\mathcal{P}(A)$ called the power set of $A$?
  \end{exercise}
  \begin{solution}
    
  \end{solution}

  \begin{exercise}[Munkres 1.9]
    Formulate and prove DeMorgan's laws for arbitrary unions and intersections.
  \end{exercise}
  \begin{solution}
    
  \end{solution}

  \begin{exercise}[Munkres 1.10]
    Let $\mathbb{R}$ denote the set of real numbers. For each of the following subsets of $\mathbb{R} \times \mathbb{R}$, determine whether it is equal to the cartesian product of two subsets of $\mathbb{R}$.
    \begin{enumerate}
      \item $\{(x, y) \mid x \text{ is an integer}\}$.
      \item $\{(x, y) \mid 0 < y \leq 1\}$.
      \item $\{(x, y) \mid y > x\}$.
      \item $\{(x, y) \mid x \text{ is not an integer and } y \text{ is an integer}\}$.
      \item $\{(x, y) \mid x^2 + y^2 < 1\}$.
    \end{enumerate}
  \end{exercise}
  \begin{solution}
    
  \end{solution}

  \begin{exercise}[Munkres 2.1]
    Let $f: A \to B$. Let $A_0 \subset A$ and $B_0 \subset B$.
    \begin{enumerate}
      \item Show that $A_0 \subset f^{-1}(f(A_0))$ and that equality holds if $f$ is injective.
      \item Show that $f(f^{-1}(B_0)) \subset B_0$ and that equality holds if $f$ is surjective.
    \end{enumerate}
  \end{exercise}
  \begin{solution}
    
  \end{solution}

  \begin{exercise}[Munkres 2.2]
    Let $f: A \to B$ and let $A_i \subset A$ and $B_i \subset B$ for $i = 0$ and $i = 1$. Show that $f^{-1}$ preserves inclusions, unions, intersections, and differences of sets:
    \begin{enumerate}
      \item $B_0 \subset B_1 \Rightarrow f^{-1}(B_0) \subset f^{-1}(B_1)$.
      \item $f^{-1}(B_0 \cup B_1) = f^{-1}(B_0) \cup f^{-1}(B_1)$.
      \item $f^{-1}(B_0 \cap B_1) = f^{-1}(B_0) \cap f^{-1}(B_1)$.
      \item $f^{-1}(B_0 - B_1) = f^{-1}(B_0) - f^{-1}(B_1)$.
    \end{enumerate}
    Show that $f$ preserves inclusions and unions only:
    \begin{enumerate}[resume]
      \item $A_0 \subset A_1 \Rightarrow f(A_0) \subset f(A_1)$.
      \item $f(A_0 \cup A_1) = f(A_0) \cup f(A_1)$.
      \item $f(A_0 \cap A_1) \subset f(A_0) \cap f(A_1)$; show that equality holds if $f$ is injective.
      \item $f(A_0 - A_1) \supset f(A_0) - f(A_1)$; show that equality holds if $f$ is injective.
    \end{enumerate}
  \end{exercise}
  \begin{solution}
    
  \end{solution}

  \begin{exercise}[Munkres 2.3]
    Show that (b), (c), (f), and (g) of Exercise 2 hold for arbitrary unions and intersections.
  \end{exercise}
  \begin{solution}
    
  \end{solution}

  \begin{exercise}[Munkres 2.4]
    Let $f: A \to B$ and $g: B \to C$.
    \begin{enumerate}
      \item If $C_0 \subset C$, show that $(g \circ f)^{-1}(C_0) = f^{-1}(g^{-1}(C_0))$.
      \item If $f$ and $g$ are injective, show that $g \circ f$ is injective.
      \item If $g \circ f$ is injective, what can you say about injectivity of $f$ and $g$?
      \item If $f$ and $g$ are surjective, show that $g \circ f$ is surjective.
      \item If $g \circ f$ is surjective, what can you say about surjectivity of $f$ and $g$?
      \item Summarize your answers to (b)--(e) in the form of a theorem.
    \end{enumerate}
  \end{exercise}
  \begin{solution}
    
  \end{solution}

  \begin{exercise}[Munkres 2.5]
    In general, let us denote the identity function for a set $C$ by $i_C$. That is, define $i_C: C \to C$ to be the function given by the rule $i_C(x) = x$ for all $x \in C$. Given $f: A \to B$, we say that a function $g: B \to A$ is a left inverse for $f$ if $g \circ f = i_A$; and we say that $h: B \to A$ is a right inverse for $f$ if $f \circ h = i_B$.
    \begin{enumerate}
      \item Show that if $f$ has a left inverse, $f$ is injective; and if $f$ has a right inverse, $f$ is surjective.
      \item Give an example of a function that has a left inverse but no right inverse.
      \item Give an example of a function that has a right inverse but no left inverse.
      \item Can a function have more than one left inverse? More than one right inverse?
      \item Show that if $f$ has both a left inverse $g$ and a right inverse $h$, then $f$ is bijective and $g = h = f^{-1}$.
    \end{enumerate}
  \end{exercise}
  \begin{solution}
    
  \end{solution}

  \begin{exercise}[Munkres 2.6]
    Let $f: \mathbb{R} \to \mathbb{R}$ be the function $f(x) = x^3 - x$. By restricting the domain and range of $f$ appropriately, obtain from $f$ a bijective function $g$. Draw the graphs of $g$ and $g^{-1}$. (There are several possible choices for $g$.)
  \end{exercise}
  \begin{solution}
    
  \end{solution}

  \begin{exercise}[Munkres 3.1]
    Define two points $(x_0, y_0)$ and $(x_1, y_1)$ of the plane to be equivalent if $y_0 - x_0^2 = y_1 - x_1^2$. Check that this is an equivalence relation and describe the equivalence classes.
  \end{exercise}
  \begin{solution}
    
  \end{solution}

  \begin{exercise}[Munkres 3.2]
    Let $C$ be a relation on a set $A$. If $A_0 \subset A$, define the restriction of $C$ to $A_0$ to be the relation $C \cap (A_0 \times A_0)$. Show that the restriction of an equivalence relation is an equivalence relation.
  \end{exercise}
  \begin{solution}
    
  \end{solution}

  \begin{exercise}[Munkres 3.3]
    Here is a "proof" that every relation $C$ that is both symmetric and transitive is also reflexive: "Since $C$ is symmetric, $aCb$ implies $bCa$. Since $C$ is transitive, $aCb$ and $bCa$ together imply $aCa$, as desired." Find the flaw in this argument.
  \end{exercise}
  \begin{solution}
    
  \end{solution}

  \begin{exercise}[Munkres 3.4]
    Let $f: A \to B$ be a surjective function. Let us define a relation on $A$ by setting $a_0 \sim a_1$ if $f(a_0) = f(a_1)$.
    \begin{enumerate}
      \item Show that this is an equivalence relation.
      \item Let $A^*$ be the set of equivalence classes. Show there is a bijective correspondence of $A^*$ with $B$.
    \end{enumerate}
  \end{exercise}
  \begin{solution}
    
  \end{solution}

  \begin{exercise}[Munkres 3.5]
    Let $S$ and $S'$ be the following subsets of the plane:

    $S = \{(x, y) \mid y = x + 1 \text{ and } 0 < x < 2\}$,

    $S' = \{(x, y) \mid y - x \text{ is an integer}\}$.
    \begin{enumerate}
      \item Show that $S'$ is an equivalence relation on the real line and $S' \supset S$. Describe the equivalence classes of $S'$.
      \item Show that given any collection of equivalence relations on a set $A$, their intersection is an equivalence relation on $A$.
      \item Describe the equivalence relation $T$ on the real line that is the intersection of all equivalence relations on the real line that contain $S$. Describe the equivalence classes of $T$.
    \end{enumerate}
  \end{exercise}
  \begin{solution}
    
  \end{solution}

  \begin{exercise}[Munkres 3.6]
    Define a relation on the plane by setting $(x_0, y_0) < (x_1, y_1)$ if either $y_0 - x_0^2 < y_1 - x_1^2$, or $y_0 - x_0^2 = y_1 - x_1^2$ and $x_0 < x_1$. Show that this is an order relation on the plane, and describe it geometrically.
  \end{exercise}
  \begin{solution}
    
  \end{solution}

  \begin{exercise}[Munkres 3.7]
    Show that the restriction of an order relation is an order relation.
  \end{exercise}
  \begin{solution}
    
  \end{solution}

  \begin{exercise}[Munkres 3.8]
    Check that the relation defined in Example 7 is an order relation.
  \end{exercise}
  \begin{solution}
    
  \end{solution}

  \begin{exercise}[Munkres 3.9]
    Check that the dictionary order is an order relation.
  \end{exercise}
  \begin{solution}
    
  \end{solution}

  \begin{exercise}[Munkres 3.10]
    \begin{enumerate}
      \item Show that the map $f: (-1, 1) \to \mathbb{R}$ of Example 9 is order-preserving.
      \item Show that the equation $g(y) = 2y/[1 + (1 + 4y^2)^{1/2}]$ defines a function $g: \mathbb{R} \to (-1, 1)$ that is both a left and a right inverse of $f$.
    \end{enumerate}
  \end{exercise}
  \begin{solution}
    
  \end{solution}

  \begin{exercise}[Munkres 3.11]
    Show that an element in an ordered set has at most one immediate successor and at most one immediate predecessor. Show that a subset of an ordered set has at most one smallest element and at most one largest element.
  \end{exercise}
  \begin{solution}
    
  \end{solution}

  \begin{exercise}[Munkres 3.12]
    Let $\mathbb{Z}_+$ denote the set of positive integers. Consider the following order relations on $\mathbb{Z}_+ \times \mathbb{Z}_+$:
    \begin{enumerate}
      \item The dictionary order.
      \item $(x_0, y_0) < (x_1, y_1)$ if either $x_0 - y_0 < x_1 - y_1$, or $x_0 - y_0 = x_1 - y_1$ and $y_0 < y_1$.
      \item $(x_0, y_0) < (x_1, y_1)$ if either $x_0 + y_0 < x_1 + y_1$, or $x_0 + y_0 = x_1 + y_1$ and $y_0 < y_1$.
    \end{enumerate}
    In these order relations, which elements have immediate predecessors? Does the set have a smallest element? Show that all three order types are different.
  \end{exercise}
  \begin{solution}
    
  \end{solution}

  \begin{exercise}[Munkres 3.13]
    Prove the following theorem. If an ordered set $A$ has the least upper bound property, then it has the greatest lower bound property.
  \end{exercise}
  \begin{solution}
    
  \end{solution}

  \begin{exercise}[Munkres 3.14]
    If $C$ is a relation on a set $A$, define a new relation $D$ on $A$ by letting $(b, a) \in D$ if $(a, b) \in C$.
    \begin{enumerate}
      \item Show that $C$ is symmetric if and only if $C = D$.
      \item Show that if $C$ is an order relation, $D$ is also an order relation.
      \item Prove the converse of the theorem in Exercise 13.
    \end{enumerate}
  \end{exercise}
  \begin{solution}
    
  \end{solution}

  \begin{exercise}[Munkres 3.15]
    Assume that the real line has the least upper bound property.
    \begin{enumerate}
      \item Show that the sets

      $[0, 1] = \{x \mid 0 \leq x \leq 1\}$,

      $[0, 1) = \{x \mid 0 \leq x < 1\}$

      have the least upper bound property.
      \item Does $[0, 1] \times [0, 1]$ in the dictionary order have the least upper bound property? What about $[0, 1) \times [0, 1]$?
    \end{enumerate}
  \end{exercise}
  \begin{solution}
    
  \end{solution}
  
  \begin{exercise}[Munkres Topology 5.5] 
    Which of the following subset of $\mathbb{R}^\omega$ can be expressed as the Cartesian product of subsets of $\mathbb{R}$?\footnote{Note that the existence of these sets depend on the axiom of choice.}
  \end{exercise}
  \begin{solution}
    Listed. We will denote the sets in question as $A$. 
    \begin{enumerate}
      \item We claim that 
        \begin{equation}
          A = \mathbb{Z} \times \mathbb{Z} \times \ldots
        \end{equation}
      \item Let us denote $\mathbb{R}_{\geq i}$ be the set of reals greater than or equal to $i$. This is clearly a subset of $\mathbb{R}$. Then 
        \begin{equation}
          A = \prod_{i=1}^\infty \mathbb{R}_{\geq i}
        \end{equation}
      \item We claim 
        \begin{equation}
          A = \bigg( \prod_{i=1}^{100} \mathbb{R} \bigg) \times \bigg( \prod_{j=1}^\infty \mathbb{Z} \bigg)
        \end{equation}
      \item This is not possible
    \end{enumerate}
  \end{solution}


\section{Zermelo-Fraenkel-Choice (ZFC) Axioms} 

  So with these paradoxes in mind, we would like to construct an axiomatic formulation of sets. My take is to think that sets ``exist'' out there somewhere in the universe, and our job is to find them. Cantor with his naive set theory believed that for every meaningful property of things there is a set whose members are exactly all the things with that property. Russell shows this this cannot be the case. Nevertheless, \textit{some} sets exist, and we have intuitive experience thinking about finite sets. Therefore, the axioms of set theory are a limited list of \textit{assumptions} that we hope are true about that actually existing universe of sets. As long as they are true, then whatever we conclude from them by valid reasoning steps must also be true.\footnote{This idea is called naive Platonism.} Hence we have the following definition, which first requires the familiar property of acting like a collection of something, and then obeys the axioms we set. 
  
  \begin{definition}[Set]
    A \textbf{set} $X$ is anything 
    \begin{enumerate}
      \item that has the innate property of containing elements, and 
      \item obeys the axioms of ZFC. 
    \end{enumerate}
  \end{definition}  

  Let's first talk about the language, where they are defined formally using the axioms in the next subsection. From first-order logic, note that we have the following symbols in our alphabet $\mathcal{L}_{\mathrm{ZFC}}$. 
  \begin{enumerate}
    \item The logical connectives $\neg$, $\lor$, $\land$. 
    \item The quantifier symbols $\exists, \forall$ 
    \item Brackets $()$. 
  \end{enumerate}
  To represent sets, we also need symbols, and the membership property requires us to define a symbol for that too. 
  \begin{enumerate}
    \item A countably infinite amount of variables used for representing sets. 
    \item The set membership symbol $\in$. In fact, when we say $x \in A$, this is a proposition formed from the predicate $P(x)$. 
  \end{enumerate} 
  This is what we have to work with so far. We will construct the rest of the symbols ($=, \subset, \supset, \cup, \cap$) from the axioms. So far we don't even know if there exists any set that obeys the following axioms! Therefore, we will assert the existence of at least one set, namely the empty set. 

\subsection{ZF Axioms}

  Now we state the axioms, which is the foundation of ZF set theory. 

  \begin{axiom}[Empty Set]
    The empty set containing no elements exists. 
  \end{axiom}

  \begin{definition}[Empty Set]
    The empty set is denoted $\emptyset$. 
  \end{definition}

  This asserts the existence of at least 1 set, which we will build on to create more sets. 

  \begin{axiom}[Axiom of Extensionality]
    Two sets are equal (are the same set) if they have the same elements. 
    \begin{equation}
      \forall A \forall B \big[ \forall x (x \in A \iff x \in B) \iff A = B\big]
    \end{equation}
  \end{axiom} 

  \begin{definition}[Equality]
    This axiom allows us to define the equality operator $=$, which we now add to our alphabet. 
  \end{definition}

  \begin{theorem}[Sets Don't Contain Repeated Elements]
    Furthermore, this axiom also implies that sets are unique up to distinct elements. That is, 
    \begin{equation}
      \{1, 1, 2\} = \{1, 2\} = \{1, 1, 2, 2\}
    \end{equation}
  \end{theorem}

  \begin{axiom}[Axiom of Regularity]
    Every non-empty set $A$ contains a member $x$ such that $A$ and $x$ are disjoint sets. 
    \begin{equation}
      \forall A \big[ A \neq \emptyset \implies \exists x (x \in A \land A \cap x = \emptyset) \big]
    \end{equation}
    This, along with the axioms of pairing and union, implies that no set is an element of itself and that every set has an ordinal rank. 
  \end{axiom}

  The axiom assists us in regulating which sets are viable and which are not, preventing Russell's paradox. 

  \begin{axiom}[Axiom Schema of Restricted Comprehension, or Specification]
    Subsets, like in naive set theory, are constructed using set builder notation. In general, the \textbf{subset} of a set $A$ obeying a formula $\phi(x)$ with one free variable $x$ may be written as 
    \begin{equation}
      \{x \in A \mid \phi(x) \}
    \end{equation}
    The axiom schema of specification states that this subset always exists.\footnote{Note that this axiom does not allow the construction of entities of the more general form $\{x \mid \phi(x)\}$. This restriction is obviously needed to avoid Russell's paradox, hence the name \textit{restricted} comprehension. } 
  \end{axiom}  

  \begin{definition}[Subset, Superset]
    The axiom of specification allows us to denote subsets. Notationally, if $A$ is a subset of $B$, then we write $A \subset B$. Similarly, we say $A$ is a \textbf{superset} of $B$, written $A \supset B$, if $B \subset A$. 
  \end{definition} 
  
  \begin{definition}[Intersection]
    This also allows us to define intersection as 
    \begin{equation}
      A \cap B \coloneqq \{x \in A \mid x \in B \}
    \end{equation} 
    and we can define the intersection of an arbitrary collection of sets $\mathcal{F}$ as the following. Let $A \in \mathcal{F}$.  
    \begin{equation}
      \bigcap \mathcal{F} \coloneqq \{x \in A \mid \forall B (B \in \mathcal{F} \implies x \in B) \}
    \end{equation}
  \end{definition}

  Unfortunately, the union cannot be expressed in this specification schema, and we need a separate axiom for this. 

  \begin{definition}[Set Minus]
    We can however define set minus. Given sets $A, B$
    \begin{equation}
      A \setminus B \coloneqq \{ x \in A \mid x \not\in B \}
    \end{equation}
  \end{definition}

  \begin{definition}[Set Complement]
    Given $B$ and a subset $A \subset B$, the \textbf{complement} of $A$ with respect to $B$ is 
    \begin{equation}
      A^c \coloneqq \{ x \in B \mid x \not\in A \} = B \setminus A
    \end{equation}
  \end{definition}

  \begin{axiom}[Axiom of Pairing]
    If $A, B$ are sets, then there exists a set which contains $A$ and $B$ as elements.\footnote{For example, if $A = \{1, 2\}$ and $B = \{2, 3\}$,then $\{\{1, 2\}, \{2, 3\}\}$ exists.}
    \begin{equation}
      \forall A \forall B \exists C((A \in C) \land (B \in C))
    \end{equation}
    This allows us to construct sets from old ones. 
  \end{axiom}

  \begin{theorem}[Nested Sets]
    By the axiom of pairing, if we have a set $X$, then $\{X\}$ is also a set, since we can set $A = B = X$ which asserts the existence of $\{X, X\} = \{X\}$. 
  \end{theorem}

  \begin{axiom}[Axiom of Union]
    For any set of sets $\mathcal{F}$, there is a set $A$ containing every element that is a member of $\mathcal{F}$.
    \begin{equation}
      \forall \mathcal{F} \exists A \forall X \forall x \big[ (x \in X \land X \in \mathcal{F}) \implies x \in A \big]
    \end{equation}
  \end{axiom}

  This formula doesn't directly assert the existence of $\cup \mathcal{F}$ (?). 

  \begin{definition}[Union]
    The set $\cup \mathcal{F}$ can be constructed from $A$ in the above using the axiom schema of restricted comprehension. 
    \begin{equation}
      \cup \mathcal{F} = \{ x \in A \mid \exists X (x \in X \land X \in \mathcal{F} ) \}
    \end{equation}
  \end{definition}

  \begin{axiom}[Axiom of Infinity]
    The axiom of infinity guarantees the existence of at least one infinite set. That is, given a set $w$, let $S(w) = w \cup \{w\}$ be a set.\footnote{Since $w$ is a set, by the axiom of pairing $\{w\}$ is a set, and by the axiom of union $w \cup \{w\}$ is a set.} Then, there exists a set $X$ such that 
    \begin{enumerate}
      \item $\emptyset \in X$, and 
      \item if $w \in X$, then $S(w) \in X$. 
    \end{enumerate} 
    In logic terms, 
    \begin{equation}
      \exists X \big[ \emptyset \in X \land \forall y (y \in X \implies S(y) \in X) \big]
    \end{equation}
    Since we have axiomatically claimed the two premises to be true, by propositional logic, namely \textit{modus ponens}, this implies the existence of at least one set $X$ with infinitely many members. 
  \end{axiom}

  \begin{definition}[Von Neumann Ordinals] 
     The \textbf{Von Neumann ordinals} is the minimal set $X$ satisfying the axiom of infinity. It is the set containing 
    \begin{align*}
      0 & = \{\} = \emptyset \\
      1 & = \{0\} = \{\emptyset\} \\
      2 & = \{0,1\} = \{\emptyset,\{\emptyset\}\} \\
      3 & = \{0,1,2\} = \{\emptyset,\{\emptyset\},\{\emptyset,\{\emptyset\}\}\} \\
      4 & = \{0,1,2,3\} = \{\emptyset,\{\emptyset\},\{\emptyset,\{\emptyset\}\},\{\emptyset,\{\emptyset\},\{\emptyset,\{\emptyset\}\}\}\} \\
      \ldots & = \ldots 
    \end{align*} 
    This provides the foundation to construct the most basic mathematical sets: the natural numbers denoted $\mathbb{N}$.  
  \end{definition} 

  Now that we have constructed the Von Neumann ordinals, we are allowed to do \textit{indexing}. 

  \begin{axiom}[Axiom of Power Set]
    The axiom of power set states that for any set $A$, there is a set $B$ that contains every subset\footnote{Note that subset is defined by the axiom of restricted comprehension.} of $A$. 
    \begin{equation}
      \forall A \exists B \forall S (S \subset A \implies S \in B)
    \end{equation}
    The axiom of schema of specification is then used to define the power set as the subset of such $B$ containing the subset of $A$ exactly. 
    \begin{equation}
      2^X = \{Y \in B \mid Y \subset X \}
    \end{equation}
  \end{axiom} 

  \begin{definition}[Cartesian Product]
    The power set axiom allows for the definition of the \textbf{Cartesian product} of two sets $X$ and $Y$. Note that if $x \in X, y \in Y$, then by the axiom of union $x, y \in X \cup Y$ and by the axiom of power set $\{x\}, \{x, y\} \in \mathcal{P}(X \cup Y)$. Therefore, using the axiom of power set again we can define
    \begin{equation}
      (x, y) \coloneqq \{\{x\}, \{x, y\}\} \in \mathcal{P}(\mathcal{P}(X \cup Y))
    \end{equation} 
    and the Cartesian product is defined 
    \begin{equation}
      X \times Y \coloneqq = \{ (x, y) \in \mathcal{P}(\mathcal{P}(X \cup Y))  \mid x \in X \land y \in Y \}
    \end{equation}
    which is axiomatically a valid set by the axiom schema of specification. 
  \end{definition} 

  From this we can define the Cartesian product of any finite collection of sets recursively. It is indeed the case that $(X \times Y) \times Z$ is a different set from $X \times (Y \times Z)$, but as we will see in later functions, we can define a canonical bijection between them, treating them as equivalent. Furthermore, notice that we have not defined the Cartesian product of infinite sets yet. We can define them using functions actually. 

\subsection{Functions} 

  The definition of Cartesian products allows us to formally define \textbf{correspondences}. The most notable correspondences are \textit{functions} and \textit{relations}. 

  \begin{definition}[Function]
    Given two sets $X, Y$, a function $f: X \rightarrow Y$ is a subset $f \subset X \times Y$ satisfying the following
    \begin{enumerate}
      \item For all $x \in X$, there exists $y \in Y$ s.t. $(x, y) \in f$.\footnote{This says that $f$ must be defined for all inputs in $X$.}
      \item For all $x \in X$ and $y, y^\prime \in Y$, if $(x, y) \in f$ and $(x, y^\prime) \in f$, then $y = y^\prime$.\footnote{In other words, $f$ must map one element to exactly one element.} 
    \end{enumerate}
    The set $X$ is said to be the \textbf{domain} and $Y$ the \textbf{codomain}. 

    \begin{figure}[H]
      \centering 
      \begin{tikzcd}
        X \arrow[r, "f"'] & Y 
      \end{tikzcd}
      \caption{A diagram representing the function $f: X \rightarrow Y$.} 
      \label{fig:function at}
    \end{figure}
  \end{definition}  

  \begin{definition}[Image, Preimage]
    Given some $f: X \rightarrow Y$ and $A \subset X$, the \textbf{image} of $A$ under $f$ is defined 
    \begin{equation}
      f(A) \coloneqq \{y \in Y \mid \exists x \in X (f(x) = y)\}
    \end{equation}
    Given $B \subset Y$, the \textbf{preimage} of $B$ under $f$ is defined 
    \begin{equation}
      f^{-1} (B) \coloneqq \{ x \in X \mid f(x) \in B \}
    \end{equation}
  \end{definition} 

  \begin{axiom}[Axiom Schema of Replacement]
    This axiom asserts that the image of a set under any definable function will fall inside a set. 
  \end{axiom} 

  Again, how do we even know for sure that these axioms aren't contradictory? The answer is that we don't, and that is why we take them as axioms rather than provable theorems. Fortunately, from the formulation in the early 20th century up until now, no contradictions have been found, and if there is one, then it would be very bad news for us.  

  \begin{definition}[Injectivity, Surjectivity, Bijectivity]
    A function $f: X \rightarrow Y$ is said to be 
    \begin{enumerate}
      \item \textbf{injective} if $\forall x \in X, \forall x^\prime \in X \big( f(x) = f(x^\prime) \implies x = x^\prime \big)$. 
      \item \textbf{surjective} if $\forall y \in Y \exists x \in X (y = f(x))$. 
      \item \textbf{bijective} if it is injective and surjective. 
    \end{enumerate}
  \end{definition}

  \begin{definition}[Inverse Function]
    If a function $f: X \rightarrow Y$ is bijective, then there exists an \textbf{inverse function} $f^{-1}: Y \rightarrow X$ satisfying 
    \begin{equation}
      \forall x \in X \big[ f(f^{-1}(x)) = f^{-1} (f(x)) = x \big]
    \end{equation}
  \end{definition} 

  \begin{definition}[Restriction]
    If $f: X \rightarrow Y$ and $X_0 \subset X$, we define the \textbf{restriction} of $f$ to $X_0$ to be the function mapping to $Y$ whose rule is 
    \begin{equation}
      f|_{X_0} \coloneqq \{ (x, f(x)) \in f x \in X_0 \}
    \end{equation}
  \end{definition} 

  \begin{definition}[Composition]
    Given functions $f: X \rightarrow Y$, $g: Y \rightarrow Z$, we define the \textbf{composite}, denoted $g \circ f$ or $g(f(\cdot))$, of $f$ and $g$ as the subset
    \begin{equation}
      g \circ f \coloneqq \{ (x, z) \in X \times Z \mid \exists y \in Y (f(x) = y \land f(y) = z) \}
    \end{equation} 
  \end{definition}

  \begin{theorem}[Compositions]
    A composite is a function. 
    \begin{figure}[H]
      \centering 
      \begin{tikzcd}
        X \arrow[r, "f"] \arrow[rd, "g \circ f"'] & Y \arrow[d, "g"] \\
        & Z
      \end{tikzcd}
      \caption{Commutative diagram representing a composition of functions.} 
      \label{fig:composition}
    \end{figure}
  \end{theorem}
  \begin{proof}
    Using the definition above, we prove the two properties. 
    \begin{enumerate}
      \item For all $x \in X$, there exists $y \in Y$ s.t. $(x, y) \in f$. Similarly, for all $y \in Y$, there exists $z \in Z$ s.t. $(y, z) \in g$. Therefore, for all $x \in X$, there exists a $y \in Y$, which follows that there exists also a $z \in Z$. Therefore $g \circ f$ is defined for all inputs in $X$. 
      \item For all $x \in X$ and $z, z^\prime \in Z$, say that $(x, z), (x, z^\prime) \in g \circ f$. Then by definition of composition there exists a $y, y^\prime \in Y$ s.t. $f(x) = y, f(y) = z$ and $f(x) = y^\prime, f(y^\prime) = z^\prime$. Since $f$ is a function, $y = y^\prime$. Since $g$ is a function, $y = y^\prime \implies z = z^\prime$. Therefore $g \circ f$ is a function. 
    \end{enumerate}
  \end{proof}

  For the computer science students, note that a function behaves precisely like functional dependencies in a relational database. A composition represents a natural join. 

  \begin{theorem}[Associativity]
    Composition is associative. That is, consider $f: Y \rightarrow Z, g: X \rightarrow Y, h: W \rightarrow X$ functions. Then 
    \begin{equation}
      (f \circ g) \circ h = f \circ (g \circ h)
    \end{equation} 
    Therefore, we write this as 
    \begin{equation}
      f \circ g \circ h
    \end{equation} 
    \begin{figure}[H]
      \centering 
      \begin{tikzcd}
        & X \arrow[r, "g"] \arrow[rrd, "g \circ h"'] & Y \arrow[rd, "h"] & \\ 
        W \arrow[ru, "f"] \arrow[rru, "g \circ f"'] & & & Z 
      \end{tikzcd}
      \caption{} 
      \label{fig:associative}
    \end{figure}
  \end{theorem}
  \begin{proof}
    Consider any $w \in W$, and let us label $x = h(w)$, $y = g(x)$, $z = f(y)$, where $x, y, z$ must be uniquely determined by $w$ since it is a function. Then, 
    \begin{align}
      ((f \circ g) \circ h) (w) & = (f \circ g) (h(w)) = (f \circ g) (x) = z  \\
      (f \circ (g \circ h)) (w) & = f ((g \circ h)(w)) = f (y) = z
    \end{align}
    and they coincide for all $w \in W$. 
  \end{proof} 

  If we are familiar with algebra, this gives the set of functions $\{f: X \rightarrow X\}$ the structure of a \textit{monoid} under composition. We can also talk about commutativity. 

  \begin{definition}[Commutativity]
    Two functions $f, g: X \rightarrow X$ are said to be \textbf{commute} if 
    \begin{equation}
      f \circ g = g \circ f
    \end{equation} 
    \begin{figure}[H]
      \centering 
      \begin{tikzcd}
        X \arrow[r, "f"] \arrow[d, "g"] & X \arrow[d, "g"] \\
        X \arrow[r, "f"] & X 
      \end{tikzcd}
      \caption{Commutative diagram representing commuting functions $f, g$. } 
      \label{fig:commutative}
    \end{figure}
  \end{definition}

\subsection{Relations}

  \begin{definition}[Relation]
    A binary relation $R$ on a set $A$ is a subset of $A \times A$. We write $aRb$ if and only if $(a,b) \in R$.\footnote{It is a way of describing precisely which two elements are related to one another.} 
  \end{definition} 

  But not all relations may be meaningful or interesting. Therefore we usually like to have certain properties on these relations. 
  \begin{enumerate}
    \item \textit{Reflexive}. For all $a \in A$, $aRa$
    \item \textit{Symmetric}. For all $a,b \in A$, if $aRb$ then $bRa$
    \item \textit{Antisymmetric}. For all $a,b \in A$, if $aRb$ and $bRa$ then $a=b$
    \item \textit{Transitive}. For all $a,b,c \in A$, if $aRb$ and $bRc$ then $aRc$
    \item \textit{Total}. For all $a,b \in A$, either $aRb$ or $bRa$
  \end{enumerate}

  \begin{definition}[Equivalence Relation]
    An \textbf{equivalence relation} on a set $A$ is a relation, denoted $\sim$ satisfying 
    \begin{enumerate}
      \item \textit{Reflexive}. For all $a \in A$, $a \sim a$
      \item \textit{Symmetric}. For all $a,b \in A$, if $a \sim b$ then $b \sim a$
      \item \textit{Transitive}. For all $a,b,c \in A$, if $a \sim b$ and $b \sim c$ then $a \sim c$
    \end{enumerate}
    Given an equivalence relation, we can define an \textbf{equivalence class} as 
    \begin{equation}
      [y] \coloneqq \{ x \in A \mid x \sim y \}
    \end{equation}
  \end{definition}

  \begin{definition}[Partition]
    A \textbf{partition} of a set $X$ is a collection of disjoint nonempty subset of $X$ whose union is all of $A$. 
  \end{definition} 

  \begin{theorem}[Quotient Space, Map]
    The set of equivalence classes of a set $X$ with an equivalence relation $\sim$ is a partition of $X$, denoted as the \textbf{quotient set} $X/\sim$. Therefore, the map $\iota: X \rightarrow X/\sim$ is well-defined and is called a \textbf{quotient map}. 
  \end{theorem} 
  \begin{proof}
    Assume the contrary. If $X$ has one element, then its equivalence class is $[x]$ and this is trivially proven. If $X$ has at least 2 elements, let us call them $x, y \in X$ with $x \neq y$. $[x], [y]$ are their equivalence classes. Clearly due to reflexivity, $x \in [x]$ and $y \in [y]$ and so they are nonempty. Since we assumed that this is not a partition, there exists some $z \in X$ in both $[x], [y]$. But $z \in [x] \implies z \sim x$ and $z \in [y] \implies y \sim z$. So by transitivity, $x \sim z$, meaning that $[x] = [y]$. Therefore they must be the same element of a partition. 
  \end{proof}

  \begin{example}[Circles]
    $M$ is the set of circles in $\mathbb{R}^{2}$. Given $a, b \in M$, $a \sim b$ iff the radii are equal in length. We can denote each equivalence class by $\{ r \}$, where $r$ is the length of the radius. We can define addition as 
    \begin{equation}
      \{ a \} + \{ b \} \equiv \{ a + b\}
    \end{equation}
  \end{example}

  \begin{definition}[Order]
    An \textbf{order} is a relation $R$, usually denoted $\leq$. 
    \begin{enumerate}
      \item \textit{Reflexive}. For all $a \in A$, $a \leq a$
      \item \textit{Antisymmetric}. For all $a,b \in A$, if $a \leq b$ and $b \leq a$ then $a=b$
      \item \textit{Transitive}. For all $a,b,c \in A$, if $a \leq b$ and $b \leq c$ then $a \leq c$
    \end{enumerate} 
    If it has the final property, it is known as a \textbf{total order}/\textbf{linear order}. Otherwise it is known as a \textbf{partial order}. 
    \begin{enumerate}
      \item \textit{Total}. For all $a,b \in A$, either $a \leq b$ or $b \leq a$
    \end{enumerate} 
    Given an order $\leq$, we can define the additional symbols to mean 
    \begin{enumerate}
      \item $a < b \iff (a \leq b) \land (a \neq b)$. 
      \item $a \geq b \iff \neg(a < b)$. 
      \item $a > b \iff \neg(a \leq b)$. 
    \end{enumerate} 
  \end{definition} 

  For convenience of notation, we also write $a < x < b \iff (a < x) \land (x < b)$. 

  \begin{definition}[Interval]
    Given a totally ordered set $X$, we denote the \textbf{intervals} as 
    \begin{enumerate}
      \item $(a, b) \coloneqq \{x \in X \mid a < x < b \}$
      \item $[a, b) \coloneqq \{x \in X \mid a \leq x < b \}$
      \item $(a, b] \coloneqq \{x \in X \mid a < x \leq b \}$
      \item $[a, b] \coloneqq \{x \in X \mid a \leq x \leq b \}$
    \end{enumerate}
  \end{definition} 

  \begin{definition}[Bounds]
    Let $X$ be a set and $Y \subset X$. Then 
    \begin{enumerate}
      \item $z$ is an \textbf{upper bound} of $Y$ if $\forall y \in Y (y \leq z)$. 
      \item $z$ is a \textbf{lower bound} of $Y$ if $\forall y \in Y (z \leq y)$. 
    \end{enumerate}
  \end{definition}

\subsection{Axiom of Choice}

  The axioms up to this point are pretty much undisputed and completes ZF set theory. Now that we've defined a function, let's quickly extend the definition of a Cartesian product into an arbitrary union of sets. 
  
  \begin{definition}[Cartesian Product]
    If $\{X_\alpha\}_{\alpha \in A}$ is an indexed family of sets, then their \textbf{Cartesian product} is defined as a set of functions. That is, 
    \begin{equation}
      \prod_{\alpha \in A} X_\alpha \coloneqq \bigg\{ f: A \rightarrow \bigcup_{\alpha \in A} X_\alpha \;\Big|\; \forall \alpha \in A, f(\alpha) \in X_\alpha \bigg\} 
    \end{equation}
    Each function $f$ is called a \textbf{choice function}, which assigns to each $X_\alpha$ some element $f(\alpha) \in X_\alpha$. 
  \end{definition} 

  Therefore, we have used the power set axiom to define a finite Cartesian product, to then define a function, to then define a general Cartesian product. But this detail is irrelevant later on. Note also that this definition of Cartesian product is not the same as that of the previous definition. The binary Cartesian product is defined as $(a, b) = \{\{a\}, \{a, b\}\}$ while this defines as a function $f: \{1, 2\} \rightarrow A, B$. But once we have overwritten the old definition (which is still necessary!) we can just forget about it and use this new definition of Cartesian product since there is a canonical bijection between them. It is a lot less annoying to think of ordered tuples as just tuples rather than as sets of sets. 

  However, in our definition, we just call this a ``set of functions'' and have never proved that it actually contains anything. But we can see obviously that if this Cartesian product is nonempty then there exists a choice function, and if there exists a choice function then the Cartesian product is nonempty. It would be ideal if we can prove one of the two conditions, but it turns out we can't, and therefore we introduce the final axiom, called the \textit{axiom of choice}. Though controversial, it is required in the proofs of some notable theorems. If we include this axiom of choice, then we have ZFC set theory. The axiom of choice has many equivalent definitions. 

  Colloquially, the axiom of choice says that a Cartesian product of a collection\footnote{Note that this does not have to be finite} of non-empty sets is non-empty. That is, it is possible to construct a new set by choosing one element from each set, even if the collection is infinite. 

  \begin{axiom}[Axiom of Choice]
    Let us have an indexed family $X = \{S_i\}_{i \in I}$ of nonempty sets. Then the axiom states the following, which are all equivalent. 
    \begin{enumerate}
      \item There exists an indexed set $\{x_i\}_{i \in I}$ such that $x_i \in S_i$ for every $i \in I$. 
      \item $\prod_{i \in I} S_i$ is nonempty. 
      \item There exists a choice function $f: I \rightarrow \cup_{i \in I} S_i$. 
    \end{enumerate}
  \end{axiom}

  The existence of a choice function when $X$ is finite is easily proved from the ZF axioms, and AC only matters for certain infinite sets. One may argue that if each $S_i$ is nonempty, then choose $s_i \in S_i$ and you're done! While this is an intuition for why the axiom of choice may be true, we can't make the \textit{choice} of all the infinitely many $s_i$ in any ``canonical'' fashion. That is,k while this works for any single $i$ at a time, this doesn't define a function $i \mapsto s_i$. Note that for any sets where you \textit{can} make this choice (e.g. there is a total ordering on $X$, so choose the minimum element), AC holds as a theorem and not as an axiom. 

  \begin{example}
    Let $I$ be the set of all nonempty subsets of $\mathbb{R}$, and $X_i = i \in I$. Then an element $f$ in $\prod_{i \in I} X_i$ is a function which picks an element $f(T) \in T$ for every nonempty $T \subset \mathbb{R}$. How do you \textit{define} such an $f$? If we have $\mathbb{N}$ instead of $\mathbb{R}$, we could take $f(T) = \min(T)$, but this doesn't work for $\mathbb{R}$. Therefore, there is no canonical choice of an element in a nonempty set of real numbers. But AC tells us that we don't have to worry about this. It gives us such a function, even if we cannot ``write it down'' (which means, construct it from the other ZF axioms).  

    If we let $I$ be the set of all nonempty \textit{open} subsets of $\mathbb{R}$, then there is a choice function. Choose any bijection $\tau: \mathbb{N} \rightarrow \mathbb{Q}$, and then assign to each nonempty open subset $U \subset \mathbb{R}$ the element $\tau (\min\{n \in \mathbb{N} \mid \tau(n) \in U\})$. This works since $U \cap \mathbb{Q} \neq \emptyset$. 
  \end{example}

  It is characterized as nonconstructive because it asserts the existence of a choice function but says nothing about how to construct one, unlike the axiom of infinity. This choice function was used in the proof of the following, which turns out to be equivalent.  

  \begin{axiom}[Axiom of Well-Ordering]
    For any set $X$, there exists a binary relation $R$ which \textit{well-orders} $X$, i.e. is a total order and has the property that every nonempty subset of $X$ has a least element under the order $R$. 
    \begin{equation}
      \forall X \exists R (R \text{ well-orders } X)
    \end{equation}
  \end{axiom}

  We can see generally that we would like to use a choice function to select a representative element of each set in $X$. Then we can use these to construct an order. Finally, we state the last form of the axiom of choice. 

  \begin{axiom}[Zorn's Lemma]
    Let $X$ be a partially ordered set that satisfies the two properties. 
    \begin{enumerate}
      \item $P$ is nonempty. 
      \item Every \textit{chain} (a subset $A \subset P$ where $A$ is totally ordered) has an upper bound in $P$. 
    \end{enumerate}
    Then $P$ has at least one maximal element. 
  \end{axiom}

  Zorn's lemma is required to show that every vector space has a basis. 

\subsection{Exercises}

  \begin{exercise}[Math 531 Spring 2025, PS2.6]
    Assume that $S$ is a set with exactly $n$ elements. Assume that $T : S \to S$.
    Prove that there exists some $x \in S$ so that
    \begin{equation}
      T^j(x) = x,
    \end{equation}
    for some $j \in \{1,2,...,n\}$. Here $T^j$ means the composition of $T$ with itself
    $j$-times.
  \end{exercise}
  \begin{solution}
    Assume that the statement is false, and there exists no such $x \in S$. Let's choose any $x \in S$ and write out 
    \begin{equation}
      x = T^0 (x) , T^1 (x), T^2 (x), \ldots, T^n (x)
    \end{equation}
    These are $n+1$ elements living in a space $S$ of size $n$, so by pigeonhole principle there exists a repeat. Let us choose any of these repeats and label them $0 \leq i < j \leq n$ s.t. $T^i (x) = T^j (x)$. It cannot be the case that $i = 0$ since we assumed that it was false. Therefore, it must be the case that $1 \leq i < j \leq n \implies j - i \leq n-1$. Consider the sequence 
    \begin{equation}
      y = T^0 (y) = T^i (x), T^1 (y) = T^{i+1} (x), \ldots, T^n (y)
    \end{equation}
    Starting from $y = T^i (x) \in S$. Since $0 < j - i \leq n - 1$, we know that $T^{j-i} (y) = T^j (x)$ lies in this sequence. Since both $y = T^0 (y) = T^i (x)$ and $T^{j-i} (y) = T^j (x)$ are equal and present, we have shown an instance of when this claim is true, and the statement is true. 
  \end{solution}


\documentclass{article}

% packages
  % basic stuff for rendering math
  \usepackage[letterpaper, top=1in, bottom=1in, left=1in, right=1in]{geometry}
  \usepackage[utf8]{inputenc}
  \usepackage[english]{babel}
  \usepackage{amsmath} 
  \usepackage{amssymb}
  % \usepackage{amsthm}

  % extra math symbols and utilities
  \usepackage{mathtools}        % for extra stuff like \coloneqq
  \usepackage{mathrsfs}         % for extra stuff like \mathsrc{}
  \usepackage{centernot}        % for the centernot arrow 
  \usepackage{bm}               % for better boldsymbol/mathbf 
  \usepackage{enumitem}         % better control over enumerate, itemize
  \usepackage{hyperref}         % for hypertext linking
  \usepackage{fancyvrb}          % for better verbatim environments
  \usepackage{newverbs}         % for texttt{}
  \usepackage{xcolor}           % for colored text 
  \usepackage{listings}         % to include code
  \usepackage{lstautogobble}    % helper package for code
  \usepackage{parcolumns}       % for side by side columns for two column code
  

  % page layout
  \usepackage{fancyhdr}         % for headers and footers 
  \usepackage{lastpage}         % to include last page number in footer 
  \usepackage{parskip}          % for no indentation and space between paragraphs    
  \usepackage[T1]{fontenc}      % to include \textbackslash
  \usepackage{footnote}
  \usepackage{etoolbox}

  % for custom environments
  \usepackage{tcolorbox}        % for better colored boxes in custom environments
  \tcbuselibrary{breakable}     % to allow tcolorboxes to break across pages

  % figures
  \usepackage{pgfplots}
  \pgfplotsset{compat=1.18}
  \usepackage{float}            % for [H] figure placement
  \usepackage{tikz}
  \usepackage{tikz-cd}
  \usepackage{circuitikz}
  \usetikzlibrary{arrows}
  \usetikzlibrary{positioning}
  \usetikzlibrary{calc}
  \usepackage{graphicx}
  \usepackage{algorithmic}
  \usepackage{caption} 
  \usepackage{subcaption}
  \captionsetup{font=small}

  % for tabular stuff 
  \usepackage{dcolumn}

  \usepackage[nottoc]{tocbibind}
  \pdfsuppresswarningpagegroup=1
  \hfuzz=5.002pt                % ignore overfull hbox badness warnings below this limit

% New and replaced operators
  \DeclareMathOperator{\Tr}{Tr}
  \DeclareMathOperator{\Sym}{Sym}
  \DeclareMathOperator{\Span}{span}
  \DeclareMathOperator{\std}{std}
  \DeclareMathOperator{\Cov}{Cov}
  \DeclareMathOperator{\Var}{Var}
  \DeclareMathOperator{\Corr}{Corr}
  \DeclareMathOperator{\pos}{pos}
  \DeclareMathOperator*{\argmin}{\arg\!\min}
  \DeclareMathOperator*{\argmax}{\arg\!\max}
  \newcommand{\ket}[1]{\ensuremath{\left|#1\right\rangle}}
  \newcommand{\bra}[1]{\ensuremath{\left\langle#1\right|}}
  \newcommand{\braket}[2]{\langle #1 | #2 \rangle}
  \newcommand{\qed}{\hfill$\blacksquare$}     % I like QED squares to be black

% Custom Environments
  \newtcolorbox[auto counter, number within=section]{question}[1][]
  {
    colframe = orange!25,
    colback  = orange!10,
    coltitle = orange!20!black,  
    breakable, 
    title = \textbf{Question \thetcbcounter ~(#1)}
  }

  \newtcolorbox[auto counter, number within=section]{exercise}[1][]
  {
    colframe = teal!25,
    colback  = teal!10,
    coltitle = teal!20!black,  
    breakable, 
    title = \textbf{Exercise \thetcbcounter ~(#1)}
  }
  \newtcolorbox[auto counter, number within=section]{solution}[1][]
  {
    colframe = violet!25,
    colback  = violet!10,
    coltitle = violet!20!black,  
    breakable, 
    title = \textbf{Solution \thetcbcounter}
  }
  \newtcolorbox[auto counter, number within=section]{lemma}[1][]
  {
    colframe = red!25,
    colback  = red!10,
    coltitle = red!20!black,  
    breakable, 
    title = \textbf{Lemma \thetcbcounter ~(#1)}
  }
  \newtcolorbox[auto counter, number within=section]{theorem}[1][]
  {
    colframe = red!25,
    colback  = red!10,
    coltitle = red!20!black,  
    breakable, 
    title = \textbf{Theorem \thetcbcounter ~(#1)}
  } 
  \newtcolorbox[auto counter, number within=section]{proposition}[1][]
  {
    colframe = red!25,
    colback  = red!10,
    coltitle = red!20!black,  
    breakable, 
    title = \textbf{Proposition \thetcbcounter ~(#1)}
  } 
  \newtcolorbox[auto counter, number within=section]{corollary}[1][]
  {
    colframe = red!25,
    colback  = red!10,
    coltitle = red!20!black,  
    breakable, 
    title = \textbf{Corollary \thetcbcounter ~(#1)}
  } 
  \newtcolorbox[auto counter, number within=section]{proof}[1][]
  {
    colframe = orange!25,
    colback  = orange!10,
    coltitle = orange!20!black,  
    breakable, 
    title = \textbf{Proof. }
  } 
  \newtcolorbox[auto counter, number within=section]{definition}[1][]
  {
    colframe = yellow!25,
    colback  = yellow!10,
    coltitle = yellow!20!black,  
    breakable, 
    title = \textbf{Definition \thetcbcounter ~(#1)}
  } 
  \newtcolorbox[auto counter, number within=section]{example}[1][]
  {
    colframe = blue!25,
    colback  = blue!10,
    coltitle = blue!20!black,  
    breakable, 
    title = \textbf{Example \thetcbcounter ~(#1)}
  } 
  \newtcolorbox[auto counter, number within=section]{code}[1][]
  {
    colframe = green!25,
    colback  = green!10,
    coltitle = green!20!black,  
    breakable, 
    title = \textbf{Code \thetcbcounter ~(#1)}
  } 
  \newtcolorbox[auto counter, number within=section]{algo}[1][]
  {
    colframe = green!25,
    colback  = green!10,
    coltitle = green!20!black,  
    breakable, 
    title = \textbf{Algorithm \thetcbcounter ~(#1)}
  } 

  \BeforeBeginEnvironment{example}{\savenotes}
  \AfterEndEnvironment{example}{\spewnotes}
  \BeforeBeginEnvironment{lemma}{\savenotes}
  \AfterEndEnvironment{lemma}{\spewnotes}
  \BeforeBeginEnvironment{theorem}{\savenotes}
  \AfterEndEnvironment{theorem}{\spewnotes}
  \BeforeBeginEnvironment{corollary}{\savenotes}
  \AfterEndEnvironment{corollary}{\spewnotes}
  \BeforeBeginEnvironment{proposition}{\savenotes}
  \AfterEndEnvironment{proposition}{\spewnotes}
  \BeforeBeginEnvironment{definition}{\savenotes}
  \AfterEndEnvironment{definition}{\spewnotes}
  \BeforeBeginEnvironment{exercise}{\savenotes}
  \AfterEndEnvironment{exercise}{\spewnotes}
  \BeforeBeginEnvironment{proof}{\savenotes}
  \AfterEndEnvironment{proof}{\spewnotes}
  \BeforeBeginEnvironment{solution}{\savenotes}
  \AfterEndEnvironment{solution}{\spewnotes}
  \BeforeBeginEnvironment{question}{\savenotes}
  \AfterEndEnvironment{question}{\spewnotes}
  \BeforeBeginEnvironment{code}{\savenotes}
  \AfterEndEnvironment{code}{\spewnotes}
  \BeforeBeginEnvironment{algo}{\savenotes}
  \AfterEndEnvironment{algo}{\spewnotes}

  \definecolor{dkgreen}{rgb}{0,0.6,0}
  \definecolor{gray}{rgb}{0.5,0.5,0.5}
  \definecolor{mauve}{rgb}{0.58,0,0.82}
  \definecolor{darkblue}{rgb}{0,0,139}
  \definecolor{lightgray}{gray}{0.93}
  \renewcommand{\algorithmiccomment}[1]{\hfill$\triangleright$\textcolor{blue}{#1}}

  % default options for listings (for code)
  \lstset{
    autogobble,
    frame=ltbr,
    language=Python,
    aboveskip=3mm,
    belowskip=3mm,
    showstringspaces=false,
    columns=fullflexible,
    keepspaces=true,
    basicstyle={\small\ttfamily},
    numbers=left,
    firstnumber=1,                        % start line number at 1
    numberstyle=\tiny\color{gray},
    keywordstyle=\color{blue},
    commentstyle=\color{dkgreen},
    stringstyle=\color{mauve},
    backgroundcolor=\color{lightgray}, 
    breaklines=true,                      % break lines
    breakatwhitespace=true,
    tabsize=3, 
    xleftmargin=2em, 
    framexleftmargin=1.5em, 
    stepnumber=1
  }

% Page style
  \pagestyle{fancy}
  \fancyhead[L]{Set Theory}
  \fancyhead[C]{Muchang Bahng}
  \fancyhead[R]{Spring 2025} 
  \fancyfoot[C]{\thepage / \pageref{LastPage}}
  \renewcommand{\footrulewidth}{0.4pt}          % the footer line should be 0.4pt wide
  \renewcommand{\thispagestyle}[1]{}  % needed to include headers in title page

\begin{document}

\title{Set Theory}
\author{Muchang Bahng}
\date{Spring 2025}

\maketitle
\tableofcontents
\pagebreak

\section{Intro} 

  \subsection{Natural Numbers and Induction}

    \begin{definition}[Inductive Set, Natural Numbers]
    A set $X \subset \mathbb{R}$ is inductive if for each number $x \in X$, it also contains $x + 1$. The set of \textit{natural numbers}j, denoted $\mathbb{N}$, is the smallest inductive set containing $1$. 
    \end{definition}

    We can use this inductive property of natural numbers to prove properties of them. Note that this can only be used to prove for finite (yet unbounded) numbers! 

    \begin{lemma}[Induction Principle]
    Given $P(n)$, a property depending on positive integer $n$, 
    \begin{enumerate}
        \item if $P(n_0)$ is true for some positive integer $n_0$, and
        \item if for every $k \geq n_0$, $P(k)$ true implies $P(k+1)$ true, 
    \end{enumerate}
    then $P(n)$ is true for all $n \geq n_0$. 
    \end{lemma}

    \begin{lemma}[Strong Induction Principle]
    Given $P(n)$, a property depending on a positive integer $n$, 
    \begin{enumerate}
        \item if $P(n_0), P(n_0 + 1), \ldots, P(n_0 + m)$ are true for some positive integer $n_0$, and nonnegative integer $m$, and 
        \item if for every $k > n_0 + m, P(j)$ is true for all $n_0 \leq j \leq k$ implies $P(k)$ is true, 
    \end{enumerate}
    then $P(n)$ is true for all $n \geq n_0$. 
    \end{lemma}

    The idea behind the strong induction principle leads to the proof using infinite descent. Infinite descent combines strong induction with the fact that every subset of the positive integers has a smallest element, i.e. there is no strictly decreasing infinite sequence of positive integers. 

    \begin{lemma}[Infinite Descent]
    Given $P(n)$, a property depending on positive integer, assume that $P(n)$ is false for a set of integers $\mathcal{S}$. Let the smallest element of $\mathcal{S}$ be $n_0$. If $P(n_0)$ false implies $P(k)$ false, where $k < n_0$, then by contradiction $P(n)$ is true for all $n$. 
    \end{lemma}

    \subsection{Countable and Uncountable Sets}

    \begin{definition}[Equipotence]
    Two sets $A$ and $B$ are \textbf{equipotent}, written $A \approx B$, if there exists a bijective map $f: A \rightarrow B$. This implies that their cardinalities are the same: $|A| = |B|$. It has the following properties: 
    \begin{enumerate}
        \item Reflexive: $A \approx A$
        \item Symmetric: $A \approx B$ implies $B \approx A$
        \item Transitive: $A \approx B$ and $B \approx C$ implies $A \approx C$
    \end{enumerate}
    \end{definition}

    \begin{definition}
    For any positive integer $n$, let $J_n$ be the set whose elements are the integers $1, 2, \ldots, n$. For any set $A$, we define 
    \begin{enumerate}
        \item $A$ is \textbf{finite} if $A \approx J_n$ for some $n$. The empty set is also considered to be finite. 
        \item $A$ is \textbf{infinite} if it is not finite. 
        \item $A$ is countable if $A \approx \mathbb{N}$. 
        \item $A$ is uncountable if $A$ is neither finite nor countable. 
        \item $A$ is at most countable if $A$ is finite or countable. 
    \end{enumerate}
    \end{definition}

    At this point, we may already be familiar with the fact that $\mathbb{Q}$ is countable and $\mathbb{R}$ is uncountable. Let us formalize the statement that a countable infinity is the smallest type of infinity. We can show this by taking a countable set and showing that every infinite subset must be countable. If it was uncountable, then this would mean that a countable set contains an uncountable set. 

    \begin{theorem}
    \label{countable smallest}
    Every infinite subset of a countable set $A$ is countable. 
    \end{theorem}

    \begin{theorem}
    An at most countable union of countable sets is countable. 
    \end{theorem}

    \begin{theorem}
    A finite Cartesian product of countable sets is countable. 
    \end{theorem}

    \begin{corollary}
    $\mathbb{Q}$ is countable. 
    \end{corollary}

    Now, how do we prove that a set is uncountable? We can't really use the contrapositive of Theorem $\ref{countable smallest}$, since to prove that an arbitrary set $A$ is uncountable, then we must find an infinite subset that is not countable. But now we must prove that this subset itself is not countable, too! Therefore, we can use this theorem. 

    \begin{theorem}
    Given an arbitrary set $A$, if every countable subset $B$ is a proper subset of $A$, then $A$ is uncountable. 
    \end{theorem}
    \begin{proof}
    Assume that $A$ is countable. Then $A$ itself is a countable subset of $A$, but by the assumption, $A$ should be a proper subset of $A$, which is absurd. Therefore, $A$ is uncountable. 
    \end{proof}

    \begin{theorem}
    Let $A$ be the set of all sequences whose elements are the digits $0$ and $1$. Then, $A$ is uncountable. 
    \end{theorem}




\end{document}

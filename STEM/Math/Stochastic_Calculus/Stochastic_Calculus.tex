\documentclass{article}
\usepackage[a4paper, top=1in, bottom=1in, left=1in, right=1in]{geometry}
\usepackage[utf8]{inputenc}

\usepackage[english]{babel}
\usepackage{tikz-cd, lipsum, bm, dcolumn}
\usetikzlibrary{arrows}
\usepackage{amsmath, amssymb, amsthm, mathrsfs, mathtools, centernot, hyperref, fancyhdr, lastpage}


\renewcommand{\thispagestyle}[1]{}

\DeclareMathOperator{\Tr}{Tr}
\DeclareMathOperator{\Sym}{Sym}
\DeclareMathOperator{\Span}{span}
\DeclareMathOperator{\im}{Im}
\DeclareMathOperator{\Div}{div}
\DeclareMathOperator{\curl}{curl}
\DeclareMathOperator{\GL}{GL}
\DeclareMathOperator{\SL}{SL}
\DeclareMathOperator{\GA}{GA}
\DeclareMathOperator{\std}{std}
\DeclareMathOperator{\Cov}{Cov}
\DeclareMathOperator{\Var}{Var}
\DeclareMathOperator{\Corr}{Corr}
\DeclareMathOperator{\Int}{Int}
\DeclareMathOperator{\Id}{Id}
\DeclareMathOperator{\Lie}{Lie}
\DeclareMathOperator{\Hom}{Hom}
\DeclareMathOperator{\Alt}{Alt}
\DeclareMathOperator{\rank}{rank}
\DeclareMathOperator{\conv}{conv}
\DeclareMathOperator{\aff}{aff}
\DeclareMathOperator{\arccot}{arccot}


\newtheorem{theorem}{Theorem}[section]
\newtheorem{proposition}[theorem]{Proposition}
\newtheorem{lemma}[theorem]{Lemma}
\newtheorem{example}{Example}[section]
\newtheorem{corollary}{Corollary}[theorem]
\theoremstyle{remark}
\newtheorem*{remark}{Remark}
\theoremstyle{definition}
\newtheorem{definition}{Definition}[section]
\renewcommand{\qed}{\hfill$\blacksquare$}
\renewcommand{\footrulewidth}{0.4pt}% default is 0pt


\begin{document}
\pagestyle{fancy}

\lhead{Stochastic Calculus}
\chead{Muchang Bahng}
\rhead{\date{August 2021}}
\cfoot{\thepage / \pageref{LastPage}}

\title{Stochastic Calculus}
\author{Muchang Bahng}
\date{November 2022}

\maketitle
\tableofcontents
\pagebreak

\section{Introduction} 
  Let us start off with an ordinary differential equation 
  \[\frac{d N(t)}{d t} = \alpha (t) \, N(t)\]
  where. 
  Given an equation 
  \begin{equation}
    \frac{N(t + \Delta t) - N(t)}{\Delta t} & = \alpha (t) N(t) + \epsilon
    \label{eq:}
  \end{equation}
  where $\epsilon$ is some random variable representing noise.

\section{Stuff}

\end{document}

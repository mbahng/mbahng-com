\documentclass{article}

  % preamble
  \usepackage[letterpaper, top=1in, bottom=1in, left=1in, right=1in]{geometry}
  \usepackage[utf8]{inputenc}
  \usepackage[english]{babel}
  \usepackage{amsmath, amssymb, amsthm, mathtools} % necessary
  \usepackage{lastpage} % insert last page number
  \usepackage{centernot} % for not slash

  \usepackage{pgfplots}
  \pgfplotsset{compat=1.18}
  \usepackage{hyperref} % hyperlinks
  \usepackage{fancyhdr} % fancy headers
  \usepackage{fancyvrb} % verbatim
  \usepackage{parskip}

  \usepackage{subcaption} % captions for figures
  \definecolor{cverbbg}{gray}{0.93}

  \setlength{\parindent}{0pt} % set no indent
  \hfuzz=5.002pt % ignore overfull hbox badness warnings below this limit

  \DeclareMathOperator{\Tr}{Tr}
  \DeclareMathOperator{\Sym}{Sym}
  \DeclareMathOperator{\Span}{span}
  \DeclareMathOperator{\std}{std}
  \DeclareMathOperator{\Cov}{Cov}
  \DeclareMathOperator{\Var}{Var}
  \DeclareMathOperator{\Corr}{Corr}
  \DeclareMathOperator*{\argmin}{\arg\!\min}
  \DeclareMathOperator*{\argmax}{\arg\!\max}
  \newenvironment{question}{\color{blue}}{\ignorespacesafterend}

  \newcommand{\ket}[1]{\ensuremath{\left|#1\right\rangle}}
  \newcommand{\bra}[1]{\ensuremath{\left\langle#1\right|}}
  \newcommand{\braket}[2]{\langle #1 | #2 \rangle}

  \theoremstyle{definition}
  \newtheorem{theorem}{Theorem}[section]
  \newtheorem{proposition}[theorem]{Proposition}
  \newtheorem{lemma}[theorem]{Lemma}
  \newtheorem{example}{Example}[section]
  \newtheorem{exercise}{Exercise}[section]
  \newtheorem{corollary}{Corollary}[theorem]
  \newtheorem{definition}{Definition}[section]
  \renewcommand{\qed}{\hfill$\blacksquare$}
  \renewcommand{\footrulewidth}{0.4pt}% default is 0pt

  \newenvironment{solution}{\noindent \textit{Solution.}}{}
  \newenvironment{cverbatim}
    {\SaveVerbatim{cverb}}
    {\endSaveVerbatim
    \flushleft\fboxrule=0pt\fboxsep=.5em
    \colorbox{cverbbg}{%
      \makebox[\dimexpr\linewidth-2\fboxsep][l]{\BUseVerbatim{cverb}}%
    }
    \endflushleft
  }

  \renewcommand{\thispagestyle}[1]{} % needed for including header in title page

\begin{document}
\pagestyle{fancy}

\lhead{Quantum Mechanics}
\chead{Muchang Bahng}
\rhead{\date{Spring 2024}}
\cfoot{\thepage / \pageref{LastPage}}

\title{Quantum Mechanics}
\author{Muchang Bahng}
\date{Spring 2024}

\maketitle
\tableofcontents
\pagebreak

\section{Wavefunctions} 

  Imagine a particle of mass $m$ constrained to move along the x-axis, subject to some applied force $F(x, t)$ (time-dependent). In classical mechanics, we want to determine the position of the particle at any given time $t$ by finding the function $x(t)$. How do we do this? We simply apply \textit{Newton's second law}. 
  \begin{equation} 
    \mathbf{F} = m \mathbf{a}
  \end{equation}
  and solve the ordinary differential equation (with some initial conditions), either analytically or numerically. Once we have $x(t)$, we can find other metrics of interest, such as the velocity $v(t)$, kinetic energy $T = \frac{1}{2} mv^2$, or others. For conservative systems (the only kind we'll consider, and fortunately the only kind that exists at the microscopic level), the force can be expressed as the gradient of a potential energy function $U(\mathbf{x})$. 

  \subsection{Schr\"odingers Equation}

    Quantum mechanics approaches the same problem differently. Rather than looking for position function $\mathbf{x}(t)$, we are looking for a \textbf{wave function} $\boldsymbol{\Psi}(\mathbf{x}, t)$ of the particle, which we can get by solving the \textit{Schr\"odinger equation}. 

    \begin{definition}[Schr\"odinger Equation]
      The \textbf{Schr\"odinger equation} is defined 
      \begin{equation} 
        i \hbar \frac{\partial \boldsymbol{\Psi}}{\partial t} = - \frac{\hbar^2}{2m} \frac{\partial^2 \boldsymbol{\Psi}}{\partial \mathbf{x}^2} + V \boldsymbol{\Psi}
      \end{equation}
      where $\hbar$ is the \textbf{reduced Planck's constant}, defined 
      \begin{equation} 
        \hbar = \frac{h}{2\pi} = 1.054573 \times 10^{-34} \mathrm{J}s
      \end{equation}
    \end{definition}

    Therefore, given suitable initial conditions $\boldsymbol{\Psi}(\mathbf{x}, 0)$, the Schr\"odinger equation determines $\boldsymbol{\Psi}(\mathbf{x}, t)$ for all future time. Now let's talk about this wave function and its physical interpretation, starting with \textbf{Born's statistical interpretation}. This says that the wavefunction $\boldsymbol{\Psi}(\mathbf{x}, t)$ determines the probability of finding th particle at point $\mathbf{x}$ at time $t$. That is, the probability density function of the particle's position at time $t$ is given by 
    \begin{equation} 
      f_{X, t} (\mathbf{x}, t) = |\boldsymbol{\Psi}(\mathbf{x}, t)|^2
    \end{equation}
    Unfortunately, Schr\"odinger's equation is a linear system, so the set of solutions to this equation forms a vector space. So if $\boldsymbol{\Psi}$ is a solution, then $c \boldsymbol{\Psi}$ is also a solution for all $c \in \mathbb{C}$. This is a problem since it must be normalized. This is why we have an extra condition that 
    \begin{equation} 
    \int |\boldsymbol{\Psi}(\mathbf{x}, t)|^2 \,d \mathbf{x} = 1
    \end{equation}
    which means that the probability of finding a particle somewhere in the space $X$ at a certain point $t$ must integrate to $1$. This gives us a normalization condition, and any functions that have an integral of infinity is not within our search space. Therefore, we're really just trying to find a function in the $L^2$ space of integrable functions. 

    Furthermore, fortunately for us, Schr\"odinger's equation keeps this normalization condition as time passes. Let's prove this. 

    \begin{theorem} 
      Given a solution $\boldsymbol{\Psi}$ that has been normalized, the function will stay normalized as time passes. 
    \end{theorem}
    \begin{proof} 
      Let us take the time-derivative of the total probability and show that is is $0$. We show that 
      \begin{equation} 
        \frac{d}{dt}  \int |\boldsymbol{\Psi} (\mathbf{x}, t)|^2 \,dx = \int \frac{\partial}{\partial t} |\boldsymbol{\Psi}(\mathbf{x}, t)|^2 \,dx
      \end{equation}
      and by the product rule we have 
      \begin{equation} 
        \frac{\partial}{\partial t} |\boldsymbol{\Psi}|^2 = \frac{\partial}{\partial t} (\boldsymbol{\Psi}^\ast \boldsymbol{\Psi}) = \boldsymbol{\Psi}^\ast \frac{\partial \boldsymbol{\Psi}}{\partial t} + \frac{\partial \boldsymbol{\Psi}^\ast}{\partial t} \boldsymbol{\Psi}
      \end{equation}
      We can take the complex conjugate of the Schr\"odinger equation to get 
      \begin{align} 
        \frac{\partial \boldsymbol{\Psi}}{\partial t} & = \frac{i \hbar}{2m} \frac{\partial^2 \boldsymbol{\Psi}}{\partial x^2} - \frac{i}{\hbar} V \boldsymbol{\Psi} \\
        \frac{\partial \boldsymbol{\Psi}^\ast}{\partial t} & = - \frac{i \hbar}{2m} \frac{\partial^2 \boldsymbol{\Psi}^\ast}{\partial x^2} + \frac{i}{\hbar} V \boldsymbol{\Psi}^\ast 
      \end{align}
      and substituting both equations into the product rule gives 
      \begin{equation} 
        \frac{\partial}{\partial t} |\boldsymbol{\Psi}|^2 = \frac{i \hbar}{2m} \bigg( \boldsymbol{\Psi}^\ast \frac{\partial^2 \boldsymbol{\Psi}}{\partial x^2} - \frac{\partial^2 \boldsymbol{\Psi}^\ast}{\partial x^2} \boldsymbol{\Psi} \bigg) = \frac{\partial}{\partial x} \bigg[ \frac{i \hbar}{2m} \bigg( \boldsymbol{\Psi}^\ast \frac{\partial \boldsymbol{\Psi}}{\partial x} - \frac{\partial \boldsymbol{\Psi}^\ast}{\partial x} \boldsymbol{\Psi} \bigg)\bigg]
      \end{equation}
      and now we can evaluate the integral to be 
      \begin{equation} 
        \frac{d}{dt} \int |\boldsymbol{\Psi}(x, t)|^2 \,dx = \frac{i \hbar}{2m} \big( \boldsymbol{\Psi}^\ast \frac{\partial \boldsymbol{\Psi}}{\partial x} - \frac{\partial \boldsymbol{\Psi}^\ast}{\partial x} \boldsymbol{\Psi} \bigg) \bigg|_{-\infty}^{+\infty} 
      \end{equation}
      which evaluates to $0$ since $\boldsymbol{\Psi}$ must go to $0$ as it is a probability density. 
    \end{proof}

    Now that we have settled on what a wavefunction is, we can say that quantum mechanics can offer only statistical information about the possible results, rather than what the actual position of the particle is. Now if we measure the position of the particle and find it to be at the point $C$, then where was the particle before I measured it? 

    \subsubsection{Ket Notation} 

      We can talk about another paradigm of representing the states of a system: as a ket. We can equivalently say that the state of a system is represented by a normalized vector $\ket{\psi}$ in the Hilbert space $L^2$. 

      In classical systems, we have some configuration space represented by the position, momentum, and time $(x, p, t)$, and we want to calculate some dynamic quantities. 
      \begin{enumerate} 
        \item The position of the vector is simply the $x$ value of the tuple. 
        \item The kinetic energy is $K(x, p, t) = \frac{1}{2m} p^2$. 
        \item The potential energy is $U(x, p, t) = U(x)$, which is dependent only on $x$ for conservative systems. 
      \end{enumerate}
      Essentially, every observable quantity is some function $Q(x, p, t)$, a function $Q: X \times P \times T \rightarrow \mathbb{R}$. We would like a quantum mechanical analogue of these functions, which would act on the wavefunction $\boldsymbol{\Psi}$. Note that we can't simply get point estimates like \textit{where is the particle located at} or \textit{what is the kinetic energy}. These are all probabilistic, so what we can actually do is take their expected values and variances. 

      Second, we can take each classical observable $Q$ and associate with a quantum mechanical observable, also denoted $Q$, which would now be a random variable. To get this quantum observable $Q$, we have an associated quantum mechanical operator $\hat{Q}: L^2 \rightarrow L^2$ obtained by the canonical substitution 
      \begin{equation} 
        p \rightarrow \frac{\hbar}{i} \frac{\partial}{\partial x} 
      \end{equation}
      Note that in QM, the actual operator $\hat{Q}$ is known as the observable, and the $Q$ is like a (random) physical quantity. 

      \begin{example}
        The $\hat{Q}$ operators are: 
        \begin{enumerate} 
          \item position: $\psi \rightarrow x \psi$ 
          \item momentum: $\psi \rightarrow \frac{\partial}{\partial x} \psi $ 
          \item kinetic energy: $\psi \rightarrow \frac{1}{2m} \frac{\hbar^2}{-1} \frac{\partial^2}{\partial x^2} \psi$
        \end{enumerate}
      \end{example}

      \begin{definition}[Expectation of Operator]
        The expectation value of the operator (as a function of time) is
        \begin{equation} 
          \langle Q \rangle (t) \coloneqq \int \Psi (x, t)^\ast (\hat{Q} \Psi) (x, t) \,dx = \braket{\Psi}{\hat{Q} \Psi}(t)
        \end{equation}
      \end{definition}

      Since physical observables must be real (to see why, look \href{https://physics.stackexchange.com/questions/436462/why-is-there-a-physical-preference-to-real-numbers}{here}), it must be the case that 
      \begin{equation} 
        \braket{\Psi}{\hat{Q} \Psi} = \braket{\Psi}{\hat{Q} \Psi}^\ast = \braket{\hat{Q} \Psi}{\Psi}
      \end{equation}
      for all vectors $\ket{\Psi}$, so it must follow that $\hat{Q}$ must be a Hermitian operator. This leads to our second postulate on measurements. 

      \begin{theorem} 
        Observable quantities, $Q(x, p, t)$, are represented by Hermitian operators $\hat{Q}(x, \frac{\hbar}{i} \frac{\partial}{\partial x}, t)$. The expectation value of $Q$ in the state $\Psi$ at time $t$ is $\braket{\Psi}{\hat{Q} \Psi} (t)$. 
      \end{theorem}
     
      The output of the operator doesn't necessarily have to be a wavefunction (doesn't need to be normalized, look at the position operator), but it should be normalizable. 

      Observables and measurements are two different quantities. Observables don't describe anything about measurements. 

      You have a system in a state. Once you perform a measurement, you get a value. That value is an eigenvalue of the observable that you measured. A measurement is like taking a realization of a random varibale. Once you get that eigenvalue, the wavefunction collapses into the eigenstate (eigenfunction) of that eigenvalue. The state is a linear combination of eigenstates. There is some probility distribution for the 

      Before you measure, you want to choose the observable that you want to measure. You're essentially drawing from the (possibly uncountable) set of eigenvalues. To get probability, you expand your wavefunction in the basis of eigenfunctions, and the amplitude squared of the  coefficients gives you the respective probabilities. 

      Once you measure, the wavefunction collapses onto the delta and immediately begins to smear out again. If you measure really soon after, you can still sample with approximately probability 1 at the point you measured. 

      Position: Amplitude squared is the pdf. The moment you draw a x from that distribution, the pdf collapses to that delta function as x. So it's a realization of a random variable. 
      

  \subsection{Measurements}

  The \textit{Copenhagen interpretation} says that the particle wasn't really anywhere. It was the act of measurement that forced the particle to \textbf{collapse} from a probabilistic wavefunction to a delta bump function. However, if we made a second measurement immediately after the first, then it must return the same value. Therefore, there are two different types of physical processes: \textit{ordinary} ones, in which the wavefunction evolves under the Schr\"odinger equation, and \textit{measurements}, in which $\boldsymbol{\Psi}$ suddenly and discontinuously collapses. 

  \begin{theorem} 
    The expected position of the particle can be written
    \begin{equation} 
      \langle \mathbf{x} \rangle \coloneqq \int x |\boldsymbol{\Psi}(\mathbf{x}, t)|^2 \,d \mathbf{x}
    \end{equation}
    and the velocity of the expected value can be evaluated to: 
    \begin{equation} 
      \frac{d}{dt} \langle \mathbf{x} \rangle = -\frac{i \hbar}{m} \int \boldsymbol{\Psi}^\ast \frac{\partial \boldsymbol{\Psi}}{\partial x} \,dx
    \end{equation}
  \end{theorem}
  \begin{proof} 
    We can see that using integration by parts in the penultimate step, 
    \begin{align} 
      \frac{d \langle x \rangle}{dt} & = \int x \frac{\partial}{\partial t} |\boldsymbol{\Psi}|^2 \,dx \\
                                     & = \frac{i \hbar}{2m} \int x \frac{\partial}{\partial x} \bigg( \boldsymbol{\Psi}^\ast \frac{\partial \boldsymbol{\Psi}{\partial x}} - \frac{\partial \boldsymbol{\Psi}^\ast}{\partial x} \boldsymbol{\Psi} \bigg) \\
                                     & = - \frac{i \hbar}{2m} \int \bigg( \boldsymbol{\Psi}^\ast \frac{\partial \boldsymbol{\Psi}}{\partial x} - \frac{\partial \boldsymbol{\Psi}^\ast}{\partial x} \boldsymbol{\Psi} \bigg) \,dx \\
                                     & = -\frac{i \hbar}{m} \int \boldsymbol{\Psi}^\ast \frac{\partial \boldsymbol{\Psi}}{\partial x} \,dx
    \end{align}
  \end{proof}

  Note that this does not mean that if we take a single particle and measure it multiple times, we will get the expected value, since it will first evolve and second we will get the exact same measurement. Rather, if we take an \textit{ensemble} of particles all in the same state $\mathbf{\Psi}$ and measure them all at once, then the histogram of measurements can be used as an unbiased estimator of $\langle x \rangle$. Therefore, we can define momentum as the following. 

  \begin{definition}[Momentum]
    The momentum is defined as 
    \begin{equation} 
      \langle p \rangle \coloneqq m \frac{d \langle x \rangle}{dt} = - i\hbar \int \bigg( \boldsymbol{\Psi}^\ast \frac{\partial \boldsymbol{\Psi}}{\partial x} \bigg)  \,dx
    \end{equation}
  \end{definition}

\section{Time Independent Schr\"odinger Equation}

  Let's talk about solving the Schr\"odinger function itself by looking at the simple case when the Schr\"odinger equation is time independent and we can thus solve it by the method of separation of variables. That is, we look for solutions of the form 
  \begin{equation} 
    \boldsymbol{\Psi}(x, t) = \psi(x) f(t)
  \end{equation}

  We have 
  \begin{equation} 
    \frac{\partial \Psi}{\partial t} = \psi \frac{d f}{\partial t} , \;\;\; \frac{\partial^2}{\partial x^2} = \frac{d^2 \psi}{\partial x^2} = \frac{d^2 \psi}{d x^2} f
  \end{equation}
  and the Schr\"odinger equation becomes 
  \begin{equation} 
    i \hbar \psi \frac{d f}{d t} = -\frac{\hbar^2}{2m}\frac{d^2 \psi}{d x^2} f + V \psi f
  \end{equation}
  and dividing by $f$ on both sides gives us 
  \begin{equation} 
    i \hbar \frac{1}{f} \frac{df}{dt} = -\frac{\hbar^2}{2m} \frac{1}{\psi} \frac{d^2 \psi}{d x^2} + V
  \end{equation}
  The LHS as a function of $t$ alone and RHS as a function of $x$ alone. This is only possible if both sides are constant (since we can simply change $t$ or $x$ to get different values). Now setting the LHS as $E$, we can rewrite the partial differential equation as a system of two ODEs. 
  \begin{align} 
    \frac{df}{dt} = - \frac{i E}{\hbar} f \\
    - \frac{\hbar^2}{2m} \frac{d^2 \psi}{dx^2} + V \psi = E \psi
  \end{align}
  The first equation is easy to solve since we can just integrate, which gives $f(t) = e^{-i E t/ \hbar}$. The second equation is called the time-independent Schr\"odinger equation, which we will need the potential $V$ to solve, but it is of form $\boldsymbol{\Psi}(x, t) = \psi(x) e^{-i E t/\hbar}$. 

  \begin{definition}[Stationary State]
    A wavefunction $\boldsymbol{\Psi}(x, t)$ is a \textbf{stationary state} if its corresponding probability density does not depend on $t$. That is, 
    \begin{equation} 
      |\boldsymbol{\Psi}(x, t)|^2 = |\boldsymbol{\Psi}(x, t^\prime)|^2  \text{ for all } t, t^\prime \in \mathbb{R}
    \end{equation}
    This means that the expected position is always constant since the probability density function never changes with time. 
  \end{definition}

  It turns out that solutions to separable equations all stationary states since we have 
  \begin{equation} 
    |\boldsymbol{\Psi}(x, t)|^2 = \boldsymbol{\Psi}^\ast \boldsymbol{\Psi} = \psi^\ast e^{+i E t/\hbar} \psi e^{- i E t/\hbar} = |\psi (x)|^2
  \end{equation}



  In classical mechanics, the Hamiltonian is defined as the total (kinetic plus potential) energy. It is a scalar valued function.  
  \begin{equation} 
    H(x, p) \coloneqq \frac{1}{2m} p^2 + V(x) 
  \end{equation}

  \begin{definition}[Hamiltonian in Quantum Mechanics]
  The corresponding Hamiltonian operator is an operator on the $L^2$ function space of wavefunctions. It can be obtained by the canonical substitution $p \rightarrow (\hbar/i)(\partial /\partial x)$ and is 
    \begin{equation} 
      \hat{H} \coloneqq - \frac{\hbar^2}{2m} \frac{\partial^2}{\partial x^2} + V(x)
    \end{equation}  
  \end{definition}

  Therefore, the time-independent Schr\"odinger equation can be written as 
  \begin{equation} 
    \hat{H} \psi = E \psi 
  \end{equation}
  and the expectation value of the total energy is 
  \begin{equation} 
    \langle H \rangle = \int \psi^\ast \hat{H} \psi \,dx = E \int |\psi|^2 \,dx = E
  \end{equation}

\section{Generalized Statistical Interpretation}

  



\end{document} 

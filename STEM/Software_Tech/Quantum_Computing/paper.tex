\documentclass{article}

% preamble 
  \usepackage[letterpaper, top=1in, bottom=1in, left=1in, right=1in]{geometry}
  \usepackage[utf8]{inputenc}
  \usepackage[english]{babel}
  \usepackage{amsmath, amssymb, amsthm, mathtools} % necessary 
  \usepackage{lastpage} % insert last page number 
  \usepackage{centernot} % for not slash 

  \usepackage{pgfplots}
  \pgfplotsset{compat=1.18}
  \usepackage{hyperref} % hyperlinks 
  \usepackage{fancyhdr} % fancy headers 
  \usepackage{fancyvrb} % verbatim 

  \usepackage{subcaption} % captions for figures 
  \definecolor{cverbbg}{gray}{0.93}

  \setlength{\parindent}{0pt} % set no indent
  \hfuzz=5.002pt % ignore overfull hbox badness warnings below this limit

  \DeclareMathOperator{\Tr}{Tr}
  \DeclareMathOperator{\Sym}{Sym}
  \DeclareMathOperator{\Span}{span}
  \DeclareMathOperator{\std}{std}
  \DeclareMathOperator{\Cov}{Cov}
  \DeclareMathOperator{\Var}{Var}
  \DeclareMathOperator{\Corr}{Corr}
  \DeclareMathOperator*{\argmin}{\arg\!\min}
  \DeclareMathOperator*{\argmax}{\arg\!\max}

  \theoremstyle{definition}
  \newtheorem{theorem}{Theorem}[section]
  \newtheorem{proposition}[theorem]{Proposition}
  \newtheorem{lemma}[theorem]{Lemma}
  \newtheorem{example}{Example}[section]
  \newtheorem{exercise}{Exercise}[section]
  \newtheorem{corollary}{Corollary}[theorem]
  \newtheorem{definition}{Definition}[section]
  \renewcommand{\qed}{\hfill$\blacksquare$}
  \renewcommand{\footrulewidth}{0.4pt}% default is 0pt
  
  \newenvironment{solution}{\noindent \textit{Solution.}}{}
  \newenvironment{cverbatim}
    {\SaveVerbatim{cverb}}
    {\endSaveVerbatim
    \flushleft\fboxrule=0pt\fboxsep=.5em
    \colorbox{cverbbg}{%
      \makebox[\dimexpr\linewidth-2\fboxsep][l]{\BUseVerbatim{cverb}}%
    }
    \endflushleft
  }

  \renewcommand{\thispagestyle}[1]{} % needed for including header in title page

\begin{document}
\pagestyle{fancy}

\lhead{Quantum Computing}
\chead{Muchang Bahng}
\rhead{\date{Spring 2024}}
\cfoot{\thepage / \pageref{LastPage}}

\title{Quantum Computing}
\author{Muchang Bahng}
\date{Spring 2024}

\maketitle
\tableofcontents
\pagebreak 


\section{Introduction}

In quantum mechanics we usually work in a Hilbert space $V$ over field $\mathbb{C}$. A \textbf{ket} is of the form $\mid v \rangle$, which mathematically denotes a vector $v$ in $V$. A \textbf{bra} is of the form $\langle f \mid$, which denotes a covector $f \in V^ast$, the dual space. With the usual construction, we say if $\mid m \rangle$ defines a column vector in $\mathbb{C}^n$, then $\langle m \mid$ is its conjugate transpose. Furthermore, let us write the shorthand notation for the classical bit as such: 

  \[\mid 0 \rangle \coloneqq \begin{pmatrix} 1 \\ 0 \end{pmatrix} , \;\;\; \mid 1 \rangle \coloneqq \begin{pmatrix} 0 \\ 1 \end{pmatrix}\] 

When can represent a set of classical bits as the tensor product, which is realized as simply the outer product. For a system of two bits, we have 

  \[\begin{pmatrix} x_0 \\ x_1 \end{pmatrix} \otimes \begin{pmatrix} y_0 \\ y_1 \end{pmatrix} = \begin{pmatrix} x_0 y_0 \\ x_0 y_1 \\ x_1 y_0 \\ x_1 y_1 \end{pmatrix}\]

We can take our familiar logic gates and represent them as matrix operations. For example, the \texttt{NOT} gate, which flips the state of the bit, can be represented as 

  \[N = \begin{pmatrix} 0 & 1 \\ 1 & 0 \end{pmatrix}\]

and the \texttt{CNOT} gate, which takes gate, which takes two bits and flips the state of the second bit if the first is a $|0\rangle$ and does nothing otherwise, can be represented as 
  \[
    C = \begin{pmatrix} 1&0&0&0\\0&1&0&0\\0&0&0&1\\0&0&1&0 \end{pmatrix} \implies \begin{cases} C \Bigg( \begin{pmatrix} 1 \\ 0 \end{pmatrix} \otimes \begin{pmatrix} 0 \\ 1 \end{pmatrix} \Bigg) = C \begin{pmatrix} 0\\1\\0\\0 \end{pmatrix} = \begin{pmatrix} 0\\1\\0\\0 \end{pmatrix} = \begin{pmatrix} 1 \\ 0 \end{pmatrix} \otimes \begin{pmatrix} 0 \\ 1 \end{pmatrix} = |01\rangle\\
    \ldots \\
    C \Bigg( \begin{pmatrix} 0 \\ 1 \end{pmatrix} \otimes \begin{pmatrix} 0 \\ 1 \end{pmatrix} \Bigg) = C \begin{pmatrix} 0\\0\\0\\1 \end{pmatrix} = \begin{pmatrix} 0\\0\\1\\0 \end{pmatrix} = \begin{pmatrix} 0 \\ 1 \end{pmatrix} \otimes \begin{pmatrix} 1 \\ 0 \end{pmatrix} = |10\rangle 
    \end{cases}
  \]


\end{document}

\documentclass{article}

  \usepackage[letterpaper, top=1in, bottom=1in, left=1in, right=1in]{geometry}
  \usepackage[utf8]{inputenc}
  \usepackage[english]{babel}
  \usepackage{amsmath, amssymb, amsthm, mathrsfs, mathtools, centernot, hyperref, fancyhdr, lastpage}
  \usepackage{graphicx} 
  \usepackage{import}
  \usepackage{caption, subcaption}
  \usepackage{enumitem}
  \usepackage{fancyvrb,newverbs,xcolor}
  \definecolor{cverbbg}{gray}{0.93}

  \renewcommand{\thispagestyle}[1]{}

  \newenvironment{cverbatim}
   {\SaveVerbatim{cverb}}
   {\endSaveVerbatim
    \flushleft\fboxrule=0pt\fboxsep=.5em
    \colorbox{cverbbg}{%
      \makebox[\dimexpr\linewidth-2\fboxsep][l]{\BUseVerbatim{cverb}}%
    }
    \endflushleft
  }

  \newcommand{\incfig}[2][1]{%
    \def\svgwidth{#1\columnwidth}
    \import{./fig/}{#2.pdf\_tex}
  }

\begin{document}
\pagestyle{fancy}

\lhead{Computer Systems}
\chead{Muchang Bahng}
\rhead{\date{Spring 2024}}
\cfoot{\thepage / \pageref{LastPage}}


\title{Linux}
\author{Muchang Bahng}
\date{Spring 2024}

\maketitle

\tableofcontents

\pagebreak 

\section{Memory Management} 

Let's talk about memory, which can be visualized as a long array of ``boxes'' that each contain a byte. 

\begin{figure}[ht]
  \centering
  \incfig{memory-visual}
  \caption{}
  \label{fig:memory-visual}
\end{figure}

\end{document}

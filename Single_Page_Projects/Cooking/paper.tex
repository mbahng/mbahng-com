\documentclass{article}

% packages
  % basic stuff for rendering math
  \usepackage[letterpaper, top=1in, bottom=1in, left=1in, right=1in]{geometry}
  \usepackage[utf8]{inputenc}
  \usepackage[english]{babel}

  % for korean
  \usepackage{kotex}

  % extra math symbols and utilities
  \usepackage{mathtools}        % for extra stuff like \coloneqq
  \usepackage{mathrsfs}         % for extra stuff like \mathsrc{}
  \usepackage{centernot}        % for the centernot arrow 
  \usepackage{bm}               % for better boldsymbol/mathbf 
  \usepackage{enumitem}         % better control over enumerate, itemize
  \usepackage{hyperref}         % for hypertext linking
  \usepackage{fancyvrb}          % for better verbatim environments
  \usepackage{newverbs}         % for texttt{}
  \usepackage{xcolor}           % for colored text 
  \usepackage{listings}         % to include code
  \usepackage{lstautogobble}    % helper package for code
  \usepackage{parcolumns}       % for side by side columns for two column code
  

  % page layout
  \usepackage{fancyhdr}         % for headers and footers 
  \usepackage{lastpage}         % to include last page number in footer 
  \usepackage{parskip}          % for no indentation and space between paragraphs    
  \usepackage[T1]{fontenc}      % to include \textbackslash
  \usepackage{footnote}
  \usepackage{etoolbox}

  % for custom environments
  \usepackage{tcolorbox}        % for better colored boxes in custom environments
  \tcbuselibrary{breakable}     % to allow tcolorboxes to break across pages

  % figures
  \usepackage{pgfplots}
  \pgfplotsset{compat=1.18}
  \usepackage{float}            % for [H] figure placement
  \usepackage{caption} 
  \usepackage{subcaption}
  \captionsetup{font=small}

  % for tabular stuff 
  \usepackage{dcolumn}

  \usepackage[nottoc]{tocbibind}
  \pdfsuppresswarningpagegroup=1
  \hfuzz=5.002pt                % ignore overfull hbox badness warnings below this limit

% Custom Environments
  \newtcolorbox[auto counter, number within=section]{theorem}[1][]
  {
    colframe = red!25,
    colback  = red!10,
    coltitle = red!20!black,  
    breakable, 
    title = \textbf{Procedure \thetcbcounter ~(#1)}
  } 
  \newtcolorbox[auto counter, number within=section]{definition}[1][]
  {
    colframe = yellow!25,
    colback  = yellow!10,
    coltitle = yellow!20!black,  
    breakable, 
    title = \textbf{Ingredients \thetcbcounter ~(#1)}
  } 

  \BeforeBeginEnvironment{theorem}{\savenotes}
  \AfterEndEnvironment{theorem}{\spewnotes}
  \BeforeBeginEnvironment{definition}{\savenotes}
  \AfterEndEnvironment{definition}{\spewnotes}

  \definecolor{dkgreen}{rgb}{0,0.6,0}
  \definecolor{gray}{rgb}{0.5,0.5,0.5}
  \definecolor{mauve}{rgb}{0.58,0,0.82}
  \definecolor{darkblue}{rgb}{0,0,139}
  \definecolor{lightgray}{gray}{0.93}


% Page style
  \pagestyle{fancy}
  \fancyhead[L]{}
  \fancyhead[C]{Muchang Bahng}
  \fancyhead[R]{Fall 2024} 
  \fancyfoot[C]{\thepage / \pageref{LastPage}}
  \renewcommand{\footrulewidth}{0.4pt}          % the footer line should be 0.4pt wide
  \renewcommand{\thispagestyle}[1]{}  % needed to include headers in title page

\begin{document}

\title{Cookbook}
\author{Muchang Bahng}
\date{Winter 2024}

\maketitle
\tableofcontents
\pagebreak

I made these set of notes since I lose track of what I know how to cook and don't. It sometimes becomes a pain to keep track of what I need to buy in order to cook something when I go shopping. Most of these are simple enough to make, as I've been cooking during my senior year in college.  

\section{Salads} 

  \subsection{Steak Eggplant Salad} 

    \begin{definition}
      For the steak and eggplant.  
      \begin{enumerate}
        \item \textit{Steak}. 
        \item \textit{Cooking Oil}. 
        \item \textit{Salt}. 
        \item \textit{Pepper}. 
        \item \textit{Butter}. 
        \item \textit{Garlic}
        \item \textit{Eggplant}. 
        \item \textit{Asparagus}. 
      \end{enumerate}
      For the salad. 
      \begin{enumerate}
        \item \textit{Nappa Cabbage}. 
        \item \textit{Seasame Dressing}. 
        \item \textit{Black Pepper}. 
        \item \textit{Lemon/Lime Juice}. 
        \item \textit{Parmesan Cheese}
        \item \textit{Olives}. 
        \item \textit{Small Tomatoes}. 
        \item \textit{Mushrooms}. 
        \item \textit{Squash}. Optional. 
      \end{enumerate}
    \end{definition}

\section{Pastas} 

  \subsection{Vongole and Aglio Olio}

    \begin{definition}
      \begin{enumerate}
        \item \textit{Pasta Noodles}.
        \item \textit{Garlic}. Chopped or minced. 
        \item \textit{Olive Oil}. Better if extra virgin. 
        \item \textit{Chili Flakes}. For spice but optional. 
        \item \textit{Clams}. If you are making Vongole. 
        \item \textit{Shrimp}. Optional. 
      \end{enumerate}
    \end{definition}

    \begin{theorem}
      
    \end{theorem} 

  \subsection{Cream Pasta} 

  \subsection{Tomato Spaghetti}

  \subsection{Rose Pasta}

\section{Stir-Fried Sweet and Spicy Pork} 

  This was the first Korean food that I made. 

  \begin{definition}
    \begin{enumerate}
      \item \textit{Pork}. This can be pork belly (삼겹살), 목살, or anything that has some amount of fat on it. 
      \item \textit{Gochujang (고추장) and/or Gochugaru (고추가루)}.  
      \item \textit{Garlic}. Can be whole, minced, or sliced. 
      \item \textit{Ginger}. 
      \item \textit{Green Onions}. 
      \item \textit{Onions}. 
      \item \textit{Bell Pepper}. Optional but I like to put more vegetables. 
      \item \textit{Carrots}. Optional but I like to put more vegetables. 
      \item \textit{Soy Sauce}. 
    \end{enumerate}
  \end{definition}

  \subsection{Burritos} 

    During the winter of 2024, I was looking for an easy way to meal prep, and I found out that I can simply wrap these in a burrito.    

    \begin{definition}
      \begin{enumerate}
        \item \textit{Pork}. This can be pork belly (삼겹살), 목살, or anything that has some amount of fat on it. 
        \item \textit{Gochujang (고추장) and/or Gochugaru (고추가루)}. 
        \item \textit{Bacon}. Optional. 
        \item \textit{Onions}. 
        \item \textit{Cheddar Cheese}. 
        \item \textit{White rice}. 
      \end{enumerate}
    \end{definition}

\section{LA Galbi}

\section{Spicy Chicken Feet} 

\section{닭갈비} 

\section{Spicy Cheese Back Ribs} 

  \begin{definition}
    Essentials. 
    \begin{enumerate}
      \item \textit{Pork Ribs}. Such as baby pork ribs. 
      \item \textit{Rice Cake}. 
      \item \textit{Green Onions}. 
      \item \textit{Onions}.
      \item \textit{Mozzarella Cheese}.  
      \item \textit{Soy Sauce}. 
      \item \textit{Rice Wine} 
      \item \textit{물엿}. Can substitute with another portion of sugar. 
      \item \textit{Sugar}. 2 tbsp. 
      \item \textit{Gochugaru}. 
      \item \textit{Minced Garlic}. 
    \end{enumerate}
  \end{definition}

  \begin{theorem}
    \begin{enumerate}
      \item Rinse the ribs in 1-2 hours in cold water to drain the blood. tbh you don't need to do it that long, but it's better. 
      \item When water boils, put the ribs in and boil for 10 minutes. Then drain the water and rinse the ribs in cold water. 
      \item Then put the ribs in a new pot of water. Add the starch, sugar, and rice wine. 
      \item Slice the onions and green onions and put it in when the water boils. 
      \item Add the gochugaru, minced garlic, and soy sauce and boil on high for about 30 minutes. 
      \item Add rice cakes and boil for 5 more minutes on medium heat.  
      \item Then add more green onions and black pepper. 
    \end{enumerate}
  \end{theorem}

\section{떡볶이} 

\section{Soondubu (순두부)}

\section{Kimchi Stew (김치찌개)}

\section{Ramen} 

  Here are different variants to make your ramen a bit better. 

  \subsection{Stir Fried} 

    \begin{definition}
      Essentials. 
      \begin{enumerate}
        \item \textit{Ramen}. Should be spicy and broth based. 
        \item \textit{Eggs}. 
        \item \textit{Vegetable Oil}. 
      \end{enumerate}
      Optional. 
      \begin{enumerate}
        \item \textit{Green Onions. }
      \end{enumerate}
    \end{definition}

    \begin{theorem}
      \begin{enumerate}
        \item Boil the ramen without the sauce. 
        \item While ramen is boiling, put 3 tablespoons of vegetable oil. Start scrambling the eggs in it. 
        \item Then add the ramen sauce in the eggs. 
        \item Then add the boiled ramen. 
        \item Add some green onions. 
      \end{enumerate}
    \end{theorem}

\section{Fried Rice} 

  \subsection{Egg Fried Rice} 

    \begin{definition}
      \begin{enumerate}
        \item \textit{Rice}. Preferably leftover ones in the fridge.  
        \item \textit{Eggs}. 
        \item \textit{Salt}. 
        \item \textit{Black Peppers}. 
        \item \textit{Soy Sauce}. 
        \item \textit{Green Onions}. 
        \item \textit{MSG}. Optional. 
      \end{enumerate}
    \end{definition}

  \subsection{Kimchi Fried Rice}

    \begin{definition}
      \begin{enumerate}
        \item \textit{Rice}. Preferably leftover ones in the fridge.  
        \item \textit{Eggs}. Optional. 
        \item \textit{Salt}. 
        \item \textit{Black Peppers}. 
        \item \textit{Soy Sauce}. 
        \item \textit{Green Onions}. 
        \item \textit{MSG}. Optional. 
        \item \textit{Kimchi}. 
        \item \textit{Bacon}. 
      \end{enumerate}
    \end{definition} 

\section{Korean Pancakes (전)} 

  There are many different types of pancakes. 

  \subsection{Cabbage Pancake} 

    Originally from Japan. 
  
    \begin{definition}
      Essentials. 
      \begin{enumerate}
        \item \textit{Cabbage}. 
        \item 
      \end{enumerate}

      Optional. 
    \end{definition}

  \subsection{Rice and Tuna Pancake} 

    \begin{definition}
      Essentials. 
      \begin{enumerate}
        \item \textit{Canned Tuna} 
        \item \textit{Rice}. Leftover white rice ideally. 
        \item \textit{Frying Powder}. 부침가루/튀김가루 (both okay). 
        \item \textit{Eggs}
        \item \textit{Salt}. Some MSG too is better. 
        \item \textit{Green Onions}. Don't need too much. 
        \item \textit{Onions}. Don't need too much. 
        \item \textit{Carrots}. Don't need too much. 
      \end{enumerate}
      Optional. 
      \begin{enumerate}
        \item \textit{Peppers}. For spice, can be substituted with paprika. Optional. 
        \item \textit{Ketchup}. For the sauce
      \end{enumerate}
    \end{definition}

\section{Peanut Noodles}

  \begin{definition}
    \begin{enumerate}
      \item \textit{Noodles}. 
      \item \textit{Rice Vinegar/Wine}. 
      \item \textit{Lao Gan Ma}. 
      \item \textit{Soy Sauce}. 
      \item \textit{Minced Garlic}. 
      \item \textit{Sugar}. 
      \item \textit{Sesame Oil}. 
      \item \textit{Peanut Butter}. 
    \end{enumerate}
  \end{definition}


\end{document}

\documentclass{article}

  % packages
    % basic stuff for rendering math
    \usepackage[letterpaper, top=1in, bottom=1in, left=1in, right=1in]{geometry}
    \usepackage[utf8]{inputenc}
    \usepackage[english]{babel}
    \usepackage{amsmath} 
    \usepackage{amssymb}

    % extra math symbols and utilities
    \usepackage{mathtools}        % for extra stuff like \coloneqq
    \usepackage{mathrsfs}         % for extra stuff like \mathsrc{}
    \usepackage{centernot}        % for the centernot arrow 
    \usepackage{bm}               % for better boldsymbol/mathbf 
    \usepackage{enumitem}         % better control over enumerate, itemize
    \usepackage{hyperref}         % for hypertext linking
    \usepackage{xr-hyper}
    \usepackage{fancyvrb}          % for better verbatim environments
    \usepackage{newverbs}         % for texttt{}
    \usepackage{xcolor}           % for colored text 
    \usepackage{listings}         % to include code
    \usepackage{lstautogobble}    % helper package for code
    \usepackage{parcolumns}       % for side by side columns for two column code
    

    % page layout
    \usepackage{fancyhdr}         % for headers and footers 
    \usepackage{lastpage}         % to include last page number in footer 
    \usepackage{parskip}          % for no indentation and space between paragraphs    
    \usepackage[T1]{fontenc}      % to include \textbackslash
    \usepackage{footnote}
    \usepackage{etoolbox}

    % for custom environments
    \usepackage{tcolorbox}        % for better colored boxes in custom environments
    \tcbuselibrary{breakable}     % to allow tcolorboxes to break across pages

    % figures
    \usepackage{pgfplots}
    \pgfplotsset{compat=1.18}
    \usepackage{float}            % for [H] figure placement
    \usepackage{tikz}
    \usepackage{tikz-cd}
    \usepackage{circuitikz}
    \usetikzlibrary{arrows}
    \usetikzlibrary{positioning}
    \usetikzlibrary{calc}
    \usepackage{graphicx}
    \usepackage{algorithmic}
    \usepackage{caption} 
    \usepackage{subcaption}
    \captionsetup{font=small}

    % for tabular stuff 
    \usepackage{dcolumn}

    \usepackage[nottoc]{tocbibind}
    \pdfsuppresswarningpagegroup=1
    \hfuzz=5.002pt                % ignore overfull hbox badness warnings below this limit

  % New and replaced operators
    \DeclareMathOperator{\Tr}{Tr}
    \DeclareMathOperator{\Sym}{Sym}
    \DeclareMathOperator{\Span}{span}
    \DeclareMathOperator{\std}{std}
    \DeclareMathOperator{\Cov}{Cov}
    \DeclareMathOperator{\Var}{Var}
    \DeclareMathOperator{\Corr}{Corr}
    \DeclareMathOperator{\pos}{pos}
    \DeclareMathOperator*{\argmin}{\arg\!\min}
    \DeclareMathOperator*{\argmax}{\arg\!\max}
    \newcommand{\ket}[1]{\ensuremath{\left|#1\right\rangle}}
    \newcommand{\bra}[1]{\ensuremath{\left\langle#1\right|}}
    \newcommand{\braket}[2]{\langle #1 | #2 \rangle}
    \newcommand{\qed}{\hfill$\blacksquare$}     % I like QED squares to be black

  % Custom Environments
    \newtcolorbox[auto counter, number within=section]{question}[1][]
    {
      colframe = orange!25,
      colback  = orange!10,
      coltitle = orange!20!black,  
      breakable, 
      title = \textbf{Question \thetcbcounter ~(#1)}
    }

    \newtcolorbox[auto counter, number within=section]{exercise}[1][]
    {
      colframe = teal!25,
      colback  = teal!10,
      coltitle = teal!20!black,  
      breakable, 
      title = \textbf{Exercise \thetcbcounter ~(#1)}
    }
    \newtcolorbox[auto counter, number within=section]{solution}[1][]
    {
      colframe = violet!25,
      colback  = violet!10,
      coltitle = violet!20!black,  
      breakable, 
      title = \textbf{Solution \thetcbcounter}
    }
    \newtcolorbox[auto counter, number within=section]{lemma}[1][]
    {
      colframe = red!25,
      colback  = red!10,
      coltitle = red!20!black,  
      breakable, 
      title = \textbf{Lemma \thetcbcounter ~(#1)}
    }
    \newtcolorbox[auto counter, number within=section]{theorem}[1][]
    {
      colframe = red!25,
      colback  = red!10,
      coltitle = red!20!black,  
      breakable, 
      title = \textbf{Theorem \thetcbcounter ~(#1)}
    } 
    \newtcolorbox[auto counter, number within=section]{proposition}[1][]
    {
      colframe = red!25,
      colback  = red!10,
      coltitle = red!20!black,  
      breakable, 
      title = \textbf{Proposition \thetcbcounter ~(#1)}
    } 
    \newtcolorbox[auto counter, number within=section]{corollary}[1][]
    {
      colframe = red!25,
      colback  = red!10,
      coltitle = red!20!black,  
      breakable, 
      title = \textbf{Corollary \thetcbcounter ~(#1)}
    } 
    \newtcolorbox[auto counter, number within=section]{proof}[1][]
    {
      colframe = orange!25,
      colback  = orange!10,
      coltitle = orange!20!black,  
      breakable, 
      title = \textbf{Proof. }
    } 
    \newtcolorbox[auto counter, number within=section]{definition}[1][]
    {
      colframe = yellow!25,
      colback  = yellow!10,
      coltitle = yellow!20!black,  
      breakable, 
      title = \textbf{Definition \thetcbcounter ~(#1)}
    } 
    \newtcolorbox[auto counter, number within=section]{example}[1][]
    {
      colframe = blue!25,
      colback  = blue!10,
      coltitle = blue!20!black,  
      breakable, 
      title = \textbf{Example \thetcbcounter ~(#1)}
    } 
    \newtcolorbox[auto counter, number within=section]{code}[1][]
    {
      colframe = green!25,
      colback  = green!10,
      coltitle = green!20!black,  
      breakable, 
      title = \textbf{Code \thetcbcounter ~(#1)}
    } 
    \newtcolorbox[auto counter, number within=section]{algo}[1][]
    {
      colframe = green!25,
      colback  = green!10,
      coltitle = green!20!black,  
      breakable, 
      title = \textbf{Algorithm \thetcbcounter ~(#1)}
    } 

    \definecolor{dkgreen}{rgb}{0,0.6,0}
    \definecolor{gray}{rgb}{0.5,0.5,0.5}
    \definecolor{mauve}{rgb}{0.58,0,0.82}
    \definecolor{darkblue}{rgb}{0,0,139}
    \definecolor{lightgray}{gray}{0.93}
    \renewcommand{\algorithmiccomment}[1]{\hfill$\triangleright$\textcolor{blue}{#1}}

    % default options for listings (for code)
    \lstset{
      autogobble,
      frame=ltbr,
      language=Python,
      aboveskip=3mm,
      belowskip=3mm,
      showstringspaces=false,
      columns=fullflexible,
      keepspaces=true,
      basicstyle={\small\ttfamily},
      numbers=left,
      firstnumber=1,                        % start line number at 1
      numberstyle=\tiny\color{gray},
      keywordstyle=\color{blue},
      commentstyle=\color{dkgreen},
      stringstyle=\color{mauve},
      backgroundcolor=\color{lightgray}, 
      breaklines=true,                      % break lines
      breakatwhitespace=true,
      tabsize=3, 
      xleftmargin=2em, 
      framexleftmargin=1.5em, 
      stepnumber=1
    }

  % Page style
    \pagestyle{fancy}
    \fancyhead[L]{}
    \fancyhead[C]{Muchang Bahng}
    \fancyhead[R]{Spring 2025} 
    \fancyfoot[C]{\thepage / \pageref{LastPage}}
    \renewcommand{\footrulewidth}{0.4pt}          % the footer line should be 0.4pt wide
    \renewcommand{\thispagestyle}[1]{}  % needed to include headers in title page

  % external documents 
  %  \externaldocument[place-]{../Machine_Learning/paper}[../Machine_Learning/paper.pdf] 

\begin{document}

\title{Psychology}
\author{Muchang Bahng}
\date{Spring 2025}

\maketitle
\tableofcontents
\pagebreak

  A series of informal lectures by Sara Shao. 
  
\section{Erik Erikson's Stages of Psychosocial Development} 

  At each of these (seemingly arbitrarily chosen) stages of life, there is a central dilemma. 
  \begin{definition}[0-18 Months, Trust vs Mistrust]
    Can you trust the world around you? For people to take care of you? Or do you have to fend for youself?
  \end{definition}

  \begin{definition}[18 Months to 3 Years, \textbf{Autonomy vs Shame/Doubt}. ]
    Basically how the world treats you when you learn. You're beginning to explore the world (basic things, such as potty training, eat food by yourself). If you are encouraged and rewarded, you embrace taking on new challenges, and if not, you become doubtful. 
  \end{definition}

  \begin{definition}[3 to 5 Years, \textbf{Initiative vs Guilt}.]
    The way that she differentiates is that Sara thinks about playing. You're exploring your creativity largely through play (e.g. inventing worlds, inventing games). These things are central to toddler-hood. 
  \end{definition}

  \begin{definition}[5 to 13 Years, \textbf{Industry vs Inferiority}. ]
    This is similar to the previous two stages, but autonomy vs doubt is on a simpler scale, and initiative vs guilt is more academic and more assessment heavy. This stage is more like concluding that you are a competent person and have the confidence, or you are inferior and not capable. 
  \end{definition}

  \begin{definition}[13 to 21 Years, \textbf{Identity vs Role Confusion}. ]
    Adolescence is important since you grow a lot and it's a transition between being a kid and an adult. This stage is about grappling with who you are, and you're measuring (stems from industry vs inferiority) what makes you same or different than others. You're wondering: Am I a social person? Am I popular? Do people like me? Am I a smart person? Am I a STEM person? If you come out of this stage without identity, you fall into role confusion, where you don't know who you are and can be very scary. Or you think you know what your role is but it's hard to accept, leading to turmoil. 
  \end{definition}

  \begin{definition}[21 to 39 Years, \textbf{Intimacy vs Isolation}. ]
    This is pretty straightforward. As you become an adult, it's a lot harder to form social connections, so you either further develop intimacy with current connections (relationships, friendships, e.g. get married). The relationships and how close you are to people becomes very important. People may struggle when they don't have anybody to confide in. 
  \end{definition}

  \begin{definition}[40 to 65 Years, \textbf{Generativity vs Stagnation}. ]
    By now, you've probably settled into some role/identity. It's around this time that you really start thinking about your legacy. Perhaps by this time, your children have grown up so you don't care as much about them. You ask yourself: Are you a productive member of society or are you a bum? This leads to the next stage. 
  \end{definition}

  \begin{definition}[65+ Years, \textbf{Integrity vs Despair}.]
    This is roughly retirement age to end of your life. Hopefully you've found your identity, developed relationships, produced something, etc. You're getting very aware that they're maybe one big surprise left. You reflect on your life as it's been. Integrity means you reflect and conclude that you had a fulfilling life in relationships, accomplished a lot, vs despair when you were unable to resolve any of the previous dilemmas (e.g. you didn't think you did enough with your life). In despair, you have a lot of regrets. 
  \end{definition}




\end{document}
